% !TEX root = ./patterns-paper.tex
\begin{figure}[h]
$\arraycolsep=4pt\begin{array}{lll}
\tau & ::= &
  \tnum ~\vert~
  \tarr{\tau_1}{\tau_2} ~\vert~
  \tprod{\tau_1}{\tau_2} ~\vert~
  \tsum{\tau_1}{\tau_2} ~\vert~
  \tunit ~\vert~
  \tvoid \\
e & ::= &
  x ~\vert~
  \hnum{n} \\
  & ~\vert~ &
  \hlam{x}{\tau}{e} ~\vert~
  \hap{e_1}{e_2} \\
  & ~\vert~ &
  \hpair{e_1}{e_2} ~\vert~
  \htriv \\
  & ~\vert~ &
  \hinl{\tau}{e} ~\vert~
  \hinr{\tau}{e} ~\vert~
  \hmatch{e}{rs} \\
rs & ::= &
  \cdot ~\vert~ \hrules{r}{rs'} \\
r & ::= &
  \hrul{p}{e} \\
p & ::= &
  x ~\vert~
  \hnum{n} ~\vert~
  \_ ~\vert~
  \hpair{p_1}{p_2} ~\vert~
  \htriv ~\vert~
  \hinlp{p} ~\vert~
  \hinrp{p}
\end{array}$
\end{figure}

\begin{figure}[h]
\fbox{$\hexptyp{\Gamma}{e}{\tau}$}
\begin{subequations}
\begin{equation}
  \inferrule[VAR]{ }{
    \hexptyp{\Gamma, x : \tau}{x}{\tau}
  }
\end{equation}
\begin{equation}
\inferrule[LAM]{
  \hexptyp{\Gamma, x : \tau_1}{e}{\tau_2}
}{
  \hexptyp{\Gamma}{\hlam{x}{\tau_1}{e}}{\tarr{\tau_1}{\tau_2}}
}
\end{equation}
\begin{equation}
\inferrule[AP]{
  \hexptyp{\Gamma}{e_1}{\tarr{\tau_2}{\tau}} \\
  \hexptyp{\Gamma}{e_2}{\tau_2}
}{
  \hexptyp{\Gamma}{\hap{e_1}{e_2}}{\tau}
}
\end{equation}
\begin{equation}
\inferrule[PAIR]{
  \hexptyp{\Gamma}{e_1}{\tau_1} \\
  \hexptyp{\Gamma}{e_2}{\tau_2}
}{
  \hexptyp{\Gamma}{\hpair{e_1}{e_2}}{\tprod{\tau_1}{\tau_2}}
}
\end{equation}
\begin{equation}
\inferrule[TRIV]{ }{
  \hexptyp{\Gamma}{\htriv}{\tunit}
}
\end{equation}
\begin{equation}
\inferrule[INL]{
  \hexptyp{\Gamma}{e}{\tau_1}
}{
  \hexptyp{\Gamma}{\hinl{\tau_2}{e}}{\tsum{\tau_1}{\tau_2}}
}
\end{equation}
\begin{equation}
\inferrule[INR]{
  \hexptyp{\Gamma}{e}{\tau_2}
}{
  \hexptyp{\Gamma}{\hinr{\tau_1}{e}}{\tsum{\tau_1}{\tau_2}}
}
\end{equation}
\begin{equation}
\inferrule[MATCH]{
  \hexptyp{\Gamma}{e_1}{\tau_1} \\
  \hexptyp{\Gamma}{rs}{\trul{\tau_1}{\tau}}
}{
  \hexptyp{\Gamma}{\hmatch{e_1}{rs}}{\tau}
}
\end{equation}
\end{subequations}
\end{figure}

\begin{figure}[h]
\fbox{$\hexptyp{\Gamma}{rs}{\hrul{\tau_1}{\tau_2}}$}
\begin{subequations}
\begin{equation}
\inferrule[RULE]{
  \hpattyp{p}{\tau_1}{\Gamma'} \\
  \hexptyp{\Gamma \uplus \Gamma'}{e}{\tau_2}
}{
  \hexptyp{\Gamma}{\hrul{p}{e}}{\hrul{\tau_1}{\tau_2}}
}
\end{equation}
\begin{equation}
\inferrule[RULES]{
  \hexptyp{\Gamma}{r}{\hrul{\tau_1}{\tau_2}} \\
  \hexptyp{\Gamma}{rs'}{\hrul{\tau_1}{\tau_2}}
}{
  \hexptyp{\Gamma}{\hrules{r}{rs'}}{\hrul{\tau_1}{\tau_2}}
}
\end{equation}
\end{subequations}
\end{figure}

\begin{figure}[h]
\fbox{$\hpattyp{p}{\tau}{\Gamma}$}
\begin{subequations}
\begin{equation}
\inferrule[VAR]{ }{
  \hpattyp{x}{\tau}{x : \tau}
}
\end{equation}
\begin{equation}
\inferrule[NUM]{ }{
  \hpattyp{\hnum{n}}{\tnum}{\cdot}
}
\end{equation}
\begin{equation}
\inferrule[WILD]{ }{
  \hpattyp{\_}{\tau}{\cdot}
}
\end{equation}
\begin{equation}
\inferrule[PAIR]{
  \hpattyp{p_1}{\tau_1}{\Gamma_1} \\
  \hpattyp{p_2}{\tau_2}{\Gamma_2}
}{
  \hpattyp{\hpair{p_1}{p_2}}{\tprod{\tau_1}{\tau_2}}{\Gamma_1 \uplus \Gamma_2}
}
\end{equation}
\begin{equation}
\inferrule[TRIV]{ }{
  \hpattyp{\htriv}{\tunit}{\cdot}
}
\end{equation}
\begin{equation}
\inferrule[INL]{
  \hpattyp{p}{\tau_1}{\Gamma}
}{
  \hpattyp{\hinlp{p}}{\tsum{\tau_1}{\tau_2}}{\Gamma}
}
\end{equation}
\begin{equation}
\inferrule[INR]{
  \hpattyp{p}{\tau_2}{\Gamma}
}{
  \hpattyp{\hinrp{p}}{\tsum{\tau_1}{\tau_2}}{\Gamma}
}
\end{equation}
\end{subequations}
\end{figure}

\begin{figure}[h]
\fbox{$\theta : \Gamma$}
\begin{subequations}
\begin{equation}
\inferrule[EMPTY]{ }{
  \phi : \cdot
}
\end{equation}
\begin{equation}
\inferrule[EXTENSION]{
  \theta : \Gamma \\
  \hexptyp{\cdot}{e}{\tau}
}{
  \theta , x \hookrightarrow e :
    \Gamma , e : \tau
}
\end{equation}
\end{subequations}
\end{figure}

\begin{figure}[h]
\fbox{$\hpatmatch{e}{p}{\theta}$}
\begin{subequations}
\begin{equation}
\inferrule[VAR]{ }{
  \hpatmatch{e}{x}{x \hookrightarrow e}
}
\end{equation}
\begin{equation}
\inferrule[NUM]{ }{
  \hpatmatch{\hnum{n}}{\hnum{n}}{\cdot}
}
\end{equation}
\begin{equation}
\inferrule[WILD]{ }{
  \hpatmatch{e}{\_}{\cdot}
}
\end{equation}
\begin{equation}
\inferrule[UNIT]{ }{
  \hpatmatch{\htriv}{\htriv}{\cdot}
}
\end{equation}
\begin{equation}
\inferrule[PAIR]{
  \hpatmatch{e_1}{p_1}{\theta_1} \\
  \hpatmatch{e_2}{p_2}{\theta_2}
}{
  \hpatmatch{\hpair{e_1}{e_2}}{\hpair{p_1}{p_2}}{\theta_1 \uplus \theta_2}
}
\end{equation}
\begin{equation}
\inferrule[INL]{
  \hpatmatch{e}{p}{\theta}
}{
  \hpatmatch{\hinl{\tau}{e}}{\hinlp{p}}{\theta}
}
\end{equation}
\begin{equation}
\inferrule[INR]{
  \hpatmatch{e}{p}{\theta}
}{
  \hpatmatch{\hinr{\tau}{e}}{\hinrp{p}}{\theta}
}
\end{equation}
\end{subequations}
\end{figure}

\begin{figure}[h]
\fbox{$e \bot p$}
\begin{subequations}
\begin{equation}
\inferrule[PAIR1]{
  e_1 \bot p_1
}{
  \hpair{e_1}{e_2} \bot \hpair{p_1}{p_2}
}
\end{equation}
\begin{equation}
\inferrule[PAIR2]{
  e_2 \bot p_2
}{
  \hpair{e_1}{e_2} \bot \hpair{p_1}{p_2}
}
\end{equation}
\begin{equation}
\inferrule[OPPOR]{ }{
  \hinl{\tau}{e} \bot \hinrp{p}
}
\end{equation}
\begin{equation}
\inferrule[OPPOL]{ }{
  \hinr{\tau}{e} \bot \hinlp{p}
}
\end{equation}
\begin{equation}
\inferrule[INL]{
  e \bot p
}{
  \hinl{\tau}{e} \bot \hinlp{p}
}
\end{equation}
\begin{equation}
\inferrule[INR]{
  e \bot p
}{
  \hinr{\tau}{e} \bot \hinrp{p}
}
\end{equation}
\end{subequations}
\end{figure}

\begin{figure}[h]
\fbox{$\hval{e}$}
\begin{subequations}
\begin{equation}
\inferrule[NUM]{ }{
  \hval{\hnum{n}}
}
\end{equation}
\begin{equation}
\inferrule[TRIV]{ }{
  \hval{\htriv}
}
\end{equation}
\begin{equation}
\inferrule[LAM]{ }{
  \hval{\hlam{x}{\tau}{e}}
}
\end{equation}
\begin{equation}
\inferrule[PAIR]{
  \hval{e_1} \\
  \hval{e_2}
}{\hval{\hpair{e_1}{e_2}}}
\end{equation}
\begin{equation}
\inferrule[INL]{
  \hval{e}
}{
  \hval{\hinl{\tau}{e}}
}
\end{equation}
\begin{equation}
\inferrule[INR]{
  \hval{e}
}{
  \hval{\hinr{\tau}{e}}
}
\end{equation}
\end{subequations}
\end{figure}

\begin{figure}[h]
\fbox{$e \mapsto e'$}
\begin{subequations}
\begin{equation}
\inferrule[AP1]{
  e_1 \mapsto e_1'
}{
  \hap{e_1}{e_2} \mapsto \hap{e_1'}{e_2}
}
\end{equation}
\begin{equation}
\inferrule[AP2]{
  \hval{e_1} \\
  e_2 \mapsto e_2'
}{
  \hap{e_1}{e_2} \mapsto \hap{e_1}{e_2'}
}
\end{equation}
\begin{equation}
\inferrule[AP]{
  \hval{e_2}
}{
  \hap{\hlam{x}{\tau}{e_1}}{e_2} \mapsto
    [e_2/x]e_1
}
\end{equation}
\begin{equation}
\inferrule[PAIR1]{
  e_1 \mapsto e_1'
}{
  \hpair{e_1}{e_2} \mapsto \hpair{e_1'}{e_2}
}
\end{equation}
\begin{equation}
\inferrule[PAIR2]{
  \hval{e_1} \\
  e_2 \mapsto e_2'
}{
  \hpair{e_1}{e_2} \mapsto \hpair{e_1}{e_2'}
}
\end{equation}
\begin{equation}
\inferrule[INL]{
  e \mapsto e'
}{
  \hinl{\tau}{e} \mapsto \hinl{\tau}{e'}
}
\end{equation}
\begin{equation}
\inferrule[INR]{
  e \mapsto e'
}{
  \hinr{\tau}{e} \mapsto \hinr{\tau}{e'}
}
\end{equation}
\begin{equation}
\inferrule[EXPMATCH]{
  e \mapsto e'
}{
  \hmatch{e}{rs} \mapsto \hmatch{e'}{rs}
}
\end{equation}
\begin{equation}
\inferrule[EXHMATCH]{
  \hval{e}
}{
  \herr{\hmatch{e}{}}
}
\end{equation}
\begin{equation}
\inferrule[SUCCMATCH]{
  \hval{e} \\
  \hpatmatch{e}{p_0}{\theta}
}{
  \hmatch{e}{\hrules{\hrul{p_0}{e_0}}{rs}} \mapsto \hat{\theta}(e_0)
}
\end{equation}
\begin{equation}
\inferrule[FAILMATCH]{
  \hval{e} \\
  e \bot p_0 \\
  \hmatch{e}{rs} \mapsto e'
}{
  \hmatch{e}{\hrules{\hrul{p_0}{e_0}}{rs}} \mapsto e'
}
\end{equation}
\end{subequations}
\end{figure}