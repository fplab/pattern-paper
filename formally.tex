% !TEX root = ./patterns-paper.tex
\begin{figure}[t]
$\arraycolsep=4pt\begin{array}{lll}
\tau & ::= &
  \tnum ~\vert~
  \tarr{\tau_1}{\tau_2} ~\vert~
  \tprod{\tau_1}{\tau_2} ~\vert~
  \tsum{\tau_1}{\tau_2} ~\vert~
  \tunit ~\vert~
  \tvoid \\
e & ::= &
  x ~\vert~
  \hnum{n} \\
  & ~\vert~ &
  \hlam{x}{\tau}{e} ~\vert~
  \hap{e_1}{e_2} \\
  & ~\vert~ &
  \hpair{e_1}{e_2} ~\vert~
  \htriv \\
  & ~\vert~ &
  \hinl{\tau}{e} ~\vert~
  \hinr{\tau}{e} ~\vert~
  \hmatch{e}{rs} \\
rs & ::= &
  \cdot ~\vert~ \hrulesP{r}{rs'} \\
r & ::= &
  \hrul{p}{e} \\
p & ::= &
  x ~\vert~
  \hnum{n} ~\vert~
  \_ ~\vert~
  \hpair{p_1}{p_2} ~\vert~
  \htriv ~\vert~
  \hinlp{p} ~\vert~
  \hinrp{p} ~\vert~
  \hehole{u} ~\vert~
  \hhole{p}{u}
\end{array}$
\end{figure}

\begin{figure}[t]
\fbox{$\hexptyp{\Gamma}{e}{\tau}$}~~\text{$e$ is of type $\tau$}
\begin{subequations}
\begin{equation}
  \inferrule[TVar]{ }{
    \hexptyp{\Gamma, x : \tau}{x}{\tau}
  }
\end{equation}
\begin{equation}
\inferrule[TLam]{
  \hexptyp{\Gamma, x : \tau_1}{e}{\tau_2}
}{
  \hexptyp{\Gamma}{\hlam{x}{\tau_1}{e}}{\tarr{\tau_1}{\tau_2}}
}
\end{equation}
\begin{equation}
\inferrule[TAp]{
  \hexptyp{\Gamma}{e_1}{\tarr{\tau_2}{\tau}} \\
  \hexptyp{\Gamma}{e_2}{\tau_2}
}{
  \hexptyp{\Gamma}{\hap{e_1}{e_2}}{\tau}
}
\end{equation}
\begin{equation}
\inferrule[TPair]{
  \hexptyp{\Gamma}{e_1}{\tau_1} \\
  \hexptyp{\Gamma}{e_2}{\tau_2}
}{
  \hexptyp{\Gamma}{\hpair{e_1}{e_2}}{\tprod{\tau_1}{\tau_2}}
}
\end{equation}
\begin{equation}
\inferrule[TTriv]{ }{
  \hexptyp{\Gamma}{\htriv}{\tunit}
}
\end{equation}
\begin{equation}
\inferrule[TInL]{
  \hexptyp{\Gamma}{e}{\tau_1}
}{
  \hexptyp{\Gamma}{\hinl{\tau_2}{e}}{\tsum{\tau_1}{\tau_2}}
}
\end{equation}
\begin{equation}
\inferrule[TInR]{
  \hexptyp{\Gamma}{e}{\tau_2}
}{
  \hexptyp{\Gamma}{\hinr{\tau_1}{e}}{\tsum{\tau_1}{\tau_2}}
}
\end{equation}
\begin{equation}
\inferrule[TMatch]{
  \hexptyp{\Gamma}{e_1}{\tau_1} \\
  \hexptyp{\Gamma}{rs}{\trul{\tau_1}{\tau}}
}{
  \hexptyp{\Gamma}{\hmatch{e_1}{rs}}{\tau}
}
\end{equation}
\end{subequations}
\end{figure}

\begin{figure}[t]
\fbox{$\hexptyp{\Gamma}{r}{\trul{\tau_1}{\tau_2}}$}~~\text{$r$ is of type $\trulnp{\tau_1}{\tau_2}$}
\begin{subequations}
\begin{equation}
\inferrule[TRule]{
  \hpattyp{p}{\tau_1}{\Gamma'} \\
  \hexptyp{\Gamma \uplus \Gamma'}{e}{\tau_2}
}{
  \hexptyp{\Gamma}{\hrulP{p}{e}}{\trul{\tau_1}{\tau_2}}
}
\end{equation}
\end{subequations}
\end{figure}

\begin{figure}[t]
\fbox{$\hexptyp{\Gamma}{rs}{\trul{\tau_1}{\tau_2}}$}~~\text{$rs$ is of type $\trulnp{\tau_1}{\tau_2}$}
\begin{subequations}
\begin{equation}
\inferrule[TZeroRule]{ }{
  \hexptyp{\Gamma}{\cdot}{\trul{\tau_1}{\tau_2}}
}
\end{equation}
\begin{equation}
\inferrule[TRules]{
  \hexptyp{\Gamma}{r}{\trul{\tau_1}{\tau_2}} \\
  \hexptyp{\Gamma}{rs}{\trul{\tau_1}{\tau_2}}
}{
  \hexptyp{\Gamma}{\hrulesP{r}{rs}}{\trul{\tau_1}{\tau_2}}
}
\end{equation}
\end{subequations}
\end{figure}

\begin{figure}[t]
\fbox{$\hpattyp{p}{\tau}{\Gamma}{\Sigma}$}~~\text{$p$ matches expression of type $\tau$ and generates context $\Gamma$ and hole context $\Sigma$}
\begin{subequations}
\begin{equation}
\inferrule[PTVar]{ }{
  \hpattyp{x}{\tau}{x : \tau}{\cdot}
}
\end{equation}
\begin{equation}
\inferrule[PTNum]{ }{
  \hpattyp{\hnum{n}}{\tnum}{\cdot}{\cdot}
}
\end{equation}
\begin{equation}
\inferrule[PTWild]{ }{
  \hpattyp{\_}{\tau}{\cdot}{\cdot}
}
\end{equation}
\begin{equation}
\inferrule[PTPair]{
  \hpattyp{p_1}{\tau_1}{\Gamma_1}{\Sigma_1} \\
  \hpattyp{p_2}{\tau_2}{\Gamma_2}{\Sigma_2}
}{
  \hpattyp{\hpair{p_1}{p_2}}{\tprod{\tau_1}{\tau_2}}{\Gamma_1 \uplus \Gamma_2}{\Sigma_1 \cup \Sigma_2}
}
\end{equation}
\begin{equation}
\inferrule[PTTriv]{ }{
  \hpattyp{\htriv}{\tunit}{\cdot}{\cdot}
}
\end{equation}
\begin{equation}
\inferrule[PTInL]{
  \hpattyp{p}{\tau_1}{\Gamma}{\Sigma}
}{
  \hpattyp{\hinlp{p}}{\tsum{\tau_1}{\tau_2}}{\Gamma}{\Sigma}
}
\end{equation}
\begin{equation}
\inferrule[PTInR]{
  \hpattyp{p}{\tau_2}{\Gamma}{\Sigma}
}{
  \hpattyp{\hinrp{p}}{\tsum{\tau_1}{\tau_2}}{\Gamma}{\Sigma}
}
\end{equation}
\begin{equation}
\inferrule[PTEHole]{ }{
  \hpattyp{\hehole{u}}{\tau}{\cdot}{u :: \tau}
}
\end{equation}
\begin{equation}
\inferrule[PTHole]{
  \hpattyp{p}{\tau'}{\Gamma}{\Sigma}
}{
  \hpattyp{\hhole{p}{u}}{\tau}
  {\Gamma}{\Sigma , u :: \tau}
}
\end{equation}
\end{subequations}
\end{figure}

\begin{figure}[t]
\fbox{$\hsubstyp{\theta}{\Gamma}$}~~\text{Substitution $\theta$ is of type $\Gamma$}
\begin{subequations}
\begin{equation}
\inferrule[STEmpty]{ }{
  \hsubstyp{\emptyset}{\cdot}
}
\end{equation}
\begin{equation}
\inferrule[STExtend]{
  \hsubstyp{\theta}{\Gamma} \\
  \hexptyp{\cdot}{e}{\tau}
}{
  \hsubstyp{\theta , x / e}{\Gamma , e : \tau}
}
\end{equation}
\end{subequations}
\end{figure}

\begin{figure}[t]
\fbox{$\hpatmatch{e}{p}{\theta}$}~~\text{Pattern matching $e$ on $p$ emits $\theta$}
\begin{subequations}
\begin{equation}
\inferrule[PMVar]{ }{
  \hpatmatch{e}{x}{x / e}
}
\end{equation}
\begin{equation}
\inferrule[PMNum]{ }{
  \hpatmatch{\hnum{n}}{\hnum{n}}{\cdot}
}
\end{equation}
\begin{equation}
\inferrule[PMWild]{ }{
  \hpatmatch{e}{\_}{\cdot}
}
\end{equation}
\begin{equation}
\inferrule[PMTriv]{ }{
  \hpatmatch{\htriv}{\htriv}{\cdot}
}
\end{equation}
\begin{equation}
\inferrule[PMPair]{
  \hpatmatch{e_1}{p_1}{\theta_1} \\
  \hpatmatch{e_2}{p_2}{\theta_2}
}{
  \hpatmatch{\hpair{e_1}{e_2}}{\hpair{p_1}{p_2}}{\theta_1 \uplus \theta_2}
}
\end{equation}
\begin{equation}
\inferrule[PMInL]{
  \hpatmatch{e}{p}{\theta}
}{
  \hpatmatch{\hinl{\tau}{e}}{\hinlp{p}}{\theta}
}
\end{equation}
\begin{equation}
\inferrule[PMInR]{
  \hpatmatch{e}{p}{\theta}
}{
  \hpatmatch{\hinr{\tau}{e}}{\hinrp{p}}{\theta}
}
\end{equation}
\end{subequations}
\end{figure}

\begin{figure}[t]
\fbox{$\hmaymatch{e}{p}$}~~\text{$e$ may match $p$}
\begin{subequations}
\begin{equation}
\inferrule[MMEHole]{ }{
  \hmaymatch{e}{\hehole{u}}
}
\end{equation}
\begin{equation}
\inferrule[MMPair1]{
  \hmaymatch{e_1}{p_1} \\
  \hpatmatch{e_2}{p_2}{\theta_2}
}{
  \hmaymatch{\hpair{e_1}{e_2}}{\hpair{p_1}{p_2}}
}
\end{equation}
\begin{equation}
\inferrule[MMPair2]{
  \hpatmatch{e_1}{p_1}{\theta_1} \\
  \hmaymatch{e_2}{p_2}
}{
  \hmaymatch{\hpair{e_1}{e_2}}{\hpair{p_1}{p_2}}
}
\end{equation}
\begin{equation}
\inferrule[MMPair3]{
  \hmaymatch{e_1}{p_1} \\
  \hmaymatch{e_2}{p_2}
}{
  \hmaymatch{\hpair{e_1}{e_2}}{\hpair{p_1}{p_2}}
}
\end{equation}
\begin{equation}
\inferrule[MMInL]{
  \hmaymatch{e}{p}
}{
  \hmaymatch{\hinl{\tau}{e}}{\hinlp{p}}
}
\end{equation}
\begin{equation}
\inferrule[MMInR]{
  \hmaymatch{e}{p}
}{
  \hmaymatch{\hinr{\tau}{e}}{\hinrp{p}}
}
\end{equation}
\end{subequations}
\end{figure}
\begin{figure}[t]
\fbox{$\hnotmatch{e}{p}$}~~\text{$e$ doesn't match $p$}
\begin{subequations}
\begin{equation}
\inferrule[NMPair1]{
  \hnotmatch{e_1}{p_1}
}{
  \hnotmatch{\hpair{e_1}{e_2}}{\hpair{p_1}{p_2}}
}
\end{equation}
\begin{equation}
\inferrule[NMPair2]{
  \hnotmatch{e_2}{p_2}
}{
  \hnotmatch{\hpair{e_1}{e_2}}{\hpair{p_1}{p_2}}
}
\end{equation}
\begin{equation}
\inferrule[NMConfL]{ }{
  \hnotmatch{\hinr{\tau}{e}}{\hinlp{p}}
}
\end{equation}
\begin{equation}
\inferrule[NMConfR]{ }{
  \hnotmatch{\hinl{\tau}{e}}{\hinrp{p}}
}
\end{equation}
\begin{equation}
\inferrule[NMInjL]{
  \hnotmatch{e}{p}
}{
  \hnotmatch{\hinr{\tau}{e}}{\hinlp{p}}
}
\end{equation}
\begin{equation}
\inferrule[NMInjR]{
  \hnotmatch{e}{p}
}{
  \hnotmatch{\hinl{\tau}{e}}{\hinrp{p}}
}
\end{equation}
\end{subequations}
\end{figure}

\begin{figure}[t]
\fbox{$\isVal{e}$}~~\text{$e$ is a value}
\begin{subequations}
\begin{equation}
\inferrule[VNum]{ }{
  \isVal{\hnum{n}}
}
\end{equation}
\begin{equation}
\inferrule[VTriv]{ }{
  \isVal{\htriv}
}
\end{equation}
\begin{equation}
\inferrule[VLam]{ }{
  \isVal{\hlam{x}{\tau}{e}}
}
\end{equation}
\begin{equation}
\inferrule[VPair]{
  \isVal{e_1} \\
  \isVal{e_2}
}{\isVal{\hpair{e_1}{e_2}}}
\end{equation}
\begin{equation}
\inferrule[VInL]{
  \isVal{e}
}{
  \isVal{\hinl{\tau}{e}}
}
\end{equation}
\begin{equation}
\inferrule[VinR]{
  \isVal{e}
}{
  \isVal{\hinr{\tau}{e}}
}
\end{equation}
\end{subequations}
\end{figure}

\begin{figure}[t]
\fbox{$\isIndet{e}$}~~\text{$e$ is indeterminate}
\begin{subequations}
\begin{equation}
\inferrule[IAp1]{
  \isIndet{e_1}
}{
  \isIndet{\hap{e_1}{e_2}}
}
\end{equation}
\begin{equation}
\inferrule[IAp2]{
  \isVal{e_1} \\ \isIndet{e_2}
}{
  \isIndet{\hap{e_1}{e_2}}
}
\end{equation}
\begin{equation}
\inferrule[IPair1]{
  \isIndet{e_1}
}{
  \isIndet{\hpair{e_1}{e_2}}
}
\end{equation}
\begin{equation}
\inferrule[IPair2]{
  \isVal{e_1} \\
  \isIndet{e_2}
}{
  \isIndet{\hpair{e_1}{e_2}}
}
\end{equation}
\begin{equation}
\inferrule[IInL]{
  \isIndet{e}
}{
  \isIndet{\hinl{\tau}{e}}
}
\end{equation}
\begin{equation}
\inferrule[IInR]{
  \isIndet{e}
}{
  \isIndet{\hinr{\tau}{e}}
}
\end{equation}
\begin{equation}
\inferrule[IExpMatch]{
  \isIndet{e}
}{
  \isIndet{\hmatch{e}{rs}}
}
\end{equation}
\begin{equation}
\inferrule[IMayMatch]{
  \isVal{e} \\
  \hmaymatch{e}{p_0}
}{
  \isIndet{\hmatch{e}{\hrules{\hrul{p_0}{e_0}}{rs}}}
}
\end{equation}
\end{subequations}
\end{figure}

\begin{figure}[t]
\fbox{$\isErr{e}$}~~\text{$e$ results in a checked error}
\begin{subequations}
\begin{equation}
\inferrule[EAp1]{
  \isErr{e_1}
}{
  \isErr{\hap{e_1}{e_2}}
}
\end{equation}
\begin{equation}
\inferrule[EAp2]{
  \isVal{e_1} \\ \isErr{e_2}
}{
  \isErr{\hap{e_1}{e_2}}
}
\end{equation}
\begin{equation}
\inferrule[EPair1]{
  \isErr{e_1}
}{
  \isErr{\hpair{e_1}{e_2}}
}
\end{equation}
\begin{equation}
\inferrule[EPair2]{
  \isVal{e_1} \\
  \isErr{e_2}
}{
  \isErr{\hpair{e_1}{e_2}}
}
\end{equation}
\begin{equation}
\inferrule[EInL]{
  \isErr{e}
}{
  \isErr{\hinl{\tau}{e}}
}
\end{equation}
\begin{equation}
\inferrule[EInR]{
  \isErr{e}
}{
  \isErr{\hinr{\tau}{e}}
}
\end{equation}
\begin{equation}
\inferrule[EExpMatch]{
  \isErr{e}
}{
  \isErr{\hmatch{e}{rs}}
}
\end{equation}
\begin{equation}
\inferrule[EExhMatch]{
  \isVal{e}
}{
  \isErr{\hmatch{e}{\cdot}}
}
\end{equation}
\end{subequations}
\end{figure}

\begin{figure}[t]
\fbox{$e \mapsto e'$}~~\text{$e$ takes an instruction transition to $e'$}
\begin{subequations}
\begin{equation}
\inferrule[ITAp1]{
  \htrans{e_1}{e_1'}
}{
  \htrans{\hap{e_1}{e_2}}{\hap{e_1'}{e_2}}
}
\end{equation}
\begin{equation}
\inferrule[ITAp2]{
  \isVal{e_1} \\
  \htrans{e_2}{e_2'}
}{
  \htrans{\hap{e_1}{e_2}}{\hap{e_1}{e_2'}}
}
\end{equation}
\begin{equation}
\inferrule[ITAP]{
  \isVal{e_2}
}{
  \hap{\hlam{x}{\tau}{e_1}}{e_2} \mapsto
    [e_2/x]e_1
}
\end{equation}
\begin{equation}
\inferrule[ITPair1]{
  \htrans{e_1}{e_1'}
}{
  \htrans{\hpair{e_1}{e_2}}{\hpair{e_1'}{e_2}}
}
\end{equation}
\begin{equation}
\inferrule[ITPair2]{
  \isVal{e_1} \\
  \htrans{e_2}{e_2'}
}{
  \htrans{\hpair{e_1}{e_2}}{\hpair{e_1}{e_2'}}
}
\end{equation}
\begin{equation}
\inferrule[ITInL]{
  \htrans{e}{e'}
}{
  \htrans{\hinl{\tau}{e}}{\hinl{\tau}{e'}}
}
\end{equation}
\begin{equation}
\inferrule[ITInR]{
  \htrans{e}{e'}
}{
  \htrans{\hinr{\tau}{e}}{\hinr{\tau}{e'}}
}
\end{equation}
\begin{equation}
\inferrule[ITExpMatch]{
  \htrans{e}{e'}
}{
  \htrans{\hmatch{e}{rs}}{\hmatch{e'}{rs}}
}
\end{equation}
\begin{equation}
\inferrule[ITSuccMatch]{
  \isVal{e} \\
  \hpatmatch{e}{p_0}{\theta}
}{
  \htrans{
    \hmatch{e}{\hrules{\hrul{p_0}{e_0}}{rs}}
  }{
    [\theta](e_0)
  }
}
\end{equation}
\begin{equation}
\inferrule[ITFailMatch]{
  \isVal{e} \\
  \hnotmatch{e}{p_0} \\
}{
  \htrans{
    \hmatch{e}{\hrules{\hrul{p_0}{e_0}}{rs}}
  }{
    \hmatch{e}{rs}
  }
}
\end{equation}
\end{subequations}
\end{figure}


\begin{thm}[Preservation]
  \label{thrm:preservation}
  If $\hexptyp{\cdot}{e}{\tau}$ and $\htrans{e}{e'}$
  then $\hexptyp{\cdot}{e'}{\tau}$
\end{thm}

\begin{lem}[Substitution]
  \label{lemma:substitution}
  If $\hexptyp{\Gamma, x : \tau}{e_0}{\tau_0}$ and $\hexptyp{\Gamma}{e}{\tau}$
  then $\hexptyp{\Gamma}{[e/x]e_0}{\tau_0}$
\end{lem}

\begin{lem}[Simultaneous Substitution]
  \label{lemma:simult-substitution}
  If $\hexptyp{\Gamma \uplus \Gamma'}{e}{\tau}$ and $\hsubstyp{\theta}{\Gamma'}$
  then $\hexptyp{\Gamma}{[\theta]e}{\tau}$
\end{lem}
Proof by induction on the derivation of $\hexptyp{\Gamma \uplus \Gamma'}{e}{\tau}$.

\begin{lem}[Substitution Typing]
  \label{lemma:subs-typing}
  If $\hpatmatch{e}{p}{\theta}$ and $\hexptyp{\cdot}{e}{\tau}$ and $\hpattyp{p}{\tau}{\Gamma}$
  then $\hsubstyp{\theta}{\Gamma}$
\end{lem}
Proof by induction on the derivation of $\hpatmatch{e}{p}{\theta}$.

To apply this lemma in ITSuccMatch case, first apply inversion lemma on premise of preservation theorem,
\emph{i.e.}, $\hexptyp{\cdot}{\hpatmatch{e}{p}{\theta}}{\trul{\tau_1}{\tau_2}}$

\begin{thm}[Progress]
  \label{thrm:progrs}
  If $\hexptyp{\cdot}{e}{\tau}$
  then $\isVal{e}$ or $\isErr{e}$ or $\isIndet{e}$ or $\htrans{e}{e'}$ for some $e'$
\end{thm}

\begin{lem}[Matching Progress]
  \label{lemma:match-progress}
  If $\hexptyp{\cdot}{e}{\tau}$ and $\hpattyp{p}{\tau}{\Gamma}$
  then $\hnotmatch{e}{p}$ or $\hmaymatch{e}{p}$ or $\hpatmatch{e}{p}{\theta}$ for some $\theta$
\end{lem}
Proof by induction on two premises.

\begin{thm}[Stepping Determinism]
  \label{thrm:step-determinism}
  If $\htrans{e}{e'}$ and $\htrans{e}{e''}$ then $e' = e''$
\end{thm}

\begin{thm}[Determinism]
  \label{thrm:determinism}
  If $\hexptyp{\cdot}{e}{\tau}$ then exactly one of the following holds
  \begin{enumerate}
    \item $\isVal{e}$
    \item $\isErr{e}$
    \item $\isIndet{e}$
    \item $\htrans{e}{e'}$ for some unique $e'$
  \end{enumerate}
\end{thm}