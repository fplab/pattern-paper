\section{Labeled Sums}

We can generalize binary sums (with holes) to labeled sums (with holes).

(syntax)


A labeled sum consists of a set of label-type pairings.
For such a pairing, we call the label a ``tag,'' and the type its ``argument type.''
The syntax for labeled sums introduces two new sorts: tags and sum bodies.
Each sort requires a new hole form: sum body holes, and tag holes.
Sum body holes bridge the gap between empty sums and non-empty but fully-indeterminate sums, enabling gap-free manipulation of sum types with no tags defined.
Tag holes, on the other hand, are necessary for gap-free manipulation of tag declarations (in types) and uses (in expressions).
Correspondingly, there are two distinct forms of tag hole: duplicate tags (in types) and undefined tags (in expressions).
Since the syntax of tags is independent of sort, we also include a single form of hole for invalid tag names.

TODO: introduce exps: intro vs elim forms

(statics)

The presence of holes poses interesting challenges for the formal treatment of labeled sums.
Consistency of labeled sum types is more nuanced than for binary sums.
For example, consistency of sums of the same cardinality depends on whether there is a consistent mapping between tags as well as argument types.
(Say something about the algorithmic complexity of naive vs our efficient implementation? Cite even more efficient approaches?)

consistency: (efficient?) matching (we build a bipartite graph of all possible consistent matchings and use Hopcroft-Karp maximum-cardinality matching algo to determine whether a maximum matching exists)

% dont need: elided sums: easy to synthesize and enable a broader notion of consistency
% dont need: casts between sums: (we do it componentwise)
% dont need: (matched finite sums?)

(dynamics)

substitution does not affect tags directly (but does affect argument types)

injection constraints (e.g. "not inl")
+ any other modifications to the formalism to make labeled sums work

any interesting details w.r.t. exhaustiveness or redundancy?
- duals
- see if existing definitions or operations need a richer understanding of sums than binary
