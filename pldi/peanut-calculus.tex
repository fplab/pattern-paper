\section{Peanut: The Core Calculus}
% We now formalize the intuitions developed in \autoref{sec:examples} as a core calculus that supports pattern matching with typed holes in both expression and pattern position.
Traditional matching of an expression to a pattern leads to success or failure (or divergence in the lazy setting). In the setting of typed holes, there is an additional outcome: \emph{indeterminate}.
An indeterminate match occurs when the pattern or expression contains holes and the match would succeed or fail depending on how the programmer modifies those holes. Such dependence on yet unspecified programmer input similarly colors the outcomes of corresponding coverage and redundancy checkers: a collection of patterns may be \emph{indeterminately inexhaustive}, and a pattern may be \emph{indeterminately redundant} with respect to some preceding collection of patterns.

In this section, we formalize these concepts in a core calculus called Peanut that supports pattern matching with typed holes in both expression and pattern position. 
We begin with the syntax of the calculus in \autoref{sec:Syntax}\todo{change autoref settings to say Sec. 3.1 instead of subsection 3.1}.
In \autoref{sec:dynamics}, we define the dynamic semantics as a small-step operational semantics with support for evaluating incomplete programs.
Next, in \autoref{sec:statics}, we define a corresponding static semantics as a type assignment system together with a match constraint language that we use to reason hypothetically about exhaustiveness and redundancy in the presence of
holes.
Finally, in \autoref{sec:algorithm}, we give an algorithmic formulation of the exhaustiveness and redundancy checker and prove it sound with respect to the declarative formulation.

% !TEX root= pattern-paper.tex

\begin{figure}[ht]
    \centering
    \begin{minipage}{.5\linewidth}
  $\arraycolsep=4pt\begin{array}{lll}
    e & ::= &
      x ~\vert~
      \hnum{n} \\
      & ~\vert~ &
      \hlam{x}{\tau}{e} ~\vert~
      \hap{e_1}{e_2} \\
      & ~\vert~ &
      \hpair{e_1}{e_2} ~\vert~
      \hfst{e} ~\vert~ \hsnd{e} \\
      & ~\vert~ &
      \hinl{\tau}{e} ~\vert~
      \hinr{\tau}{e} \\
      & ~\vert~ &
      \hmatch{e}{\hat{rs}} \\
      & ~\vert~ &
      \hehole{u} ~\vert~
      \hhole{e}{u} \\
    \end{array}$
    \end{minipage}%
    \begin{minipage}{.5\linewidth}
  $\arraycolsep=4pt\begin{array}{lll}
    \tau & ::= &
      \tnum ~\vert~
      \tarr{\tau_1}{\tau_2} ~\vert~
      \tprod{\tau_1}{\tau_2} ~\vert~
      \tsum{\tau_1}{\tau_2} \\
    \hat{rs} & ::= &
      \zrulsP{rs}{r}{rs} \\
    rs & ::= &
      \cdot ~\vert~ \hrulesP{r}{rs'} \\
    r & ::= &
      \hrul{p}{e} \\
    p & ::= &
      x ~\vert~
      \_ ~\vert~
      \hnum{n} ~\vert~
      \hpair{p_1}{p_2} \\
      & ~\vert~ &
      \hinlp{p} ~\vert~
      \hinrp{p} ~\vert~
      \heholep{w} ~\vert~
      \hholep{p}{w}{\tau}
    \end{array}$
    \end{minipage}
\caption{Syntax}
\label{fig:syntax}
\end{figure}
% !TEX root= pattern-paper.tex

\begin{figure}[ht]

\judgbox{\rmpointer{\zrules} = rs}
        {$rs$ can be obtained by erasing pointer from $\zrules$}
\begin{align*}
  \rmpointer{\zruls{\cdot}{r}{rs}} &= \hrules{r}{rs} \\
  \rmpointer{\zruls{\hrulesP{r'}{rs'}}{r}{rs}} &= \hrules{r'}{\rmpointer{\zruls{rs'}{r}{rs}}}
\end{align*}

\caption{Pointer Eraser flattens the rules with pointer.}
\label{fig:pointer-eraser}
\end{figure}


\subsection{Syntax}
\label{sec:Syntax}
\figurename~\ref{fig:syntax} presents the syntax of Peanut.
Peanut is based on the internal language of Hazelnut Live, a typed lambda calculus featuring holes in expression position \cite{DBLP:journals/pacmpl/OmarVCH19}.
We choose numbers as the base type and add binary sums and binary products so that we have interesting
patterns to consider. We also remove the machinery related to gradual typing (casts and failed casts) to focus our attention on pattern matching in particular. Most forms are standard (we base our formulation on \cite{Harper2012}). We include functions, function application, pairs, explicit projection operators (for reasons we will consider below), and left and right injections, which are the introductory forms for sum types. Functions and injections include type annotations so that we can define a simple type assignment system. The forms of particular interest here are holes and match expressions.

Empty expression holes are written $\hehole{u}$ and non-empty expression holes, which serve as membranes around type inconsistencies, are written $\hholep{e}{u}{\tau}$. Similarly, empty pattern holes are written $\heholep{w}$ and non-empty pattern holes, which are analogous, are written $\hhole{p}{w}{\tau}$. Here, $u$ are expression hole identifiers and $w$ are pattern hole identifiers.
We are modeling an internal language, so we assume that hole identifiers 
correspond to unique holes in the user's original program, which we do not model here. We do not impose a uniqueness constraint, however, because a hole can 
appear multiple times during evaluation due to substitution.

A match expression, $\hmatch{e}{\zrules}$, 
consists of a scrutinee, $e$, and a zipper of rules, $\zrules$, i.e. a sequence of one or more rules with a pointer marking the rule currently being considered during evaluation (we assume that the marker is on the first rule initially). Syntactically, this is a triple, $\zruls{rs_{pre}}{r}{rs_{post}}$, consisting of a prefix rule sequence, $rs_{pre}$, a current rule, $r$, and a suffix rule sequence, $rs_{post}$. We can erase the pointer using the pointer erasure operator, $\rmpointer{\zrules}$, defined in \autoref{fig:pointer-eraser}. 
Each rule, $r$, consists of a pattern and an expression, which we call the branch expression.



\subsection{Dynamic Semantics}\label{sec:dynamics}

% !TEX root= pattern-paper.tex

\begin{figure}[ht]

\judgbox{\htrans{e}{e'}}{$e$ takes a step to $e'$}

\begin{mathpar}
\Infer{\ITHole}{
  \htrans{e}{e'}
}{
  \htrans{\hhole{e}{u}}{\hhole{e'}{u}}
}

\Infer{\ITApFun}{
  \htrans{e_1}{e_1'}
}{
  \htrans{\hap{e_1}{e_2}}{\hap{e_1'}{e_2}}
}

\Infer{\ITApArg}{
  \isFinal{e_1} \\
  \htrans{e_2}{e_2'}
}{
  \htrans{\hap{e_1}{e_2}}{\hap{e_1}{e_2'}}
}

\Infer{\ITAp}{
  \isFinal{e_2}
}{
  \hap{\hlam{x}{\tau}{e_1}}{e_2} \mapsto
    [e_2/x]e_1
}

\Infer{\ITPairL}{
  \htrans{e_1}{e_1'}
}{
  \htrans{\hpair{e_1}{e_2}}{\hpair{e_1'}{e_2}}
}

\Infer{\ITPairR}{
  \isFinal{e_1} \\
  \htrans{e_2}{e_2'}
}{
  \htrans{\hpair{e_1}{e_2}}{\hpair{e_1}{e_2'}}
}

\Infer{\ITPrl}{
  \isFinal{\hpair{e_1}{e_2}}
}{
  \htrans{\hprl{\hpair{e_1}{e_2}}}{e_1}
}

\Infer{\ITPrr}{
  \isFinal{\hpair{e_1}{e_2}}
}{
  \htrans{\hprr{\hpair{e_1}{e_2}}}{e_2}
}

\Infer{\ITInl}{
  \htrans{e}{e'}
}{
  \htrans{\hinl{\tau}{e}}{\hinl{\tau}{e'}}
}

\Infer{\ITInr}{
  \htrans{e}{e'}
}{
  \htrans{\hinr{\tau}{e}}{\hinr{\tau}{e'}}
}

\Infer{\ITExpMatch}{
  \htrans{e}{e'}
}{
  \htrans{\hmatch{e}{\zrules}}{\hmatch{e'}{\zrules}}
}

\Infer{\ITSuccMatch}{
  \isFinal{e} \\
  \hpatmatch{e}{p_r}{\theta}
}{
  \htrans{
    \hmatch{e}{\zruls{rs_{pre}}{\hrulP{p_r}{e_r}}{rs_{post}}}
  }{
    [\theta](e_r)
  }
}

\Infer{\ITFailMatch}{
  \isFinal{e} \\
  \hnotmatch{e}{p_r}
}{
  \htrans{
    \hmatch{e}{\zruls{rs}{\hrulP{p_r}{e_r}}{\hrulesP{r'}{rs'}}}
  }{
    \hmatch{e}{
      \zruls{
        \rmpointer{\zruls{rs}{\hrulP{p_r}{e_r}}{\cdot}}
      }{r'}{rs'}
    }
  }
}
\end{mathpar}

  \caption{Stepping}
  \label{fig:step}
\end{figure}

% !TEX root= pattern-paper.tex

\begin{figure}[th]

\judgbox{\isVal{e}}{$e$ is a value}

\begin{mathpar}
\Infer{\VNum}{ }{
  \isVal{\hnum{n}}
}

\Infer{\VLam}{ }{
  \isVal{\hlam{x}{\tau}{e}}
}

\Infer{\VPair}{
  \isVal{e_1} \\
  \isVal{e_2}
}{
  \isVal{\hpair{e_1}{e_2}}
}

\Infer{\VInl}{
  \isVal{e}
}{
  \isVal{\hinl{\tau}{e}}
}

\Infer{\VInr}{
  \isVal{e}
}{
  \isVal{\hinr{\tau}{e}}
}
\end{mathpar}

\judgbox{\isIndet{e}}{$e$ is indeterminate}

\begin{mathpar}
\Infer{\IEHole}{ }{
  \isIndet{\hehole{u}}
}

\Infer{\IHole}{
  \isFinal{e}
}{
  \isIndet{\hhole{e}{u}}
}

\Infer{\IAp}{
  \isIndet{e_1} \\ \isFinal{e_2}
}{
  \isIndet{\hap{e_1}{e_2}}
}

\Infer{\IPairL}{
  \isIndet{e_1} \\ \isVal{e_2}
}{
  \isIndet{\hpair{e_1}{e_2}}
}

\Infer{\IPairR}{
  \isVal{e_1} \\ \isIndet{e_2}
}{
  \isIndet{\hpair{e_1}{e_2}}
}

\Infer{\IPair}{
  \isIndet{e_1} \\ \isIndet{e_2}
}{
  \isIndet{\hpair{e_1}{e_2}}
}

\Infer{\IFst}{
  \isIndet{e} \\ e \neq \hpair{e_1}{e_2}
}{
  \isIndet{\hfst{e}}
}

\Infer{\ISnd}{
  \isIndet{e} \\ e \neq \hpair{e_1}{e_2}
}{
  \isIndet{\hsnd{e}}
}

\Infer{\IInl}{
  \isIndet{e}
}{
  \isIndet{\hinl{\tau}{e}}
}

\Infer{\IInr}{
  \isIndet{e}
}{
  \isIndet{\hinr{\tau}{e}}
}

\Infer{\IMatch}{
  \isFinal{e} \\
  \hmaymatch{e}{p_r}
}{
  \isIndet{
    \hmatch{e}{\zruls{rs_{pre}}{\hrulP{p_r}{e_r}}{rs_{post}}}
  }
}
\end{mathpar}

\judgbox{\isFinal{e}}{$e$ is final}

\begin{mathpar}
\Infer{\FVal}{
  \isVal{e}
}{
  \isFinal{e}
}

\Infer{\FIndet}{
  \isIndet{e}
}{
  \isFinal{e}
}
\end{mathpar}

  \caption{Final expressions}
  \label{fig:final}
\end{figure}


Hazelnut Live \cite{DBLP:journals/pacmpl/OmarVCH19} features a dynamic semantics capable of evaluating expressions with holes.
Rather than stopping upon encountering a hole, evaluation proceeds around it, taking all evaluation steps that do not depend on the eventual contents of the hole.
\autoref{fig:step} presents the corresponding stepping judgment $\htrans{e}{e'}$ of Peanut, many of the rules of which are directly adapted from Hazelnut Live.\footnote{Hazelnut Live is a contextual small-step semantics, whereas we choose to define a standard small-step semantics for Peanut.}
The rules of interest in this work are those concerning match expressions, namely (match rules\todo{}).
These inference rules use the zipper representation of the Peanut syntax to record the pattern matching progress.

We use zipper rules of the form $\zrulsP{rs}{r}{rs}$ in match expressions to represent the intermediate states when we are performing pattern matching.
The former $rs$ represents the preceding rules that has already been considered and fails to match the scrutinee; the rule $r$ in the middle represents the current rule being considered; the latter $rs$ represents the remaining rules to be considered.
\figurename~\ref{fig:pointer-eraser} defines a helper function to flatten the zipper rules.
If we need to move the rule pointer to the next rule, we can append the current rule after the preceding rules, and regard the initial rule in the remaining rule as the new current rule.
The conclusion of Rule \ITFailMatch demonstrates how it works.
\todo{edit this paragraph, bringing up successful and failed matches in connection to traditional pattern matching. possibly note some details regarding finality of match scrutinee. see commented text.}
% \figurename~\ref{fig:step} defines the stepping judgment
% $\htrans{e}{e'}$. We will focus on stepping judgments of match expressions.
% Rule \ITExpMatch specifies that the match expression can take a step as its its
% scrutinee can take a step. Note that Rule \ITFailMatch and Rule \ITSuccMatch
% share the premise, $\isFinal{e}$, which is defined in \figurename~\ref{fig:final}. It
% means that expression $e$ is \textit{final} in the sense that it is either already a
% \textit{value} or \textit{indeterminate}, \ie, cannot be evaluated further due to unfilled holes
% \cite{DBLP:journals/pacmpl/OmarVCH19}. And only when the scrutinee is final,
% shall we consider the constinuent rules of the match expression in order. As in
% the examples shown in \listfigurename~\ref{fig:exp-hole-step},\ref{fig:pat-hole-step},

% \begin{itemize}
%   \item
%     if the final scrutinee does not match the pattern in the current rule,
%     then we move the pointer to the next rule (Rule \ITFailMatch)

%   \item
%     if the final scrutinee does match the pattern in the current rule, 
%     then we take the emitted substitution $\theta$ and apply it on the subexpression of that rule (Rule \ITSuccMatch)

%   \item 
%     if the final scrutinee may or may not match the pattern in the current rule,
%     then the match expression is said to be indeterminate (Rule \IMatch)
% \end{itemize}


The final result $e$ of evaluation may be, as characterized by the judgment $\isFinal{e}$ in \autoref{fig:final}, either a value ($\isVal{e}$) or an \emph{indeterminate} form ($\isIndet{e}$) whose continued evaluation depends on the remaining holes' contents. For example, Rules \IAp and \IEHole together derive that a function application with a hole in function position is indeterminate.

% !TEX root= pattern-paper.tex

\begin{figure}[!b]

\judgbox{
  \hpatmatch{e}{p}{\theta}
}{
  $e$ matches $p$, emitting $\theta$
}

\begin{mathpar}
\Infer{\MVar}{ }{
  \hpatmatch{e}{x}{e / x}
}

\Infer{\MWild}{ }{
  \hpatmatch{e}{\_}{\cdot}
}

\Infer{\MNum}{ }{
  \hpatmatch{\hnum{n}}{\hnum{n}}{\cdot}
}

\Infer{\MPair}{
  \hpatmatch{e_1}{p_1}{\theta_1} \\
  \hpatmatch{e_2}{p_2}{\theta_2}
}{
  \hpatmatch{\hpair{e_1}{e_2}}{\hpair{p_1}{p_2}}{\theta_1 \uplus \theta_2}
}

\Infer{\MInl}{
  \hpatmatch{e}{p}{\theta}
}{
  \hpatmatch{\hinl{\tau}{e}}{\hinlp{p}}{\theta}
}

\Infer{\MInr}{
  \hpatmatch{e}{p}{\theta}
}{
  \hpatmatch{\hinr{\tau}{e}}{\hinrp{p}}{\theta}
}

\Infer{\MNotIntroPair}{
  \notIntro{e} \\
  \hpatmatch{\hprl{e}}{p_1}{\theta_1} \\
  \hpatmatch{\hprr{e}}{p_2}{\theta_2}
}{
  \hpatmatch{e}{\hpair{p_1}{p_2}}{\theta_1 \uplus \theta_2}
}
\end{mathpar}

\judgbox{
  \hnotmatch{e}{p}
}{
  $e$ does not match $p$
}

\begin{mathpar}

\Infer{\NMNum}{
  n_1 \neq n_2
}{
  \hnotmatch{\hnum{n_1}}{\hnum{n_2}}
}

\Infer{\NMPairL}{
  \hnotmatch{e_1}{p_1}
}{
  \hnotmatch{\hpair{e_1}{e_2}}{\hpair{p_1}{p_2}}
}

\Infer{\NMPairR}{
  \hnotmatch{e_2}{p_2}
}{
  \hnotmatch{\hpair{e_1}{e_2}}{\hpair{p_1}{p_2}}
}

\Infer{\NMConfL}{ }{
  \hnotmatch{\hinr{\tau}{e}}{\hinlp{p}}
}

\Infer{\NMConfR}{ }{
  \hnotmatch{\hinl{\tau}{e}}{\hinrp{p}}
}

\Infer{\NMInl}{
  \hnotmatch{e}{p}
}{
  \hnotmatch{\hinr{\tau}{e}}{\hinlp{p}}
}

\Infer{\NMInr}{
  \hnotmatch{e}{p}
}{
  \hnotmatch{\hinl{\tau}{e}}{\hinrp{p}}
}
\end{mathpar}

\judgbox{
  \hmaymatch{e}{p}
}{
  $e$ indeterminately matches $p$
}

\begin{mathpar}
\Infer{\MMEHole}{ }{
  \hmaymatch{e}{\heholep{w}}
}

\Infer{\MMHole}{ }{
  \hmaymatch{e}{\hholep{p}{w}{\tau}}
}
\\
\Infer{\MMNotIntro}{
  \notIntro{e} \\
  \refutable{p}
}{
  \hmaymatch{e}{p}
}

\Infer{\MMPairL}{
  \hmaymatch{e_1}{p_1} \\
  \hpatmatch{e_2}{p_2}{\theta_2}
}{
  \hmaymatch{\hpair{e_1}{e_2}}{\hpair{p_1}{p_2}}
}

\Infer{\MMPairR}{
  \hpatmatch{e_1}{p_1}{\theta_1} \\
  \hmaymatch{e_2}{p_2}
}{
  \hmaymatch{\hpair{e_1}{e_2}}{\hpair{p_1}{p_2}}
}

\Infer{\MMPair}{
  \hmaymatch{e_1}{p_1} \\
  \hmaymatch{e_2}{p_2}
}{
  \hmaymatch{\hpair{e_1}{e_2}}{\hpair{p_1}{p_2}}
}

\Infer{\MMInl}{
  \hmaymatch{e}{p}
}{
  \hmaymatch{\hinl{\tau}{e}}{\hinlp{p}}
}

\Infer{\MMInr}{
  \hmaymatch{e}{p}
}{
  \hmaymatch{\hinr{\tau}{e}}{\hinrp{p}}
}
\end{mathpar}

\caption{Three possible outcomes of pattern matching}
\label{fig:patmatch}
\end{figure}


Peanut extends this concept of indeterminacy to pattern matching.
\autoref{fig:patmatch} presents the three possible outcomes of matching an expression $e$ to a pattern $p$.
First, the judgment $\hpatmatch{e}{p}{\theta}$ denotes a successful match as witnessed by the substitution $\theta$ consisting of the variables bound in $p$. Second, the judgment $\hnotmatch{e}{p}$ denotes that $e$ does not match $p$. Third, the judgment $\hmaymatch{e}{p}$ indicates an \emph{indeterminate match} due to the presence of holes in $e$ or $p$.
\todo{add mention of assupmtion of finality of expression}
\todo{recall how we saw how successful/failed matches contextualize within stepping judgment, note how indeterminate match appears in indet judgment}
% \figurename~\ref{fig:patmatch} defines three judgments corresponding to the
% three possible results of pattern matching. We consider a final scrutinee $e$
% and a pattern $p$ that are of the same type (see \autoref{sec:statics}). \autoref{fig:notintro} defines the judgment for expressions that cannot be values by simply looking at its form. Hence, when some final expression is of such form, they must be indeterminate. Besides, there is not need to recursively consider its inner expression during pattern matching.


\todo{add paragraph opener here bridging to going into some detail regarding specifics of matching}

% !TEX root= pattern-paper.tex

\begin{figure}[!ht]
\judgbox{
  \refutable{p}
}{$p$ is refutable}

\begin{mathpar}
\Infer{\RNum}{ }{
  \refutable{\hnum{n}}
}

\Infer{\REHole}{ }{
  \refutable{\heholep{w}}
}

\Infer{\RHole}{ }{
  \refutable{\hholep{p}{w}{\tau}}
}

\Infer{\RInl}{ }{
  \refutable{\hinlp{p}}
}

\Infer{\RInr}{ }{
  \refutable{\hinrp{p}}
}

\Infer{\RPairL}{
  \refutable{p_1}
}{
  \refutable{\hpair{p_1}{p_2}}
}

\Infer{\RPairR}{
  \refutable{p_2}
}{
  \refutable{\hpair{p_1}{p_2}}
}
\end{mathpar}
\caption{Refutable Patterns}
\label{fig:refutable}
\end{figure}
% !TEX root= pattern-paper.tex

\begin{figure}[ht]
\judgbox{
  \notIntro{e}
}{
  $e$ cannot be a value syntactically
}

\begin{mathpar}
\Infer{NVEHole}{ }{
  \notIntro{\hehole{u}}
}

\Infer{NVHole}{ }{
  \notIntro{\hhole{e}{u}}
}

\Infer{NVAp}{ }{
  \notIntro{\hap{e_1}{e_2}}
}

\Infer{NVMatch}{ }{
  \notIntro{\hmatch{e}{\zrules}}
}

\Infer{NVPrl}{ }{
  \notIntro{\hprl{e}}
}

\Infer{NVPrr}{ }{
  \notIntro{\hprr{e}}
}
\end{mathpar}

  \caption{Expressions that are syntactically not values}
  \label{fig:notintro}
\end{figure}

\todo{this paragraph below is floating content, needs to be worked into its surroundings}
The judgment $\hpatmatch{e}{p}{\theta}$ denotes that $e$ matches $p$, as witnessed by
the substitution $\theta$ defined on the variables in $p$, while the judgment
$\hnotmatch{e}{p}$ denotes that $e$ does not match $p$. Notably, we introduce the
third possibility $\hmaymatch{e}{p}$ --- $e$ may match $p$, which is also the
\textit{key concept}
\todo{d: the concept itself should be emphasized (if at all), not the phrase ``key concept''}
in pattern matching with holes. Rules \MMEHole and \MMHole
specify that any final expression may match a pattern hole, regardless of
whether it is empty or non-empty. On the other hand, Rule \MMNotIntro specifies that
an indeterminate expression may match a pattern $p$ only when $p$ is refutable
(see \figurename~\ref{fig:refutable}).
It is because any final expressions
must match an irrefutable pattern of the same type, and it does not make sense to allow
$\hpatmatch{e}{p}{\theta}$ and $\hmaymatch{e}{p}$ to be derivable at the same time.
Actually,
the three possible results of pattern matching are mutually exclusive and at
least one of judgments can be derived under certain assumptions, as shown in the
following lemma.

The judgment $\hexptyp{\Gamma}{\Delta}{e}{\tau}$ specifies that expression $e$ is well-typed. And the judgment $\hpattyp{p}{\tau}{\Gamma}{\Delta}$ specifies that we can assign type $\tau$ to pattern $p$.
\begin{lemma}[Matching Determinism]
  \label{lemma:match-determinism}
  If $\isFinal{e}$ and $\hexptyp{\cdot}{\Delta}{e}{\tau}$ and $\hpattyp{p}{\tau}{\Gamma}{\Delta}$ then exactly one of the following holds
  \begin{enumerate}
    \item $\hpatmatch{e}{p}{\theta}$ for some $\theta$
    \item $\hmaymatch{e}{p}$
    \item $\hnotmatch{e}{p}$
  \end{enumerate}
\end{lemma}

The determinism of pattern matching ensures the determinism of dynamic semantics.

\begin{theorem}[Determinism]
  \label{theorem:determinism}
  If $\hexptyp{\cdot}{\Delta}{e}{\tau}$ then exactly one of the following holds
  \begin{enumerate}
    \item $\isVal{e}$
    \item $\isIndet{e}$
    \item $\htrans{e}{e'}$ for some unique $e'$
  \end{enumerate}
\end{theorem}
\todo{prove unique e'?}

\subsection{Static Semantics}\label{sec:statics}

% !TEX root= pattern-paper.tex

\begin{figure}[ht]
\judgbox{
  \hexptyp{\Gamma}{\Delta}{e}{\tau}
}{
  $e$ is of type \(\tau\)
}
  \begin{mathpar}
%  \Infer{\TVar}{ }{
%    \hexptyp{\Gamma, x:\tau}{\Delta}{x}{\tau}
%  }
%
%  \Infer{\TEHole}{ }{
%    \hexptyp{\Gamma}{\Delta, u::\tau}{\hehole{u}}{\tau}
%  }
%
%  \Infer{\THole}{
%    \hexptyp{\Gamma}{\Delta, u::\tau}{e}{\tau'}
%  }{
%    \hexptyp{\Gamma}{\Delta, u::\tau}{\hhole{e}{u}}{\tau}
%  }
%
%  \Infer{\TNum}{ }{
%    \hexptyp{\Gamma}{\Delta}{\hnum{n}}{\tnum}
%  }
%
%  \Infer{\TLam}{
%    \hexptyp{\Gamma, x:\tau_1}{\Delta}{e}{\tau_2}
%  }{
%    \hexptyp{\Gamma}{\Delta}{\hlam{x}{\tau_1}{e}}{\tarr{\tau_1}{\tau_2}}
%  }
%
%  \Infer{\TAp}{
%    \hexptyp{\Gamma}{\Delta}{e_1}{\tarr{\tau_2}{\tau}} \\
%    \hexptyp{\Gamma}{\Delta}{e_2}{\tau_2}
%  }{
%    \hexptyp{\Gamma}{\Delta}{\hap{e_1}{e_2}}{\tau}
%  }
%
%  \Infer{\TPair}{
%    \hexptyp{\Gamma}{\Delta}{e_1}{\tau_1} \\
%    \hexptyp{\Gamma}{\Delta}{e_2}{\tau_2}
%  }{
%    \hexptyp{\Gamma}{\Delta}{\hpair{e_1}{e_2}}{\tprod{\tau_1}{\tau_2}}
%  }
%
%  \Infer{\TFst}{
%    \hexptyp{\Gamma}{\Delta}{e}{\tprod{\tau_1}{\tau_2}}
%  }{
%    \hexptyp{\Gamma}{\Delta}{\hfst{e}}{\tau_1}
%  }
%  
%  \Infer{\TSnd}{
%    \hexptyp{\Gamma}{\Delta}{e}{\tprod{\tau_1}{\tau_2}}
%  }{
%    \hexptyp{\Gamma}{\Delta}{\hsnd{e}}{\tau_2}
%  }
%  
%  \Infer{\TInl}{
%    \hexptyp{\Gamma}{\Delta}{e}{\tau_1}
%  }{
%    \hexptyp{\Gamma}{\Delta}{\hinl{\tau_2}{e}}{\tsum{\tau_1}{\tau_2}}
%  }
%
%  \Infer{\TInr}{
%    \hexptyp{\Gamma}{\Delta}{e}{\tau_2}
%  }{
%    \hexptyp{\Gamma}{\Delta}{\hinr{\tau_1}{e}}{\tsum{\tau_1}{\tau_2}}
%  }
  
  \Infer{TMatchZPre}{
    \hexptyp{\Gamma}{\Delta}{e}{\tau} \\
    \chrulstyp{\Gamma}{\Delta}{\cfalsity}{\hrules{r}{rs}}{\tau}{\xi}{\tau'} \\
    \csatisfyormay{\ctruth}{\xi}
  }{
  \hexptyp{\Gamma}{\Delta}{\hmatch{e}{\zruls{\cdot}{r}{rs}}}{\tau'}
  }

  \Infer{TMatchNZPre}{
    \hexptyp{\Gamma}{\Delta}{e}{\tau} \\
    \isFinal{e} \\
    \chrulstyp{\Gamma}{\Delta}{\cfalsity}{rs_{pre}}{\tau}{\xi_{pre}}{\tau'} \\
    \chrulstyp{\Gamma}{\Delta}{\xi_{pre}}{\hrules{r}{rs_{post}}}{\tau}{\xi_{rest}}{\tau'} \\
    \cnotsatisfyormay{e}{\xi_{pre}} \\
    \csatisfyormay{\ctruth}{\cor{\xi_{pre}}{\xi_{rest}}}
  }{
    \hexptyp{\Gamma}{\Delta}{\hmatch{e}{\zruls{rs_{pre}}{r}{rs_{post}}}}{\tau'}
  }
  \end{mathpar}

\caption{Match Expression Typing}
\label{fig:exptyp}
\end{figure}


We have already introduced how pattern matching with holes works. Now, we want
to predict the runtime behavior of match expressions through checking
exhaustiveness and redundancy in static semantics. However, in order to do exhaustiveness checking and redundancy
checking of match expressions in \autoref{fig:exhaustiveness},
\ref{fig:redundancy}, we need to predict the result of pattern matching in static
semantics. We start by introducing \textit{match constraint language}, which
extends the idea in \cite{Harper2012}. Then, we build a similar type system to
\cite{DBLP:journals/pacmpl/OmarVCH19} by defining typing judgments for both
patterns and expressions. The former generates variable contexts $\Gamma$ and
hole contexts $\Delta$ (see \figurename~\ref{fig:pat-rulestyp}) while the latter 
takes variable contexts $\Gamma$ and hole contexts $\Delta$ as hypothesis (see
\autoref{fig:exptyp}).

\subsubsection{Typing of Expressions and Exhaustiveness Checking} \label{sec:exptyp}

We start by specifying the typing of expressions, particularly, match
expressions. And we will see that the definition of the typing judgments of
match expressions enforce exhaustiveness of the constituent rules.
\autoref{fig:exptyp} defines the typing judgment of expressions.


Rule \TMatchZPre corresponds to the case that we have not started pattern
matching. The first premise specifies that the scrutinee $e$ is of type $\tau$,
and the second premise specifies that the constituent rules $\hrul{r}{rs}$ are not only
well-typed but also transforms a final expression of the same type as the
scrutinee, into a final expression of type $\tau'$. Notably, it generates a
constraint $\xi$ associated with the constituent rules $rs$. Then we use the
third premise $\csatisfyormay{\ctruth}{\xi}$ to ensure that there is at least
one rule whose pattern does match or may match the final scrutinee (see
\autoref{sec:constraint}). In other words, for a well-typed match expression,
it is impossible that the final scrutinee fails to match all the patterns as we
consider rules $rs$ in order.

Rule \TMatchNZPre corresponds to the case that we have already started pattern
matching and have already considered preceding rules $rs_{pre}$. First of all,
the scrutinee should not only be well-typed but also be final. Next, other than
ensuring the exhaustiveness of the constituent rules, we want to make sure that
at least one of the remaining rules $r_{post}$ would be taken. Note
that only when the final scrutinee $e$ cannot match the pattern $p$, \ie,
$\hnotmatch{e}{p}$, can we move the rule pointer. By
\autoref{lemma:match-determinism}, for all patterns $p$ in the preceding
rules, neither $\hpatmatch{e}{p}{\theta}$ nor $\hmaymatch{e}{p}$ is derivable.
Then by \autoref{lemma:const-matching-coherence}, we can derive the premise
in Rule \TMatchNZPre, $\cnotsatisfyormay{e}{\xi_{pre}}$. And thus, the type of
the match expression is preserved (\autoref{theorem:preservation})as we consider rules in order.
\todo{d: This section needs to be much more clear as to how it relates to exhaustiveness checking. There should be some summarizing idea at the top before getting into the details of the rules. It might help to have some discussion of our overall approach using constraints at the top before talking about the specifics of exhaustiveness or redundancy checks. As it stands, constraints kinda come out of nowhere.}

\subsubsection{Typing of Patterns and Rules, and Redundancy Checking}
\label{sec:pattyp}

% !TEX root= pattern-paper.tex

\begin{figure}[!ht]
  \judgbox{
    \chpattyp{p}{\tau}{\xi}{\Gamma}{\Delta}
  }{
    $p$ is assigned type $\tau$ and emits constraint $\xi$
  }

  \begin{mathpar}
  \Infer{\PTVar}{ }{
    \chpattyp{x}{\tau}{\ctruth}{x : \tau}{\cdot}
  }

  \Infer{\PTWild}{ }{
    \chpattyp{\_}{\tau}{\ctruth}{\cdot}{\cdot}
  }

  \Infer{\PTEHole}{ }{
    \chpattyp{\heholep{w}}{\tau}{\cunknown}{\cdot}{\Delta , w :: \tau}
  }

  \Infer{\PTHole}{
    \chpattyp{p}{\tau}{\xi}{\Gamma}{\Delta , w :: \tau'}
  }{
    \chpattyp{\hholep{p}{w}{\tau}}{\tau'}{\cunknown}
    {\Gamma}{\Delta , w :: \tau'}
  }
  
  \Infer{\PTNum}{ }{
    \chpattyp{\hnum{n}}{\tnum}{\cnum{n}}{\cdot}{\Delta}
  }

  \Infer{\PTInl}{
    \chpattyp{p}{\tau_1}{\xi}{\Gamma}{\Delta}
  }{
    \chpattyp{\hinlp{p}}{\tsum{\tau_1}{\tau_2}}{\cinl{\xi}}{\Gamma}{\Delta}
  }

  \Infer{\PTInr}{
    \chpattyp{p}{\tau_2}{\xi}{\Gamma}{\Delta}
  }{
    \chpattyp{\hinrp{p}}{\tsum{\tau_1}{\tau_2}}{\cinr{\xi}}{\Gamma}{\Delta}
  }

  \Infer{\PTPair}{
    \chpattyp{p_1}{\tau_1}{\xi_1}{\Gamma_1}{\Delta} \\
    \chpattyp{p_2}{\tau_2}{\xi_2}{\Gamma_2}{\Delta}
  }{
    \chpattyp{\hpair{p_1}{p_2}}{\tprod{\tau_1}{\tau_2}}
    {\cpair{\xi_1}{\xi_2}}{\Gamma_1 \uplus \Gamma_2}{\Delta}
  }

  \end{mathpar}

  \judgbox{
    \chrultyp{\Gamma}{\Delta}{r}{\tau}{\xi}{\tau'}
  }{$r$ transforms a final expression of type $\tau$ \\ to a final expression of type $\tau'$}

  \begin{mathpar}
    \Infer{\TRule}{
      \chpattyp{p}{\tau}{\xi}{\Gamma_p}{\Delta_p} \\
      \hexptyp{\Gamma \uplus \Gamma_p}{\Delta \uplus \Delta_p}{e}{\tau'}
    }{
      \chrultyp{\Gamma}{\Delta}{\hrulP{p}{e}}{\tau}{\xi}{\tau'}
    }
  \end{mathpar}

  \judgbox{\chrulstyp{\Gamma}{\Delta}{\xi_{pre}}{rs}{\tau}{\xi_{rs}}{\tau'}}
  {$rs$ transforms a final expression of type $\tau$ \\ to a final expression of type $\tau'$}

  \begin{mathpar}
  \Infer{TOneRules}{
    \chrultyp{\Gamma}{\Delta}{r}{\tau}{\xi_r}{\tau'} \\
    \cnotsatisfy{\xi_r}{\xi_{pre}}
  }{
    \chrulstyp{\Gamma}{\Delta}{\xi_{pre}}{\hrulesP{r}{\cdot}}{\tau}{\xi_r}{\tau'}
  }

  \Infer{TRules}{
    \chrultyp{\Gamma}{\Delta}{r}{\tau}{\xi_r}{\tau'} \\
    \chrulstyp{\Gamma}{\Delta}{\cor{\xi_{pre}}{\xi_r}}{rs}
    {\tau}{\xi_{rs}}{\tau'} \\
    \cnotsatisfy{\xi_r}{\xi_{pre}}
  }{
    \chrulstyp{\Gamma}{\Delta}{\xi_{pre}}{\hrules{r}{rs}}
    {\tau}{\cor{\xi_r}{\xi_{rs}}}{\tau'}
  }
\end{mathpar}
\caption{Typing of Patterns, Single Rules, and Series of Rules}
\label{fig:pat-rulestyp}
\end{figure}


\figurename~\ref{fig:pat-rulestyp} defines the typing judgment for patterns $p$,
single rules $r$, and series of rules $rs$. We will see how constraint $\xi$ is
generated, accumulated, and used to check redundancy of a rule $r$ with respect
to its preceding ones.
\todo{d: Similarly this summarizing sentence needs to be more simply
stated and perhaps more descriptive as to what will follow, not just leaning
on the phrase ``we will see'' to defer responsibility to later details)}

The typing judgment of series of rules $rs$ is of the form
$\chrulstyp{\Gamma}{\Delta}{\xi_{pre}}{rs}{\tau_1}{\xi_{rs}}{\tau_2}$. As shown
in Rules \TMatchZPre and \TMatchNZPre, the constituent rules inherit the
variable context $\Gamma$ and hole context $\Delta$ from the outer match
expression. When we check the type of a series of rules, we consider each rule
in order, just as how we do pattern matching in \autoref{sec:dynamics}.

Rule \TRules corresponds to the inductive case. The first premise is to check
the type of the initial rule $r$. It specifies that each rule takes a final expression
of type $\tau_1$ and returns a final expression of type $\tau_2$. It also emits
a constraint $\xi_r$, which is actually emitted from the pattern of rule $r$ as
we will see later. In order to determine if the initial rule $r$ of the rules
$\hrules{r}{rs}$ is redundant with respect to its preceding rules, we use
$\xi_{pre}$ to keep track of the pattern matching information of preceding
rules. To accomplish that, as we drop the initial rule $r$, we append the
constraint $\xi_r$ emitted from the pattern of $r$, to the constraint
$\xi_{pre}$, and use $\cor{\xi_{pre}}{\xi_r}$ as the new input to inductively check the type
of the trailing rules $rs$ in the second premise. Now that we have shown how to maintain the
constraint $\xi_{pre}$ associated with the preceding rules, we can compare it
with the constraint of the current rule, $\xi_r$. As we check the
type of rules, we consider each rule in order and use
$\cnotsatisfy{\xi_r}{\xi_{pre}}$ to ensure that the current rule $r$ doesn't
have to be redundant with respect to its preceding rules. We will see in
\autoref{sec:constraint} that $\csatisfy{\xi_r}{\xi_{pre}}$ corresponds to
``must redundant''. At the same time, the judgment also outputs the accumulated
constraint collected from rules $\hrules{r}{rs}$, which is used to check
exhaustiveness of rules, as we have shown in \autoref{sec:exptyp}

Rule \TOneRules corresponds to the base case that the series of rules contains
only one rule. The premises are similar to that of Rule \TRules except that
there is no trailing rules to check the type of. The reason why we regard one
rule as the base case instead of empty rules, is that since our match expression
takes zipper rules, we will never need to check the type of empty rules. The
only case that it makes sense to allow a match expression to contain empty
rules, is when we match on a final expression of \textit{Void} type and thus
don't need to worry about exhaustiveness checking. It turns out that we do not
have to sacrifice the generality (see Appendix \todo{match(){.}}).

As we have briefly mentioned above, Rule \TRule specifies that rule
$\hrul{p}{e}$ transforms final expressions of type $\tau_1$ to final expressions
of type $\tau_2$. The first premise is the typing judgment of patterns---by
assigning pattern $p$ with type $\tau_1$, we collect the typing for all the
variables and holes involved in the pattern $p$ and generate variable context
$\Gamma_p$ and hole context $\Delta_p$. Additionally, it emits constraint $\xi$,
which is closely associated with the pattern itself. While constraint is used to
identify the set of final expressions that match $p$, we introduce
\textit{Unknown} constraint $\cunknown$ to denote the set of final expression
matching $p$ is yet to be determined (Rules \PTEHole and \PTHole). We will elaborate on what constraint is in
\autoref{sec:constraint}. The second premise strictly extends two contexts of
rule $r$ with that generated from pattern $p$, and check the type of
sub-expression $e$.

\subsubsection{Type Safety}
The type safety of the language is established by
\autoref{theorem:determinism} and \autoref{theorem:preservation}.

\begin{theorem}[Preservation]
  \label{theorem:preservation}
  If $\hexptyp{\cdot}{\Delta}{e}{\tau}$ and $\htrans{e}{e'}$
  then $\hexptyp{\cdot}{\Delta}{e'}{\tau}$
\end{theorem}

% !TEX root= pattern-paper.tex

\begin{figure}[ht]

\judgbox{
  \hsubstyp{\theta}{\Gamma}
}{
  $\theta$ is of type $\Gamma$
}

\begin{mathpar}
\Infer{\STEmpty}{ }{
  \hsubstyp{\emptyset}{\cdot}
}

\Infer{\STExtend}{
  \hsubstyp{\theta}{\Gamma_\theta} \\
  \hexptyp{\Gamma}{\Delta}{e}{\tau}
}{
  \hsubstyp{\theta , x / e}{\Gamma_\theta , x : \tau}
}
\end{mathpar}

\caption{Typing of Substitution}
\label{fig:substyp}
\end{figure}


\figurename~\ref{fig:substyp} defines the typing of substitution $\theta$.

To prove \autoref{theorem:preservation}, we need the following three lemmas.
When considering Rule \ITAp, \autoref{lemma:substitution} is needed.
When considering Rule \ITSuccMatch, \autoref{lemma:subs-typing} is needed
to show the typing of substitution $\theta$, and we use
\autoref{lemma:simult-substitution} to show that type is preserved when pattern
matching succeeds.

\begin{lemma}[Substitution]
  \label{lemma:substitution}
  If $\hexptyp{\Gamma, x : \tau}{\Delta}{e_0}{\tau_0}$ and $\hexptyp{\Gamma}{\Delta}{e}{\tau}$
  then $\hexptyp{\Gamma}{\Delta}{[e/x]e_0}{\tau_0}$
\end{lemma}

\begin{lemma}[Substitution Typing]
  \label{lemma:subs-typing}
  If $\hpatmatch{e}{p}{\theta}$ and $\hexptyp{\cdot}{\Delta_e}{e}{\tau}$ and $\chpattyp{p}{\tau}{\xi}{\Gamma}{\Delta}$
  then $\hsubstyp{\theta}{\Gamma}$
\end{lemma}

\begin{lemma}[Simultaneous Substitution]
  \label{lemma:simult-substitution}
  If $\hexptyp{\Gamma \uplus \Gamma'}{\Delta}{e}{\tau}$ and $\hsubstyp{\theta}{\Gamma'}$
  then $\hexptyp{\Gamma}{\Delta}{[\theta]e}{\tau}$
\end{lemma}

\subsubsection{Match Constraint Language}\label{sec:constraint}
% !TEX root= pattern-paper.tex

\begin{figure}[ht]
$\arraycolsep=4pt\begin{array}{lll}
\xi & ::= &
  \ctruth ~\vert~
  \cfalsity ~\vert~
  \cunknown ~\vert~
  \cnum{n} ~\vert~
  \cnotnum{n} ~\vert~
  \cand{\xi_1}{\xi_2} ~\vert~
  \cor{\xi_1}{\xi_2} ~\vert~
  \cinl{\xi} ~\vert~
  \cinr{\xi} ~\vert~
  \cpair{\xi_1}{\xi_2}
\end{array}$

\judgbox{\ctyp{\xi}{\tau}}{$\xi$ constrains values of type $\tau$}

\begin{mathpar}
\Infer{\CTTruth}{ }{
  \ctyp{\ctruth}{\tau}
}

\Infer{\CTFalsity}{ }{
  \ctyp{\cfalsity}{\tau}
}

\Infer{\CTUnknown}{ }{
  \ctyp{\cunknown}{\tau}
}

\Infer{\CTNum}{ }{
  \ctyp{\cnum{n}}{\tnum}
}

\Infer{\CTNotNum}{ }{
  \ctyp{\cnotnum{n}}{\tnum}
}

\Infer{\CTAnd}{
  \ctyp{\xi_1}{\tau} \\ \ctyp{\xi_2}{\tau}
}{
  \ctyp{\cand{\xi_1}{\xi_2}}{\tau}
}

\Infer{\CTOr}{
  \ctyp{\xi_1}{\tau} \\ \ctyp{\xi_2}{\tau}
}{
  \ctyp{\cor{\xi_1}{\xi_2}}{\tau}
}

\Infer{\CTInl}{
  \ctyp{\xi_1}{\tau_1}
}{
  \ctyp{\cinl{\xi_1}}{\tsum{\tau_1}{\tau_2}}
}

\Infer{\CTInr}{
  \ctyp{\xi_2}{\tau_2}
}{
  \ctyp{\cinr{\xi_2}}{\tsum{\tau_1}{\tau_2}}
}

\Infer{\CTPair}{
  \ctyp{\xi_1}{\tau} \\ \ctyp{\xi_2}{\tau}
}{
  \ctyp{\cpair{\xi_1}{\xi_2}}{\tau}
}
\end{mathpar}

\judgbox{\cdual{\xi_1} = \xi_2}{dual of $\xi_1$ is $\xi_2$}
\begin{subequations}\label{defn:dual}
\begin{align}
  \cdual{\ctruth} &= \cfalsity \\
  \cdual{\cfalsity} &= \ctruth \\
  \cdual{\cunknown} &= \cunknown \\
  \cdual{\cnum{n}} &= \cnotnum{n} \\
  \cdual{\cnotnum{n}} &= \cnum{n} \\
  \cdual{\cand{\xi_1}{\xi_2}} &= \cor{\cdual{\xi_1}}{\cdual{\xi_2}} \\
  \cdual{\cor{\xi_1}{\xi_2}} &= \cand{\cdual{\xi_1}}{\cdual{\xi_2}} \\
  \cdual{\cinl{\xi_1}} &= \cor{ \cinl{\cdual{\xi_1}} }{ \cinr{\ctruth} } \\
  \cdual{\cinr{\xi_2}} &= \cor{ \cinr{\cdual{\xi_2}} }{ \cinl{\ctruth} } \\
  \cdual{\cpair{\xi_1}{\xi_2}} &=
  \cor{ \cor{ 
    \cpair{\xi_1}{\cdual{\xi_2}}
  }{
    \cpair{\cdual{\xi_1}}{\xi_2}
  }}{
    \cpair{\cdual{\xi_1}}{\cdual{\xi_2}}
  }
\end{align}
\end{subequations}
  \caption{Match Constraints}
  \label{fig:constraint}
\end{figure}


\autoref{fig:constraint} introduces match constraint language, which is
used to identify a subset of the final expressions of a type. The judgment
$\ctyp{\xi}{\tau}$ specifies that constraint $\xi$ constrains the final
expressions of type $\tau$. The dual of $\xi$, $\cdual{\xi}$ represents the
complement of the subset identified by $\xi$ under the set of the final
expressions of a type.
% !TEX root= pattern-paper.tex

\begin{figure}[!ht]
\judgbox{
  \refutable{\xi}
}{$\xi$ is refutable}

\begin{mathpar}
\Infer{RXFalsity}{ }{
  \refutable{\cfalsity}
}

\Infer{\RXUnknown}{ }{
  \refutable{\cunknown}
}

\Infer{\RXNum}{ }{
  \refutable{\cnum{n}}
}

\Infer{\RXNotNum}{ }{
  \refutable{\cnotnum{n}}
}

\Infer{\RXInl}{ }{
  \refutable{\cinl{\xi}}
}

\Infer{\RXInr}{ }{
  \refutable{\cinr{\xi}}
}

\Infer{\RXPairL}{
  \refutable{\xi_1}
}{
  \refutable{\cpair{\xi_1}{\xi_2}}
}

\Infer{\RXPairR}{
  \refutable{\xi_2}
}{
  \refutable{\cpair{\xi_1}{\xi_2}}
}
\end{mathpar}
\caption{Refutable Constraints}
\label{fig:xi-refutable}
\end{figure}

\todo{xi refutable}
% !TEX root= pattern-paper.tex

\begin{figure}[!ht]

\judgbox{\csatisfy{e}{\xi}}{$e$ satisfies $\xi$}

\begin{mathpar}
\Infer{\CSTruth}{ }{
  \csatisfy{e}{\ctruth}
}

\Infer{\CSNum}{ }{
  \csatisfy{\hnum{n}}{\cnum{n}}
}

\Infer{\CSNotNum}{
  n_1 \neq n_2
}{
  \csatisfy{\hnum{n_1}}{\cnotnum{n_2}}
}

\Infer{\CSAnd}{
  \csatisfy{e}{\xi_1} \\
  \csatisfy{e}{\xi_2}
}{
  \csatisfy{e}{\cand{\xi_1}{\xi_2}}
}

\Infer{\CSOrL}{
  \csatisfy{e}{\xi_1}
}{
  \csatisfy{e}{\cor{\xi_1}{\xi_2}}
}

\Infer{\CSOrR}{
  \csatisfy{e}{\xi_2}
}{
  \csatisfy{e}{\cor{\xi_1}{\xi_2}}
}

\Infer{\CSInl}{
  \csatisfy{e_1}{\xi_1}
}{
  \csatisfy{
    \hinl{\tau_2}{e_1}
  }{
    \cinl{\xi_1}
  }
}

\Infer{\CSInr}{
  \csatisfy{e_2}{\xi_2}
}{
  \csatisfy{
    \hinr{\tau_1}{e_2}
  }{
    \cinr{\xi_2}
  }
}

\Infer{\CSPair}{
  \csatisfy{e_1}{\xi_1} \\
  \csatisfy{e_2}{\xi_2}
}{
\csatisfy{\hpair{e_1}{e_2}}{\cpair{\xi_1}{\xi_2}}
}

\Infer{\CSNotIntroPair}{
  \notIntro{e} \\
  \csatisfy{\hprl{e}}{\xi_1} \\
  \csatisfy{\hprr{e}}{\xi_2}
}{
  \csatisfy{e}{\cpair{\xi_1}{\xi_2}}
}
\end{mathpar}

\judgbox{\cmaysatisfy{e}{\xi}}{$e$ may satisfy $\xi$}

\begin{mathpar}
\Infer{\CMSUnknown}{ }{
  \cmaysatisfy{e}{\cunknown}
}

\Infer{\CMSNotIntro}{
  \notIntro{e} \\
  \refutable{\xi}
}{
  \cmaysatisfy{e}{\xi}
}

\Infer{\CMSAndL}{
  \cmaysatisfy{e}{\xi_1} \\
  \csatisfy{e}{\xi_2}
}{
  \cmaysatisfy{e}{\cand{\xi_1}{\xi_2}}
}

\Infer{\CMSAndR}{
  \csatisfy{e}{\xi_1} \\
  \cmaysatisfy{e}{\xi_2}
}{
  \cmaysatisfy{e}{\cand{\xi_1}{\xi_2}}
}

\Infer{\CMSAnd}{
  \cmaysatisfy{e}{\xi_1} \\
  \cmaysatisfy{e}{\xi_2}
}{
  \cmaysatisfy{e}{\cand{\xi_1}{\xi_2}}
}

\Infer{\CMSOrL}{
  \cmaysatisfy{e}{\xi_1} \\
  \cnotsatisfy{e}{\xi_2}
}{
  \cmaysatisfy{e}{\cor{\xi_1}{\xi_2}}
}

\Infer{\CMSOrR}{
  \cnotsatisfy{e}{\xi_1} \\
  \cmaysatisfy{e}{\xi_2}
}{
  \cmaysatisfy{e}{\cor{\xi_1}{\xi_2}}
}

\Infer{\CMSInl}{
  \cmaysatisfy{e_1}{\xi_1}
}{
  \cmaysatisfy{
    \hinl{\tau_2}{e_1}
  }{
    \cinl{\xi_1}
  }
}

\Infer{\CMSInr}{
  \cmaysatisfy{e_2}{\xi_2}
}{
  \cmaysatisfy{
    \hinr{\tau_1}{e_2}
  }{
    \cinr{\xi_2}
  }
}

\Infer{\CMSPairL}{
  \cmaysatisfy{e_1}{\xi_1} \\
  \csatisfy{e_2}{\xi_2}
}{
  \cmaysatisfy{\hpair{e_1}{e_2}}{\cpair{\xi_1}{\xi_2}}
}

\Infer{\CMSPairR}{
  \csatisfy{e_1}{\xi_1} \\
  \cmaysatisfy{e_2}{\xi_2}
}{
  \cmaysatisfy{\hpair{e_1}{e_2}}{\cpair{\xi_1}{\xi_2}}
}

\Infer{\CMSPair}{
  \cmaysatisfy{e_1}{\xi_1} \\
  \cmaysatisfy{e_2}{\xi_2}
}{
  \cmaysatisfy{\hpair{e_1}{e_2}}{\cpair{\xi_1}{\xi_2}}
}
\end{mathpar}

\judgbox{\csatisfyormay{e}{\xi}}{$e$ satisfies or may satisfy $\xi$}

\begin{mathpar}
\Infer{\CSMSMay}{
  \cmaysatisfy{e}{\xi}
}{
  \csatisfyormay{e}{\xi}
}

\Infer{\CSMSSat}{
  \csatisfy{e}{\xi}
}{
  \csatisfyormay{e}{\xi}
}
\end{mathpar}

  \caption{Satisfaction Judgments}
  \label{fig:satisfy}
\end{figure}


\autoref{fig:satisfy} defines satisfaction judgments. As we only
consider final expressions and patterns of the same type when talking about
pattern matching, a constraint only constrains final expressions of the same
type. And the satisfaction judgments does not make sense when the expression is
not final or the expression and the constraint are of different type. The
judgment $\csatisfy{e}{\xi}$ specifies that expression $e$ satisfies $\xi$ while
the judgment $\cmaysatisfy{e}{\xi}$ specifies that expression $e$ may or may not
satisfy $\xi$. The judgment $\csatisfyormay{e}{\xi}$ is the combination of the
previous two cases. It turns out that the remaining case where
$\csatisfyormay{e}{\xi}$ is not derivable, can be represented by
$\csatisfy{e}{\cdual{\xi}}$.

\begin{theorem}[Exclusiveness of Constraint Satisfaction]
  \label{theorem:exclusive-constraint-satisfaction}
  If $\ctyp{\xi}{\tau}$ and $\hexptyp{\cdot}{\Delta}{e}{\tau}$ and $\isFinal{e}$ then exactly one of the following holds
  \begin{enumerate}
  \item $\csatisfy{e}{\xi}$
  \item $\cmaysatisfy{e}{\xi}$
  \item $\cnotsatisfyormay{e}{\xi}$
  \end{enumerate}
\end{theorem}

\autoref{lemma:const-matching-coherence} establishes a correspondence
between pattern matching results and satisfaction judgments. That makes
reasoning pattern matching in type system possible and helps prove
\autoref{theorem:preservation}.

\begin{lemma}[Matching Coherence of Constraint]
  \label{lemma:const-matching-coherence}
  Suppose that $\hexptyp{\cdot}{\Delta_e}{e}{\tau}$ and $\isFinal{e}$ and $\chpattyp{p}{\tau}{\xi}{\Gamma}{\Delta}$. Then we have
  \begin{enumerate}
  \item $\csatisfy{e}{\xi}$ iff $\hpatmatch{e}{p}{\theta}$
  \item $\cmaysatisfy{e}{\xi}$ iff $\hmaymatch{e}{p}$
  \item $\cnotsatisfyormay{e}{\xi}$ iff $\hnotmatch{e}{p}$
  \end{enumerate}
\end{lemma}

The following two definitions take advantage satisfaction judgments and
corresponds to redundancy and exhaustiveness respectively. We will see how they
can be determined in \autoref{sec:algorithm}

\begin{definition}[Entailment of Constraints]
  \label{definition:const-entailment}
  Suppose that $\ctyp{\xi_1}{\tau}$ and $\ctyp{\xi_2}{\tau}$.
  Then $\csatisfy{\xi_1}{\xi_2}$ iff for all $e$ such that $\hexptyp{\cdot}{\Delta}{e}{\tau}$ and $\isVal{e}$ we have $\csatisfyormay{e}{\xi_1}$ implies $\csatisfy{e}{\xi_2}$
\end{definition}

Recall in Rules \TOneRules and \TRules, we use $\cnotsatisfy{\xi_r}{\xi_{pre}}$ to ensure rule $r$ does not have to be redundant with respect to its preceding rules $rs_{pre}$. When considering the redundancy of a specific rule in a match expression, the programmer only want to be warned when the rule is destined be redundant, regardless of how the programmer fills the holes at the end. Therefore, only when all values that must or may match the pattern of rule $r$, must have already matched one of the patterns in its preceding rules $rs_{pre}$, can we say $r$ must be redundant with respect to $rs_{pre}$.
\todo{d: is there a connective word missing in this last sentence? not sure what this is saying}

\begin{definition}[Potential Entailment of Constraints]
  \label{definition:nn-entailment}
  Suppose that $\ctyp{\xi_1}{\tau}$ and $\ctyp{\xi_2}{\tau}$. Then $\csatisfyormay{\xi_1}{\xi_2}$ iff for all $e$ such that $\hexptyp{\cdot}{\Delta}{e}{\tau}$ and $\isFinal{e}$ we have $\csatisfyormay{e}{\xi_1}$ implies $\csatisfyormay{e}{\xi_2}$ 
\end{definition}

Recall in Rules \TMatchZPre and \TMatchNZPre, we use $\csatisfyormay{\ctruth}{\xi}$ to ensure that the rules associated with constraint $\xi$ either must or may be exhaustive. When considering the exhaustiveness of a sequence of rules, the programmer only want to be warned when the rules cannot be exhaustive, regardless of how the programmer fills the holes at the end. Then, we just need to ensure that for all values $e$, $e$ either must or may match one of the patterns in the sequence of the rules. For simplicity when proving the progress part of  \autoref{theorem:determinism}, we consider all final expressions instead. In this way, even when the match expression is not complete and we may match on an indeterminate expression, we can still be confident that we won't have a match failure error. And actually, as we will later show in \autoref{sec:algorithm}, it is legitimate to replace values with final expressions in the quantifier.

\subsection{Decidability}\label{sec:algorithm}

\autoref{sec:statics} has already described exhaustiveness checking and redundancy checking using entail judgments -- \autoref{definition:const-entailment} and
\autoref{definition:nn-entailment}. The following section will discuss the decision process of entail judgments and thus show the decidability of type checking.

As discussed in previous sections, if we are going to reason about the exhaustiveness of an incomplete match expression, only when the match expression must be inexhaustive no matter how the programmer fills it, shall the type checking fail. As a human being, such a process can be naturally performed by thinking of all the pattern holes as wildcards and checking if the transformed patterns cover all the possible cases. 

Similarly, if we are going to reason about the redundancy of one branch with respect to its previous ones in the presence of pattern holes, the programmer shall only get an error/warning if that branch will never be matched no matter how he/she fills the pattern holes. It is natural to imagine all the previous branches with pattern holes would be refuted during runtime as the programmer may fill them with any refutable patterns. On the other hand, it makes sense to have the branch of our interest to cover more cases by thinking of any of its pattern holes as wild cards. Then, we may see if there is a case that is covered by the branch but not its previous branches.

As discussed in previous sections, we use a constraint language to capture the semantics of patterns and thus reason about the exhaustiveness and redundancy of patterns on their corresponding constraints. Naturally, the aforementioned intuition is implemented through transformation on constraints. Specifically, as a pattern hole emits an unknown constraint \cunknown, replacing the pattern hole with a wildcard corresponds to "truify" $\cunknown$, \ie, turning $\cunknown$ into $\ctruth$. Similarly, we turn $\cunknown$ into $\cfalsity$ -- dubbed "falsify" -- to encode the idea of replacing the pattern hole with some pattern that a specific set of expressions won't match. Both "truify" operation $\ctruify{\xi}$ and "falsify" operation $\cfalsify{\xi}$ are defined as functions in \autoref{fig:truify-falsify}. 

% !TEX root= pattern-paper.tex

\begin{figure}[ht]
\judgbox{\ctruify{\xi_1} = \xi_2}{}

\begin{align*}
  \ctruify{\ctruth} &= \ctruth \\
  \ctruify{\cfalsity} &= \cfalsity \\
  \ctruify{\cunknown} &= \ctruth \\
  \ctruify{\cnum{n}} &= \cnum{n} \\
  \ctruify{\cnotnum{n}} &= \cnotnum{n} \\
  \ctruify{\cand{\xi_1}{\xi_2}} &= \cand{\ctruify{\xi_1}}{\ctruify{\xi_2}} \\
  \ctruify{\cor{\xi_1}{\xi_2}} &= \cor{\ctruify{\xi_1}}{\ctruify{\xi_2}} \\
  \ctruify{\cinl{\xi}} &= \cinl{\ctruify{\xi}} \\
  \ctruify{\cinr{\xi}} &= \cinr{\ctruify{\xi}} \\
  \ctruify{\cpair{\xi_1}{\xi_2}} &= \cpair{\ctruify{\xi_1}}{\ctruify{\xi_2}}
\end{align*}

\judgbox{\cfalsify{\xi_1} = \xi_2}{}
\begin{align*}
  \cfalsify{\ctruth} &= \ctruth \\
  \cfalsify{\cfalsity} &= \cfalsity \\
  \cfalsify{\cunknown} &= \cfalsity \\
  \cfalsify{\cnum{n}} &= \cnum{n} \\
  \cfalsify{\cnotnum{n}} &= \cnotnum{n} \\
  \cfalsify{\cand{\xi_1}{\xi_2}} &= \cand{\cfalsify{\xi_1}}{\cfalsify{\xi_2}} \\
  \cfalsify{\cor{\xi_1}{\xi_2}} &= \cor{\cfalsify{\xi_1}}{\cfalsify{\xi_2}} \\
  \cfalsify{\cinl{\xi}} &= \cinl{\cfalsify{\xi}} \\
  \cfalsify{\cinr{\xi}} &= \cinr{\cfalsify{\xi}} \\
  \cfalsify{\cpair{\xi_1}{\xi_2}} &= \cpair{\cfalsify{\xi_1}}{\cfalsify{\xi_2}}
\end{align*}
  \caption{Truify and Falsify Constraints}
  \label{fig:truify-falsify}
\end{figure}

Leveraging these two operation, \autoref{theorem:exhaustive-truify} and \autoref{theorem:redundant-truify-falsify} establish the "fully-known" equivalent of the entail judgments for exhaustiveness checking $\csatisfyormay{\ctruth}{\xi}$ and redundancy checking $\csatisfy{\xi_r}{\xi_{rs}}$.

\begin{theorem}
\label{theorem:exhaustive-truify}
  $\csatisfyormay{\ctruth}{\xi}$ iff $\csatisfy{\ctruth}{\ctruify{\xi}}$.
\end{theorem}

\begin{theorem}
\label{theorem:redundant-truify-falsify}
  $\csatisfy{\xi_r}{\xi_{rs}}$ iff $\csatisfy{\ctruth}{\cor{\cdual{\ctruify{\xi_r}}}{\cfalsify{\xi_{rs}}}}$.
\end{theorem}

Now, let's revisit the examples in \autoref{sec:examples}.

Consider the program in \autoref{fig:may-exhaustive}, it is unclear whether the match expression is truly exhaustive. But by replacing the pattern hole with a wild card pattern, the match expression may cover every case of the input list. And thus we know the match expression may be exhaustive and the program should type check. To Since the Unknown constraint $\cunknown$ is directly emitted from the pattern hole $w$, analogously, we replace it with Truth constraint $\ctruth$.

Consider the example in \autoref{fig:must-redundant}, we want to tell if the third rule is redundant with respect to the first and second rules. To maximize the subset of values of list type that matches the pattern $\heholep{w}::xs$, we again replace the pattern hole $w$ with a variable pattern or a wild card pattern and realize that it is redundant. When there are hole(s) in the patterns of preceding rules, it is not obvious what pattern to replace the hole with. Consider the example in \figurename~\ref{fig:may-redundant1}, we want to tell if the third rule is redundant. To accomplish that, we want to minimize the subset of values of list type that matches $x::\heholep{w}$ so that it does not cover all the cases of the last branch. Intuitively, we may fill hole $\heholep{w}$ with $2::\nil$ so that the trailing part of the second pattern is more specific than $xs$ in the third pattern. However, it is not always the case that we can find a more specific (sub)pattern. Consider the example in \figurename~\ref{fig:may-redundant2}, the third pattern only matches one values and thus we cannot find a more specific pattern. Particularly, $2::\nil$ in the third pattern corresponds to the hole pattern $\heholep{w}$ in the second pattern. In this case, we can always find a different pattern to substitute the hole pattern. For example, we can replace $\heholep{w}$ with $3::[]$ and find that the third rule is not redundant.

The reason why we can always find a different pattern is because we have number as the base type of our language. What if we add unit type? (We actually need unit type to represent empty list!) Then we no longer have infinite patterns that is of arbitrary given type. We can simply emit Truth constraint $\ctruth$ when assigning unit type to pattern holes (see Appendix \todo{}).

\figurename~\ref{fig:incon} defines \textit{inconsistent} judgment to further determine $\csatisfy{\ctruth}{\xi}$. Note that at this stage, constraint $\xi$ does not involve any Unknown constraint $\cunknown$. The inconsistent judgment $\cincon{\xi}$ specifies that constraint $\xi$ is inconsistent in the sense that no value of the same type as $\xi$ satisfies $\xi$.

% !TEX root= pattern-paper.tex

\begin{figure}[bp]
\judgbox{\cincon{\Xi}}{}

\begin{mathpar}
\Infer{\CINCTruth}{
  \cincon{\Xi}
}{
  \cincon{\Xi, \ctruth}
}

\Infer{\CINCFalsity}{ }{
  \cincon{\Xi, \cfalsity}
}

\Infer{\CINCNum}{
  n_1 \neq n_2
}{
  \cincon{\Xi, \cnum{n_1}, \cnum{n_2}}
}

\Infer{\CINCNotNum}{ }{
  \cincon{\Xi, \cnum{n}, \cnotnum{n}}
}

\Infer{\CINCAnd}{
  \cincon{\Xi, \xi_1, \xi_2}
}{
  \cincon{\Xi, \cand{\xi_1}{\xi_2}}
}

\Infer{\CINCOr}{
  \cincon{\Xi, \xi_1} \\
  \cincon{\Xi, \xi_2}
}{
  \cincon{\Xi, \cor{\xi_1}{\xi_2}}
}

\Infer{\CINCInj}{ }{
  \cincon{\Xi, \cinl{\xi_1}, \cinr{\xi_2}}
}

\Infer{\CINCInl}{
  \cincon{\setof{\xi' | \cinl{\xi'} \in \Xi},\xi}
}{
  \cincon{\Xi, \cinl{\xi}}
}

\Infer{\CINCInr}{
  \cincon{\setof{\xi' | \cinr{\xi'} \in \Xi},\xi}
}{
  \cincon{\Xi, \cinr{\xi}}
}

\Infer{\CINCPairL}{
    \cincon{\setof{\xi_1' | \cpair{\xi_1'}{\xi_2'} \in \Xi},\xi_1}
}{
    \cincon{\Xi, \cpair{\xi_1}{\xi_2}}
}

\Infer{\CINCPairR}{
    \cincon{\setof{\xi_2' | \cpair{\xi_1'}{\xi_2'} \in \Xi},\xi_2}
}{
    \cincon{\Xi, \cpair{\xi_1}{\xi_2}}
}
\end{mathpar}

  \caption{Inconsistency of Constraints}
  \label{fig:incon}
\end{figure}

\begin{theorem}
\textbf{}  Assume $\ctruify{\xi}=\xi$. It is decidable whether $\cincon{\xi}$.
\end{theorem}

\begin{theorem}
  Assume $\ctruify{\xi}=\xi$. Then $\cincon{\cdual{\xi}}$ iff $\csatisfy{\ctruth}{\xi}$.
\end{theorem}
