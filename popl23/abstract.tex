\begin{abstract}
    Several modern programming systems, including GHC Haskell, Agda, Idris, and Hazel,  support \emph{typed holes}. 
    Assigning static and, to varying degree, dynamic meaning to programs with holes allows program editors and other tools to offer meaningful  feedback and assistance throughout editing, i.e. in a \emph{live} manner.
    Prior work, however, has considered only holes appearing in expressions and types. 
    This paper considers, from type theoretic and logical first principles,
    the problem of typed pattern holes.
    We confront two main difficulties, (1) statically reasoning about exhaustiveness and irredundancy 
    when patterns are not fully known, and (2) live evaluation of expressions
    containing both pattern and expression holes. 
    In both cases, this requires reasoning conservatively about all 
    possible hole fillings.
    We develop a typed lambda calculus, Peanut, 
    where reasoning about exhaustiveness and redundancy is mapped to the
    problem of deriving first order entailments. 
    We equip Peanut with an operational semantics in the style of Hazelnut Live that allows us 
    to evaluate around holes in both expressions and patterns.
    We mechanize the metatheory of Peanut in Agda and formalize a procedure capable of 
    deciding the necessary entailments.
    We scale up the core calculus to support finite labeled sums, adding support for \emph{datatype constructor holes} in the process. Finally, we implement these mechanisms within Hazel, a  
    programming environment for a dialect of Elm that automatically inserts holes during editing to provide static and dynamic feedback to the programmer in a maximally live manner, i.e. for every possible editor state. Hazel is the first maximally live environment for a general-purpose functional language.
    % Hazel is the first maximally live environment for a general-purpose functional language. 
        % In these languages, empty holes serve as placeholders for missing program terms, while
    % non-empty holes operate as semantic membranes around erroneous terms, 
    % isolating them so that the remainder of the program remains statically meaningful,
    % and, in certain cases, even dynamically meaningful.

  \keywords{pattern matching \and typed holes \and live programming}
\end{abstract}