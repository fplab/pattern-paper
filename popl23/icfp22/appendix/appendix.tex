\RequirePackage{etex}
\documentclass{article}

\usepackage[T1]{fontenc} % fix missing font cmtt
\usepackage{amsmath}
\usepackage{amssymb} % Vdash
\usepackage{amsthm} % proof
\usepackage{graphicx} % rotatebox
\usepackage{stmaryrd} % llparenthesis
\usepackage{anyfontsize} % workaround for font size difference warning
\usepackage{todonotes}
\usepackage{listings}
\usepackage{tikz}
\usetikzlibrary{calc,fit,tikzmark,plotmarks,arrows.meta,positioning,overlay-beamer-styles}
\usepackage[caption=false]{subfig}


\usepackage{cancel} % slash over symbol
\usepackage{hyperref}
\renewcommand\UrlFont{\color{blue}\rmfamily}
\let\figureautorefname\figurename
\def\sectionautorefname{Sec.}
\def\lemmaautorefname{Lemma}
\def\equationautorefname{Definition}
\def\definitionautorefname{Definition}
\def\theoremautorefname{Theorem}
\def\corollaryautorefname{Corollary}
\let\subsectionautorefname\sectionautorefname
\let\subsubsectionautorefname\sectionautorefname
\newcommand{\rulesref}[1]{Rules (\ref{#1})}
\newcommand{\ruleref}[1]{Rule (\ref{#1})}

\usepackage{xcolor}
\definecolor{hazelgreen}{RGB}{7,63,36}
\definecolor{hazellightgreen}{RGB}{103,138,97}
\definecolor{hazelyellow}{RGB}{245,222,179}
\definecolor{hazellightyellow}{RGB}{254,254,234}

\newcommand{\highlight}[1]{\colorbox{yellow}{$\displaystyle #1$}}

%% Joshua Dunfield macros
\def\OPTIONConf{1}
\usepackage{jdunfield}
\usepackage{pfsteps}
\makeatletter
\newcommand{\savelocalsteps}[1]{
  \@ifundefined{c@#1}
    {% the counter doesn't exist
     \newcounter{#1}
   }{}
  \setcounter{#1}{\value{pfsteps@pfc@local}}
}
\makeatother
\newcommand{\restorelocalsteps}[1]{\setcounter{pfsteps@pfc@local}{\value{#1}}}

\newtheorem{theorem}{Theorem}[section]
\newtheorem{corollary}{Corollary}[theorem]
\newtheorem{lemma}{Lemma}[theorem]
\newtheorem{definition}{Definition}[theorem]

% !TEX root = pattern-paper.tex
\usepackage{centernot}
% reverse Vdash
\newcommand{\dashV}{\mathbin{\rotatebox[origin=c]{180}{$\Vdash$}}}

% Violet hotdogs; highlight color helps distinguish them
\newcommand{\llparenthesiscolor}{\textcolor{violet}{\llparenthesis}}
\newcommand{\rrparenthesiscolor}{\textcolor{violet}{\rrparenthesis}}

% HTyp and HExp
\newcommand{\hcomplete}[1]{#1~\mathsf{complete}}

% HTyp
\newcommand{\htau}{\dot{\tau}}
\newcommand{\tarr}[2]{\inparens{#1 \rightarrow #2}}
\newcommand{\tarrnp}[2]{#1 \rightarrow #2}
\newcommand{\trul}[2]{\inparens{#1 \Rightarrow #2}}
\newcommand{\trulnp}[2]{#1 \Rightarrow #2}
\newcommand{\tnum}{\mathtt{num}}
\newcommand{\tehole}{\llparenthesiscolor\rrparenthesiscolor}
\newcommand{\tsum}[2]{\inparens{#1 + #2}}
\newcommand{\tprod}[2]{\inparens{#1 \times #2}}
\newcommand{\tunit}{\mathtt{1}}
\newcommand{\tvoid}{\mathtt{0}}

\newcommand{\tlabeledsum}[1]{+\mathopen{}\left\{#1\right\}}

\newcommand{\tcompat}[2]{#1 \sim #2}
\newcommand{\tincompat}[2]{#1 \nsim #2}


% HTag
\newcommand{\tagC}{\underline{\mathbf{C}}}
\newcommand{\tagset}{\mathcal{C}}
\newcommand{\tagehole}[1]{\llparenthesiscolor\rrparenthesiscolor^{#1}}
\newcommand{\taghole}[2]{\llparenthesiscolor#1\rrparenthesiscolor^{#2}}

\newcommand{\tagmaymatch}[2]{#1 \mathbin{?} #2}

% HExp
\newcommand{\hexp}{\dot{e}}
\newcommand{\hlam}[3]{\lambda #1:#2.#3}
\newcommand{\hap}[2]{#1(#2)}
\newcommand{\hapP}[2]{(#1)~(#2)} % Extra paren around function term
\newcommand{\hnum}[1]{\underline{#1}}
\newcommand{\hadd}[2]{\inparens{#1 + #2}}
\newcommand{\hpair}[2]{(#1 , #2)}
\newcommand{\htriv}{()}
\newcommand{\hehole}[1]{\llparenthesiscolor\rrparenthesiscolor^{#1}}
\newcommand{\hhole}[2]{\llparenthesiscolor#1\rrparenthesiscolor^{#2}}
\newcommand{\heholep}[1]{\llparenthesiscolor\rrparenthesiscolor^{#1}}
\newcommand{\hholep}[3]{\llparenthesiscolor#1\rrparenthesiscolor^{#2}_{#3}}
\newcommand{\hindet}[1]{\lceil#1\rceil}
\newcommand{\hinj}[3]{\mathtt{inj}_{#1}^{#2}({#3})}
\newcommand{\hinl}[2]{\mathtt{inl}_{#1}({#2})}
\newcommand{\hinr}[2]{\mathtt{inr}_{#1}({#2})}
\newcommand{\hinjp}[2]{\mathtt{inj}_{#1}(#2)}
\newcommand{\hinlp}[1]{\mathtt{inl}(#1)}
\newcommand{\hinrp}[1]{\mathtt{inr}(#1)}
\newcommand{\hfst}[1]{\mathtt{fst}(#1)}
\newcommand{\hsnd}[1]{\mathtt{snd}(#1)}
\newcommand{\hmatch}[2]{\mathtt{match}(#1)~\{#2\}}
\newcommand{\hcase}[5]{\mathtt{case}({#1},{#2}.{#3},{#4}.{#5})}
\newcommand{\hrules}[2]{#1 \mid #2}
\newcommand{\hrulesP}[2]{\inparens{#1 \mid #2}}
\newcommand{\hrul}[2]{#1 \Rightarrow #2}
\newcommand{\hrulP}[2]{\inparens{#1 \Rightarrow #2}}

\newcommand{\refutable}[1]{#1 ~\mathtt{refutable}_?}
\newcommand{\frefutable}[1]{\textit{refutable}_?\inparens{#1}}
\newcommand{\possible}[1]{#1 ~\mathtt{possible}}
\newcommand{\fpossible}[1]{\textit{possible}\inparens{#1}}
\newcommand{\inValues}[3]{#1 \in \mathtt{values}[#2]\inparens{#3}}

\newcommand{\hGamma}{\dot{\Gamma}}
\newcommand{\domof}[1]{\text{dom}(#1)}
\newcommand{\hsyn}[3]{#1 \vdash #2 \Rightarrow #3}
\newcommand{\hana}[3]{#1 \vdash #2 \Leftarrow #3}
\newcommand{\hexptyp}[4]{#1 \mathbin{;} #2 \vdash #3 : #4}
\newcommand{\hpattyp}[4]{#4 \vdash #1 : #2 \dashV #3}
\newcommand{\hsubstyp}[4]{#1 \mathbin{;} #2 \vdash #3 : #4}
\newcommand{\hpatmatch}[3]{#1 \vartriangleright #2 \dashV #3}
\newcommand{\hnotnotmatch}[2]{#1 \mathbin{\cancel{\bot}} #2}
\newcommand{\hnotmatch}[2]{#1 \mathbin{\bot} #2}
\newcommand{\hmaymatch}[2]{#1 \mathbin{?} #2}
\newcommand{\htrans}[2]{#1 \mapsto #2}
\newcommand{\hlongtrans}[2]{#1 \newline \mapsto #2}

\newcommand{\isVal}[1]{#1 ~\mathtt{val}}
\newcommand{\isErr}[1]{#1 ~\mathtt{err}}
\newcommand{\isIndet}[1]{#1 ~\mathtt{indet}}
\newcommand{\isFinal}[1]{#1 ~\mathtt{final}}
\newcommand{\notIntro}[1]{#1 ~\mathtt{notintro}}
\newcommand{\fnotIntro}[1]{\textit{notintro}\inparens{#1}}

\newcommand{\setof}[1]{\{#1\}}

% ZTyp and ZExp
\newcommand{\zlsel}[1]{{\bowtie}{#1}}
\newcommand{\zrsel}[1]{{#1}{\bowtie}}
\newcommand{\zwsel}[1]{
  \setlength{\fboxsep}{0pt}
  \colorbox{green!10!white!100}{
    \ensuremath{{{\textcolor{Green}{{\hspace{-2px}\triangleright}}}}{#1}{\textcolor{Green}{\triangleleft{\vphantom{\tehole}}}}}}
}

\newcommand{\removeSel}[1]{#1^{\diamond}}

% ZTyp
\newcommand{\ztau}{\hat{\tau}}

% ZExp
\newcommand{\zexp}{\hat{e}}

% rules with pointer

\newcommand{\zrulsP}[3]{\inparens{#1 \mid #2 \mid #3}}
\newcommand{\zruls}[3]{#1 \mid #2 \mid #3}
\newcommand{\zrules}{\hat{rs}}
\newcommand{\rmpointer}[1]{\inparens{#1}^\diamond}

% Constraint
\newcommand{\hxi}{\dot{\xi}}
\newcommand{\ctyp}[2]{#1 : #2}
\newcommand{\cdual}[1]{\overline{#1}}
\newcommand{\ctruify}[1]{\dot{\top}\inparens{#1}}
\newcommand{\cfalsify}[1]{\dot{\bot}\inparens{#1}}
\newcommand{\ctruth}{\top}
\newcommand{\cfalsity}{\bot}
\newcommand{\cnum}[1]{\underline{#1}}
\newcommand{\cnotnum}[1]{\underline{\cancel{#1}}}
\newcommand{\cunit}{()}
\newcommand{\cunknown}{\mathtt{?}}
\newcommand{\cor}[2]{#1 \vee #2}
\newcommand{\cand}[2]{#1 \wedge #2}
\newcommand{\cinj}[3]{\mathtt{inj}_{#1}^{#2}(#3)}
\newcommand{\notC}{\cancel{C}}
\newcommand{\nottagC}{\cancel{\tagC}}
\newcommand{\cinl}[1]{\mathtt{inl}(#1)}
\newcommand{\cinr}[1]{\mathtt{inr}(#1)}
\newcommand{\cpair}[2]{(#1 , #2)}
\newcommand{\csatisfy}[2]{#1 \models #2}
\newcommand{\cnotsatisfy}[2]{#1 \centernot{\models} #2}
\newcommand{\cmaysatisfy}[2]{#1 \models_{?} #2}
\newcommand{\cnotmaysatisfy}[2]{#1 \centernot{\models}_{?} #2}
\newcommand{\csatisfyormay}[2]{#1 \models_{?}^{\dag} #2}
\newcommand{\cnotsatisfyormay}[2]{#1 \centernot{\models}_{?}^{\dag} #2}
%temporary change, need to get it back to make appendix right
%\newcommand{\csatisfy}[2]{#1 \dot{\models} #2}
%\newcommand{\cnotsatisfy}[2]{#1 \centernot{\dot{\models}} #2}
%\newcommand{\cmaysatisfy}[2]{#1 \dot{\models}_{?} #2}
%\newcommand{\cnotmaysatisfy}[2]{#1 \centernot{\dot{\models}}_{?} #2}
%\newcommand{\csatisfyormay}[2]{#1 \dot{\models}_{?}^{\dag} #2}
%\newcommand{\cnotsatisfyormay}[2]{#1 \centernot{\dot{\models}}_{?}^{\dag} #2}
\newcommand{\ccsatisfy}[2]{#1 \models #2}
\newcommand{\ccnotsatisfy}[2]{#1 \centernot{\models} #2}
\newcommand{\cincon}[1]{#1 ~\mathtt{incon}}
\newcommand{\cmayincon}[1]{#1 ~\cancel{\mathtt{not~incon}}}

\newcommand{\fsatisfy}[2]{\textit{satisfy}\inparens{#1, #2}}
\newcommand{\fmaysatisfy}[2]{\textit{maysatisfy}\inparens{#1, #2}}
\newcommand{\fsatisfyormay}[2]{\textit{satisfyormay}\inparens{#1, #2}}

\newcommand{\chpattyp}[5]{#5 \vdash #1 : #2 [#3] \dashV #4}
\newcommand{\crultyp}[5]{#1 \vdash #2 : #3 [#4] \Rightarrow #5}
\newcommand{\chrultyp}[6]{#1 \mathbin{;} #2 \vdash #3 : #4 [#5] \Rightarrow #6}
\newcommand{\crulstyp}[6]{#1 \vdash [#2] #3 : #4 [#5] \Rightarrow #6}
\newcommand{\chrulstyp}[7]{#1 \mathbin{;} #2 \vdash [#3] #4 : #5 [#6] \Rightarrow #7}
\newcommand{\czrulstyp}[7]{#1 \mathbin{;} #2 \vdash [#3] #4 [#5] : #6 \Rightarrow #7}

% Direction
\newcommand{\dParent}{\mathtt{parent}}
\newcommand{\dChildn}[1]{\mathtt{child}~\mathtt{{#1}}}
\newcommand{\dChildnm}[1]{\mathtt{child}~{#1}}

% Action
\newcommand{\aMove}[1]{\mathtt{move}~#1}
	\newcommand{\zrightmost}[1]{\mathsf{rightmost}(#1)}
	\newcommand{\zleftmost}[1]{\mathsf{leftmost}(#1)}
\newcommand{\aSelect}[1]{\mathtt{sel}~#1}
\newcommand{\aDel}{\mathtt{del}}
\newcommand{\aReplace}[1]{\mathtt{replace}~#1}
\newcommand{\aConstruct}[1]{\mathtt{construct}~#1}
\newcommand{\aConstructx}[1]{#1}
\newcommand{\aFinish}{\mathtt{finish}}

\newcommand{\performAna}[5]{#1 \vdash #2 \xlongrightarrow{#4} #5 \Leftarrow #3}
\newcommand{\performAnaI}[5]{#1 \vdash #2 \xlongrightarrow{#4}\hspace{-3px}{}^{*}~ #5 \Leftarrow #3}
\newcommand{\performSyn}[6]{#1 \vdash #2 \Rightarrow #3 \xlongrightarrow{#4} #5 \Rightarrow #6}
\newcommand{\performSynI}[6]{#1 \vdash #2 \Rightarrow #3 \xlongrightarrow{#4}\hspace{-3px}{}^{*}~ #5 \Rightarrow #6}
\newcommand{\performTyp}[3]{#1 \xlongrightarrow{#2} #3}
\newcommand{\performTypI}[3]{#1 \xlongrightarrow{#2}\hspace{-3px}{}^{*}~#3}

\newcommand{\performMove}[3]{#1 \xlongrightarrow{#2} #3}
\newcommand{\performDel}[2]{#1 \xlongrightarrow{\aDel} #2}

% Form
\newcommand{\farr}{\mathtt{arrow}}
\newcommand{\fnum}{\mathtt{num}}
\newcommand{\fsum}{\mathtt{sum}}

\newcommand{\fasc}{\mathtt{asc}}
\newcommand{\fvar}[1]{\mathtt{var}~#1}
\newcommand{\flam}[1]{\mathtt{lam}~#1}
\newcommand{\fap}{\mathtt{ap}}
% \newcommand{\farg}{\mathtt{arg}}
\newcommand{\fnumlit}[1]{\mathtt{lit}~#1}
\newcommand{\fplus}{\mathtt{plus}}
\newcommand{\fhole}{\mathtt{hole}}
\newcommand{\fnehole}{\mathtt{nehole}}

\newcommand{\finj}[1]{\mathtt{inj}~#1}
\newcommand{\fcase}[2]{\mathtt{case}~#1~#2}

% Talk about formal rules in example
\newcommand{\refrule}[1]{\textrm{Rule~(#1)}}

\newcommand{\herase}[1]{\left|#1\right|_\textsf{erase}}

\newcommand{\arrmatch}[2]{#1 \blacktriangleright_{\rightarrow} #2}


\newcommand{\TABperformAna}[5]{#1 \vdash & #2                & \xlongrightarrow{#4} & #5 & \Leftarrow #3}
\newcommand{\TABperformSyn}[6]{#1 \vdash & #2 \Rightarrow #3 & \xlongrightarrow{#4} & #5 \Rightarrow #6}
\newcommand{\TABperformTyp}[3]{& #1 & \xlongrightarrow{#2} & #3}

\newcommand{\TABperformMove}[3]{#1 & \xlongrightarrow{#2} & #3}
\newcommand{\TABperformDel}[2]{#1 \xlongrightarrow{\aDel} #2}

\newcommand{\sumhasmatched}[2]{#1 \mathrel{\textcolor{black}{\blacktriangleright_{+}}} #2}

\newcommand{\subminsyn}[1]{\mathsf{submin}_{\Rightarrow}(#1)}
\newcommand{\subminana}[1]{\mathsf{submin}_{\Leftarrow}(#1)}


\newcommand{\inparens}[1]{{\color{gray}(}#1{\color{gray})}}
\newcommand{\true}{\text{true}}
\newcommand{\false}{\text{false}}

\newcommand{\textnot}{\text{not }}
\newcommand{\textor}{\text{ or }}
\newcommand{\textand}{\text{ and }}
\newcommand{\textif}{\quad\text{if }}

%% rule names for appendix
\newcommand{\rname}[1]{\textsc{#1}}
\newcommand{\gap}{\vspace{7pt}}

%% just for examples
\newcommand{\match}{\mathtt{match}}
\newcommand{\?}{\mathtt{?}}
\newcommand{\nil}{[~]}
\newcommand{\some}[1]{\mathtt{Some\inparens{#1}}}
\newcommand{\none}{\mathtt{None}}

\let\Autoref\undefined
\makeatletter
% an adaption from 
% https://tex.stackexchange.com/questions/15728/multiple-references-with-autoref
% define a macro \Autoref to allow multiple references to be passed to \autoref
\newcommand\Autoref[1]{\@first@ref#1,@}
\def\@throw@dot#1.#2@{#1}% discard everything after the dot
\def\@set@refname#1{%    % set \@refname to autoefname+s using \getrefbykeydefault
    \edef\@tmp{\getrefbykeydefault{#1}{anchor}{}}%
    \xdef\@tmp{\expandafter\@throw@dot\@tmp.@}%
    \ltx@IfUndefined{\@tmp autorefnameplural}%
         {\def\@refname{\@nameuse{\@tmp autorefname}}}%
         {\def\@refname{\@nameuse{\@tmp autorefnameplural}}}%
}
\def\@first@ref#1,#2{%
  \ifx#2@\autoref{#1}\let\@nextref\@gobble% only one ref, revert to normal \autoref
  \else%
    \@set@refname{#1}%  set \@refname to autoref name
    \@refname~\ref{#1}% add autoefname and first reference
    \let\@nextref\@next@ref% push processing to \@next@ref
  \fi%
  \@nextref#2%
}
\def\@next@ref#1,#2{%
   \ifx#2@ and~\ref{#1}\let\@nextref\@gobble% at end: print and+\ref and stop
   \else, \ref{#1}% print  ,+\ref and continue
   \fi%
   \@nextref#2%
}
\makeatother

%%% Local Variables:
%%% mode: latex
%%% TeX-master: "appendix/appendix"
%%% End:


\usepackage{comment}
\excludecomment{proof}
\begin{document}

In the main paper, we present only a single constraint language. However, conceptually, we work with this language in two distinct stages: first, the constraints directly emitted by lists of rules, then, for use in redundancy and exhaustiveness checking, the constraints which are in the image of the truify and falsify functions and their duals. While irrelevant to the overall theory, to simplify some proofs, it is useful to make this distinction explicit.

In \autoref{sec:incompleteconstraint}, we present the first stage of constraints, called the \emph{incomplete constraint language}. This consists of any constraint emitted by a pattern, and in particular, includes the $\cunknown$ constraint. In order to define the constraint emitted by a list of rules, we also include $\cfalsity$ and allow taking the $\vee$ of incomplete constraints. At this stage, we often require two versions of each judgement: one describing a determinate result, and one describing a result which is indeterminate due to the presence of the $\cunknown$ constraint.

In turn, in \autoref{sec:completeconstraint}, we discuss those constraints in the image of the truify and falsify functions, as well as their duals. We call this the \emph{complete constraint language}, and it includes almost all of the incomplete language, but excludes the $\cunknown$ constraint. To support the dual operation, we also may take the $\wedge$ of complete constraints, and we add a $\cnotnum{n}$ constraint. Due to the absence of $\cunknown$, judgements related to the complete language do not have to consider indeterminacy, and thus are often simpler than their counterparts in the incomplete language. This is the primary motivation for distinguishing these languages at all.

\section{Match Constraint Language}
$\arraycolsep=4pt\begin{array}{lll}
\hxi & ::= &
  \ctruth ~\vert~
  \cfalsity ~\vert~
  \cunknown ~\vert~
  \cnum{n} ~\vert~
  \cinl{\hxi} ~\vert~
  \cinr{\hxi} ~\vert~
  \cpair{\hxi}{\hxi} ~\vert~
  \cor{\hxi}{\hxi}
\end{array}$

\judgboxa{\ctyp{\hxi}{\tau}}{$\hxi$ constrains final expressions of type $\tau$}
\begin{subequations}\label{rules:CTyp}
\begin{equation}\label{rule:CTTruth}
\inferrule[CTTruth]{ }{
  \ctyp{\ctruth}{\tau}
}
\end{equation}
\begin{equation}\label{rule:CTUnknown}
\inferrule[CTUnknown]{ }{
  \ctyp{\cunknown}{\tau}
}
\end{equation}
\begin{equation}\label{rule:CTNum}
\inferrule[CTNum]{ }{
  \ctyp{\cnum{n}}{\tnum}
}
\end{equation}
\begin{equation}\label{rule:CTInl}
\inferrule[CTInl]{
  \ctyp{\hxi_1}{\tau_1}
}{
  \ctyp{\cinl{\hxi_1}}{\tsum{\tau_1}{\tau_2}}
}
\end{equation}
\begin{equation}\label{rule:CTInr}
\inferrule[CTInr]{
  \ctyp{\hxi_2}{\tau_2}
}{
  \ctyp{\cinr{\hxi_2}}{\tsum{\tau_1}{\tau_2}}
}
\end{equation}
\begin{equation}\label{rule:CTPair}
\inferrule[CTPair]{
  \ctyp{\hxi_1}{\tau_1} \\ \ctyp{\hxi_2}{\tau_2}
}{
  \ctyp{\cpair{\hxi_1}{\hxi_2}}{\tprod{\tau_1}{\tau_2}}
}
\end{equation}
\begin{equation}\label{rule:CTOr}
\inferrule[CTOr]{
  \ctyp{\hxi_1}{\tau} \\ \ctyp{\hxi_2}{\tau}
}{
  \ctyp{\cor{\hxi_1}{\hxi_2}}{\tau}
}
\end{equation}
\end{subequations}

\judgboxa{
  \refutable{\hxi}
}{$\hxi$ is refutable}

\begin{subequations}\label{rules:xi-refutable}
\begin{equation}\label{rule:RXNum}
\inferrule[RXNum]{ }{
  \refutable{\cnum{n}}
}
\end{equation}
\begin{equation}\label{rule:RXUnknown}
\inferrule[RXUnknown]{ }{
  \refutable{\cunknown}
}
\end{equation}
\begin{equation}\label{rule:RXInl}
\inferrule[RXInl]{ }{
  \refutable{\cinl{\hxi}}
}
\end{equation}
\begin{equation}\label{rule:RXInr}
\inferrule[RXInr]{ }{
  \refutable{\cinr{\hxi}}
}
\end{equation}
\begin{equation}\label{rule:RXPairL}
\inferrule[RXPairL]{
  \refutable{\hxi_1}
}{
  \refutable{\cpair{\hxi_1}{\hxi_2}}
}
\end{equation}
\begin{equation}\label{rule:RXPairR}
\inferrule[RXPairR]{
  \refutable{\hxi_2}
}{
  \refutable{\cpair{\hxi_1}{\hxi_2}}
}
\end{equation}
\begin{equation}\label{rule:RXOr}
  \inferrule[RXOr]{
    \refutable{\hxi_1} \\
    \refutable{\hxi_2}
  }{
    \refutable{\cor{\hxi_1}{\hxi_2}}
  }
  \end{equation}
\end{subequations}

\judgboxa{\frefutable{\hxi}}{}
\begin{subequations}\label{defn:xi-refutable}
\begin{align}
    \frefutable{\cnum{n}} &= \true \\
    \frefutable{\cunknown} &= \true \\
    \frefutable{\cinl{\hxi}} &= \frefutable{\hxi} \\
    \frefutable{\cinr{\hxi}} &= \frefutable{\hxi} \\
    \frefutable{\cpair{\hxi_1}{\hxi_2}} &= \frefutable{\hxi_1} \text{ or } \frefutable{\hxi_2} \\
    \frefutable{\cor{\hxi_1}{\hxi_2}} &= \frefutable{\hxi_1} \text{ and } \frefutable{\hxi_2} \\
  \text{Otherwise}\quad \frefutable{\hxi} &= \false 
\end{align}
\end{subequations}


\judgboxa{\csatisfy{e}{\hxi}}{$e$ satisfies $\hxi$}
\begin{subequations}\label{rules:Satisfy}
\begin{equation}\label{rule:CSTruth}
\inferrule[CSTruth]{ }{
  \csatisfy{e}{\ctruth}
}
\end{equation}
\begin{equation}\label{rule:CSNum}
\inferrule[CSNum]{ }{
  \csatisfy{\hnum{n}}{\cnum{n}}
}
\end{equation}
\begin{equation}\label{rule:CSInl}
\inferrule[CSInl]{
  \csatisfy{e_1}{\hxi_1}
}{
  \csatisfy{
    \hinl{\tau_2}{e_1}
  }{
    \cinl{\hxi_1}
  }
}
\end{equation}
\begin{equation}\label{rule:CSInr}
\inferrule[CSInr]{
  \csatisfy{e_2}{\hxi_2}
}{
  \csatisfy{
    \hinr{\tau_1}{e_2}
  }{
    \cinr{\hxi_2}
  }
}
\end{equation}
\begin{equation}\label{rule:CSPair}
\inferrule[CSPair]{
  \csatisfy{e_1}{\hxi_1} \\
  \csatisfy{e_2}{\hxi_2}
}{
\csatisfy{\hpair{e_1}{e_2}}{\cpair{\hxi_1}{\hxi_2}}
}
\end{equation}
\begin{equation}\label{rule:CSNotIntroPair}
\inferrule[CSNotIntroPair]{
  \notIntro{e} \\
  \csatisfy{\hprl{e}}{\hxi_1} \\
  \csatisfy{\hprr{e}}{\hxi_2}
}{
  \csatisfy{e}{\cpair{\hxi_1}{\hxi_2}}
}
\end{equation}
\begin{equation}\label{rule:CSOr1}
\inferrule[CSOrL]{
  \csatisfy{e}{\hxi_1}
}{
  \csatisfy{e}{\cor{\hxi_1}{\hxi_2}}
}
\end{equation}
\begin{equation}\label{rule:CSOr2}
\inferrule[CSOrR]{
  \csatisfy{e}{\hxi_2}
}{
  \csatisfy{e}{\cor{\hxi_1}{\hxi_2}}
}
\end{equation}
\end{subequations}

\judgboxa{\fsatisfy{e}{\hxi}}{}
\begin{subequations}\label{defn:satisfy}
\begin{align}
  \fsatisfy{e}{\ctruth} ={}& \true \label{defn:satisfy-truth}\\
  \fsatisfy{\hnum{n_1}}{\cnum{n_2}} ={}& (n_1 = n_2) \label{defn:num-satisfy-num}\\
  \fsatisfy{e}{\cor{\hxi_1}{\hxi_2}} ={}& \fsatisfy{e}{\hxi_1} \text{ or } \fsatisfy{e}{\hxi_2} \label{defn:satisfy-or}\\
  \fsatisfy{\hinl{\tau_2}{e_1}}{\cinl{\hxi_1}} ={}& \fsatisfy{e_1}{\hxi_1} \label{defn:inl-satisfy-inl}\\
  \fsatisfy{\hinr{\tau_1}{e_2}}{\cinr{\hxi_2}} ={}& \fsatisfy{e_2}{\hxi_2} \label{defn:inr-satisfy-inr}\\
  \fsatisfy{\hpair{e_1}{e_2}}{\cpair{\hxi_1}{\hxi_2}} ={}& \fsatisfy{e_1}{\hxi_1} \text{ and } \fsatisfy{e_2}{\hxi_2} \label{defn:pair-satisfy-pair}\\
  \fsatisfy{\hehole{u}}{\cpair{\hxi_1}{\hxi_2}} ={}& \fsatisfy{\hprl{\hehole{u}}}{\hxi_1} \text{ and } \fsatisfy{\hprr{\hehole{u}}}{\hxi_2}
  \label{defn:ehole-satisfy-pair} \\
  \fsatisfy{\hhole{e}{u}}{\cpair{\hxi_1}{\hxi_2}} ={}& \fsatisfy{\hprl{\hhole{e}{u}}}{\hxi_1} \text{ and } \fsatisfy{\hprr{\hhole{e}{u}}}{\hxi_2}
  \label{defn:hole-satisfy-pair} \\
  \fsatisfy{\hap{e_1}{e_2}}{\cpair{\hxi_1}{\hxi_2}} ={}& \fsatisfy{\hprl{\hap{e_1}{e_2}}}{\hxi_1} \notag\\
  &\text{ and } \fsatisfy{\hprr{\hap{e_1}{e_2}}}{\hxi_2}
  \label{defn:ap-satisfy-pair} \\
  \fsatisfy{\hmatch{e}{\zrules}}{\cpair{\hxi_1}{\hxi_2}} ={}& \fsatisfy{\hprl{\hmatch{e}{\zrules}}}{\hxi_1} \notag\\
  &\text{ and } \fsatisfy{\hprr{\hmatch{e}{\zrules}}}{\hxi_2}
  \label{defn:match-satisfy-pair} \\
  \fsatisfy{\hprl{e}}{\cpair{\hxi_1}{\hxi_2}} ={}& \fsatisfy{\hprl{\hprl{e}}}{\hxi_1} \notag\\
  &\text{ and } \fsatisfy{\hprr{\hprl{e}}}{\hxi_2}
  \label{defn:prl-satisfy-pair} \\
  \fsatisfy{\hprr{e}}{\cpair{\hxi_1}{\hxi_2}} ={}& \fsatisfy{\hprl{\hprr{e}}}{\hxi_1} \notag\\
  &\text{ and } \fsatisfy{\hprr{\hprr{e}}}{\hxi_2}
  \label{defn:prr-satisfy-pair} \\
  \text{Otherwise}\quad \fsatisfy{e}{\hxi} ={}& \false \label{defn:not-satisfy}
\end{align}
\end{subequations}

\judgboxa{\cmaysatisfy{e}{\hxi}}{$e$ may satisfy $\hxi$}
\begin{subequations}\label{rules:MaySatisfy}
\begin{equation}\label{rule:CMSUnknown}
\inferrule[CMSUnknown]{ }{
  \cmaysatisfy{e}{\cunknown}
}
\end{equation}
\begin{equation}\label{rule:CMSInl}
\inferrule[CMSInl]{
  \cmaysatisfy{e_1}{\hxi_1}
}{
  \cmaysatisfy{
    \hinl{\tau_2}{e_1}
  }{
    \cinl{\hxi_1}
  }
}
\end{equation}
\begin{equation}\label{rule:CMSInr}
\inferrule[CMSInr]{
  \cmaysatisfy{e_2}{\hxi_2}
}{
  \cmaysatisfy{
    \hinr{\tau_1}{e_2}
  }{
    \cinr{\hxi_2}
  }
}
\end{equation}
\begin{equation}\label{rule:CMSPair1}
\inferrule[CMSPairL]{
  \cmaysatisfy{e_1}{\hxi_1} \\
  \csatisfy{e_2}{\hxi_2}
}{
  \cmaysatisfy{\hpair{e_1}{e_2}}{\cpair{\hxi_1}{\hxi_2}}
}
\end{equation}
\begin{equation}\label{rule:CMSPair2}
\inferrule[CMSPairR]{
  \csatisfy{e_1}{\hxi_1} \\
  \cmaysatisfy{e_2}{\hxi_2}
}{
  \cmaysatisfy{\hpair{e_1}{e_2}}{\cpair{\hxi_1}{\hxi_2}}
}
\end{equation}
\begin{equation}\label{rule:CMSPair3}
\inferrule[CMSPair]{
  \cmaysatisfy{e_1}{\hxi_1} \\
  \cmaysatisfy{e_2}{\hxi_2}
}{
  \cmaysatisfy{\hpair{e_1}{e_2}}{\cpair{\hxi_1}{\hxi_2}}
}
\end{equation}
\begin{equation}\label{rule:CMSOr1}
\inferrule[CMSOrL]{
  \cmaysatisfy{e}{\hxi_1} \\
  \cnotsatisfy{e}{\hxi_2}
}{
  \cmaysatisfy{e}{\cor{\hxi_1}{\hxi_2}}
}
\end{equation}
\begin{equation}\label{rule:CMSOr2}
\inferrule[CMSOrR]{
  \cnotsatisfy{e}{\hxi_1} \\
  \cmaysatisfy{e}{\hxi_2}
}{
  \cmaysatisfy{e}{\cor{\hxi_1}{\hxi_2}}
}
\end{equation}
\begin{equation}\label{rule:CMSNotIntro}
\inferrule[CMSNotIntro]{
  \notIntro{e} \\
  \refutable{\hxi}
}{
  \cmaysatisfy{e}{\hxi}
}
\end{equation}
\end{subequations}

\judgboxa{\csatisfyormay{e}{\hxi}}{$e$ satisfies or may satisfy $\hxi$}
\begin{subequations}\label{rules:satormay}
\begin{equation}\label{rule:CSMSMay}
\inferrule[CSMSMay]{
  \cmaysatisfy{e}{\hxi}
}{
  \csatisfyormay{e}{\hxi}
}
\end{equation}
\begin{equation}\label{rule:CSMSSat}
\inferrule[CSMSSat]{
  \csatisfy{e}{\hxi}
}{
  \csatisfyormay{e}{\hxi}
}
\end{equation}
\end{subequations}

\begin{lemma}
  \label{lem:no-e-satisfy-falsity}
  $\cnotsatisfy{e}{\cfalsity}$
\end{lemma}
\begin{proof}
  By rule induction over Rules (\ref{rules:Satisfy}), we notice that $\csatisfy{e}{\cfalsity}$ is in syntactic contradiction with all rules, hence not derivable.
\end{proof}

\begin{lemma}
  \label{lem:no-e-may-satisfy-falsity}
  $\cnotmaysatisfy{e}{\cfalsity}$
\end{lemma}
\begin{proof}
  Assume $\cmaysatisfy{e}{\cfalsity}$.
  By rule induction over Rules (\ref{rules:MaySatisfy}) on $\cmaysatisfy{e}{\cfalsity}$, only one case applies.
  \begin{byCases}
  \item[\text{(\ref{rule:CMSNotIntro})}]
    \begin{pfsteps*}
    \item $\refutable{\cfalsity}$ \BY{assumption} \pflabel{refutable}
    \end{pfsteps*}
    By rule induction over Rules (\ref{rules:xi-refutable}) on \pfref{refutable}, no case applies due to syntactic contradiction.
  \end{byCases}
  Therefore, $\cmaysatisfy{e}{\cfalsity}$ is not derivable.
  \resetpfcounter
\end{proof}

\begin{lemma}
  \label{lem:no-e-may-satisfy-truth}
  $\cnotmaysatisfy{e}{\ctruth}$
\end{lemma}
\begin{proof}
  Assume $\cmaysatisfy{e}{\ctruth}$.
  By rule induction over Rules (\ref{rules:MaySatisfy}) on $\cmaysatisfy{e}{\ctruth}$, only one case applies.
  \begin{byCases}
  \item[\text{(\ref{rule:CMSNotIntro})}]
    \begin{pfsteps*}
    \item $\refutable{\ctruth}$ \BY{assumption} \pflabel{refutable}
    \end{pfsteps*}
    By rule induction over Rules (\ref{rules:xi-refutable}) on \pfref{refutable}, no case applies due to syntactic contradiction.
  \end{byCases}
  Therefore, $\cmaysatisfy{e}{\ctruth}$ is not derivable.
  \resetpfcounter
\end{proof}

\begin{lemma}
  \label{lem:no-e-satisfy-unknown}
  $\cnotsatisfy{e}{\cunknown}$
\end{lemma}
\begin{proof}
  By rule induction over Rules (\ref{rules:Satisfy}), we notice that $\csatisfy{e}{\cunknown}$ is in syntactic contradiction with all the cases, hence not derivable.
\end{proof}

\begin{lemma}
  \label{lem:relax-nn-satisfy}
  $\csatisfyormay{e}{\hxi}$ iff $\csatisfyormay{e}{\cor{\hxi}{\cfalsity}}$
\end{lemma}
\begin{proof}
  We prove sufficiency and necessity separately.
  \begin{enumerate}
    \item Sufficiency:
      \begin{pfsteps*}
      \item $\csatisfyormay{e}{\hxi}$ \BY{assumption} \pflabel{nnsatisfyxi}
      \end{pfsteps*}
      By rule induction over Rules (\ref{rules:satormay}) on \pfref{nnsatisfyxi}.
      \begin{byCases}

      \savelocalsteps{lem:relax-nn-satisfy-suff-1}
      \item[\text{(\ref{rule:CSMSMay})}]
        \begin{pfsteps*}
        \item $\cmaysatisfy{e}{\hxi}$ \BY{assumption} \pflabel{maysatisfyxi}
        \item $\cmaysatisfy{e}{\cor{\hxi}{\cfalsity}}$ \BY{Rule (\ref{rule:CMSOr1}) on \pfref{maysatisfyxi} and Lemma \ref{lem:no-e-satisfy-falsity}} \pflabel{maysatisfyxi+bot}
        \item $\csatisfyormay{e}{\cor{\hxi}{\cfalsity}}$ \BY{Rule (\ref{rule:CSMSMay}) on \pfref{maysatisfyxi+bot}}
        \end{pfsteps*}

      \restorelocalsteps{lem:relax-nn-satisfy-suff-1}
      \item[\text{(\ref{rule:CSMSSat})}]
        \begin{pfsteps*}
        \item $\csatisfy{e}{\hxi}$ \BY{assumption} \pflabel{satisfyxi}
        \item $\csatisfy{e}{\cor{\hxi}{\cfalsity}} \BY{Rule (\ref{rule:CSOr1}) on \pfref{satisfyxi}} \pflabel{satisfyxi+bot}
        \item $\csatisfyormay{e}{\cor{\hxi}{\cfalsity}} \BY{Rule (\ref{rule:CSMSSat}) on \pfref{satisfyxi+bot}}
        \end{pfsteps*}

      \end{byCases}

    \resetpfcounter

    \item Necessity:
      \begin{pfsteps*}
      \item $\csatisfyormay{e}{\cor{\hxi}{\cfalsity}}$ \BY{assumption} \pflabel{nnsatisfyxi+bot}
      \end{pfsteps*}
      By rule induction over Rules (\ref{rules:satormay}) on \pfref{nnsatisfyxi+bot}.
      \begin{byCases}

      \savelocalsteps{lem:relax-nn-satisfy-necs-1}
      \item[\text{(\ref{rule:CSMSMay})}]
        \begin{pfsteps*}
        \item $\cmaysatisfy{e}{\cor{\hxi}{\cfalsity}}$ \BY{assumption} \pflabel{maysatisfyxi+bot}
        \end{pfsteps*}
        By rule induction over Rules (\ref{rules:MaySatisfy}) on \pfref{maysatisfyxi+bot}, only two of them apply.
        \begin{byCases}

        \savelocalsteps{lem:relax-nn-satisfy-necs-2}
        \item[\text{(\ref{rule:CMSOr1})}]
          \begin{pfsteps*}
          \item $\cmaysatisfy{e}{\hxi}$ \BY{assumption} \pflabel{maysatisfyxi}
          \item $\csatisfyormay{e}{\hxi}$ \BY{Rule (\ref{rule:CSMSMay}) on \pfref{maysatisfyxi}}
          \end{pfsteps*}

        \restorelocalsteps{lem:relax-nn-satisfy-necs-2}
        \item[\text{(\ref{rule:CMSOr2})}]
          \begin{pfsteps*}
          \item $\cmaysatisfy{e}{\cfalsity}$ \BY{assumption} \pflabel{maysatisfybot}
          \item $\cnotmaysatisfy{e}{\cfalsity}$ \BY{Lemma \ref{lem:no-e-may-satisfy-falsity}} \pflabel{notmaysatisfybot}
          \end{pfsteps*}
          \pfref{maysatisfybot} contradicts \pfref{notmaysatisfybot}.

        \end{byCases}

      \restorelocalsteps{lem:relax-nn-satisfy-necs-1}
      \item[\text{(\ref{rule:CSMSSat})}]
        \begin{pfsteps*}
        \item $\csatisfy{e}{\cor{\hxi}{\cfalsity}}$ \BY{assumption} \pflabel{satisfyxi+bot}
        \end{pfsteps*}
        By rule induction over Rules (\ref{rules:Satisfy}) on \pfref{satisfyxi+bot}, only two of them apply.
        \begin{byCases}

        \savelocalsteps{lem:relax-nn-satisfy-necs-2}
        \item[\text{(\ref{rule:CSOr1})}]
          \begin{pfsteps*}
          \item $\csatisfy{e}{\hxi}$ \BY{assumption} \pflabel{satisfyxi}
          \item $\csatisfyormay{e}{\hxi}$ \BY{Rule (\ref{rule:CSMSSat}) on \pfref{satisfyxi}}
          \end{pfsteps*}

        \restorelocalsteps{lem:relax-nn-satisfy-necs-2}
        \item[\text{(\ref{rule:CSOr2})}]
          \begin{pfsteps*}
          \item $\csatisfy{e}{\cfalsity}$ \BY{assumption} \pflabel{satisfybot}
          \item $\cnotsatisfy{e}{\cfalsity}$ \BY{Lemma \ref{lem:no-e-satisfy-falsity}} \pflabel{notsatisfybot}
          \end{pfsteps*}
          \pfref{satisfybot} contradicts \pfref{notsatisfybot}.
        \end{byCases}

      \resetpfcounter
      \end{byCases}
  \end{enumerate}
\end{proof}

\begin{corollary}
  \label{corol:relax-nn-entail}
  $\csatisfyormay{\ctruth}{\hxi}$ iff $\csatisfyormay{\ctruth}{\cor{\hxi}{\cfalsity}}$
\end{corollary}
\begin{proof}
  Follows directly from Definition \ref{defn:nn-entailment} and Lemma \ref{lem:relax-nn-satisfy}.
\end{proof}

\begin{lemma}
  \label{lem:relax-not-redundant}
  Suppose that $\ctyp{\hxi_1}{\tau}$ and $\ctyp{\hxi_2}{\tau}$. Then $\cnotsatisfy{\hxi_1}{\hxi_2}$ iff $\cnotsatisfy{\hxi_1}{\cor{\hxi_2}{\cfalsity}}$
\end{lemma}
\begin{proof}
  \begin{pfsteps*}
  \item $\ctyp{\hxi_1}{\tau}$ \BY{assumption} \pflabel{1CTyp}
  \item $\ctyp{\hxi_2}{\tau}$ \BY{assumption} \pflabel{2CTyp}
  \item $\ctyp{\cfalsity}{\tau}$ \BY{Rule (\ref{rule:CTFalsity})} \pflabel{fCTyp}
  \item $\ctyp{\cor{\hxi_2}{\cfalsity}}{\tau}$ \BY{Rule (\ref{rule:CTOr}) on \pfref{2CTyp} and \pfref{fCTyp}} \pflabel{2+botCTyp}
  \end{pfsteps*}
  Then we prove sufficiency and necessity separately.
\begin{enumerate}
  
\savelocalsteps{lem:relax-not-redundant-0}
\item Sufficiency:
  \begin{pfsteps*}
  \item $\cnotsatisfy{\hxi_1}{\hxi_2}$ \BY{assumption} \pflabel{1notsatisfy2}
  \end{pfsteps*}
  To prove $\cnotsatisfy{\hxi_1}{\cor{\hxi_2}{\cfalsity}}$, assume $\csatisfy{\hxi_1}{\cor{\hxi_2}{\cfalsity}}$. 
  \begin{pfsteps*}
  \item $\csatisfy{\hxi_1}{\cor{\hxi_2}{\cfalsity}}$ \BY{assumption} \pflabel{1=>2+bot}
  \end{pfsteps*}
  For all $e$ such that $\hexptyp{\cdot}{\Delta}{e}{\tau}$ and $\isFinal{e}$ we have $\csatisfyormay{e}{\hxi_1}$ implies
  \begin{pfsteps*}
  \item $\csatisfy{e}{\cor{\hxi_2}{\cfalsity}}$ \BY{Definition \ref{defn:const-entailment} on \pfref{1CTyp} and \pfref{2+botCTyp} and \pfref{1=>2+bot}} \pflabel{satisfy2+bot}
  \end{pfsteps*}
  By rule induction over Rules (\ref{rules:Satisfy}) on \pfref{satisfy2+bot}.
  \begin{byCases}

  \savelocalsteps{lem:relax-not-redundant-1}
  \item[\text{(\ref{rule:CSOr1})}]
    \begin{pfsteps*}
    \item $\csatisfy{e}{\hxi_2}$ \BY{assumption} \pflabel{satisfy2}
    \item $\csatisfy{\hxi_1}{\hxi_2}$ \BY{Definition \ref{defn:const-entailment} on \pfref{satisfy2}} \pflabel{1satisfy2}
    \end{pfsteps*}
    \pfref{1notsatisfy2} contradicts \pfref{1satisfy2}.


  \restorelocalsteps{lem:relax-not-redundant-1}
  \item[\text{(\ref{rule:CSOr2})}]
    \begin{pfsteps*}
    \item $\csatisfy{e}{\cfalsity}$ \BY{assumption} \pflabel{satisfybot}
    \item $\cnotsatisfy{e}{\cfalsity}$ \BY{Lemma \ref{lem:no-e-satisfy-falsity}} \pflabel{notsatisfybot}
    \end{pfsteps*}
    \pfref{satisfybot} contradicts \pfref{notsatisfybot}.
  \end{byCases}
  The conclusion holds as follows:
  \begin{enumerate}
    \item $\cnotsatisfy{\hxi_1}{\cor{\hxi_2}{\cfalsity}}$
  \end{enumerate}

\restorelocalsteps{lem:relax-not-redundant-0}
\item Necessity:
  \begin{pfsteps*}
  \item $\cnotsatisfy{\hxi_1}{\cor{\hxi_2}{\cfalsity}}$ \BY{assumption} \pflabel{1notentail2+bot}
  \end{pfsteps*}
  To prove $\cnotsatisfy{\hxi_1}{\hxi_2}$, assume $\csatisfy{\hxi_1}{\hxi_2}$.
  \begin{pfsteps*}
  \item $\csatisfy{\hxi_1}{\hxi_2}$ \BY{assumption} \pflabel{1entail2}
  \end{pfsteps*}
  For all $e$ such that $\hexptyp{\cdot}{\Delta}{e}{\tau}$ and $\isFinal{e}$ we have $\csatisfyormay{e}{\hxi_1}$ implies
  \begin{pfsteps*}
  \item $\csatisfy{e}{\hxi_2}$ \BY{Definition \ref{defn:const-entailment} on \pfref{1CTyp} and \pfref{2CTyp} and \pfref{1entail2}} \pflabel{esatisfy2}
  \item $\csatisfy{e}{\cor{\hxi_2}{\cfalsity}}$ \BY{Rule (\ref{rule:CSOr1}) on \pfref{esatisfy2}} \pflabel{esatisfy2+bot}
  \item $\csatisfy{\hxi_1}{\cor{\hxi_2}{\cfalsity}}$ \BY{Definition \ref{defn:const-entailment} on \pfref{esatisfy2+bot}} \pflabel{1|=2vbot}
  \end{pfsteps*}
  \pfref{1|=2vbot} contradicts \pfref{1notentail2+bot}.

  The conclusion holds as follows:
  \begin{enumerate}
    \item $\cnotsatisfy{\hxi_1}{\hxi_2}$
  \end{enumerate}
\end{enumerate}
\resetpfcounter
\end{proof}

\begin{lemma}
  \label{lem:or-nn-satisfy}
  If $\cnotsatisfyormay{e}{\hxi_1}$ and $\cnotsatisfyormay{e}{\hxi_2}$ then $\cnotsatisfyormay{e}{\cor{\hxi_1}{\hxi_2}}$
\end{lemma}
\begin{proof}
  Assume, for the sake of contradiction, that $\csatisfyormay{e}{\cor{\hxi_1}{\hxi_2}}$.
\begin{pfsteps*}
\item $\csatisfyormay{e}{\cor{\hxi_1}{\hxi_2}}$ \BY{assumption} \pflabel{satormay1+2}
\item $\cnotsatisfyormay{e}{\hxi_1}$ \BY{assumption} \pflabel{notsatormay1}
\item $\cnotsatisfyormay{e}{\hxi_2}$ \BY{assumption} \pflabel{notsatormay2}
\end{pfsteps*}
By rule induction over Rules (\ref{rules:satormay}) on \pfref{satormay1+2}.
\begin{byCases}

\savelocalsteps{lem:or-nn-satisfy-1}
\item[\text{(\ref{rule:CSMSSat})}]
  \begin{pfsteps*}
  \item $\csatisfy{e}{\cor{\hxi_1}{\hxi_2}}$ \BY{assumption} \pflabel{satisfy1+2}
  \end{pfsteps*}
  By rule induction over Rules (\ref{rules:Satisfy}) on \pfref{satisfy1+2} and only two of them apply.
  \begin{byCases}

  \savelocalsteps{lem:or-nn-satisfy-2}
  \item[\text{(\ref{rule:CSOr1})}]
    \begin{pfsteps*}
    \item $\csatisfy{e}{\hxi_1}$ \BY{assumption} \pflabel{satisfy1}
    \item $\csatisfyormay{e}{\hxi_1}$ \BY{Rule (\ref{rule:CSMSSat}) on \pfref{satisfy1}} \pflabel{satormay1}
    \end{pfsteps*}
    \pfref{satormay1} contradicts \pfref{notsatormay1}.

  \restorelocalsteps{lem:or-nn-satisfy-2}
  \item[\text{(\ref{rule:CSOr2})}]
    \begin{pfsteps*}
    \item $\csatisfy{e}{\hxi_2}$ \BY{assumption} \pflabel{satisfy2}
    \item $\csatisfyormay{e}{\hxi_2}$ \BY{Rule (\ref{rule:CSMSSat}) on \pfref{satisfy2}} \pflabel{satormay2}
    \end{pfsteps*}
    \pfref{satormay2} contradicts \pfref{notsatormay2}.

  \end{byCases}

\restorelocalsteps{lem:or-nn-satisfy-1}
\item[\text{(\ref{rule:CSMSMay})}]
  \begin{pfsteps*}
  \item $\cmaysatisfy{e}{\cor{\hxi_1}{\hxi_2}}$ \BY{assumption} \pflabel{maysatisfy1+2}
  \end{pfsteps*}
  By rule induction over Rules (\ref{rules:MaySatisfy}) on \pfref{maysatisfy1+2} and only two of them apply.
  \begin{byCases}

  \savelocalsteps{lem:or-nn-satisfy-2}
  \item[\text{(\ref{rule:CMSOr1})}]
    \begin{pfsteps*}
    \item $\cmaysatisfy{e}{\hxi_1}$ \BY{assumption} \pflabel{maysatisfy1}
    \item $\csatisfyormay{e}{\hxi_1}$ \BY{Rule (\ref{rule:CSMSMay}) on \pfref{maysatisfy1}} \pflabel{satormay1'}
    \end{pfsteps*}
    \pfref{satormay1'} contradicts \pfref{notsatormay1}.

  \restorelocalsteps{lem:or-nn-satisfy-2}
  \item[\text{(\ref{rule:CMSOr2})}]
    \begin{pfsteps*}
    \item $\cmaysatisfy{e}{\hxi_2}$ \BY{assumption} \pflabel{maysatisfy2}
    \item $\csatisfyormay{e}{\hxi_2}$ \BY{Rule (\ref{rule:CSMSMay}) on \pfref{maysatisfy2}} \pflabel{satormay2'}
    \end{pfsteps*}
    \pfref{satormay2'} contradicts \pfref{notsatormay2}.

  \end{byCases}
\end{byCases}
The conclusion holds as follows:
\begin{enumerate}
  \item $\cnotsatisfyormay{e}{\cor{\hxi_1}{\hxi_2}}$
\end{enumerate}
\resetpfcounter
\end{proof}

\begin{lemma}
  \label{lem:satisfy-substraction}
  If $\csatisfyormay{e}{\cor{\hxi_1}{\hxi_2}}$ and $\cnotsatisfyormay{e}{\hxi_1}$ then $\csatisfyormay{e}{\hxi_2}$
\end{lemma}
\begin{proof}
  \begin{pfsteps*}
  \item $\csatisfyormay{e}{\cor{\hxi_1}{\hxi_2}}$ \BY{assumption} \pflabel{satormay1+2}
  \item $\cnotsatisfyormay{e}{\hxi_1}$ \BY{assumption} \pflabel{notsatormay1}
  \end{pfsteps*}
  By rule induction over Rules (\ref{rules:satormay}) on \pfref{satormay1+2}.
  \begin{byCases}

  \savelocalsteps{lem:satisfy-substraction-1}
  \item[\text{(\ref{rule:CSMSSat})}]
    \begin{pfsteps*}
    \item $\csatisfy{e}{\cor{\hxi_1}{\hxi_2}}$ \BY{assumption} \pflabel{satisfy1+2}
    \end{pfsteps*}
    By rule induction over Rules (\ref{rules:Satisfy}) on \pfref{satisfy1+2} and only two of them apply.
    \begin{byCases}

    \savelocalsteps{lem:satisfy-substraction-2}
    \item[\text{(\ref{rule:CSOr1})}]
      \begin{pfsteps*}
      \item $\csatisfy{e}{\hxi_1}$ \BY{assumption} \pflabel{satisfy1}
      \item $\csatisfyormay{e}{\hxi_1}$ \BY{Rule (\ref{rule:CSMSSat}) on \pfref{satisfy1}} \pflabel{[sat]satormay1}
      \end{pfsteps*}
      \pfref{[sat]satormay1} contradicts \pfref{notsatormay1}.

    \restorelocalsteps{lem:satisfy-substraction-2}
    \item[\text{(\ref{rule:CSOr2})}]
      \begin{pfsteps*}
      \item $\csatisfy{e}{\hxi_2}$ \BY{assumption} \pflabel{satisfy2}
      \item $\csatisfyormay{e}{\hxi_2}$ \BY{Rule (\ref{rule:CSMSSat}) on \pfref{satisfy2}}
      \end{pfsteps*}
    \end{byCases}

  \restorelocalsteps{lem:satisfy-substraction-1}
  \item[\text{(\ref{rule:CSMSMay})}]
    \begin{pfsteps*}
    \item $\cmaysatisfy{e}{\cor{\hxi_1}{\hxi_2}}$ \BY{assumption} \pflabel{maysatisfy1+2}
    \end{pfsteps*}
    By rule induction over Rules (\ref{rules:MaySatisfy}) on \pfref{maysatisfy1+2} and only two of them apply.
    \begin{byCases}

    \savelocalsteps{lem:satisfy-substraction-2}
    \item[\text{(\ref{rule:CMSOr1})}]
      \begin{pfsteps*}
      \item $\cmaysatisfy{e}{\hxi_1}$ \BY{assumption} \pflabel{maysatisfy1}
      \item $\csatisfyormay{e}{\hxi_1}$ \BY{Rule (\ref{rule:CSMSMay}) on \pfref{maysatisfy1}} \pflabel{[may]satormay1}
      \end{pfsteps*}
      \pfref{[may]satormay1} contradicts \pfref{notsatormay1}.

    \restorelocalsteps{lem:satisfy-substraction-2}
    \item[\text{(\ref{rule:CMSOr2})}]
      \begin{pfsteps*}
      \item $\cmaysatisfy{e}{\hxi_2}$ \BY{assumption} \pflabel{maysatisfy2}
      \item $\csatisfyormay{e}{\hxi_2}$ \BY{Rule (\ref{rule:CSMSMay}) on \pfref{maysatisfy2}}
      \end{pfsteps*}
    \end{byCases}
    \resetpfcounter
  \end{byCases}
\end{proof}

\begin{lemma}
  \label{lem:satormay-and}
  If $\csatisfyormay{e}{\hxi_1}$ and $\csatisfyormay{e}{\hxi_2}$ then $\csatisfyormay{e}{\cand{\hxi_1}{\hxi_2}}$
\end{lemma}

\begin{lemma}
  \label{lem:satormay-or}
  If $\csatisfyormay{e}{\hxi_1}$ then $\csatisfyormay{e}{\cor{\hxi_1}{\hxi_2}}$ and
$\csatisfyormay{e}{\cor{\hxi_2}{\hxi_1}}$
\end{lemma}
\begin{proof}
\begin{pfsteps*}
\item $\csatisfyormay{e}{\hxi_1}$ \BY{assumption} \pflabel{satormay1},
\end{pfsteps*}
By rule induction over Rules (\ref{rules:satormay}) on \pfref{satormay1},
\savelocalsteps{lem:satormay-or-1}
\begin{byCases}

\item[\text{(\ref{rule:CSMSSat})}]
  \begin{pfsteps*}
  \item $\csatisfy{e}{\hxi_1}$ \BY{assumption} \pflabel{[sat]satisfy1}
  \item $\csatisfy{e}{\cor{\hxi_1}{\hxi_2}}$ \BY{Rule (\ref{rule:CSOr1}) on \pfref{[sat]satisfy1}} \pflabel{[sat]satisfyOr12}
  \item $\csatisfy{e}{\cor{\hxi_2}{\hxi_1}}$ \BY{Rule (\ref{rule:CSOr2}) on \pfref{[sat]satisfy1}} \pflabel{[sat]satisfyOr21}
  \item $\csatisfyormay{e}{\cor{\hxi_1}{\hxi_2}}$ \BY{Rule (\ref{rule:CSMSSat}) on \pfref{[sat]satisfyOr12}}
  \item $\csatisfyormay{e}{\cor{\hxi_2}{\hxi_1}}$ \BY{Rule (\ref{rule:CSMSSat}) on \pfref{[sat]satisfyOr21}}
  \end{pfsteps*}

\restorelocalsteps{lem:satormay-or-1}

\item[\text{(\ref{rule:CSMSMay})}]
  \begin{pfsteps*}
  \item $\cmaysatisfy{e}{\hxi_1}$ \BY{assumption} \pflabel{maysat1}
  \end{pfsteps*}
  By case analysis on the result of $\fsatisfy{e}{\hxi_2}$.
  \begin{byCases}

  \savelocalsteps{lem:satormay-or-2}
  \item[\true]
    \begin{pfsteps*}
    \item $\fsatisfy{e}{\hxi_2} = \true$ \BY{assumption} \pflabel{fsatisfy2}
    \item $\csatisfy{e}{\hxi_2}$ \BY{Lemma \ref{lem:sound-complete-satisfy} on \pfref{fsatisfy2}} \pflabel{satisfy2}
    \item $\csatisfy{e}{\cor{\hxi_1}{\hxi_2}}$ \BY{Rule (\ref{rule:CSOr2}) on \pfref{satisfy2}} \pflabel{satisfyOr12}
    \item $\csatisfy{e}{\cor{\hxi_2}{\hxi_1}}$ \BY{Rule (\ref{rule:CSOr1}) on \pfref{satisfy2}} \pflabel{satisfyOr21}
    \item $\csatisfyormay{e}{\cor{\hxi_1}{\hxi_2}}$ \BY{Rule (\ref{rule:CSMSSat}) on \pfref{satisfyOr12}}
    \item $\csatisfyormay{e}{\cor{\hxi_2}{\hxi_1}}$ \BY{Rule (\ref{rule:CSMSSat}) on \pfref{satisfyOr21}}
    \end{pfsteps*}

  \restorelocalsteps{lem:satormay-or-2}
  \item[\false]
    \begin{pfsteps*}
    \item $\fsatisfy{e}{\hxi_2} = \false$ \BY{assumption} \pflabel{fnotsatisfy2}
    \item $\cnotsatisfy{e}{\hxi_2}$ \BY{Lemma \ref{lem:sound-complete-satisfy} on \pfref{fnotsatisfy2}} \pflabel{notSatisfy2}
    \item $\cmaysatisfy{e}{\cor{\hxi_1}{\hxi_2}}$ \BY{Rule (\ref{rule:CMSOr1}) on \pfref{maysat1} and \pfref{notSatisfy2}} \pflabel{maysatOr}
    \item $\csatisfyormay{e}{\cor{\hxi_1}{\hxi_2}}$ \BY{Rule (\ref{rule:CSMSMay}) on \pfref{maysatOr}}
    \end{pfsteps*}
  \end{byCases}

\end{byCases}
\resetpfcounter
\end{proof}

\begin{lemma}
  \label{lem:satormay-inl}
  If $\csatisfyormay{e_1}{\hxi_1}$ then $\csatisfyormay{\hinl{\tau_2}{e_1}}{\cinl{\hxi_1}}$
\end{lemma}
\begin{proof}
  \begin{pfsteps*}
  \item $\csatisfyormay{e_1}{\hxi_1}$ \BY{assumption} \pflabel{satormay1}
  \end{pfsteps*}
  By rule induction over Rules (\ref{rules:satormay}) on \pfref{satormay1}.
  \begin{byCases}

  \savelocalsteps{lem:satormay-inl-1}
  \item[\text{(\ref{rule:CSMSSat})}]
    \begin{pfsteps*}
    \item $\csatisfy{e_1}{\hxi_1}$ \BY{assumption} \pflabel{satisfy1}
    \item $\csatisfy{\hinl{\tau_2}{e_1}}{\cinl{\hxi_1}}$ \BY{Rule (\ref{rule:CSInl}) on \pfref{satisfy1}} \pflabel{satisfyInl}
    \item $\csatisfyormay{\hinl{\tau_2}{e_1}}{\cinl{\hxi_1}}$ \BY{Rule (\ref{rule:CSMSSat}) on \pfref{satisfyInl}}
    \end{pfsteps*}

  \restorelocalsteps{lem:satormay-inl-1}
  \item[\text{(\ref{rule:CSMSMay})}]
    \begin{pfsteps*}
    \item $\cmaysatisfy{e_1}{\hxi_1}$ \BY{assumption} \pflabel{maysat1}
    \item $\cmaysatisfy{\hinl{\tau_2}{e_1}}{\cinl{\hxi_1}}$ \BY{Rule (\ref{rule:CMSInl}) on \pfref{maysat1}} \pflabel{maysatInl}
    \item $\csatisfyormay{\hinl{\tau_2}{e_1}}{\cinl{\hxi_1}}$ \BY{Rule (\ref{rule:CSMSMay}) on \pfref{maysatInl}}
    \end{pfsteps*}
  \end{byCases}
  \resetpfcounter
\end{proof}

\begin{lemma}
  \label{lem:satormay-inr}
  If $\csatisfyormay{e_2}{\hxi_2}$ then $\csatisfyormay{\hinr{\tau_1}{e_2}}{\cinr{\hxi_2}}$
\end{lemma}
\begin{proof}
  \begin{pfsteps*}
  \item $\csatisfyormay{e_2}{\hxi_2}$ \BY{assumption} \pflabel{satormay2}
  \end{pfsteps*}
  By rule induction over Rules (\ref{rules:satormay}) on \pfref{satormay2}.
  \begin{byCases}

  \savelocalsteps{lem:satormay-inr-1}
  \item[\text{(\ref{rule:CSMSSat})}]
    \begin{pfsteps*}
    \item $\csatisfy{e_2}{\hxi_2}$ \BY{assumption} \pflabel{satisfy2}
    \item $\csatisfy{\hinr{\tau_1}{e_2}}{\cinr{\hxi_2}}$ \BY{Rule (\ref{rule:CSInr}) on \pfref{satisfy2}} \pflabel{satisfyInr}
    \item $\csatisfyormay{\hinr{\tau_1}{e_2}}{\cinr{\hxi_2}}$ \BY{Rule (\ref{rule:CSMSSat}) on \pfref{satisfyInr}}
    \end{pfsteps*}

  \restorelocalsteps{lem:satormay-inr-1}
  \item[\text{(\ref{rule:CSMSMay})}]
    \begin{pfsteps*}
    \item $\cmaysatisfy{e_2}{\hxi_2}$ \BY{assumption} \pflabel{maysat2}
    \item $\cmaysatisfy{\hinl{\tau_1}{e_2}}{\cinr{\hxi_2}}$ \BY{Rule (\ref{rule:CMSInr}) on \pfref{maysat2}} \pflabel{maysatInr}
    \item $\csatisfyormay{\hinl{\tau_1}{e_2}}{\cinr{\hxi_2}}$ \BY{Rule (\ref{rule:CSMSMay}) on \pfref{maysatInr}}
    \end{pfsteps*}
  \end{byCases}
  \resetpfcounter
\end{proof}

\begin{lemma}
  \label{lem:satormay-pair}
  If $\csatisfyormay{e_1}{\hxi_1}$ and $\csatisfyormay{e_2}{\hxi_2}$ then $\csatisfyormay{\hpair{e_1}{e_2}}{\cpair{\hxi_1}{\hxi_2}}$
\end{lemma}

\begin{lemma}[Soundness and Completeness of Refutable Constraints]
  \label{lem:sound-complete-xi-refutable}
  $\refutable{\hxi}$ iff $\frefutable{\hxi} = \true$.
\end{lemma}

\begin{lemma}
\label{lem:no-and-refutable}
There does not exist such a constraint $\cand{\hxi_1}{\hxi_2}$ such that $\refutable{\cand{\hxi_1}{\hxi_2}}$.
\end{lemma}
\begin{proof}
By rule induction over Rules (\ref{rules:xi-refutable}), we notice that $\refutable{\cand{\hxi_1}{\hxi_2}}$ is in syntactic contradiction with all the cases, hence not derivable.
\end{proof}

\begin{lemma}
\label{lem:no-or-refutable}
There does not exist such a constraint $\cor{\hxi_1}{\hxi_2}$ such that $\refutable{\cor{\hxi_1}{\hxi_2}}$.
\end{lemma}
\begin{proof}
By rule induction over Rules (\ref{rules:xi-refutable}), we notice that $\refutable{\cor{\hxi_1}{\hxi_2}}$ is in syntactic contradiction with all the cases, hence not derivable.
\end{proof}

\begin{lemma}
\label{lem:satisfy-not-refutable}
If $\notIntro{e}$ and $\csatisfy{e}{\hxi}$ then $\cancel{\refutable{\hxi}}$.
\end{lemma}
\begin{proof}
\begin{pfsteps*}
\item $\notIntro{e}$ \BY{assumption} \pflabel{e-notintro}
\item $\csatisfy{e}{\hxi}$ \BY{assumption} \pflabel{satisfy}
\end{pfsteps*}
By rule induction over \rulesref{rules:Satisfy} on \pfref{satisfy}.
\begin{byCases}
\savelocalsteps{0}
\item[\text{(\ref{rule:CSTruth})}]
    \begin{pfsteps*}
    \item $\hxi=\ctruth$ \BY{assumption}
    \end{pfsteps*}
    Assume $\refutable{\ctruth}$. By rule induction over \rulesref{rules:xi-refutable}, no case applies due to syntactic contradiction.\\
    Therefore, $\cancel{\refutable{\ctruth}}$.
\restorelocalsteps{0}
\item[\text{(\ref{rule:CSOr1}),(\ref{rule:CSOr2})}]
    \begin{pfsteps*}
    \item $\hxi=\cor{\hxi_1}{\hxi_2}$ \BY{assumption}
    \item $\cancel{\refutable{\cor{\hxi_1}{\hxi_2}}}$ \BY{\autoref{lem:no-or-refutable}}
    \end{pfsteps*}
\restorelocalsteps{0}
\item[\text{(\ref{rule:CSAnd})}]
    \begin{pfsteps*}
    \item $\hxi=\cand{\hxi_1}{\hxi_2}$ \BY{assumption}
    \item $\cancel{\refutable{\cand{\hxi_1}{\hxi_2}}}$ \BY{\autoref{lem:no-and-refutable}}
    \end{pfsteps*}
\restorelocalsteps{0}
\item[\text{(\ref{rule:CSNotIntroPair})}]
    \begin{pfsteps*}
    \item $\hxi=\cpair{\hxi_1}{\hxi_2}$ \BY{assumption}
    \item $\csatisfy{\hprl{e}}{\hxi_1}$ \BY{assumption} \pflabel{prl-satisfy}
    \item $\csatisfy{\hprr{e}}{\hxi_2}$ \BY{assumption} \pflabel{prr-satisfy}
    \item $\notIntro{\hprl{e}}$ \BY{\ruleref{rule:NVPrl}} \pflabel{prl-notintro}
    \item $\notIntro{\hprr{e}}$ \BY{\ruleref{rule:NVPrr}} \pflabel{prr-notintro}
    \item $\cancel{\refutable{\hxi_1}}$ \BY{IH on \pfref{prl-notintro} and \pfref{prl-satisfy}} \pflabel{not-rft1}
    \item $\cancel{\refutable{\hxi_2}}$ \BY{IH on \pfref{prr-notintro} and \pfref{prr-satisfy}} \pflabel{not-rft2}
    \end{pfsteps*}
    Assume $\refutable{\cpair{\hxi_1}{\hxi_2}}$. By rule induction over \rulesref{rules:xi-refutable} on it, only two cases apply.
    \begin{byCases}
    \savelocalsteps{1}
    \item[\text{(\ref{rule:RXPairL})}]
        \begin{pfsteps*}
        \item $\refutable{\hxi_1}$ \BY{assumption}
        \end{pfsteps*}
        Contradicts \pfref{not-rft1}.
    \restorelocalsteps{1}
    \item[\text{(\ref{rule:RXPairR})}]
        \begin{pfsteps*}
        \item $\refutable{\hxi_2}$ \BY{assumption}
        \end{pfsteps*}
        Contradicts \pfref{not-rft2}.
    \end{byCases}
    Therefore, $\cancel{\refutable{\cpair{\hxi_1}{\hxi_2}}}$.
\restorelocalsteps{0}
\item
    \begin{pfsteps*}
    \item $e=\hnum{n},\hinl{\tau_2}{e_1},\hinr{\tau_1}{e_2},\hpair{e_1}{e_2}$ \BY{assumption}
    \end{pfsteps*}
    By rule induction over \rulesref{rules:notintro} on \pfref{e-notintro}, no case applies due to syntactic contradiction.
\end{byCases}
\resetpfcounter
\end{proof}

\begin{lemma}[Soundness and Completeness of Satisfaction Judgment]
  \label{lem:sound-complete-satisfy}
  $\csatisfy{e}{\hxi}$ iff $\fsatisfy{e}{\hxi} = \true$.
\end{lemma}
\begin{proof}
  We prove soundness and completeness separately.
  \begin{enumerate}
    \item Soundness:
      \begin{pfsteps*}
      \item $\csatisfy{e}{\hxi}$ \BY{assumption} \pflabel{satisfy}
      \end{pfsteps*}
      By rule induction over Rules (\ref{rules:Satisfy}) on \pfref{satisfy}.
      \begin{byCases}

      \savelocalsteps{lem:sound-complete-satisfy-1}
      \item[\text{(\ref{rule:CSTruth})}]
        \begin{pfsteps*}
        \item $\hxi = \ctruth$ \BY{assumption}
        \item $\fsatisfy{e}{\ctruth} = \true$ \BY{Definition \ref{defn:satisfy-truth}}
        \end{pfsteps*}

      \restorelocalsteps{lem:sound-complete-satisfy-1}
      \item[\text{(\ref{rule:CSNum})}]
        \begin{pfsteps*}
        \item $e = \hnum{n}$ \BY{assumption}
        \item $\hxi = \cnum{n}$ \BY{assumption}
        \item $\fsatisfy{\hnum{n}}{\cnum{n}} = (n = n) = \true$ \BY{Definition \ref{defn:num-satisfy-num}}
        \end{pfsteps*}

      \restorelocalsteps{lem:sound-complete-satisfy-1}
      \item[\text{(\ref{rule:CSNotNum})}]
        \begin{pfsteps*}
        \item $e = \hnum{n_1}$ \BY{assumption}
        \item $\hxi = \cnotnum{n_2}$ \BY{assumption}
        \item $n_1 \neq n_2$ \BY{assumption} \pflabel{numnotequal}
        \item $\fsatisfy{\hnum{n_1}}{\cnotnum{n_2}} = (n_1 \neq n_2) = \true$ \BY{Definition \ref{defn:num-satisfy-notnum} on \pfref{numnotequal}}
        \end{pfsteps*}

      \restorelocalsteps{lem:sound-complete-satisfy-1}
      \item[\text{(\ref{rule:CSAnd})}]
        \begin{pfsteps*}
        \item $\hxi = \cand{\hxi_1}{\hxi_2}$ \BY{assumption}
        \item $\csatisfy{e}{\hxi_1}$ \BY{assumption} \pflabel{[and]csatisfy1}
        \item $\csatisfy{e}{\hxi_2}$ \BY{assumption} \pflabel{[and]csatisfy2}
        \item $\fsatisfy{e}{\hxi_1} = \true$ \BY{IH on \pfref{[and]csatisfy1}} \pflabel{[and]fsatisfy1}
        \item $\fsatisfy{e}{\hxi_2} = \true$ \BY{IH on \pfref{[and]csatisfy2}} \pflabel{[and]fsatisfy2}
        \item $\fsatisfy{e}{\cand{\hxi_1}{\hxi_2}} = \fsatisfy{e}{\hxi_1} \text{ and } \fsatisfy{e}{\hxi_2} = \true$ \BY{Definition \ref{defn:satisfy-and} on \pfref{[and]fsatisfy1} and \pfref{[and]fsatisfy2}}
        \end{pfsteps*}

      \restorelocalsteps{lem:sound-complete-satisfy-1}
      \item[\text{(\ref{rule:CSOr1})}]
        \begin{pfsteps*}
        \item $\hxi = \cor{\hxi_1}{\hxi_2}$ \BY{assumption}
        \item $\csatisfy{e}{\hxi_1}$ \BY{assumption} \pflabel{[or1]csatisfy1}
        \item $\fsatisfy{e}{\hxi_1} = \true$ \BY{IH on \pfref{[or1]csatisfy1}} \pflabel{[or1]fsatisfy1}
        \item $\fsatisfy{e}{\cor{\hxi_1}{\hxi_2}} = \fsatisfy{e}{\hxi_1} \text{ or } \fsatisfy{e}{\hxi_2} = \true$ \BY{Definition \ref{defn:satisfy-or} on \pfref{[or1]fsatisfy1}}
        \end{pfsteps*}

      \restorelocalsteps{lem:sound-complete-satisfy-1}
      \item[\text{(\ref{rule:CSOr2})}]
        \begin{pfsteps*}
        \item $\hxi = \cor{\hxi_1}{\hxi_2}$ \BY{assumption}
        \item $\csatisfy{e}{\hxi_2}$ \BY{assumption} \pflabel{[or2]csatisfy2}
        \item $\fsatisfy{e}{\hxi_2} = \true$ \BY{IH on \pfref{[or2]csatisfy2}} \pflabel{[or2]fsatisfy2}
        \item $\fsatisfy{e}{\cor{\hxi_1}{\hxi_2}} = \fsatisfy{e}{\hxi_1} \text{ or } \fsatisfy{e}{\hxi_2} = \true$ \BY{Definition \ref{defn:satisfy-or} on \pfref{[or2]fsatisfy2}}
        \end{pfsteps*}

      \restorelocalsteps{lem:sound-complete-satisfy-1}
      \item[\text{(\ref{rule:CSInl})}]
        \begin{pfsteps*}
        \item $e = \hinl{\tau_2}{e_1}$ \BY{assumption}
        \item $\hxi = \cinl{\hxi_1}$ \BY{assumption}
        \item $\csatisfy{e_1}{\hxi_1}$ \BY{assumption} \pflabel{[inl]csatisfy1}
        \item $\fsatisfy{e_1}{\hxi_1} = \true$ \BY{IH on \pfref{[inl]csatisfy1}} \pflabel{[inl]fsatisfy1}
        \item $\fsatisfy{\hinl{\tau_2}{e_1}}{\cinl{\hxi_1}} = \fsatisfy{e_1}{\hxi_1} = \true$ \BY{Definition \ref{defn:inl-satisfy-inl} on \pfref{[inl]fsatisfy1}}
        \end{pfsteps*}

      \restorelocalsteps{lem:sound-complete-satisfy-1}
      \item[\text{(\ref{rule:CSInr})}]
        \begin{pfsteps*}
        \item $e = \hinr{\tau_1}{e_2}$ \BY{assumption}
        \item $\hxi = \cinl{\hxi_2}$ \BY{assumption}
        \item $\csatisfy{e_2}{\hxi_2}$ \BY{assumption} \pflabel{[inr]csatisfy2}
        \item $\fsatisfy{e_2}{\hxi_2} = \true$ \BY{IH on \pfref{[inr]csatisfy2}} \pflabel{[inr]fsatisfy2}
        \item $\fsatisfy{\hinr{\tau_1}{e_2}}{\cinr{\hxi_2}} = \fsatisfy{e_2}{\hxi_2} = \true$ \BY{Definition \ref{defn:inr-satisfy-inr} on \pfref{[inr]fsatisfy2}}
        \end{pfsteps*}

      \restorelocalsteps{lem:sound-complete-satisfy-1}
      \item[\text{(\ref{rule:CSPair})}]
        \begin{pfsteps*}
        \item $e = \hpair{e_1}{e_2}$ \BY{assumption}
        \item $\hxi = \cpair{\hxi_1}{\hxi_2}$ \BY{assumption}
        \item $\csatisfy{e_1}{\hxi_1}$ \BY{assumption} \pflabel{[pair]csatisfy1}
        \item $\csatisfy{e_2}{\hxi_2}$ \BY{assumption} \pflabel{[pair]csatisfy2}
        \item $\fsatisfy{e_1}{\hxi_1} = \true$ \BY{IH on \pfref{[pair]csatisfy1}} \pflabel{[pair]fsatisfy1}
        \item $\fsatisfy{e_2}{\hxi_2} = \true$ \BY{IH on \pfref{[pair]csatisfy2}} \pflabel{[pair]fsatisfy2}
        \item $\fsatisfy{\hpair{e_1}{e_2}}{\cpair{\hxi_1}{\hxi_2}} = \fsatisfy{e_1}{\hxi_1} \text{ and } \fsatisfy{e_2}{\hxi_2} = \true$ \BY{Definition \ref{defn:pair-satisfy-pair} on \pfref{[pair]fsatisfy1} and \pfref{[pair]fsatisfy2}}
        \end{pfsteps*}
      
      \restorelocalsteps{lem:sound-complete-satisfy-1}
      \item[\text{(\ref{rule:CSNotIntroPair})}]
        \begin{pfsteps*}
        \item $\hxi = \cpair{\hxi_1}{\hxi_2}$ \BY{assumption}
        \item $\notIntro{e}$ \BY{assumption} \pflabel{[notintro]notintro}
        \item $\csatisfy{\hprl{e}}{\hxi_1}$ \BY{assumption} \pflabel{[notintro]csatisfy1}
        \item $\csatisfy{\hprr{e}}{\hxi_2}$ \BY{assumption} \pflabel{[notintro]csatisfy2}
        \item $\fsatisfy{\hprl{e}}{\hxi_1} = \true$ \BY{IH on \pfref{[notintro]csatisfy1}} \pflabel{[notintro]fsatisfy1}
        \item $\fsatisfy{\hprr{e}}{\hxi_2} = \true$ \BY{IH on \pfref{[notintro]csatisfy2}} \pflabel{[notintro]fsatisfy2}
        \end{pfsteps*}
        By rule induction over \rulesref{rules:notintro} on \pfref{[notintro]notintro}.
        \begin{byCases}
        \item
          \begin{pfsteps*}
          \item $e=\hehole{u},\hhole{e_0}{u},\hap{e_1}{e_2},\hprl{e_0},\hprr{e_0},\hmatch{e_0}{\zrules}$ \BY{assumption}
          \item $\fsatisfy{e}{\cpair{\hxi_1}{\hxi_2}} = \fsatisfy{\hprl{e}}{\hxi_1} \text{ and } \fsatisfy{\hprr{e}}{\hxi_2} = \true$ \BY{Definition \ref{defn:satisfy} on \pfref{[notintro]fsatisfy1} and \pfref{[notintro]fsatisfy2}}
          \end{pfsteps*}
        \end{byCases}
      \end{byCases}

    \resetpfcounter
    
    \item Completeness:
      \begin{pfsteps*}
      \item $\fsatisfy{e}{\hxi} = \true$ \BY{assumption} \pflabel{fsatisfy}
      \end{pfsteps*}
      By structural induction on $\hxi$.
      \begin{byCases}

      \savelocalsteps{lem:sound-complete-satisfy-1}
      \item[\hxi=\ctruth]
        \begin{pfsteps*}
        \item $\csatisfy{e}{\ctruth}$ \BY{Rule (\ref{rule:CSTruth})}
        \end{pfsteps*}

      \restorelocalsteps{lem:sound-complete-satisfy-1}
      \item[\hxi=\cfalsity, \cunknown]
        \begin{pfsteps*}
        \item $\fsatisfy{e}{\hxi} = \false$ \BY{Definition \ref{defn:not-satisfy}} \pflabel{fsatisfy-unknown}
        \end{pfsteps*}
        \pfref{fsatisfy-unknown} contradicts \pfref{fsatisfy} and thus vacuously true.

      \restorelocalsteps{lem:sound-complete-satisfy-1}
      \item[\hxi=\cnum{n}]\ \\
        By structural induction on $e$.
        \begin{byCases}

          \savelocalsteps{lem:sound-complete-satisfy-2}
          \item[e=\hnum{n'}]
            \begin{pfsteps*}
              \item $n' = n$ \BY{Definition \ref{defn:num-satisfy-num} on \pfref{fsatisfy}} \pflabel{numequal}
              \item $\csatisfy{\hnum{n'}}{\cnum{n}}$ \BY{Rule (\ref{rule:CSNum}) on \pfref{numequal}}
            \end{pfsteps*}

          \restorelocalsteps{lem:sound-complete-satisfy-2}
          \item
            \begin{pfsteps*}
            \item $\fsatisfy{e}{\cnum{n}} = \false$ \BY{Definition \ref{defn:not-satisfy}} \pflabel{[cnum]fsatisfy}
            \end{pfsteps*}
            \pfref{[cnum]fsatisfy} contradicts \pfref{fsatisfy} and thus vacuously true. 
        \end{byCases}
        
      \restorelocalsteps{lem:sound-complete-satisfy-1}
      \item[\hxi=\cnotnum{n}]\ \\
        By structural induction on $e$.
        \begin{byCases}

          \savelocalsteps{lem:sound-complete-satisfy-2}
          \item[e=\hnum{n'}]
            \begin{pfsteps*}
            \item $n' \neq n$ \BY{Definition \ref{defn:num-satisfy-notnum} on \pfref{fsatisfy}} \pflabel{numnotequal}
            \item $\csatisfy{\hnum{n'}}{\cnotnum{n}}$ \BY{Rule (\ref{rule:CSNotNum}) on \pfref{numnotequal}}
            \end{pfsteps*}
            
          \restorelocalsteps{lem:sound-complete-satisfy-2}
          \item
            \begin{pfsteps*}
            \item $\fsatisfy{e}{\cnotnum{n}} = \false$ \BY{Definition \ref{defn:not-satisfy}} \pflabel{[cnotnum]fsatisfy}
            \end{pfsteps*}
            \pfref{[cnotnum]fsatisfy} contradicts \pfref{fsatisfy} and thus vacuously true. 
        \end{byCases}

      \restorelocalsteps{lem:sound-complete-satisfy-1}
      \item[\hxi=\cand{\hxi_1}{\hxi_2}]
        \begin{pfsteps*}
        \item $\fsatisfy{e}{\hxi_1} = \true$ \BY{Definition \ref{defn:satisfy-and} on \pfref{fsatisfy}} \pflabel{[and]fsatisfy1}
        \item $\fsatisfy{e}{\hxi_2} = \true$ \BY{Definition \ref{defn:satisfy-and} on \pfref{fsatisfy}} \pflabel{[and]fsatisfy2}
        \item $\csatisfy{e}{\hxi_1}$ \BY{IH on \pfref{[and]fsatisfy1}} \pflabel{[and]csatisfy1}
        \item $\csatisfy{e}{\hxi_2}$ \BY{IH on \pfref{[and]fsatisfy2}} \pflabel{[and]csatisfy2}
        \item $\csatisfy{e}{\cand{\hxi_1}{\hxi_2}}$ \BY{Rule (\ref{rule:CSAnd}) on \pfref{[and]csatisfy1} and \pfref{[and]csatisfy2}}
        \end{pfsteps*}

      \restorelocalsteps{lem:sound-complete-satisfy-1}
      \item[\hxi=\cor{\hxi_1}{\hxi_2}]
        \begin{pfsteps*}
        \item $\fsatisfy{e}{\hxi_1} \text{ or } \fsatisfy{e}{\hxi_2} = \true$ \BY{Definition \ref{defn:satisfy-or} on \pfref{fsatisfy}} \pflabel{[or]fsatisfy}
        \end{pfsteps*}
        By case analysis on \pfref{[or]fsatisfy}.
        \begin{byCases}

          \savelocalsteps{lem:sound-complete-satisfy-2}
          \item[\fsatisfy{e}{\hxi_1}=\true]
          \begin{pfsteps*}
          \item $\fsatisfy{e}{\hxi_1} = \true$ \BY{assumption} \pflabel{[or]fsatisfy1}
          \item $\csatisfy{e}{\hxi_1}$ \BY{IH on \pfref{[or]fsatisfy1}} \pflabel{[or]csatisfy1}
          \item $\csatisfy{e}{\cor{\hxi_1}{\hxi_2}}$ \BY{Rule (\ref{rule:CSOr1}) on \pfref{[or]csatisfy1}}
          \end{pfsteps*}

          \restorelocalsteps{lem:sound-complete-satisfy-2}
          \item[\fsatisfy{e}{\hxi_2}=\true]
          \begin{pfsteps*}
          \item $\fsatisfy{e}{\hxi_2} = \true$ \BY{assumption} \pflabel{[or]fsatisfy2}
          \item $\csatisfy{e}{\hxi_2}$ \BY{IH on \pfref{[or]fsatisfy2}} \pflabel{[or]csatisfy2}
          \item $\csatisfy{e}{\cor{\hxi_1}{\hxi_2}}$ \BY{Rule (\ref{rule:CSOr2}) on \pfref{[or]csatisfy2}}
          \end{pfsteps*}
        \end{byCases}

      \restorelocalsteps{lem:sound-complete-satisfy-1}
      \item[\hxi=\cinl{\hxi_1}]\ \\
        By structural induction on $e$.
        \begin{byCases}

          \savelocalsteps{lem:sound-complete-satisfy-2}
          \item[e=\hinl{\tau_2}{e_1}]
            \begin{pfsteps*}
              \item $\fsatisfy{e_1}{\hxi_1}=\true$ \BY{Definition \ref{defn:inl-satisfy-inl} on \pfref{fsatisfy}} \pflabel{[inl]fsatisfy1}
              \item $\csatisfy{e_1}{\hxi_1}$ \BY{IH on \pfref{[inl]fsatisfy1}} \pflabel{[inl]csatisfy1}
              \item $\csatisfy{\hinl{\tau_2}{e_1}}{\cinl{\hxi_1}}$ \BY{Rule (\ref{rule:CSInl}) on \pfref{[inl]csatisfy1}}
            \end{pfsteps*}

            \restorelocalsteps{lem:sound-complete-satisfy-2}
          \item
            \begin{pfsteps*}
            \item $\fsatisfy{e}{\cinl{\hxi_1}} = \false$ \BY{Definition \ref{defn:not-satisfy}} \pflabel{[inl]fnotsatisfy}
            \end{pfsteps*}
            \pfref{[inl]fnotsatisfy} contradicts \pfref{fsatisfy} and thus vacuously true. 
        \end{byCases}

      \restorelocalsteps{lem:sound-complete-satisfy-1}
      \item[\hxi=\cinr{\hxi_2}]\ \\
        By structural induction on $e$.
        \begin{byCases}

          \savelocalsteps{lem:sound-complete-satisfy-2}
          \item[e=\hinr{\tau_1}{e_2}]
            \begin{pfsteps*}
              \item $\fsatisfy{e_2}{\hxi_2}=\true$ \BY{Definition \ref{defn:inr-satisfy-inr} on \pfref{fsatisfy}} \pflabel{[inr]fsatisfy2}
              \item $\csatisfy{e_2}{\hxi_2}$ \BY{IH on \pfref{[inr]fsatisfy2}} \pflabel{[inr]csatisfy2}
              \item $\csatisfy{\hinr{\tau_1}{e_2}}{\cinr{\hxi_2}}$ \BY{Rule (\ref{rule:CSInr}) on \pfref{[inr]csatisfy2}}
            \end{pfsteps*}

          \restorelocalsteps{lem:sound-complete-satisfy-2}
          \item
            \begin{pfsteps*}
            \item $\fsatisfy{e}{\cinr{\hxi_2}} = \false$ \BY{Definition \ref{defn:not-satisfy}} \pflabel{[inr]fnotsatisfy}
            \end{pfsteps*}
            \pfref{[inr]fnotsatisfy} contradicts \pfref{fsatisfy} and thus vacuously true. 
        \end{byCases}

      \restorelocalsteps{lem:sound-complete-satisfy-1}
      \item[\hxi=\cpair{\hxi_1}{\hxi_2}]\ \\
        By structural induction on $e$.
        \begin{byCases}

          \savelocalsteps{lem:sound-complete-satisfy-2}
          \item[e=\hpair{e_1}{e_2}]
            \begin{pfsteps*}
            \item $\fsatisfy{e_1}{\hxi_1} = \true$ \BY{Definition \ref{defn:pair-satisfy-pair} on \pfref{fsatisfy}} \pflabel{[pair]fsatisfy1}
            \item $\fsatisfy{e_2}{\hxi_2} = \true$ \BY{Definition \ref{defn:pair-satisfy-pair} on \pfref{fsatisfy}} \pflabel{[pair]fsatisfy2}
            \item $\csatisfy{e_1}{\hxi_1}$ \BY{IH on \pfref{[pair]fsatisfy1}} \pflabel{[pair]csatisfy1}
            \item $\csatisfy{e_2}{\hxi_2}$ \BY{IH on \pfref{[pair]fsatisfy2}} \pflabel{[pair]csatisfy2}
            \item $\csatisfy{\hpair{e_1}{e_2}}{\cpair{\hxi_1}{\hxi_2}}$ \BY{Rule (\ref{rule:CSPair}) on \pfref{[pair]csatisfy1} and \pfref{[pair]csatisfy2}}
            \end{pfsteps*}
        
          \restorelocalsteps{lem:sound-complete-satisfy-2}
          \item[e=\hehole{u},\hhole{e_0}{u},\hap{e_1}{e_2},\hprl{e_0},\hprr{e_0},\hmatch{e_0}{\zrules}]
            \begin{pfsteps*}
            \item $\fsatisfy{\hprl{e}}{\hxi_1} = \true$ \BY{Definition \ref{defn:pair-satisfy-pair} on \pfref{fsatisfy}} \pflabel{[notintro]fsatisfy1}
            \item $\fsatisfy{\hprr{e}}{\hxi_2} = \true$ \BY{Definition \ref{defn:pair-satisfy-pair} on \pfref{fsatisfy}} \pflabel{[notintro]fsatisfy2}
            \item $\csatisfy{\hprl{e}}{\hxi_1}$ \BY{IH on \pfref{[notintro]fsatisfy1}} \pflabel{[notintro]csatisfy1}
            \item $\csatisfy{\hprr{e}}{\hxi_2}$ \BY{IH on \pfref{[notintro]fsatisfy2}} \pflabel{[notintro]csatisfy2}
            \item $\notIntro{e}$ \BY{each rule in \rulesref{rules:notintro}} \pflabel{[notintro]notintro}
            \item $\csatisfy{\hpair{e_1}{e_2}}{\cpair{\hxi_1}{\hxi_2}}$ \BY{Rule (\ref{rule:CSNotIntroPair}) on \pfref{[notintro]notintro} and \pfref{[notintro]csatisfy1} and \pfref{[notintro]csatisfy2}}
            \end{pfsteps*}
          \restorelocalsteps{lem:sound-complete-satisfy-2}
          \item
            \begin{pfsteps*}
            \item $\fsatisfy{e}{\cpair{\hxi_1}{\hxi_2}} = \false$ \BY{Definition \ref{defn:not-satisfy}} \pflabel{[pair]fnotsatisfy}
            \end{pfsteps*}
            \pfref{[pair]fnotsatisfy} contradicts \pfref{fsatisfy} and thus vacuously true. 
        \end{byCases}
      \end{byCases}
  \end{enumerate}
  \resetpfcounter
\end{proof}


\begin{lemma}
  \label{lem:not-satormay}
  $\cnotsatisfy{e}{\hxi}$ and $\cnotmaysatisfy{e}{\hxi}$ iff $\cnotsatisfyormay{e}{\hxi}$.
\end{lemma}
\begin{proof}
\begin{enumerate}
    \item Sufficiency:
    \begin{pfsteps*}
    \item $\cnotsatisfy{e}{\hxi}$ \BY{assumption} \pflabel{notsatisfy}
    \item $\cnotmaysatisfy{e}{\hxi}$ \BY{assumption} \pflabel{notmaysat}
    \end{pfsteps*}
    Assume $\csatisfyormay{e}{\hxi}$. By rule induction over Rules (\ref{rules:satormay}) on it.
    \begin{byCases}
    \savelocalsteps{0}
    \item[\text{(\ref{rule:CSMSMay})}]
        \begin{pfsteps*}
        \item $\csatisfy{e}{\hxi}$ \BY{assumption}
        \end{pfsteps*}
        Contradicts \pfref{notsatisfy}.
    \restorelocalsteps{0}
    \item[\text{(\ref{rule:CSMSSat})}]
        \begin{pfsteps*}
        \item $\cmaysatisfy{e}{\hxi}$ \BY{assumption}
        \end{pfsteps*}
        Contradicts \pfref{notmaysat}.
    \end{byCases}
    Therefore, $\csatisfyormay{e}{\hxi}$ is not derivable.
    \resetpfcounter
    \item Necessity:
    \begin{pfsteps*}
    \item $\cnotsatisfyormay{e}{\hxi}$ \BY{assumption} \pflabel{notsatormay}
    \end{pfsteps*}
    Assume $\csatisfy{e}{\hxi}$.
    \begin{pfsteps*}
    \item $\csatisfyormay{e}{\hxi}$ \BY{\ruleref{rule:CSMSSat} on assumption}
    \end{pfsteps*}
    Contradicts \pfref{notsatormay}. Therefore, $\cnotsatisfy{e}{\hxi}$.
    Assume $\cmaysatisfy{e}{\hxi}$.
    \begin{pfsteps*}
    \item $\csatisfyormay{e}{\hxi}$ \BY{\ruleref{rule:CSMSMay} on assumption}
    \end{pfsteps*}
    Contradicts \pfref{notsatormay}. Therefore, $\cnotmaysatisfy{e}{\hxi}$.
\end{enumerate}
\end{proof}

\begin{theorem}[Exclusiveness of Satisfaction Judgment]
  \label{thrm:exclusive-constraint-satisfaction}
  If $\ctyp{\hxi}{\tau}$ and $\hexptyp{\cdot}{\Delta}{e}{\tau}$ and $\isFinal{e}$ then exactly one of the following holds
  \begin{enumerate}
    \item $\csatisfy{e}{\hxi}$
    \item $\cmaysatisfy{e}{\hxi}$
    \item $\cnotsatisfyormay{e}{\hxi}$
  \end{enumerate}
\end{theorem}
\begin{proof}
\begin{pfsteps*}
\item $\ctyp{\hxi}{\tau}$ \BY{assumption} \pflabel{cTyp}
\item $\hexptyp{\cdot}{\Delta}{e}{\tau}$ \BY{assumption} \pflabel{eTyp}
\item $\isFinal{e}$ \BY{assumption} \pflabel{eFinal}
\end{pfsteps*}
By rule induction over Rules (\ref{rules:CTyp}) on \pfref{cTyp}, we would show one conclusion is derivable while the other two are not.
\begin{byCases}

\savelocalsteps{0}
\item[\text{(\ref{rule:CTTruth})}]
    \begin{pfsteps*}
    \item $\hxi=\ctruth$ \BY{assumption}
    \item $\csatisfy{e}{\ctruth}$ \BY{Rule (\ref{rule:CSTruth})} \pflabel{[truth]satisfy}
    \item $\cnotmaysatisfy{e}{\ctruth}$ \BY{\autoref{lem:no-e-may-satisfy-truth}}
    \item $\csatisfyormay{e}{\ctruth}$ \BY{\ruleref{rule:CSMSSat} on \pfref{[truth]satisfy}}
    \end{pfsteps*}
    
\restorelocalsteps{0}
\item[\text{(\ref{rule:CTFalsity})}]
    \begin{pfsteps*}
    \item $\hxi=\cfalsity$ \BY{assumption}
    \item $\cnotsatisfy{e}{\cfalsity}$ \BY{\autoref{lem:no-e-satisfy-falsity}} \pflabel{[falsity]notsatisfy}
    \item $\cnotmaysatisfy{e}{\cfalsity}$ \BY{\autoref{lem:no-e-may-satisfy-falsity}} \pflabel{[falsity]notmaysat}
    \item $\cnotsatisfyormay{e}{\cfalsity}$ \BY{\autoref{lem:not-satormay} on \pfref{[falsity]notsatisfy} and \pfref{[falsity]notmaysat}}
    \end{pfsteps*}
    
\restorelocalsteps{0}
\item[\text{(\ref{rule:CTUnknown})}]
    \begin{pfsteps*}
    \item $\hxi=\cunknown$ \BY{assumption}
    \item $\cnotsatisfy{e}{\cunknown}$ \BY{\autoref{lem:no-e-satisfy-unknown}}
    \item $\cmaysatisfy{e}{\cunknown}$ \BY{Rule (\ref{rule:CMSUnknown})} \pflabel{[unknown]maysat}
    \item $\csatisfyormay{e}{\cunknown}$ \BY{\ruleref{rule:CSMSMay} on \pfref{[unknown]maysat}} 
    \end{pfsteps*}
    
\restorelocalsteps{0}
\item[\text{(\ref{rule:CTNum})}]
    \begin{pfsteps*}
    \item $\hxi=\cnum{n_2}$ \BY{assumption}
    \item $\tau=\tnum$ \BY{assumption}
    \end{pfsteps*}
    By rule induction over Rules (\ref{rules:TExp}) on \pfref{eTyp}, the following cases apply.
    \begin{byCases}
    \savelocalsteps{1}
    \item[\text{(\ref{rule:TEHole}),(\ref{rule:THole}),(\ref{rule:TAp}),(\ref{rule:TPrl}),(\ref{rule:TPrr}),(\ref{rule:TMatchZPre}),(\ref{rule:TMatchNZPre})}]
        \begin{pfsteps*}
        \item $e=\hehole{u},\hhole{e_0}{u},\hap{e_1}{e_2},\hprl{e_0},\hprr{e_0},\hmatch{e_0}{\zrules}$ \BY{assumption}
        \item $\notIntro{e}$ \BY{Rule (\ref{rule:NVEHole}),(\ref{rule:NVHole}),(\ref{rule:NVAp}),(\ref{rule:NVMatch}),(\ref{rule:NVPrl}),(\ref{rule:NVPrr})} \pflabel{[num]notintro}
        \end{pfsteps*}
        Assume $\csatisfy{e}{\cnum{n_2}}$. By rule induction over Rules (\ref{rules:Satisfy}) on it, no case applies due to syntactic contradiction on $\hxi$.\\
        \begin{pfsteps*}
        \item $\cnotsatisfy{e}{\cnum{n_2}}$ \BY{contradiction}
        \item $\refutable{\cnum{n_2}}$ \BY{\ruleref{rule:RXNum}} \pflabel{[num]rft}
        \item $\cmaysatisfy{e}{\cnum{n_2}}$ \BY{Rule (\ref{rule:CMSNotIntro}) on \pfref{[num]notintro} and \pfref{[num]rft}} \pflabel{[num]maysat}
        \item $\csatisfyormay{e}{\cnum{n_2}}$ \BY{\ruleref{rule:CSMSMay} on \pfref{[num]maysat}} 
        \end{pfsteps*}
    \restorelocalsteps{1}
    \item[\text{(\ref{rule:TNum})}]
        \begin{pfsteps*}
        \item $e=\hnum{n_1}$ \BY{assumption}
        \end{pfsteps*}
        Assume $\cmaysatisfy{\hnum{n_1}}{\cnum{n_2}}$. By rule induction over Rules (\ref{rules:MaySatisfy}), only one case applies.
        \begin{byCases}
        \item[\text{(\ref{rule:CMSNotIntro})}]
            \begin{pfsteps*}
            \item $\notIntro{\hnum{n_1}}$ \BY{assumption}
            \end{pfsteps*}
            Contradicts \autoref{lem:no-num-notintro}.
        \end{byCases}
        \begin{pfsteps*}
        \item $\cnotmaysatisfy{\hnum{n_1}}{\cnum{n_2}}$ \BY{contradiction} \pflabel{[num]num-notmaysat}
        \end{pfsteps*}
        By case analysis on whether $n_1$ is equal to $n_2$.
        \begin{byCases}
        \savelocalsteps{2}
        \item[n_1=n_2]
            \begin{pfsteps*}
            \item $\fsatisfy{\hnum{n_1}}{\cnum{n_2}}=\true$ \BY{Definition \ref{defn:satisfy}} \pflabel{fsatisfy-num-num-true}
            \item $\csatisfy{\hnum{n_1}}{\cnum{n_2}}$ \BY{Lemma \ref{lem:sound-complete-satisfy} on \pfref{fsatisfy-num-num-true}} \pflabel{[num]num-satisfy}
            \item $\csatisfyormay{e}{\cnum{n_2}}$ \BY{\ruleref{rule:CSMSSat} on \pfref{[num]num-satisfy}} 
            \end{pfsteps*}
        \restorelocalsteps{2}
        \item[n_1\neq n_2]
            \begin{pfsteps*}
            \item $\fsatisfy{\hnum{n_1}}{\cnum{n_2}}=\false$ \BY{Definition \ref{defn:satisfy}} \pflabel{fsatisfy-num-num-false}
            \item $\cnotsatisfy{\hnum{n_1}}{\cnum{n_2}}$ \BY{Lemma \ref{lem:sound-complete-satisfy} on \pfref{fsatisfy-num-num-false}} \pflabel{[num]num-notsatisfy}
            \item $\cnotsatisfyormay{e}{\cnum{n_2}}$ \BY{\autoref{lem:not-satormay} on \pfref{[num]num-notmaysat} and \pfref{[num]num-notsatisfy}} 
            \end{pfsteps*}
        \end{byCases}
    \end{byCases}
 
\restorelocalsteps{0}
\item[\text{(\ref{rule:CTOr})}]
    \begin{pfsteps*}
    \item $\hxi=\cor{\hxi_1}{\hxi_2}$ \BY{assumption}
    \end{pfsteps*}
    By inductive hypothesis on \pfref{eTyp} and \pfref{eFinal}, exactly one of $\csatisfy{e}{\hxi_1}$, $\cmaysatisfy{e}{\hxi_1}$, and $\cnotsatisfyormay{e}{\hxi_1}$ holds. The same goes for $\hxi_2$. By case analysis on which conclusion holds for $\hxi_1$ and $\hxi_2$.
    \begin{byCases}
    \savelocalsteps{1}
    \item[\csatisfy{e}{\hxi_1},\csatisfy{e}{\hxi_2}]
        \begin{pfsteps*}
        \item $\csatisfy{e}{\hxi_1}$ \BY{assumption} \pflabel{[or1]satisfy1}
        \item $\cnotmaysatisfy{e}{\hxi_1}$ \BY{assumption} \pflabel{[or1]notmaysat1}
        \item $\csatisfy{e}{\hxi_2}$ \BY{assumption} \pflabel{[or1]satisfy2}
        \item $\cnotmaysatisfy{e}{\hxi_2}$ \BY{assumption} \pflabel{[or1]notmaysat2}
        \item $\csatisfy{e}{\cor{\hxi_1}{\hxi_2}}$ \BY{Rule (\ref{rule:CSOr1}) on \pfref{[or1]satisfy1}} \pflabel{[or1]satisfy-or}
        \item $\csatisfyormay{e}{\cor{\hxi_1}{\hxi_2}}$ \BY{\ruleref{rule:CSMSSat} on \pfref{[or1]satisfy-or}}
        \end{pfsteps*}
        Assume $\cmaysatisfy{e}{\cor{\hxi_1}{\hxi_2}}$. By rule induction over Rules (\ref{rules:MaySatisfy}) on it, the following cases apply.
        \begin{byCases}
        \savelocalsteps{2}
        \item[\text{(\ref{rule:CMSNotIntro})}]
            \begin{pfsteps*}
            \item $\refutable{\cor{\hxi_1}{\hxi_2}}$ \BY{assumption}
            \end{pfsteps*}
            Contradicts \autoref{lem:no-or-refutable}.
        \restorelocalsteps{2}
        \item[\text{(\ref{rule:CMSOr1})}]
            \begin{pfsteps*}
            \item $\cmaysatisfy{e}{\hxi_1}$ \BY{assumption}
            \end{pfsteps*}
            Contradicts \pfref{[or1]notmaysat1}.
        \restorelocalsteps{2}
        \item[\text{(\ref{rule:CMSOr2})}]
            \begin{pfsteps*}
            \item $\cmaysatisfy{e}{\hxi_2}$ \BY{assumption}
            \end{pfsteps*}
            Contradicts \pfref{[or1]notmaysat2}.
        \end{byCases}
        \begin{pfsteps*}
        \item $\cnotmaysatisfy{e}{\cor{\hxi_1}{\hxi_2}}$ \BY{contradiction}
        \end{pfsteps*}
        
    \restorelocalsteps{1}
    \item[\csatisfy{e}{\hxi_1},\cmaysatisfy{e}{\hxi_2}]
        \begin{pfsteps*}
        \item $\csatisfy{e}{\hxi_1}$ \BY{assumption} \pflabel{[or2]satisfy1}
        \item $\cnotmaysatisfy{e}{\hxi_1}$ \BY{assumption} \pflabel{[or2]notmaysat1}
        \item $\cnotsatisfy{e}{\hxi_2}$ \BY{assumption} \pflabel{[or2]notsatisfy2}
        \item $\cmaysatisfy{e}{\hxi_2}$ \BY{assumption} \pflabel{[or2]maysat2}
        \item $\csatisfy{e}{\cor{\hxi_1}{\hxi_2}}$ \BY{Rule (\ref{rule:CSOr1}) on \pfref{[or2]satisfy1}} \pflabel{[or2]satisfy-or}
        \item $\csatisfyormay{e}{\cor{\hxi_1}{\hxi_2}}$ \BY{\ruleref{rule:CSMSSat} on \pfref{[or2]satisfy-or}}
        \end{pfsteps*}
        Assume $\cmaysatisfy{e}{\cor{\hxi_1}{\hxi_2}}$. By rule induction over Rules (\ref{rules:MaySatisfy}) on it, the following cases apply.
        \begin{byCases}
        \savelocalsteps{2}
        \item[\text{(\ref{rule:CMSNotIntro})}]
            \begin{pfsteps*}
            \item $\refutable{\cor{\hxi_1}{\hxi_2}}$ \BY{assumption}
            \end{pfsteps*}
            Contradicts \autoref{lem:no-or-refutable}.
        \restorelocalsteps{2}
        \item[\text{(\ref{rule:CMSOr1})}]
            \begin{pfsteps*}
            \item $\cmaysatisfy{e}{\hxi_1}$ \BY{assumption}
            \end{pfsteps*}
            Contradicts \pfref{[or2]notmaysat1}.
        \restorelocalsteps{2}
        \item[\text{(\ref{rule:CMSOr2})}]
            \begin{pfsteps*}
            \item $\cnotsatisfy{e}{\hxi_1}$ \BY{assumption}
            \end{pfsteps*}
            Contradicts \pfref{[or2]satisfy1}.
        \end{byCases}
        \begin{pfsteps*}
        \item $\cnotmaysatisfy{e}{\cor{\hxi_1}{\hxi_2}}$ \BY{contradiction}
        \end{pfsteps*}
    \restorelocalsteps{1}
    \item[\csatisfy{e}{\hxi_1},\cnotsatisfyormay{e}{\hxi_2}]
        \begin{pfsteps*}
        \item $\csatisfy{e}{\hxi_1}$ \BY{assumption} \pflabel{[or3]satisfy1}
        \item $\cnotmaysatisfy{e}{\hxi_1}$ \BY{assumption} \pflabel{[or3]notmaysat1}
        \item $\cnotsatisfy{e}{\hxi_2}$ \BY{assumption} \pflabel{[or3]notsatisfy2}
        \item $\cnotmaysatisfy{e}{\hxi_2}$ \BY{assumption} \pflabel{[or3]notmaysat2}
        \item $\csatisfy{e}{\cor{\hxi_1}{\hxi_2}}$ \BY{Rule (\ref{rule:CSOr1}) on \pfref{[or3]satisfy1}} \pflabel{[or3]satisfy-or}
        \item $\csatisfyormay{e}{\cor{\hxi_1}{\hxi_2}}$ \BY{\ruleref{rule:CSMSSat} on \pfref{[or3]satisfy-or}}
        \end{pfsteps*}
        Assume $\cmaysatisfy{e}{\cor{\hxi_1}{\hxi_2}}$. By rule induction over Rules (\ref{rules:MaySatisfy}) on it, the following cases apply.
        \begin{byCases}
        \savelocalsteps{2}
        \item[\text{(\ref{rule:CMSNotIntro})}]
            \begin{pfsteps*}
            \item $\refutable{\cor{\hxi_1}{\hxi_2}}$ \BY{assumption}
            \end{pfsteps*}
            Contradicts \autoref{lem:no-or-refutable}.
        \restorelocalsteps{2}
        \item[\text{(\ref{rule:CMSOr1})}]
            \begin{pfsteps*}
            \item $\cmaysatisfy{e}{\hxi_1}$ \BY{assumption}
            \end{pfsteps*}
            Contradicts \pfref{[or3]notmaysat1}.
        \restorelocalsteps{2}
        \item[\text{(\ref{rule:CMSOr2})}]
            \begin{pfsteps*}
            \item $\cnotsatisfy{e}{\hxi_1}$ \BY{assumption}
            \end{pfsteps*}
            Contradicts \pfref{[or3]satisfy1}.
        \end{byCases}
        \begin{pfsteps*}
        \item $\cnotmaysatisfy{e}{\cor{\hxi_1}{\hxi_2}}$ \BY{contradiction}
        \end{pfsteps*}
    \restorelocalsteps{1}
    \item[\cmaysatisfy{e}{\hxi_1},\csatisfy{e}{\hxi_2}]
        \begin{pfsteps*}
        \item $\cnotsatisfy{e}{\hxi_1}$ \BY{assumption} \pflabel{[or4]notsatisfy1}
        \item $\cmaysatisfy{e}{\hxi_1}$ \BY{assumption} \pflabel{[or4]maysat1}
        \item $\csatisfy{e}{\hxi_2}$ \BY{assumption} \pflabel{[or4]satisfy2}
        \item $\cnotmaysatisfy{e}{\hxi_2}$ \BY{assumption} \pflabel{[or4]notmaysat2}
        \item $\csatisfy{e}{\cor{\hxi_1}{\hxi_2}}$ \BY{Rule (\ref{rule:CSOr2}) on \pfref{[or4]satisfy2}} \pflabel{[or4]satisfy-or}
        \item $\csatisfyormay{e}{\cor{\hxi_1}{\hxi_2}}$ \BY{\ruleref{rule:CSMSSat} on \pfref{[or4]satisfy-or}}
        \end{pfsteps*}
        Assume $\cmaysatisfy{e}{\cor{\hxi_1}{\hxi_2}}$. By rule induction over Rules (\ref{rules:MaySatisfy}) on it, the following cases apply.
        \begin{byCases}
        \savelocalsteps{2}
        \item[\text{(\ref{rule:CMSNotIntro})}]
            \begin{pfsteps*}
            \item $\refutable{\cor{\hxi_1}{\hxi_2}}$ \BY{assumption}
            \end{pfsteps*}
            Contradicts \autoref{lem:no-or-refutable}.
        \restorelocalsteps{2}
        \item[\text{(\ref{rule:CMSOr1})}]
            \begin{pfsteps*}
            \item $\cnotsatisfy{e}{\hxi_2}$ \BY{assumption}
            \end{pfsteps*}
            Contradicts \pfref{[or4]satisfy2}.
        \restorelocalsteps{2}
        \item[\text{(\ref{rule:CMSOr2})}]
            \begin{pfsteps*}
            \item $\cmaysatisfy{e}{\hxi_2}$ \BY{assumption}
            \end{pfsteps*}
            Contradicts \pfref{[or4]notmaysat2}.
        \end{byCases}
        \begin{pfsteps*}
        \item $\cnotmaysatisfy{e}{\cor{\hxi_1}{\hxi_2}}$ \BY{contradiction}
        \end{pfsteps*}
        
    \restorelocalsteps{1}
    \item[\cmaysatisfy{e}{\hxi_1},\cmaysatisfy{e}{\hxi_2}]
        \begin{pfsteps*}
        \item $\cnotsatisfy{e}{\hxi_1}$ \BY{assumption} \pflabel{[or5]notsatisfy1}
        \item $\cmaysatisfy{e}{\hxi_1}$ \BY{assumption} \pflabel{[or5]maysat1}
        \item $\cnotsatisfy{e}{\hxi_2}$ \BY{assumption} \pflabel{[or5]notsatisfy2}
        \item $\cmaysatisfy{e}{\hxi_2}$ \BY{assumption} \pflabel{[or5]maysat2}
        \item $\cmaysatisfy{e}{\cor{\hxi_1}{\hxi_2}}$ \BY{Rule (\ref{rule:CMSOr1}) on \pfref{[or5]maysat1} and \pfref{[or5]notsatisfy2}} \pflabel{[or5]maysat-or}
        \item $\csatisfyormay{e}{\cor{\hxi_1}{\hxi_2}}$ \BY{\ruleref{rule:CSMSMay} on \pfref{[or5]maysat-or}}
        \end{pfsteps*}
        Assume $\csatisfy{e}{\cor{\hxi_1}{\hxi_2}}$. By rule induction over Rules (\ref{rules:Satisfy}), only two cases apply.
        \begin{byCases}
        \savelocalsteps{2}
        \item[\text{(\ref{rule:CSOr1})}]
            \begin{pfsteps*}
            \item $\csatisfy{e}{\hxi_1}$ \BY{assumption}
            \end{pfsteps*}
            Contradicts $\pfref{[or5]notsatisfy1}$
        \restorelocalsteps{2}
        \item[\text{(\ref{rule:CSOr2})}]
            \begin{pfsteps*}
            \item $\csatisfy{e}{\hxi_2}$ \BY{assumption}
            \end{pfsteps*}
            Contradicts $\pfref{[or5]notsatisfy2}$
        \end{byCases}
        \begin{pfsteps*}
        \item $\cnotsatisfy{e}{\cor{\hxi_1}{\hxi_2}}$ \BY{contradiction}
        \end{pfsteps*}
    \restorelocalsteps{1}
    \item[\cmaysatisfy{e}{\hxi_1},\cnotsatisfyormay{e}{\hxi_2}]
        \begin{pfsteps*}
        \item $\cnotsatisfy{e}{\hxi_1}$ \BY{assumption} \pflabel{[or6]notsatisfy1}
        \item $\cmaysatisfy{e}{\hxi_1}$ \BY{assumption} \pflabel{[or6]maysat1}
        \item $\cnotsatisfy{e}{\hxi_2}$ \BY{assumption} \pflabel{[or6]notsatisfy2}
        \item $\cnotmaysatisfy{e}{\hxi_2}$ \BY{assumption} \pflabel{[or6]notmaysat2}
        \item $\cmaysatisfy{e}{\cor{\hxi_1}{\hxi_2}}$ \BY{Rule (\ref{rule:CMSOr1}) on \pfref{[or6]maysat1} and \pfref{[or6]notsatisfy2}} \pflabel{[or6]maysat-or}
        \item $\csatisfyormay{e}{\cor{\hxi_1}{\hxi_2}}$ \BY{\ruleref{rule:CSMSMay} on \pfref{[or6]maysat-or}}
        \end{pfsteps*}
        Assume $\csatisfy{e}{\cor{\hxi_1}{\hxi_2}}$. By rule induction over Rules (\ref{rules:Satisfy}), only two cases apply.
        \begin{byCases}
        \savelocalsteps{2}
        \item[\text{(\ref{rule:CSOr1})}]
            \begin{pfsteps*}
            \item $\csatisfy{e}{\hxi_1}$ \BY{assumption}
            \end{pfsteps*}
            Contradicts \pfref{[or6]notsatisfy1}.
        \restorelocalsteps{2}
        \item[\text{(\ref{rule:CSOr2})}]
            \begin{pfsteps*}
            \item $\csatisfy{e}{\hxi_2}$ \BY{assumption}
            \end{pfsteps*}
            Contradicts \pfref{[or6]notsatisfy2}.
        \end{byCases}
        \begin{pfsteps*}
        \item $\cnotsatisfy{e}{\cor{\hxi_1}{\hxi_2}}$ \BY{contradiction}
        \end{pfsteps*}
    \restorelocalsteps{1}
    \item[\cnotsatisfyormay{e}{\hxi_1},\csatisfy{e}{\hxi_2}]
        \begin{pfsteps*}
        \item $\cnotsatisfy{e}{\hxi_1}$ \BY{assumption} \pflabel{[or7]notsatisfy1}
        \item $\cnotmaysatisfy{e}{\hxi_1}$ \BY{assumption} \pflabel{[or7]notmaysat1}
        \item $\csatisfy{e}{\hxi_2}$ \BY{assumption} \pflabel{[or7]satisfy2}
        \item $\cnotmaysatisfy{e}{\hxi_2}$ \BY{assumption} \pflabel{[or7]notmaysat2}
        \item $\csatisfy{e}{\cor{\hxi_1}{\hxi_2}}$ \BY{Rule (\ref{rule:CSOr2}) on \pfref{[or7]satisfy2}} \pflabel{[or7]satisfy-or}
        \item $\csatisfyormay{e}{\cor{\hxi_1}{\hxi_2}}$ \BY{\ruleref{rule:CSMSSat} on \pfref{[or7]satisfy-or}}
        \end{pfsteps*}
        Assume $\cmaysatisfy{e}{\cor{\hxi_1}{\hxi_2}}$. By rule induction over Rules (\ref{rules:MaySatisfy}) on it, the following cases apply.
        \begin{byCases}
        \savelocalsteps{2}
        \item[\text{(\ref{rule:CMSNotIntro})}]
            \begin{pfsteps*}
            \item $\refutable{\cor{\hxi_1}{\hxi_2}}$ \BY{assumption}
            \end{pfsteps*}
            Contradicts \autoref{lem:no-or-refutable}.
        \restorelocalsteps{2}
        \item[\text{(\ref{rule:CMSOr1})}]
            \begin{pfsteps*}
            \item $\cnotsatisfy{e}{\hxi_2}$ \BY{assumption}
            \end{pfsteps*}
            Contradicts \pfref{[or7]satisfy2}.
        \restorelocalsteps{2}
        \item[\text{(\ref{rule:CMSOr2})}]
            \begin{pfsteps*}
            \item $\cmaysatisfy{e}{\hxi_2}$ \BY{assumption}
            \end{pfsteps*}
            Contradicts \pfref{[or7]notmaysat2}.
        \end{byCases}
        \begin{pfsteps*}
        \item $\cnotmaysatisfy{e}{\cor{\hxi_1}{\hxi_2}}$ \BY{contradiction}
        \end{pfsteps*}
        
    \restorelocalsteps{1}
    \item[\cnotsatisfyormay{e}{\hxi_1},\cmaysatisfy{e}{\hxi_2}]
        \begin{pfsteps*}
        \item $\cnotsatisfy{e}{\hxi_1}$ \BY{assumption} \pflabel{[or8]notsatisfy1}
        \item $\cnotmaysatisfy{e}{\hxi_1}$ \BY{assumption} \pflabel{[or8]notmaysat1}
        \item $\cnotsatisfy{e}{\hxi_2}$ \BY{assumption} \pflabel{[or8]notsatisfy2}
        \item $\cmaysatisfy{e}{\hxi_2}$ \BY{assumption} \pflabel{[or8]maysat2}
        \item $\cmaysatisfy{e}{\cor{\hxi_1}{\hxi_2}}$ \BY{Rule (\ref{rule:CMSOr2}) on \pfref{[or8]maysat2} and \pfref{[or8]notsatisfy1}} \pflabel{[or8]maysat-or}
        \item $\csatisfyormay{e}{\cor{\hxi_1}{\hxi_2}}$ \BY{\ruleref{rule:CSMSMay} on \pfref{[or8]maysat-or}}
        \end{pfsteps*}
        Assume $\csatisfy{e}{\cor{\hxi_1}{\hxi_2}}$. By rule induction over Rules (\ref{rules:Satisfy}), only two cases apply.
        \begin{byCases}
        \savelocalsteps{2}
        \item[\text{(\ref{rule:CSOr1})}]
            \begin{pfsteps*}
            \item $\csatisfy{e}{\hxi_1}$ \BY{assumption}
            \end{pfsteps*}
            Contradicts $\pfref{[or8]notsatisfy1}$
        \restorelocalsteps{2}
        \item[\text{(\ref{rule:CSOr2})}]
            \begin{pfsteps*}
            \item $\csatisfy{e}{\hxi_2}$ \BY{assumption}
            \end{pfsteps*}
            Contradicts $\pfref{[or8]notsatisfy2}$
        \end{byCases}
        \begin{pfsteps*}
        \item $\cnotsatisfy{e}{\cor{\hxi_1}{\hxi_2}}$ \BY{contradiction}
        \end{pfsteps*}
    \restorelocalsteps{1}
    \item[\cnotsatisfyormay{e}{\hxi_1},\cnotsatisfyormay{e}{\hxi_2}]
        \begin{pfsteps*}
        \item $\cnotsatisfy{e}{\hxi_1}$ \BY{assumption} \pflabel{[or9]notsatisfy1}
        \item $\cnotmaysatisfy{e}{\hxi_1}$ \BY{assumption} \pflabel{[or9]notmaysat1}
        \item $\cnotsatisfy{e}{\hxi_2}$ \BY{assumption} \pflabel{[or9]notsatisfy2}
        \item $\cnotmaysatisfy{e}{\hxi_2}$ \BY{assumption} \pflabel{[or9]notmaysat2}
        \end{pfsteps*}
        Assume $\csatisfy{e}{\cor{\hxi_1}{\hxi_2}}$. By rule induction over Rules (\ref{rules:Satisfy}) on it, only two cases apply.
        \begin{byCases}
        \savelocalsteps{2}
        \item[\text{(\ref{rule:CSOr1})}]
            \begin{pfsteps*}
            \item $\csatisfy{e}{\hxi_1}$ \BY{assumption}
            \end{pfsteps*}
            Contradicts \pfref{[or9]notsatisfy1}.
        \restorelocalsteps{2}
        \item[\text{(\ref{rule:CSOr2})}]
            \begin{pfsteps*}
            \item $\csatisfy{e}{\hxi_2}$ \BY{assumption}
            \end{pfsteps*}
            Contradicts \pfref{[or9]notsatisfy2}.
        \end{byCases}
        \begin{pfsteps*}
        \item $\cnotsatisfy{e}{\cor{\hxi_1}{\hxi_2}}$ \BY{contradiction} \pflabel{[or9]notsatisfy}
        \end{pfsteps*}
        Assume $\cmaysatisfy{e}{\cor{\hxi_1}{\hxi_2}}$. By rule induction over Rules (\ref{rules:MaySatisfy}) on it, the following cases apply.
        \begin{byCases}
        \savelocalsteps{2}
        \item[\text{(\ref{rule:CMSNotIntro})}]
            \begin{pfsteps*}
            \item $\refutable{\cor{\hxi_1}{\hxi_2}}$ \BY{assumption}
            \end{pfsteps*}
            Contradicts \autoref{lem:no-or-refutable}.
        \restorelocalsteps{2}
        \item[\text{(\ref{rule:CMSOr1})}]
            \begin{pfsteps*}
            \item $\cmaysatisfy{e}{\hxi_1}$ \BY{assumption}
            \end{pfsteps*}
            Contradicts \pfref{[or9]notmaysat1}.
        \restorelocalsteps{2}
        \item[\text{(\ref{rule:CMSOr2})}]
            \begin{pfsteps*}
            \item $\cmaysatisfy{e}{\hxi_2}$ \BY{assumption}
            \end{pfsteps*}
            Contradicts \pfref{[or9]notmaysat2}.
        \end{byCases}
        \begin{pfsteps*}
        \item $\cnotmaysatisfy{e}{\cor{\hxi_1}{\hxi_2}}$ \BY{contradiction} \pflabel{[or9]notmaysat}
        \item $\cnotsatisfyormay{e}{\cor{\hxi_1}{\hxi_2}}$ \BY{\autoref{lem:not-satormay} on \pfref{[or9]notsatisfy} and \pfref{[or9]notmaysat}}
        \end{pfsteps*}
        
        
    \end{byCases}
\restorelocalsteps{0}
\item[\text{(\ref{rule:CTInl})}]
    \begin{pfsteps*}
    \item $\hxi=\cinl{\hxi_1}$ \BY{assumption}
    \item $\tau=\tsum{\tau_1}{\tau_2}$ \BY{assumption}
    \item $\ctyp{\hxi_1}{\tau_1}$ \BY{assumption} \pflabel{c1Typ}
    \end{pfsteps*}
    By rule induction over Rules (\ref{rules:TExp}) on \pfref{eTyp}, the following cases apply.
    \begin{byCases}
    \savelocalsteps{1}
    \item[\text{(\ref{rule:TEHole}),(\ref{rule:THole}),(\ref{rule:TAp}),(\ref{rule:TPrl}),(\ref{rule:TPrr}),(\ref{rule:TMatchZPre}),(\ref{rule:TMatchNZPre})}]
        \begin{pfsteps*}
        \item $e=\hehole{u},\hhole{e_0}{u},\hap{e_1}{e_2},\hprl{e_0},\hprr{e_0},\hmatch{e_0}{\zrules}$ \BY{assumption}
        \item $\notIntro{e}$ \BY{Rule (\ref{rule:NVEHole}),(\ref{rule:NVHole}),(\ref{rule:NVAp}),(\ref{rule:NVMatch}),(\ref{rule:NVPrl}),(\ref{rule:NVPrr})} \pflabel{[inl]notintro}
        \end{pfsteps*}
        Assume $\csatisfy{e}{\cinl{\hxi_1}}$. By rule induction over Rules (\ref{rules:Satisfy}) on it, no case applies due to syntactic contradiction.
        \begin{pfsteps*}
        \item $\cnotsatisfy{e}{\cinl{\hxi_1}}$ \BY{contradiction} \pflabel{[inl]notsat-inl}
        \end{pfsteps*}
        By case analysis on the value of $\frefutable{\cinl{\hxi_1}}$.
        \begin{byCases}
        \savelocalsteps{2}
        \item[\frefutable{\cinl{\hxi_1}}=\true]
            \begin{pfsteps*}
            \item $\frefutable{\cinl{\hxi_1}}=\true$ \BY{assumption} \pflabel{[inl]frft-true}
            \item $\refutable{\cinl{\hxi_1}}$ \BY{\autoref{lem:sound-complete-xi-refutable} on \pfref{[inl]frft-true}} \pflabel{[inl]rft-true}
            \item $\cmaysatisfy{e}{\cinl{\hxi_1}}$ \BY{\ruleref{rule:CMSNotIntro} on \pfref{[inl]notintro} and \pfref{[inl]rft-true}} \pflabel{[inl]maysat}
            \item $\csatisfyormay{e}{\cinl{\hxi_1}}$ \BY{\ruleref{rule:CSMSMay} on \pfref{[inl]maysat}}
            \end{pfsteps*}
        \restorelocalsteps{2}
        \item[\frefutable{\cinl{\hxi_1}}=\false]
            \begin{pfsteps*}
            \item $\frefutable{\cinl{\hxi_1}}=\false$ \BY{assumption} \pflabel{[inl]frft-false}
            \item $\cancel{\refutable{\cinl{\hxi_1}}}$ \BY{\autoref{lem:sound-complete-xi-refutable} on \pfref{[inl]frft-false}} \pflabel{[inl]rft-false}
            \end{pfsteps*}
            Assume $\cmaysatisfy{e}{\cinl{\hxi_1}}$. By rule induction over \rulesref{rules:MaySatisfy} on it, only one case applies.
            \begin{byCases}
            \item[\text{(\ref{rule:CMSNotIntro})}]
                \begin{pfsteps*}
                \item $\refutable{\cinl{\hxi_1}}$ \BY{assumption}
                \end{pfsteps*}
                Contradicts \pfref{[inl]rft-false}.
            \end{byCases}
            \begin{pfsteps*}
            \item $\cnotmaysatisfy{e}{\cinl{\hxi_1}}$ \BY{contradiction} \pflabel{[inl]notmaysat-inl}
            \item $\cnotsatisfyormay{e}{\cinl{\hxi_1}}$ \BY{\autoref{lem:not-satormay} on \pfref{[inl]notsat-inl} and \pfref{[inl]notmaysat-inl}}
            \end{pfsteps*}
        \end{byCases}
    \restorelocalsteps{1}
    \item[\text{(\ref{rule:TInl})}]
        \begin{pfsteps*}
        \item $e=\hinl{\tau_2}{e_1}$ \BY{assumption}
        \item $\hexptyp{\cdot}{\Delta}{e_1}{\tau_1}$ \BY{assumption} \pflabel{e1Typ}
        \item $\isFinal{e_1}$ \BY{\autoref{lem:inl-final} on \pfref{eFinal}} \pflabel{e1Final}
        \end{pfsteps*}
        By inductive hypothesis on \pfref{c1Typ} and \pfref{e1Typ} and \pfref{e1Final}, exactly one of $\csatisfy{e_1}{\hxi_1}$, $\cmaysatisfy{e_1}{\hxi_1}$, and $\cnotsatisfyormay{e_1}{\hxi_1}$ holds. By case analysis on which one holds.
        \begin{byCases}
        \savelocalsteps{2}
        \item[\csatisfy{e_1}{\hxi_1}]
            \begin{pfsteps*}
            \item $\csatisfy{e_1}{\hxi_1}$ \BY{assumption} \pflabel{[inl1]satisfy1}
            \item $\cnotmaysatisfy{e_1}{\hxi_1}$ \BY{assumption} \pflabel{[inl1]notmaysat1}
            \item $\csatisfy{\hinl{\tau_2}{e_1}}{\cinl{\hxi_1}}$ \BY{Rule (\ref{rule:CSInl}) on \pfref{[inl1]satisfy1}} \pflabel{[inl1]satisfy}
            \item $\csatisfyormay{\hinl{\tau_2}{e_1}}{\cinl{\hxi_1}}$ \BY{\ruleref{rule:CSMSSat} on \pfref{[inl1]satisfy}}
            \end{pfsteps*}
            Assume $\cmaysatisfy{\hinl{\tau_2}{e_1}}{\cinl{\hxi_1}}$. By rule induction over Rules (\ref{rules:MaySatisfy}) on it, only two cases apply.
            \begin{byCases}
            \savelocalsteps{3}
            \item[\text{(\ref{rule:CMSNotIntro})}]
                \begin{pfsteps*}
                \item $\notIntro{\hinl{\tau_2}{e_1}}$ \BY{assumption} \pflabel{[inl1]notintro-inl}
                \end{pfsteps*}
                By rule induction over Rules (\ref{rules:notintro}) on \pfref{[inl1]notintro-inl}, no case applies due to syntactic contradiction.
            \restorelocalsteps{3}
            \item[\text{(\ref{rule:CMSInl})}]
                \begin{pfsteps*}
                \item $\cmaysatisfy{e_1}{\hxi_1}$
                \end{pfsteps*}
                Contradicts \pfref{[inl1]notmaysat1}.
            \end{byCases}
            \begin{pfsteps*}
            \item $\cnotmaysatisfy{\hinl{\tau_2}{e_1}}{\cinl{\hxi_1}}$ \BY{contradiction}
            \end{pfsteps*}
            
        \restorelocalsteps{2}
        \item[\cmaysatisfy{e_1}{\hxi_1}]
            \begin{pfsteps*}
            \item $\cnotsatisfy{e_1}{\hxi_1}$ \BY{assumption} \pflabel{[inl2]notsatisfy1}
            \item $\cmaysatisfy{e_1}{\hxi_1}$ \BY{assumption} \pflabel{[inl2]maysat1}
            \item $\cmaysatisfy{\hinl{\tau_2}{e_1}}{\cinl{\hxi_1}}$ \BY{Rule (\ref{rule:CMSInl}) on \pfref{[inl2]maysat1}} \pflabel{[inl2]maysat}
            \item $\csatisfyormay{\hinl{\tau_2}{e_1}}{\cinl{\hxi_1}}$ \BY{\ruleref{rule:CSMSMay} on \pfref{[inl2]maysat}}
            \end{pfsteps*}
            Assume $\csatisfy{\hinl{\tau_2}{e_1}}{\cinl{\hxi_1}}$. By rule induction over Rules (\ref{rules:Satisfy}) on it, only one case applies.
            \begin{byCases}
            \item[\text{(\ref{rule:CSInl})}]
                \begin{pfsteps*}
                \item $\csatisfy{e_1}{\hxi_1}$
                \end{pfsteps*}
                Contradicts \pfref{[inl2]notsatisfy1}.
            \end{byCases}
            \begin{pfsteps*}
            \item $\cnotsatisfy{\hinl{\tau_2}{e_1}}{\cinl{\hxi_1}}$ \BY{contradiction}
            \end{pfsteps*}
           
        \restorelocalsteps{2}
        \item[\cnotsatisfyormay{e_1}{\hxi_1}]
            \begin{pfsteps*}
            \item $\cnotsatisfy{e_1}{\hxi_1}$ \BY{assumption} \pflabel{[inl3]notsatisfy1}
            \item $\cnotmaysatisfy{e_1}{\hxi_1}$ \BY{assumption} \pflabel{[inl3]notmaysat1}
            \end{pfsteps*}
            Assume $\csatisfy{\hinl{\tau_2}{e_1}}{\cinl{\hxi_1}}$. By rule induction over Rules (\ref{rules:Satisfy}) on it, only one case applies.
            \begin{byCases}
            \item[\text{(\ref{rule:CSInl})}]
                \begin{pfsteps*}
                \item $\csatisfy{e_1}{\hxi_1}$
                \end{pfsteps*}
                Contradicts \pfref{[inl3]notsatisfy1}.
            \end{byCases}
            \begin{pfsteps*}
            \item $\cnotsatisfy{\hinl{\tau_2}{e_1}}{\cinl{\hxi_1}}$ \BY{contradiction} \pflabel{[inl3]notsatisfy}
            \end{pfsteps*}
            Assume $\cmaysatisfy{\hinl{\tau_2}{e_1}}{\cinl{\hxi_1}}$. By rule induction over Rules (\ref{rules:MaySatisfy}) on it, only one case applies.
            \begin{byCases}
            \savelocalsteps{3}
            \item[\text{(\ref{rule:CMSNotIntro})}]
                \begin{pfsteps*}
                \item $\notIntro{\hinl{\tau_2}{e_1}}$ \BY{assumption} \pflabel{[inl3]notintro-inl}
                \end{pfsteps*}
                By rule induction over Rules (\ref{rules:notintro}) on \pfref{[inl3]notintro-inl}, no case applies due to syntactic contradiction.
            \restorelocalsteps{3}
            \item[\text{(\ref{rule:CMSInl})}]
                \begin{pfsteps*}
                \item $\cmaysatisfy{e_1}{\hxi_1}$
                \end{pfsteps*}
                Contradicts \pfref{[inl3]notmaysat1}.
            \end{byCases}
            \begin{pfsteps*}
            \item $\cnotmaysatisfy{\hinl{\tau_2}{e_1}}{\cinl{\hxi_1}}$ \BY{contradiction} \pflabel{[inl3]notmaysatisfy}
            \item $\cnotsatisfyormay{\hinl{\tau_2}{e_1}}{\cinl{\hxi_1}}$ \BY{\autoref{lem:not-satormay} on \pfref{[inl3]notsatisfy} and \pfref{[inl3]notmaysatisfy}}
            \end{pfsteps*}
        \end{byCases}
    \restorelocalsteps{1}
    \item[\text{(\ref{rule:TInr})}]
        \begin{pfsteps*}
        \item $e=\hinr{\tau_1}{e_2}$ \BY{assumption}
        \end{pfsteps*}
        Assume $\csatisfy{\hinr{\tau_1}{e_2}}{\cinl{\hxi_1}}$. By rule induction over Rules (\ref{rules:Satisfy}) on it, no case applies due to syntactic contradiction.
        \begin{pfsteps*}
        \item $\cnotsatisfy{\hinr{\tau_1}{e_2}}{\cinl{\hxi_1}}$ \BY{contradiction} \pflabel{[inl]notsat-conf}
        \end{pfsteps*}
        Assume $\cmaysatisfy{\hinr{\tau_1}{e_2}}{\cinl{\hxi_1}}$. By rule induction over Rules (\ref{rules:MaySatisfy}) on it, only one case applies.
        \begin{byCases}
        \item[\text{(\ref{rule:CMSNotIntro})}]
            \begin{pfsteps*}
            \item $\notIntro{\hinr{\tau_1}{e_2}}$ \BY{assumption} \pflabel{[inl]notintro-inr}
            \end{pfsteps*}
            By rule induction over Rules (\ref{rules:notintro}) on \pfref{[inl]notintro-inr}, no case applies due to syntactic contradiction.
        \end{byCases}
        \begin{pfsteps*}
        \item $\cnotmaysatisfy{\hinr{\tau_1}{e_2}}{\cinl{\hxi_1}}$ \BY{contradiction} \pflabel{[inl]notmaysat-conf}
        \item $\cnotsatisfyormay{\hinr{\tau_1}{e_2}}{\cinl{\hxi_1}}$ \BY{\autoref{lem:not-satormay} on \pfref{[inl]notsat-conf} and \pfref{[inl]notmaysat-conf}}
        \end{pfsteps*}
    \end{byCases}
\restorelocalsteps{0}
\item[\text{(\ref{rule:CTInr})}]
    \begin{pfsteps*}
    \item $\hxi=\cinr{\hxi_2}$ \BY{assumption}
    \item $\tau=\tsum{\tau_1}{\tau_2}$ \BY{assumption}
    \item $\ctyp{\hxi_2}{\tau_2}$ \BY{assumption} \pflabel{c2Typ}
    \end{pfsteps*}
    By rule induction over Rules (\ref{rules:TExp}) on \pfref{eTyp}, the following cases apply.
    \begin{byCases}
    \savelocalsteps{1}
    \item[\text{(\ref{rule:TEHole}),(\ref{rule:THole}),(\ref{rule:TAp}),(\ref{rule:TPrl}),(\ref{rule:TPrr}),(\ref{rule:TMatchZPre}),(\ref{rule:TMatchNZPre})}]
        \begin{pfsteps*}
        \item $e=\hehole{u},\hhole{e_0}{u},\hap{e_1}{e_2},\hprl{e_0},\hprr{e_0},\hmatch{e_0}{\zrules}$ \BY{assumption}
        \item $\notIntro{e}$ \BY{Rule (\ref{rule:NVEHole}),(\ref{rule:NVHole}),(\ref{rule:NVAp}),(\ref{rule:NVMatch}),(\ref{rule:NVPrl}),(\ref{rule:NVPrr})} \pflabel{[inr]notintro}
        \end{pfsteps*}
        Assume $\csatisfy{e}{\cinr{\hxi_2}}$. By rule induction over Rules (\ref{rules:Satisfy}) on it, no case applies due to syntactic contradiction.
        \begin{pfsteps*}
        \item $\cnotsatisfy{e}{\cinr{\hxi_2}}$ \BY{contradiction} \pflabel{[inr]notsat-inr}
        \end{pfsteps*}
        By case analysis on the value of $\frefutable{\cinr{\hxi_2}}$. \todo{inr is refutable}
        \begin{byCases}
        \savelocalsteps{2}
        \item[\frefutable{\cinr{\hxi_2}}=\true]
            \begin{pfsteps*}
            \item $\frefutable{\cinr{\hxi_2}}=\true$ \BY{assumption} \pflabel{[inr]frft-true}
            \item $\refutable{\cinr{\hxi_2}}$ \BY{\autoref{lem:sound-complete-xi-refutable} on \pfref{[inr]frft-true}} \pflabel{[inr]rft-true}
            \item $\cmaysatisfy{e}{\cinr{\hxi_2}}$ \BY{\ruleref{rule:CMSNotIntro} on \pfref{[inr]notintro} and \pfref{[inr]rft-true}} \pflabel{[inr]maysat}
            \item $\csatisfyormay{e}{\cinr{\hxi_2}}$ \BY{\ruleref{rule:CSMSMay} on \pfref{[inr]maysat}}
            \end{pfsteps*}
        \restorelocalsteps{2}
        \item[\frefutable{\cinr{\hxi_2}}=\false]
            \begin{pfsteps*}
            \item $\frefutable{\cinr{\hxi_2}}=\false$ \BY{assumption} \pflabel{[inr]frft-false}
            \item $\cancel{\refutable{\cinr{\hxi_2}}}$ \BY{\autoref{lem:sound-complete-xi-refutable} on \pfref{[inr]frft-false}} \pflabel{[inr]rft-false}
            \end{pfsteps*}
            Assume $\cmaysatisfy{e}{\cinr{\hxi_2}}$. By rule induction over \rulesref{rules:MaySatisfy} on it, only one case applies.
            \begin{byCases}
            \item[\text{(\ref{rule:CMSNotIntro})}]
                \begin{pfsteps*}
                \item $\refutable{\cinr{\hxi_2}}$ \BY{assumption}
                \end{pfsteps*}
                Contradicts \pfref{[inr]rft-false}.
            \end{byCases}
            \begin{pfsteps*}
            \item $\cnotmaysatisfy{e}{\cinr{\hxi_2}}$ \BY{contradiction} \pflabel{[inr]notmaysat-inr}
            \item $\cnotsatisfyormay{e}{\cinr{\hxi_2}}$ \BY{\autoref{lem:not-satormay} on \pfref{[inr]notsat-inr} and \pfref{[inr]notmaysat-inr}}
            \end{pfsteps*}
        \end{byCases}
    \restorelocalsteps{1}
    \item[\text{(\ref{rule:TInl})}]
        \begin{pfsteps*}
        \item $e=\hinl{\tau_2}{e_1}$ \BY{assumption}
        \end{pfsteps*}
        Assume $\csatisfy{\hinl{\tau_2}{e_1}}{\cinr{\hxi_2}}$. By rule induction over Rules (\ref{rules:Satisfy}) on it, no case applies due to syntactic contradiction.
        \begin{pfsteps*}
        \item $\cnotsatisfy{\hinl{\tau_2}{e_1}}{\cinr{\hxi_2}}$ \BY{contradiction} \pflabel{[inr]notsat-conf}
        \end{pfsteps*}
        Assume $\cmaysatisfy{\hinl{\tau_2}{e_1}}{\cinr{\hxi_2}}$. By rule induction over Rules (\ref{rules:MaySatisfy}) on it, only one case applies.
        \begin{byCases}
        \item[\text{(\ref{rule:CMSNotIntro})}]
            \begin{pfsteps*}
            \item $\notIntro{\hinl{\tau_2}{e_1}}$ \BY{assumption} \pflabel{[inr]notintro-inl}
            \end{pfsteps*}
            By rule induction over Rules (\ref{rules:notintro}) on \pfref{[inr]notintro-inl}, no case applies due to syntactic contradiction.
        \end{byCases}
        \begin{pfsteps*}
        \item $\cnotmaysatisfy{\hinl{\tau_2}{e_1}}{\cinr{\hxi_2}}$ \BY{contradiction} \pflabel{[inr]notmaysat-conf}
        \item $\cnotsatisfyormay{\hinl{\tau_2}{e_1}}{\cinr{\hxi_2}}$ \BY{\autoref{lem:not-satormay} on \pfref{[inr]notsat-conf} and \pfref{[inr]notmaysat-conf}}
        \end{pfsteps*}
        
    \restorelocalsteps{1}
    \item[\text{(\ref{rule:TInr})}]
        \begin{pfsteps*}
        \item $e=\hinr{\tau_1}{e_2}$ \BY{assumption}
        \item $\hexptyp{\cdot}{\Delta}{e_2}{\tau_2}$ \BY{assumption} \pflabel{e2Typ}
        \item $\isFinal{e_2}$ \BY{\autoref{lem:inr-final} on \pfref{eFinal}} \pflabel{e2Final}
        \end{pfsteps*}
        By inductive hypothesis on \pfref{c2Typ} and \pfref{e2Typ} and \pfref{e2Final}, exactly one of $\csatisfy{e_2}{\hxi_2}$, $\cmaysatisfy{e_2}{\hxi_2}$, and $\cnotsatisfyormay{e_2}{\hxi_2}$ holds. By case analysis on which one holds.
        \begin{byCases}
        \savelocalsteps{2}
        \item[\csatisfy{e_2}{\hxi_2}]
            \begin{pfsteps*}
            \item $\csatisfy{e_2}{\hxi_2}$ \BY{assumption} \pflabel{[inr1]satisfy2}
            \item $\cnotmaysatisfy{e_2}{\hxi_2}$ \BY{assumption} \pflabel{[inr1]notmaysat2}
            \item $\csatisfy{\hinr{\tau_1}{e_2}}{\cinr{\hxi_2}}$ \BY{Rule (\ref{rule:CSInl}) on \pfref{[inr1]satisfy2}} \pflabel{[inr1]satisfy}
            \item $\csatisfyormay{\hinr{\tau_1}{e_2}}{\cinr{\hxi_2}}$ \BY{\ruleref{rule:CSMSSat} on \pfref{[inr1]satisfy}}
            \end{pfsteps*}
            Assume $\cmaysatisfy{\hinr{\tau_1}{e_2}}{\cinr{\hxi_2}}$. By rule induction over Rules (\ref{rules:MaySatisfy}) on it, only two cases apply.
            \begin{byCases}
            \savelocalsteps{3}
            \item[\text{(\ref{rule:CMSNotIntro})}]
                \begin{pfsteps*}
                \item $\notIntro{\hinr{\tau_1}{e_2}}$ \BY{assumption} \pflabel{[inr1]notintro-inr}
                \end{pfsteps*}
                By rule induction over Rules (\ref{rules:notintro}) on \pfref{[inr1]notintro-inr}, no case applies due to syntactic contradiction.
            \restorelocalsteps{3}
            \item[\text{(\ref{rule:CMSInr})}]
                \begin{pfsteps*}
                \item $\cmaysatisfy{e_2}{\hxi_2}$
                \end{pfsteps*}
                Contradicts \pfref{[inr1]notmaysat2}.
            \end{byCases}
            \begin{pfsteps*}
            \item $\cnotmaysatisfy{\hinr{\tau_1}{e_2}}{\cinr{\hxi_2}}$ \BY{contradiction}
            \end{pfsteps*}
        \restorelocalsteps{2}
        \item[\cmaysatisfy{e_2}{\hxi_2}]
            \begin{pfsteps*}
            \item $\cnotsatisfy{e_2}{\hxi_2}$ \BY{assumption} \pflabel{[inr2]notsatisfy2}
            \item $\cmaysatisfy{e_2}{\hxi_2}$ \BY{assumption} \pflabel{[inr2]maysat2}
            \item $\cmaysatisfy{\hinr{\tau_1}{e_2}}{\cinr{\hxi_2}}$ \BY{Rule (\ref{rule:CMSInr}) on \pfref{[inr2]maysat2}} \pflabel{[inr2]maysat}
            \item $\csatisfyormay{\hinr{\tau_1}{e_2}}{\cinr{\hxi_2}}$ \BY{\ruleref{rule:CSMSMay} on \pfref{[inr2]maysat}}
            \end{pfsteps*}
            Assume $\csatisfy{\hinr{\tau_1}{e_2}}{\cinr{\hxi_2}}$. By rule induction over Rules (\ref{rules:Satisfy}) on it, only one case applies.
            \begin{byCases}
            \item[\text{(\ref{rule:CSInr})}]
                \begin{pfsteps*}
                \item $\csatisfy{e_2}{\hxi_2}$
                \end{pfsteps*}
                Contradicts \pfref{[inr2]notsatisfy2}.
            \end{byCases}
            \begin{pfsteps*}
            \item $\cnotsatisfy{\hinr{\tau_1}{e_2}}{\cinr{\hxi_2}}$ \BY{contradiction}
            \end{pfsteps*}
        \restorelocalsteps{2}
        \item[\cnotsatisfyormay{e_2}{\hxi_2}]
            \begin{pfsteps*}
            \item $\cnotsatisfy{e_2}{\hxi_2}$ \BY{assumption} \pflabel{[inr3]notsatisfy2}
            \item $\cnotmaysatisfy{e_2}{\hxi_2}$ \BY{assumption} \pflabel{[inr3]notmaysat2}
            \end{pfsteps*}
            Assume $\csatisfy{\hinr{\tau_1}{e_2}}{\cinr{\hxi_2}}$. By rule induction over Rules (\ref{rules:Satisfy}) on it, only one case applies.
            \begin{byCases}
            \item[\text{(\ref{rule:CSInr})}]
                \begin{pfsteps*}
                \item $\csatisfy{e_2}{\hxi_2}$
                \end{pfsteps*}
                Contradicts \pfref{[inr3]notsatisfy2}.
            \end{byCases}
            \begin{pfsteps*}
            \item $\cnotsatisfy{\hinr{\tau_1}{e_2}}{\cinr{\hxi_2}}$ \BY{contradiction} \pflabel{[inr3]notsatisfy}
            \end{pfsteps*}
            Assume $\cmaysatisfy{\hinr{\tau_1}{e_2}}{\cinr{\hxi_2}}$. By rule induction over Rules (\ref{rules:MaySatisfy}) on it, only one case applies.
            \begin{byCases}
            \savelocalsteps{3}
            \item[\text{(\ref{rule:CMSNotIntro})}]
                \begin{pfsteps*}
                \item $\notIntro{\hinr{\tau_1}{e_2}}$ \BY{assumption} \pflabel{[inr3]notintro-inr}
                \end{pfsteps*}
                By rule induction over Rules (\ref{rules:notintro}) on \pfref{[inr3]notintro-inr}, no case applies due to syntactic contradiction.
            \restorelocalsteps{3}
            \item[\text{(\ref{rule:CMSInr})}]
                \begin{pfsteps*}
                \item $\cmaysatisfy{e_2}{\hxi_2}$
                \end{pfsteps*}
                Contradicts \pfref{[inr3]notmaysat2}.
            \end{byCases}
            \begin{pfsteps*}
            \item $\cnotmaysatisfy{\hinr{\tau_1}{e_2}}{\cinr{\hxi_2}}$ \BY{contradiction} \pflabel{[inr3]notmaysatisfy}
            \item $\cnotsatisfyormay{\hinl{\tau_2}{e_1}}{\cinl{\hxi_1}}$ \BY{\autoref{lem:not-satormay} on \pfref{[inr3]notsatisfy} and \pfref{[inr3]notmaysatisfy}}
            \end{pfsteps*}
        \end{byCases}
    \end{byCases}

\restorelocalsteps{0}
\item[\text{(\ref{rule:CSPair})}]
\begin{pfsteps*}
    \item $\hxi=\cpair{\hxi_1}{\hxi_2}$ \BY{assumption}
    \item $\tau=\tprod{\tau_1}{\tau_2}$ \BY{assumption}
    \item $\ctyp{\hxi_1}{\tau_1}$ \BY{assumption} \pflabel{[pair]c1Typ}
    \item $\ctyp{\hxi_2}{\tau_2}$ \BY{assumption} \pflabel{[pair]c2Typ}
    \end{pfsteps*}
    By rule induction over Rules (\ref{rules:TExp}) on \pfref{eTyp}, the following cases apply.
    \begin{byCases}
    \savelocalsteps{00}
    \item[\text{(\ref{rule:TEHole}),(\ref{rule:THole}),(\ref{rule:TAp}),(\ref{rule:TPrl}),(\ref{rule:TPrr}),(\ref{rule:TMatchZPre}),(\ref{rule:TMatchNZPre})}]
        \begin{pfsteps*}
        \item $e=\hehole{u},\hhole{e_0}{u},\hap{e_1}{e_2},\hprl{e_0},\hprr{e_0},\hmatch{e_0}{\zrules}$ \BY{assumption}
        \item $\notIntro{e}$ \BY{Rule (\ref{rule:NVEHole}),(\ref{rule:NVHole}),(\ref{rule:NVAp}),(\ref{rule:NVMatch}),(\ref{rule:NVPrl}),(\ref{rule:NVPrr})} \pflabel{[pair]notintro}
        \item $\isIndet{e}$ \BY{\autoref{lem:final-notintro-indet} on \pfref{eFinal} and \pfref{[pair]notintro}} \pflabel{[pair]indet}
        \item $\isIndet{\hprl{e}}$ \BY{Rule (\ref{rule:IPrl}) on \pfref{[pair]indet}} \pflabel{[pair]prl-indet}
        \item $\isFinal{\hprl{e}}$ \BY{Rule (\ref{rule:FIndet}) on \pfref{[pair]prl-indet}} \pflabel{[pair]prl-final}
        \item $\isIndet{\hprr{e}}$ \BY{Rule (\ref{rule:IPrr}) on \pfref{[pair]indet}} \pflabel{[pair]prr-indet}
        \item $\isFinal{\hprr{e}}$ \BY{Rule (\ref{rule:FIndet}) on \pfref{[pair]prr-indet}} \pflabel{[pair]prr-final}
        \item $\hexptyp{\cdot}{\Delta}{\hprl{e}}{\tau_1}$ \BY{Rule (\ref{rule:TPrl}) on \pfref{eTyp}} \pflabel{[pair]prl-typ}
        \item $\hexptyp{\cdot}{\Delta}{\hprr{e}}{\tau_2}$ \BY{Rule (\ref{rule:TPrr}) on \pfref{eTyp}} \pflabel{[pair]prr-typ}
        \end{pfsteps*}
        By inductive hypothesis on \pfref{[pair]c1Typ} and \pfref{[pair]prl-typ} and \pfref{[pair]prl-final}, exactly one of $\csatisfy{\hprl{e}}{\hxi_1}$, $\cmaysatisfy{\hprl{e}}{\hxi_1}$, and $\cnotsatisfyormay{\hprl{e}}{\hxi_1}$ holds. \\
        By inductive hypothesis on \pfref{[pair]c2Typ} and \pfref{[pair]prr-typ} and \pfref{[pair]prr-final}, exactly one of $\csatisfy{\hprr{e}}{\hxi_2}$, $\cmaysatisfy{\hprr{e}}{\hxi_2}$, and $\cnotsatisfyormay{\hprr{e}}{\hxi_2}$ holds. \\
        By case analysis on which conclusion holds for $\hxi_1$ and $\hxi_2$.
        \begin{byCases}
        \savelocalsteps{1}
        \item[\csatisfy{\hprl{e}}{\hxi_1},\csatisfy{\hprr{e}}{\hxi_2}]
            \begin{pfsteps*}
            \item $\csatisfy{\hprl{e}}{\hxi_1}$ \BY{assumption} \pflabel{[pair1]satisfy1}
            \item $\cnotmaysatisfy{\hprl{e}}{\hxi_1}$ \BY{assumption} \pflabel{[pair1]notmaysat1}
            \item $\csatisfy{\hprr{e}}{\hxi_2}$ \BY{assumption} \pflabel{[pair1]satisfy2}
            \item $\cnotmaysatisfy{\hprr{e}}{\hxi_2}$ \BY{assumption} \pflabel{[pair1]notmaysat2}
            \item $\csatisfy{e}{\cpair{\hxi_1}{\hxi_2}}$ \BY{Rule (\ref{rule:CSNotIntroPair}) on \pfref{[pair]notintro} and \pfref{[pair1]satisfy1} and \pfref{[pair1]satisfy2}} \pflabel{[pair1]satisfy-pair}
            \item $\csatisfyormay{e}{\cpair{\hxi_1}{\hxi_2}}$ \BY{\ruleref{rule:CSMSSat} on \pfref{[pair1]satisfy-pair}}
            \item $\cancel{\refutable{\cpair{\hxi_1}{\hxi_2}}}$ \BY{\autoref{lem:satisfy-not-refutable} on \pfref{[pair]notintro} and \pfref{[pair1]satisfy-pair}} \pflabel{[pair1]not-rft-pair}
            \end{pfsteps*}
            Assume $\cmaysatisfy{e}{\cpair{\hxi_1}{\hxi_2}}$. By rule induction over Rules (\ref{rules:MaySatisfy}) on it, only one case applies.
            \begin{byCases}
            \item[\text{(\ref{rule:CMSNotIntro})}]
                \begin{pfsteps*}
                \item $\refutable{\cpair{\hxi_1}{\hxi_2}}$ \BY{assumption}
                \end{pfsteps*}
                Contradicts \pfref{[pair1]not-rft-pair}.
            \end{byCases}
            \begin{pfsteps*}
            \item $\cnotmaysatisfy{e}{\cpair{\hxi_1}{\hxi_2}}$ \BY{contradiction}
            \end{pfsteps*}
            
        \restorelocalsteps{1}
        \item[\csatisfy{\hprl{e}}{\hxi_1},\cmaysatisfy{\hprr{e}}{\hxi_2}]
            \begin{pfsteps*}
            \item $\csatisfy{\hprl{e}}{\hxi_1}$ \BY{assumption} \pflabel{[pair2]satisfy1}
            \item $\cnotmaysatisfy{\hprl{e}}{\hxi_1}$ \BY{assumption} \pflabel{[pair2]notmaysat1}
            \item $\cnotsatisfy{\hprr{e}}{\hxi_2}$ \BY{assumption} \pflabel{[pair2]notsatisfy2}
            \item $\cmaysatisfy{\hprr{e}}{\hxi_2}$ \BY{assumption} \pflabel{[pair2]maysat2}
            \end{pfsteps*}
            Assume $\csatisfy{e}{\cpair{\hxi_1}{\hxi_2}}$. By rule induction over Rules (\ref{rules:Satisfy}), only one case applies.
            \begin{byCases}
            \item[\text{(\ref{rule:CSNotIntroPair})}]
                \begin{pfsteps*}
                \item $\csatisfy{\hprr{e}}{\hxi_2}$ \BY{assumption}
                \end{pfsteps*}
                Contradicts \pfref{[pair2]notsatisfy2}
            \end{byCases}
            \begin{pfsteps*}
            \item $\cnotsatisfy{e}{\cpair{\hxi_1}{\hxi_2}}$ \BY{contradiction} \pflabel{[pair2]notsatisfy-pair}
            \end{pfsteps*}
            By rule induction over Rules (\ref{rules:MaySatisfy}) on \pfref{[pair2]maysat2}, only one case applies. \todo{assume no "or" and "and" in pair}
            \begin{byCases}
            \item[\text{(\ref{rule:CMSNotIntro})}]
                \begin{pfsteps*}
                \item $\refutable{\hxi_2}$ \BY{assumption} \pflabel{[pair2]rft2}
                \item $\refutable{\cpair{\hxi_1}{\hxi_2}}$ \BY{\ruleref{rule:RXPairR} on \pfref{[pair2]rft2}} \pflabel{[pair2]rft-pair}
                \item $\cmaysatisfy{e}{\cpair{\hxi_1}{\hxi_2}}$ \BY{\ruleref{rule:CMSNotIntro} on \pfref{[pair]notintro} and \pfref{[pair2]rft-pair}} \pflabel{[pair2]maysat-pair}
                \item $\csatisfyormay{e}{\cpair{\hxi_1}{\hxi_2}}$ \BY{\ruleref{rule:CSMSMay} on \pfref{[pair2]maysat-pair}}
                \end{pfsteps*}
            \end{byCases}
            
        \restorelocalsteps{1}
        \item[\csatisfy{\hprl{e}}{\hxi_1},\cnotsatisfyormay{\hprr{e}}{\hxi_2}]
            \begin{pfsteps*}
            \item $\csatisfy{\hprl{e}}{\hxi_1}$ \BY{assumption} \pflabel{[pair3]satisfy1}
            \item $\cnotmaysatisfy{\hprl{e}}{\hxi_1}$ \BY{assumption} \pflabel{[pair3]notmaysat1}
            \item $\cnotsatisfy{\hprr{e}}{\hxi_2}$ \BY{assumption} \pflabel{[pair3]notsatisfy2}
            \item $\cnotmaysatisfy{\hprr{e}}{\hxi_2}$ \BY{assumption} \pflabel{[pair3]notmaysat2}
            \end{pfsteps*}
            Assume $\csatisfy{e}{\cpair{\hxi_1}{\hxi_2}}$. By rule induction over Rules (\ref{rules:Satisfy}) on it, only one case applies.
            \begin{byCases}
            \item[\text{(\ref{rule:CSNotIntroPair})}]
                \begin{pfsteps*}
                \item $\csatisfy{\hprr{e}}{\hxi_2}$ \BY{assumption}
                \end{pfsteps*}
                Contradicts \pfref{[pair3]notsatisfy2}.
            \end{byCases}
            \begin{pfsteps*}
            \item $\cnotsatisfy{e}{\cpair{\hxi_1}{\hxi_2}}$ \BY{contradiction} \pflabel{[pair3]notsat}
            \end{pfsteps*}
            Assume $\cmaysatisfy{e}{\cpair{\hxi_1}{\hxi_2}}$. By rule induction over Rules (\ref{rules:MaySatisfy}) on it, only one case applies.
            \begin{byCases}
            \item[\text{(\ref{rule:CMSNotIntro})}]
                \begin{pfsteps*}
                \item $\refutable{\cpair{\hxi_1}{\hxi_2}}$ \BY{assumption} \pflabel{[pair3]rft-pair}
                \end{pfsteps*}
                By rule induction over \rulesref{rules:xi-refutable} on \pfref{[pair3]rft-pair}, only two cases apply.
                \begin{byCases}
                \savelocalsteps{2}
                \item[\text{(\ref{rule:RXPairL})}]
                    \begin{pfsteps*}
                    \item $\refutable{\hxi_1}$ \BY{assumption} \pflabel{[pair3]rft1}
                    \item $\notIntro{\hprl{e}}$ \BY{\ruleref{rule:NVPrl}} \pflabel{[pair3]prl-notintro}
                    \item $\cmaysatisfy{\hprl{e}}{\hxi_1}$ \BY{\ruleref{rule:CMSNotIntro} on \pfref{[pair3]prl-notintro} and \pfref{[pair3]rft1}}
                    \end{pfsteps*}
                    Contradicts \pfref{[pair3]notmaysat1}.
                \restorelocalsteps{2}
                \item[\text{(\ref{rule:RXPairR})}]
                    \begin{pfsteps*}
                    \item $\refutable{\hxi_2}$ \BY{assumption} \pflabel{[pair3]rft2}
                    \item $\notIntro{\hprr{e}}$ \BY{\ruleref{rule:NVPrr}} \pflabel{[pair3]prr-notintro}
                    \item $\cmaysatisfy{\hprr{e}}{\hxi_2}$ \BY{\ruleref{rule:CMSNotIntro} on \pfref{[pair3]prr-notintro} and \pfref{[pair3]rft2}}
                    \end{pfsteps*}
                    Contradicts \pfref{[pair3]notmaysat2}.
                \end{byCases}
            \end{byCases}
            \begin{pfsteps*}
            \item $\cnotmaysatisfy{e}{\cpair{\hxi_1}{\hxi_2}}$ \BY{contradiction} \pflabel{[pair3]notmaysat}
            \item $\cnotsatisfyormay{e}{\cpair{\hxi_1}{\hxi_2}}$ \BY{\autoref{lem:not-satormay} on \pfref{[pair3]notsat} and \pfref{[pair3]notmaysat}}
            \end{pfsteps*}
            
        \restorelocalsteps{1}
        \item[\cmaysatisfy{\hprl{e}}{\hxi_1},\csatisfy{\hprr{e}}{\hxi_2}]
            \begin{pfsteps*}
            \item $\cnotsatisfy{\hprl{e}}{\hxi_1}$ \BY{assumption} \pflabel{[pair4]notsatisfy1}
            \item $\cmaysatisfy{\hprl{e}}{\hxi_1}$ \BY{assumption} \pflabel{[pair4]maysat1}
            \item $\csatisfy{\hprr{e}}{\hxi_2}$ \BY{assumption} \pflabel{[pair4]satisfy2}
            \item $\cnotmaysatisfy{\hprr{e}}{\hxi_2}$ \BY{assumption} \pflabel{[pair4]notmaysat2}
            \end{pfsteps*}
            Assume $\csatisfy{e}{\cpair{\hxi_1}{\hxi_2}}$. By rule induction over Rules (\ref{rules:Satisfy}), only one case applies.
            \begin{byCases}
            \item[\text{(\ref{rule:CSNotIntroPair})}]
                \begin{pfsteps*}
                \item $\csatisfy{\hprl{e}}{\hxi_1}$ \BY{assumption}
                \end{pfsteps*}
                Contradicts \pfref{[pair4]notsatisfy1}.
            \end{byCases}
            \begin{pfsteps*}
            \item $\cnotsatisfy{e}{\cpair{\hxi_1}{\hxi_2}}$ \BY{contradiction} \pflabel{[pair4]notsatisfy-pair}
            \end{pfsteps*}
            By rule induction over Rules (\ref{rules:MaySatisfy}) on \pfref{[pair4]maysat1}, only one case applies. \todo{assume no "or" and "and" in pair}
            \begin{byCases}
            \item[\text{(\ref{rule:CMSNotIntro})}]
                \begin{pfsteps*}
                \item $\refutable{\hxi_1}$ \BY{assumption} \pflabel{[pair4]rft1}
                \item $\refutable{\cpair{\hxi_1}{\hxi_2}}$ \BY{\ruleref{rule:RXPairR} on \pfref{[pair4]rft1}} \pflabel{[pair4]rft-pair}
                \item $\cmaysatisfy{e}{\cpair{\hxi_1}{\hxi_2}}$ \BY{\ruleref{rule:CMSNotIntro} on \pfref{[pair]notintro} and \pfref{[pair4]rft-pair}} \pflabel{[pair4]maysat-pair}
                \item $\csatisfyormay{e}{\cpair{\hxi_1}{\hxi_2}}$ \BY{\ruleref{rule:CSMSMay} on \pfref{[pair4]maysat-pair}}
                \end{pfsteps*}
            \end{byCases}
            
        \restorelocalsteps{1}
        \item[\cmaysatisfy{\hprl{e}}{\hxi_1},\cmaysatisfy{\hprr{e}}{\hxi_2}]
            \begin{pfsteps*}
            \item $\cnotsatisfy{\hprl{e}}{\hxi_1}$ \BY{assumption} \pflabel{[pair5]notsatisfy1}
            \item $\cmaysatisfy{\hprl{e}}{\hxi_1}$ \BY{assumption} \pflabel{[pair5]maysat1}
            \item $\cnotsatisfy{\hprr{e}}{\hxi_2}$ \BY{assumption} \pflabel{[pair5]notsatisfy2}
            \item $\cmaysatisfy{\hprr{e}}{\hxi_2}$ \BY{assumption} \pflabel{[pair5]maysat2}
            \end{pfsteps*}
            Assume $\csatisfy{e}{\cpair{\hxi_1}{\hxi_2}}$. By rule induction over Rules (\ref{rules:Satisfy}), only one case applies.
            \begin{byCases}
            \item[\text{(\ref{rule:CSNotIntroPair})}]
                \begin{pfsteps*}
                \item $\csatisfy{\hprl{e}}{\hxi_1}$ \BY{assumption}
                \end{pfsteps*}
                Contradicts \pfref{[pair5]notsatisfy1}.
            \end{byCases}
            \begin{pfsteps*}
            \item $\cnotsatisfy{e}{\cpair{\hxi_1}{\hxi_2}}$ \BY{contradiction} \pflabel{[pair5]notsatisfy-pair}
            \end{pfsteps*}
            By rule induction over Rules (\ref{rules:MaySatisfy}) on \pfref{[pair5]maysat2}, only one case applies. \todo{assume no "or" and "and" in pair}
            \begin{byCases}
            \item[\text{(\ref{rule:CMSNotIntro})}]
                \begin{pfsteps*}
                \item $\refutable{\hxi_2}$ \BY{assumption} \pflabel{[pair5]rft2}
                \item $\refutable{\cpair{\hxi_1}{\hxi_2}}$ \BY{\ruleref{rule:RXPairR} on \pfref{[pair5]rft2}} \pflabel{[pair5]rft-pair}
                \item $\cmaysatisfy{e}{\cpair{\hxi_1}{\hxi_2}}$ \BY{\ruleref{rule:CMSNotIntro} on \pfref{[pair]notintro} and \pfref{[pair5]rft-pair}} \pflabel{[pair5]maysat-pair}
                \item $\csatisfyormay{e}{\cpair{\hxi_1}{\hxi_2}}$ \BY{\ruleref{rule:CSMSMay} on \pfref{[pair5]maysat-pair}}
                \end{pfsteps*}
            \end{byCases}
        \restorelocalsteps{1}
        \item[\cmaysatisfy{\hprl{e}}{\hxi_1},\cnotsatisfyormay{\hprr{e}}{\hxi_2}]
            \begin{pfsteps*}
            \item $\cnotsatisfy{\hprl{e}}{\hxi_1}$ \BY{assumption} \pflabel{[pair6]notsatisfy1}
            \item $\cmaysatisfy{\hprl{e}}{\hxi_1}$ \BY{assumption} \pflabel{[pair6]maysat1}
            \item $\cnotsatisfy{\hprr{e}}{\hxi_2}$ \BY{assumption} \pflabel{[pair6]notsatisfy2}
            \item $\cnotmaysatisfy{\hprr{e}}{\hxi_2}$ \BY{assumption} \pflabel{[pair6]notmaysat2}
            \end{pfsteps*}
            Assume $\csatisfy{e}{\cpair{\hxi_1}{\hxi_2}}$. By rule induction over Rules (\ref{rules:Satisfy}), only one case applies.
            \begin{byCases}
            \item[\text{(\ref{rule:CSNotIntroPair})}]
                \begin{pfsteps*}
                \item $\csatisfy{\hprl{e}}{\hxi_1}$ \BY{assumption}
                \end{pfsteps*}
                Contradicts \pfref{[pair6]notsatisfy1}
            \end{byCases}
            \begin{pfsteps*}
            \item $\cnotsatisfy{e}{\cpair{\hxi_1}{\hxi_2}}$ \BY{contradiction} \pflabel{[pair6]notsatisfy-pair}
            \end{pfsteps*}
            By rule induction over Rules (\ref{rules:MaySatisfy}) on \pfref{[pair6]maysat1}, only one case applies. \todo{assume no "or" and "and" in pair}
            \begin{byCases}
            \item[\text{(\ref{rule:CMSNotIntro})}]
                \begin{pfsteps*}
                \item $\refutable{\hxi_1}$ \BY{assumption} \pflabel{[pair6]rft1}
                \item $\refutable{\cpair{\hxi_1}{\hxi_2}}$ \BY{\ruleref{rule:RXPairR} on \pfref{[pair6]rft1}} \pflabel{[pair6]rft-pair}
                \item $\cmaysatisfy{e}{\cpair{\hxi_1}{\hxi_2}}$ \BY{\ruleref{rule:CMSNotIntro} on \pfref{[pair]notintro} and \pfref{[pair6]rft-pair}} \pflabel{[pair6]maysat-pair}
                \item $\csatisfyormay{e}{\cpair{\hxi_1}{\hxi_2}}$ \BY{\ruleref{rule:CSMSMay} on \pfref{[pair6]maysat-pair}}
                \end{pfsteps*}
            \end{byCases}
        \restorelocalsteps{1}
        \item[\cnotsatisfyormay{\hprl{e}}{\hxi_1},\csatisfy{\hprr{e}}{\hxi_2}]
            \begin{pfsteps*}
            \item $\cnotsatisfy{\hprl{e}}{\hxi_1}$ \BY{assumption} \pflabel{[pair7]notsatisfy1}
            \item $\cnotmaysatisfy{\hprl{e}}{\hxi_1}$ \BY{assumption} \pflabel{[pair7]notmaysat1}
            \item $\csatisfy{\hprr{e}}{\hxi_2}$ \BY{assumption} \pflabel{[pair7]satisfy2}
            \item $\cnotmaysatisfy{\hprr{e}}{\hxi_2}$ \BY{assumption} \pflabel{[pair7]notmaysat2}
            \end{pfsteps*}
            Assume $\csatisfy{e}{\cpair{\hxi_1}{\hxi_2}}$. By rule induction over Rules (\ref{rules:Satisfy}) on it, only one case applies.
            \begin{byCases}
            \item[\text{(\ref{rule:CSNotIntroPair})}]
                \begin{pfsteps*}
                \item $\csatisfy{\hprl{e}}{\hxi_1}$ \BY{assumption}
                \end{pfsteps*}
                Contradicts \pfref{[pair7]notsatisfy1}
            \end{byCases}
            \begin{pfsteps*}
            \item $\cnotsatisfy{e}{\cpair{\hxi_1}{\hxi_2}}$ \BY{contradiction} \pflabel{[pair7]notsat}
            \end{pfsteps*}
            Assume $\cmaysatisfy{e}{\cpair{\hxi_1}{\hxi_2}}$. By rule induction over Rules (\ref{rules:MaySatisfy}) on it, only one case applies.
            \begin{byCases}
            \item[\text{(\ref{rule:CMSNotIntro})}]
                \begin{pfsteps*}
                \item $\refutable{\cpair{\hxi_1}{\hxi_2}}$ \BY{assumption} \pflabel{[pair7]rft-pair}
                \end{pfsteps*}
                By rule induction over \rulesref{rules:xi-refutable} on \pfref{[pair7]rft-pair}, only two cases apply.
                \begin{byCases}
                \savelocalsteps{2}
                \item[\text{(\ref{rule:RXPairL})}]
                    \begin{pfsteps*}
                    \item $\refutable{\hxi_1}$ \BY{assumption} \pflabel{[pair7]rft1}
                    \item $\notIntro{\hprl{e}}$ \BY{\ruleref{rule:NVPrl}} \pflabel{[pair7]prl-notintro}
                    \item $\cmaysatisfy{\hprl{e}}{\hxi_1}$ \BY{\ruleref{rule:CMSNotIntro} on \pfref{[pair7]prl-notintro} and \pfref{[pair7]rft1}}
                    \end{pfsteps*}
                    Contradicts \pfref{[pair3]notmaysat1}.
                \restorelocalsteps{2}
                \item[\text{(\ref{rule:RXPairR})}]
                    \begin{pfsteps*}
                    \item $\refutable{\hxi_2}$ \BY{assumption} \pflabel{[pair7]rft2}
                    \item $\notIntro{\hprr{e}}$ \BY{\ruleref{rule:NVPrr}} \pflabel{[pair7]prr-notintro}
                    \item $\cmaysatisfy{\hprr{e}}{\hxi_2}$ \BY{\ruleref{rule:CMSNotIntro} on \pfref{[pair7]prr-notintro} and \pfref{[pair7]rft2}}
                    \end{pfsteps*}
                    Contradicts \pfref{[pair7]notmaysat2}.
                \end{byCases}
            \end{byCases}
            \begin{pfsteps*}
            \item $\cnotmaysatisfy{e}{\cpair{\hxi_1}{\hxi_2}}$ \BY{contradiction} \pflabel{[pair7]notmaysat}
            \item $\cnotsatisfyormay{e}{\cpair{\hxi_1}{\hxi_2}}$ \BY{\autoref{lem:not-satormay} on \pfref{[pair7]notsat} and \pfref{[pair7]notmaysat}}
            \end{pfsteps*}
            
        \restorelocalsteps{1}
        \item[\cnotsatisfyormay{\hprl{e}}{\hxi_1},\cmaysatisfy{\hprr{e}}{\hxi_2}]
            \begin{pfsteps*}
            \item $\cnotsatisfy{\hprl{e}}{\hxi_1}$ \BY{assumption} \pflabel{[pair8]notsatisfy1}
            \item $\cnotmaysatisfy{\hprl{e}}{\hxi_1}$ \BY{assumption} \pflabel{[pair8]notmaysat1}
            \item $\cnotsatisfy{\hprr{e}}{\hxi_2}$ \BY{assumption} \pflabel{[pair8]notsatisfy2}
            \item $\cmaysatisfy{\hprr{e}}{\hxi_2}$ \BY{assumption} \pflabel{[pair8]maysat2}
            \end{pfsteps*}
            Assume $\csatisfy{e}{\cpair{\hxi_1}{\hxi_2}}$. By rule induction over Rules (\ref{rules:Satisfy}), only one case applies.
            \begin{byCases}
            \item[\text{(\ref{rule:CSNotIntroPair})}]
                \begin{pfsteps*}
                \item $\csatisfy{\hprl{e}}{\hxi_1}$ \BY{assumption}
                \end{pfsteps*}
                Contradicts \pfref{[pair8]notsatisfy1}.
            \end{byCases}
            \begin{pfsteps*}
            \item $\cnotsatisfy{e}{\cpair{\hxi_1}{\hxi_2}}$ \BY{contradiction} \pflabel{[pair8]notsatisfy-pair}
            \end{pfsteps*}
            By rule induction over Rules (\ref{rules:MaySatisfy}) on \pfref{[pair8]maysat2}, only one case applies. \todo{assume no "or" and "and" in pair}
            \begin{byCases}
            \item[\text{(\ref{rule:CMSNotIntro})}]
                \begin{pfsteps*}
                \item $\refutable{\hxi_2}$ \BY{assumption} \pflabel{[pair8]rft2}
                \item $\refutable{\cpair{\hxi_1}{\hxi_2}}$ \BY{\ruleref{rule:RXPairR} on \pfref{[pair8]rft2}} \pflabel{[pair8]rft-pair}
                \item $\cmaysatisfy{e}{\cpair{\hxi_1}{\hxi_2}}$ \BY{\ruleref{rule:CMSNotIntro} on \pfref{[pair]notintro} and \pfref{[pair8]rft-pair}} \pflabel{[pair8]maysat-pair}
                \item $\csatisfyormay{e}{\cpair{\hxi_1}{\hxi_2}}$ \BY{\ruleref{rule:CSMSMay} on \pfref{[pair8]maysat-pair}}
                \end{pfsteps*}
            \end{byCases}
        \restorelocalsteps{1}
        \item[\cnotsatisfyormay{\hprl{e}}{\hxi_1},\cnotsatisfyormay{\hprr{e}}{\hxi_2}]
            \begin{pfsteps*}
            \item $\cnotsatisfy{\hprl{e}}{\hxi_1}$ \BY{assumption} \pflabel{[pair9]notsatisfy1}
            \item $\cnotmaysatisfy{\hprl{e}}{\hxi_1}$ \BY{assumption} \pflabel{[pair9]notmaysat1}
            \item $\cnotsatisfy{\hprr{e}}{\hxi_2}$ \BY{assumption} \pflabel{[pair9]notsatisfy2}
            \item $\cnotmaysatisfy{\hprr{e}}{\hxi_2}$ \BY{assumption} \pflabel{[pair9]notmaysat2}
            \end{pfsteps*}
            Assume $\csatisfy{e}{\cpair{\hxi_1}{\hxi_2}}$. By rule induction over Rules (\ref{rules:Satisfy}) on it, only one case applies.
            \begin{byCases}
            \item[\text{(\ref{rule:CSNotIntroPair})}]
                \begin{pfsteps*}
                \item $\csatisfy{\hprl{e}}{\hxi_1}$ \BY{assumption}
                \end{pfsteps*}
                Contradicts \pfref{[pair9]notsatisfy1}
            \end{byCases}
            \begin{pfsteps*}
            \item $\cnotsatisfy{e}{\cpair{\hxi_1}{\hxi_2}}$ \BY{contradiction} \pflabel{[pair9]notsat}
            \end{pfsteps*}
            Assume $\cmaysatisfy{e}{\cpair{\hxi_1}{\hxi_2}}$. By rule induction over Rules (\ref{rules:MaySatisfy}) on it, only one case applies.
            \begin{byCases}
            \item[\text{(\ref{rule:CMSNotIntro})}]
                \begin{pfsteps*}
                \item $\refutable{\cpair{\hxi_1}{\hxi_2}}$ \BY{assumption} \pflabel{[pair9]rft-pair}
                \end{pfsteps*}
                By rule induction over \rulesref{rules:xi-refutable} on \pfref{[pair9]rft-pair}, only two cases apply.
                \begin{byCases}
                \savelocalsteps{2}
                \item[\text{(\ref{rule:RXPairL})}]
                    \begin{pfsteps*}
                    \item $\refutable{\hxi_1}$ \BY{assumption} \pflabel{[pair9]rft1}
                    \item $\notIntro{\hprl{e}}$ \BY{\ruleref{rule:NVPrl}} \pflabel{[pair9]prl-notintro}
                    \item $\cmaysatisfy{\hprl{e}}{\hxi_1}$ \BY{\ruleref{rule:CMSNotIntro} on \pfref{[pair9]prl-notintro} and \pfref{[pair9]rft1}}
                    \end{pfsteps*}
                    Contradicts \pfref{[pair9]notmaysat1}.
                \restorelocalsteps{2}
                \item[\text{(\ref{rule:RXPairR})}]
                    \begin{pfsteps*}
                    \item $\refutable{\hxi_2}$ \BY{assumption} \pflabel{[pair9]rft2}
                    \item $\notIntro{\hprr{e}}$ \BY{\ruleref{rule:NVPrr}} \pflabel{[pair9]prr-notintro}
                    \item $\cmaysatisfy{\hprr{e}}{\hxi_2}$ \BY{\ruleref{rule:CMSNotIntro} on \pfref{[pair9]prr-notintro} and \pfref{[pair9]rft2}}
                    \end{pfsteps*}
                    Contradicts \pfref{[pair9]notmaysat2}.
                \end{byCases}
            \end{byCases}
            \begin{pfsteps*}
            \item $\cnotmaysatisfy{e}{\cpair{\hxi_1}{\hxi_2}}$ \BY{contradiction} \pflabel{[pair9]notmaysat}
            \item $\cnotsatisfyormay{e}{\cpair{\hxi_1}{\hxi_2}}$ \BY{\autoref{lem:not-satormay} on \pfref{[pair9]notsat} and \pfref{[pair9]notmaysat}}
            \end{pfsteps*}
            
        \end{byCases}
    \restorelocalsteps{00}
    \item[\text{(\ref{rule:TPair})}]
        \begin{pfsteps*}
        \item $e=\hpair{e_1}{e_2}$ \BY{assumption}
        \item $\hexptyp{\cdot}{\Delta}{e_1}{\tau_1}$ \BY{assumption} \pflabel{[epair]e1-typ}
        \item $\hexptyp{\cdot}{\Delta}{e_2}{\tau_2}$ \BY{assumption} \pflabel{[epair]e2-typ}
        \item $\isFinal{e_1}$ \BY{\autoref{lem:pair-final} on \pfref{eFinal}} \pflabel{[epair]e1-final}
        \item $\isFinal{e_2}$ \BY{\autoref{lem:pair-final} on \pfref{eFinal}} \pflabel{[epair]e2-final}
        \end{pfsteps*}
        By inductive hypothesis on \pfref{[pair]c1Typ} and \pfref{[epair]e1-typ} and \pfref{[epair]e1-final}, exactly one of $\csatisfy{e_1}{\hxi_1}$, $\cmaysatisfy{e_1}{\hxi_1}$, and $\csatisfy{e_1}{\cdual{\hxi_1}}$ holds. \\
        By inductive hypothesis on \pfref{[pair]c2Typ} and \pfref{[epair]e2-typ} and \pfref{[epair]e2-final}, exactly one of $\csatisfy{e_2}{\hxi_2}$, $\cmaysatisfy{e_2}{\hxi_2}$, and $\csatisfy{e_2}{\cdual{\hxi_2}}$ holds. \\
        By case analysis on which conclusion holds for $\hxi_1$ and $\hxi_2$.
        \begin{byCases}
        \savelocalsteps{1}
        \item[\csatisfy{e_1}{\hxi_1},\csatisfy{e_2}{\hxi_2}]
            \begin{pfsteps*}
            \item $\csatisfy{e_1}{\hxi_1}$ \BY{assumption} \pflabel{[epair1]satisfy1}
            \item $\cnotmaysatisfy{e_1}{\hxi_1}$ \BY{assumption} \pflabel{[epair1]notmaysat1}
            \item $\csatisfy{e_2}{\hxi_2}$ \BY{assumption} \pflabel{[epair1]satisfy2}
            \item $\cnotmaysatisfy{e_2}{\hxi_2}$ \BY{assumption} \pflabel{[epair1]notmaysat2}
            \item $\csatisfy{\hpair{e_1}{e_2}}{\cpair{\hxi_1}{\hxi_2}}$ \BY{Rule (\ref{rule:CSPair}) on \pfref{[epair1]satisfy1} and \pfref{[epair1]satisfy2}} \pflabel{[epair1]satisfy}
            \item $\csatisfyormay{\hpair{e_1}{e_2}}{\cpair{\hxi_1}{\hxi_2}}$ \BY{\ruleref{rule:CSMSSat} on \pfref{[epair1]satisfy}}
            \end{pfsteps*}
            Assume $\cmaysatisfy{\hpair{e_1}{e_2}}{\cpair{\hxi_1}{\hxi_2}}$. By rule induction over Rules (\ref{rules:MaySatisfy}) on it, the following cases apply.
            \begin{byCases}
            \savelocalsteps{2}
            \item[\text{(\ref{rule:CMSNotIntro})}]
                \begin{pfsteps*}
                \item $\notIntro{\hpair{e_1}{e_2}}$ \BY{assumption}
                \end{pfsteps*}
                Contradicts \autoref{lem:no-pair-notintro}.
            \restorelocalsteps{2}
            \item[\text{(\ref{rule:CMSPair1})}]
                \begin{pfsteps*}
                \item $\cmaysatisfy{e_1}{\hxi_1}$ \BY{assumption}
                \end{pfsteps*}
                Contradicts \pfref{[epair1]notmaysat1}.
            \restorelocalsteps{2}
            \item[\text{(\ref{rule:CMSPair2})}]
                \begin{pfsteps*}
                \item $\cmaysatisfy{e_2}{\hxi_2}$ \BY{assumption}
                \end{pfsteps*}
                Contradicts \pfref{[epair1]notmaysat2}.
            \restorelocalsteps{2}
            \item[\text{(\ref{rule:CMSPair3})}]
                \begin{pfsteps*}
                \item $\cmaysatisfy{e_1}{\hxi_1}$ \BY{assumption}
                \end{pfsteps*}
                Contradicts \pfref{[epair1]notmaysat1}.
            \end{byCases}
            \begin{pfsteps*}
            \item $\cnotmaysatisfy{\hpair{e_1}{e_2}}{\cpair{\hxi_1}{\hxi_2}}$ \BY{contradiction}
            \end{pfsteps*}
            
        \restorelocalsteps{1}
        \item[\csatisfy{e_1}{\hxi_1},\cmaysatisfy{e_2}{\hxi_2}]
            \begin{pfsteps*}
            \item $\csatisfy{e_1}{\hxi_1}$ \BY{assumption} \pflabel{[epair2]satisfy1}
            \item $\cnotmaysatisfy{e_1}{\hxi_1}$ \BY{assumption} \pflabel{[epair2]notmaysat1}
            \item $\cnotsatisfy{e_2}{\hxi_2}$ \BY{assumption} \pflabel{[epair2]notsatisfy2}
            \item $\cmaysatisfy{e_2}{\hxi_2}$ \BY{assumption} \pflabel{[epair2]maysat2}
            \item $\cmaysatisfy{\hpair{e_1}{e_2}}{\cpair{\hxi_1}{\hxi_2}}$ \BY{Rule (\ref{rule:CMSPair2}) on \pfref{[epair2]satisfy1} and \pfref{[epair2]maysat2}} \pflabel{[epair2]maysat}
            \item $\csatisfyormay{\hpair{e_1}{e_2}}{\cpair{\hxi_1}{\hxi_2}}$ \BY{\ruleref{rule:CSMSMay} on \pfref{[epair2]maysat}}
            \end{pfsteps*}
            Assume $\csatisfy{\hpair{e_1}{e_2}}{\cpair{\hxi_1}{\hxi_2}}$. By rule induction over Rules (\ref{rules:Satisfy}) on it, only two cases apply. 
           \begin{byCases}
            \savelocalsteps{2}
            \item[\text{(\ref{rule:CSNotIntroPair})}]
                \begin{pfsteps*}
                \item $\notIntro{\hpair{e_1}{e_2}}$ \BY{assumption}
                \end{pfsteps*}
                Contradicts \autoref{lem:no-pair-notintro}.
            \restorelocalsteps{2}
            \item[\text{(\ref{rule:CSPair})}]
                \begin{pfsteps*}
                \item $\csatisfy{e_2}{\hxi_2}$ \BY{assumption}
                \end{pfsteps*}
                Contradicts \pfref{[epair2]notsatisfy2}.
            \end{byCases}
            \begin{pfsteps*}
            \item $\cnotsatisfy{\hpair{e_1}{e_2}}{\cpair{\hxi_1}{\hxi_2}}$ \BY{contradiction}
            \end{pfsteps*}
            
        \restorelocalsteps{1}
        \item[\csatisfy{e_1}{\hxi_1},\cnotsatisfyormay{e_2}{\hxi_2}]
            \begin{pfsteps*}
            \item $\csatisfy{e_1}{\hxi_1}$ \BY{assumption} \pflabel{[epair3]satisfy1}
            \item $\cnotmaysatisfy{e_1}{\hxi_1}$ \BY{assumption} \pflabel{[epair3]notmaysat1}
            \item $\cnotsatisfy{e_2}{\hxi_2}$ \BY{assumption} \pflabel{[epair3]notsatisfy2}
            \item $\cnotmaysatisfy{e_2}{\hxi_2}$ \BY{assumption} \pflabel{[epair3]notmaysat2}
            \end{pfsteps*}
            Assume $\csatisfy{\hpair{e_1}{e_2}}{\cpair{\hxi_1}{\hxi_2}}$. By rule induction over Rules (\ref{rules:Satisfy}) on it, only two cases apply. 
           \begin{byCases}
            \savelocalsteps{2}
            \item[\text{(\ref{rule:CSNotIntroPair})}]
                \begin{pfsteps*}
                \item $\notIntro{\hpair{e_1}{e_2}}$ \BY{assumption}
                \end{pfsteps*}
                Contradicts \autoref{lem:no-pair-notintro}.
            \restorelocalsteps{2}
            \item[\text{(\ref{rule:CSPair})}]
                \begin{pfsteps*}
                \item $\csatisfy{e_2}{\hxi_2}$ \BY{assumption}
                \end{pfsteps*}
                Contradicts \pfref{[epair3]notsatisfy2}.
            \end{byCases}
            \begin{pfsteps*}
            \item $\cnotsatisfy{\hpair{e_1}{e_2}}{\cpair{\hxi_1}{\hxi_2}}$ \BY{contradiction} \pflabel{[epair3]notsat}
            \end{pfsteps*}
            Assume $\cmaysatisfy{\hpair{e_1}{e_2}}{\cpair{\hxi_1}{\hxi_2}}$. By rule induction over Rules (\ref{rules:MaySatisfy}) on it, the following cases apply.
            \begin{byCases}
            \savelocalsteps{2}
            \item[\text{(\ref{rule:CMSNotIntro})}]
                \begin{pfsteps*}
                \item $\notIntro{\hpair{e_1}{e_2}}$ \BY{assumption}
                \end{pfsteps*}
                Contradicts \autoref{lem:no-pair-notintro}.
            \restorelocalsteps{2}
            \item[\text{(\ref{rule:CMSPair1})}]
                \begin{pfsteps*}
                \item $\cmaysatisfy{e_1}{\hxi_1}$ \BY{assumption}
                \end{pfsteps*}
                Contradicts \pfref{[epair3]notmaysat1}.
            \restorelocalsteps{2}
            \item[\text{(\ref{rule:CMSPair2})}]
                \begin{pfsteps*}
                \item $\cmaysatisfy{e_2}{\hxi_2}$ \BY{assumption}
                \end{pfsteps*}
                Contradicts \pfref{[epair3]notmaysat2}.
            \restorelocalsteps{2}
            \item[\text{(\ref{rule:CMSPair3})}]
                \begin{pfsteps*}
                \item $\cmaysatisfy{e_1}{\hxi_1}$ \BY{assumption}
                \end{pfsteps*}
                Contradicts \pfref{[epair3]notmaysat1}.
            \end{byCases}
            \begin{pfsteps*}
            \item $\cnotmaysatisfy{\hpair{e_1}{e_2}}{\cpair{\hxi_1}{\hxi_2}}$ \BY{contradiction} \pflabel{[epair3]notmaysat}
            \item $\cnotsatisfyormay{\hpair{e_1}{e_2}}{\cpair{\hxi_1}{\hxi_2}}$ \BY{\autoref{lem:not-satormay} on \pfref{[epair3]notsat} and \pfref{[epair3]notmaysat}}
            \end{pfsteps*}
        \restorelocalsteps{1}
        \item[\cmaysatisfy{e_1}{\hxi_1},\csatisfy{e_2}{\hxi_2}]
            \begin{pfsteps*}
            \item $\cnotsatisfy{e_1}{\hxi_1}$ \BY{assumption} \pflabel{[epair4]notsatisfy1}
            \item $\cmaysatisfy{e_1}{\hxi_1}$ \BY{assumption} \pflabel{[epair4]maysat1}
            \item $\csatisfy{e_2}{\hxi_2}$ \BY{assumption} \pflabel{[epair4]satisfy2}
            \item $\cnotmaysatisfy{e_2}{\hxi_2}$ \BY{assumption} \pflabel{[epair4]notmaysat2}
            \item $\cmaysatisfy{\hpair{e_1}{e_2}}{\cpair{\hxi_1}{\hxi_2}}$ \BY{Rule (\ref{rule:CMSPair1}) on \pfref{[epair4]maysat1} and \pfref{[epair4]satisfy2}} \pflabel{[epair4]maysat}
            \item $\csatisfyormay{\hpair{e_1}{e_2}}{\cpair{\hxi_1}{\hxi_2}}$ \BY{\ruleref{rule:CSMSMay} on \pfref{[epair4]maysat}}
            \end{pfsteps*}
            Assume $\csatisfy{\hpair{e_1}{e_2}}{\cpair{\hxi_1}{\hxi_2}}$. By rule induction over Rules (\ref{rules:Satisfy}) on it, only two cases apply. 
           \begin{byCases}
            \savelocalsteps{2}
            \item[\text{(\ref{rule:CSNotIntroPair})}]
                \begin{pfsteps*}
                \item $\notIntro{\hpair{e_1}{e_2}}$ \BY{assumption}
                \end{pfsteps*}
                Contradicts \autoref{lem:no-pair-notintro}.
            \restorelocalsteps{2}
            \item[\text{(\ref{rule:CSPair})}]
                \begin{pfsteps*}
                \item $\csatisfy{e_1}{\hxi_1}$ \BY{assumption}
                \end{pfsteps*}
                Contradicts \pfref{[epair4]notsatisfy1}.
            \end{byCases}
            \begin{pfsteps*}
            \item $\cnotsatisfy{\hpair{e_1}{e_2}}{\cpair{\hxi_1}{\hxi_2}}$ \BY{contradiction} \pflabel{[epair4]notsat}
            \end{pfsteps*}
            
        \restorelocalsteps{1}
        \item[\cmaysatisfy{e_1}{\hxi_1},\cmaysatisfy{e_2}{\hxi_2}]
            \begin{pfsteps*}
            \item $\cnotsatisfy{e_1}{\hxi_1}$ \BY{assumption} \pflabel{[epair5]notsatisfy1}
            \item $\cmaysatisfy{e_1}{\hxi_1}$ \BY{assumption} \pflabel{[epair5]maysat1}
            \item $\cnotsatisfy{e_2}{\hxi_2}$ \BY{assumption} \pflabel{[epair5]notsatisfy2}
            \item $\cmaysatisfy{e_2}{\hxi_2}$ \BY{assumption} \pflabel{[epair5]maysat2}
            \item $\cmaysatisfy{\hpair{e_1}{e_2}}{\cpair{\hxi_1}{\hxi_2}}$ \BY{Rule (\ref{rule:CMSPair3}) on \pfref{[epair5]maysat1} and \pfref{[epair5]maysat2}} \pflabel{[epair5]maysat}
            \item $\csatisfyormay{\hpair{e_1}{e_2}}{\cpair{\hxi_1}{\hxi_2}}$ \BY{\ruleref{rule:CSMSMay} on \pfref{[epair5]maysat}}
            \end{pfsteps*}
            Assume $\csatisfy{\hpair{e_1}{e_2}}{\cpair{\hxi_1}{\hxi_2}}$. By rule induction over Rules (\ref{rules:Satisfy}) on it, only two cases apply. 
           \begin{byCases}
            \savelocalsteps{2}
            \item[\text{(\ref{rule:CSNotIntroPair})}]
                \begin{pfsteps*}
                \item $\notIntro{\hpair{e_1}{e_2}}$ \BY{assumption}
                \end{pfsteps*}
                Contradicts \autoref{lem:no-pair-notintro}.
            \restorelocalsteps{2}
            \item[\text{(\ref{rule:CSPair})}]
                \begin{pfsteps*}
                \item $\csatisfy{e_1}{\hxi_1}$ \BY{assumption}
                \end{pfsteps*}
                Contradicts \pfref{[epair5]notsatisfy1}.
            \end{byCases}
            \begin{pfsteps*}
            \item $\cnotsatisfy{\hpair{e_1}{e_2}}{\cpair{\hxi_1}{\hxi_2}}$ \BY{contradiction} \pflabel{[epair5]notsat}
            \end{pfsteps*}
            
        \restorelocalsteps{1}
        \item[\cmaysatisfy{e_1}{\hxi_1},\cnotsatisfyormay{e_2}{\hxi_2}]
            \begin{pfsteps*}
            \item $\cnotsatisfy{e_1}{\hxi_1}$ \BY{assumption} \pflabel{[epair6]notsatisfy1}
            \item $\cmaysatisfy{e_1}{\hxi_1}$ \BY{assumption} \pflabel{[epair6]maysat1}
            \item $\cnotsatisfy{e_2}{\hxi_2}$ \BY{assumption} \pflabel{[epair6]notsatisfy2}
            \item $\cnotmaysatisfy{e_2}{\hxi_2}$ \BY{assumption} \pflabel{[epair6]notmaysat2}
            \end{pfsteps*}
            Assume $\csatisfy{\hpair{e_1}{e_2}}{\cpair{\hxi_1}{\hxi_2}}$. By rule induction over Rules (\ref{rules:Satisfy}) on it, only two cases apply. 
           \begin{byCases}
            \savelocalsteps{2}
            \item[\text{(\ref{rule:CSNotIntroPair})}]
                \begin{pfsteps*}
                \item $\notIntro{\hpair{e_1}{e_2}}$ \BY{assumption}
                \end{pfsteps*}
                Contradicts \autoref{lem:no-pair-notintro}.
            \restorelocalsteps{2}
            \item[\text{(\ref{rule:CSPair})}]
                \begin{pfsteps*}
                \item $\csatisfy{e_1}{\hxi_1}$ \BY{assumption}
                \end{pfsteps*}
                Contradicts \pfref{[epair6]notsatisfy1}.
            \end{byCases}
            \begin{pfsteps*}
            \item $\cnotsatisfy{\hpair{e_1}{e_2}}{\cpair{\hxi_1}{\hxi_2}}$ \BY{contradiction} \pflabel{[epair6]notsat}
            \end{pfsteps*}
            Assume $\cmaysatisfy{\hpair{e_1}{e_2}}{\cpair{\hxi_1}{\hxi_2}}$. By rule induction over Rules (\ref{rules:MaySatisfy}) on it, the following cases apply.
            \begin{byCases}
            \savelocalsteps{2}
            \item[\text{(\ref{rule:CMSNotIntro})}]
                \begin{pfsteps*}
                \item $\notIntro{\hpair{e_1}{e_2}}$ \BY{assumption}
                \end{pfsteps*}
                Contradicts \autoref{lem:no-pair-notintro}.
            \restorelocalsteps{2}
            \item[\text{(\ref{rule:CMSPair1})}]
                \begin{pfsteps*}
                \item $\csatisfy{e_2}{\hxi_2}$ \BY{assumption}
                \end{pfsteps*}
                Contradicts \pfref{[epair6]notsatisfy2}.
            \restorelocalsteps{2}
            \item[\text{(\ref{rule:CMSPair2})}]
                \begin{pfsteps*}
                \item $\cmaysatisfy{e_2}{\hxi_2}$ \BY{assumption}
                \end{pfsteps*}
                Contradicts \pfref{[epair6]notmaysat2}.
            \restorelocalsteps{2}
            \item[\text{(\ref{rule:CMSPair3})}]
                \begin{pfsteps*}
                \item $\cmaysatisfy{e_2}{\hxi_2}$ \BY{assumption}
                \end{pfsteps*}
                Contradicts \pfref{[epair6]notmaysat2}.
            \end{byCases}
            \begin{pfsteps*}
            \item $\cnotmaysatisfy{\hpair{e_1}{e_2}}{\cpair{\hxi_1}{\hxi_2}}$ \BY{contradiction} \pflabel{[epair6]notmaysat}
            \item $\cnotsatisfyormay{\hpair{e_1}{e_2}}{\cpair{\hxi_1}{\hxi_2}}$ \BY{\autoref{lem:not-satormay} on \pfref{[epair6]notsat} and \pfref{[epair6]notmaysat}}
            \end{pfsteps*}
            
        \restorelocalsteps{1}
        \item[\cnotsatisfyormay{e_1}{\hxi_1},\csatisfy{e_2}{\hxi_2}]
            \begin{pfsteps*}
            \item $\cnotsatisfy{e_1}{\hxi_1}$ \BY{assumption} \pflabel{[epair7]notsatisfy1}
            \item $\cnotmaysatisfy{e_1}{\hxi_1}$ \BY{assumption} \pflabel{[epair7]notmaysat1}
            \item $\csatisfy{e_2}{\hxi_2}$ \BY{assumption} \pflabel{[epair7]satisfy2}
            \item $\cnotmaysatisfy{e_2}{\hxi_2}$ \BY{assumption} \pflabel{[epair7]notmaysat2}
            \end{pfsteps*}
            Assume $\csatisfy{\hpair{e_1}{e_2}}{\cpair{\hxi_1}{\hxi_2}}$. By rule induction over Rules (\ref{rules:Satisfy}) on it, only two cases apply. 
           \begin{byCases}
            \savelocalsteps{2}
            \item[\text{(\ref{rule:CSNotIntroPair})}]
                \begin{pfsteps*}
                \item $\notIntro{\hpair{e_1}{e_2}}$ \BY{assumption}
                \end{pfsteps*}
                Contradicts \autoref{lem:no-pair-notintro}.
            \restorelocalsteps{2}
            \item[\text{(\ref{rule:CSPair})}]
                \begin{pfsteps*}
                \item $\csatisfy{e_1}{\hxi_1}$ \BY{assumption}
                \end{pfsteps*}
                Contradicts \pfref{[epair7]notsatisfy1}.
            \end{byCases}
            \begin{pfsteps*}
            \item $\cnotsatisfy{\hpair{e_1}{e_2}}{\cpair{\hxi_1}{\hxi_2}}$ \BY{contradiction} \pflabel{[epair7]notsat}
            \end{pfsteps*}
            Assume $\cmaysatisfy{\hpair{e_1}{e_2}}{\cpair{\hxi_1}{\hxi_2}}$. By rule induction over Rules (\ref{rules:MaySatisfy}) on it, the following cases apply.
            \begin{byCases}
            \savelocalsteps{2}
            \item[\text{(\ref{rule:CMSNotIntro})}]
                \begin{pfsteps*}
                \item $\notIntro{\hpair{e_1}{e_2}}$ \BY{assumption}
                \end{pfsteps*}
                Contradicts \autoref{lem:no-pair-notintro}.
            \restorelocalsteps{2}
            \item[\text{(\ref{rule:CMSPair1})}]
                \begin{pfsteps*}
                \item $\cmaysatisfy{e_1}{\hxi_1}$ \BY{assumption}
                \end{pfsteps*}
                Contradicts \pfref{[epair7]notmaysat1}.
            \restorelocalsteps{2}
            \item[\text{(\ref{rule:CMSPair2})}]
                \begin{pfsteps*}
                \item $\cmaysatisfy{e_2}{\hxi_2}$ \BY{assumption}
                \end{pfsteps*}
                Contradicts \pfref{[epair7]notmaysat2}.
            \restorelocalsteps{2}
            \item[\text{(\ref{rule:CMSPair3})}]
                \begin{pfsteps*}
                \item $\cmaysatisfy{e_1}{\hxi_1}$ \BY{assumption}
                \end{pfsteps*}
                Contradicts \pfref{[epair7]notmaysat1}.
            \end{byCases}
            \begin{pfsteps*}
            \item $\cnotmaysatisfy{\hpair{e_1}{e_2}}{\cpair{\hxi_1}{\hxi_2}}$ \BY{contradiction} \pflabel{[epair7]notmaysat}
            \item $\cnotsatisfyormay{\hpair{e_1}{e_2}}{\cpair{\hxi_1}{\hxi_2}}$ \BY{\autoref{lem:not-satormay} on \pfref{[epair7]notsat} and \pfref{[epair7]notmaysat}}
            \end{pfsteps*}
            
        \restorelocalsteps{1}
        \item[\cnotsatisfyormay{e_1}{\hxi_1},\cmaysatisfy{e_2}{\hxi_2}]
            \begin{pfsteps*}
            \item $\cnotsatisfy{e_1}{\hxi_1}$ \BY{assumption} \pflabel{[epair8]notsatisfy1}
            \item $\cnotmaysatisfy{e_1}{\hxi_1}$ \BY{assumption} \pflabel{[epair8]notmaysat1}
            \item $\cnotsatisfy{e_2}{\hxi_2}$ \BY{assumption} \pflabel{[epair8]notsatisfy2}
            \item $\cmaysatisfy{e_2}{\hxi_2}$ \BY{assumption} \pflabel{[epair8]maysat2}
            \end{pfsteps*}
            Assume $\csatisfy{\hpair{e_1}{e_2}}{\cpair{\hxi_1}{\hxi_2}}$. By rule induction over Rules (\ref{rules:Satisfy}) on it, only two cases apply. 
           \begin{byCases}
            \savelocalsteps{2}
            \item[\text{(\ref{rule:CSNotIntroPair})}]
                \begin{pfsteps*}
                \item $\notIntro{\hpair{e_1}{e_2}}$ \BY{assumption}
                \end{pfsteps*}
                Contradicts \autoref{lem:no-pair-notintro}.
            \restorelocalsteps{2}
            \item[\text{(\ref{rule:CSPair})}]
                \begin{pfsteps*}
                \item $\csatisfy{e_2}{\hxi_2}$ \BY{assumption}
                \end{pfsteps*}
                Contradicts \pfref{[epair8]notsatisfy2}.
            \end{byCases}
            \begin{pfsteps*}
            \item $\cnotsatisfy{\hpair{e_1}{e_2}}{\cpair{\hxi_1}{\hxi_2}}$ \BY{contradiction} \pflabel{[epair8]notsat}
            \end{pfsteps*}
            Assume $\cmaysatisfy{\hpair{e_1}{e_2}}{\cpair{\hxi_1}{\hxi_2}}$. By rule induction over Rules (\ref{rules:MaySatisfy}) on it, the following cases apply.
            \begin{byCases}
            \savelocalsteps{2}
            \item[\text{(\ref{rule:CMSNotIntro})}]
                \begin{pfsteps*}
                \item $\notIntro{\hpair{e_1}{e_2}}$ \BY{assumption}
                \end{pfsteps*}
                Contradicts \autoref{lem:no-pair-notintro}.
            \restorelocalsteps{2}
            \item[\text{(\ref{rule:CMSPair1})}]
                \begin{pfsteps*}
                \item $\cmaysatisfy{e_1}{\hxi_1}$ \BY{assumption}
                \end{pfsteps*}
                Contradicts \pfref{[epair8]notmaysat1}.
            \restorelocalsteps{2}
            \item[\text{(\ref{rule:CMSPair2})}]
                \begin{pfsteps*}
                \item $\csatisfy{e_1}{\hxi_1}$ \BY{assumption}
                \end{pfsteps*}
                Contradicts \pfref{[epair8]notsatisfy1}.
            \restorelocalsteps{2}
            \item[\text{(\ref{rule:CMSPair3})}]
                \begin{pfsteps*}
                \item $\cmaysatisfy{e_1}{\hxi_1}$ \BY{assumption}
                \end{pfsteps*}
                Contradicts \pfref{[epair8]notmaysat1}.
            \end{byCases}
            \begin{pfsteps*}
            \item $\cnotmaysatisfy{\hpair{e_1}{e_2}}{\cpair{\hxi_1}{\hxi_2}}$ \BY{contradiction} \pflabel{[epair8]notmaysat}
            \item $\cnotsatisfyormay{\hpair{e_1}{e_2}}{\cpair{\hxi_1}{\hxi_2}}$ \BY{\autoref{lem:not-satormay} on \pfref{[epair8]notsat} and \pfref{[epair8]notmaysat}}
            \end{pfsteps*}
            
        \restorelocalsteps{1}
        \item[\cnotsatisfyormay{e_1}{\hxi_1},\cnotsatisfyormay{e_2}{\hxi_2}]
            \begin{pfsteps*}
            \item $\cnotsatisfy{e_1}{\hxi_1}$ \BY{assumption} \pflabel{[epair9]notsatisfy1}
            \item $\cnotmaysatisfy{e_1}{\hxi_1}$ \BY{assumption} \pflabel{[epair9]notmaysat1}
            \item $\cnotsatisfy{e_2}{\hxi_2}$ \BY{assumption} \pflabel{[epair9]notsatisfy2}
            \item $\cnotmaysatisfy{e_2}{\hxi_2}$ \BY{assumption} \pflabel{[epair9]notmaysat2}
            \end{pfsteps*}
            Assume $\csatisfy{\hpair{e_1}{e_2}}{\cpair{\hxi_1}{\hxi_2}}$. By rule induction over Rules (\ref{rules:Satisfy}) on it, only two cases apply. 
           \begin{byCases}
            \savelocalsteps{2}
            \item[\text{(\ref{rule:CSNotIntroPair})}]
                \begin{pfsteps*}
                \item $\notIntro{\hpair{e_1}{e_2}}$ \BY{assumption}
                \end{pfsteps*}
                Contradicts \autoref{lem:no-pair-notintro}.
            \restorelocalsteps{2}
            \item[\text{(\ref{rule:CSPair})}]
                \begin{pfsteps*}
                \item $\csatisfy{e_2}{\hxi_2}$ \BY{assumption}
                \end{pfsteps*}
                Contradicts \pfref{[epair9]notsatisfy2}.
            \end{byCases}
            \begin{pfsteps*}
            \item $\cnotsatisfy{\hpair{e_1}{e_2}}{\cpair{\hxi_1}{\hxi_2}}$ \BY{contradiction} \pflabel{[epair9]notsat}
            \end{pfsteps*}
            Assume $\cmaysatisfy{\hpair{e_1}{e_2}}{\cpair{\hxi_1}{\hxi_2}}$. By rule induction over Rules (\ref{rules:MaySatisfy}) on it, the following cases apply.
            \begin{byCases}
            \savelocalsteps{2}
            \item[\text{(\ref{rule:CMSNotIntro})}]
                \begin{pfsteps*}
                \item $\notIntro{\hpair{e_1}{e_2}}$ \BY{assumption}
                \end{pfsteps*}
                Contradicts \autoref{lem:no-pair-notintro}.
            \restorelocalsteps{2}
            \item[\text{(\ref{rule:CMSPair1})}]
                \begin{pfsteps*}
                \item $\cmaysatisfy{e_1}{\hxi_1}$ \BY{assumption}
                \end{pfsteps*}
                Contradicts \pfref{[epair9]notmaysat1}.
            \restorelocalsteps{2}
            \item[\text{(\ref{rule:CMSPair2})}]
                \begin{pfsteps*}
                \item $\cmaysatisfy{e_2}{\hxi_2}$ \BY{assumption}
                \end{pfsteps*}
                Contradicts \pfref{[epair9]notmaysat2}.
            \restorelocalsteps{2}
            \item[\text{(\ref{rule:CMSPair3})}]
                \begin{pfsteps*}
                \item $\cmaysatisfy{e_1}{\hxi_1}$ \BY{assumption}
                \end{pfsteps*}
                Contradicts \pfref{[epair9]notmaysat1}.
            \end{byCases}
            \begin{pfsteps*}
            \item $\cnotmaysatisfy{\hpair{e_1}{e_2}}{\cpair{\hxi_1}{\hxi_2}}$ \BY{contradiction} \pflabel{[epair9]notmaysat}
            \item $\cnotsatisfyormay{\hpair{e_1}{e_2}}{\cpair{\hxi_1}{\hxi_2}}$ \BY{\autoref{lem:not-satormay} on \pfref{[epair9]notsat} and \pfref{[epair9]notmaysat}}
            \end{pfsteps*}
        \end{byCases}
    \end{byCases}
\resetpfcounter
\end{byCases}
\end{proof}

\begin{definition}[Entailment of Constraints]
  \label{defn:const-entailment}
  Suppose that $\ctyp{\hxi_1}{\tau}$ and $\ctyp{\hxi_2}{\tau}$.
  Then $\csatisfy{\hxi_1}{\hxi_2}$ iff for all $e$ such that $\hexptyp{\cdot}{\Delta}{e}{\tau}$ and $\isVal{e}$ we have $\csatisfyormay{e}{\hxi_1}$ implies $\csatisfy{e}{\hxi_2}$
\end{definition}

\begin{definition}[Potential Entailment of Constraints]
  \label{defn:nn-entailment}
  Suppose that $\ctyp{\hxi_1}{\tau}$ and $\ctyp{\hxi_2}{\tau}$. Then $\csatisfyormay{\hxi_1}{\hxi_2}$ iff for all $e$ such that $\hexptyp{\cdot}{\Delta}{e}{\tau}$ and $\isFinal{e}$ we have $\csatisfyormay{e}{\hxi_1}$ implies $\csatisfyormay{e}{\hxi_2}$ 
\end{definition}

\begin{corollary}
  \label{corol:nn-exhaust}
  Suppose that $\ctyp{\hxi}{\tau}$ and $\hexptyp{\cdot}{\Delta}{e}{\tau}$ and $\isFinal{e}$. Then $\csatisfyormay{\ctruth}{\hxi}$ implies $\csatisfyormay{e}{\hxi}$
\end{corollary}
\begin{proof}
  \begin{pfsteps*}
  \item $\ctyp{\hxi}{\tau}$ \BY{assumption} \pflabel{CTyp}
  \item $\hexptyp{\cdot}{\Gamma}{e}{\tau}$ \BY{assumption} \pflabel{eTyp}
  \item $\isFinal{e}$ \BY{assumption} \pflabel{eFinal}
  \item $\csatisfyormay{\ctruth}{\hxi}$ \BY{assumption} \pflabel{entailormay}
  \item $\csatisfy{e_1}{\ctruth}$ \BY{Rule (\ref{rule:CSTruth})} \pflabel{satisfytop}
  \item $\csatisfyormay{e_1}{\ctruth}$ \BY{Rule (\ref{rule:CSMSSat}) on \pfref{satisfytop}} \pflabel{satisfyormaytop}
  \item $\ctyp{\ctruth}{\tau}$ \BY{Rule (\ref{rule:CTTruth})} \pflabel{topCTyp}
  \item $\csatisfyormay{e_1}{\hxi_r}$ \BY{Definition \ref{defn:nn-entailment} of \pfref{entailormay} on \pfref{topCTyp} and \pfref{CTyp} and \pfref{eTyp} and \pfref{eFinal} and \pfref{satisfyormaytop}}
  \end{pfsteps*}
  \resetpfcounter
\end{proof}
\section{Normal Match Constraint Language}
$\arraycolsep=4pt\begin{array}{lll}
\xi & ::= &
  \ctruth ~\vert~
  \cfalsity ~\vert~
  \cnum{n} ~\vert~
  \cnotnum{n} ~\vert~
  \cand{\xi_1}{\xi_2} ~\vert~
  \cor{\xi_1}{\xi_2} ~\vert~
  \cinl{\xi} ~\vert~
  \cinr{\xi} ~\vert~
  \cpair{\xi_1}{\xi_2}
\end{array}$

\judgboxa{\ctyp{\xi}{\tau}}{$\xi$ constrains final expressions of type $\tau$}
\begin{subequations}\label{rules:CCTyp}
\begin{equation}\label{rule:CCTTruth}
\inferrule[CTTruth]{ }{
  \ctyp{\ctruth}{\tau}
}
\end{equation}
\begin{equation}\label{rule:CCTFalsity}
\inferrule[CTFalsity]{ }{
  \ctyp{\cfalsity}{\tau}
}
\end{equation}
\begin{equation}\label{rule:CCTNum}
\inferrule[CTNum]{ }{
  \ctyp{\cnum{n}}{\tnum}
}
\end{equation}
\begin{equation}\label{rule:CCTNotNum}
\inferrule[CTNotNum]{ }{
  \ctyp{\cnotnum{n}}{\tnum}
}
\end{equation}
\begin{equation}\label{rule:CCTAnd}
\inferrule[CTAnd]{
  \ctyp{\xi_1}{\tau} \\ \ctyp{\xi_2}{\tau}
}{
  \ctyp{\cand{\xi_1}{\xi_2}}{\tau}
}
\end{equation}
\begin{equation}\label{rule:CCTOr}
\inferrule[CTOr]{
  \ctyp{\xi_1}{\tau} \\ \ctyp{\xi_2}{\tau}
}{
  \ctyp{\cor{\xi_1}{\xi_2}}{\tau}
}
\end{equation}
\begin{equation}\label{rule:CCTInl}
\inferrule[CTInl]{
  \ctyp{\xi_1}{\tau_1}
}{
  \ctyp{\cinl{\xi_1}}{\tsum{\tau_1}{\tau_2}}
}
\end{equation}
\begin{equation}\label{rule:CCTInr}
\inferrule[CTInr]{
  \ctyp{\xi_2}{\tau_2}
}{
  \ctyp{\cinr{\xi_2}}{\tsum{\tau_1}{\tau_2}}
}
\end{equation}
\begin{equation}\label{rule:CCTPair}
\inferrule[CTPair]{
  \ctyp{\xi_1}{\tau_1} \\ \ctyp{\xi_2}{\tau_2}
}{
  \ctyp{\cpair{\xi_1}{\xi_2}}{\tprod{\tau_1}{\tau_2}}
}
\end{equation}
\end{subequations}

\judgboxa{\cdual{\xi_1} = \xi_2}{dual of $\xi_1$ is $\xi_2$}
\begin{subequations}\label{defn:dual}
\begin{align*}
  \cdual{\ctruth} &= \cfalsity \\
  \cdual{\cfalsity} &= \ctruth \\
  \cdual{\cnum{n}} &= \cnotnum{n} \\
  \cdual{\cnotnum{n}} &= \cnum{n} \\
  \cdual{\cand{\xi_1}{\xi_2}} &= \cor{\cdual{\xi_1}}{\cdual{\xi_2}} \\
  \cdual{\cor{\xi_1}{\xi_2}} &= \cand{\cdual{\xi_1}}{\cdual{\xi_2}} \\
  \cdual{\cinl{\xi_1}} &= \cor{ \cinl{\cdual{\xi_1}} }{ \cinr{\ctruth} } \\
  \cdual{\cinr{\xi_2}} &= \cor{ \cinr{\cdual{\xi_2}} }{ \cinl{\ctruth} } \\
  \cdual{\cpair{\xi_1}{\xi_2}} &=
  \cor{ \cor{ 
    \cpair{\xi_1}{\cdual{\xi_2}}
  }{
    \cpair{\cdual{\xi_1}}{\xi_2}
  }}{
    \cpair{\cdual{\xi_1}}{\cdual{\xi_2}}
  } \\
  \cdual{\cdual{\xi}} &= \xi
\end{align*}
\end{subequations}

\judgboxa{\ccsatisfy{e}{\xi}}{$e$ satisfies $\xi$}
\begin{subequations}\label{rules:cSatisfy}
\begin{equation}\label{rule:CCSTruth}
\inferrule[CSTruth]{ }{
  \ccsatisfy{e}{\ctruth}
}
\end{equation}
\begin{equation}\label{rule:CCSNum}
\inferrule[CSNum]{ }{
  \ccsatisfy{\hnum{n}}{\cnum{n}}
}
\end{equation}
\begin{equation}\label{rule:CCSNotNum}
\inferrule[CSNotNum]{
  n_1 \neq n_2
}{
  \ccsatisfy{\hnum{n_1}}{\cnotnum{n_2}}
}
\end{equation}
\begin{equation}\label{rule:CCSAnd}
\inferrule[CSAnd]{
  \ccsatisfy{e}{\xi_1} \\
  \ccsatisfy{e}{\xi_2}
}{
  \ccsatisfy{e}{\cand{\xi_1}{\xi_2}}
}
\end{equation}
\begin{equation}\label{rule:CCSOr1}
\inferrule[CSOrL]{
  \ccsatisfy{e}{\xi_1}
}{
  \ccsatisfy{e}{\cor{\xi_1}{\xi_2}}
}
\end{equation}
\begin{equation}\label{rule:CCSOr2}
\inferrule[CSOrR]{
  \ccsatisfy{e}{\xi_2}
}{
  \ccsatisfy{e}{\cor{\xi_1}{\xi_2}}
}
\end{equation}
\begin{equation}\label{rule:CCSInl}
\inferrule[CSInl]{
  \ccsatisfy{e_1}{\xi_1}
}{
  \ccsatisfy{
    \hinl{\tau_2}{e_1}
  }{
    \cinl{\xi_1}
  }
}
\end{equation}
\begin{equation}\label{rule:CCSInr}
\inferrule[CSInr]{
  \ccsatisfy{e_2}{\xi_2}
}{
  \ccsatisfy{
    \hinr{\tau_1}{e_2}
  }{
    \cinr{\xi_2}
  }
}
\end{equation}
\begin{equation}\label{rule:CCSPair}
\inferrule[CSPair]{
  \ccsatisfy{e_1}{\xi_1} \\
  \ccsatisfy{e_2}{\xi_2}
}{
\ccsatisfy{\hpair{e_1}{e_2}}{\cpair{\xi_1}{\xi_2}}
}
\end{equation}
\end{subequations}

\begin{lemma}
  \label{lem:notsatisfy-dual}
  Assume $\isVal{e}$. Then $\ccnotsatisfy{e}{\xi}$ iff $\ccsatisfy{e}{\cdual{\xi}}$.
\end{lemma}

\begin{theorem}[Exclusiveness of Satisfaction Judgment]
  \label{thrm:exclusive-complete-constraint-satisfaction}
  If $\ctyp{\xi}{\tau}$ and $\hexptyp{\cdot}{\Delta}{e}{\tau}$ and $\isVal{e}$ then exactly one of the following holds
  \begin{enumerate}
    \item $\ccsatisfy{e}{\xi}$
    \item $\ccsatisfy{e}{\cdual{\xi}}$
  \end{enumerate}
\end{theorem}
\begin{proof}
\end{proof}

\begin{definition}[Entailment of Constraints]
  \label{defn:complete-constraint-entailment}
  Suppose that $\ctyp{\xi_1}{\tau}$ and $\ctyp{\xi_2}{\tau}$.
  Then $\ccsatisfy{\xi_1}{\xi_2}$ iff for all $e$ such that $\hexptyp{\cdot}{\Delta}{e}{\tau}$ and $\isVal{e}$ we have $\ccsatisfy{e}{\xi_1}$ implies $\ccsatisfy{e}{\xi_2}$
\end{definition}

\subsection{Relationship with Incomplete Constraint Language}

\begin{lemma}
  \label{lem:val-satisfy-truify}
  Assume that $\isVal{e}$. Then $\csatisfyormay{e}{\hxi}$ iff $\ccsatisfy{e}{\ctruify{\hxi}}$.
\end{lemma}
\begin{proof}\mbox{}\\
  We prove sufficiency and necessity separately.
  \begin{enumerate}
    \item Sufficiency:
      \begin{pfsteps}
      \item \isVal{e} \BY{assumption} \pflabel{val}
      \item \csatisfyormay{e}{\hxi} \BY{assumption} \pflabel{satormay}
      \end{pfsteps}
      By rule induction over Rules (\ref{rules:satormay}) on \pfref{satormay}.
      \begin{byCases}

      \savelocalsteps{1}
      \item[\text{(\ref{rule:CSMSSat})}]
        \begin{pfsteps*}
        \item $\csatisfy{e}{\hxi}$ \BY{assumption} \pflabel{satisfy}
        \end{pfsteps*}
        By rule induction over \rulesref{rules:Satisfy} on \pfref{satisfy}.
        \begin{byCases}
          \savelocalsteps{2}
          \item[\text{(\ref{rule:CSTruth})}]
          \begin{pfsteps}
          \item \hxi = \ctruth \BY{assumption}
          \item \ctruify{\hxi} = \ctruth \BY{\autoref{defn:truify}}
          \item \ccsatisfy{e}{\ctruth} \BY{\ruleref{rule:CCSTruth}}
          \end{pfsteps} 
          \restorelocalsteps{2} 
          \item[\text{(\ref{rule:CSNum})}] 
          \begin{pfsteps}
          \item e = \hnum{n} \BY{assumption}
          \item \hxi = \cnum{n} \BY{assumption}
          \item \ctruify{\cnum{n}} = \cnum{n} \BY{\autoref{defn:truify}}
          \item \ccsatisfy{e}{\cnum{n}} \BY{\ruleref{rule:CCSNum}}
          \end{pfsteps} 
          \restorelocalsteps{2} 
          \item[\text{(\ref{rule:CSInl})}]
          \begin{pfsteps}
          \item e = \hinl{\tau_2}{e_1} \BY{assumption}
          \item \hxi = \cinl{\hxi_1} \BY{assumption}
          \item \csatisfy{e_1}{\hxi_1} \BY{assumption} \pflabel{[inl]satisfy1}
          \item \csatisfyormay{e_1}{\hxi_1} \BY{\ruleref{rule:CSMSSat} on \pfref{[inl]satisfy1}} \pflabel{[inl]satormay1}
          \item \ctruify{\cinl{\hxi_1}} = \cinl{\ctruify{\hxi_1}} \BY{\autoref{defn:truify}}
          \end{pfsteps}
          By rule induction over \rulesref{rules:Value} on \pfref{val}, only one rule applies.
          \begin{byCases}
            \item[\text{(\ref{rule:VInl})}]
            \begin{pfsteps}
            \item \isVal{e_1} \BY{assumption} \pflabel{[inl]val1}
            \item \ccsatisfy{e_1}{\ctruify{\hxi_1}} \BY{IH on \pfref{[inl]val1} and \pfref{[inl]satormay1}} \pflabel{[inl]ccsat1}
            \item \ccsatisfy{\hinl{\tau_2}{e_1}}{\cinl{\ctruify{\hxi_1}}} \BY{\ruleref{rule:CCSInl} on \pfref{[inl]ccsat1}}
            \end{pfsteps}    
          \end{byCases}
          \restorelocalsteps{2} 
          \item[\text{(\ref{rule:CSInr})}]
          \begin{pfsteps}
          \item e = \hinr{\tau_1}{e_2} \BY{assumption}
          \item \hxi = \cinr{\hxi_2} \BY{assumption}
          \item \csatisfy{e_2}{\hxi_2} \BY{assumption} \pflabel{[inr]satisfy2}
          \item \csatisfyormay{e_2}{\hxi_2} \BY{\ruleref{rule:CSMSSat} on \pfref{[inr]satisfy2}} \pflabel{[inr]satormay2}
          \item \ctruify{\cinr{\hxi_2}} = \cinr{\ctruify{\hxi_2}} \BY{\autoref{defn:truify}}
          \end{pfsteps}
          By rule induction over \rulesref{rules:Value} on \pfref{val}, only one rule applies.
          \begin{byCases}
            \item[\text{(\ref{rule:VInr})}]
            \begin{pfsteps}
            \item \isVal{e_2} \BY{assumption} \pflabel{[inr]val2}
            \item \ccsatisfy{e_2}{\ctruify{\hxi_2}} \BY{IH on \pfref{[inr]val2} and \pfref{[inr]satormay2}} \pflabel{[inr]ccsat2}
            \item \ccsatisfy{\hinr{\tau_1}{e_2}}{\cinr{\ctruify{\hxi_2}}} \BY{\ruleref{rule:CCSInr} on \pfref{[inr]ccsat2}}
            \end{pfsteps}    
          \end{byCases}
          \restorelocalsteps{2} 
          \item[\text{(\ref{rule:CSPair})}]
          \begin{pfsteps}
          \item e = \hpair{e_1}{e_2} \BY{assumption}
          \item \hxi = \cpair{\hxi_1}{\hxi_2} \BY{assumption}
          \item \csatisfy{e_1}{\hxi_1} \BY{assumption} \pflabel{[pair]satisfy1}
          \item \csatisfy{e_2}{\hxi_2} \BY{assumption} \pflabel{[pair]satisfy2}
          \item \csatisfyormay{e_1}{\hxi_1} \BY{\ruleref{rule:CSMSSat} on \pfref{[pair]satisfy1}} \pflabel{[pair]satormay1}
          \item \csatisfyormay{e_2}{\hxi_2} \BY{\ruleref{rule:CSMSSat} on \pfref{[pair]satisfy2}} \pflabel{[pair]satormay2}
          \item \ctruify{\cpair{\hxi_1}{\hxi_2}} = \cpair{\ctruify{\hxi_1}}{\ctruify{\hxi_2}} \BY{\autoref{defn:truify}}
          \end{pfsteps}
          By rule induction over \rulesref{rules:Value} on \pfref{val}, only one rule applies.
          \begin{byCases}
            \item[\text{(\ref{rule:VPair})}]
            \begin{pfsteps}
            \item \isVal{e_1} \BY{assumption} \pflabel{[pair]val1}
            \item \isVal{e_2} \BY{assumption} \pflabel{[pair]val2}
            \item \ccsatisfy{e_1}{\ctruify{\hxi_1}} \BY{IH on \pfref{[pair]val1} and \pfref{[pair]satormay1}} \pflabel{[pair]ccsat1}
            \item \ccsatisfy{e_2}{\ctruify{\hxi_2}} \BY{IH on \pfref{[pair]val2} and \pfref{[pair]satormay2}} \pflabel{[pair]ccsat2}
            \item \ccsatisfy{\hpair{e_1}{e_2}}{\cpair{\ctruify{\hxi_1}}{\ctruify{\hxi_2}}} \BY{\ruleref{rule:CCSPair} on \pfref{[pair]ccsat1} and \pfref{[pair]ccsat2}}
            \end{pfsteps}    
          \end{byCases}
          \restorelocalsteps{2} 
          \item[\text{(\ref{rule:CSNotIntroPair})}]
          \begin{pfsteps}
          \item \notIntro{e} \BY{assumption}
          \end{pfsteps} 
          Contradicts \pfref{val} by \autoref{lem:val-not-notintro}.
          \restorelocalsteps{2} 
          \item[\text{(\ref{rule:CSOr1})}]
          \begin{pfsteps}
          \item \hxi = \cor{\hxi_1}{\hxi_2} \BY{assumption}
          \item \csatisfy{e}{\hxi_1} \BY{assumption} \pflabel{[or1]sat1}
          \item \csatisfyormay{e}{\hxi_1} \BY{\ruleref{rule:CSMSSat} on \pfref{[or1]sat1}} \pflabel{[or1]satormay1}
          \item \ctruify{\cor{\hxi_1}{\hxi_2}} = \cor{\ctruify{\hxi_1}}{\ctruify{\hxi_2}} \BY{\autoref{defn:truify}}
          \item \ccsatisfy{e}{\ctruify{\hxi_1}} \BY{IH on \pfref{val} and \pfref{[or1]satormay1}} \pflabel{[or1]ccsat1}
          \item \ccsatisfy{e}{\cor{\ctruify{\hxi_1}}{\ctruify{\hxi_2}}} \BY{\ruleref{rule:CCSOr1} on \pfref{[or1]ccsat1}}
          \end{pfsteps} 
          \restorelocalsteps{2} 
          \item[\text{(\ref{rule:CSOr2})}]
          \begin{pfsteps}
          \item \hxi = \cor{\hxi_1}{\hxi_2} \BY{assumption}
          \item \csatisfy{e}{\hxi_2} \BY{assumption} \pflabel{[or2]sat2}
          \item \csatisfyormay{e}{\hxi_2} \BY{\ruleref{rule:CSMSSat} on \pfref{[or2]sat2}} \pflabel{[or2]satormay2}
          \item \ctruify{\cor{\hxi_1}{\hxi_2}} = \cor{\ctruify{\hxi_1}}{\ctruify{\hxi_2}} \BY{\autoref{defn:truify}}
          \item \ccsatisfy{e}{\ctruify{\hxi_2}} \BY{IH on \pfref{val} and \pfref{[or2]satormay2}} \pflabel{[or2]ccsat2}
          \item \ccsatisfy{e}{\cor{\ctruify{\hxi_1}}{\ctruify{\hxi_2}}} \BY{\ruleref{rule:CCSOr2} on \pfref{[or2]ccsat2}}
          \end{pfsteps} 
        \end{byCases}
      
      \restorelocalsteps{1}
      \item[\text{(\ref{rule:CSMSMay})}]
        \begin{pfsteps*}
        \item $\cmaysatisfy{e}{\hxi}$ \BY{assumption} \pflabel{maysat}
        \end{pfsteps*}
        By rule induction over Rules (\ref{rules:MaySatisfy}) on \pfref{maysat}.
        \begin{byCases}

        \savelocalsteps{2}
        \item[\text{(\ref{rule:CMSUnknown})}]
          \begin{pfsteps}
          \item \hxi=\cunknown \BY{assumption}
          \item \ctruify{\cunknown} = \ctruth \BY{\autoref{defn:truify}} 
          \item \ccsatisfy{e}{\ctruth} \BY{Rule (\ref{rule:CCSTruth})}
          \end{pfsteps}
        
        \restorelocalsteps{2} 
        \item[\text{(\ref{rule:CMSInl})}] 
          \begin{pfsteps}
          \item e = \hinl{\tau_2}{e_1} \BY{assumption}
          \item \hxi=\cinl{\hxi_1} \BY{assumption}
          \item \cmaysatisfy{e_1}{\hxi_1} \BY{assumption} \pflabel{[inl1]maysat}
          \item \csatisfyormay{e_1}{\hxi_1} \BY{\ruleref{rule:CSMSMay} on \pfref{[inl1]maysat}} \pflabel{[inl1]satormay}
          \item \ctruify{\cinl{\hxi_1}}=\cinl{\ctruify{\xi_1}} \BY{\autoref{defn:truify}}
          \end{pfsteps}
          By rule induction over \rulesref{rules:Value} on \pfref{val}, only one rule applies.
          \begin{byCases}
            \item[\text{(\ref{rule:VInl})}]
            \begin{pfsteps}
            \item \isVal{e_1} \BY{assumption} \pflabel{[may][inl]val1}
            \item \ccsatisfy{e_1}{\ctruify{\hxi_1}} \BY{IH on \pfref{[may][inl]val1} and \pfref{[inl1]satormay}} \pflabel{[inl1]sat-truify}
            \item \ccsatisfy{\hinl{\tau_2}{e_1}}{\cinl{\ctruify{\hxi_1}}} \BY{\ruleref{rule:CSInl} on \pfref{[inl1]sat-truify}}
            \end{pfsteps}
          \end{byCases}
        
        \restorelocalsteps{2} 
        \item[\text{(\ref{rule:CMSInr})}] 
          \begin{pfsteps}
          \item e = \hinr{\tau_1}{e_2} \BY{assumption}
          \item \hxi=\cinr{\hxi_2} \BY{assumption}
          \item \cmaysatisfy{e_2}{\hxi_2} \BY{assumption} \pflabel{[inr]maysat}
          \item \csatisfyormay{e_2}{\hxi_2} \BY{\ruleref{rule:CSMSMay} on \pfref{[inr]maysat}} \pflabel{[inr]satormay}
          \item \ctruify{\cinr{\hxi_2}}=\cinr{\ctruify{\hxi_2}} \BY{\autoref{defn:truify}}
          \end{pfsteps}
          By rule induction over \rulesref{rules:Value} on \pfref{val}, only one rule applies.
          \begin{byCases}
            \item[\text{(\ref{rule:VInr})}]
            \begin{pfsteps}
            \item \isVal{e_2} \BY{assumption} \pflabel{[may][inr]val2}
            \item \ccsatisfy{e_2}{\ctruify{\hxi_2}} \BY{IH on \pfref{[may][inr]val2} and \pfref{[inr]satormay}} \pflabel{[inr]sat-truify}
            \item \ccsatisfy{\hinr{\tau_1}{e_2}}{\cinr{\ctruify{\hxi_2}}} \BY{\ruleref{rule:CSInr} on \pfref{[inr]sat-truify}}
            \end{pfsteps}
          \end{byCases}
        
        \restorelocalsteps{2} 
        \item[\text{(\ref{rule:CMSPair1})}] 
          \begin{pfsteps}
          \item e=\hpair{e_1}{e_2} \BY{assumption}
          \item \hxi=\cpair{\hxi_1}{\hxi_2} \BY{assumption}
          \item \cmaysatisfy{e_1}{\hxi_1} \BY{assumption} \pflabel{[pair1]maysat1}
          \item \csatisfy{e_2}{\hxi_2} \BY{assumption} \pflabel{[pair1]satisfy2}
          \item \ctruify{\hxi}=\cpair{\ctruify{\hxi_1}}{\ctruify{\hxi_2}} \BY{\autoref{defn:truify}}
          \item \csatisfyormay{e_1}{\hxi_1} \BY{\ruleref{rule:CSMSMay} on \pfref{[pair1]maysat1}} \pflabel{[pair1]satormay1}
          \item \csatisfyormay{e_2}{\hxi_2} \BY{\ruleref{rule:CSMSSat} on \pfref{[pair1]satisfy2}} \pflabel{[pair1]satormay2}
          \end{pfsteps}
          By rule induction over \rulesref{rules:Value} on \pfref{val}, only one rule applies.
          \begin{byCases}
            \item[\text{(\ref{rule:VPair})}]
            \begin{pfsteps}
            \item \isVal{e_1} \BY{assumption} \pflabel{[pair1]val1}
            \item \isVal{e_2} \BY{assumption} \pflabel{[pair1]val2}
            \item \ccsatisfy{e_1}{\ctruify{\hxi_1}} \BY{IH on \pfref{[pair1]val1} and \pfref{[pair1]satormay1}} \pflabel{[pair1]sat-truify1}
            \item \ccsatisfy{e_2}{\ctruify{\hxi_2}} \BY{IH on \pfref{[pair1]val2} and \pfref{[pair1]satormay2}} \pflabel{[pair1]sat-truify2}
            \item \ccsatisfy{\hpair{e_1}{e_2}}{\cpair{\ctruify{\hxi_1}}{\ctruify{\hxi_2}}} \BY{\ruleref{rule:CSPair} on \pfref{[pair1]sat-truify1} and \pfref{[pair1]sat-truify2}}
            \end{pfsteps}
          \end{byCases}
        
        \restorelocalsteps{2} 
        \item[\text{(\ref{rule:CMSPair2})}] 
          \begin{pfsteps}
          \item e=\hpair{e_1}{e_2} \BY{assumption}
          \item \hxi=\cpair{\hxi_1}{\hxi_2} \BY{assumption}
          \item \csatisfy{e_1}{\hxi_1} \BY{assumption} \pflabel{[pair2]satisfy1}
          \item \cmaysatisfy{e_2}{\hxi_2} \BY{assumption} \pflabel{[pair2]maysat2}
          \item \ctruify{\cpair{\hxi_1}{\hxi_2}}=\cpair{\ctruify{\hxi_1}}{\ctruify{\xi_2}} \BY{\autoref{defn:truify}}
          \item \csatisfyormay{e_1}{\hxi_1} \BY{\ruleref{rule:CSMSSat} on \pfref{[pair2]satisfy1}} \pflabel{[pair2]satormay1}
          \item \csatisfyormay{e_2}{\hxi_2} \BY{\ruleref{rule:CSMSMay} on \pfref{[pair2]maysat2}} \pflabel{[pair2]satormay2}
          \end{pfsteps}
          By rule induction over \rulesref{rules:Value} on \pfref{val}, only one rule applies.
          \begin{byCases}
            \item[\text{(\ref{rule:VPair})}]
            \begin{pfsteps}
            \item \isVal{e_1} \BY{assumption} \pflabel{[pair2]val1}
            \item \isVal{e_2} \BY{assumption} \pflabel{[pair2]val2}
            \item \ccsatisfy{e_1}{\ctruify{\hxi_1}} \BY{IH on \pfref{[pair2]val1} and \pfref{[pair2]satormay1}} \pflabel{[pair2]sat-truify1}
            \item \ccsatisfy{e_2}{\ctruify{\hxi_2}} \BY{IH on \pfref{[pair2]val2} and \pfref{[pair2]satormay2}} \pflabel{[pair2]sat-truify2}
            \item \ccsatisfy{\hpair{e_1}{e_2}}{\cpair{\ctruify{\hxi_1}}{\ctruify{\hxi_2}}} \BY{\ruleref{rule:CSPair} on \pfref{[pair2]sat-truify1} and \pfref{[pair2]sat-truify2}}
            \end{pfsteps}
          \end{byCases}
        
        \restorelocalsteps{2} 
        \item[\text{(\ref{rule:CMSPair3})}] 
          \begin{pfsteps}
          \item e=\hpair{e_1}{e_2} \BY{assumption}
          \item \hxi=\cpair{\hxi_1}{\hxi_2} \BY{assumption}
          \item \cmaysatisfy{e_1}{\hxi_1} \BY{assumption} \pflabel{[pair3]maysat1}
          \item \cmaysatisfy{e_2}{\hxi_2} \BY{assumption} \pflabel{[pair3]maysat2}
          \item \ctruify{\cpair{\hxi_1}{\hxi_2}}=\cpair{\ctruify{\hxi_1}}{\ctruify{\hxi_2}} \BY{\autoref{defn:truify}}
          \item \csatisfyormay{e_1}{\hxi_1} \BY{\ruleref{rule:CSMSMay} on \pfref{[pair3]maysat1}} \pflabel{[pair3]satormay1}
          \item \csatisfyormay{e_2}{\hxi_2} \BY{\ruleref{rule:CSMSMay} on \pfref{[pair3]maysat2}} \pflabel{[pair3]satormay2}
          \item \ccsatisfy{e_1}{\ctruify{\hxi_1}} \BY{IH on \pfref{[pair3]satormay1}} \pflabel{[pair3]sat-truify1}
          \item \ccsatisfy{e_2}{\ctruify{\hxi_2}} \BY{IH on \pfref{[pair3]satormay2}} \pflabel{[pair3]sat-truify2}
          \item \ccsatisfy{\hpair{e_1}{e_2}}{\cpair{\ctruify{\hxi_1}}{\ctruify{\hxi_2}}} \BY{\ruleref{rule:CSPair} on \pfref{[pair3]sat-truify1} and \pfref{[pair3]sat-truify2}}
          \end{pfsteps}
        
        \restorelocalsteps{2} 
        \item[\text{(\ref{rule:CMSOr1})}] 
          \begin{pfsteps}
          \item \hxi=\cor{\hxi_1}{\hxi_2} \BY{assumption}
          \item \cmaysatisfy{e}{\hxi_1} \BY{assumption} \pflabel{[or1]maysat}
          \item \ctruify{\cor{\hxi_1}{\hxi_2}}=\cor{\ctruify{\hxi_1}}{\ctruify{\hxi_2}} \BY{\autoref{defn:truify}}
          \item \csatisfyormay{e}{\hxi_1} \BY{\ruleref{rule:CSMSMay} on \pfref{[or1]maysat}} \pflabel{[or1]satormay}
          \item \ccsatisfy{e}{\ctruify{\hxi_1}} \BY{IH on \pfref{val} and \pfref{[or1]satormay}} \pflabel{[or1]sat-truify}
          \item \ccsatisfy{e}{\cor{\ctruify{\hxi_1}}{\ctruify{\hxi_2}}} \BY{\ruleref{rule:CCSOr1} on \pfref{[or1]sat-truify}}
          \end{pfsteps}
        
        \restorelocalsteps{2} 
        \item[\text{(\ref{rule:CMSOr2})}] 
          \begin{pfsteps}
          \item \hxi=\cor{\hxi_1}{\hxi_2} \BY{assumption}
          \item \cmaysatisfy{e}{\hxi_2} \BY{assumption} \pflabel{[or2]maysat}
          \item \ctruify{\hxi}=\cor{\ctruify{\hxi_1}}{\ctruify{\hxi_2}} \BY{\autoref{defn:truify}}
          \item \csatisfyormay{e}{\hxi_2} \BY{\ruleref{rule:CSMSMay} on \pfref{[or2]maysat}} \pflabel{[or2]satormay}
          \item \ccsatisfy{e}{\ctruify{\hxi_2}} \BY{IH on \pfref{val} and \pfref{[or2]satormay}} \pflabel{[or2]sat-truify}
          \item \ccsatisfy{e}{\cor{\ctruify{\hxi_1}}{\ctruify{\hxi_2}}} \BY{\ruleref{rule:CSOr2} on \pfref{[or2]sat-truify}}
          \end{pfsteps}
          
        \restorelocalsteps{2} 
        \item[\text{(\ref{rule:CMSNotIntro})}] 
          \begin{pfsteps}
          \item \notIntro{e} \BY{assumption} \pflabel{notintro}
          \end{pfsteps}
          Contradicts \pfref{val} by \autoref{lem:val-not-notintro}.
        
        \end{byCases}
      \end{byCases}

    \resetpfcounter

    \item Necessity:
    \begin{pfsteps}
    \item \isVal{e} \BY{assumption} \pflabel{val}
    \item \ccsatisfy{e}{\ctruify{\hxi}} \BY{assumption} \pflabel{ccsatisfy}
    \end{pfsteps}
    By structural induction on $\hxi$.
    \begin{byCases}

      \savelocalsteps{1}
      \item[\hxi=\ctruth]
      \begin{pfsteps}
      \item \csatisfy{e}{\ctruth} \BY{\ruleref{rule:CSTruth}} \pflabel{[truth]satisfy}
      \item \csatisfyormay{e}{\ctruth} \BY{\ruleref{rule:CSMSSat} on \pfref{[truth]satisfy}}
      \end{pfsteps}
      
      \restorelocalsteps{1}
      \item[\hxi=\cnum{n}]
      \begin{pfsteps}
      \item \ctruify{\cnum{n}} = \cnum{n} \BY{assumption}
      \end{pfsteps}
      By rule induction over \rulesref{rules:cSatisfy} on \pfref{ccsatisfy}, only one rule applies.
      \begin{byCases}
        \item[\text{(\ref{rule:CCSNum})}]
        \begin{pfsteps}
        \item e = \hnum{n} \BY{assumption}
        \item \csatisfy{\hnum{n}}{\cnum{n}} \BY{\ruleref{rule:CSNum}} \pflabel{[num]satisfy}
        \item \csatisfyormay{\hnum{n}}{\cnum{n}} \BY{\ruleref{rule:CSMSSat} on \pfref{[num]satisfy}}
        \end{pfsteps}
      \end{byCases}

      \restorelocalsteps{1}
      \item[\hxi=\cunknown]
        \begin{pfsteps}
        \item \cmaysatisfy{e}{\cunknown} \BY{\ruleref{rule:CMSUnknown}} \pflabel{[unknown]maysat}
        \item \csatisfyormay{e}{\cunknown} \BY{\ruleref{rule:CSMSMay} on \pfref{[unknown]maysat}}
        \end{pfsteps}

      \restorelocalsteps{1}
      \item[\hxi = \cor{\hxi_1}{\hxi_2}]
      \begin{pfsteps*}
      \item $\ctruify{\cor{\hxi_1}{\hxi_2}}=\cor{\ctruify{\hxi_1}}{\ctruify{\hxi_2}}$ \BY{\autoref{defn:truify}}
      \end{pfsteps*}
      By rule induction over \rulesref{rules:cSatisfy} on \pfref{ccsatisfy}, only two rules apply.
      \begin{byCases}

        \savelocalsteps{2}
        \item[\text{(\ref{rule:CCSOr1})}]
        \begin{pfsteps*}
        \item $\ccsatisfy{e}{\ctruify{\hxi_1}}$ \BY{assumption} \pflabel{[or]satisfy-falsify1}
        \item $\csatisfyormay{e}{\hxi_1}$ \BY{IH on \pfref{val} and \pfref{[or]satisfy-falsify1}} \pflabel{[or]satisfy1}
        \item $\csatisfyormay{e}{\cor{\hxi_1}{\hxi_2}}$ \BY{\autoref{lem:satormay-or} on \pfref{[or]satisfy1}}
        \end{pfsteps*}

        \restorelocalsteps{2}
        \item[\text{(\ref{rule:CCSOr2})}]
        \begin{pfsteps*}
        \item $\ccsatisfy{e}{\ctruify{\hxi_2}}$ \BY{assumption} \pflabel{[or]satisfy-falsify2}
        \item $\csatisfyormay{e}{\hxi_2}$ \BY{IH on \pfref{val} and \pfref{[or]satisfy-falsify2}} \pflabel{[or]satisfy2}
        \item $\csatisfyormay{e}{\cor{\hxi_1}{\hxi_2}}$ \BY{\autoref{lem:satormay-or} on \pfref{[or]satisfy2}}
        \end{pfsteps*}
      \end{byCases}
    
      \restorelocalsteps{1}
      \item[\hxi=\cinl{\hxi_1}]
        \begin{pfsteps*}
        \item $\ctruify{\cinl{\hxi_1}}=\cinl{\ctruify{\hxi_1}}$ \BY{\autoref{defn:truify}}
        \end{pfsteps*}
        By rule induction over \rulesref{rules:cSatisfy} on \pfref{ccsatisfy}, only one rule applies.
        \begin{byCases}
          \item[\text{(\ref{rule:CCSInl})}]
            \begin{pfsteps*}
            \item $e = \hinl{\tau_2}{e_1}$ \BY{assumption}
            \item $\ccsatisfy{e_1}{\ctruify{\hxi_1}}$ \BY{assumption} \pflabel{[inl]satisfy-falsify1}
            \end{pfsteps*}
            By rule induction over \rulesref{rules:Value} on \pfref{val}, only one rule applies.
            \begin{byCases}
              \item[\text{(\ref{rule:VInl})}]
              \begin{pfsteps*}
              \item $\isVal{e_1}$ \BY{assumption} \pflabel{[inl]val1}
              \item $\csatisfyormay{e_1}{\hxi_1}$ \BY{IH on \pfref{[inl]val1} and \pfref{[inl]satisfy-falsify1}} \pflabel{[inl]satisfy1}
              \item $\csatisfyormay{\hinl{\tau_2}{e_1}}{\cinl{\hxi_1}}$ \BY{\autoref{lem:satormay-inl} on \pfref{[inl]satisfy1}}
              \end{pfsteps*} 
            \end{byCases}
        \end{byCases}

      \restorelocalsteps{1}
      \item[\hxi=\cinr{\hxi_2}]
        \begin{pfsteps*}
        \item $\ctruify{\cinr{\hxi_2}}=\cinr{\ctruify{\hxi_2}}$ \BY{\autoref{defn:truify}}
        \end{pfsteps*}
        By rule induction over \rulesref{rules:cSatisfy} on \pfref{ccsatisfy}, only one rule applies.
        \begin{byCases}
          \item[\text{(\ref{rule:CSInr})}]
            \begin{pfsteps*}
            \item $e = \hinr{\tau_1}{e_2}$ \BY{assumption}
            \item $\ccsatisfy{e_2}{\ctruify{\hxi_2}}$ \BY{assumption} \pflabel{[inr]satisfy-falsify2}
            \end{pfsteps*}
            By rule induction over \rulesref{rules:Value} on \pfref{val}, only one rule applies.
            \begin{byCases}
              \item[\text{(\ref{rule:VInr})}]
              \begin{pfsteps*}
              \item $\isVal{e_2}$ \BY{assumption} \pflabel{[inr]val2}
              \item $\csatisfyormay{e_2}{\hxi_2}$ \BY{IH on \pfref{[inr]val2} and \pfref{[inr]satisfy-falsify2}} \pflabel{[inr]satisfy2}
              \item $\csatisfyormay{\hinr{\tau_1}{e_2}}{\cinr{\hxi_2}}$ \BY{\autoref{lem:satormay-inr} on \pfref{[inr]satisfy2}}
              \end{pfsteps*} 
            \end{byCases}
        \end{byCases}
      
      \restorelocalsteps{1}
      \item[\hxi=\cpair{\hxi_1}{\hxi_2}]
        \begin{pfsteps*}
        \item $\ctruify{\cpair{\hxi_1}{\hxi_2}}=\cpair{\ctruify{\hxi_1}}{\ctruify{\hxi_2}}$ \BY{\autoref{defn:truify}}
        \end{pfsteps*}
        By rule induction over \rulesref{rules:cSatisfy} on \pfref{ccsatisfy}, only one rule applies.
        \begin{byCases}
          \item[\text{(\ref{rule:CSPair})}]
            \begin{pfsteps*}
            \item $e=\hpair{e_1}{e_2}$ \BY{assumption}
            \item $\ccsatisfy{e_1}{\cfalsify{\hxi_1}}$ \BY{assumption} \pflabel{[pair]satisfy-falsify1}
            \item $\ccsatisfy{e_2}{\cfalsify{\hxi_2}}$ \BY{assumption} \pflabel{[pair]satisfy-falsify2}
            \end{pfsteps*}
            By rule induction over \rulesref{rules:Value} on \pfref{val}, only one rule applies.
            \begin{byCases}
              \item[\text{(\ref{rule:VPair})}]
              \begin{pfsteps*}
              \item $\isVal{e_1}$ \BY{assumption} \pflabel{[pair]val1}
              \item $\isVal{e_2}$ \BY{assumption} \pflabel{[pair]val2}
              \item $\csatisfyormay{e_1}{\hxi_1}$ \BY{IH on \pfref{[pair]val1} and \pfref{[pair]satisfy-falsify1}} \pflabel{[pair]satisfy1}
              \item $\csatisfyormay{e_2}{\hxi_2}$ \BY{IH on \pfref{[pair]val2} and \pfref{[pair]satisfy-falsify2}} \pflabel{[pair]satisfy2}
              \item $\csatisfyormay{\hpair{e_1}{e_2}}{\cpair{\hxi_1}{\hxi_2}}$ \BY{\autoref{lem:satormay-pair} on \pfref{[pair]satisfy1} and \pfref{[pair]satisfy2}}
              \end{pfsteps*}
            \end{byCases}
        \end{byCases}
    \end{byCases}
  \end{enumerate}
  \resetpfcounter
\end{proof}

\begin{lemma}
  \label{lem:satisfy-falsify}
  $\csatisfy{e}{\hxi}$ iff $\ccsatisfy{e}{\cfalsify{\hxi}}$
\end{lemma}
\begin{proof}
  We prove sufficiency and necessity separately.
  \begin{enumerate}
    \item Sufficiency:
    \begin{pfsteps*}
    \item $\csatisfy{e}{\hxi}$ \BY{assumption} \pflabel{satisfy}
    \end{pfsteps*}
    By rule induction over Rules (\ref{rules:Satisfy}) on \pfref{satisfy}.
    \begin{byCases}
      
      \savelocalsteps{1}
      \item[\text{(\ref{rule:CSTruth})}]
        \begin{pfsteps*}
        \item $\hxi = \ctruth$ \BY{assumption}
        \item $\cfalsify{\ctruth}=\ctruth$ \BY{Definition \ref{defn:falsify}}
        \item $\ccsatisfy{e}{\ctruth}$ \BY{\ruleref{rule:CCSTruth}}
        \end{pfsteps*}

      \restorelocalsteps{1}
      \item[\text{(\ref{rule:CSNum})}]
        \begin{pfsteps*}
        \item $\hxi = \cnum{n}$ \BY{assumption}
        \item $e = \hnum{n}$ \BY{assumption}
        \item $\cfalsify{\hnum{n}}=\cnum{n}$ \BY{Definition \ref{defn:falsify}}
        \item $\ccsatisfy{\hnum{n}}{\cnum{n}}$ \BY{\ruleref{rule:CCSNum}}
        \end{pfsteps*}

      \restorelocalsteps{1}
      \item[\text{(\ref{rule:CSOr1})}]
        \begin{pfsteps*}
        \item $\hxi = \cor{\hxi_1}{\hxi_2}$ \BY{assumption}
        \item $\csatisfy{e}{\hxi_1}$ \BY{assumption} \pflabel{[or]satisfy1}
        \item $\cfalsify{\cor{\hxi_1}{\hxi_2}}=\cor{\cfalsify{\hxi_1}}{\cfalsify{\hxi_2}}$ \BY{Definition \ref{defn:falsify}}
        \item $\ccsatisfy{e}{\cfalsify{\hxi_1}}$ \BY{IH on \pfref{[or]satisfy1}} \pflabel{[or]satisfy-falsify1}
        \item $\ccsatisfy{e}{\cor{\cfalsify{\hxi_1}}{\cfalsify{\hxi_2}}}$ \BY{Rule (\ref{rule:CCSOr1}) on \pfref{[or]satisfy-falsify1}}
        \end{pfsteps*}
      
      \restorelocalsteps{1}
      \item[\text{(\ref{rule:CSOr2})}]
        \begin{pfsteps*}
        \item $\hxi = \cor{\hxi_1}{\hxi_2}$ \BY{assumption}
        \item $\csatisfy{e}{\hxi_2}$ \BY{assumption} \pflabel{[or]satisfy2}
        \item $\cfalsify{\cor{\hxi_1}{\hxi_2}}=\cor{\cfalsify{\hxi_1}}{\cfalsify{\hxi_2}}$ \BY{Definition \ref{defn:falsify}}
        \item $\ccsatisfy{e}{\cfalsify{\hxi_2}}$ \BY{IH on \pfref{[or]satisfy2}} \pflabel{[or]satisfy-falsify2}
        \item $\ccsatisfy{e}{\cor{\cfalsify{\hxi_1}}{\cfalsify{\hxi_2}}}$ \BY{Rule (\ref{rule:CCSOr2}) on \pfref{[or]satisfy-falsify2}}
        \end{pfsteps*}

      \restorelocalsteps{1}
      \item[\text{(\ref{rule:CSInl})}]
        \begin{pfsteps*}
        \item $e = \hinl{\tau_2}{e_1}$ \BY{assumption}
        \item $\hxi = \cinl{\hxi_1}$ \BY{assumption}
        \item $\csatisfy{e_1}{\hxi_1}$ \BY{assumption} \pflabel{[inl]satisfy1}
        \item $\cfalsify{\cinl{\hxi_1}} = \cinl{\cfalsify{\hxi_1}}$ \BY{\autoref{defn:falsify}}
        \item $\ccsatisfy{e_1}{\cfalsify{\hxi_1}}$ \BY{IH on \pfref{[inl]satisfy1}} \pflabel{[inl]satisfy-falsify1}
        \item $\ccsatisfy{\hinl{\tau_2}{e_1}}{\cinl{\cfalsify{\hxi_1}}}$ \BY{Rule (\ref{rule:CCSInl}) on \pfref{[inl]satisfy-falsify1}}
        \end{pfsteps*}

      \restorelocalsteps{1}
      \item[\text{(\ref{rule:CSInr})}]
        \begin{pfsteps*}
        \item $e = \hinr{\tau_1}{e_2}$ \BY{assumption}
        \item $\hxi = \cinr{\hxi_2}$ \BY{assumption}
        \item $\csatisfy{e_2}{\hxi_2}$ \BY{assumption} \pflabel{[inr]satisfy2}
        \item $\cfalsify{\cinr{\hxi_2}} = \cinr{\cfalsify{\hxi_2}}$ \BY{\autoref{defn:falsify}}
        \item $\ccsatisfy{e_2}{\cfalsify{\hxi_2}}$ \BY{IH on \pfref{[inr]satisfy2}} \pflabel{[inr]satisfy-falsify2}
        \item $\ccsatisfy{\hinr{\tau_1}{e_2}}{\cinr{\cfalsify{\hxi_2}}}$ \BY{Rule (\ref{rule:CCSInr}) on \pfref{[inr]satisfy-falsify2}}
        \end{pfsteps*}
      
      \restorelocalsteps{1}
      \item[\text{(\ref{rule:CSPair})}]
        \begin{pfsteps*}
        \item $e = \hpair{e_1}{e_2}$ \BY{assumption}
        \item $\hxi = \cpair{\hxi_1}{\hxi_2}$ \BY{assumption}
        \item $\csatisfy{e_1}{\hxi_1}$ \BY{assumption} \pflabel{[pair]satisfy1}
        \item $\csatisfy{e_2}{\hxi_2}$ \BY{assumption} \pflabel{[pair]satisfy2}
        \item $\cfalsify{\cpair{\hxi_1}{\hxi_2}}=\cpair{\cfalsify{\hxi_1}}{\cfalsify{\hxi_2}}$ \BY{Definition \ref{defn:falsify}}
        \item $\ccsatisfy{e_1}{\cfalsify{\hxi_1}}$ \BY{IH on \pfref{[pair]satisfy1}} \pflabel{[pair]satisfy-falsify1}
        \item $\ccsatisfy{e_2}{\cfalsify{\hxi_2}}$ \BY{IH on \pfref{[pair]satisfy2}} \pflabel{[pair]satisfy-falsify2}
        \item $\ccsatisfy{\hpair{e_1}{e_2}}{\cpair{\cfalsify{\hxi_1}}{\cfalsify{\hxi_2}}}$ \BY{Rule (\ref{rule:CCSPair}) on \pfref{[pair]satisfy-falsify1} and \pfref{[pair]satisfy-falsify2}}
        \end{pfsteps*}
    \end{byCases}

    \resetpfcounter

    \item Necessity:
    \begin{pfsteps*}
    \item $\ccsatisfy{e}{\cfalsify{\hxi}}$ \BY{assumption} \pflabel{satisfy-falsify}
    \end{pfsteps*}
    By structural induction on $\hxi$.
    \begin{byCases}

      \savelocalsteps{1}
      \item[\hxi=\ctruth]
      \begin{pfsteps*}
      \item $\cfalsify{\ctruth} = \ctruth$ \BY{\autoref{defn:falsify}}
      \item $\csatisfy{e}{\ctruth}$ \BY{\ruleref{rule:CSTruth}}
      \end{pfsteps*}

      \restorelocalsteps{1}
      \item[\hxi=\cunknown]
      \begin{pfsteps*}
      \item $\cfalsify{\cunknown} = \cfalsity$ \BY{\autoref{defn:falsify}}
      \end{pfsteps*}
      By rule induction over \rulesref{rules:cSatisfy} on \pfref{satisfy-falsify}, no rule applies due to syntactic contradiction.
        
      \restorelocalsteps{1}
      \item[\hxi=\cnum{n}]
      \begin{pfsteps*}
      \item $\cfalsify{\cnum{n}} = \cnum{n}$ \BY{\autoref{defn:falsify}}
      \end{pfsteps*}
      By rule induction over \rulesref{rules:cSatisfy} on \pfref{satisfy-falsify}, only one rule applies.
      \begin{byCases}
        \item[\text{(\ref{rule:CCSNum})}]
        \begin{pfsteps}
        \item e = \hnum{n} \BY{assumption}
        \item \csatisfy{\hnum{n}}{\cnum{n}} \BY{\ruleref{rule:CSNum}}
        \end{pfsteps}
      \end{byCases}

      \restorelocalsteps{1}
      \item[\hxi = \cor{\hxi_1}{\hxi_2}]
      \begin{pfsteps*}
      \item $\cfalsify{\cor{\hxi_1}{\hxi_2}} = \cor{\cfalsify{\hxi_1}}{\cfalsify{\hxi_2}}$ \BY{\autoref{defn:falsify}}
      \end{pfsteps*}
      By rule induction over \rulesref{rules:cSatisfy} on \pfref{satisfy-falsify}, only two rules apply.
      \begin{byCases}

        \savelocalsteps{2}
        \item[\text{(\ref{rule:CCSOr1})}]
        \begin{pfsteps*}
        \item $\ccsatisfy{e}{\cfalsify{\hxi_1}}$ \BY{assumption} \pflabel{[or]satisfy-falsify1}
        \item $\csatisfy{e}{\hxi_1}$ \BY{IH on \pfref{[or]satisfy-falsify1}} \pflabel{[or]satisfy1}
        \item $\csatisfy{e}{\cor{\hxi_1}{\hxi_2}}$ \BY{Rule (\ref{rule:CSOr1}) on \pfref{[or]satisfy1}}
        \end{pfsteps*}

        \restorelocalsteps{2}
        \item[\text{(\ref{rule:CCSOr2})}]
        \begin{pfsteps*}
        \item $\ccsatisfy{e}{\cfalsify{\hxi_2}}$ \BY{assumption} \pflabel{[or]satisfy-falsify2}
        \item $\csatisfy{e}{\hxi_2}$ \BY{IH on \pfref{[or]satisfy-falsify2}} \pflabel{[or]satisfy2}
        \item $\csatisfy{e}{\cor{\hxi_1}{\hxi_2}}$ \BY{Rule (\ref{rule:CSOr2}) on \pfref{[or]satisfy2}}
        \end{pfsteps*}
      \end{byCases}
    
    \restorelocalsteps{1}
    \item[\hxi=\cinl{\hxi_1}]
    \begin{pfsteps*}
    \item $\cfalsify{\cinl{\hxi_1}} = \cinl{\cfalsify{\hxi_1}}$ \BY{\autoref{defn:falsify}}
    \end{pfsteps*}
    By rule induction over \rulesref{rules:cSatisfy} on \pfref{satisfy-falsify}, only one rule applies.
    \begin{byCases}
      \item[\text{(\ref{rule:CCSInl})}]
      \begin{pfsteps*}
      \item $e = \hinl{\tau_2}{e_1}$ \BY{assumption}
      \item $\ccsatisfy{e_1}{\cfalsify{\hxi_1}}$ \BY{assumption} \pflabel{[inl]satisfy-falsify1}
      \item $\csatisfy{e_1}{\hxi_1}$ \BY{IH on \pfref{[inl]satisfy-falsify1}} \pflabel{[inl]satisfy1}
      \item $\csatisfy{e}{\cinl{\hxi_1}}$ \BY{Rule (\ref{rule:CSInl}) on \pfref{[inl]satisfy1}}
      \end{pfsteps*} 
    \end{byCases}

    \restorelocalsteps{1}
    \item[\hxi=\cinr{\hxi_2}]
    \begin{pfsteps*}
    \item $\cfalsify{\cinr{\hxi_2}} = \cinr{\cfalsify{\hxi_2}}$ \BY{\autoref{defn:falsify}}
    \end{pfsteps*}
    By rule induction over \rulesref{rules:cSatisfy} on \pfref{satisfy-falsify}, only one rule applies.
    \begin{byCases}
      \item[\text{(\ref{rule:CCSInr})}]
      \begin{pfsteps*}
      \item $e = \hinr{\tau_1}{e_2}$ \BY{assumption}
      \item $\ccsatisfy{e_2}{\cfalsify{\hxi_2}}$ \BY{assumption} \pflabel{[inr]satisfy-falsify2}
      \item $\csatisfy{e_2}{\hxi_2}$ \BY{IH on \pfref{[inr]satisfy-falsify2}} \pflabel{[inr]satisfy2}
      \item $\csatisfy{e}{\cinr{\hxi_2}}$ \BY{Rule (\ref{rule:CSInr}) on \pfref{[inr]satisfy2}}
      \end{pfsteps*} 
    \end{byCases}
    
    \restorelocalsteps{1}
    \item[\hxi=\cpair{\hxi_1}{\hxi_2}]
    \begin{pfsteps*}
    \item $\cfalsify{\cpair{\hxi_1}{\hxi_2}} = \cpair{\cfalsify{\hxi_1}}{\cfalsify{\hxi_2}}$ \BY{\autoref{defn:falsify}}
    \end{pfsteps*}
    By rule induction over \rulesref{rules:cSatisfy} on \pfref{satisfy-falsify}, only one rule applies.
    \begin{byCases}
      \item[\text{(\ref{rule:CSPair})}]
      \begin{pfsteps*}
      \item $e=\hpair{e_1}{e_2}$ \BY{assumption}
      \item $\ccsatisfy{e_1}{\cfalsify{\hxi_1}}$ \BY{assumption} \pflabel{[pair]satisfy-falsify1}
      \item $\ccsatisfy{e_2}{\cfalsify{\hxi_2}}$ \BY{assumption} \pflabel{[pair]satisfy-falsify2}
      \item $\csatisfy{e_1}{\hxi_1}$ \BY{IH on \pfref{[pair]satisfy-falsify1}} \pflabel{[pair]satisfy1}
      \item $\csatisfy{e_2}{\hxi_2}$ \BY{IH on \pfref{[pair]satisfy-falsify2}} \pflabel{[pair]satisfy2}
      \item $\csatisfy{e}{\cpair{\hxi_1}{\hxi_2}}$ \BY{Rule (\ref{rule:CSPair}) on \pfref{[pair]satisfy1} and \pfref{[pair]satisfy2}}
      \end{pfsteps*}
    \end{byCases} 
  \end{byCases}
  \resetpfcounter
  \end{enumerate}
\end{proof}

\begin{lemma}
  \label{lem:val-final-satormay}
  Suppose $\ctyp{\hxi}{\tau}$. Then $\csatisfyormay{e}{\hxi}$ for all $e$ such that $\hexptyp{\cdot}{\Delta}{e}{\tau}$ and $\isFinal{e}$ \textbf{iff} $\csatisfyormay{e}{\hxi}$ for all $e$ such that $\hexptyp{\cdot}{\Delta}{e}{\tau}$ and $\isVal{e}$.
\end{lemma}
\begin{proof}
  The sufficiency can be easily proved by observing that $\isVal{e}$ implies $\isFinal{e}$, due to \ruleref{rule:FVal}.
  
  Now, we want to prove the necessity. By rule induction over \rulesref{rules:Final}, we notice that $\isFinal{e}$ iff $\isVal{e}$ or $\isIndet{e}$. Therefore, we only need to prove that if $\csatisfyormay{e}{\hxi}$ for all $e$ such that $\hexptyp{\cdot}{\Delta}{e}{\tau}$ and $\isVal{e}$, then $\csatisfyormay{e}{\hxi}$ for all $e$ such that $\hexptyp{\cdot}{\Delta}{e}{\tau}$ and $\isIndet{e}$.

  By \autoref{lem:sound-complete-satormay}, we know that $\csatisfyormay{e}{\hxi}$ is decidable. Therefore, we just need to prove the contraposition of the above statement \textemdash if there exists $e$ such that $\isIndet{e}$ and $\hexptyp{\cdot}{\Delta}{e}{\tau}$ and $\cnotsatisfyormay{e}{\hxi}$, then there exists $e$ such that $\isVal{e}$ and $\hexptyp{\cdot}{\Delta}{e}{\tau}$ and $\cnotsatisfyormay{e}{\hxi}$.
  
  Now, we just need to show the implication.
  \begin{pfsteps}
  \item \ctyp{\hxi}{\tau} \BY{assumption} \pflabel{1}
  \item \isIndet{e} \BY{assumption} \pflabel{2}
  \item \hexptyp{\cdot}{\Delta}{e}{\tau} \BY{assumption} \pflabel{3}
  \item \cnotsatisfyormay{e}{\hxi} \BY{assumption} \pflabel{4}
  \item \inValues{e'}{e} \text{ implies } \cnotsatisfyormay{e'}{\hxi} \BY{\autoref{lem:complete-not-satormay} on \pfref{1} and \pfref{2} and \pfref{3} and \pfref{4}} \pflabel{notsat}
  \item \inValues{e'}{e} \BY{\autoref{lem:invalues-derivable} on \pfref{2}} \pflabel{invalues}
  \item \hexptyp{\cdot}{\Delta}{e'}{\tau} \BY{\autoref{lem:invalues-typ} on \pfref{invalues} and \pfref{3}}
  \item \isVal{e'} \BY{\autoref{lem:invalues-val} on \pfref{invalues}}
  \item \cnotsatisfyormay{e'}{\hxi} \BY{\pfref{notsat} on \pfref{invalues}}
  \end{pfsteps}
  \resetpfcounter
\end{proof}

\begin{theorem}
\label{thrm:demystify-exhaustiveness}
$\csatisfyormay{\ctruth}{\hxi}$ iff $\ccsatisfy{\ctruth}{\ctruify{\hxi}}$.
\end{theorem}
\begin{proof}\mbox{}\\

  By \ruleref{rule:CSTruth} and \ruleref{rule:CSMSSat}, we have $\csatisfyormay{e}{\ctruth}$ for any $e$.
  By \ruleref{rule:CCSTruth}, we have $\ccsatisfy{e}{\ctruth}$ for any $e$.

  Therefore, by \autoref{defn:nn-entailment}, $\csatisfyormay{\ctruth}{\hxi}$ iff $\csatisfyormay{e}{\hxi}$ for all $e$ such that $\hexptyp{\cdot}{\Delta}{e}{\tau}$ and $\isFinal{e}$.

  And by \autoref{defn:complete-constraint-entailment}, $\ccsatisfy{\ctruth}{\ctruify{\hxi}}$ 
  iff $\ccsatisfy{e}{\ctruify{\hxi}}$ for all $e$ such that $\hexptyp{\cdot}{\Delta}{e}{\tau}$ and $\isVal{e}$, 
  by \autoref{lem:val-satisfy-truify}, iff $\csatisfyormay{e}{\hxi}$ for all $e$ such that $\hexptyp{\cdot}{\Delta}{e}{\tau}$ and $\isVal{e}$.
  
  And the equivalence between the two is proved by \autoref{lem:val-final-satormay}.
\end{proof}

\begin{theorem}
\label{thrm:demystify-redundancy}
$\csatisfy{\hxi_1}{\hxi_2}$ iff $\ccsatisfy{\ctruify{\hxi_1}}{\cfalsify{\hxi_2}}$.
\end{theorem}
\begin{proof}\mbox{}\\
  
  By \autoref{defn:const-entailment}, $\csatisfy{\hxi_1}{\hxi_2}$ iff for all $e$ such that $\isVal{e}$ and $\hexptyp{\cdot}{\Delta}{e}{\tau}$, $\csatisfyormay{e}{\hxi_1}$ implies $\csatisfy{e}{\hxi_2}$.
  
  By \autoref{defn:complete-constraint-entailment}, $\ccsatisfy{\ctruify{\hxi_1}}{\cfalsify{\hxi_2}}$ iff for all $e$ such that $\isVal{e}$ and $\hexptyp{\cdot}{\Delta}{e}{\tau}$, $\ccsatisfy{e}{\ctruify{\hxi_1}}$ implies $\ccsatisfy{e}{\cfalsify{\hxi_2}}$.
  
  And the equivalence between the two is proved by \autoref{lem:val-satisfy-truify} and \autoref{lem:satisfy-falsify}.
\end{proof}
\pagebreak
\section{Dynamic Semantics}
\judgboxa{\isVal{e}}{$e$ is a value}
\begin{subequations}\label{rules:Value}
\begin{equation}
\inferrule[VNum]{ }{
  \isVal{\hnum{n}}
}
\end{equation}
\begin{equation}
\inferrule[VLam]{ }{
  \isVal{\hlam{x}{\tau}{e}}
}
\end{equation}
\begin{equation}
\inferrule[VPair]{
  \isVal{e_1} \\
  \isVal{e_2}
}{\isVal{\hpair{e_1}{e_2}}}
\end{equation}
\begin{equation}
\inferrule[VInl]{
  \isVal{e}
}{
  \isVal{\hinl{\tau}{e}}
}
\end{equation}
\begin{equation}
\inferrule[Vinr]{
  \isVal{e}
}{
  \isVal{\hinr{\tau}{e}}
}
\end{equation}
\end{subequations}

\judgboxa{\isIndet{e}}{$e$ is indeterminate}
\begin{subequations}\label{rules:Indet}
\begin{equation}\label{rule:IEHole}
\inferrule[IEHole]{ }{
  \isIndet{\hehole{u}}
}
\end{equation}
\begin{equation}\label{rule:IHole}
\inferrule[IHole]{
  \isFinal{e}
}{
  \isIndet{\hhole{e}{u}}
}
\end{equation}
\begin{equation}\label{rule:IAp}
\inferrule[IAp]{
  \isIndet{e_1} \\ \isFinal{e_2}
}{
  \isIndet{\hap{e_1}{e_2}}
}
\end{equation}
\begin{equation}\label{rule:IPairL}
\inferrule[IPairL]{
  \isIndet{e_1} \\ \isVal{e_2}
}{
  \isIndet{\hpair{e_1}{e_2}}
}
\end{equation}
\begin{equation}\label{rule:IPairR}
\inferrule[IPairR]{
  \isVal{e_1} \\
  \isIndet{e_2}
}{
  \isIndet{\hpair{e_1}{e_2}}
}
\end{equation}
\begin{equation}
\inferrule[IPair]{
  \isIndet{e_1} \\ \isIndet{e_2}
}{
  \isIndet{\hpair{e_1}{e_2}}
}
\end{equation}
\begin{equation}
\inferrule[IPrl]{
  \isIndet{e}
}{
  \isIndet{\hprl{e}}
}
\end{equation}
\begin{equation}
\inferrule[IPrr]{
  \isIndet{e}
}{
  \isIndet{\hprr{e}}
}
\end{equation}
\begin{equation}\label{rule:IInl}
\inferrule[IInL]{
  \isIndet{e}
}{
  \isIndet{\hinl{\tau}{e}}
}
\end{equation}
\begin{equation}\label{rule:IInR}
\inferrule[IInR]{
  \isIndet{e}
}{
  \isIndet{\hinr{\tau}{e}}
}
\end{equation}
\begin{equation}\label{rule:IMatch}
\inferrule[IMatch]{
  \isFinal{e} \\
  \hmaymatch{e}{p_r}
}{
  \isIndet{
    \hmatch{e}{\zruls{rs_{pre}}{\hrulP{p_r}{e_r}}{rs_{post}}}
  }
}
\end{equation}
\end{subequations}

\judgboxa{\isFinal{e}}{$e$ is final}
\begin{subequations}\label{rules:Final}
  \begin{equation}\label{rule:FVal}
\inferrule[FVal]{
  \isVal{e}
}{
  \isFinal{e}
}
\end{equation}
\begin{equation}\label{rule:FIndet}
\inferrule[FIndet]{
  \isIndet{e}
}{
  \isFinal{e}
}
\end{equation}
\end{subequations}

\judgboxa{
  \isntVal{e}
}{
  $e$ cannot be a value syntactically
}
\begin{subequations}
\begin{equation}
\inferrule[NVEHole]{ }{
  \isntVal{\hehole{u}}
}
\end{equation}
\begin{equation}
\inferrule[NVHole]{ }{
  \isntVal{\hhole{e}{u}}
}
\end{equation}
\begin{equation}
\inferrule[NVAp]{ }{
  \isntVal{\hap{e_1}{e_2}}
}
\end{equation}
\begin{equation}
\inferrule[NVMatch]{ }{
  \isntVal{\hmatch{e}{\zrules}}
}
\end{equation}
\begin{equation}
\inferrule[NVPrl]{ }{
  \isntVal{\hprl{e}}
}
\end{equation}
\begin{equation}
\inferrule[NVPrr]{ }{
  \isntVal{\hprr{e}}
}
\end{equation}
\end{subequations}

\judgboxa{
  \hsubstyp{\theta}{\Gamma}
}{
  $\theta$ is of type $\Gamma$
}
\begin{subequations}
\begin{equation}
\inferrule[STEmpty]{ }{
  \hsubstyp{\emptyset}{\cdot}
}
\end{equation}
\begin{equation}
\inferrule[STExtend]{
  \hsubstyp{\theta}{\Gamma_\theta} \\
  \hexptyp{\Gamma}{\Delta}{e}{\tau}
}{
  \hsubstyp{\theta , x / e}{\Gamma_\theta , x : \tau}
}
\end{equation}
\end{subequations}

\judgboxa{
  \refutable{p}
}{$p$ is refutable}
\begin{subequations}
\begin{equation}
\inferrule[RNum]{ }{
  \refutable{\hnum{n}}
}
\end{equation}
\begin{equation}
\inferrule[REHole]{ }{
  \refutable{\hehole{w}}
}
\end{equation}
\begin{equation}
\inferrule[RHole]{ }{
  \refutable{\hhole{p}{w}}
}
\end{equation}
\begin{equation}
\inferrule[RInl]{ }{
  \refutable{\hinlp{p}}
}
\end{equation}
\begin{equation}
\inferrule[RInr]{ }{
  \refutable{\hinrp{p}}
}
\end{equation}
\begin{equation}
\inferrule[RPairL]{
  \refutable{p_1}
}{
  \refutable{\hpair{p_1}{p_2}}
}
\end{equation}
\begin{equation}
\inferrule[RPairR]{
  \refutable{p_2}
}{
  \refutable{\hpair{p_1}{p_2}}
}
\end{equation}
\end{subequations}

\judgboxa{
  \hpatmatch{e}{p}{\theta}
}{
  $e$ matches $p$, emitting $\theta$
}
\begin{subequations}\label{rules:match}
\begin{equation}\label{rule:MVar}
\inferrule[MVar]{ }{
  \hpatmatch{e}{x}{e / x}
}
\end{equation}
\begin{equation}\label{rule:MWild}
\inferrule[MWild]{ }{
  \hpatmatch{e}{\_}{\cdot}
}
\end{equation}
\begin{equation}\label{rule:MNum}
\inferrule[MNum]{ }{
  \hpatmatch{\hnum{n}}{\hnum{n}}{\cdot}
}
\end{equation}
\begin{equation}\label{rule:MPair}
\inferrule[MPair]{
  \hpatmatch{e_1}{p_1}{\theta_1} \\
  \hpatmatch{e_2}{p_2}{\theta_2}
}{
  \hpatmatch{\hpair{e_1}{e_2}}{\hpair{p_1}{p_2}}{\theta_1 \uplus \theta_2}
}
\end{equation}
\begin{equation}\label{rule:MInl}
\inferrule[MInl]{
  \hpatmatch{e}{p}{\theta}
}{
  \hpatmatch{\hinl{\tau}{e}}{\hinlp{p}}{\theta}
}
\end{equation}
\begin{equation}\label{rule:MInr}
\inferrule[MInr]{
  \hpatmatch{e}{p}{\theta}
}{
  \hpatmatch{\hinr{\tau}{e}}{\hinrp{p}}{\theta}
}
\end{equation}
\begin{equation}
\inferrule[MEHolePair]{
  \hpatmatch{\hprl{\hehole{u}}}{p_1}{\theta_1} \\
  \hpatmatch{\hprr{\hehole{u}}}{p_2}{\theta_2}
}{
  \hpatmatch{\hehole{u}}{\hpair{p_1}{p_2}}{\theta_1 \uplus \theta_2}
}
\end{equation}
\begin{equation}
\inferrule[MHolePair]{
  \hpatmatch{\hprl{\hhole{e}{u}}}{p_1}{\theta_1} \\
  \hpatmatch{\hprr{\hhole{e}{u}}}{p_2}{\theta_2}
}{
  \hpatmatch{\hhole{e}{u}}{\hpair{p_1}{p_2}}{\theta_1 \uplus \theta_2}
}
\end{equation}
\begin{equation}
\inferrule[MApPair]{
  \hpatmatch{\hprl{\hap{e_1}{e_2}}}{p_1}{\theta_1} \\
  \hpatmatch{\hprr{\hap{e_1}{e_2}}}{p_2}{\theta_2}
}{
  \hpatmatch{\hap{e_1}{e_2}}{\hpair{p_1}{p_2}}{\theta_1 \uplus \theta_2}
}
\end{equation}
\begin{equation}
\inferrule[MMatchPair]{
  \hpatmatch{\hprl{\hmatch{e}{\zrules}}}{p_1}{\theta_1} \\
  \hpatmatch{\hprr{\hmatch{e}{\zrules}}}{p_2}{\theta_2}
}{
  \hpatmatch{\hmatch{e}{\zrules}}{\hpair{p_1}{p_2}}{\theta_1 \uplus \theta_2}
}
\end{equation}
\begin{equation}
\inferrule[MPrlPair]{
  \hpatmatch{\hprl{\hprl{e}}}{p_1}{\theta_1} \\
  \hpatmatch{\hprr{\hprl{e}}}{p_2}{\theta_2}
}{
  \hpatmatch{\hprl{e}}{\hpair{p_1}{p_2}}{\theta_1 \uplus \theta_2}
}
\end{equation}
\begin{equation}
\inferrule[MPrrPair]{
  \hpatmatch{\hprl{\hprr{e}}}{p_1}{\theta_1} \\
  \hpatmatch{\hprr{\hprr{e}}}{p_2}{\theta_2}
}{
  \hpatmatch{\hprr{e}}{\hpair{p_1}{p_2}}{\theta_1 \uplus \theta_2}
}
\end{equation}
\end{subequations}

\judgboxa{
  \hmaymatch{e}{p}
}{
  $e$ may match $p$
}
\begin{subequations}\label{rules:maymatch}
\begin{equation}\label{rule:MMEHole}
\inferrule[MMEHole]{ }{
  \hmaymatch{e}{\hehole{w}}
}
\end{equation}
\begin{equation}\label{rule:MMHole}
\inferrule[MMHole]{ }{
  \hmaymatch{e}{\hhole{p}{w}}
}
\end{equation}
\begin{equation}\label{rule:MMExpEHole}
\inferrule[MMExpEHole]{
  \refutable{p}
}{
  \hmaymatch{\hehole{u}}{p}
}
\end{equation}
\begin{equation}\label{rule:MMExpHole}
\inferrule[MMExpHole]{
  \refutable{p}
}{
  \hmaymatch{\hhole{e}{u}}{p}
}
\end{equation}
\begin{equation}\label{rule:MMAp}
\inferrule[MMAp]{
  \refutable{p}
}{
  \hmaymatch{\hap{e_1}{e_2}}{p}
}
\end{equation}
\begin{equation}
\inferrule[MMMatch]{
  \refutable{p}
}{
  \hmaymatch{\hmatch{e}{\zrules}}{p}
}
\end{equation}
\begin{equation}
\inferrule[MMPrl]{
  \refutable{p}
}{
  \hmaymatch{\hprl{e}}{p}
}
\end{equation}
\begin{equation}
\inferrule[MMPrr]{
  \refutable{p}
}{
  \hmaymatch{\hprr{e}}{p}
}
\end{equation}
\begin{equation}\label{rule:MMPairL}
\inferrule[MMPairL]{
  \hmaymatch{e_1}{p_1} \\
  \hpatmatch{e_2}{p_2}{\theta_2}
}{
  \hmaymatch{\hpair{e_1}{e_2}}{\hpair{p_1}{p_2}}
}
\end{equation}
\begin{equation}\label{rule:MMPairR}
\inferrule[MMPairR]{
  \hpatmatch{e_1}{p_1}{\theta_1} \\
  \hmaymatch{e_2}{p_2}
}{
  \hmaymatch{\hpair{e_1}{e_2}}{\hpair{p_1}{p_2}}
}
\end{equation}
\begin{equation}\label{rule:MMPair}
\inferrule[MMPair]{
  \hmaymatch{e_1}{p_1} \\
  \hmaymatch{e_2}{p_2}
}{
  \hmaymatch{\hpair{e_1}{e_2}}{\hpair{p_1}{p_2}}
}
\end{equation}
\begin{equation}\label{rule:MMInl}
\inferrule[MMInl]{
  \hmaymatch{e}{p}
}{
  \hmaymatch{\hinl{\tau}{e}}{\hinlp{p}}
}
\end{equation}
\begin{equation}\label{rule:MMInr}
\inferrule[MMInr]{
  \hmaymatch{e}{p}
}{
  \hmaymatch{\hinr{\tau}{e}}{\hinrp{p}}
}
\end{equation}
\end{subequations}

\judgboxa{
  \hnotmatch{e}{p}
}{
  $e$ does not match $p$
}
\begin{subequations}\label{rules:notmatch}
\begin{equation}
\inferrule[NMNum]{
  n_1 \neq n_2
}{
  \hnotmatch{\hnum{n_1}}{\hnum{n_2}}
}
\end{equation}
\begin{equation}
\inferrule[NMPairL]{
  \hnotmatch{e_1}{p_1}
}{
  \hnotmatch{\hpair{e_1}{e_2}}{\hpair{p_1}{p_2}}
}
\end{equation}
\begin{equation}
\inferrule[NMPairR]{
  \hnotmatch{e_2}{p_2}
}{
  \hnotmatch{\hpair{e_1}{e_2}}{\hpair{p_1}{p_2}}
}
\end{equation}
\begin{equation}
\inferrule[NMConfL]{ }{
  \hnotmatch{\hinr{\tau}{e}}{\hinlp{p}}
}
\end{equation}
\begin{equation}
\inferrule[NMConfR]{ }{
  \hnotmatch{\hinl{\tau}{e}}{\hinrp{p}}
}
\end{equation}
\begin{equation}
\inferrule[NMInl]{
  \hnotmatch{e}{p}
}{
  \hnotmatch{\hinr{\tau}{e}}{\hinlp{p}}
}
\end{equation}
\begin{equation}
\inferrule[NMInr]{
  \hnotmatch{e}{p}
}{
  \hnotmatch{\hinl{\tau}{e}}{\hinrp{p}}
}
\end{equation}
\end{subequations}

\judgboxa{\htrans{e}{e'}}{$e$ takes a step to $e'$}
\begin{subequations}\label{rules:ITExp}
\begin{equation}
\inferrule[ITHole]{
  \htrans{e}{e'}
}{
  \htrans{\hhole{e}{u}}{\hhole{e'}{u}}
}
\end{equation}
\begin{equation}
\inferrule[ITApFun]{
  \htrans{e_1}{e_1'}
}{
  \htrans{\hap{e_1}{e_2}}{\hap{e_1'}{e_2}}
}
\end{equation}
\begin{equation}
\inferrule[ITApArg]{
  \isVal{e_1} \\
  \htrans{e_2}{e_2'}
}{
  \htrans{\hap{e_1}{e_2}}{\hap{e_1}{e_2'}}
}
\end{equation}
\begin{equation}
\inferrule[ITAP]{
  \isVal{e_2}
}{
  \hap{\hlam{x}{\tau}{e_1}}{e_2} \mapsto
    [e_2/x]e_1
}
\end{equation}
\begin{equation}
\inferrule[ITPairL]{
  \htrans{e_1}{e_1'}
}{
  \htrans{\hpair{e_1}{e_2}}{\hpair{e_1'}{e_2}}
}
\end{equation}
\begin{equation}
\inferrule[ITPairR]{
  \isVal{e_1} \\
  \htrans{e_2}{e_2'}
}{
  \htrans{\hpair{e_1}{e_2}}{\hpair{e_1}{e_2'}}
}
\end{equation}
\begin{equation}
\inferrule[ITPrl]{
  \isFinal{\hpair{e_1}{e_2}}
}{
  \htrans{\hprl{\hpair{e_1}{e_2}}}{e_1}
}
\end{equation}
\begin{equation}
\inferrule[ITPrr]{
  \isFinal{\hpair{e_1}{e_2}}
}{
  \htrans{\hprr{\hpair{e_1}{e_2}}}{e_2}
}
\end{equation}
\begin{equation}
\inferrule[ITInl]{
  \htrans{e}{e'}
}{
  \htrans{\hinl{\tau}{e}}{\hinl{\tau}{e'}}
}
\end{equation}
\begin{equation}
\inferrule[ITInr]{
  \htrans{e}{e'}
}{
  \htrans{\hinr{\tau}{e}}{\hinr{\tau}{e'}}
}
\end{equation}
\begin{equation}\label{rule:ITExpMatch}
\inferrule[ITExpMatch]{
  \htrans{e}{e'}
}{
  \htrans{\hmatch{e}{\zrules}}{\hmatch{e'}{\zrules}}
}
\end{equation}
\begin{equation}\label{rule:ITSuccMatch}
\inferrule[ITSuccMatch]{
  \isFinal{e} \\
  \hpatmatch{e}{p_r}{\theta}
}{
  \htrans{
    \hmatch{e}{\zruls{rs_{pre}}{\hrulP{p_r}{e_r}}{rs_{post}}}
  }{
    [\theta](e_r)
  }
}
\end{equation}
\begin{equation}\label{rule:ITFailMatch}
\inferrule[ITFailMatch]{
  \isFinal{e} \\
  \hnotmatch{e}{p_r}
}{
  \htrans{
    \hmatch{e}{\zruls{rs}{\hrulP{p_r}{e_r}}{\hrulesP{r'}{rs'}}}
  }{
    \hmatch{e}{
      \zruls{
        \rmpointer{\zruls{rs}{\hrulP{p_r}{e_r}}{\cdot}}
      }{r'}{rs'}
    }
  }
}
\end{equation}
\end{subequations}

\begin{lemma}[Matching Coherence of Constraint]
  \label{lem:const-matching-coherence}
  Suppose that $\hexptyp{\cdot}{\Delta_e}{e}{\tau}$ and $\isFinal{e}$ and $\chpattyp{p}{\tau}{\xi}{\Gamma}{\Delta}$. Then we have
  \begin{enumerate}
  \item $\csatisfy{e}{\xi}$ iff $\hpatmatch{e}{p}{\theta}$
  \item $\csatisfy{e}{\cdual{\xi}}$ iff $\hnotmatch{e}{p}$
  \item $\cmaysatisfy{e}{\xi}$ iff $\hmaymatch{e}{p}$
  \end{enumerate}
\end{lemma}
\begin{proof}
\begin{pfsteps*}
\item $\hexptyp{\cdot}{\Delta_e}{e}{\tau}$ \BY{assumption} \pflabel{eTyp}
\item $\isFinal{e}$ \BY{assumption} \pflabel{eFinal}
\item $\chpattyp{p}{\tau}{\xi}{\Gamma}{\Delta}$ \BY{assumption} \pflabel{patTyp}
\end{pfsteps*}
By rule induction over Rules (\ref{rules:PatTyp}) on \pfref{patTyp}.
\begin{byCases}
\savelocalsteps{0}
\item[\text{(\ref{rule:PTVar})}]
    \begin{pfsteps*}
    \item $p=x$ \BY{assumption}
    \item $\xi=\ctruth$ \BY{assumption}
    \end{pfsteps*}
    \begin{enumerate}
    \savelocalsteps{1}
    \item Prove $\csatisfy{e}{\ctruth}$ implies $\hpatmatch{e}{x}{\theta}$ for some $\theta$.
        \begin{pfsteps*}
        \item $\hpatmatch{e}{x}{e / x}$ \BY{Rule (\ref{rule:MVar})}
        \end{pfsteps*}
    \restorelocalsteps{1}
    \item Prove $\hpatmatch{e}{x}{\theta}$ implies $\csatisfy{e}{\ctruth}$.
        \begin{pfsteps*}
        \item $\csatisfy{e}{\ctruth}$ \BY{Rule (\ref{rule:CSTruth})}
        \end{pfsteps*}
    \restorelocalsteps{1}
    \item Prove $\csatisfy{e}{\cdual{\ctruth}}$ implies $\hnotmatch{e}{x}$.
        \begin{pfsteps*}
        \item $\cdual{\ctruth}=\cfalsity$ \BY{Definition \ref{defn:dual}}
        \item $\cnotsatisfy{e}{\cfalsity}$ \BY{\autoref{lem:no-e-satisfy-falsity}}
        \end{pfsteps*}
        Vacuously true.
    \restorelocalsteps{1}
    \item Prove $\hnotmatch{e}{x}$ implies $\csatisfy{e}{\cdual{\ctruth}}$.

        By rule induction over Rules (\ref{rules:notmatch}), we notice that $\hnotmatch{e}{x}$ is in syntactic contradiction with all the cases, hence not derivable. And thus vacuously true.
    \restorelocalsteps{1}
    \item Prove $\cmaysatisfy{e}{\ctruth}$ implies $\hmaymatch{e}{x}$.
        \begin{pfsteps*}
        \item $\cnotmaysatisfy{e}{\ctruth}$ \BY{\autoref{lem:no-e-may-satisfy-truth}}
        \end{pfsteps*}
        Vacuously true.
    \restorelocalsteps{1}
    \item Prove $\hmaymatch{e}{x}$ implies $\cmaysatisfy{e}{\xi}$.
    
        By rule induction over Rules (\ref{rules:maymatch}), we notice that either, $\hmaymatch{e}{x}$ is in syntactic contradiction with all the cases, or the premise $\refutable{x}$ is not derivable. Hence, $\hmaymatch{e}{x}$ are not derivable. And thus vacuously true.
    \end{enumerate}
    
\restorelocalsteps{0}
\item[\text{(\ref{rule:PTWild})}]
    \begin{pfsteps*}
    \item $p=\_$ \BY{assumption}
    \item $\xi=\ctruth$ \BY{assumption}
    \end{pfsteps*}
    \begin{enumerate}
    \savelocalsteps{1}
    \item Prove $\csatisfy{e}{\ctruth}$ implies $\hpatmatch{e}{\_}{\theta}$ for some $\theta$.
        \begin{pfsteps*}
        \item $\hpatmatch{e}{\_}{\cdot}$ \BY{Rule (\ref{rule:MVar})}
        \end{pfsteps*}
    \restorelocalsteps{1}
    \item Prove $\hpatmatch{e}{\_}{\theta}$ implies $\csatisfy{e}{\ctruth}$.
        \begin{pfsteps*}
        \item $\csatisfy{e}{\ctruth}$ \BY{Rule (\ref{rule:CSTruth})}
        \end{pfsteps*}
    \restorelocalsteps{1}
    \item Prove $\csatisfy{e}{\cdual{\ctruth}}$ implies $\hnotmatch{e}{\_}$.
        \begin{pfsteps*}
        \item $\cdual{\ctruth}=\cfalsity$ \BY{Definition \ref{defn:dual}}
        \item $\cnotsatisfy{e}{\cfalsity}$ \BY{\autoref{lem:no-e-satisfy-falsity}}
        \end{pfsteps*}
        Vacuously true.
    \restorelocalsteps{1}
    \item Prove $\hnotmatch{e}{\_}$ implies $\csatisfy{e}{\cdual{\ctruth}}$.

        By rule induction over Rules (\ref{rules:notmatch}), we notice that $\hnotmatch{e}{\_}$ is in syntactic contradiction with all the cases, hence not derivable. And thus vacuously true.
    \restorelocalsteps{1}
    \item Prove $\cmaysatisfy{e}{\ctruth}$ implies $\hmaymatch{e}{\_}$.
        \begin{pfsteps*}
        \item $\cnotmaysatisfy{e}{\ctruth}$ \BY{\autoref{lem:no-e-may-satisfy-truth}}
        \end{pfsteps*}
        Vacuously true.
    \restorelocalsteps{1}
    \item Prove $\hmaymatch{e}{\_}$ implies $\cmaysatisfy{e}{\xi}$.
    
        By rule induction over Rules (\ref{rules:maymatch}), we notice that either, $\hmaymatch{e}{\_}$ is in syntactic contradiction with all the cases, or the premise $\refutable{\_}$ is not derivable. Hence, $\hmaymatch{e}{\_}$ are not derivable. And thus vacuously true.
    \end{enumerate}
    
\restorelocalsteps{0}
\item[\text{(\ref{rule:PTEHole})}]
    \begin{pfsteps*}
    \item $p=\hehole{w}$ \BY{assumption}
    \item $\xi=\cunknown$ \BY{assumption}
    \item $\cdual{\xi}=\cunknown$ \BY{Definition \ref{defn:dual}}
    \end{pfsteps*}
    \begin{enumerate}
    \savelocalsteps{1}
    \item Prove $\csatisfy{e}{\cunknown}$ implies $\hpatmatch{e}{\hehole{w}}{\theta}$ for some $\theta$.
        \begin{pfsteps*}
        \item $\cnotsatisfy{e}{\cunknown}$ \BY{Rule (\ref{rule:MVar})}
        \end{pfsteps*}
        Vacuously true.
    \restorelocalsteps{1}
    \item Prove $\hpatmatch{e}{\hehole{w}}{\theta}$ implies $\csatisfy{e}{\cunknown}$.\\
        By rule induction over Rules (\ref{rules:match}), we notice that $\hpatmatch{e}{\hehole{w}}{\theta}$ is in syntactic contradiction with all the cases, hence not derivable. And thus vacuously true.
    \restorelocalsteps{1}
    \item Prove $\csatisfy{e}{\cunknown}$ implies $\hnotmatch{e}{\hehole{w}}$.
        \begin{pfsteps*}
        \item $\cnotsatisfy{e}{\cunknown}$ \BY{Rule (\ref{rule:MVar})}
        \end{pfsteps*}
        Vacuously true.
    \restorelocalsteps{1}
    \item Prove $\hnotmatch{e}{\hehole{w}}$ implies $\csatisfy{e}{\cunknown}$.\\
        By rule induction over Rules (\ref{rules:notmatch}), we notice that $\hnotmatch{e}{\hehole{w}}$ is in syntactic contradiction with all the cases, hence not derivable. And thus vacuously true.
    \restorelocalsteps{1}
    \item Prove $\cmaysatisfy{e}{\cunknown}$ implies $\hmaymatch{e}{\hehole{w}}$.
        \begin{pfsteps*}
        \item $\hmaymatch{e}{\hehole{w}}$ \BY{Rule (\ref{rule:MMEHole})}
        \end{pfsteps*}
    \restorelocalsteps{1}
    \item Prove $\hmaymatch{e}{\hehole{w}}$ implies $\cmaysatisfy{e}{\cunknown}$.
        \begin{pfsteps*}
        \item $\cmaysatisfy{e}{\cunknown}$ \BY{Rule (\ref{rule:CMSUnknown})}
        \end{pfsteps*}
    \end{enumerate}

\restorelocalsteps{0}
\item[\text{(\ref{rule:PTHole})}]
    \begin{pfsteps*}
    \item $p=\hhole{p_0}{w}$ \BY{assumption}
    \item $\xi=\cunknown$ \BY{assumption}
    \item $\cdual{\xi}=\cunknown$ \BY{Definition \ref{defn:dual}}
    \end{pfsteps*}
    \begin{enumerate}
    \savelocalsteps{1}
    \item Prove $\csatisfy{e}{\cunknown}$ implies $\hpatmatch{e}{\hhole{p_0}{w}}{\theta}$ for some $\theta$.
        \begin{pfsteps*}
        \item $\cnotsatisfy{e}{\cunknown}$ \BY{Rule (\ref{rule:MVar})}
        \end{pfsteps*}
        Vacuously true.
    \restorelocalsteps{1}
    \item Prove $\hpatmatch{e}{\hhole{p_0}{w}}{\theta}$ implies $\csatisfy{e}{\cunknown}$.\\
        By rule induction over Rules (\ref{rules:match}), we notice that $\hpatmatch{e}{\hhole{p_0}{w}}{\theta}$ is in syntactic contradiction with all the cases, hence not derivable. And thus vacuously true.
    \restorelocalsteps{1}
    \item Prove $\csatisfy{e}{\cunknown}$ implies $\hnotmatch{e}{\hhole{p_0}{w}}$.
        \begin{pfsteps*}
        \item $\cnotsatisfy{e}{\cunknown}$ \BY{Rule (\ref{rule:MVar})}
        \end{pfsteps*}
        Vacuously true.
    \restorelocalsteps{1}
    \item Prove $\hnotmatch{e}{\hhole{p_0}{w}}$ implies $\csatisfy{e}{\cunknown}$.\\
        By rule induction over Rules (\ref{rules:notmatch}), we notice that $\hnotmatch{e}{\hhole{p_0}{w}}$ is in syntactic contradiction with all the cases, hence not derivable. And thus vacuously true.
    \restorelocalsteps{1}
    \item Prove $\cmaysatisfy{e}{\cunknown}$ implies $\hmaymatch{e}{\hhole{p_0}{w}}$.
        \begin{pfsteps*}
        \item $\hmaymatch{e}{\hhole{p_0}{w}}$ \BY{Rule (\ref{rule:MMHole})}
        \end{pfsteps*}
    \restorelocalsteps{1}
    \item Prove $\hmaymatch{e}{\hhole{p_0}{w}}$ implies $\cmaysatisfy{e}{\cunknown}$.
        \begin{pfsteps*}
        \item $\cmaysatisfy{e}{\cunknown}$ \BY{Rule (\ref{rule:CMSUnknown})}
        \end{pfsteps*}
    \end{enumerate}
    
\restorelocalsteps{0}
\item[\text{(\ref{rule:PTNum})}]
    \begin{pfsteps*}
    \item $p=\hhole{p_0}{w}$ \BY{assumption}
    \item $\xi=\cunknown$ \BY{assumption}
    \item $\cdual{\xi}=\cunknown$ \BY{Definition \ref{defn:dual}}
    \end{pfsteps*}
    \begin{enumerate}
    \savelocalsteps{1}
    \item Prove $\csatisfy{e}{\cunknown}$ implies $\hpatmatch{e}{\hhole{p_0}{w}}{\theta}$ for some $\theta$.
        \begin{pfsteps*}
        \item $\cnotsatisfy{e}{\cunknown}$ \BY{Rule (\ref{rule:MVar})}
        \end{pfsteps*}
        Vacuously true.
    \restorelocalsteps{1}
    \item Prove $\hpatmatch{e}{\hhole{p_0}{w}}{\theta}$ implies $\csatisfy{e}{\cunknown}$.\\
        By rule induction over Rules (\ref{rules:match}), we notice that $\hpatmatch{e}{\hhole{p_0}{w}}{\theta}$ is in syntactic contradiction with all the cases, hence not derivable. And thus vacuously true.
    \restorelocalsteps{1}
    \item Prove $\csatisfy{e}{\cunknown}$ implies $\hnotmatch{e}{\hhole{p_0}{w}}$.
        \begin{pfsteps*}
        \item $\cnotsatisfy{e}{\cunknown}$ \BY{Rule (\ref{rule:MVar})}
        \end{pfsteps*}
        Vacuously true.
    \restorelocalsteps{1}
    \item Prove $\hnotmatch{e}{\hhole{p_0}{w}}$ implies $\csatisfy{e}{\cunknown}$.\\
        By rule induction over Rules (\ref{rules:notmatch}), we notice that $\hnotmatch{e}{\hhole{p_0}{w}}$ is in syntactic contradiction with all the cases, hence not derivable. And thus vacuously true.
    \restorelocalsteps{1}
    \item Prove $\cmaysatisfy{e}{\cunknown}$ implies $\hmaymatch{e}{\hhole{p_0}{w}}$.
        \begin{pfsteps*}
        \item $\hmaymatch{e}{\hhole{p_0}{w}}$ \BY{Rule (\ref{rule:MMHole})}
        \end{pfsteps*}
    \restorelocalsteps{1}
    \item Prove $\hmaymatch{e}{\hhole{p_0}{w}}$ implies $\cmaysatisfy{e}{\cunknown}$.
        \begin{pfsteps*}
        \item $\cmaysatisfy{e}{\cunknown}$ \BY{Rule (\ref{rule:CMSUnknown})}
        \end{pfsteps*}
    \end{enumerate}
\end{byCases}
\end{proof}
\section{Static Semantics}
$\arraycolsep=4pt\begin{array}{lll}
\tau & ::= &
  \tnum ~\vert~
  \tarr{\tau_1}{\tau_2} ~\vert~
  \tprod{\tau_1}{\tau_2} ~\vert~
  \tsum{\tau_1}{\tau_2} \\
e & ::= &
  x ~\vert~
  \hnum{n} \\
  & ~\vert~ &
  \hlam{x}{\tau}{e} ~\vert~
  \hap{e_1}{e_2} \\
  & ~\vert~ &
  \hpair{e_1}{e_2} \\
  & ~\vert~ &
  \hinl{\tau}{e} ~\vert~
  \hinr{\tau}{e} ~\vert~
  \hmatch{e}{\hat{rs}} \\
  & ~\vert~ &
  \hehole{u} ~\vert~
  \hhole{e}{u} \\
\hat{rs} & ::= &
  \inparens{\zruls{rs}{r}{rs}} \\
rs & ::= &
  \cdot ~\vert~ \hrulesP{r}{rs'} \\
r & ::= &
  \hrul{p}{e} \\
p & ::= &
  x ~\vert~
  \hnum{n} ~\vert~
  \_ ~\vert~
  \hpair{p_1}{p_2} ~\vert~
  \hinlp{p} ~\vert~
  \hinrp{p} ~\vert~
  \hehole{w} ~\vert~
  \hhole{p}{w}
\end{array}$

\judgboxa{\rmpointer{\zrules} = rs}
        {$rs$ can be obtained by erasing pointer from $\zrules$}
\begin{subequations}\label{defn:rmpointer}
\begin{align}
  \rmpointer{\zruls{\cdot}{r}{rs}} &= \hrules{r}{rs} \\
  \rmpointer{\zruls{\hrulesP{r'}{rs'}}{r}{rs}} &= \hrules{r'}{\rmpointer{\zruls{rs'}{r}{rs}}}
\end{align}
\end{subequations}

\judgboxa{
  \hexptyp{\Gamma}{\Delta}{e}{\tau}
}{
  $e$ is of type \(\tau\)
}
\begin{subequations}\label{rules:TExp}
\begin{equation}\label{rule:TVar}
\inferrule[TVar]{ }{
  \hexptyp{\Gamma, x : \tau}{\Delta}{x}{\tau}
}
\end{equation}
\begin{equation}\label{rule:TEHole}
\inferrule[TEHole]{ }{
  \hexptyp{\Gamma}{\Delta, u::\tau}{\hehole{u}}{\tau}
}
\end{equation}
\begin{equation}\label{rule:THole}
\inferrule[THole]{
  \hexptyp{\Gamma}{\Delta, u::\tau}{e}{\tau'}
}{
  \hexptyp{\Gamma}{\Delta, u::\tau}{\hhole{e}{u}}{\tau}
}
\end{equation}
\begin{equation}\label{rule:TNum}
\inferrule[TNum]{ }{
  \hexptyp{\Gamma}{\Delta}{\hnum{n}}{\tnum}
}
\end{equation}
\begin{equation}\label{rule:TLam}
\inferrule[TLam]{
  \hexptyp{\Gamma, x : \tau_1}{\Delta}{e}{\tau_2}
}{
  \hexptyp{\Gamma}{\Delta}{\hlam{x}{\tau_1}{e}}{\tarr{\tau_1}{\tau_2}}
}
\end{equation}
\begin{equation}\label{rule:TAp}
\inferrule[TAp]{
  \hexptyp{\Gamma}{\Delta}{e_1}{\tarr{\tau_2}{\tau}} \\
  \hexptyp{\Gamma}{\Delta}{e_2}{\tau_2}
}{
  \hexptyp{\Gamma}{\Delta}{\hap{e_1}{e_2}}{\tau}
}
\end{equation}
\begin{equation}\label{rule:TPair}
\inferrule[TPair]{
  \hexptyp{\Gamma}{\Delta}{e_1}{\tau_1} \\
  \hexptyp{\Gamma}{\Delta}{e_2}{\tau_2}
}{
  \hexptyp{\Gamma}{\Delta}{\hpair{e_1}{e_2}}{\tprod{\tau_1}{\tau_2}}
}
\end{equation}
\begin{equation}\label{rule:TPrl}
\inferrule[TPrl]{
    \hexptyp{\Gamma}{\Delta}{e}{\tprod{\tau_1}{\tau_2}}
}{
    \hexptyp{\Gamma}{\Delta}{\hprl{e}}{\tau_1}
} 
\end{equation}
\begin{equation}\label{rule:TPrr}
  \inferrule[TPrr]{
    \hexptyp{\Gamma}{\Delta}{e}{\tprod{\tau_1}{\tau_2}}
  }{
    \hexptyp{\Gamma}{\Delta}{\hprr{e}}{\tau_2}
  }
\end{equation}
\begin{equation}\label{rule:TRDenl}
\inferrule[TInl]{
  \hexptyp{\Gamma}{\Delta}{e}{\tau_1}
}{
  \hexptyp{\Gamma}{\Delta}{\hinl{\tau_2}{e}}{\tsum{\tau_1}{\tau_2}}
}
\end{equation}
\begin{equation}\label{rule:TInr}
\inferrule[TInr]{
  \hexptyp{\Gamma}{\Delta}{e}{\tau_2}
}{
  \hexptyp{\Gamma}{\Delta}{\hinr{\tau_1}{e}}{\tsum{\tau_1}{\tau_2}}
}
\end{equation}
\begin{equation}\label{rule:TMatchZPre}
\inferrule[TMatchZPre]{
  \hexptyp{\Gamma}{\Delta}{e}{\tau} \\
  \chrulstyp{\Gamma}{\Delta}{\cfalsity}{\hrules{r}{rs}}{\tau}{\xi}{\tau'} \\
  \csatisfyormay{\ctruth}{\xi}
}{
\hexptyp{\Gamma}{\Delta}{\hmatch{e}{\zruls{\cdot}{r}{rs}}}{\tau'}
}
\end{equation}
\begin{equation}\label{rule:TMatchNZPre}
\inferrule[TMatchNZPre]{
  \hexptyp{\Gamma}{\Delta}{e}{\tau} \\
  \isFinal{e} \\
  \chrulstyp{\Gamma}{\Delta}{\cfalsity}{rs_{pre}}{\tau}{\xi_{pre}}{\tau'} \\
  \chrulstyp{\Gamma}{\Delta}{\cor{\cfalsity}{\xi_{pre}}}{\hrules{r}{rs_{post}}}{\tau}{\xi_{rest}}{\tau'} \\
  \cnotsatisfyormay{e}{\xi_{pre}} \\
  \csatisfyormay{\ctruth}{\cor{\xi_{pre}}{\xi_{rest}}}
}{
  \hexptyp{\Gamma}{\Delta}{\hmatch{e}{\zruls{rs_{pre}}{r}{rs_{post}}}}{\tau'}
}
\end{equation}
\end{subequations}

\judgboxa{
    \chpattyp{p}{\tau}{\xi}{\Gamma}{\Delta}
  }{
    $p$ is assigned type $\tau$ and emits constraint $\xi$
  }
\begin{subequations}
\begin{equation}
\inferrule[PTVar]{ }{
  \chpattyp{x}{\tau}{\ctruth}{\cdot}{x : \tau}
}
\end{equation}
\begin{equation}
\inferrule[PTWild]{ }{
  \chpattyp{\_}{\tau}{\ctruth}{\cdot}{\cdot}
}
\end{equation}
\begin{equation}
\inferrule[PTEHole]{ }{
  \chpattyp{\hehole{w}}{\tau}{\cunknown}{\cdot}{w :: \tau}
}
\end{equation}
\begin{equation}
\inferrule[PTHole]{
  \chpattyp{p}{\tau}{\xi}{\Gamma}{\Delta}
}{
  \chpattyp{\hhole{p}{w}}{\tau'}{\cunknown}
  {\Gamma}{\Delta , w :: \tau'}
}
\end{equation}
\begin{equation}
\inferrule[PTNum]{ }{
  \chpattyp{\hnum{n}}{\tnum}{\cnum{n}}{\cdot}{\cdot}
}
\end{equation}
\begin{equation}
\inferrule[PTInl]{
  \chpattyp{p}{\tau_1}{\xi}{\Gamma}{\Delta}
}{
  \chpattyp{\hinlp{p}}{\tsum{\tau_1}{\tau_2}}{\cinl{\xi}}{\Gamma}{\Delta}
}
\end{equation}
\begin{equation}
\inferrule[PTInr]{
  \chpattyp{p}{\tau_2}{\xi}{\Gamma}{\Delta}
}{
  \chpattyp{\hinrp{p}}{\tsum{\tau_1}{\tau_2}}{\cinr{\xi}}{\Gamma}{\Delta}
}
\end{equation}
\begin{equation}
\inferrule[PTPair]{
  \chpattyp{p_1}{\tau_1}{\xi_1}{\Gamma_1}{\Delta_1} \\
  \chpattyp{p_2}{\tau_2}{\xi_2}{\Gamma_2}{\Delta_2}
}{
  \chpattyp{\hpair{p_1}{p_2}}{\tprod{\tau_1}{\tau_2}}
  {\cpair{\xi_1}{\xi_2}}{\Gamma_1 \uplus \Gamma_2}{\Delta_1 \uplus \Delta_2}
}
\end{equation}
\end{subequations}

  \judgboxa{
    \chrultyp{\Gamma}{\Delta}{\hrulP{p}{e}}{\tau}{\xi}{\tau'}
  }{$r$ transforms a final expression of type $\tau$ \\ to a final expression of type $\tau'$}
\begin{equation}\label{rule:CTRule}
\inferrule[CTRule]{
    \chpattyp{p}{\tau}{\xi}{\Gamma_p}{\Delta_p} \\
    \hexptyp{\Gamma \uplus \Gamma_p}{\Delta \uplus \Delta_p}{e}{\tau'}
}{
  \chrultyp{\Gamma}{\Delta}{\hrul{p}{e}}{\tau}{\xi}{\tau'}
}
\end{equation}

  \judgboxa{\chrulstyp{\Gamma}{\Delta}{\xi_{pre}}{rs}{\tau}{\xi_{rs}}{\tau'}}
  {$rs$ transforms a final expression of type $\tau$ \\ to a final expression of type $\tau'$}
\begin{subequations}\label{rules:CTRules}
\begin{equation}\label{rule:CTOneRules}
\inferrule[CTOneRules]{
  \chrultyp{\Gamma}{\Delta}{r}{\tau}{\xi_r}{\tau'} \\
  \cnotsatisfy{\xi_r}{\xi_{pre}}
}{
  \chrulstyp{\Gamma}{\Delta}{\xi_{pre}}{\hrulesP{r}{\cdot}}{\tau}{\xi_r}{\tau'}
}
\end{equation}
\begin{equation}\label{rule:CTRules}
\inferrule[CTRules]{
  \chrultyp{\Gamma}{\Delta}{r}{\tau}{\xi_r}{\tau'} \\
  \chrulstyp{\Gamma}{\Delta}{\cor{\xi_{pre}}{\xi_r}}{rs}
  {\tau}{\xi_{rs}}{\tau'} \\
  \cnotsatisfy{\xi_r}{\xi_{pre}}
}{
  \chrulstyp{\Gamma}{\Delta}{\xi_{pre}}{\hrules{r}{rs}}
  {\tau}{\cor{\xi_r}{\xi_{rs}}}{\tau'}
}
\end{equation}
\end{subequations}


\judgboxa{
  \hsubstyp{\theta}{\Gamma}
}{
  $\theta$ is of type $\Gamma$
}
\begin{subequations}
\begin{equation}
\inferrule[STEmpty]{ }{
  \hsubstyp{\emptyset}{\cdot}
}
\end{equation}
\begin{equation}
\inferrule[STExtend]{
  \hsubstyp{\theta}{\Gamma_\theta} \\
  \hexptyp{\Gamma}{\Delta}{e}{\tau}
}{
  \hsubstyp{\theta , x / e}{\Gamma_\theta , x : \tau}
}
\end{equation}
\end{subequations}

\judgboxa{
  \refutable{p}
}{$p$ is refutable}
\begin{subequations}
\begin{equation}
\inferrule[RNum]{ }{
  \refutable{\hnum{n}}
}
\end{equation}
\begin{equation}
\inferrule[REHole]{ }{
  \refutable{\hehole{w}}
}
\end{equation}
\begin{equation}
\inferrule[RHole]{ }{
  \refutable{\hhole{p}{w}}
}
\end{equation}
\begin{equation}
\inferrule[RInl]{ }{
  \refutable{\hinlp{p}}
}
\end{equation}
\begin{equation}
\inferrule[RInr]{ }{
  \refutable{\hinrp{p}}
}
\end{equation}
\begin{equation}
\inferrule[RPairL]{
  \refutable{p_1}
}{
  \refutable{\hpair{p_1}{p_2}}
}
\end{equation}
\begin{equation}
\inferrule[RPairR]{
  \refutable{p_2}
}{
  \refutable{\hpair{p_1}{p_2}}
}
\end{equation}
\end{subequations}

\judgboxa{
  \hpatmatch{e}{p}{\theta}
}{
  $e$ matches $p$, emitting $\theta$
}
\begin{subequations}
\begin{equation}
\inferrule[MVar]{ }{
  \hpatmatch{e}{x}{e / x}
}
\end{equation}
\begin{equation}
\inferrule[MWild]{ }{
  \hpatmatch{e}{\_}{\cdot}
}
\end{equation}
\begin{equation}
\inferrule[MNum]{ }{
  \hpatmatch{\hnum{n}}{\hnum{n}}{\cdot}
}
\end{equation}
\begin{equation}
\inferrule[MPair]{
  \hpatmatch{e_1}{p_1}{\theta_1} \\
  \hpatmatch{e_2}{p_2}{\theta_2}
}{
  \hpatmatch{\hpair{e_1}{e_2}}{\hpair{p_1}{p_2}}{\theta_1 \uplus \theta_2}
}
\end{equation}
\begin{equation}
\inferrule[MInl]{
  \hpatmatch{e}{p}{\theta}
}{
  \hpatmatch{\hinl{\tau}{e}}{\hinlp{p}}{\theta}
}
\end{equation}
\begin{equation}
\inferrule[MInr]{
  \hpatmatch{e}{p}{\theta}
}{
  \hpatmatch{\hinr{\tau}{e}}{\hinrp{p}}{\theta}
}
\end{equation}
\begin{equation}
\inferrule[MEHolePair]{
  \hpatmatch{\hprl{\hehole{u}}}{p_1}{\theta_1} \\
  \hpatmatch{\hprr{\hehole{u}}}{p_2}{\theta_2}
}{
  \hpatmatch{\hehole{u}}{\hpair{p_1}{p_2}}{\theta_1 \uplus \theta_2}
}
\end{equation}
\begin{equation}
\inferrule[MHolePair]{
  \hpatmatch{\hprl{\hhole{e}{u}}}{p_1}{\theta_1} \\
  \hpatmatch{\hprr{\hhole{e}{u}}}{p_2}{\theta_2}
}{
  \hpatmatch{\hhole{e}{u}}{\hpair{p_1}{p_2}}{\theta_1 \uplus \theta_2}
}
\end{equation}
\begin{equation}
\inferrule[MApPair]{
  \hpatmatch{\hprl{\hap{e_1}{e_2}}}{p_1}{\theta_1} \\
  \hpatmatch{\hprr{\hap{e_1}{e_2}}}{p_2}{\theta_2}
}{
  \hpatmatch{\hap{e_1}{e_2}}{\hpair{p_1}{p_2}}{\theta_1 \uplus \theta_2}
}
\end{equation}
\begin{equation}
\inferrule[MMatchPair]{
  \hpatmatch{\hprl{\hmatch{e}{\zrules}}}{p_1}{\theta_1} \\
  \hpatmatch{\hprr{\hmatch{e}{\zrules}}}{p_2}{\theta_2}
}{
  \hpatmatch{\hmatch{e}{\zrules}}{\hpair{p_1}{p_2}}{\theta_1 \uplus \theta_2}
}
\end{equation}
\begin{equation}
\inferrule[MPrlPair]{
  \hpatmatch{\hprl{\hprl{e}}}{p_1}{\theta_1} \\
  \hpatmatch{\hprr{\hprl{e}}}{p_2}{\theta_2}
}{
  \hpatmatch{\hprl{e}}{\hpair{p_1}{p_2}}{\theta_1 \uplus \theta_2}
}
\end{equation}
\begin{equation}
\inferrule[MPrrPair]{
  \hpatmatch{\hprl{\hprr{e}}}{p_1}{\theta_1} \\
  \hpatmatch{\hprr{\hprr{e}}}{p_2}{\theta_2}
}{
  \hpatmatch{\hprr{e}}{\hpair{p_1}{p_2}}{\theta_1 \uplus \theta_2}
}
\end{equation}
\end{subequations}

\judgbox{
  \hmaymatch{e}{p}
}{
  $e$ may match $p$
}
\begin{subequations}
\begin{equation}
\inferrule[MMEHole]{ }{
  \hmaymatch{e}{\hehole{w}}
}
\end{equation}
\begin{equation}
\inferrule[MMHole]{ }{
  \hmaymatch{e}{\hhole{p}{w}}
}
\end{equation}
\begin{equation}
\inferrule[MMExpEHole]{
  \refutable{p}
}{
  \hmaymatch{\hehole{u}}{p}
}
\end{equation}
\begin{equation}
\inferrule[MMExpHole]{
  \refutable{p}
}{
  \hmaymatch{\hhole{e}{u}}{p}
}
\end{equation}
\begin{equation}
\inferrule[MMAp]{
  \refutable{p}
}{
  \hmaymatch{\hap{e_1}{e_2}}{p}
}
\end{equation}
\begin{equation}
\inferrule[MMMatch]{
  \refutable{p}
}{
  \hmaymatch{\hmatch{e}{\zrules}}{p}
}
\end{equation}
\begin{equation}
\inferrule[MMPrl]{
  \refutable{p}
}{
  \hmaymatch{\hprl{e}}{p}
}
\end{equation}
\begin{equation}
\inferrule[MMPrr]{
  \refutable{p}
}{
  \hmaymatch{\hprr{e}}{p}
}
\end{equation}
\begin{equation}
\inferrule[MMPair1]{
  \hmaymatch{e_1}{p_1} \\
  \hpatmatch{e_2}{p_2}{\theta_2}
}{
  \hmaymatch{\hpair{e_1}{e_2}}{\hpair{p_1}{p_2}}
}
\end{equation}
\begin{equation}
\inferrule[MMPair2]{
  \hpatmatch{e_1}{p_1}{\theta_1} \\
  \hmaymatch{e_2}{p_2}
}{
  \hmaymatch{\hpair{e_1}{e_2}}{\hpair{p_1}{p_2}}
}
\end{equation}
\begin{equation}
\inferrule[MMPair3]{
  \hmaymatch{e_1}{p_1} \\
  \hmaymatch{e_2}{p_2}
}{
  \hmaymatch{\hpair{e_1}{e_2}}{\hpair{p_1}{p_2}}
}
\end{equation}
\begin{equation}
\inferrule[MMInL]{
  \hmaymatch{e}{p}
}{
  \hmaymatch{\hinl{\tau}{e}}{\hinlp{p}}
}
\end{equation}
\begin{equation}
\inferrule[MMInR]{
  \hmaymatch{e}{p}
}{
  \hmaymatch{\hinr{\tau}{e}}{\hinrp{p}}
}
\end{equation}
\end{subequations}

\fbox{$\hnotmatch{e}{p}$}~~\text{$e$ doesn't match $p$}
\begin{subequations}
\begin{equation}
\inferrule[NMNum]{
  n_1 \neq n_2
}{
  \hnotmatch{\hnum{n_1}}{\hnum{n_2}}
}
\end{equation}
\begin{equation}
\inferrule[NMPair1]{
  \hnotmatch{e_1}{p_1}
}{
  \hnotmatch{\hpair{e_1}{e_2}}{\hpair{p_1}{p_2}}
}
\end{equation}
\begin{equation}
\inferrule[NMPair2]{
  \hnotmatch{e_2}{p_2}
}{
  \hnotmatch{\hpair{e_1}{e_2}}{\hpair{p_1}{p_2}}
}
\end{equation}
\begin{equation}
\inferrule[NMConfL]{ }{
  \hnotmatch{\hinr{\tau}{e}}{\hinlp{p}}
}
\end{equation}
\begin{equation}
\inferrule[NMConfR]{ }{
  \hnotmatch{\hinl{\tau}{e}}{\hinrp{p}}
}
\end{equation}
\begin{equation}
\inferrule[NMInjL]{
  \hnotmatch{e}{p}
}{
  \hnotmatch{\hinr{\tau}{e}}{\hinlp{p}}
}
\end{equation}
\begin{equation}
\inferrule[NMInjR]{
  \hnotmatch{e}{p}
}{
  \hnotmatch{\hinl{\tau}{e}}{\hinrp{p}}
}
\end{equation}
\end{subequations}

\fbox{$\isVal{e}$}~~\text{$e$ is a value}
\begin{subequations}\label{rules:Value}
\begin{equation}
\inferrule[VNum]{ }{
  \isVal{\hnum{n}}
}
\end{equation}
\begin{equation}
\inferrule[VLam]{ }{
  \isVal{\hlam{x}{\tau}{e}}
}
\end{equation}
\begin{equation}
\inferrule[VPair]{
  \isVal{e_1} \\
  \isVal{e_2}
}{\isVal{\hpair{e_1}{e_2}}}
\end{equation}
\begin{equation}
\inferrule[VInL]{
  \isVal{e}
}{
  \isVal{\hinl{\tau}{e}}
}
\end{equation}
\begin{equation}
\inferrule[VinR]{
  \isVal{e}
}{
  \isVal{\hinr{\tau}{e}}
}
\end{equation}
\end{subequations}

\fbox{$\isIndet{e}$}~~\text{$e$ is indeterminate}
\begin{subequations}\label{rules:Indet}
\begin{equation}\label{rule:IEHole}
\inferrule[IEHole]{ }{
  \isIndet{\hehole{u}}
}
\end{equation}
\begin{equation}\label{rule:IHole}
\inferrule[IHole]{
  \isFinal{e}
}{
  \isIndet{\hhole{e}{u}}
}
\end{equation}
\begin{equation}\label{rule:IAp1}
\inferrule[IAp1]{
  \isIndet{e_1}
}{
  \isIndet{\hap{e_1}{e_2}}
}
\end{equation}
\begin{equation}\label{rule:IAp2}
\inferrule[IAp2]{
  \isVal{e_1} \\ \isIndet{e_2}
}{
  \isIndet{\hap{e_1}{e_2}}
}
\end{equation}
\begin{equation}\label{rule:IPair1}
\inferrule[IPair1]{
  \isIndet{e_1}
}{
  \isIndet{\hpair{e_1}{e_2}}
}
\end{equation}
\begin{equation}\label{rule:IPair2}
\inferrule[IPair2]{
  \isVal{e_1} \\
  \isIndet{e_2}
}{
  \isIndet{\hpair{e_1}{e_2}}
}
\end{equation}
\begin{equation}\label{rule:IInl}
\inferrule[IInL]{
  \isIndet{e}
}{
  \isIndet{\hinl{\tau}{e}}
}
\end{equation}
\begin{equation}\label{rule:IInR}
\inferrule[IInR]{
  \isIndet{e}
}{
  \isIndet{\hinr{\tau}{e}}
}
\end{equation}
\begin{equation}\label{rule:IMatch}
\inferrule[IMatch]{
  \isFinal{e} \\
  \hmaymatch{e}{p_r}
}{
  \isIndet{
    \hmatch{e}{\zruls{rs_{pre}}{\hrulP{p_r}{e_r}}{rs_{post}}}
  }
}
\end{equation}
\end{subequations}

\fbox{$\isFinal{e}$}~~\text{$e$ is final}
\begin{subequations}\label{rules:Final}
  \begin{equation}\label{rule:FVal}
\inferrule[FVal]{
  \isVal{e}
}{
  \isFinal{e}
}
\end{equation}
\begin{equation}\label{rule:FIndet}
\inferrule[FIndet]{
  \isIndet{e}
}{
  \isFinal{e}
}
\end{equation}
\end{subequations}

\fbox{$e \mapsto e'$}~~\text{$e$ takes an instruction transition to $e'$}
\begin{subequations}\label{rules:ITExp}
\begin{equation}
\inferrule[ITHole]{
  \htrans{e}{e'}
}{
  \htrans{\hhole{e}{u}}{\hhole{e'}{u}}
}
\end{equation}
\begin{equation}
\inferrule[ITAp1]{
  \htrans{e_1}{e_1'}
}{
  \htrans{\hap{e_1}{e_2}}{\hap{e_1'}{e_2}}
}
\end{equation}
\begin{equation}
\inferrule[ITAp2]{
  \isVal{e_1} \\
  \htrans{e_2}{e_2'}
}{
  \htrans{\hap{e_1}{e_2}}{\hap{e_1}{e_2'}}
}
\end{equation}
\begin{equation}
\inferrule[ITAP]{
  \isVal{e_2}
}{
  \hap{\hlam{x}{\tau}{e_1}}{e_2} \mapsto
    [e_2/x]e_1
}
\end{equation}
\begin{equation}
\inferrule[ITPair1]{
  \htrans{e_1}{e_1'}
}{
  \htrans{\hpair{e_1}{e_2}}{\hpair{e_1'}{e_2}}
}
\end{equation}
\begin{equation}
\inferrule[ITPair2]{
  \isVal{e_1} \\
  \htrans{e_2}{e_2'}
}{
  \htrans{\hpair{e_1}{e_2}}{\hpair{e_1}{e_2'}}
}
\end{equation}
\begin{equation}
\inferrule[ITInL]{
  \htrans{e}{e'}
}{
  \htrans{\hinl{\tau}{e}}{\hinl{\tau}{e'}}
}
\end{equation}
\begin{equation}
\inferrule[ITInR]{
  \htrans{e}{e'}
}{
  \htrans{\hinr{\tau}{e}}{\hinr{\tau}{e'}}
}
\end{equation}
\begin{equation}\label{rule:ITExpMatch}
\inferrule[ITExpMatch]{
  \htrans{e}{e'}
}{
  \htrans{\hmatch{e}{\zrules}}{\hmatch{e'}{\zrules}}
}
\end{equation}
\begin{equation}\label{rule:ITSuccMatch}
\inferrule[ITSuccMatch]{
  \isFinal{e} \\
  \hpatmatch{e}{p_r}{\theta}
}{
  \htrans{
    \hmatch{e}{\zruls{rs_{pre}}{\hrulP{p_r}{e_r}}{rs_{post}}}
  }{
    [\theta](e_r)
  }
}
\end{equation}
\begin{equation}\label{rule:ITFailMatch}
\inferrule[ITFailMatch]{
  \isFinal{e} \\
  \hnotmatch{e}{p_r}
}{
  \htrans{
    \hmatch{e}{\zruls{rs}{\hrulP{p_r}{e_r}}{\hrulesP{r'}{rs'}}}
  }{
    \hmatch{e}{
      \zruls{
        \rmpointer{\zruls{rs}{\hrulP{p_r}{e_r}}{\cdot}}
      }{r'}{rs'}
    }
  }
}
\end{equation}
\end{subequations}

\begin{lemma}[Substitution]
  \label{lem:substitution}
  If $\hexptyp{\Gamma, x : \tau}{\Delta}{e_0}{\tau_0}$ and $\hexptyp{\Gamma}{\Delta}{e}{\tau}$
  then $\hexptyp{\Gamma}{\Delta}{[e/x]e_0}{\tau_0}$
\end{lemma}

\begin{lemma}[Simultaneous Substitution]
  \label{lem:simult-substitution}
  If $\hexptyp{\Gamma \uplus \Gamma'}{\Delta}{e}{\tau}$ and $\hsubstyp{\theta}{\Gamma'}$
  then $\hexptyp{\Gamma}{\Delta}{[\theta]e}{\tau}$
\end{lemma}
Proof by induction on the derivation of $\hexptyp{\Gamma \uplus \Gamma'}{\Delta}{e}{\tau}$.

\begin{lemma}[Substitution Typing]
  \label{lem:subs-typing}
  If $\hpatmatch{e}{p}{\theta}$ and $\hexptyp{\cdot}{\Delta}{e}{\tau}$ and $\hpattyp{p}{\tau}{\Gamma}{\Delta'}$
  then $\hsubstyp{\theta}{\Gamma}$
\end{lemma}
Proof by induction on the derivation of $\hpatmatch{e}{p}{\theta}$.

To apply this lemma in ITSuccMatch case, first apply inversion lemma on premise of preservation theorem,

\begin{lemma}[Matching Progress]
  \label{lem:match-progress}
  If $\isFinal{e}$ and $\hexptyp{\cdot}{\Delta}{e}{\tau}$ and $\hpattyp{p}{\tau}{\Gamma}{\Delta'}$
  then $\hnotmatch{e}{p}$ or $\hmaymatch{e}{p}$ or $\hpatmatch{e}{p}{\theta}$ for some $\theta$
\end{lemma}
Proof by induction on two premises.

\begin{theorem}[Stepping Determinism]
  \label{thrm:step-determinism}
  If $\htrans{e}{e'}$ and $\htrans{e}{e''}$ then $e' = e''$
\end{theorem}

\begin{theorem}[Determinism]
  \label{thrm:determinism}
  If $\hexptyp{\cdot}{\Delta}{e}{\tau}$ then exactly one of the following holds
  \begin{enumerate}
    \item $\isVal{e}$
    \item $\isIndet{e}$
    \item $\htrans{e}{e'}$ for some unique $e'$
  \end{enumerate}
\end{theorem}

\begin{lemma}[Matching Determinism]
  \label{lem:match-determinism}
  If $\isFinal{e}$ and $\hexptyp{\cdot}{\Delta}{e}{\tau}$ and $\hpattyp{p}{\tau}{\Gamma}{\Delta}$ then exactly one of the following holds
  \begin{enumerate}
    \item $\hpatmatch{e}{p}{\theta}$ for some $\theta$
    \item $\hmaymatch{e}{p}$
    \item $\hnotmatch{e}{p}$
  \end{enumerate}
\end{lemma}
\subsection{Decidability}
\label{sec:decidability}
In this subsection, we show that the validity of a "fully known" constraint (\autoref{definition:valid-constraint}) is decidable.

% !TEX root= pattern-paper.tex

\begin{figure}[ht]
\judgbox{\cincon{\Xi}}{}

\begin{mathpar}
\Infer{\CINCTruth}{
  \cincon{\Xi}
}{
  \cincon{\Xi, \ctruth}
}

\Infer{\CINCFalsity}{ }{
  \cincon{\Xi, \cfalsity}
}

\Infer{\CINCNum}{
  n_1 \neq n_2
}{
  \cincon{\Xi, \cnum{n_1}, \cnum{n_2}}
}

\Infer{\CINCNotNum}{ }{
  \cincon{\Xi, \cnum{n}, \cnotnum{n}}
}

\Infer{\CINCAnd}{
  \cincon{\Xi, \xi_1, \xi_2}
}{
  \cincon{\Xi, \cand{\xi_1}{\xi_2}}
}

\Infer{\CINCOr}{
  \cincon{\Xi, \xi_1} \\
  \cincon{\Xi, \xi_2}
}{
  \cincon{\Xi, \cor{\xi_1}{\xi_2}}
}

\Infer{\CINCInj}{ }{
  \cincon{\Xi, \cinl{\xi_1}, \cinr{\xi_2}}
}

\Infer{\CINCInl}{
  \cincon{\setof{\xi' | \cinl{\xi'} \in \Xi},\xi}
}{
  \cincon{\Xi, \cinl{\xi}}
}

\Infer{\CINCInr}{
  \cincon{\setof{\xi' | \cinr{\xi'} \in \Xi},\xi}
}{
  \cincon{\Xi, \cinr{\xi}}
}

\Infer{\CINCPairL}{
    \cincon{\setof{\xi_1' | \cpair{\xi_1'}{\xi_2'} \in \Xi},\xi_1}
}{
    \cincon{\Xi, \cpair{\xi_1}{\xi_2}}
}

\Infer{\CINCPairR}{
    \cincon{\setof{\xi_2' | \cpair{\xi_1'}{\xi_2'} \in \Xi},\xi_2}
}{
    \cincon{\Xi, \cpair{\xi_1}{\xi_2}}
}
\end{mathpar}

  \caption{Inconsistency of Constraints}
  \label{fig:incon}
\end{figure}

\subsubsection{SAT Encoding}
 One approach is to reduce the validity checking to a boolean satisfiability problem (SAT). 
If we revisit the analogy between constraint and set of expressions discussed in \autoref{sec:constraint}, we can think of constraints as subsets of values of type $\tau$.
Then $\ccsatisfy{}{\xi}$ basically says that $\xi$ exactly represents the set of all values of type $\tau$. 
However, such set may be infinite (e.g. top constraint $\ctruth$),
and thus defining operations on such infinite sets is nontrivial. 

Nevertheless, we may use logical predicates to encode the subset of values corresponding to a constraint. 
For example, $\xi=\cnum{2}$ represents a set with one value $\hnum{2}$, and thus can be encoded as a predicate $x=2$. 
If there is any connectives ($\cand{}{}$ and $\cor{}{}$) in $\xi$, we can use the connectives of the same form in logical formula. 
It is tricky to encode $inl$, $inr$, and $pair$ constraint as predicate. 
If we think of a constraint as a set again, a value $e=\cpair{e_1}{e_2}$ belongs to $\xi=\cpair{\xi_1}{\xi_2}$ iff $e_1$ belongs to $\xi_1$ and $e_2$ belongs to $\xi_2$. 
Therefore, the logical encoding of $\cpair{\xi_1}{\xi_2}$ would be a conjunction of encoding of both side of the pair, with a variable for each.
As for injections, we can use a boolean value $b$ to denote whether a constraint is $inl$ or $inr$, and conjoin it with the encoding of $\xi_1$. In the following example, we use $b=\mathtt{true}$ for $inl$ and $b=\mathtt{false}$ for $inr$.

One last thing to notice here is that we need to make sure when transforming constraints on the same set of values into a predicate, the same variable would be used. To demonstrate how that might work, let's consider a more involved example:
\[ \cpair{\cinl{\cnum{1}}}{\cinl{\cnum{3}}} \vee \cpair{\cinl{\cnum{2}}}{\cinr{\cnum{1}}} \]
Both $\cpair{\cinl{\cnum{1}}}{\cinl{\cnum{3}}}$ and $\cpair{\cinl{\cnum{2}}}{\cinr{\cnum{1}}}$ place constraints on the same variable $x$.
Therefore, their left/right side also place constraints on the same variable, though unnecessary to be introduced explicitly.
If we let $b_{x_l}$($b_{x_r}$) correspond to the left(right) side of the pairs, and $x_l'$($x_r'$) encode the number constraints under injections, 
then 
we encode $\cinl{\cnum{1}}$ as $b_{x_l} \wedge (x_l'=1)$, 
$\cinl{\cnum{3}}$ as $b_{x_r} \wedge (x_r'=3)$,
$\cinl{\cnum{2}}$ as $b_{x_l} \wedge (x_l'=2)$,
$\cinr{\cnum{1}}$ as $\neg b_{x_r} \wedge (x_r'=1)$. 
Put them together and we get the logical encoding of the entire constraint, 
\[
(b_{x_l} \wedge (x_l'=1) \wedge
b_{x_r} \wedge (x_r'=3))
\vee
(b_{x_l} \wedge (x_l'=2) \wedge
\neg b_{x_r} \wedge (x_r'=1))
\]

As a result, the validity of a constraint $\xi$, $\ccsatisfy{}{\xi}$, is equivalent to the validity of its logical encoding. Exhaustiveness and redundancy checking are reduced to boolean satisfiability problem, which is NP-complete but decidable, and several tools exist for doing so. For handling numeric patterns we only need SAT modulo numeric equality and disequality.

\subsubsection{Constraint Inconsistency Checking}\label{sec:incon}

Using an SMT solver to decide constraint entailments is, however, an overkill. Moreover, it may incur run-time and space overhead in a  development environment. When incorporating Peanut into Hazel, we use a different but more lightweight decision procedure. \figurename~\ref{fig:incon} describes such a procedure by defining a new judgment $\cincon{\Xi}$, where $\Xi$ represents a list of constraint $\xi$. Assuming constraint $\xi$ is fully known and is of type $\tau$, $\cincon{\xi}$ means constraint $\xi$ is inconsistent in the sense that no values of type $\tau$ satisfy $\xi$, which corresponds to the insatisfiability of $\xi$'s logical encoding. Therefore, a constraint is valid, written as $\ccsatisfy{}{\xi}$, iff its dual is inconsistent, written as $\cincon{\cdual{\xi}}$. Note that this is not fully mechanized in Agda. Such proofs require reasoning about finite sets in a non-structurally recursive way, making them inordinately difficult to verify in Agda, but our implementation uses this algorithm.


%\begin{theorem}
%\textbf{}  Assume $\ctruify{\xi}=\xi$. It is decidable whether $\cincon{\xi}$.
%\end{theorem}
%
%\begin{theorem}
%  Assume $\ctruify{\xi}=\xi$. Then $\cincon{\cdual{\xi}}$ iff $\csatisfy{\ctruth}{\xi}$.
%\end{theorem}
%%% Local Variables:
%%% mode: latex
%%% TeX-master: "pattern-paper"
%%% End:


\end{document}
%%% Local Variables:
%%% mode: latex
%%% TeX-master: t
%%% End:
