% !TEX root= pattern-paper.tex

\begin{figure}[t]
    \centering
    \begin{minipage}{.44\linewidth}
  $\arraycolsep=4pt\begin{array}{lcll}
    \mathsf{Exp} & e & ::= &
      x ~\vert~
      \hnum{n} \\&
      & ~\vert~ &
      \hlam{x}{\tau}{e} ~\vert~
      \hap{e_1}{e_2} \\&
      & ~\vert~ &
      \hpair{e_1}{e_2} ~\vert~
      \hfst{e} ~\vert~ \hsnd{e} \\&
      & ~\vert~ &
      \hinl{\tau}{e} ~\vert~
      \hinr{\tau}{e} \\&
      & ~\vert~ &
      \hmatch{e}{\hat{rs}} \\&
      & ~\vert~ &
      \hehole{u} ~\vert~
      \hhole{e}{u}
    \end{array}$
    \end{minipage}%
    \begin{minipage}{.55\linewidth}
  $\arraycolsep=4pt\begin{array}{llcl}
    \mathsf{Typ} & \tau & ::= &
      \tnum ~\vert~
      \tarr{\tau_1}{\tau_2} ~\vert~
      \tprod{\tau_1}{\tau_2} ~\vert~
      \tsum{\tau_1}{\tau_2} \\
    \mathsf{ZRules} & \hat{rs} & ::= &
      \zrulsP{rs}{r}{rs} \\
    \mathsf{Rules} & rs & ::= &
      \cdot ~\vert~ \hrulesP{r}{rs'} \\
    \mathsf{Rule} & r & ::= &
      \hrul{p}{e} \\
    \mathsf{Pat} & p & ::= &
      x ~\vert~
      \_ ~\vert~
      \hnum{n} ~\vert~
      \hpair{p_1}{p_2} \\&
      & ~\vert~ &
      \hinlp{p} ~\vert~
      \hinrp{p} ~\vert~
      \heholep{w} ~\vert~
      \hholep{p}{w}{\tau}
    \end{array}$
    \end{minipage}
\caption{Syntax of Peanut. Here, $n$ ranges over mathematical numbers (of any suitable sort), $x$ over variables, $u$ over expression hole names, and $w$ over pattern hole names.}
\label{fig:syntax}
\end{figure}