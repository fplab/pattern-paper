\section{Related Work}

\subsubsection{Typed Holes}
Peanut is based on Hazelnut Live, a typed lambda calculus that includes only expression and type holes.
The Hazel programming environment, which our implementation extends, is based directly on Hazelnut Live. 
Peanut is derived from Hazelnut Live, retaining expression holes and introducing  
structural pattern matching and pattern holes. (Some small technical differences are described in \todo{refer part of Sec 3 where we say more about technical distinctions}.) 
Like Hazelnut Live, evaluation is able to proceed around holes, including pattern holes, in Peanut. An evaluation step is taken 
only if it would be justified for all possible hole fillings.

Peanut omits type holes, i.e. the unknown types from gradual type theory\todo{cite Siek and Taha + Siek et al SNAPL 2015}{}. Type holes obscure type information, so 
is possible to reason statically about exhaustiveness and redundancy in only a limited manner when the scrutinee is of unknown type.\footnote{Constraint-based type inference could be deployed to discover a type for the scrutinee in some cases, at which point it would be possible to use our mechanisms as described. Whether inference is deployed for this purpose is an orthogonal consideration.} Allowing scrutinees of unknown type would also require performing run-time checks, e.g. in the form of casts inserted on terms matched by variables of unknown type. 
We have implemented this cast insertion machinery, which follows straightforwardly from prior work on cast insertion for binary sum types\todo{SNAPL15 paper has it}{}, into Hazel. This machinery is orthogonal to the machinery
we consider in this paper, so we did not include type holes and casts in Peanut.

Hazelnut Live augments each expression hole instance with a closure, which serves to record deferred substitutions 
(it is therefore a contextual type theory\todo{cite cmtt}). This information can be presented to the user and it enables soundly resuming from the current evaluation state when the programmer fills a hole (as long as there are no non-commutative side effects in the language).
The addition of pattern holes does not interfere with this mechanism. Pattern holes do not themselves need closures because patterns bind,
rather than consume, variables. (In a language where patterns contain expressions, e.g. when guards are integrated into patterns\todo{cite ML workshop paper by Reppy}, closures on pattern holes would be necessary to support resumption.) It would not be possible to resume evaluation after filling a pattern hole because doing so can, in general, change the binding structure of the program by introducing shadowing. We leave to future work consideration of 
conditional resumption when pattern hole filling happens not to cause shadowing.

In Hazel, holes are inserted automatically during editing. Formally, Hazel is a type-aware structure editor governed by an edit action semantics derived from Hazelnut, a type-aware structure editor calculus\todo{cite}{} 
for the same language as Hazelnut Live. 
Hazel, by combining machinery from Hazelnut and Hazelnut Live, maintains a powerful continuity invariant: every edit state has a well-defined type
and a well-defined result, both possibly containing holes.
Our extension of Hazel maintains the same invariant, now with the addition of match expressions 
as described in this paper. 
Extending the edit action semantics to allow us to enter patterns presents no special challenges relative to prior work, so we omit formal consideration of editing from this paper.

Moreover, our contributions do not require a structure editor: they are also relevant to languages where typed holes are inserted 
manually by programmers, rather than automatically by an editor. 
For example, GHC Haskell, Agda, and Idris, all feature manually inserted typed holes, both empty and non-empty, in expression position. 
None of these systems support pattern holes as of this writing, however. Haskell does support unbound data constructors in patterns, but these bring compilation to a halt and do not interact with the exhaustiveness and redundancy checker, much less the run-time system. With the appropriate flags set, Haskell will attempt to evaluate programs with expression holes, but it stops with an exception whenever a hole is reached\todo{cite}{}. In contrast, our system supports evaluating around holes of all sorts, as described in \autoref{sec:examples} and formalized in \todo{refer to dynamics section}{}. We hope that this paper will prompt other languages to 
consider adding pattern holes. Our work on the Peanut calculus provides a minimal formal characterization 
that captures the essential character of pattern matching with typed holes and our implementation in Hazel 
demonstrates a practical realization, complete with editor integration.

\subsubsection{Pattern Matching}
Pattern matching has a long history, first appearing in the 1970s in early functional languages
such as NPL, SASL, and ML.
All prior work on pattern matching assumes that the patterns and expressions being matched are
complete, i.e., do not contain any holes.
Uniquely, the indeterminate matching and exhaustiveness and redundancy checks in Peanut
take into account how the programmer may fill or correct these holes in a future edit state.

Much of the early work on pattern matching focused on efficient compilation. Some methods construct decision trees encoding the matching procedure, the goal being to minimize the sizes of the constructed trees, meanwhile obtaining exhaustiveness and redundancy checks as easy by-products \cite{Aitken92smlnj,Baudinet85treepattern,Sestoft96mlpattern}.
Other methods compile pattern matching to backtracking automata \cite{DBLP:journals/jfp/Maranget07}(cite Maranget 1994);
while these methods avoid potential exponential behavior exhibited by decision trees, they are less directly amenable to pattern match analyses and require separate methods for reporting inexhaustive and redundant patterns (cite Maranget 2007).
In our setting of incomplete programs, the question of compilation is not (yet) relevant. Our goal in this paper is to provide a complete semantics of pattern matching with holes that extends the incomplete program evaluation of Hazelnut Live; in so doing, we integrate exhaustiveness and redundancy checks into the type system.
We leave questions of optimization to future work.

Contemporary work focuses on extending pattern match analyses to settings with more sophisticated types and pattern matching features (cite Vaozou et al 2017, Cockx and Abel 2018, Sozeau 2010, GMTM, LYG).
In a similar spirit, our work extends pattern matching to the previously unexplored setting of gradually typed patterns and expressions with holes.
Whereas prior work of this sort concerns itself with analyzing increasingly sophisticated predicates delineating a match from a failure (or divergence in the lazy setting), our work introduces a new pattern matching \emph{outcome}---the indeterminate match---and specifies its static and dynamic semantic underpinnings.
To that end, we narrow our focus to the novel aspects of our system and leave integration with existing, richer pattern matching features and checkers to future work.

% The current state of the art is Lower Your Guards by Graf et al., which can handle GADTs and the wide variety of pattern matching features in Haskell, e.g., guards, view patterns, pattern synonyms, etc.
% This work abandons simple structural pattern matching altogether, instead compiling the various pattern forms into a simpler intermediate representation called a guard tree.



% It is straightforward to check for exhaustiveness and redundancy when pattern-matching on simple algebraic datatypes.
% Much of the literature on pattern matching focuses on extending these checks in the presence of more sophisticated types and pattern matching features.
% \begin{itemize}
%     \item Lower Your Guards
%     \begin{itemize}
%         \item various pattern matching features are compiled to simple guard
%         trees, which are then proceeds into annotated trees that decorate
%         branches with refinement types
%         \item pattern match analyses are performed by generating inhabitants
%         of annotated refinement types
%         \item we do not consider sophisticated pattern matching features
%         and need not resort to such machinery, simple structural matching
%         is sufficient
%         \item matching a guard tree may succeed, fail, or diverge; no indeterminate case
%         \item likely their work could be extended to integrate indeterminate matching
%     \end{itemize}
%     \item other checkers for handling potentially undecidable coverage
%     \begin{itemize}
%         \item GADTs Meet Their Match, Karachalias et al 2015
%         \begin{itemize}
%             \item note: defines redundant clause as one where pattern has no well-typed
%             value that matches (in the context of GADTs)
%         \end{itemize}
%         \item SMT solver for handling guards, Kalvoda and Kerckhove 2019
%         \item case trees for dependently typed / refinement type languages
%         \item emphasize that the failure/uncertainty of checking with these
%         other tools is not the same as indeterminate matching in our work
%     \end{itemize}
% \end{itemize}
