\RequirePackage{etex}
\documentclass[runningheads,envcountsame,a4paper]{llncs}

\usepackage[T1]{fontenc} % fix missing font cmtt
\usepackage{amsmath}
\usepackage{amssymb} % Vdash
\usepackage{graphicx} % rotatebox
\usepackage{stmaryrd} % llparenthesis
\usepackage{anyfontsize} % workaround for font size difference warning
\usepackage{todonotes}
\usepackage{listings}
\usepackage{tikz}
\usetikzlibrary{calc,fit,tikzmark,plotmarks,arrows.meta,positioning,overlay-beamer-styles}
\usepackage[caption=false]{subfig}

\usepackage{cancel} % slash over symbol
\usepackage{hyperref}
\renewcommand\UrlFont{\color{blue}\rmfamily}

\usepackage{xcolor}
\definecolor{hazelgreen}{RGB}{7,63,36}
\definecolor{hazellightgreen}{RGB}{103,138,97}
\definecolor{hazelyellow}{RGB}{245,222,179}
\definecolor{hazellightyellow}{RGB}{254,254,234}

\newcommand{\highlight}[1]{\colorbox{yellow}{$\displaystyle #1$}}

\newcommand{\todo}[1]{{\color{red} TODO: #1}}

%% Joshua Dunfield macros
\def\OPTIONConf{1}
\usepackage{jdunfield}
\usepackage{rulelinks} % hyperlink of rule name
\usepackage{pfsteps}
\makeatletter
\newcommand{\savelocalsteps}[1]{
  \@ifundefined{c@#1}
    {% the counter doesn't exist
     \newcounter{#1}
   }{}
  \setcounter{#1}{\value{pfsteps@pfc@local}}
}
\makeatother
\newcommand{\restorelocalsteps}[1]{\setcounter{pfsteps@pfc@local}{\value{#1}}}

% !TEX root = ./patterns-paper.tex

% reverse Vdash
\newcommand{\dashV}{\mathbin{\rotatebox[origin=c]{180}{$\Vdash$}}}

% Violet hotdogs; highlight color helps distinguish them
\newcommand{\llparenthesiscolor}{\textcolor{violet}{\llparenthesis}}
\newcommand{\rrparenthesiscolor}{\textcolor{violet}{\rrparenthesis}}

% HTyp and HExp
\newcommand{\hcomplete}[1]{#1~\mathsf{complete}}

% HTyp
\newcommand{\htau}{\dot{\tau}}
\newcommand{\tarr}[2]{\inparens{#1 \rightarrow #2}}
\newcommand{\tarrnp}[2]{#1 \rightarrow #2}
\newcommand{\trul}[2]{\inparens{#1 \Rightarrow #2}}
\newcommand{\tnum}{\mathtt{num}}
\newcommand{\tehole}{\llparenthesiscolor\rrparenthesiscolor}
\newcommand{\tsum}[2]{\inparens{{#1} + {#2}}}
\newcommand{\tprod}[2]{\inparens{{#1} \times {#2}}}
\newcommand{\tunit}{\mathtt{1}}
\newcommand{\tvoid}{\mathtt{0}}

\newcommand{\tcompat}[2]{#1 \sim #2}
\newcommand{\tincompat}[2]{#1 \nsim #2}

% HExp
\newcommand{\hexp}{\dot{e}}
\newcommand{\hlam}[3]{\inparens{\lambda #1:#2.#3}}
\newcommand{\hap}[2]{#1(#2)}
\newcommand{\hapP}[2]{(#1)~(#2)} % Extra paren around function term
\newcommand{\hnum}[1]{\underline{#1}}
\newcommand{\hadd}[2]{\inparens{#1 + #2}}
\newcommand{\hpair}[2]{\inparens{#1 , #2}}
\newcommand{\htriv}{()}
\newcommand{\hehole}{\llparenthesiscolor\rrparenthesiscolor}
\newcommand{\hhole}[1]{\llparenthesiscolor#1\rrparenthesiscolor}
\newcommand{\hindet}[1]{\lceil#1\rceil}
\newcommand{\hinj}[2]{\mathtt{inj}_{#1}({#2})}
\newcommand{\hinl}[2]{\mathtt{inl}_{#1}({#2})}
\newcommand{\hinr}[2]{\mathtt{inr}_{#1}({#2})}
\newcommand{\hinlp}[1]{\mathtt{inl}(#1)}
\newcommand{\hinrp}[1]{\mathtt{inr}(#1)}
\newcommand{\hmatch}[2]{\mathtt{match}(#1) \{#2\}}
\newcommand{\hcase}[5]{\mathtt{case}({#1},{#2}.{#3},{#4}.{#5})}
\newcommand{\hrules}[2]{\inparens{#1 \mid #2}}
\newcommand{\hrul}[2]{#1 \Rightarrow #2}

\newcommand{\hGamma}{\dot{\Gamma}}
\newcommand{\domof}[1]{\text{dom}(#1)}
\newcommand{\hsyn}[3]{#1 \vdash #2 \Rightarrow #3}
\newcommand{\hana}[3]{#1 \vdash #2 \Leftarrow #3}
\newcommand{\hexptyp}[3]{#1 \vdash #2 : #3}
\newcommand{\hpattyp}[3]{#1 : #2 \dashV #3}
\newcommand{\hpatmatch}[3]{#1 \vartriangleright #2 \dashV #3}
\newcommand{\hval}[1]{#1 ~\mathtt{val}}
\newcommand{\herr}[1]{#1 ~\mathtt{err}}

% ZTyp and ZExp
\newcommand{\zlsel}[1]{{\bowtie}{#1}}
\newcommand{\zrsel}[1]{{#1}{\bowtie}}
\newcommand{\zwsel}[1]{
  \setlength{\fboxsep}{0pt}
  \colorbox{green!10!white!100}{
    \ensuremath{{{\textcolor{Green}{{\hspace{-2px}\triangleright}}}}{#1}{\textcolor{Green}{\triangleleft{\vphantom{\tehole}}}}}}
}

\newcommand{\removeSel}[1]{#1^{\diamond}}

% ZTyp
\newcommand{\ztau}{\hat{\tau}}

% ZExp
\newcommand{\zexp}{\hat{e}}

% Direction
\newcommand{\dParent}{\mathtt{parent}}
\newcommand{\dChildn}[1]{\mathtt{child}~\mathtt{{#1}}}
\newcommand{\dChildnm}[1]{\mathtt{child}~{#1}}

% Action
\newcommand{\aMove}[1]{\mathtt{move}~#1}
	\newcommand{\zrightmost}[1]{\mathsf{rightmost}(#1)}
	\newcommand{\zleftmost}[1]{\mathsf{leftmost}(#1)}
\newcommand{\aSelect}[1]{\mathtt{sel}~#1}
\newcommand{\aDel}{\mathtt{del}}
\newcommand{\aReplace}[1]{\mathtt{replace}~#1}
\newcommand{\aConstruct}[1]{\mathtt{construct}~#1}
\newcommand{\aConstructx}[1]{#1}
\newcommand{\aFinish}{\mathtt{finish}}

\newcommand{\performAna}[5]{#1 \vdash #2 \xlongrightarrow{#4} #5 \Leftarrow #3}
\newcommand{\performAnaI}[5]{#1 \vdash #2 \xlongrightarrow{#4}\hspace{-3px}{}^{*}~ #5 \Leftarrow #3}
\newcommand{\performSyn}[6]{#1 \vdash #2 \Rightarrow #3 \xlongrightarrow{#4} #5 \Rightarrow #6}
\newcommand{\performSynI}[6]{#1 \vdash #2 \Rightarrow #3 \xlongrightarrow{#4}\hspace{-3px}{}^{*}~ #5 \Rightarrow #6}
\newcommand{\performTyp}[3]{#1 \xlongrightarrow{#2} #3}
\newcommand{\performTypI}[3]{#1 \xlongrightarrow{#2}\hspace{-3px}{}^{*}~#3}

\newcommand{\performMove}[3]{#1 \xlongrightarrow{#2} #3}
\newcommand{\performDel}[2]{#1 \xlongrightarrow{\aDel} #2}

% Form
\newcommand{\farr}{\mathtt{arrow}}
\newcommand{\fnum}{\mathtt{num}}
\newcommand{\fsum}{\mathtt{sum}}

\newcommand{\fasc}{\mathtt{asc}}
\newcommand{\fvar}[1]{\mathtt{var}~#1}
\newcommand{\flam}[1]{\mathtt{lam}~#1}
\newcommand{\fap}{\mathtt{ap}}
% \newcommand{\farg}{\mathtt{arg}}
\newcommand{\fnumlit}[1]{\mathtt{lit}~#1}
\newcommand{\fplus}{\mathtt{plus}}
\newcommand{\fhole}{\mathtt{hole}}
\newcommand{\fnehole}{\mathtt{nehole}}

\newcommand{\finj}[1]{\mathtt{inj}~#1}
\newcommand{\fcase}[2]{\mathtt{case}~#1~#2}

% Talk about formal rules in example
\newcommand{\refrule}[1]{\textrm{Rule~(#1)}}

\newcommand{\herase}[1]{\left|#1\right|_\textsf{erase}}

\newcommand{\arrmatch}[2]{#1 \blacktriangleright_{\rightarrow} #2}


\newcommand{\TABperformAna}[5]{#1 \vdash & #2                & \xlongrightarrow{#4} & #5 & \Leftarrow #3}
\newcommand{\TABperformSyn}[6]{#1 \vdash & #2 \Rightarrow #3 & \xlongrightarrow{#4} & #5 \Rightarrow #6}
\newcommand{\TABperformTyp}[3]{& #1 & \xlongrightarrow{#2} & #3}

\newcommand{\TABperformMove}[3]{#1 & \xlongrightarrow{#2} & #3}
\newcommand{\TABperformDel}[2]{#1 \xlongrightarrow{\aDel} #2}

\newcommand{\sumhasmatched}[2]{#1 \mathrel{\textcolor{black}{\blacktriangleright_{+}}} #2}

\newcommand{\subminsyn}[1]{\mathsf{submin}_{\Rightarrow}(#1)}
\newcommand{\subminana}[1]{\mathsf{submin}_{\Leftarrow}(#1)}


\newcommand{\inparens}[1]{{\color{gray}(}#1{\color{gray})}}

%% rule names for appendix
\newcommand{\rname}[1]{\textsc{#1}}
\newcommand{\gap}{\vspace{7pt}}


% !TEX root = pattern-paper.tex
%% Statics

% typing of internal expressions
\newrulecommand{TVar}{TVar}
\newrulecommand{TNum}{TNum}
\newrulecommand{TLam}{TLam}
\newrulecommand{TAp}{TAp}
\newrulecommand{TPair}{TPair}
\newrulecommand{TPrl}{TPrl}
\newrulecommand{TPrr}{TPrr}
\newrulecommand{TInl}{TInl}
\newrulecommand{TInr}{TInr}
\newrulecommand{TMatchZPre}{TMatchZPre}
\newrulecommand{TMatchNZPre}{TMatchNZPre}
\newrulecommand{TEHole}{TEHole}
\newrulecommand{THole}{THole}

% typing of rule and rules
\newrulecommand{TRule}{TRule}
\newrulecommand{TOneRules}{TOneRules}
\newrulecommand{TRules}{TRules}

% typing of pattern
\newrulecommand{PTVar}{PTVar}
\newrulecommand{PTNum}{PTNum}
\newrulecommand{PTWild}{PTWild}
\newrulecommand{PTPair}{PTPair}
\newrulecommand{PTInl}{PTInl}
\newrulecommand{PTInr}{PTInr}
\newrulecommand{PTEHole}{PTEHole}
\newrulecommand{PTHole}{PTHole}

%% Dynamics

% typing of substitution
\newrulecommand{STEmpty}{STEmpty}
\newrulecommand{STExtend}{STExtend}

% refutable pattern
\newrulecommand{RNum}{RNum}
\newrulecommand{REHole}{REHole}
\newrulecommand{RHole}{RHole}
\newrulecommand{RInl}{RInl}
\newrulecommand{RInr}{RInr}
\newrulecommand{RPairL}{RPairL}
\newrulecommand{RPairR}{RPairR}

% match
\newrulecommand{MVar}{MVar}
\newrulecommand{MNum}{MNum}
\newrulecommand{MWild}{MWild}
\newrulecommand{MPair}{MPair}
\newrulecommand{MEHolePair}{MEHolePair}
\newrulecommand{MHolePair}{MHolePair}
\newrulecommand{MApPair}{MApPair}
\newrulecommand{MMatchPair}{MMatchPair}
\newrulecommand{MPrlPair}{MPrlPair}
\newrulecommand{MPrrPair}{MPrrPair}
\newrulecommand{MInl}{MInl}
\newrulecommand{MInr}{MInr}

% may match
\newrulecommand{MMEHole}{MMEHole}
\newrulecommand{MMHole}{MMHole}
\newrulecommand{MMExpEHole}{MMExpEHole}
\newrulecommand{MMExpHole}{MMExpHole}
\newrulecommand{MMAp}{MMAp}
\newrulecommand{MMMatch}{MMMatch}
\newrulecommand{MMPrl}{MMPrl}
\newrulecommand{MMPrr}{MMPrr}
\newrulecommand{MMPairL}{MMPairL}
\newrulecommand{MMPairR}{MMPairR}
\newrulecommand{MMPair}{MMPair}
\newrulecommand{MMInl}{MMInl}
\newrulecommand{MMInr}{MMInr}

% not match
\newrulecommand{NMNum}{NMNum}
\newrulecommand{NMPairL}{NMPairL}
\newrulecommand{NMPairR}{NMPairR}
\newrulecommand{NMConfL}{NMConfL}
\newrulecommand{NMConfR}{NMConfR}
\newrulecommand{NMInl}{NMInl}
\newrulecommand{NMInr}{NMInr}

% value
\newrulecommand{VNum}{VNum}
\newrulecommand{VLam}{VLam}
\newrulecommand{VPair}{VPair}
\newrulecommand{VInl}{VInl}
\newrulecommand{VInr}{VInr}

% indeterminate
\newrulecommand{IEHole}{IEHole}
\newrulecommand{IHole}{IHole}
\newrulecommand{IAp}{IAp}
\newrulecommand{IPairL}{IPairL}
\newrulecommand{IPairR}{IPairR}
\newrulecommand{IPair}{IPair}
\newrulecommand{IPrl}{IPrl}
\newrulecommand{IPrr}{IPrr}
\newrulecommand{IInl}{IInl}
\newrulecommand{IInr}{IInr}
\newrulecommand{IMatch}{IMatch}

% final
\newrulecommand{FVal}{FVal}
\newrulecommand{FIndet}{FIndet}

% step
\newrulecommand{ITHole}{ITHole}
\newrulecommand{ITApFun}{ITApFun}
\newrulecommand{ITApArg}{ITApArg}
\newrulecommand{ITAp}{ITAp}
\newrulecommand{ITPairL}{ITPairL}
\newrulecommand{ITPairR}{ITPairR}
\newrulecommand{ITPrl}{ITPrl}
\newrulecommand{ITPrr}{ITPrr}
\newrulecommand{ITInl}{ITInl}
\newrulecommand{ITInr}{ITInr}
\newrulecommand{ITExpMatch}{ITExpMatch}
\newrulecommand{ITSuccMatch}{ITSuccMatch}
\newrulecommand{ITFailMatch}{ITFailMatch}

% typing of constraints
\newrulecommand{CTTruth}{CTTruth}
\newrulecommand{CTFalsity}{CTFalsity}
\newrulecommand{CTUnknown}{CTUnknown}
\newrulecommand{CTNum}{CTNum}
\newrulecommand{CTNotNum}{CTNotNum}
\newrulecommand{CTAnd}{CTAnd}
\newrulecommand{CTOr}{CTOr}
\newrulecommand{CTInl}{CTInl}
\newrulecommand{CTInr}{CTInr}
\newrulecommand{CTPair}{CTPair}

% satisfaction judgment
\newrulecommand{CSTruth}{CSTruth}
\newrulecommand{CSNum}{CSNum}
\newrulecommand{CSNotNum}{CSNotNum}
\newrulecommand{CSAnd}{CSAnd}
\newrulecommandONE{CSOrL}{CSOrL}
\newrulecommandONE{CSOrR}{CSOrR}
\newrulecommand{CSInl}{CSInl}
\newrulecommand{CSInr}{CSInr}
\newrulecommand{CSPair}{CSPair}
\newrulecommand{CSEHolePair}{CSEHolePair}
\newrulecommand{CSHolePair}{CSHolePair}
\newrulecommand{CSApPair}{CSApPair}
\newrulecommand{CSMatchPair}{CSMatchPair}
\newrulecommand{CSPrlPair}{CSPrlPair}
\newrulecommand{CSPrrPair}{CSPrrPair}

% maybe satisfaction judgment
\newrulecommand{CMSUnknown}{CMSUnknown}
\newrulecommand{CMSExpEHole}{CMSExpEHole}
\newrulecommand{CMSExpHole}{CMSExpHole}
\newrulecommand{CMSAp}{CMSAp}
\newrulecommand{CMSMatch}{CMSMatch}
\newrulecommand{CMSPrl}{CMSPrl}
\newrulecommand{CMSPrr}{CMSPrr}
\newrulecommand{CMSAndL}{CMSAndL}
\newrulecommand{CMSAndR}{CMSAndR}
\newrulecommand{CMSAnd}{CMSAnd}
\newrulecommand{CMSOrL}{CMSOrL}
\newrulecommand{CMSOrR}{CMSOrR}
\newrulecommand{CMSInl}{CMSInl}
\newrulecommand{CMSInr}{CMSInr}
\newrulecommand{CMSPairL}{CMSPairL}
\newrulecommand{CMSPairR}{CMSPairR}
\newrulecommand{CMSPair}{CMSPair}

% must or maybe satisfaction judgment
\newrulecommand{CSMSMay}{CSMSMay}
\newrulecommand{CSMSSat}{CSMSSat}

% incon
\newrulecommand{CINCTruth}{CINCTruth}
\newrulecommand{CINCFalsity}{CINCFalsity}
\newrulecommand{CINCNum}{CINCNum}
\newrulecommand{CINCNotNum}{CINCNotNum}
\newrulecommand{CINCAnd}{CINCAnd}
\newrulecommand{CINCOr}{CINCOr}
\newrulecommand{CINCInj}{CINCInj}
\newrulecommand{CINCInl}{CINCInl}
\newrulecommand{CINCInr}{CINCInr}
\newrulecommand{CINCPairL}{CINCPairL}
\newrulecommand{CINCPairR}{CINCPairR}

\begin{document}
  
\title{Pattern Matching with Typed Holes}

\author{
  Yongwei Yuan\inst{1} \and 
}
\authorrunning{Y. Yuan et al.}
\institute{
  Purdue University, West Lafayette, USA \\
  \email{yuan311@purdue.edu} \and
  University of Michigan, Ann Arbor, USA
  \email{\{dmoo,hkpotter,wyuning,comar\}@umich.edu}}


\maketitle

\begin{abstract}
  The abstract should briefly summarize the contents of the paper in
  15--250 words.
  
  \keywords{live programming  \and typed holes \and contextual modal type theory.}
\end{abstract}

\section{Introduction}
\label{sec:intro}
Most contemporary programming environments either provide meaningful feedback only for complete programs, or fall back to limited heuristic approaches when the program is incomplete.
Recent work on modeling incomplete programs as typed expressions with \emph{holes} has focused on tackling this problem. \cite{DBLP:conf/popl/OmarVHAH17} describes a static semantics for incomplete functional programs. Based on that, \cite{DBLP:journals/pacmpl/OmarVCH19} develops a dynamic semantics to evaluate such incomplete programs.
These foundational treatments are being implemented and extended in the Hazel programming environment (hazel.org).

\emph{Pattern matching} is a cornerstone of functional programming languages in the ML family. 
However, \cite{DBLP:journals/pacmpl/OmarVCH19} only supports simple case analysis on binary sum types and does not support nested patterns nor pattern holes.
This paper addresses this problem, focusing on adding full ML-style pattern matching with support for pattern holes to the Hazelnut core calculus and implementing 
this system at full scale into Hazel. Consider the examples below, which contain expression and pattern holes, denoted $\hehole{u}$ (where each hole has a unique name, $u$). 

For the match expression shown in \Eqref{eq:complete}, the scrutinee contains a hole.
By considering the match two rules in order, we observe that
no matter how the hole $u$ is filled, $\hpair{\hinl{\tnum}{\hehole{u}}}{2}$ doesn't match $\hpair{\hinrp{x}}{\_}$,
and always matches $\hpair{\_}{x}$. And thus the value of the match expression is $2$, despite the fact that the program is incomplete.
If, however, we replaced the pattern in the first rule with $(\hinlp{3}, \_)$, it would be impossible to decide whether the pattern matches
or not, and so the match expression would be \emph{indeterminate}, i.e. awaiting hole filling.
%Similarly, if we had replaced the pattern in the first rule with $(\hinlp{x}, \hehole{w})$,
%then it is impossible to tell whether the pattern hole $w$ will match $2$ or not,
%so the match would also be indeterminate.

One of the benefits of pattern matching is that it allows for static reasoning about exhaustiveness and redundancy.
However, reasoning about these matters is subtle and complicated in the presence of holes. 
For example, consider the problem of checking if the rules in \Eqref{eq:pathole-exhaustive} cover all the possibilities, \ie \emph{exhaustiveness}, or if the second rule in \Eqref{eq:pathole-redundant} can never be reached no matter how the holes are filled, \ie \emph{redundancy}. 

This abstract focuses on the formalism of full-fledged pattern matching with typed holes so that the programming environment could give this sort of feedback even when considering incomplete match expressions.

\section{Live Pattern Matching}\label{sec:examples}

\begin{figure}[ht]
  \begin{tikzpicture}[remember picture, overlay]
    \node[rotate=-90,mark size=5pt,color=hazellightgreen,shift={(-0.1,-0.2)}] at (pic cs:exp-hole-1) {\pgfuseplotmark{triangle*}};
    \node[rotate=-90,mark size=5pt,color=hazellightgreen,shift={(-0.1,-0.2)}] at (pic cs:exp-hole-2) {\pgfuseplotmark{triangle*}};
  \end{tikzpicture}

  \centering
  \hspace*{\fill}%
  \captionsetup[subfloat]{labelformat=empty}
  \subfloat[]{%
    \makebox[\width]{
      $\begin{aligned}
        &\match (1::\?) \{ \\
        &\tikzmark{exp-hole-1}|~ \hrul{\nil}{\none} \\
        &|~ \hrul{x::xs}{\some{xs}} \\
        &\}
      \end{aligned}$
    }
  }%
  $\mapsto$
  \subfloat[]{
    \makebox[1.2\width]{
      $\begin{aligned}
        &\match (1::\?) \{ \\
        &|~ \hrul{\nil}{\none} \\
        &\tikzmark{exp-hole-2}|~ \hrul{1::xs}{\some{xs}} \\
        &\}
      \end{aligned}$
    }
  }%
  $\mapsto$
  \subfloat[]{
    $\some{?}$
  }
  \hspace*{\fill}%
  \caption{Pattern Matching with Expression Holes}
  \label{fig:exp-hole-step}
\end{figure}

\begin{figure}[ht]
  \begin{tikzpicture}[remember picture, overlay]
    \node[rotate=-90,mark size=5pt,color=hazellightgreen,shift={(-0.1,-0.2)}] at (pic cs:pat-hole-1) {\pgfuseplotmark{triangle*}};
    \node[rotate=-90,mark size=5pt,color=hazellightgreen,shift={(-0.1,-0.2)}] at (pic cs:pat-hole-2) {\pgfuseplotmark{triangle*}};
  \end{tikzpicture}

  \centering
  \hspace*{\fill}%
  \captionsetup[subfloat]{labelformat=empty}
  \subfloat[]{%
    \makebox[\width]{
      $\begin{aligned}
        &\match (1::\nil) \{ \\
        &\tikzmark{pat-hole-1}|~ \hrul{\nil}{\none} \\
        &|~ \hrul{\?::xs}{\some{xs}} \\
        &\}
      \end{aligned}$
    }
  }%
  $\mapsto$
  \subfloat[]{
    \makebox[1.2\width]{
      $\begin{aligned}
        &\match (1::\nil) \{ \\
        &|~ \hrul{\nil}{\none} \\
        &\tikzmark{pat-hole-2}|~ \hrul{\?::xs}{\some{xs}} \\
        &\}
      \end{aligned}$
    }
  }%
  $\not\mapsto$(indeterminate)
  \hspace*{\fill}%
  \caption{Pattern Matching with Pattern Holes}
  \label{fig:pat-hole-step}
\end{figure}

\begin{figure}[ht]
  \centering
  \hspace*{\fill}%
  \subfloat[May or may not be exhaustive.\label{fig:may-exhaustive}]{
    \makebox[2\width]{
      $\begin{aligned}
        &\match (1::\nil) \{ \\
        &|~ \hrul{\nil}{\none} \\
        &|~ \hrul{\?::xs}{\some{xs}} \\
        &\}
      \end{aligned}$
    }
  }%
  \hfill%
  \subfloat[Must not be exhaustive.\label{fig:not-exhasutive}]{
    \makebox[2\width]{
      $\begin{aligned}
        &\match (1::\nil) \{ \\
        &|~ \hrul{\nil}{\none} \\
        &|~ \hrul{1::\?}{\?} \\
        &\}
      \end{aligned}$
    }
  }%
  \hspace*{\fill}%
  \caption{Exhaustiveness Checking in Incomplete Match Expressions}
  \label{fig:exh-hole}
\end{figure}

\begin{figure}[ht]
  \centering
  \hspace*{\fill}%
  \subfloat[May or may not be redundant.\label{fig:may-redundant1}]{
    \makebox[0.8\width]{
      $\begin{aligned}
        &\match (1::2::\nil) \{ \\
        &|~ \hrul{\nil}{\none} \\
        &|~ \hrul{x::\?}{\?} \\
        &|~ \hrul{1::xs}{\some{xs}} \\
        &\}
      \end{aligned}$
    }
  }%
  \hfill%
  \subfloat[May or may not be redundant.\label{fig:may-redundant2}]{
    \makebox[0.8\width]{
      $\begin{aligned}
        &\match (1::2::\nil) \{ \\
        &|~ \hrul{\nil}{\none} \\
        &|~ \hrul{x::\?}{\?} \\
        &|~ \hrul{1::2::\nil}{\some{2:\nil}} \\
        &\}
      \end{aligned}$
    }
  }%
  \hfill%
  \subfloat[Must be redundant.\label{fig:redundant}]{
    \makebox[0.7\width]{
      $\begin{aligned}
        &\match (1::2::\nil) \{ \\
        &|~ \hrul{\nil}{\none} \\
        &|~ \hrul{x::xs}{\some{xs}} \\
        &|~ \hrul{\?::xs}{\some{xs}} \\
        &\}
      \end{aligned}$
    }
  }%
  \hspace*{\fill}%
  \caption{Redundancy Checking in Incomplete Match Expressions}
  \label{fig:red-hole}
\end{figure}

\section{Peanut: The Core Calculus}
With the intuition developed in the previous section, we will now formally specify the dynamic semantics and the static semantics, including exhaustiveness and redundancy checking, of a core calculus with support for pattern matching with typed holes in both expression and pattern position. 

% !TEX root= pattern-paper.tex

\begin{figure}[ht]
    \centering
    \begin{minipage}{.5\linewidth}
  $\arraycolsep=4pt\begin{array}{lll}
    e & ::= &
      x ~\vert~
      \hnum{n} \\
      & ~\vert~ &
      \hlam{x}{\tau}{e} ~\vert~
      \hap{e_1}{e_2} \\
      & ~\vert~ &
      \hpair{e_1}{e_2} ~\vert~
      \hfst{e} ~\vert~ \hsnd{e} \\
      & ~\vert~ &
      \hinl{\tau}{e} ~\vert~
      \hinr{\tau}{e} \\
      & ~\vert~ &
      \hmatch{e}{\hat{rs}} \\
      & ~\vert~ &
      \hehole{u} ~\vert~
      \hhole{e}{u} \\
    \end{array}$
    \end{minipage}%
    \begin{minipage}{.5\linewidth}
  $\arraycolsep=4pt\begin{array}{lll}
    \tau & ::= &
      \tnum ~\vert~
      \tarr{\tau_1}{\tau_2} ~\vert~
      \tprod{\tau_1}{\tau_2} ~\vert~
      \tsum{\tau_1}{\tau_2} \\
    \hat{rs} & ::= &
      \zrulsP{rs}{r}{rs} \\
    rs & ::= &
      \cdot ~\vert~ \hrulesP{r}{rs'} \\
    r & ::= &
      \hrul{p}{e} \\
    p & ::= &
      x ~\vert~
      \_ ~\vert~
      \hnum{n} ~\vert~
      \hpair{p_1}{p_2} \\
      & ~\vert~ &
      \hinlp{p} ~\vert~
      \hinrp{p} ~\vert~
      \heholep{w} ~\vert~
      \hholep{p}{w}{\tau}
    \end{array}$
    \end{minipage}
\caption{Syntax}
\label{fig:syntax}
\end{figure}
% !TEX root= pattern-paper.tex

\begin{figure}[ht]

\judgbox{\rmpointer{\zrules} = rs}
        {$rs$ can be obtained by erasing pointer from $\zrules$}
\begin{align*}
  \rmpointer{\zruls{\cdot}{r}{rs}} &= \hrules{r}{rs} \\
  \rmpointer{\zruls{\hrulesP{r'}{rs'}}{r}{rs}} &= \hrules{r'}{\rmpointer{\zruls{rs'}{r}{rs}}}
\end{align*}

\caption{Pointer Eraser flattens the rules with pointer.}
\label{fig:pointer-eraser}
\end{figure}


\paragraph{Overview.}
We will first go over the syntax of the core calculus in \autoref{sec:Syntax}\todo{change autoref settings to say Sec. 3.1 instead of subsection 3.1}. Then we will define the dynamic semantics as a small-step operational semantics with support for evaluating incomplete programs in \autoref{sec:dynamics}. Next, we define a corresponding static
semantics in \autoref{sec:statics} as a type assignment system together with a match constraint language
that we use to reason hypothetically about exhaustiveness and redundancy in the presence of
holes. Finally, in \autoref{sec:algorithm} we give an algorithmic formulation of the exhaustiveness and redundancy checker and prove it sound with respect to the
declarative formulation.

\subsection{Syntax}
\label{sec:Syntax}
The syntax of Peanut is given in \figurename~\ref{fig:syntax}. 
Peanut is based on the internal language of Hazelnut Live, a typed lambda calculus with only expression holes \cite{DBLP:journals/pacmpl/OmarVCH19}.
We choose numbers as the base type and add binary sums and binary products so that we have interesting
patterns to consider. We also remove the machinery related to gradual typing (casts and failed casts) to focus our attention on pattern matching in particular. Most forms are standard (we base our formulation on \cite{Harper2012}). We include functions, function application, pairs, explicit projection operators (for reasons we will consider below), and left and right injections, which are the introductory forms for sum types. Functions and injections include type annotations so that we can define a simple type assignment system. The forms of particular interest here are holes and match expressions.

Empty expression holes are written $\hehole{u}$ and non-empty expression holes, which serve as membranes around type inconsistencies, are written $\hhole{e}{u}$. Similarly, empty pattern holes are written $\hehole{w}$ and non-empty pattern holes, which are analogous, are written $\hhole{p}{w}$. Here, $u$ are expression hole identifiers and $w$ are pattern hole identifiers.
We are modeling an internal language, so we assume that hole identifiers 
correspond to unique holes in the user's original program, which we do not model here. We do not impose a uniqueness constraint, however, because a hole can 
appear multiple times during evaluation due to substitution.

A match expression, $\hmatch{e}{\zrules}$, 
consists of a scrutinee, $e$, and a zipper of rules, $\zrules$, i.e. a sequence of one or more rules with a pointer marking the rule currently being considered during evaluation (we assume that the marker is on the first rule initially). Syntactically, this is a triple, $\zruls{rs_{pre}}{r}{rs_{post}}$, consisting of a prefix rule sequence, $rs_{pre}$, a current rule, $r$, and a suffix rule sequence, $rs_{post}$. We can erase the pointer using the pointer erasure operator, $\rmpointer{\zrules}$, defined in \autoref{fig:pointer-eraser}. 
Each rule, $r$, consists of a pattern and an expression, which we call the branch expression.



\subsection{Dynamic Semantics}\label{sec:dynamics}
We will now define the dynamic semantics of Peanut. 
As with Hazelnut Live, and as suggested in \autoref{sec:examples}, it is possible to evaluate well-typed programs with holes. We will define the type system in \autoref{sec:statics}.

% !TEX root= pattern-paper.tex

\begin{figure}[ht]

\judgbox{\htrans{e}{e'}}{$e$ takes a step to $e'$}

\begin{mathpar}
\Infer{\ITHole}{
  \htrans{e}{e'}
}{
  \htrans{\hhole{e}{u}}{\hhole{e'}{u}}
}

\Infer{\ITApFun}{
  \htrans{e_1}{e_1'}
}{
  \htrans{\hap{e_1}{e_2}}{\hap{e_1'}{e_2}}
}

\Infer{\ITApArg}{
  \isFinal{e_1} \\
  \htrans{e_2}{e_2'}
}{
  \htrans{\hap{e_1}{e_2}}{\hap{e_1}{e_2'}}
}

\Infer{\ITAp}{
  \isFinal{e_2}
}{
  \hap{\hlam{x}{\tau}{e_1}}{e_2} \mapsto
    [e_2/x]e_1
}

\Infer{\ITPairL}{
  \htrans{e_1}{e_1'}
}{
  \htrans{\hpair{e_1}{e_2}}{\hpair{e_1'}{e_2}}
}

\Infer{\ITPairR}{
  \isFinal{e_1} \\
  \htrans{e_2}{e_2'}
}{
  \htrans{\hpair{e_1}{e_2}}{\hpair{e_1}{e_2'}}
}

\Infer{\ITPrl}{
  \isFinal{\hpair{e_1}{e_2}}
}{
  \htrans{\hprl{\hpair{e_1}{e_2}}}{e_1}
}

\Infer{\ITPrr}{
  \isFinal{\hpair{e_1}{e_2}}
}{
  \htrans{\hprr{\hpair{e_1}{e_2}}}{e_2}
}

\Infer{\ITInl}{
  \htrans{e}{e'}
}{
  \htrans{\hinl{\tau}{e}}{\hinl{\tau}{e'}}
}

\Infer{\ITInr}{
  \htrans{e}{e'}
}{
  \htrans{\hinr{\tau}{e}}{\hinr{\tau}{e'}}
}

\Infer{\ITExpMatch}{
  \htrans{e}{e'}
}{
  \htrans{\hmatch{e}{\zrules}}{\hmatch{e'}{\zrules}}
}

\Infer{\ITSuccMatch}{
  \isFinal{e} \\
  \hpatmatch{e}{p_r}{\theta}
}{
  \htrans{
    \hmatch{e}{\zruls{rs_{pre}}{\hrulP{p_r}{e_r}}{rs_{post}}}
  }{
    [\theta](e_r)
  }
}

\Infer{\ITFailMatch}{
  \isFinal{e} \\
  \hnotmatch{e}{p_r}
}{
  \htrans{
    \hmatch{e}{\zruls{rs}{\hrulP{p_r}{e_r}}{\hrulesP{r'}{rs'}}}
  }{
    \hmatch{e}{
      \zruls{
        \rmpointer{\zruls{rs}{\hrulP{p_r}{e_r}}{\cdot}}
      }{r'}{rs'}
    }
  }
}
\end{mathpar}

  \caption{Stepping}
  \label{fig:step}
\end{figure}


We define the dynamic semantics as a small-step operational semantics with a stepping judgment
$\htrans{e}{e'}$ defined in \autoref{fig:step} and irreducible states defined by the judgement $\isFinal{e}$ defined in \autoref{fig:final}. 

Most of the rules in \autoref{fig:step} and \autoref{fig:final} are adapted directly from Hazelnut Live \cite{DBLP:journals/pacmpl/OmarVCH19} (which was a contextual small-step semantics, whereas we choose to define a standard small-step semantics.) 

Final expressions are those that cannot be further reduced. These include
standard values, given by the judgement $\isVal{e}$ in \autoref{fig:final}, 
and \emph{indeterminate} expressions, which are those that are not values but cannot be further reduced due to one or more holes in a critical position, 
e.g. as the function position of an application. Indeterminate expressions
are specified by the judgement $\isIndet{e}$ in \autoref{fig:final}, following Hazelnut Live \cite{DBLP:journals/pacmpl/OmarVCH19}.

We will focus on rules related to match expressions.

In Sec.~\ref{sec:examples}, we use a rule pointer to simulate the process in
which we consider the rules in order. Correspondingly, in the syntax of the core
calculus, we use zipper rules of the form $\zrulsP{rs}{r}{rs}$ in match
expressions to represent the intermediate states when we are performing pattern
matching. The former $rs$ represents the preceding rules that has already been
considered and fails to match the scrutinee; the rule $r$ in the middle
represents the current rule being considered; the latter $rs$ represents the
remaining rules to be considered. \figurename~\ref{fig:pointer-eraser} defines a
helper function to flatten the zipper rules. If we need to move the rule pointer
to the next rule, we can append the current rule after the preceding rules, and
regard the initial rule in the remaining rule as the new current rule. The
conclusion of Rule \ITFailMatch demonstrates how it works.

% !TEX root= pattern-paper.tex

\begin{figure}[th]

\judgbox{\isVal{e}}{$e$ is a value}

\begin{mathpar}
\Infer{\VNum}{ }{
  \isVal{\hnum{n}}
}

\Infer{\VLam}{ }{
  \isVal{\hlam{x}{\tau}{e}}
}

\Infer{\VPair}{
  \isVal{e_1} \\
  \isVal{e_2}
}{
  \isVal{\hpair{e_1}{e_2}}
}

\Infer{\VInl}{
  \isVal{e}
}{
  \isVal{\hinl{\tau}{e}}
}

\Infer{\VInr}{
  \isVal{e}
}{
  \isVal{\hinr{\tau}{e}}
}
\end{mathpar}

\judgbox{\isIndet{e}}{$e$ is indeterminate}

\begin{mathpar}
\Infer{\IEHole}{ }{
  \isIndet{\hehole{u}}
}

\Infer{\IHole}{
  \isFinal{e}
}{
  \isIndet{\hhole{e}{u}}
}

\Infer{\IAp}{
  \isIndet{e_1} \\ \isFinal{e_2}
}{
  \isIndet{\hap{e_1}{e_2}}
}

\Infer{\IPairL}{
  \isIndet{e_1} \\ \isVal{e_2}
}{
  \isIndet{\hpair{e_1}{e_2}}
}

\Infer{\IPairR}{
  \isVal{e_1} \\ \isIndet{e_2}
}{
  \isIndet{\hpair{e_1}{e_2}}
}

\Infer{\IPair}{
  \isIndet{e_1} \\ \isIndet{e_2}
}{
  \isIndet{\hpair{e_1}{e_2}}
}

\Infer{\IFst}{
  \isIndet{e} \\ e \neq \hpair{e_1}{e_2}
}{
  \isIndet{\hfst{e}}
}

\Infer{\ISnd}{
  \isIndet{e} \\ e \neq \hpair{e_1}{e_2}
}{
  \isIndet{\hsnd{e}}
}

\Infer{\IInl}{
  \isIndet{e}
}{
  \isIndet{\hinl{\tau}{e}}
}

\Infer{\IInr}{
  \isIndet{e}
}{
  \isIndet{\hinr{\tau}{e}}
}

\Infer{\IMatch}{
  \isFinal{e} \\
  \hmaymatch{e}{p_r}
}{
  \isIndet{
    \hmatch{e}{\zruls{rs_{pre}}{\hrulP{p_r}{e_r}}{rs_{post}}}
  }
}
\end{mathpar}

\judgbox{\isFinal{e}}{$e$ is final}

\begin{mathpar}
\Infer{\FVal}{
  \isVal{e}
}{
  \isFinal{e}
}

\Infer{\FIndet}{
  \isIndet{e}
}{
  \isFinal{e}
}
\end{mathpar}

  \caption{Final expressions}
  \label{fig:final}
\end{figure}


\figurename~\ref{fig:step} defines the stepping judgment
$\htrans{e}{e'}$. We will focus on stepping judgments of match expressions.
Rule \ITExpMatch specifies that the match expression can take a step as its its
scrutinee can take a step. Note that Rule \ITFailMatch and Rule \ITSuccMatch
share the premise, $\isFinal{e}$, which is defined in \figurename~\ref{fig:final}. It
means that expression $e$ is \textit{final} in the sense that it is either already a
\textit{value} or \textit{indeterminate}, \ie, cannot be evaluated further due to unfilled holes
\cite{DBLP:journals/pacmpl/OmarVCH19}. And only when the scrutinee is final,
shall we consider the constinuent rules of the match expression in order. As in
the examples shown in \listfigurename~\ref{fig:exp-hole-step},\ref{fig:pat-hole-step},

\begin{itemize}
  \item
    if the final scrutinee does not match the pattern in the current rule,
    then we move the pointer to the next rule (Rule \ITFailMatch)

  \item
    if the final scrutinee does match the pattern in the current rule, 
    then we take the emitted substitution $\theta$ and apply it on the subexpression of that rule (Rule \ITSuccMatch)

  \item 
    if the final scrutinee may or may not match the pattern in the current rule,
    then the match expression is said to be indeterminate (Rule \IMatch)
\end{itemize}

% !TEX root= pattern-paper.tex

\begin{figure}[!b]

\judgbox{
  \hpatmatch{e}{p}{\theta}
}{
  $e$ matches $p$, emitting $\theta$
}

\begin{mathpar}
\Infer{\MVar}{ }{
  \hpatmatch{e}{x}{e / x}
}

\Infer{\MWild}{ }{
  \hpatmatch{e}{\_}{\cdot}
}

\Infer{\MNum}{ }{
  \hpatmatch{\hnum{n}}{\hnum{n}}{\cdot}
}

\Infer{\MPair}{
  \hpatmatch{e_1}{p_1}{\theta_1} \\
  \hpatmatch{e_2}{p_2}{\theta_2}
}{
  \hpatmatch{\hpair{e_1}{e_2}}{\hpair{p_1}{p_2}}{\theta_1 \uplus \theta_2}
}

\Infer{\MInl}{
  \hpatmatch{e}{p}{\theta}
}{
  \hpatmatch{\hinl{\tau}{e}}{\hinlp{p}}{\theta}
}

\Infer{\MInr}{
  \hpatmatch{e}{p}{\theta}
}{
  \hpatmatch{\hinr{\tau}{e}}{\hinrp{p}}{\theta}
}

\Infer{\MNotIntroPair}{
  \notIntro{e} \\
  \hpatmatch{\hprl{e}}{p_1}{\theta_1} \\
  \hpatmatch{\hprr{e}}{p_2}{\theta_2}
}{
  \hpatmatch{e}{\hpair{p_1}{p_2}}{\theta_1 \uplus \theta_2}
}
\end{mathpar}

\judgbox{
  \hnotmatch{e}{p}
}{
  $e$ does not match $p$
}

\begin{mathpar}

\Infer{\NMNum}{
  n_1 \neq n_2
}{
  \hnotmatch{\hnum{n_1}}{\hnum{n_2}}
}

\Infer{\NMPairL}{
  \hnotmatch{e_1}{p_1}
}{
  \hnotmatch{\hpair{e_1}{e_2}}{\hpair{p_1}{p_2}}
}

\Infer{\NMPairR}{
  \hnotmatch{e_2}{p_2}
}{
  \hnotmatch{\hpair{e_1}{e_2}}{\hpair{p_1}{p_2}}
}

\Infer{\NMConfL}{ }{
  \hnotmatch{\hinr{\tau}{e}}{\hinlp{p}}
}

\Infer{\NMConfR}{ }{
  \hnotmatch{\hinl{\tau}{e}}{\hinrp{p}}
}

\Infer{\NMInl}{
  \hnotmatch{e}{p}
}{
  \hnotmatch{\hinr{\tau}{e}}{\hinlp{p}}
}

\Infer{\NMInr}{
  \hnotmatch{e}{p}
}{
  \hnotmatch{\hinl{\tau}{e}}{\hinrp{p}}
}
\end{mathpar}

\judgbox{
  \hmaymatch{e}{p}
}{
  $e$ indeterminately matches $p$
}

\begin{mathpar}
\Infer{\MMEHole}{ }{
  \hmaymatch{e}{\heholep{w}}
}

\Infer{\MMHole}{ }{
  \hmaymatch{e}{\hholep{p}{w}{\tau}}
}
\\
\Infer{\MMNotIntro}{
  \notIntro{e} \\
  \refutable{p}
}{
  \hmaymatch{e}{p}
}

\Infer{\MMPairL}{
  \hmaymatch{e_1}{p_1} \\
  \hpatmatch{e_2}{p_2}{\theta_2}
}{
  \hmaymatch{\hpair{e_1}{e_2}}{\hpair{p_1}{p_2}}
}

\Infer{\MMPairR}{
  \hpatmatch{e_1}{p_1}{\theta_1} \\
  \hmaymatch{e_2}{p_2}
}{
  \hmaymatch{\hpair{e_1}{e_2}}{\hpair{p_1}{p_2}}
}

\Infer{\MMPair}{
  \hmaymatch{e_1}{p_1} \\
  \hmaymatch{e_2}{p_2}
}{
  \hmaymatch{\hpair{e_1}{e_2}}{\hpair{p_1}{p_2}}
}

\Infer{\MMInl}{
  \hmaymatch{e}{p}
}{
  \hmaymatch{\hinl{\tau}{e}}{\hinlp{p}}
}

\Infer{\MMInr}{
  \hmaymatch{e}{p}
}{
  \hmaymatch{\hinr{\tau}{e}}{\hinrp{p}}
}
\end{mathpar}

\caption{Three possible outcomes of pattern matching}
\label{fig:patmatch}
\end{figure}

% !TEX root= pattern-paper.tex

\begin{figure}[!ht]
\judgbox{
  \refutable{p}
}{$p$ is refutable}

\begin{mathpar}
\Infer{\RNum}{ }{
  \refutable{\hnum{n}}
}

\Infer{\REHole}{ }{
  \refutable{\heholep{w}}
}

\Infer{\RHole}{ }{
  \refutable{\hholep{p}{w}{\tau}}
}

\Infer{\RInl}{ }{
  \refutable{\hinlp{p}}
}

\Infer{\RInr}{ }{
  \refutable{\hinrp{p}}
}

\Infer{\RPairL}{
  \refutable{p_1}
}{
  \refutable{\hpair{p_1}{p_2}}
}

\Infer{\RPairR}{
  \refutable{p_2}
}{
  \refutable{\hpair{p_1}{p_2}}
}
\end{mathpar}
\caption{Refutable Patterns}
\label{fig:refutable}
\end{figure}

\figurename~\ref{fig:patmatch} defines three judgments corresponding to the
three possible results of pattern matching. We consider a final scrutinee $e$
and a pattern $p$ that are of the same type (see Sec.~\ref{sec:statics}). The
judgment $\hpatmatch{e}{p}{\theta}$ denotes that $e$ matches $p$, as witness by
the substitution $\theta$ defined on the variables in $p$, while the judgment
$\hnotmatch{e}{p}$ denotes that $e$ doesn't match $p$. Notably, we introduce the
third possibility $\hmaymatch{e}{p}$ --- $e$ may match $p$, which is also the
\textit{key concept} in pattern matching with holes. Rules \MMEHole and \MMHole
specifies that any final expression may match a pattern hole, regardless of
whether it is empty or non-empty. On the other hand, only when pattern $p$ is
refutable (see \figurename~\ref{fig:refutable}), can we say that an indeterminate
expression may or may not match the pattern. It is because any final expressions
must match an irrefutable pattern of the same type, and it does not make sense to allow
$\hpatmatch{e}{p}{\theta}$ and $\hmaymatch{e}{p}$ to be derivable at the same time. Actually,
the three possible results of pattern matching are mutually exclusive and at
least one of judgments can be derived under certain assumptions, as shown in the
following lemma.

\begin{lemma}[Matching Determinism]
  \label{lem:match-determinism}
  If $\isFinal{e}$ and $\hexptyp{\cdot}{\Delta}{e}{\tau}$ and $\hpattyp{p}{\tau}{\Gamma}{\Delta}$ then exactly one of the following holds
  \begin{enumerate}
    \item $\hpatmatch{e}{p}{\theta}$ for some $\theta$
    \item $\hmaymatch{e}{p}$
    \item $\hnotmatch{e}{p}$
  \end{enumerate}
\end{lemma}

The typing judgment $\hexptyp{\cdot}{\Delta}{e}{\tau}$ \todo{exptyp}The following theorem states that determinism of pattern matching ensures the determinism of dynamic semantics.

\begin{theorem}[Determinism]
  \label{thrm:determinism}
  If $\hexptyp{\cdot}{\Delta}{e}{\tau}$ then exactly one of the following holds
  \begin{enumerate}
    \item $\isVal{e}$
    \item $\isIndet{e}$
    \item $\htrans{e}{e'}$ for some unique $e'$
  \end{enumerate}
\end{theorem}

\subsection{Static Semantics}\label{sec:statics}

We have already introduced how pattern matching with holes works. Now, we want
to predict the runtime behavior of match expressions through checking
exhaustiveness and redundancy in static semantics. However, in order to do exhaustiveness checking and redundancy
checking of match expressions in \listfigurename~\ref{fig:exh-hole},
\ref{fig:red-hole}, we need to predict the result of pattern matching in static
semantics. We start by introducing \textit{match constraint language}, which
extends the idea in \cite{Harper2012}. Then, we build a similar type system to
\cite{DBLP:journals/pacmpl/OmarVCH19} by defining typing judgments for both
patterns and expressions. The former generates variable contexts $\Gamma$ and
hole contexts $\Delta$ (see \figurename~\ref{fig:pat-rulestyp}) while the latter 
takes variable contexts $\Gamma$ and hole contexts $\Delta$ as hypothesis (see
\figurename~\ref{fig:exptyp}).

\subsubsection{Typing of Expressions and Exhaustiveness Checking} \label{sec:exptyp}

We start by specifying the typing of expressions, particularly, match
expressions. And we will see that the definition of the typing judgments of
match expressions enforce exhaustiveness of the constituent rules.
\figurename~\ref{fig:exptyp} defines the typing judgment of expressions.

% !TEX root= pattern-paper.tex

\begin{figure}[ht]
\judgbox{
  \hexptyp{\Gamma}{\Delta}{e}{\tau}
}{
  $e$ is of type \(\tau\)
}
  \begin{mathpar}
%  \Infer{\TVar}{ }{
%    \hexptyp{\Gamma, x:\tau}{\Delta}{x}{\tau}
%  }
%
%  \Infer{\TEHole}{ }{
%    \hexptyp{\Gamma}{\Delta, u::\tau}{\hehole{u}}{\tau}
%  }
%
%  \Infer{\THole}{
%    \hexptyp{\Gamma}{\Delta, u::\tau}{e}{\tau'}
%  }{
%    \hexptyp{\Gamma}{\Delta, u::\tau}{\hhole{e}{u}}{\tau}
%  }
%
%  \Infer{\TNum}{ }{
%    \hexptyp{\Gamma}{\Delta}{\hnum{n}}{\tnum}
%  }
%
%  \Infer{\TLam}{
%    \hexptyp{\Gamma, x:\tau_1}{\Delta}{e}{\tau_2}
%  }{
%    \hexptyp{\Gamma}{\Delta}{\hlam{x}{\tau_1}{e}}{\tarr{\tau_1}{\tau_2}}
%  }
%
%  \Infer{\TAp}{
%    \hexptyp{\Gamma}{\Delta}{e_1}{\tarr{\tau_2}{\tau}} \\
%    \hexptyp{\Gamma}{\Delta}{e_2}{\tau_2}
%  }{
%    \hexptyp{\Gamma}{\Delta}{\hap{e_1}{e_2}}{\tau}
%  }
%
%  \Infer{\TPair}{
%    \hexptyp{\Gamma}{\Delta}{e_1}{\tau_1} \\
%    \hexptyp{\Gamma}{\Delta}{e_2}{\tau_2}
%  }{
%    \hexptyp{\Gamma}{\Delta}{\hpair{e_1}{e_2}}{\tprod{\tau_1}{\tau_2}}
%  }
%
%  \Infer{\TFst}{
%    \hexptyp{\Gamma}{\Delta}{e}{\tprod{\tau_1}{\tau_2}}
%  }{
%    \hexptyp{\Gamma}{\Delta}{\hfst{e}}{\tau_1}
%  }
%  
%  \Infer{\TSnd}{
%    \hexptyp{\Gamma}{\Delta}{e}{\tprod{\tau_1}{\tau_2}}
%  }{
%    \hexptyp{\Gamma}{\Delta}{\hsnd{e}}{\tau_2}
%  }
%  
%  \Infer{\TInl}{
%    \hexptyp{\Gamma}{\Delta}{e}{\tau_1}
%  }{
%    \hexptyp{\Gamma}{\Delta}{\hinl{\tau_2}{e}}{\tsum{\tau_1}{\tau_2}}
%  }
%
%  \Infer{\TInr}{
%    \hexptyp{\Gamma}{\Delta}{e}{\tau_2}
%  }{
%    \hexptyp{\Gamma}{\Delta}{\hinr{\tau_1}{e}}{\tsum{\tau_1}{\tau_2}}
%  }
  
  \Infer{TMatchZPre}{
    \hexptyp{\Gamma}{\Delta}{e}{\tau} \\
    \chrulstyp{\Gamma}{\Delta}{\cfalsity}{\hrules{r}{rs}}{\tau}{\xi}{\tau'} \\
    \csatisfyormay{\ctruth}{\xi}
  }{
  \hexptyp{\Gamma}{\Delta}{\hmatch{e}{\zruls{\cdot}{r}{rs}}}{\tau'}
  }

  \Infer{TMatchNZPre}{
    \hexptyp{\Gamma}{\Delta}{e}{\tau} \\
    \isFinal{e} \\
    \chrulstyp{\Gamma}{\Delta}{\cfalsity}{rs_{pre}}{\tau}{\xi_{pre}}{\tau'} \\
    \chrulstyp{\Gamma}{\Delta}{\xi_{pre}}{\hrules{r}{rs_{post}}}{\tau}{\xi_{rest}}{\tau'} \\
    \cnotsatisfyormay{e}{\xi_{pre}} \\
    \csatisfyormay{\ctruth}{\cor{\xi_{pre}}{\xi_{rest}}}
  }{
    \hexptyp{\Gamma}{\Delta}{\hmatch{e}{\zruls{rs_{pre}}{r}{rs_{post}}}}{\tau'}
  }
  \end{mathpar}

\caption{Match Expression Typing}
\label{fig:exptyp}
\end{figure}


Rule \TMatchZPre corresponds to the case that we have not started pattern
matching. The first premise specifies that the scrutinee $e$ is of type $\tau$,
and the second premise specifies that the constituent rules $\hrul{r}{rs}$ are not only
well-typed but also transforms a final expression of the same type as the
scrutinee, into a final expression of type $\tau'$. Notably, it generates a
constraint $\xi$ associated with the constituent rules $rs$. Then we use the
third premise $\csatisfyormay{\ctruth}{\xi}$ to ensure that there is at least
one rule whose pattern does match or may match the final scrutinee (see
Sec.~\ref{sec:constraint}). In other words, for a well-typed match expression,
it is impossible that the final scrutinee fails to match all the patterns as we
consider rules $rs$ in order.

Rule \TMatchNZPre corresponds to the case that we have already started pattern
matching and have already considered preceding rules $rs_{pre}$. First of all,
the scrutinee should not only be well-typed but also be final. Next, other than
ensuring the exhaustiveness of the constituent rules, we want to make sure that
at least one of the remaining rules $r_{post}$ would be taken. Note
that only when the final scrutinee $e$ cannot match the pattern $p$, \ie,
$\hnotmatch{e}{p}$, can we move the rule pointer. By
\lemmaname~\ref{lem:match-determinism}, for all patterns $p$ in the preceding
rules, neither $\hpatmatch{e}{p}{\theta}$ nor $\hmaymatch{e}{p}$ is derivable.
Then by \lemmaname~\ref{lem:const-matching-coherence}, we can derive the premise
in Rule \TMatchNZPre, $\cnotsatisfyormay{e}{\xi_{pre}}$. And thus, the type of
the match expression is preserved (\theoremname~\ref{thrm:preservation})as we consider rules in order.

\subsubsection{Typing of Patterns and Rules, and Redundancy Checking}
\label{sec:pattyp}
% !TEX root= pattern-paper.tex

\begin{figure}[!ht]
  \judgbox{
    \chpattyp{p}{\tau}{\xi}{\Gamma}{\Delta}
  }{
    $p$ is assigned type $\tau$ and emits constraint $\xi$
  }

  \begin{mathpar}
  \Infer{\PTVar}{ }{
    \chpattyp{x}{\tau}{\ctruth}{x : \tau}{\cdot}
  }

  \Infer{\PTWild}{ }{
    \chpattyp{\_}{\tau}{\ctruth}{\cdot}{\cdot}
  }

  \Infer{\PTEHole}{ }{
    \chpattyp{\heholep{w}}{\tau}{\cunknown}{\cdot}{\Delta , w :: \tau}
  }

  \Infer{\PTHole}{
    \chpattyp{p}{\tau}{\xi}{\Gamma}{\Delta , w :: \tau'}
  }{
    \chpattyp{\hholep{p}{w}{\tau}}{\tau'}{\cunknown}
    {\Gamma}{\Delta , w :: \tau'}
  }
  
  \Infer{\PTNum}{ }{
    \chpattyp{\hnum{n}}{\tnum}{\cnum{n}}{\cdot}{\Delta}
  }

  \Infer{\PTInl}{
    \chpattyp{p}{\tau_1}{\xi}{\Gamma}{\Delta}
  }{
    \chpattyp{\hinlp{p}}{\tsum{\tau_1}{\tau_2}}{\cinl{\xi}}{\Gamma}{\Delta}
  }

  \Infer{\PTInr}{
    \chpattyp{p}{\tau_2}{\xi}{\Gamma}{\Delta}
  }{
    \chpattyp{\hinrp{p}}{\tsum{\tau_1}{\tau_2}}{\cinr{\xi}}{\Gamma}{\Delta}
  }

  \Infer{\PTPair}{
    \chpattyp{p_1}{\tau_1}{\xi_1}{\Gamma_1}{\Delta} \\
    \chpattyp{p_2}{\tau_2}{\xi_2}{\Gamma_2}{\Delta}
  }{
    \chpattyp{\hpair{p_1}{p_2}}{\tprod{\tau_1}{\tau_2}}
    {\cpair{\xi_1}{\xi_2}}{\Gamma_1 \uplus \Gamma_2}{\Delta}
  }

  \end{mathpar}

  \judgbox{
    \chrultyp{\Gamma}{\Delta}{r}{\tau}{\xi}{\tau'}
  }{$r$ transforms a final expression of type $\tau$ \\ to a final expression of type $\tau'$}

  \begin{mathpar}
    \Infer{\TRule}{
      \chpattyp{p}{\tau}{\xi}{\Gamma_p}{\Delta_p} \\
      \hexptyp{\Gamma \uplus \Gamma_p}{\Delta \uplus \Delta_p}{e}{\tau'}
    }{
      \chrultyp{\Gamma}{\Delta}{\hrulP{p}{e}}{\tau}{\xi}{\tau'}
    }
  \end{mathpar}

  \judgbox{\chrulstyp{\Gamma}{\Delta}{\xi_{pre}}{rs}{\tau}{\xi_{rs}}{\tau'}}
  {$rs$ transforms a final expression of type $\tau$ \\ to a final expression of type $\tau'$}

  \begin{mathpar}
  \Infer{TOneRules}{
    \chrultyp{\Gamma}{\Delta}{r}{\tau}{\xi_r}{\tau'} \\
    \cnotsatisfy{\xi_r}{\xi_{pre}}
  }{
    \chrulstyp{\Gamma}{\Delta}{\xi_{pre}}{\hrulesP{r}{\cdot}}{\tau}{\xi_r}{\tau'}
  }

  \Infer{TRules}{
    \chrultyp{\Gamma}{\Delta}{r}{\tau}{\xi_r}{\tau'} \\
    \chrulstyp{\Gamma}{\Delta}{\cor{\xi_{pre}}{\xi_r}}{rs}
    {\tau}{\xi_{rs}}{\tau'} \\
    \cnotsatisfy{\xi_r}{\xi_{pre}}
  }{
    \chrulstyp{\Gamma}{\Delta}{\xi_{pre}}{\hrules{r}{rs}}
    {\tau}{\cor{\xi_r}{\xi_{rs}}}{\tau'}
  }
\end{mathpar}
\caption{Typing of Patterns, Single Rules, and Series of Rules}
\label{fig:pat-rulestyp}
\end{figure}


\figurename~\ref{fig:pat-rulestyp} defines the typing judgment for patterns $p$,
single rules $r$, and series of rules $rs$. We will see how constraint $\xi$ is
generated, accumulated, and used to check redundancy of a rule $r$ with respect
to its preceding ones.

The typing judgment of series of rules $rs$ is of the form
$\chrulstyp{\Gamma}{\Delta}{\xi_{pre}}{rs}{\tau_1}{\xi_{rs}}{\tau_2}$. As shown
in Rules \TMatchZPre and \TMatchNZPre, the constituent rules inherit the
variable context $\Gamma$ and hole context $\Delta$ from the outer match
expression. When we check the type of a series of rules, we consider each rule
in order, just as how we do pattern matching in Sec.~\ref{sec:dynamics}.

Rule \TRules corresponds to the inductive case. The first premise is to check
the type of the initial rule $r$. It specifies that each rule takes a final expression
of type $\tau_1$ and returns a final expression of type $\tau_2$. It also emits
a constraint $\xi_r$, which is actually emitted from the pattern of rule $r$ as
we will see later. In order to determine if the initial rule $r$ of the rules
$\hrules{r}{rs}$ is redundant with respect to its preceding rules, we use
$\xi_{pre}$ to keep track of the pattern matching information of preceding
rules. To accomplish that, as we drop the initial rule $r$, we append the
constraint $\xi_r$ emitted from the pattern of $r$, to the constraint
$\xi_{pre}$, and use $\cor{\xi_{pre}}{\xi_r}$ as the new input to inductively check the type
of the trailing rules $rs$ in the second premise. Now that we have shown how to maintain the
constraint $\xi_{pre}$ associated with the preceding rules, we can compare it
with the constraint of the current rule, $\xi_r$. As we check the
type of rules, we consider each rule in order and use
$\cnotsatisfy{\xi_r}{\xi_{pre}}$ to ensure that the current rule $r$ doesn't
have to be redundant with respect to its preceding rules. We will see in
Sec.~\ref{sec:constraint} that $\csatisfy{\xi_r}{\xi_{pre}}$ corresponds to
``must redundant''. At the same time, the judgment also outputs the accumulated
constraint collected from rules $\hrules{r}{rs}$, which is used to check
exhaustiveness of rules, as we have shown in Sec.~\ref{sec:exptyp}

Rule \TOneRules corresponds to the base case that the series of rules contains
only one rule. The premises are similar to that of Rule \TRules except that
there is no trailing rules to check the type of. The reason why we regard one
rule as the base case instead of empty rules, is that since our match expression
takes zipper rules, we will never need to check the type of empty rules. The
only case that it makes sense to allow a match expression to contain empty
rules, is when we match on a final expression of \textit{Void} type and thus
don't need to worry about exhaustiveness checking. It turns out that we do not
have to sacrifice the generality (see Appendix \todo{match(){.}}).

As we have briefly mentioned above, Rule \TRule specifies that rule
$\hrul{p}{e}$ transforms final expressions of type $\tau_1$ to final expressions
of type $\tau_2$. The first premise is the typing judgment of patterns---by
assigning pattern $p$ with type $\tau_1$, we collect the typing for all the
variables and holes involved in the pattern $p$ and generate variable context
$\Gamma_p$ and hole context $\Delta_p$. Additionally, it emits constraint $\xi$,
which is closely associated with the pattern itself. While constraint is used to
identify the set of final expressions that match $p$, we introduce
\textit{Unknown} constraint $\cunknown$ to denote the set of final expression
matching $p$ is yet to be determined (Rules \PTEHole and \PTHole). We will elaborate on what constraint is in
Sec.~\ref{sec:constraint}. The second premise strictly extends two contexts of
rule $r$ with that generated from pattern $p$, and check the type of
sub-expression $e$.

\subsubsection{Type Safety}
The type safety of the language is established by
\theoremname~\ref{thrm:determinism} and \theoremname~\ref{thrm:preservation}.

\begin{theorem}[Preservation]
  \label{thrm:preservation}
  If $\hexptyp{\cdot}{\Delta}{e}{\tau}$ and $\htrans{e}{e'}$
  then $\hexptyp{\cdot}{\Delta}{e'}{\tau}$
\end{theorem}

% !TEX root= pattern-paper.tex

\begin{figure}[ht]

\judgbox{
  \hsubstyp{\theta}{\Gamma}
}{
  $\theta$ is of type $\Gamma$
}

\begin{mathpar}
\Infer{\STEmpty}{ }{
  \hsubstyp{\emptyset}{\cdot}
}

\Infer{\STExtend}{
  \hsubstyp{\theta}{\Gamma_\theta} \\
  \hexptyp{\Gamma}{\Delta}{e}{\tau}
}{
  \hsubstyp{\theta , x / e}{\Gamma_\theta , x : \tau}
}
\end{mathpar}

\caption{Typing of Substitution}
\label{fig:substyp}
\end{figure}


\figurename~\ref{fig:substyp} defines the typing of substitution $\theta$.

To prove \theoremname~\ref{thrm:preservation}, we need the following three lemmas.
When considering Rule \ITAp, \lemmaname~\ref{lem:substitution} is needed.
When considering Rule \ITSuccMatch, \lemmaname~\ref{lem:subs-typing} is needed
to show the typing of substitution $\theta$, and we use
\lemmaname~\ref{lem:simult-substitution} to show that type is preserved when pattern
matching succeeds.

\begin{lemma}[Substitution]
  \label{lem:substitution}
  If $\hexptyp{\Gamma, x : \tau}{\Delta}{e_0}{\tau_0}$ and $\hexptyp{\Gamma}{\Delta}{e}{\tau}$
  then $\hexptyp{\Gamma}{\Delta}{[e/x]e_0}{\tau_0}$
\end{lemma}

\begin{lemma}[Substitution Typing]
  \label{lem:subs-typing}
  If $\hpatmatch{e}{p}{\theta}$ and $\hexptyp{\cdot}{\Delta}{e}{\tau}$ and $\hpattyp{p}{\tau}{\Gamma}{\Delta'}$
  then $\hsubstyp{\theta}{\Gamma}$
\end{lemma}

\begin{lemma}[Simultaneous Substitution]
  \label{lem:simult-substitution}
  If $\hexptyp{\Gamma \uplus \Gamma'}{\Delta}{e}{\tau}$ and $\hsubstyp{\theta}{\Gamma'}$
  then $\hexptyp{\Gamma}{\Delta}{[\theta]e}{\tau}$
\end{lemma}

\subsubsection{Match Constraint Language}\label{sec:constraint}
% !TEX root= pattern-paper.tex

\begin{figure}[ht]
$\arraycolsep=4pt\begin{array}{lll}
\xi & ::= &
  \ctruth ~\vert~
  \cfalsity ~\vert~
  \cunknown ~\vert~
  \cnum{n} ~\vert~
  \cnotnum{n} ~\vert~
  \cand{\xi_1}{\xi_2} ~\vert~
  \cor{\xi_1}{\xi_2} ~\vert~
  \cinl{\xi} ~\vert~
  \cinr{\xi} ~\vert~
  \cpair{\xi_1}{\xi_2}
\end{array}$

\judgbox{\ctyp{\xi}{\tau}}{$\xi$ constrains values of type $\tau$}

\begin{mathpar}
\Infer{\CTTruth}{ }{
  \ctyp{\ctruth}{\tau}
}

\Infer{\CTFalsity}{ }{
  \ctyp{\cfalsity}{\tau}
}

\Infer{\CTUnknown}{ }{
  \ctyp{\cunknown}{\tau}
}

\Infer{\CTNum}{ }{
  \ctyp{\cnum{n}}{\tnum}
}

\Infer{\CTNotNum}{ }{
  \ctyp{\cnotnum{n}}{\tnum}
}

\Infer{\CTAnd}{
  \ctyp{\xi_1}{\tau} \\ \ctyp{\xi_2}{\tau}
}{
  \ctyp{\cand{\xi_1}{\xi_2}}{\tau}
}

\Infer{\CTOr}{
  \ctyp{\xi_1}{\tau} \\ \ctyp{\xi_2}{\tau}
}{
  \ctyp{\cor{\xi_1}{\xi_2}}{\tau}
}

\Infer{\CTInl}{
  \ctyp{\xi_1}{\tau_1}
}{
  \ctyp{\cinl{\xi_1}}{\tsum{\tau_1}{\tau_2}}
}

\Infer{\CTInr}{
  \ctyp{\xi_2}{\tau_2}
}{
  \ctyp{\cinr{\xi_2}}{\tsum{\tau_1}{\tau_2}}
}

\Infer{\CTPair}{
  \ctyp{\xi_1}{\tau} \\ \ctyp{\xi_2}{\tau}
}{
  \ctyp{\cpair{\xi_1}{\xi_2}}{\tau}
}
\end{mathpar}

\judgbox{\cdual{\xi_1} = \xi_2}{dual of $\xi_1$ is $\xi_2$}
\begin{subequations}\label{defn:dual}
\begin{align}
  \cdual{\ctruth} &= \cfalsity \\
  \cdual{\cfalsity} &= \ctruth \\
  \cdual{\cunknown} &= \cunknown \\
  \cdual{\cnum{n}} &= \cnotnum{n} \\
  \cdual{\cnotnum{n}} &= \cnum{n} \\
  \cdual{\cand{\xi_1}{\xi_2}} &= \cor{\cdual{\xi_1}}{\cdual{\xi_2}} \\
  \cdual{\cor{\xi_1}{\xi_2}} &= \cand{\cdual{\xi_1}}{\cdual{\xi_2}} \\
  \cdual{\cinl{\xi_1}} &= \cor{ \cinl{\cdual{\xi_1}} }{ \cinr{\ctruth} } \\
  \cdual{\cinr{\xi_2}} &= \cor{ \cinr{\cdual{\xi_2}} }{ \cinl{\ctruth} } \\
  \cdual{\cpair{\xi_1}{\xi_2}} &=
  \cor{ \cor{ 
    \cpair{\xi_1}{\cdual{\xi_2}}
  }{
    \cpair{\cdual{\xi_1}}{\xi_2}
  }}{
    \cpair{\cdual{\xi_1}}{\cdual{\xi_2}}
  }
\end{align}
\end{subequations}
  \caption{Match Constraints}
  \label{fig:constraint}
\end{figure}


\figurename~\ref{fig:constraint} introduces match constraint language, which is
used to identify a subset of the final expressions of a type. The judgment
$\ctyp{\xi}{\tau}$ specifies that constraint $\xi$ constrains the final
expressions of type $\tau$. The dual of $\xi$, $\cdual{\xi}$ represents the
complement of the subset identified by $\xi$ under the set of the final
expressions of a type.

% !TEX root= pattern-paper.tex

\begin{figure}[!ht]

\judgbox{\csatisfy{e}{\xi}}{$e$ satisfies $\xi$}

\begin{mathpar}
\Infer{\CSTruth}{ }{
  \csatisfy{e}{\ctruth}
}

\Infer{\CSNum}{ }{
  \csatisfy{\hnum{n}}{\cnum{n}}
}

\Infer{\CSNotNum}{
  n_1 \neq n_2
}{
  \csatisfy{\hnum{n_1}}{\cnotnum{n_2}}
}

\Infer{\CSAnd}{
  \csatisfy{e}{\xi_1} \\
  \csatisfy{e}{\xi_2}
}{
  \csatisfy{e}{\cand{\xi_1}{\xi_2}}
}

\Infer{\CSOrL}{
  \csatisfy{e}{\xi_1}
}{
  \csatisfy{e}{\cor{\xi_1}{\xi_2}}
}

\Infer{\CSOrR}{
  \csatisfy{e}{\xi_2}
}{
  \csatisfy{e}{\cor{\xi_1}{\xi_2}}
}

\Infer{\CSInl}{
  \csatisfy{e_1}{\xi_1}
}{
  \csatisfy{
    \hinl{\tau_2}{e_1}
  }{
    \cinl{\xi_1}
  }
}

\Infer{\CSInr}{
  \csatisfy{e_2}{\xi_2}
}{
  \csatisfy{
    \hinr{\tau_1}{e_2}
  }{
    \cinr{\xi_2}
  }
}

\Infer{\CSPair}{
  \csatisfy{e_1}{\xi_1} \\
  \csatisfy{e_2}{\xi_2}
}{
\csatisfy{\hpair{e_1}{e_2}}{\cpair{\xi_1}{\xi_2}}
}

\Infer{\CSNotIntroPair}{
  \notIntro{e} \\
  \csatisfy{\hprl{e}}{\xi_1} \\
  \csatisfy{\hprr{e}}{\xi_2}
}{
  \csatisfy{e}{\cpair{\xi_1}{\xi_2}}
}
\end{mathpar}

\judgbox{\cmaysatisfy{e}{\xi}}{$e$ may satisfy $\xi$}

\begin{mathpar}
\Infer{\CMSUnknown}{ }{
  \cmaysatisfy{e}{\cunknown}
}

\Infer{\CMSNotIntro}{
  \notIntro{e} \\
  \refutable{\xi}
}{
  \cmaysatisfy{e}{\xi}
}

\Infer{\CMSAndL}{
  \cmaysatisfy{e}{\xi_1} \\
  \csatisfy{e}{\xi_2}
}{
  \cmaysatisfy{e}{\cand{\xi_1}{\xi_2}}
}

\Infer{\CMSAndR}{
  \csatisfy{e}{\xi_1} \\
  \cmaysatisfy{e}{\xi_2}
}{
  \cmaysatisfy{e}{\cand{\xi_1}{\xi_2}}
}

\Infer{\CMSAnd}{
  \cmaysatisfy{e}{\xi_1} \\
  \cmaysatisfy{e}{\xi_2}
}{
  \cmaysatisfy{e}{\cand{\xi_1}{\xi_2}}
}

\Infer{\CMSOrL}{
  \cmaysatisfy{e}{\xi_1} \\
  \cnotsatisfy{e}{\xi_2}
}{
  \cmaysatisfy{e}{\cor{\xi_1}{\xi_2}}
}

\Infer{\CMSOrR}{
  \cnotsatisfy{e}{\xi_1} \\
  \cmaysatisfy{e}{\xi_2}
}{
  \cmaysatisfy{e}{\cor{\xi_1}{\xi_2}}
}

\Infer{\CMSInl}{
  \cmaysatisfy{e_1}{\xi_1}
}{
  \cmaysatisfy{
    \hinl{\tau_2}{e_1}
  }{
    \cinl{\xi_1}
  }
}

\Infer{\CMSInr}{
  \cmaysatisfy{e_2}{\xi_2}
}{
  \cmaysatisfy{
    \hinr{\tau_1}{e_2}
  }{
    \cinr{\xi_2}
  }
}

\Infer{\CMSPairL}{
  \cmaysatisfy{e_1}{\xi_1} \\
  \csatisfy{e_2}{\xi_2}
}{
  \cmaysatisfy{\hpair{e_1}{e_2}}{\cpair{\xi_1}{\xi_2}}
}

\Infer{\CMSPairR}{
  \csatisfy{e_1}{\xi_1} \\
  \cmaysatisfy{e_2}{\xi_2}
}{
  \cmaysatisfy{\hpair{e_1}{e_2}}{\cpair{\xi_1}{\xi_2}}
}

\Infer{\CMSPair}{
  \cmaysatisfy{e_1}{\xi_1} \\
  \cmaysatisfy{e_2}{\xi_2}
}{
  \cmaysatisfy{\hpair{e_1}{e_2}}{\cpair{\xi_1}{\xi_2}}
}
\end{mathpar}

\judgbox{\csatisfyormay{e}{\xi}}{$e$ satisfies or may satisfy $\xi$}

\begin{mathpar}
\Infer{\CSMSMay}{
  \cmaysatisfy{e}{\xi}
}{
  \csatisfyormay{e}{\xi}
}

\Infer{\CSMSSat}{
  \csatisfy{e}{\xi}
}{
  \csatisfyormay{e}{\xi}
}
\end{mathpar}

  \caption{Satisfaction Judgments}
  \label{fig:satisfy}
\end{figure}


\figurename~\ref{fig:satisfy} defines satisfaction judgments. As we only
consider final expressions and patterns of the same type when talking about
pattern matching, a constraint only constrains final expressions of the same
type. And the satisfaction judgments does not make sense when the expression is
not final or the expression and the constraint are of different type. The
judgment $\csatisfy{e}{\xi}$ specifies that expression $e$ satisfies $\xi$ while
the judgment $\cmaysatisfy{e}{\xi}$ specifies that expression $e$ may or may not
satisfy $\xi$. The judgment $\csatisfyormay{e}{\xi}$ is the combination of the
previous two cases. It turns out that the remaining case where
$\csatisfyormay{e}{\xi}$ is not derivable, can be represented by
$\csatisfy{e}{\cdual{\xi}}$.

\begin{theorem}[Exclusiveness of Constraint Satisfaction]
  \label{thrm:exclusive-constraint-satisfaction}
  If $\ctyp{\xi}{\tau}$ and $\hexptyp{\cdot}{\Delta}{e}{\tau}$ and $\isFinal{e}$ then exactly one of the following holds
  \begin{enumerate}
  \item $\csatisfy{e}{\xi}$
  \item $\cmaysatisfy{e}{\xi}$
  \item $\csatisfy{e}{\cdual{\xi}}$
  \end{enumerate}
\end{theorem}

\lemmaname~\ref{lem:const-matching-coherence} establishes a correspondence
between pattern matching results and satisfaction judgments. That makes
reasoning pattern matching in type system possible and helps prove
\theoremname~\ref{thrm:preservation}.

\begin{lemma}[Matching Coherence of Constraint]
  \label{lem:const-matching-coherence}
  Suppose that $\hexptyp{\cdot}{\Delta}{e}{\tau}$ and $\isFinal{e}$ and $\chpattyp{p}{\tau}{\xi}{\Gamma}{\Delta}$. Then we have
  \begin{enumerate}
  \item $\csatisfy{e}{\xi}$ iff $\hpatmatch{e}{p}{\theta}$
  \item $\csatisfy{e}{\cdual{\xi}}$ iff $\hnotmatch{e}{p}$
  \item $\cmaysatisfy{e}{\xi}$ iff $\hmaymatch{e}{p}$
  \end{enumerate}
\end{lemma}

The following two definitions take advantage satisfaction judgments and
corresponds to redundancy and exhaustiveness respectively. We will see how they
can be determined in Sec.~\ref{sec:algorithm}

\begin{definition}[Entailment of Constraints]
  \label{defn:const-entailment}
  Suppose that $\ctyp{\xi_1}{\tau}$ and $\ctyp{\xi_2}{\tau}$.
  Then $\csatisfy{\xi_1}{\xi_2}$ iff for all $e$ such that $\hexptyp{\cdot}{\Delta}{e}{\tau}$ and $\isVal{e}$ we have $\csatisfyormay{e}{\xi_1}$ implies $\csatisfy{e}{\xi_2}$
\end{definition}

Recall in Rules \TOneRules and \TRules, we use $\cnotsatisfy{\xi_r}{\xi_{pre}}$ to ensure rule $r$ does not have to be redundant with respect to its preceding rules $rs_{pre}$. When considering the redundancy of a specific rule in a match expression, the programmer only want to be warned when the rule is destined be redundant, regardless of how the programmer fills the holes at the end. Therefore, only when all values that must or may match the pattern of rule $r$, must have already matched one of the patterns in its preceding rules $rs_{pre}$, can we say $r$ must be redundant with respect to $rs_{pre}$.

\begin{definition}[Not Not Entailment of Constraints]
  \label{defn:nn-entailment}
  Suppose that $\ctyp{\xi_1}{\tau}$ and $\ctyp{\xi_2}{\tau}$. Then $\csatisfyormay{\xi_1}{\xi_2}$ iff for all $e$ such that $\hexptyp{\cdot}{\Delta}{e}{\tau}$ and $\isFinal{e}$ we have $\csatisfyormay{e}{\xi_1}$ implies $\csatisfyormay{e}{\xi_2}$ 
\end{definition}

Recall in Rules \TMatchZPre and \TMatchNZPre, we use $\csatisfyormay{\ctruth}{\xi}$ to ensure that the rules associated with constraint $\xi$ either must or may be exhaustive. When considering the exhaustiveness of a sequence of rules, the programmer only want to be warned when the rules cannot be exhaustive, regardless of how the programmer fills the holes at the end. Then, we just need to ensure that for all values $e$, $e$ either must or may match one of the patterns in the sequence of the rules. For simplicity when proving the progress part of  \theoremname~\ref{thrm:determinism}, we consider all final expressions instead. In this way, even when the match expression is not complete and we may match on an indeterminate expression, we can still be confident that we won't have a match failure error. And actually, as we will later show in Sec.~\ref{sec:algorithm}, it is legitimate to replace values with final expressions in the quantifier.

\subsection{Decidability}\label{sec:algorithm}

We have already shown in Sec.~\ref{sec:statics} how to check redundancy and
exhaustiveness using \definitionname~\ref{defn:const-entailment} and
\definitionname~\ref{defn:nn-entailment}, but it remains unclear how to
determine whether they are true or false.

% !TEX root= pattern-paper.tex

\begin{figure}[ht]
\judgbox{\ctruify{\xi_1} = \xi_2}{}

\begin{align*}
  \ctruify{\ctruth} &= \ctruth \\
  \ctruify{\cfalsity} &= \cfalsity \\
  \ctruify{\cunknown} &= \ctruth \\
  \ctruify{\cnum{n}} &= \cnum{n} \\
  \ctruify{\cnotnum{n}} &= \cnotnum{n} \\
  \ctruify{\cand{\xi_1}{\xi_2}} &= \cand{\ctruify{\xi_1}}{\ctruify{\xi_2}} \\
  \ctruify{\cor{\xi_1}{\xi_2}} &= \cor{\ctruify{\xi_1}}{\ctruify{\xi_2}} \\
  \ctruify{\cinl{\xi}} &= \cinl{\ctruify{\xi}} \\
  \ctruify{\cinr{\xi}} &= \cinr{\ctruify{\xi}} \\
  \ctruify{\cpair{\xi_1}{\xi_2}} &= \cpair{\ctruify{\xi_1}}{\ctruify{\xi_2}}
\end{align*}

\judgbox{\cfalsify{\xi_1} = \xi_2}{}
\begin{align*}
  \cfalsify{\ctruth} &= \ctruth \\
  \cfalsify{\cfalsity} &= \cfalsity \\
  \cfalsify{\cunknown} &= \cfalsity \\
  \cfalsify{\cnum{n}} &= \cnum{n} \\
  \cfalsify{\cnotnum{n}} &= \cnotnum{n} \\
  \cfalsify{\cand{\xi_1}{\xi_2}} &= \cand{\cfalsify{\xi_1}}{\cfalsify{\xi_2}} \\
  \cfalsify{\cor{\xi_1}{\xi_2}} &= \cor{\cfalsify{\xi_1}}{\cfalsify{\xi_2}} \\
  \cfalsify{\cinl{\xi}} &= \cinl{\cfalsify{\xi}} \\
  \cfalsify{\cinr{\xi}} &= \cinr{\cfalsify{\xi}} \\
  \cfalsify{\cpair{\xi_1}{\xi_2}} &= \cpair{\cfalsify{\xi_1}}{\cfalsify{\xi_2}}
\end{align*}
  \caption{Truify and Falsify Constraints}
  \label{fig:truify-falsify}
\end{figure}

\figurename~\ref{fig:truify-falsify} defines function \textit{truify} and \textit{falsify},
which substitute unknown constraint with truth constraint $\ctruth$ and falsity
constraint $\cfalsity$ respectively. That mechanism turns to actually closely follow us, as human, as for how to determine exhaustiveness and redundancy when it comes to incomplete match expression.

Consider the examples in \figurename~\ref{fig:exh-hole}, to make it exhaustive, we would naturally replace pattern hole $w$ with a variable pattern or a wild card pattern. Since the Unknown constraint $\cunknown$ is directly emitted from the pattern hole $w$, analogously, we replace it with Truth constraint $\ctruth$.
\begin{theorem}
  $\csatisfyormay{\ctruth}{\xi}$ iff $\csatisfy{\ctruth}{\ctruify{\xi}}$.
\end{theorem}

\begin{lemma}
  $\csatisfyormay{e}{\xi}$ iff $\csatisfyormay{e}{\ctruify{\xi}}$.
\end{lemma}

\begin{lemma}
  Assume $\ctruify{\xi}=\xi$. Then $\csatisfyormay{\ctruth}{\xi}$ iff $\csatisfy{\ctruth}{\xi}$.
\end{lemma}

Consider the example in \figurename~\ref{fig:redundant}, we want to tell if the third rule is redundant with respect to the first and second rules. To maximize the subset of values of list type that matches the pattern $\hehole{w}::xs$, we again replace the pattern hole $w$ with a variable pattern or a wild card pattern and realize that it is redundant. When there are hole(s) in the patterns of preceding rules, it is not obvious what pattern to replace the hole with. Consider the example in \figurename~\ref{fig:may-redundant1}, we want to tell if the third rule is redundant. To accomplish that, we want to minimize the subset of values of list type that matches $x::\hehole{w}$ so that it does not cover all the cases of the last branch. Intuitively, we may fill hole $\hehole{w}$ with $2::\nil$ so that the trailing part of the second pattern is more specific than $xs$ in the third pattern. However, it is not always the case that we can find a more specific (sub)pattern. Consider the example in \figurename~\ref{fig:may-redundant2}, the third pattern only matches one values and thus we cannot find a more specific pattern. Particularly, $2::\nil$ in the third pattern corresponds to the hole pattern $\hehole{w}$ in the second pattern. In this case, we can always find a different pattern to substitute the hole pattern. For example, we can replace $\hehole{w}$ with $3::[]$ and find that the third rule is not redundant.
\begin{theorem}
  $\csatisfy{\xi_r}{\xi_{rs}}$ iff $\csatisfy{\ctruth}{\cor{\cdual{\ctruify{\xi_r}}}{\cfalsify{\xi_{rs}}}}$.
\end{theorem}

\begin{lemma}
  Assume $\isVal{e}$. Then $\csatisfyormay{e}{\xi}$ iff $\csatisfy{e}{\ctruify{\xi}}$.
\end{lemma}

\begin{lemma}
  $\csatisfy{e}{\xi}$ iff $\csatisfy{e}{\cfalsify{\xi}}$.
\end{lemma}

\begin{lemma}
  Assume $\isVal{e}$ and $\ctruify{\xi}=\xi$. Then $\cnotsatisfy{e}{\xi}$ iff $\csatisfy{e}{\cdual{\xi}}$.
\end{lemma}

The reason why we can always find a different pattern is because we have number as the base type of our language. What if we add unit type? (We actually need unit type to represent empty list!) Then we no longer have infinite patterns that is of arbitrary given type. We can simply emit Truth constraint $\ctruth$ when assigning unit type to pattern holes (see Appendix \todo{}).

\figurename~\ref{fig:incon} defines \textit{inconsistent} judgment to further determine $\csatisfy{\ctruth}{\xi}$. Note that at this stage, constraint $\xi$ does not involve any Unknown constraint $\cunknown$. The inconsistent judgment $\cincon{\xi}$ specifies that constraint $\xi$ is inconsistent in the sense that no value of the same type as $\xi$ satisfies $\xi$.

% !TEX root= pattern-paper.tex

\begin{figure}[bp]
\judgbox{\cincon{\Xi}}{}

\begin{mathpar}
\Infer{\CINCTruth}{
  \cincon{\Xi}
}{
  \cincon{\Xi, \ctruth}
}

\Infer{\CINCFalsity}{ }{
  \cincon{\Xi, \cfalsity}
}

\Infer{\CINCNum}{
  n_1 \neq n_2
}{
  \cincon{\Xi, \cnum{n_1}, \cnum{n_2}}
}

\Infer{\CINCNotNum}{ }{
  \cincon{\Xi, \cnum{n}, \cnotnum{n}}
}

\Infer{\CINCAnd}{
  \cincon{\Xi, \xi_1, \xi_2}
}{
  \cincon{\Xi, \cand{\xi_1}{\xi_2}}
}

\Infer{\CINCOr}{
  \cincon{\Xi, \xi_1} \\
  \cincon{\Xi, \xi_2}
}{
  \cincon{\Xi, \cor{\xi_1}{\xi_2}}
}

\Infer{\CINCInj}{ }{
  \cincon{\Xi, \cinl{\xi_1}, \cinr{\xi_2}}
}

\Infer{\CINCInl}{
  \cincon{\setof{\xi' | \cinl{\xi'} \in \Xi},\xi}
}{
  \cincon{\Xi, \cinl{\xi}}
}

\Infer{\CINCInr}{
  \cincon{\setof{\xi' | \cinr{\xi'} \in \Xi},\xi}
}{
  \cincon{\Xi, \cinr{\xi}}
}

\Infer{\CINCPairL}{
    \cincon{\setof{\xi_1' | \cpair{\xi_1'}{\xi_2'} \in \Xi},\xi_1}
}{
    \cincon{\Xi, \cpair{\xi_1}{\xi_2}}
}

\Infer{\CINCPairR}{
    \cincon{\setof{\xi_2' | \cpair{\xi_1'}{\xi_2'} \in \Xi},\xi_2}
}{
    \cincon{\Xi, \cpair{\xi_1}{\xi_2}}
}
\end{mathpar}

  \caption{Inconsistency of Constraints}
  \label{fig:incon}
\end{figure}

\begin{theorem}
  Assume $\ctruify{\xi}=\xi$. It is decidable whether $\cincon{\xi}$.
\end{theorem}

\begin{theorem}
  Assume $\ctruify{\xi}=\xi$. Then $\cincon{\cdual{\xi}}$ iff $\csatisfy{\ctruth}{\xi}$.
\end{theorem}

\section{Related Work}



\begin{itemize}
    \item typed holes
    \begin{itemize}
        \item Hazelnut, Hazelnut Live
        \begin{itemize}
            \item both include only simple sum types, do not consider holes in patterns
        \end{itemize}
        \item Haskell, Agda, neither have support for holes in patterns
    \end{itemize}
    \item specifying pattern matching
    \item Compilng pattern matching
    \begin{itemize}
        \item Compiling a functional language, Cardelli 1984
        \item Compiling pattern matching, Augustsson 1985
        \item in this work we do not focus on compilation but rather
        spec/correctness
        \item probably should say something more about the path toward
        efficiency
    \end{itemize}
    \item Lower Your Guards
    \begin{itemize}
        \item compiles a myriad of pattern forms/features into a guard tree
        \item matching a guard tree may succeed, fail, or diverge
        \item (draw some comparison with guard trees and our constraints)
        \item does not account for indeterminate matches
        \item likely their work could be extended to integrate indeterminate matching
    \end{itemize}
    \item other checkers for handling potentially undecidable coverage
    \begin{itemize}
        \item GADTs Meet Their Match, Karachalias et al 2015
        \item SMT solver for handling guards, Kalvoda and Kerckhove 2019
        \item case trees for dependently typed / refinement type languages
        \item emphasize that the failure/uncertainty of checking with these
        other tools is not the same as indeterminate matching in our work
    \end{itemize}
\end{itemize}

\section{Conclusion}
We can adapt our core calculus to fit in a more expressive setting.
\clearpage

\bibliographystyle{splncs04}
\bibliography{references.bib}

\appendix
\section{Match Constraint Language}
$\arraycolsep=4pt\begin{array}{lll}
\xi & ::= &
  \ctruth ~\vert~
  \cfalsity ~\vert~
  \cunknown ~\vert~
  \cnum{n} ~\vert~
  \cnotnum{n} ~\vert~
  \cand{\xi_1}{\xi_2} ~\vert~
  \cor{\xi_1}{\xi_2} ~\vert~
  \cinl{\xi} ~\vert~
  \cinr{\xi} ~\vert~
  \cpair{\xi_1}{\xi_2}
\end{array}$

\judgboxa{\ctyp{\xi}{\tau}}{$\xi$ constrains final expressions of type $\tau$}
\begin{subequations}\label{rules:CTyp}
\begin{equation}\label{rule:CTTruth}
\inferrule[CTTruth]{ }{
  \ctyp{\ctruth}{\tau}
}
\end{equation}
\begin{equation}\label{rule:CTFalsity}
\inferrule[CTFalsity]{ }{
  \ctyp{\cfalsity}{\tau}
}
\end{equation}
\begin{equation}\label{rule:CTUnknown}
\inferrule[CTUnknown]{ }{
  \ctyp{\cunknown}{\tau}
}
\end{equation}
\begin{equation}\label{rule:CTNum}
\inferrule[CTNum]{ }{
  \ctyp{\cnum{n}}{\tnum}
}
\end{equation}
\begin{equation}\label{rule:CTNotNum}
\inferrule[CTNotNum]{ }{
  \ctyp{\cnotnum{n}}{\tnum}
}
\end{equation}
\begin{equation}\label{rule:CTAnd}
\inferrule[CTAnd]{
  \ctyp{\xi_1}{\tau} \\ \ctyp{\xi_2}{\tau}
}{
  \ctyp{\cand{\xi_1}{\xi_2}}{\tau}
}
\end{equation}
\begin{equation}\label{rule:CTOr}
\inferrule[CTOr]{
  \ctyp{\xi_1}{\tau} \\ \ctyp{\xi_2}{\tau}
}{
  \ctyp{\cor{\xi_1}{\xi_2}}{\tau}
}
\end{equation}
\begin{equation}\label{rule:CTInl}
\inferrule[CTInl]{
  \ctyp{\xi_1}{\tau_1}
}{
  \ctyp{\cinl{\xi_1}}{\tsum{\tau_1}{\tau_2}}
}
\end{equation}
\begin{equation}\label{rule:CTInr}
\inferrule[CTInr]{
  \ctyp{\xi_2}{\tau_2}
}{
  \ctyp{\cinr{\xi_2}}{\tsum{\tau_1}{\tau_2}}
}
\end{equation}
\begin{equation}\label{rule:CTPair}
\inferrule[CTPair]{
  \ctyp{\xi_1}{\tau_1} \\ \ctyp{\xi_2}{\tau_2}
}{
  \ctyp{\cpair{\xi_1}{\xi_2}}{\tprod{\tau_1}{\tau_2}}
}
\end{equation}
\end{subequations}

\judgboxa{\cdual{\xi_1} = \xi_2}{dual of $\xi_1$ is $\xi_2$}
\begin{subequations}\label{defn:dual}
\begin{align}
  \cdual{\ctruth} &= \cfalsity \\
  \cdual{\cfalsity} &= \ctruth \\
  \cdual{\cunknown} &= \cunknown \\
  \cdual{\cnum{n}} &= \cnotnum{n} \\
  \cdual{\cnotnum{n}} &= \cnum{n} \\
  \cdual{\cand{\xi_1}{\xi_2}} &= \cor{\cdual{\xi_1}}{\cdual{\xi_2}} \\
  \cdual{\cor{\xi_1}{\xi_2}} &= \cand{\cdual{\xi_1}}{\cdual{\xi_2}} \\
  \cdual{\cinl{\xi_1}} &= \cor{ \cinl{\cdual{\xi_1}} }{ \cinr{\ctruth} } \\
  \cdual{\cinr{\xi_2}} &= \cor{ \cinr{\cdual{\xi_2}} }{ \cinl{\ctruth} } \\
  \cdual{\cpair{\xi_1}{\xi_2}} &=
  \cor{ \cor{ 
    \cpair{\xi_1}{\cdual{\xi_2}}
  }{
    \cpair{\cdual{\xi_1}}{\xi_2}
  }}{
    \cpair{\cdual{\xi_1}}{\cdual{\xi_2}}
  }
\end{align}
\end{subequations}

\judgboxa{
  \refutable{\xi}
}{$\xi$ is refutable}

\begin{subequations}\label{rules:xi-refutable}
\begin{equation}\label{rule:RXNum}
\inferrule[RXNum]{ }{
  \refutable{\cnum{n}}
}
\end{equation}
\begin{equation}\label{rule:RXNotNum}
\inferrule[RXNotNum]{ }{
  \refutable{\cnotnum{n}}
}
\end{equation}
\begin{equation}\label{rule:RXUnknown}
\inferrule[RXUnknown]{ }{
  \refutable{\cunknown}
}
\end{equation}
\begin{equation}\label{rule:RXInl}
\inferrule[RXInl]{ }{
  \refutable{\cinl{\xi}}
}
\end{equation}
\begin{equation}\label{rule:RXInr}
\inferrule[RXInr]{ }{
  \refutable{\cinr{\xi}}
}
\end{equation}
\begin{equation}\label{rule:RXPairL}
\inferrule[RXPairL]{
  \refutable{\xi_1}
}{
  \refutable{\cpair{\xi_1}{\xi_2}}
}
\end{equation}
\begin{equation}\label{rule:RXPairR}
\inferrule[RXPairR]{
  \refutable{\xi_2}
}{
  \refutable{\cpair{\xi_1}{\xi_2}}
}
\end{equation}
\end{subequations}

\judgboxa{\frefutable{\xi}}{}
\begin{subequations}\label{defn:xi-refutable}
\begin{align}
    \frefutable{\cnum{n}} &= \true \\
    \frefutable{\cnotnum{n}} &= \true \\
    \frefutable{\cunknown} &= \true \\
    \frefutable{\cinl{\xi}} &= \frefutable{\xi} \\
    \frefutable{\cinr{\xi}} &= \frefutable{\xi} \\
    \frefutable{\cpair{\xi_1}{\xi_2}} &= \frefutable{\xi_1} \text{ or } \frefutable{\xi_2} \\
  \text{Otherwise}\quad \frefutable{\xi} &= \false 
\end{align}
\end{subequations}


\judgboxa{\ctruify{\xi_1} = \xi_2}{}
\begin{subequations}\label{defn:truify}
\begin{align}
  \ctruify{\ctruth} &= \ctruth \\
  \ctruify{\cfalsity} &= \cfalsity \\
  \ctruify{\cunknown} &= \ctruth \\
  \ctruify{\cnum{n}} &= \cnum{n} \\
  \ctruify{\cnotnum{n}} &= \cnotnum{n} \\
  \ctruify{\cand{\xi_1}{\xi_2}} &= \cand{\ctruify{\xi_1}}{\ctruify{\xi_2}} \\
  \ctruify{\cor{\xi_1}{\xi_2}} &= \cor{\ctruify{\xi_1}}{\ctruify{\xi_2}} \\
  \ctruify{\cinl{\xi}} &= \cinl{\ctruify{\xi}} \\
  \ctruify{\cinr{\xi}} &= \cinr{\ctruify{\xi}} \\
  \ctruify{\cpair{\xi_1}{\xi_2}} &= \cpair{\ctruify{\xi_1}}{\ctruify{\xi_2}}
\end{align}
\end{subequations}

\judgboxa{\cfalsify{\xi_1} = \xi_2}{}
\begin{subequations}\label{defn:falsify}
\begin{align}
  \cfalsify{\ctruth} &= \ctruth \\
  \cfalsify{\cfalsity} &= \cfalsity \\
  \cfalsify{\cunknown} &= \cfalsity \\
  \cfalsify{\cnum{n}} &= \cnum{n} \\
  \cfalsify{\cnotnum{n}} &= \cnotnum{n} \\
  \cfalsify{\cand{\xi_1}{\xi_2}} &= \cand{\cfalsify{\xi_1}}{\cfalsify{\xi_2}} \\
  \cfalsify{\cor{\xi_1}{\xi_2}} &= \cor{\cfalsify{\xi_1}}{\cfalsify{\xi_2}} \\
  \cfalsify{\cinl{\xi}} &= \cinl{\cfalsify{\xi}} \\
  \cfalsify{\cinr{\xi}} &= \cinr{\cfalsify{\xi}} \\
  \cfalsify{\cpair{\xi_1}{\xi_2}} &= \cpair{\cfalsify{\xi_1}}{\cfalsify{\xi_2}}
\end{align}
\end{subequations}

\judgboxa{\csatisfy{e}{\xi}}{$e$ satisfies $\xi$}
\begin{subequations}\label{rules:Satisfy}
\begin{equation}\label{rule:CSTruth}
\inferrule[CSTruth]{ }{
  \csatisfy{e}{\ctruth}
}
\end{equation}
\begin{equation}\label{rule:CSNum}
\inferrule[CSNum]{ }{
  \csatisfy{\hnum{n}}{\cnum{n}}
}
\end{equation}
\begin{equation}\label{rule:CSNotNum}
\inferrule[CSNotNum]{
  n_1 \neq n_2
}{
  \csatisfy{\hnum{n_1}}{\cnotnum{n_2}}
}
\end{equation}
\begin{equation}\label{rule:CSAnd}
\inferrule[CSAnd]{
  \csatisfy{e}{\xi_1} \\
  \csatisfy{e}{\xi_2}
}{
  \csatisfy{e}{\cand{\xi_1}{\xi_2}}
}
\end{equation}
\begin{equation}\label{rule:CSOr1}
\inferrule[CSOrL]{
  \csatisfy{e}{\xi_1}
}{
  \csatisfy{e}{\cor{\xi_1}{\xi_2}}
}
\end{equation}
\begin{equation}\label{rule:CSOr2}
\inferrule[CSOrR]{
  \csatisfy{e}{\xi_2}
}{
  \csatisfy{e}{\cor{\xi_1}{\xi_2}}
}
\end{equation}
\begin{equation}\label{rule:CSInl}
\inferrule[CSInl]{
  \csatisfy{e_1}{\xi_1}
}{
  \csatisfy{
    \hinl{\tau_2}{e_1}
  }{
    \cinl{\xi_1}
  }
}
\end{equation}
\begin{equation}\label{rule:CSInr}
\inferrule[CSInr]{
  \csatisfy{e_2}{\xi_2}
}{
  \csatisfy{
    \hinr{\tau_1}{e_2}
  }{
    \cinr{\xi_2}
  }
}
\end{equation}
\begin{equation}\label{rule:CSPair}
\inferrule[CSPair]{
  \csatisfy{e_1}{\xi_1} \\
  \csatisfy{e_2}{\xi_2}
}{
\csatisfy{\hpair{e_1}{e_2}}{\cpair{\xi_1}{\xi_2}}
}
\end{equation}
\begin{equation}\label{rule:CSNotValPair}
\inferrule[CSNotValPair]{
  \isntVal{e} \\
  \csatisfy{\hprl{e}}{\xi_1} \\
  \csatisfy{\hprr{e}}{\xi_2}
}{
  \csatisfy{e}{\cpair{\xi_1}{\xi_2}}
}
\end{equation}
\end{subequations}

\judgboxa{\fsatisfy{e}{\xi}}{}
\begin{subequations}\label{defn:satisfy}
\begin{align}
  \fsatisfy{e}{\ctruth} ={}& \true \label{defn:satisfy-truth}\\
  \fsatisfy{\hnum{n_1}}{\cnum{n_2}} ={}& (n_1 = n_2) \label{defn:num-satisfy-num}\\
  \fsatisfy{\hnum{n_1}}{\cnotnum{n_2}} ={}& (n_1 \neq n_2) \label{defn:num-satisfy-notnum}\\
  \fsatisfy{e}{\cand{\xi_1}{\xi_2}} ={}& \fsatisfy{e}{\xi_1} \text{ and } \fsatisfy{e}{\xi_2} \label{defn:satisfy-and}\\
  \fsatisfy{e}{\cor{\xi_1}{\xi_2}} ={}& \fsatisfy{e}{\xi_1} \text{ or } \fsatisfy{e}{\xi_2} \label{defn:satisfy-or}\\
  \fsatisfy{\hinl{\tau_2}{e_1}}{\cinl{\xi_1}} ={}& \fsatisfy{e_1}{\xi_1} \label{defn:inl-satisfy-inl}\\
  \fsatisfy{\hinr{\tau_1}{e_2}}{\cinr{\xi_2}} ={}& \fsatisfy{e_2}{\xi_2} \label{defn:inr-satisfy-inr}\\
  \fsatisfy{\hpair{e_1}{e_2}}{\cpair{\xi_1}{\xi_2}} ={}& \fsatisfy{e_1}{\xi_1} \text{ and } \fsatisfy{e_2}{\xi_2} \label{defn:pair-satisfy-pair}\\
  \fsatisfy{\hehole{u}}{\cpair{\xi_1}{\xi_2}} ={}& \fsatisfy{\hprl{\hehole{u}}}{\xi_1} \text{ and } \fsatisfy{\hprr{\hehole{u}}}{\xi_2}
  \label{defn:ehole-satisfy-pair} \\
  \fsatisfy{\hhole{e}{u}}{\cpair{\xi_1}{\xi_2}} ={}& \fsatisfy{\hprl{\hhole{e}{u}}}{\xi_1} \text{ and } \fsatisfy{\hprr{\hhole{e}{u}}}{\xi_2}
  \label{defn:hole-satisfy-pair} \\
  \fsatisfy{\hap{e_1}{e_2}}{\cpair{\xi_1}{\xi_2}} ={}& \fsatisfy{\hprl{\hap{e_1}{e_2}}}{\xi_1} \notag\\
  &\text{ and } \fsatisfy{\hprr{\hap{e_1}{e_2}}}{\xi_2}
  \label{defn:ap-satisfy-pair} \\
  \fsatisfy{\hmatch{e}{\zrules}}{\cpair{\xi_1}{\xi_2}} ={}& \fsatisfy{\hprl{\hmatch{e}{\zrules}}}{\xi_1} \notag\\
  &\text{ and } \fsatisfy{\hprr{\hmatch{e}{\zrules}}}{\xi_2}
  \label{defn:match-satisfy-pair} \\
  \fsatisfy{\hprl{e}}{\cpair{\xi_1}{\xi_2}} ={}& \fsatisfy{\hprl{\hprl{e}}}{\xi_1} \notag\\
  &\text{ and } \fsatisfy{\hprr{\hprl{e}}}{\xi_2}
  \label{defn:prl-satisfy-pair} \\
  \fsatisfy{\hprr{e}}{\cpair{\xi_1}{\xi_2}} ={}& \fsatisfy{\hprl{\hprr{e}}}{\xi_1} \notag\\
  &\text{ and } \fsatisfy{\hprr{\hprr{e}}}{\xi_2}
  \label{defn:prr-satisfy-pair} \\
  \text{Otherwise}\quad \fsatisfy{e}{\xi} ={}& \false \label{defn:not-satisfy}
\end{align}
\end{subequations}

\judgboxa{\cmaysatisfy{e}{\xi}}{$e$ may satisfy $\xi$}
\begin{subequations}\label{rules:MaySatisfy}
\begin{equation}\label{rule:CMSUnknown}
\inferrule[CMSUnknown]{ }{
  \cmaysatisfy{e}{\cunknown}
}
\end{equation}
\begin{equation}\label{rule:CMSNotVal}
\inferrule[CMSNotVal]{
  \isntVal{e} \\
  \refutable{\xi}
}{
  \cmaysatisfy{e}{\xi}
}
\end{equation}
\begin{equation}\label{rule:CMSAnd1}
\inferrule[CMSAndL]{
  \cmaysatisfy{e}{\xi_1} \\
  \csatisfy{e}{\xi_2}
}{
  \cmaysatisfy{e}{\cand{\xi_1}{\xi_2}}
}
\end{equation}
\begin{equation}\label{rule:CMSAnd2}
\inferrule[CMSAndR]{
  \csatisfy{e}{\xi_1} \\
  \cmaysatisfy{e}{\xi_2}
}{
  \cmaysatisfy{e}{\cand{\xi_1}{\xi_2}}
}
\end{equation}
\begin{equation}\label{rule:CMSAnd3}
\inferrule[CMSAnd]{
  \cmaysatisfy{e}{\xi_1} \\
  \cmaysatisfy{e}{\xi_2}
}{
  \cmaysatisfy{e}{\cand{\xi_1}{\xi_2}}
}
\end{equation}
\begin{equation}\label{rule:CMSOr1}
\inferrule[CMSOrL]{
  \cmaysatisfy{e}{\xi_1} \\
  \cnotsatisfy{e}{\xi_2}
}{
  \cmaysatisfy{e}{\cor{\xi_1}{\xi_2}}
}
\end{equation}
\begin{equation}\label{rule:CMSOr2}
\inferrule[CMSOrR]{
  \cnotsatisfy{e}{\xi_1} \\
  \cmaysatisfy{e}{\xi_2}
}{
  \cmaysatisfy{e}{\cor{\xi_1}{\xi_2}}
}
\end{equation}
\begin{equation}\label{rule:CMSInl}
\inferrule[CMSInl]{
  \cmaysatisfy{e_1}{\xi_1}
}{
  \cmaysatisfy{
    \hinl{\tau_2}{e_1}
  }{
    \cinl{\xi_1}
  }
}
\end{equation}
\begin{equation}\label{rule:CMSInr}
\inferrule[CMSInr]{
  \cmaysatisfy{e_2}{\xi_2}
}{
  \cmaysatisfy{
    \hinr{\tau_1}{e_2}
  }{
    \cinr{\xi_2}
  }
}
\end{equation}
\begin{equation}\label{rule:CMSPair1}
\inferrule[CMSPairL]{
  \cmaysatisfy{e_1}{\xi_1} \\
  \csatisfy{e_2}{\xi_2}
}{
  \cmaysatisfy{\hpair{e_1}{e_2}}{\cpair{\xi_1}{\xi_2}}
}
\end{equation}
\begin{equation}\label{rule:CMSPair2}
\inferrule[CMSPairR]{
  \csatisfy{e_1}{\xi_1} \\
  \cmaysatisfy{e_2}{\xi_2}
}{
  \cmaysatisfy{\hpair{e_1}{e_2}}{\cpair{\xi_1}{\xi_2}}
}
\end{equation}
\begin{equation}\label{rule:CMSPair3}
\inferrule[CMSPair]{
  \cmaysatisfy{e_1}{\xi_1} \\
  \cmaysatisfy{e_2}{\xi_2}
}{
  \cmaysatisfy{\hpair{e_1}{e_2}}{\cpair{\xi_1}{\xi_2}}
}
\end{equation}
\end{subequations}

\judgboxa{\csatisfyormay{e}{\xi}}{$e$ satisfies or may satisfy $\xi$}
\begin{subequations}\label{rules:satormay}
\begin{equation}\label{rule:CSMSMay}
\inferrule[CSMSMay]{
  \cmaysatisfy{e}{\xi}
}{
  \csatisfyormay{e}{\xi}
}
\end{equation}
\begin{equation}\label{rule:CSMSSat}
\inferrule[CSMSSat]{
  \csatisfy{e}{\xi}
}{
  \csatisfyormay{e}{\xi}
}
\end{equation}
\end{subequations}

\begin{lemma}
  \label{lem:no-e-satisfy-falsity}
  $\cnotsatisfy{e}{\cfalsity}$
\end{lemma}
\begin{proof}
  By rule induction over Rules (\ref{rules:Satisfy}), we notice that $\csatisfy{e}{\cfalsity}$ is in syntactic contradiction with all rules, hence not derivable.
\end{proof}

\begin{lemma}
  \label{lem:no-e-may-satisfy-falsity}
  $\cnotmaysatisfy{e}{\cfalsity}$
\end{lemma}
\begin{proof}
  Assume $\cmaysatisfy{e}{\cfalsity}$.
  By rule induction over Rules (\ref{rules:MaySatisfy}) on $\cmaysatisfy{e}{\cfalsity}$, only one case applies.
  \begin{byCases}
  \item[\text{(\ref{rule:CMSNotVal})}]
    \begin{pfsteps*}
    \item $\refutable{\cfalsity}$ \BY{assumption} \pflabel{refutable}
    \end{pfsteps*}
    By rule induction over Rules (\ref{rules:xi-refutable}) on \pfref{refutable}, no case applies due to syntactic contradiction.
  \end{byCases}
  Therefore, $\cmaysatisfy{e}{\cfalsity}$ is not derivable.
  \resetpfcounter
\end{proof}

\begin{lemma}
  \label{lem:no-e-may-satisfy-truth}
  $\cnotmaysatisfy{e}{\ctruth}$
\end{lemma}
\begin{proof}
  Assume $\cmaysatisfy{e}{\ctruth}$.
  By rule induction over Rules (\ref{rules:MaySatisfy}) on $\cmaysatisfy{e}{\ctruth}$, only one case applies.
  \begin{byCases}
  \item[\text{(\ref{rule:CMSNotVal})}]
    \begin{pfsteps*}
    \item $\refutable{\ctruth}$ \BY{assumption} \pflabel{refutable}
    \end{pfsteps*}
    By rule induction over Rules (\ref{rules:xi-refutable}) on \pfref{refutable}, no case applies due to syntactic contradiction.
  \end{byCases}
  Therefore, $\cmaysatisfy{e}{\ctruth}$ is not derivable.
  \resetpfcounter
\end{proof}

\begin{lemma}
  \label{lem:no-e-satisfy-unknown}
  $\cnotsatisfy{e}{\cunknown}$
\end{lemma}
\begin{proof}
  By rule induction over Rules (\ref{rules:Satisfy}), we notice that $\csatisfy{e}{\cunknown}$ is in syntactic contradiction with all the cases, hence not derivable.
\end{proof}

\begin{lemma}
  \label{lem:relax-nn-satisfy}
  $\csatisfyormay{e}{\xi}$ iff $\csatisfyormay{e}{\cor{\xi}{\cfalsity}}$
\end{lemma}
\begin{proof}
  We prove sufficiency and necessity separately.
  \begin{enumerate}
    \item Sufficiency:
      \begin{pfsteps*}
      \item $\csatisfyormay{e}{\xi}$ \BY{assumption} \pflabel{nnsatisfyxi}
      \end{pfsteps*}
      By rule induction over Rules (\ref{rules:satormay}) on \pfref{nnsatisfyxi}.
      \begin{byCases}

      \savelocalsteps{lem:relax-nn-satisfy-suff-1}
      \item[\text{(\ref{rule:CSMSMay})}]
        \begin{pfsteps*}
        \item $\cmaysatisfy{e}{\xi}$ \BY{assumption} \pflabel{maysatisfyxi}
        \item $\cmaysatisfy{e}{\cor{\xi}{\cfalsity}}$ \BY{Rule (\ref{rule:CMSOr1}) on \pfref{maysatisfyxi} and Lemma \ref{lem:no-e-satisfy-falsity}} \pflabel{maysatisfyxi+bot}
        \item $\csatisfyormay{e}{\cor{\xi}{\cfalsity}}$ \BY{Rule (\ref{rule:CSMSMay}) on \pfref{maysatisfyxi+bot}}
        \end{pfsteps*}

      \restorelocalsteps{lem:relax-nn-satisfy-suff-1}
      \item[\text{(\ref{rule:CSMSSat})}]
        \begin{pfsteps*}
        \item $\csatisfy{e}{\xi}$ \BY{assumption} \pflabel{satisfyxi}
        \item $\csatisfy{e}{\cor{\xi}{\cfalsity}} \BY{Rule (\ref{rule:CSOr1}) on \pfref{satisfyxi}} \pflabel{satisfyxi+bot}
        \item $\csatisfyormay{e}{\cor{\xi}{\cfalsity}} \BY{Rule (\ref{rule:CSMSSat}) on \pfref{satisfyxi+bot}}
        \end{pfsteps*}

      \end{byCases}

    \resetpfcounter

    \item Necessity:
      \begin{pfsteps*}
      \item $\csatisfyormay{e}{\cor{\xi}{\cfalsity}}$ \BY{assumption} \pflabel{nnsatisfyxi+bot}
      \end{pfsteps*}
      By rule induction over Rules (\ref{rules:satormay}) on \pfref{nnsatisfyxi+bot}.
      \begin{byCases}

      \savelocalsteps{lem:relax-nn-satisfy-necs-1}
      \item[\text{(\ref{rule:CSMSMay})}]
        \begin{pfsteps*}
        \item $\cmaysatisfy{e}{\cor{\xi}{\cfalsity}}$ \BY{assumption} \pflabel{maysatisfyxi+bot}
        \end{pfsteps*}
        By rule induction over Rules (\ref{rules:MaySatisfy}) on \pfref{maysatisfyxi+bot}, only two of them apply.
        \begin{byCases}

        \savelocalsteps{lem:relax-nn-satisfy-necs-2}
        \item[\text{(\ref{rule:CMSOr1})}]
          \begin{pfsteps*}
          \item $\cmaysatisfy{e}{\xi}$ \BY{assumption} \pflabel{maysatisfyxi}
          \item $\csatisfyormay{e}{\xi}$ \BY{Rule (\ref{rule:CSMSMay}) on \pfref{maysatisfyxi}}
          \end{pfsteps*}

        \restorelocalsteps{lem:relax-nn-satisfy-necs-2}
        \item[\text{(\ref{rule:CMSOr2})}]
          \begin{pfsteps*}
          \item $\cmaysatisfy{e}{\cfalsity}$ \BY{assumption} \pflabel{maysatisfybot}
          \item $\cnotmaysatisfy{e}{\cfalsity}$ \BY{Lemma \ref{lem:no-e-may-satisfy-falsity}} \pflabel{notmaysatisfybot}
          \end{pfsteps*}
          \pfref{maysatisfybot} contradicts \pfref{notmaysatisfybot}.

        \end{byCases}

      \restorelocalsteps{lem:relax-nn-satisfy-necs-1}
      \item[\text{(\ref{rule:CSMSSat})}]
        \begin{pfsteps*}
        \item $\csatisfy{e}{\cor{\xi}{\cfalsity}}$ \BY{assumption} \pflabel{satisfyxi+bot}
        \end{pfsteps*}
        By rule induction over Rules (\ref{rules:Satisfy}) on \pfref{satisfyxi+bot}, only two of them apply.
        \begin{byCases}

        \savelocalsteps{lem:relax-nn-satisfy-necs-2}
        \item[\text{(\ref{rule:CSOr1})}]
          \begin{pfsteps*}
          \item $\csatisfy{e}{\xi}$ \BY{assumption} \pflabel{satisfyxi}
          \item $\csatisfyormay{e}{\xi}$ \BY{Rule (\ref{rule:CSMSSat}) on \pfref{satisfyxi}}
          \end{pfsteps*}

        \restorelocalsteps{lem:relax-nn-satisfy-necs-2}
        \item[\text{(\ref{rule:CSOr2})}]
          \begin{pfsteps*}
          \item $\csatisfy{e}{\cfalsity}$ \BY{assumption} \pflabel{satisfybot}
          \item $\cnotsatisfy{e}{\cfalsity}$ \BY{Lemma \ref{lem:no-e-satisfy-falsity}} \pflabel{notsatisfybot}
          \end{pfsteps*}
          \pfref{satisfybot} contradicts \pfref{notsatisfybot}.
        \end{byCases}

      \resetpfcounter
      \end{byCases}
  \end{enumerate}
\end{proof}

\begin{corollary}
  \label{corol:relax-nn-entail}
  $\csatisfyormay{\ctruth}{\xi}$ iff $\csatisfyormay{\ctruth}{\cor{\xi}{\cfalsity}}$
\end{corollary}
\begin{proof}
  By Definition \ref{defn:nn-entailment} and Lemma \ref{lem:relax-nn-satisfy}.
\end{proof}

\begin{lemma}
  \label{lem:relax-not-redundant}
  Suppose that $\ctyp{\xi_1}{\tau}$ and $\ctyp{\xi_2}{\tau}$. Then $\cnotsatisfy{\xi_1}{\xi_2}$ iff $\cnotsatisfy{\xi_1}{\cor{\xi_2}{\cfalsity}}$
\end{lemma}
\begin{proof}
  \begin{pfsteps*}
  \item $\ctyp{\xi_1}{\tau}$ \BY{assumption} \pflabel{1CTyp}
  \item $\ctyp{\xi_2}{\tau}$ \BY{assumption} \pflabel{2CTyp}
  \item $\ctyp{\cfalsity}{\tau}$ \BY{Rule (\ref{rule:CTFalsity})} \pflabel{fCTyp}
  \item $\ctyp{\cor{\xi_2}{\cfalsity}}{\tau}$ \BY{Rule (\ref{rule:CTOr}) on \pfref{2CTyp} and \pfref{fCTyp}} \pflabel{2+botCTyp}
  \end{pfsteps*}
  Then we prove sufficiency and necessity separately.
\begin{enumerate}
  
\savelocalsteps{lem:relax-not-redundant-0}
\item Sufficiency:
  \begin{pfsteps*}
  \item $\cnotsatisfy{\xi_1}{\xi_2}$ \BY{assumption} \pflabel{1notsatisfy2}
  \end{pfsteps*}
  To prove $\cnotsatisfy{\xi_1}{\cor{\xi_2}{\cfalsity}}$, assume $\csatisfy{\xi_1}{\cor{\xi_2}{\cfalsity}}$. 
  \begin{pfsteps*}
  \item $\csatisfy{\xi_1}{\cor{\xi_2}{\cfalsity}}$ \BY{assumption} \pflabel{1=>2+bot}
  \end{pfsteps*}
  For all $e$ such that $\hexptyp{\cdot}{\Delta}{e}{\tau}$ and $\isFinal{e}$ we have $\csatisfyormay{e}{\xi_1}$ implies
  \begin{pfsteps*}
  \item $\csatisfy{e}{\cor{\xi_2}{\cfalsity}}$ \BY{Definition \ref{defn:const-entailment} on \pfref{1CTyp} and \pfref{2+botCTyp} and \pfref{1=>2+bot}} \pflabel{satisfy2+bot}
  \end{pfsteps*}
  By rule induction over Rules (\ref{rules:Satisfy}) on \pfref{satisfy2+bot}.
  \begin{byCases}

  \savelocalsteps{lem:relax-not-redundant-1}
  \item[\text{(\ref{rule:CSOr1})}]
    \begin{pfsteps*}
    \item $\csatisfy{e}{\xi_2}$ \BY{assumption} \pflabel{satisfy2}
    \item $\csatisfy{\xi_1}{\xi_2}$ \BY{Definition \ref{defn:const-entailment} on \pfref{satisfy2}} \pflabel{1satisfy2}
    \end{pfsteps*}
    \pfref{1notsatisfy2} contradicts \pfref{1satisfy2}.


  \restorelocalsteps{lem:relax-not-redundant-1}
  \item[\text{(\ref{rule:CSOr2})}]
    \begin{pfsteps*}
    \item $\csatisfy{e}{\cfalsity}$ \BY{assumption} \pflabel{satisfybot}
    \item $\cnotsatisfy{e}{\cfalsity}$ \BY{Lemma \ref{lem:no-e-satisfy-falsity}} \pflabel{notsatisfybot}
    \end{pfsteps*}
    \pfref{satisfybot} contradicts \pfref{notsatisfybot}.
  \end{byCases}
  The conclusion holds as follows:
  \begin{enumerate}
    \item $\cnotsatisfy{\xi_1}{\cor{\xi_2}{\cfalsity}}$
  \end{enumerate}

\restorelocalsteps{lem:relax-not-redundant-0}
\item Necessity:
  \begin{pfsteps*}
  \item $\cnotsatisfy{\xi_1}{\cor{\xi_2}{\cfalsity}}$ \BY{assumption} \pflabel{1notentail2+bot}
  \end{pfsteps*}
  To prove $\cnotsatisfy{\xi_1}{\xi_2}$, assume $\csatisfy{\xi_1}{\xi_2}$.
  \begin{pfsteps*}
  \item $\csatisfy{\xi_1}{\xi_2}$ \BY{assumption} \pflabel{1entail2}
  \end{pfsteps*}
  For all $e$ such that $\hexptyp{\cdot}{\Delta}{e}{\tau}$ and $\isFinal{e}$ we have $\csatisfyormay{e}{\xi_1}$ implies
  \begin{pfsteps*}
  \item $\csatisfy{e}{\xi_2}$ \BY{Definition \ref{defn:const-entailment} on \pfref{1CTyp} and \pfref{2CTyp} and \pfref{1entail2}} \pflabel{esatisfy2}
  \item $\csatisfy{e}{\cor{\xi_2}{\cfalsity}}$ \BY{Rule (\ref{rule:CSOr1}) on \pfref{esatisfy2}} \pflabel{esatisfy2+bot}
  \item $\csatisfy{\xi_1}{\cor{\xi_2}{\cfalsity}}$ \BY{Definition \ref{defn:const-entailment} on \pfref{esatisfy2+bot}} \pflabel{1|=2vbot}
  \end{pfsteps*}
  \pfref{1|=2vbot} contradicts \pfref{1notentail2+bot}.

  The conclusion holds as follows:
  \begin{enumerate}
    \item $\cnotsatisfy{\xi_1}{\xi_2}$
  \end{enumerate}
\end{enumerate}
\resetpfcounter
\end{proof}

\begin{lemma}
  \label{lem:or-nn-satisfy}
  If $\cnotsatisfyormay{e}{\xi_1}$ and $\cnotsatisfyormay{e}{\xi_2}$ then $\cnotsatisfyormay{e}{\cor{\xi_1}{\xi_2}}$
\end{lemma}
\begin{proof}
  Assume, for the sake of contradiction, that $\csatisfyormay{e}{\cor{\xi_1}{\xi_2}}$.
\begin{pfsteps*}
\item $\csatisfyormay{e}{\cor{\xi_1}{\xi_2}}$ \BY{assumption} \pflabel{satormay1+2}
\item $\cnotsatisfyormay{e}{\xi_1}$ \BY{assumption} \pflabel{notsatormay1}
\item $\cnotsatisfyormay{e}{\xi_2}$ \BY{assumption} \pflabel{notsatormay2}
\end{pfsteps*}
By rule induction over Rules (\ref{rules:satormay}) on \pfref{satormay1+2}.
\begin{byCases}

\savelocalsteps{lem:or-nn-satisfy-1}
\item[\text{(\ref{rule:CSMSSat})}]
  \begin{pfsteps*}
  \item $\csatisfy{e}{\cor{\xi_1}{\xi_2}}$ \BY{assumption} \pflabel{satisfy1+2}
  \end{pfsteps*}
  By rule induction over Rules (\ref{rules:Satisfy}) on \pfref{satisfy1+2} and only two of them apply.
  \begin{byCases}

  \savelocalsteps{lem:or-nn-satisfy-2}
  \item[\text{(\ref{rule:CSOr1})}]
    \begin{pfsteps*}
    \item $\csatisfy{e}{\xi_1}$ \BY{assumption} \pflabel{satisfy1}
    \item $\csatisfyormay{e}{\xi_1}$ \BY{Rule (\ref{rule:CSMSSat}) on \pfref{satisfy1}} \pflabel{satormay1}
    \end{pfsteps*}
    \pfref{satormay1} contradicts \pfref{notsatormay1}.

  \restorelocalsteps{lem:or-nn-satisfy-2}
  \item[\text{(\ref{rule:CSOr2})}]
    \begin{pfsteps*}
    \item $\csatisfy{e}{\xi_2}$ \BY{assumption} \pflabel{satisfy2}
    \item $\csatisfyormay{e}{\xi_2}$ \BY{Rule (\ref{rule:CSMSSat}) on \pfref{satisfy2}} \pflabel{satormay2}
    \end{pfsteps*}
    \pfref{satormay2} contradicts \pfref{notsatormay2}.

  \end{byCases}

\restorelocalsteps{lem:or-nn-satisfy-1}
\item[\text{(\ref{rule:CSMSMay})}]
  \begin{pfsteps*}
  \item $\cmaysatisfy{e}{\cor{\xi_1}{\xi_2}}$ \BY{assumption} \pflabel{maysatisfy1+2}
  \end{pfsteps*}
  By rule induction over Rules (\ref{rules:MaySatisfy}) on \pfref{maysatisfy1+2} and only two of them apply.
  \begin{byCases}

  \savelocalsteps{lem:or-nn-satisfy-2}
  \item[\text{(\ref{rule:CMSOr1})}]
    \begin{pfsteps*}
    \item $\cmaysatisfy{e}{\xi_1}$ \BY{assumption} \pflabel{maysatisfy1}
    \item $\csatisfyormay{e}{\xi_1}$ \BY{Rule (\ref{rule:CSMSMay}) on \pfref{maysatisfy1}} \pflabel{satormay1'}
    \end{pfsteps*}
    \pfref{satormay1'} contradicts \pfref{notsatormay1}.

  \restorelocalsteps{lem:or-nn-satisfy-2}
  \item[\text{(\ref{rule:CMSOr2})}]
    \begin{pfsteps*}
    \item $\cmaysatisfy{e}{\xi_2}$ \BY{assumption} \pflabel{maysatisfy2}
    \item $\csatisfyormay{e}{\xi_2}$ \BY{Rule (\ref{rule:CSMSMay}) on \pfref{maysatisfy2}} \pflabel{satormay2'}
    \end{pfsteps*}
    \pfref{satormay2'} contradicts \pfref{notsatormay2}.

  \end{byCases}
\end{byCases}
The conclusion holds as follows:
\begin{enumerate}
  \item $\cnotsatisfyormay{e}{\cor{\xi_1}{\xi_2}}$
\end{enumerate}
\resetpfcounter
\end{proof}

\begin{lemma}
  \label{lem:satisfy-substraction}
  If $\csatisfyormay{e}{\cor{\xi_1}{\xi_2}}$ and $\cnotsatisfyormay{e}{\xi_1}$ then $\csatisfyormay{e}{\xi_2}$
\end{lemma}
\begin{proof}
  \begin{pfsteps*}
  \item $\csatisfyormay{e}{\cor{\xi_1}{\xi_2}}$ \BY{assumption} \pflabel{satormay1+2}
  \item $\cnotsatisfyormay{e}{\xi_1}$ \BY{assumption} \pflabel{notsatormay1}
  \end{pfsteps*}
  By rule induction over Rules (\ref{rules:satormay}) on \pfref{satormay1+2}.
  \begin{byCases}

  \savelocalsteps{lem:satisfy-substraction-1}
  \item[\text{(\ref{rule:CSMSSat})}]
    \begin{pfsteps*}
    \item $\csatisfy{e}{\cor{\xi_1}{\xi_2}}$ \BY{assumption} \pflabel{satisfy1+2}
    \end{pfsteps*}
    By rule induction over Rules (\ref{rules:Satisfy}) on \pfref{satisfy1+2} and only two of them apply.
    \begin{byCases}

    \savelocalsteps{lem:satisfy-substraction-2}
    \item[\text{(\ref{rule:CSOr1})}]
      \begin{pfsteps*}
      \item $\csatisfy{e}{\xi_1}$ \BY{assumption} \pflabel{satisfy1}
      \item $\csatisfyormay{e}{\xi_1}$ \BY{Rule (\ref{rule:CSMSSat}) on \pfref{satisfy1}} \pflabel{[sat]satormay1}
      \end{pfsteps*}
      \pfref{[sat]satormay1} contradicts \pfref{notsatormay1}.

    \restorelocalsteps{lem:satisfy-substraction-2}
    \item[\text{(\ref{rule:CSOr2})}]
      \begin{pfsteps*}
      \item $\csatisfy{e}{\xi_2}$ \BY{assumption} \pflabel{satisfy2}
      \item $\csatisfyormay{e}{\xi_2}$ \BY{Rule (\ref{rule:CSMSSat}) on \pfref{satisfy2}}
      \end{pfsteps*}
    \end{byCases}

  \restorelocalsteps{lem:satisfy-substraction-1}
  \item[\text{(\ref{rule:CSMSMay})}]
    \begin{pfsteps*}
    \item $\cmaysatisfy{e}{\cor{\xi_1}{\xi_2}}$ \BY{assumption} \pflabel{maysatisfy1+2}
    \end{pfsteps*}
    By rule induction over Rules (\ref{rules:MaySatisfy}) on \pfref{maysatisfy1+2} and only two of them apply.
    \begin{byCases}

    \savelocalsteps{lem:satisfy-substraction-2}
    \item[\text{(\ref{rule:CMSOr1})}]
      \begin{pfsteps*}
      \item $\cmaysatisfy{e}{\xi_1}$ \BY{assumption} \pflabel{maysatisfy1}
      \item $\csatisfyormay{e}{\xi_1}$ \BY{Rule (\ref{rule:CSMSMay}) on \pfref{maysatisfy1}} \pflabel{[may]satormay1}
      \end{pfsteps*}
      \pfref{[may]satormay1} contradicts \pfref{notsatormay1}.

    \restorelocalsteps{lem:satisfy-substraction-2}
    \item[\text{(\ref{rule:CMSOr2})}]
      \begin{pfsteps*}
      \item $\cmaysatisfy{e}{\xi_2}$ \BY{assumption} \pflabel{maysatisfy2}
      \item $\csatisfyormay{e}{\xi_2}$ \BY{Rule (\ref{rule:CSMSMay}) on \pfref{maysatisfy2}}
      \end{pfsteps*}
    \end{byCases}
    \resetpfcounter
  \end{byCases}
\end{proof}

\begin{lemma}
  \label{lem:satormay-and}
  If $\csatisfyormay{e}{\xi_1}$ and $\csatisfyormay{e}{\xi_2}$ then $\csatisfyormay{e}{\cand{\xi_1}{\xi_2}}$
\end{lemma}

\begin{lemma}
  \label{lem:satormay-or}
  If $\csatisfyormay{e}{\xi_1}$ then $\csatisfyormay{e}{\cor{\xi_1}{\xi_2}}$ and
$\csatisfyormay{e}{\cor{\xi_2}{\xi_1}}$
\end{lemma}
\begin{proof}
\begin{pfsteps*}
\item $\csatisfyormay{e}{\xi_1}$ \BY{assumption} \pflabel{satormay1},
\end{pfsteps*}
By rule induction over Rules (\ref{rules:satormay}) on \pfref{satormay1},
\savelocalsteps{lem:satormay-or-1}
\begin{byCases}

\item[\text{(\ref{rule:CSMSSat})}]
  \begin{pfsteps*}
  \item $\csatisfy{e}{\xi_1}$ \BY{assumption} \pflabel{[sat]satisfy1}
  \item $\csatisfy{e}{\cor{\xi_1}{\xi_2}}$ \BY{Rule (\ref{rule:CSOr1}) on \pfref{[sat]satisfy1}} \pflabel{[sat]satisfyOr12}
  \item $\csatisfy{e}{\cor{\xi_2}{\xi_1}}$ \BY{Rule (\ref{rule:CSOr2}) on \pfref{[sat]satisfy1}} \pflabel{[sat]satisfyOr21}
  \item $\csatisfyormay{e}{\cor{\xi_1}{\xi_2}}$ \BY{Rule (\ref{rule:CSMSSat}) on \pfref{[sat]satisfyOr12}}
  \item $\csatisfyormay{e}{\cor{\xi_2}{\xi_1}}$ \BY{Rule (\ref{rule:CSMSSat}) on \pfref{[sat]satisfyOr21}}
  \end{pfsteps*}

\restorelocalsteps{lem:satormay-or-1}

\item[\text{(\ref{rule:CSMSMay})}]
  \begin{pfsteps*}
  \item $\cmaysatisfy{e}{\xi_1}$ \BY{assumption} \pflabel{maysat1}
  \end{pfsteps*}
  By case analysis on the result of $\fsatisfy{e}{\xi_2}$.
  \begin{byCases}

  \savelocalsteps{lem:satormay-or-2}
  \item[\true]
    \begin{pfsteps*}
    \item $\fsatisfy{e}{\xi_2} = \true$ \BY{assumption} \pflabel{fsatisfy2}
    \item $\csatisfy{e}{\xi_2}$ \BY{Lemma \ref{lem:sound-complete-satisfy} on \pfref{fsatisfy2}} \pflabel{satisfy2}
    \item $\csatisfy{e}{\cor{\xi_1}{\xi_2}}$ \BY{Rule (\ref{rule:CSOr2}) on \pfref{satisfy2}} \pflabel{satisfyOr12}
    \item $\csatisfy{e}{\cor{\xi_2}{\xi_1}}$ \BY{Rule (\ref{rule:CSOr1}) on \pfref{satisfy2}} \pflabel{satisfyOr21}
    \item $\csatisfyormay{e}{\cor{\xi_1}{\xi_2}}$ \BY{Rule (\ref{rule:CSMSSat}) on \pfref{satisfyOr12}}
    \item $\csatisfyormay{e}{\cor{\xi_2}{\xi_1}}$ \BY{Rule (\ref{rule:CSMSSat}) on \pfref{satisfyOr21}}
    \end{pfsteps*}

  \restorelocalsteps{lem:satormay-or-2}
  \item[\false]
    \begin{pfsteps*}
    \item $\fsatisfy{e}{\xi_2} = \false$ \BY{assumption} \pflabel{fnotsatisfy2}
    \item $\cnotsatisfy{e}{\xi_2}$ \BY{Lemma \ref{lem:sound-complete-satisfy} on \pfref{fnotsatisfy2}} \pflabel{notSatisfy2}
    \item $\cmaysatisfy{e}{\cor{\xi_1}{\xi_2}}$ \BY{Rule (\ref{rule:CMSOr1}) on \pfref{maysat1} and \pfref{notSatisfy2}} \pflabel{maysatOr}
    \item $\csatisfyormay{e}{\cor{\ctruify{\xi_1}}{\ctruify{\xi_2}}}$ \BY{Rule (\ref{rule:CSMSMay}) on \pfref{maysatOr}}
    \end{pfsteps*}
  \end{byCases}

\end{byCases}
\resetpfcounter
\end{proof}

\begin{lemma}
  \label{lem:satormay-inl}
  If $\csatisfyormay{e_1}{\xi_1}$ then $\csatisfyormay{\hinl{\tau_2}{e_1}}{\cinl{\xi_1}}$
\end{lemma}
\begin{proof}
  \begin{pfsteps*}
  \item $\csatisfyormay{e_1}{\xi_1}$ \BY{assumption} \pflabel{satormay1}
  \end{pfsteps*}
  By rule induction over Rules (\ref{rules:satormay}) on \pfref{satormay1}.
  \begin{byCases}

  \savelocalsteps{lem:satormay-inl-1}
  \item[\text{(\ref{rule:CSMSSat})}]
    \begin{pfsteps*}
    \item $\csatisfy{e_1}{\xi_1}$ \BY{assumption} \pflabel{satisfy1}
    \item $\csatisfy{\hinl{\tau_2}{e_1}}{\cinl{\xi_1}}$ \BY{Rule (\ref{rule:CSInl}) on \pfref{satisfy1}} \pflabel{satisfyInl}
    \item $\csatisfyormay{\hinl{\tau_2}{e_1}}{\cinl{\xi_1}}$ \BY{Rule (\ref{rule:CSMSSat}) on \pfref{satisfyInl}}
    \end{pfsteps*}

  \restorelocalsteps{lem:satormay-inl-1}
  \item[\text{(\ref{rule:CSMSMay})}]
    \begin{pfsteps*}
    \item $\cmaysatisfy{e_1}{\xi_1}$ \BY{assumption} \pflabel{maysat1}
    \item $\cmaysatisfy{\hinl{\tau_2}{e_1}}{\cinl{\xi_1}}$ \BY{Rule (\ref{rule:CMSInl}) on \pfref{maysat1}} \pflabel{maysatInl}
    \item $\csatisfyormay{\hinl{\tau_2}{e_1}}{\cinl{\xi_1}}$ \BY{Rule (\ref{rule:CSMSMay}) on \pfref{maysatInl}}
    \end{pfsteps*}
  \end{byCases}
  \resetpfcounter
\end{proof}

\begin{lemma}
  \label{lem:satormay-inr}
  If $\csatisfyormay{e_2}{\xi_2}$ then $\csatisfyormay{\hinr{\tau_1}{e_2}}{\cinr{\xi_2}}$
\end{lemma}
\begin{proof}
  \begin{pfsteps*}
  \item $\csatisfyormay{e_2}{\xi_2}$ \BY{assumption} \pflabel{satormay2}
  \end{pfsteps*}
  By rule induction over Rules (\ref{rules:satormay}) on \pfref{satormay2}.
  \begin{byCases}

  \savelocalsteps{lem:satormay-inr-1}
  \item[\text{(\ref{rule:CSMSSat})}]
    \begin{pfsteps*}
    \item $\csatisfy{e_2}{\xi_2}$ \BY{assumption} \pflabel{satisfy2}
    \item $\csatisfy{\hinr{\tau_1}{e_2}}{\cinr{\xi_2}}$ \BY{Rule (\ref{rule:CSInr}) on \pfref{satisfy2}} \pflabel{satisfyInr}
    \item $\csatisfyormay{\hinr{\tau_1}{e_2}}{\cinr{\xi_2}}$ \BY{Rule (\ref{rule:CSMSSat}) on \pfref{satisfyInr}}
    \end{pfsteps*}

  \restorelocalsteps{lem:satormay-inr-1}
  \item[\text{(\ref{rule:CSMSMay})}]
    \begin{pfsteps*}
    \item $\cmaysatisfy{e_2}{\xi_2}$ \BY{assumption} \pflabel{maysat2}
    \item $\cmaysatisfy{\hinl{\tau_1}{e_2}}{\cinr{\xi_2}}$ \BY{Rule (\ref{rule:CMSInr}) on \pfref{maysat2}} \pflabel{maysatInr}
    \item $\csatisfyormay{\hinl{\tau_1}{e_2}}{\cinr{\xi_2}}$ \BY{Rule (\ref{rule:CSMSMay}) on \pfref{maysatInr}}
    \end{pfsteps*}
  \end{byCases}
  \resetpfcounter
\end{proof}

\begin{lemma}
  \label{lem:satormay-pair}
  If $\csatisfyormay{e_1}{\xi_1}$ and $\csatisfyormay{e_2}{\xi_2}$ then $\csatisfyormay{\hpair{e_1}{e_2}}{\cpair{\xi_1}{\xi_2}}$
\end{lemma}

\begin{lemma}[Soundness and Completeness of Refutable Constraints]
  \label{lem:sound-complete-xi-refutable}
  $\refutable{\xi}$ iff $\frefutable{\xi} = \true$.
\end{lemma}

\begin{lemma}
\label{lem:notval-refutable}
If $\isntVal{e}$ and $\refutable{\xi}$ then either $\refutable{\ctruify{\xi}}$ or $\csatisfy{e}{\ctruify{\xi}}$.
\end{lemma}
\begin{proof}
  By structural induction on $\xi$.
\end{proof}

\begin{lemma}
\label{lem:no-and-refutable}
There does not exist such a constraint $\cand{\xi_1}{\xi_2}$ such that $\refutable{\cand{\xi_1}{\xi_2}}$.
\end{lemma}
\begin{proof}
By rule induction over Rules (\ref{rules:xi-refutable}), we notice that $\refutable{\cand{\xi_1}{\xi_2}}$ is in syntactic contradiction with all the cases, hence not derivable.
\end{proof}

\begin{lemma}
\label{lem:no-or-refutable}
There does not exist such a constraint $\cor{\xi_1}{\xi_2}$ such that $\refutable{\cor{\xi_1}{\xi_2}}$.
\end{lemma}
\begin{proof}
By rule induction over Rules (\ref{rules:xi-refutable}), we notice that $\refutable{\cor{\xi_1}{\xi_2}}$ is in syntactic contradiction with all the cases, hence not derivable.
\end{proof}

\begin{lemma}
\label{lem:satisfy-not-refutable}
If $\isntVal{e}$ and $\csatisfy{e}{\xi}$ then $\cancel{\refutable{\xi}}$.
\end{lemma}
\begin{proof}
\begin{pfsteps*}
\item $\isntVal{e}$ \BY{assumption} \pflabel{e-notval}
\item $\csatisfy{e}{\xi}$ \BY{assumption} \pflabel{satisfy}
\end{pfsteps*}
By rule induction over \rulesref{rules:Satisfy} on \pfref{satisfy}.
\begin{byCases}
\savelocalsteps{0}
\item[\text{(\ref{rule:CSTruth})}]
    \begin{pfsteps*}
    \item $\xi=\ctruth$ \BY{assumption}
    \end{pfsteps*}
    Assume $\refutable{\ctruth}$. By rule induction over \rulesref{rules:xi-refutable}, no case applies due to syntactic contradiction.\\
    Therefore, $\cancel{\refutable{\ctruth}}$.
\restorelocalsteps{0}
\item[\text{(\ref{rule:CSOr1}),(\ref{rule:CSOr2})}]
    \begin{pfsteps*}
    \item $\xi=\cor{\xi_1}{\xi_2}$ \BY{assumption}
    \item $\cancel{\refutable{\cor{\xi_1}{\xi_2}}}$ \BY{\autoref{lem:no-or-refutable}}
    \end{pfsteps*}
\restorelocalsteps{0}
\item[\text{(\ref{rule:CSAnd})}]
    \begin{pfsteps*}
    \item $\xi=\cand{\xi_1}{\xi_2}$ \BY{assumption}
    \item $\cancel{\refutable{\cand{\xi_1}{\xi_2}}}$ \BY{\autoref{lem:no-and-refutable}}
    \end{pfsteps*}
\restorelocalsteps{0}
\item[\text{(\ref{rule:CSNotValPair})}]
    \begin{pfsteps*}
    \item $\xi=\cpair{\xi_1}{\xi_2}$ \BY{assumption}
    \item $\csatisfy{\hprl{e}}{\xi_1}$ \BY{assumption} \pflabel{prl-satisfy}
    \item $\csatisfy{\hprr{e}}{\xi_2}$ \BY{assumption} \pflabel{prr-satisfy}
    \item $\isntVal{\hprl{e}}$ \BY{\ruleref{rule:NVPrl}} \pflabel{prl-notval}
    \item $\isntVal{\hprr{e}}$ \BY{\ruleref{rule:NVPrr}} \pflabel{prr-notval}
    \item $\cancel{\refutable{\xi_1}}$ \BY{IH on \pfref{prl-notval} and \pfref{prl-satisfy}} \pflabel{not-rft1}
    \item $\cancel{\refutable{\xi_2}}$ \BY{IH on \pfref{prr-notval} and \pfref{prr-satisfy}} \pflabel{not-rft2}
    \end{pfsteps*}
    Assume $\refutable{\cpair{\xi_1}{\xi_2}}$. By rule induction over \rulesref{rules:xi-refutable} on it, only two cases apply.
    \begin{byCases}
    \savelocalsteps{1}
    \item[\text{(\ref{rule:RXPairL})}]
        \begin{pfsteps*}
        \item $\refutable{\xi_1}$ \BY{assumption}
        \end{pfsteps*}
        Contradicts \pfref{not-rft1}.
    \restorelocalsteps{1}
    \item[\text{(\ref{rule:RXPairR})}]
        \begin{pfsteps*}
        \item $\refutable{\xi_2}$ \BY{assumption}
        \end{pfsteps*}
        Contradicts \pfref{not-rft2}.
    \end{byCases}
    Therefore, $\cancel{\refutable{\cpair{\xi_1}{\xi_2}}}$.
\restorelocalsteps{0}
\item
    \begin{pfsteps*}
    \item $e=\hnum{n},\hinl{\tau_2}{e_1},\hinr{\tau_1}{e_2},\hpair{e_1}{e_2}$ \BY{assumption}
    \end{pfsteps*}
    By rule induction over \rulesref{rules:notval} on \pfref{e-notval}, no case applies due to syntactic contradiction.
\end{byCases}
\resetpfcounter
\end{proof}

\begin{lemma}[Soundness and Completeness of Satisfaction Judgment]
  \label{lem:sound-complete-satisfy}
  $\csatisfy{e}{\xi}$ iff $\fsatisfy{e}{\xi} = \true$.
\end{lemma}
\begin{proof}
  We prove soundness and completeness separately.
  \begin{enumerate}
    \item Soundness:
      \begin{pfsteps*}
      \item $\csatisfy{e}{\xi}$ \BY{assumption} \pflabel{satisfy}
      \end{pfsteps*}
      By rule induction over Rules (\ref{rules:Satisfy}) on \pfref{satisfy}.
      \begin{byCases}

      \savelocalsteps{lem:sound-complete-satisfy-1}
      \item[\text{(\ref{rule:CSTruth})}]
        \begin{pfsteps*}
        \item $\xi = \ctruth$ \BY{assumption}
        \item $\fsatisfy{e}{\ctruth} = \true$ \BY{Definition \ref{defn:satisfy-truth}}
        \end{pfsteps*}

      \restorelocalsteps{lem:sound-complete-satisfy-1}
      \item[\text{(\ref{rule:CSNum})}]
        \begin{pfsteps*}
        \item $e = \hnum{n}$ \BY{assumption}
        \item $\xi = \cnum{n}$ \BY{assumption}
        \item $\fsatisfy{\hnum{n}}{\cnum{n}} = (n = n) = \true$ \BY{Definition \ref{defn:num-satisfy-num}}
        \end{pfsteps*}

      \restorelocalsteps{lem:sound-complete-satisfy-1}
      \item[\text{(\ref{rule:CSNotNum})}]
        \begin{pfsteps*}
        \item $e = \hnum{n_1}$ \BY{assumption}
        \item $\xi = \cnotnum{n_2}$ \BY{assumption}
        \item $n_1 \neq n_2$ \BY{assumption} \pflabel{numnotequal}
        \item $\fsatisfy{\hnum{n_1}}{\cnotnum{n_2}} = (n_1 \neq n_2) = \true$ \BY{Definition \ref{defn:num-satisfy-notnum} on \pfref{numnotequal}}
        \end{pfsteps*}

      \restorelocalsteps{lem:sound-complete-satisfy-1}
      \item[\text{(\ref{rule:CSAnd})}]
        \begin{pfsteps*}
        \item $\xi = \cand{\xi_1}{\xi_2}$ \BY{assumption}
        \item $\csatisfy{e}{\xi_1}$ \BY{assumption} \pflabel{[and]csatisfy1}
        \item $\csatisfy{e}{\xi_2}$ \BY{assumption} \pflabel{[and]csatisfy2}
        \item $\fsatisfy{e}{\xi_1} = \true$ \BY{IH on \pfref{[and]csatisfy1}} \pflabel{[and]fsatisfy1}
        \item $\fsatisfy{e}{\xi_2} = \true$ \BY{IH on \pfref{[and]csatisfy2}} \pflabel{[and]fsatisfy2}
        \item $\fsatisfy{e}{\cand{\xi_1}{\xi_2}} = \fsatisfy{e}{\xi_1} \text{ and } \fsatisfy{e}{\xi_2} = \true$ \BY{Definition \ref{defn:satisfy-and} on \pfref{[and]fsatisfy1} and \pfref{[and]fsatisfy2}}
        \end{pfsteps*}

      \restorelocalsteps{lem:sound-complete-satisfy-1}
      \item[\text{(\ref{rule:CSOr1})}]
        \begin{pfsteps*}
        \item $\xi = \cor{\xi_1}{\xi_2}$ \BY{assumption}
        \item $\csatisfy{e}{\xi_1}$ \BY{assumption} \pflabel{[or1]csatisfy1}
        \item $\fsatisfy{e}{\xi_1} = \true$ \BY{IH on \pfref{[or1]csatisfy1}} \pflabel{[or1]fsatisfy1}
        \item $\fsatisfy{e}{\cor{\xi_1}{\xi_2}} = \fsatisfy{e}{\xi_1} \text{ or } \fsatisfy{e}{\xi_2} = \true$ \BY{Definition \ref{defn:satisfy-or} on \pfref{[or1]fsatisfy1}}
        \end{pfsteps*}

      \restorelocalsteps{lem:sound-complete-satisfy-1}
      \item[\text{(\ref{rule:CSOr2})}]
        \begin{pfsteps*}
        \item $\xi = \cor{\xi_1}{\xi_2}$ \BY{assumption}
        \item $\csatisfy{e}{\xi_2}$ \BY{assumption} \pflabel{[or2]csatisfy2}
        \item $\fsatisfy{e}{\xi_2} = \true$ \BY{IH on \pfref{[or2]csatisfy2}} \pflabel{[or2]fsatisfy2}
        \item $\fsatisfy{e}{\cor{\xi_1}{\xi_2}} = \fsatisfy{e}{\xi_1} \text{ or } \fsatisfy{e}{\xi_2} = \true$ \BY{Definition \ref{defn:satisfy-or} on \pfref{[or2]fsatisfy2}}
        \end{pfsteps*}

      \restorelocalsteps{lem:sound-complete-satisfy-1}
      \item[\text{(\ref{rule:CSInl})}]
        \begin{pfsteps*}
        \item $e = \hinl{\tau_2}{e_1}$ \BY{assumption}
        \item $\xi = \cinl{\xi_1}$ \BY{assumption}
        \item $\csatisfy{e_1}{\xi_1}$ \BY{assumption} \pflabel{[inl]csatisfy1}
        \item $\fsatisfy{e_1}{\xi_1} = \true$ \BY{IH on \pfref{[inl]csatisfy1}} \pflabel{[inl]fsatisfy1}
        \item $\fsatisfy{\hinl{\tau_2}{e_1}}{\cinl{\xi_1}} = \fsatisfy{e_1}{\xi_1} = \true$ \BY{Definition \ref{defn:inl-satisfy-inl} on \pfref{[inl]fsatisfy1}}
        \end{pfsteps*}

      \restorelocalsteps{lem:sound-complete-satisfy-1}
      \item[\text{(\ref{rule:CSInr})}]
        \begin{pfsteps*}
        \item $e = \hinr{\tau_1}{e_2}$ \BY{assumption}
        \item $\xi = \cinl{\xi_2}$ \BY{assumption}
        \item $\csatisfy{e_2}{\xi_2}$ \BY{assumption} \pflabel{[inr]csatisfy2}
        \item $\fsatisfy{e_2}{\xi_2} = \true$ \BY{IH on \pfref{[inr]csatisfy2}} \pflabel{[inr]fsatisfy2}
        \item $\fsatisfy{\hinr{\tau_1}{e_2}}{\cinr{\xi_2}} = \fsatisfy{e_2}{\xi_2} = \true$ \BY{Definition \ref{defn:inr-satisfy-inr} on \pfref{[inr]fsatisfy2}}
        \end{pfsteps*}

      \restorelocalsteps{lem:sound-complete-satisfy-1}
      \item[\text{(\ref{rule:CSPair})}]
        \begin{pfsteps*}
        \item $e = \hpair{e_1}{e_2}$ \BY{assumption}
        \item $\xi = \cpair{\xi_1}{\xi_2}$ \BY{assumption}
        \item $\csatisfy{e_1}{\xi_1}$ \BY{assumption} \pflabel{[pair]csatisfy1}
        \item $\csatisfy{e_2}{\xi_2}$ \BY{assumption} \pflabel{[pair]csatisfy2}
        \item $\fsatisfy{e_1}{\xi_1} = \true$ \BY{IH on \pfref{[pair]csatisfy1}} \pflabel{[pair]fsatisfy1}
        \item $\fsatisfy{e_2}{\xi_2} = \true$ \BY{IH on \pfref{[pair]csatisfy2}} \pflabel{[pair]fsatisfy2}
        \item $\fsatisfy{\hpair{e_1}{e_2}}{\cpair{\xi_1}{\xi_2}} = \fsatisfy{e_1}{\xi_1} \text{ and } \fsatisfy{e_2}{\xi_2} = \true$ \BY{Definition \ref{defn:pair-satisfy-pair} on \pfref{[pair]fsatisfy1} and \pfref{[pair]fsatisfy2}}
        \end{pfsteps*}
      
      \restorelocalsteps{lem:sound-complete-satisfy-1}
      \item[\text{(\ref{rule:CSNotValPair})}]
        \begin{pfsteps*}
        \item $\xi = \cpair{\xi_1}{\xi_2}$ \BY{assumption}
        \item $\isntVal{e}$ \BY{assumption} \pflabel{[notval]notval}
        \item $\csatisfy{\hprl{e}}{\xi_1}$ \BY{assumption} \pflabel{[notval]csatisfy1}
        \item $\csatisfy{\hprr{e}}{\xi_2}$ \BY{assumption} \pflabel{[notval]csatisfy2}
        \item $\fsatisfy{\hprl{e}}{\xi_1} = \true$ \BY{IH on \pfref{[notval]csatisfy1}} \pflabel{[notval]fsatisfy1}
        \item $\fsatisfy{\hprr{e}}{\xi_2} = \true$ \BY{IH on \pfref{[notval]csatisfy2}} \pflabel{[notval]fsatisfy2}
        \end{pfsteps*}
        By rule induction over \rulesref{rules:notval} on \pfref{[notval]notval}.
        \begin{byCases}
        \item
          \begin{pfsteps*}
          \item $e=\hehole{u},\hhole{e_0}{u},\hap{e_1}{e_2},\hprl{e_0},\hprr{e_0},\hmatch{e_0}{\zrules}$ \BY{assumption}
          \item $\fsatisfy{e}{\cpair{\xi_1}{\xi_2}} = \fsatisfy{\hprl{e}}{\xi_1} \text{ and } \fsatisfy{\hprr{e}}{\xi_2} = \true$ \BY{Definition \ref{defn:satisfy} on \pfref{[notval]fsatisfy1} and \pfref{[notval]fsatisfy2}}
          \end{pfsteps*}
        \end{byCases}
      \end{byCases}

    \resetpfcounter
    
    \item Completeness:
      \begin{pfsteps*}
      \item $\fsatisfy{e}{\xi} = \true$ \BY{assumption} \pflabel{fsatisfy}
      \end{pfsteps*}
      By structural induction on $\xi$.
      \begin{byCases}

      \savelocalsteps{lem:sound-complete-satisfy-1}
      \item[\xi=\ctruth]
        \begin{pfsteps*}
        \item $\csatisfy{e}{\ctruth}$ \BY{Rule (\ref{rule:CSTruth})}
        \end{pfsteps*}

      \restorelocalsteps{lem:sound-complete-satisfy-1}
      \item[\xi=\cfalsity, \cunknown]
        \begin{pfsteps*}
        \item $\fsatisfy{e}{\xi} = \false$ \BY{Definition \ref{defn:not-satisfy}} \pflabel{fsatisfy-unknown}
        \end{pfsteps*}
        \pfref{fsatisfy-unknown} contradicts \pfref{fsatisfy} and thus vacuously true.

      \restorelocalsteps{lem:sound-complete-satisfy-1}
      \item[\xi=\cnum{n}]\ \\
        By structural induction on $e$.
        \begin{byCases}

          \savelocalsteps{lem:sound-complete-satisfy-2}
          \item[e=\hnum{n'}]
            \begin{pfsteps*}
              \item $n' = n$ \BY{Definition \ref{defn:num-satisfy-num} on \pfref{fsatisfy}} \pflabel{numequal}
              \item $\csatisfy{\hnum{n'}}{\cnum{n}}$ \BY{Rule (\ref{rule:CSNum}) on \pfref{numequal}}
            \end{pfsteps*}

          \restorelocalsteps{lem:sound-complete-satisfy-2}
          \item
            \begin{pfsteps*}
            \item $\fsatisfy{e}{\cnum{n}} = \false$ \BY{Definition \ref{defn:not-satisfy}} \pflabel{[cnum]fsatisfy}
            \end{pfsteps*}
            \pfref{[cnum]fsatisfy} contradicts \pfref{fsatisfy} and thus vacuously true. 
        \end{byCases}
        
      \restorelocalsteps{lem:sound-complete-satisfy-1}
      \item[\xi=\cnotnum{n}]\ \\
        By structural induction on $e$.
        \begin{byCases}

          \savelocalsteps{lem:sound-complete-satisfy-2}
          \item[e=\hnum{n'}]
            \begin{pfsteps*}
            \item $n' \neq n$ \BY{Definition \ref{defn:num-satisfy-notnum} on \pfref{fsatisfy}} \pflabel{numnotequal}
            \item $\csatisfy{\hnum{n'}}{\cnotnum{n}}$ \BY{Rule (\ref{rule:CSNotNum}) on \pfref{numnotequal}}
            \end{pfsteps*}
            
          \restorelocalsteps{lem:sound-complete-satisfy-2}
          \item
            \begin{pfsteps*}
            \item $\fsatisfy{e}{\cnotnum{n}} = \false$ \BY{Definition \ref{defn:not-satisfy}} \pflabel{[cnotnum]fsatisfy}
            \end{pfsteps*}
            \pfref{[cnotnum]fsatisfy} contradicts \pfref{fsatisfy} and thus vacuously true. 
        \end{byCases}

      \restorelocalsteps{lem:sound-complete-satisfy-1}
      \item[\xi=\cand{\xi_1}{\xi_2}]
        \begin{pfsteps*}
        \item $\fsatisfy{e}{\xi_1} = \true$ \BY{Definition \ref{defn:satisfy-and} on \pfref{fsatisfy}} \pflabel{[and]fsatisfy1}
        \item $\fsatisfy{e}{\xi_2} = \true$ \BY{Definition \ref{defn:satisfy-and} on \pfref{fsatisfy}} \pflabel{[and]fsatisfy2}
        \item $\csatisfy{e}{\xi_1}$ \BY{IH on \pfref{[and]fsatisfy1}} \pflabel{[and]csatisfy1}
        \item $\csatisfy{e}{\xi_2}$ \BY{IH on \pfref{[and]fsatisfy2}} \pflabel{[and]csatisfy2}
        \item $\csatisfy{e}{\cand{\xi_1}{\xi_2}}$ \BY{Rule (\ref{rule:CSAnd}) on \pfref{[and]csatisfy1} and \pfref{[and]csatisfy2}}
        \end{pfsteps*}

      \restorelocalsteps{lem:sound-complete-satisfy-1}
      \item[\xi=\cor{\xi_1}{\xi_2}]
        \begin{pfsteps*}
        \item $\fsatisfy{e}{\xi_1} \text{ or } \fsatisfy{e}{\xi_2} = \true$ \BY{Definition \ref{defn:satisfy-or} on \pfref{fsatisfy}} \pflabel{[or]fsatisfy}
        \end{pfsteps*}
        By case analysis on \pfref{[or]fsatisfy}.
        \begin{byCases}

          \savelocalsteps{lem:sound-complete-satisfy-2}
          \item[\fsatisfy{e}{\xi_1}=\true]
          \begin{pfsteps*}
          \item $\fsatisfy{e}{\xi_1} = \true$ \BY{assumption} \pflabel{[or]fsatisfy1}
          \item $\csatisfy{e}{\xi_1}$ \BY{IH on \pfref{[or]fsatisfy1}} \pflabel{[or]csatisfy1}
          \item $\csatisfy{e}{\cor{\xi_1}{\xi_2}}$ \BY{Rule (\ref{rule:CSOr1}) on \pfref{[or]csatisfy1}}
          \end{pfsteps*}

          \restorelocalsteps{lem:sound-complete-satisfy-2}
          \item[\fsatisfy{e}{\xi_2}=\true]
          \begin{pfsteps*}
          \item $\fsatisfy{e}{\xi_2} = \true$ \BY{assumption} \pflabel{[or]fsatisfy2}
          \item $\csatisfy{e}{\xi_2}$ \BY{IH on \pfref{[or]fsatisfy2}} \pflabel{[or]csatisfy2}
          \item $\csatisfy{e}{\cor{\xi_1}{\xi_2}}$ \BY{Rule (\ref{rule:CSOr2}) on \pfref{[or]csatisfy2}}
          \end{pfsteps*}
        \end{byCases}

      \restorelocalsteps{lem:sound-complete-satisfy-1}
      \item[\xi=\cinl{\xi_1}]\ \\
        By structural induction on $e$.
        \begin{byCases}

          \savelocalsteps{lem:sound-complete-satisfy-2}
          \item[e=\hinl{\tau_2}{e_1}]
            \begin{pfsteps*}
              \item $\fsatisfy{e_1}{\xi_1}=\true$ \BY{Definition \ref{defn:inl-satisfy-inl} on \pfref{fsatisfy}} \pflabel{[inl]fsatisfy1}
              \item $\csatisfy{e_1}{\xi_1}$ \BY{IH on \pfref{[inl]fsatisfy1}} \pflabel{[inl]csatisfy1}
              \item $\csatisfy{\hinl{\tau_2}{e_1}}{\cinl{\xi_1}}$ \BY{Rule (\ref{rule:CSInl}) on \pfref{[inl]csatisfy1}}
            \end{pfsteps*}

            \restorelocalsteps{lem:sound-complete-satisfy-2}
          \item
            \begin{pfsteps*}
            \item $\fsatisfy{e}{\cinl{\xi_1}} = \false$ \BY{Definition \ref{defn:not-satisfy}} \pflabel{[inl]fnotsatisfy}
            \end{pfsteps*}
            \pfref{[inl]fnotsatisfy} contradicts \pfref{fsatisfy} and thus vacuously true. 
        \end{byCases}

      \restorelocalsteps{lem:sound-complete-satisfy-1}
      \item[\xi=\cinr{\xi_2}]\ \\
        By structural induction on $e$.
        \begin{byCases}

          \savelocalsteps{lem:sound-complete-satisfy-2}
          \item[e=\hinr{\tau_1}{e_2}]
            \begin{pfsteps*}
              \item $\fsatisfy{e_2}{\xi_2}=\true$ \BY{Definition \ref{defn:inr-satisfy-inr} on \pfref{fsatisfy}} \pflabel{[inr]fsatisfy2}
              \item $\csatisfy{e_2}{\xi_2}$ \BY{IH on \pfref{[inr]fsatisfy2}} \pflabel{[inr]csatisfy2}
              \item $\csatisfy{\hinr{\tau_1}{e_2}}{\cinr{\xi_2}}$ \BY{Rule (\ref{rule:CSInr}) on \pfref{[inr]csatisfy2}}
            \end{pfsteps*}

          \restorelocalsteps{lem:sound-complete-satisfy-2}
          \item
            \begin{pfsteps*}
            \item $\fsatisfy{e}{\cinr{\xi_2}} = \false$ \BY{Definition \ref{defn:not-satisfy}} \pflabel{[inr]fnotsatisfy}
            \end{pfsteps*}
            \pfref{[inr]fnotsatisfy} contradicts \pfref{fsatisfy} and thus vacuously true. 
        \end{byCases}

      \restorelocalsteps{lem:sound-complete-satisfy-1}
      \item[\xi=\cpair{\xi_1}{\xi_2}]\ \\
        By structural induction on $e$.
        \begin{byCases}

          \savelocalsteps{lem:sound-complete-satisfy-2}
          \item[e=\hpair{e_1}{e_2}]
            \begin{pfsteps*}
            \item $\fsatisfy{e_1}{\xi_1} = \true$ \BY{Definition \ref{defn:pair-satisfy-pair} on \pfref{fsatisfy}} \pflabel{[pair]fsatisfy1}
            \item $\fsatisfy{e_2}{\xi_2} = \true$ \BY{Definition \ref{defn:pair-satisfy-pair} on \pfref{fsatisfy}} \pflabel{[pair]fsatisfy2}
            \item $\csatisfy{e_1}{\xi_1}$ \BY{IH on \pfref{[pair]fsatisfy1}} \pflabel{[pair]csatisfy1}
            \item $\csatisfy{e_2}{\xi_2}$ \BY{IH on \pfref{[pair]fsatisfy2}} \pflabel{[pair]csatisfy2}
            \item $\csatisfy{\hpair{e_1}{e_2}}{\cpair{\xi_1}{\xi_2}}$ \BY{Rule (\ref{rule:CSPair}) on \pfref{[pair]csatisfy1} and \pfref{[pair]csatisfy2}}
            \end{pfsteps*}
        
          \restorelocalsteps{lem:sound-complete-satisfy-2}
          \item[e=\hehole{u},\hhole{e_0}{u},\hap{e_1}{e_2},\hprl{e_0},\hprr{e_0},\hmatch{e_0}{\zrules}]
            \begin{pfsteps*}
            \item $\fsatisfy{\hprl{e}}{\xi_1} = \true$ \BY{Definition \ref{defn:pair-satisfy-pair} on \pfref{fsatisfy}} \pflabel{[notval]fsatisfy1}
            \item $\fsatisfy{\hprr{e}}{\xi_2} = \true$ \BY{Definition \ref{defn:pair-satisfy-pair} on \pfref{fsatisfy}} \pflabel{[notval]fsatisfy2}
            \item $\csatisfy{\hprl{e}}{\xi_1}$ \BY{IH on \pfref{[notval]fsatisfy1}} \pflabel{[notval]csatisfy1}
            \item $\csatisfy{\hprr{e}}{\xi_2}$ \BY{IH on \pfref{[notval]fsatisfy2}} \pflabel{[notval]csatisfy2}
            \item $\isntVal{e}$ \BY{each rule in \rulesref{rules:notval}} \pflabel{[notval]notval}
            \item $\csatisfy{\hpair{e_1}{e_2}}{\cpair{\xi_1}{\xi_2}}$ \BY{Rule (\ref{rule:CSNotValPair}) on \pfref{[notval]notval} and \pfref{[notval]csatisfy1} and \pfref{[notval]csatisfy2}}
            \end{pfsteps*}
          \restorelocalsteps{lem:sound-complete-satisfy-2}
          \item
            \begin{pfsteps*}
            \item $\fsatisfy{e}{\cpair{\xi_1}{\xi_2}} = \false$ \BY{Definition \ref{defn:not-satisfy}} \pflabel{[pair]fnotsatisfy}
            \end{pfsteps*}
            \pfref{[pair]fnotsatisfy} contradicts \pfref{fsatisfy} and thus vacuously true. 
        \end{byCases}
      \end{byCases}
  \end{enumerate}
  \resetpfcounter
\end{proof}


\begin{lemma}
  \label{lem:not-satormay}
  $\cnotsatisfy{e}{\xi}$ and $\cnotmaysatisfy{e}{\xi}$ iff $\cnotsatisfyormay{e}{\xi}$.
\end{lemma}
\begin{proof}
\begin{enumerate}
    \item Sufficiency:
    \begin{pfsteps*}
    \item $\cnotsatisfy{e}{\xi}$ \BY{assumption} \pflabel{notsatisfy}
    \item $\cnotmaysatisfy{e}{\xi}$ \BY{assumption} \pflabel{notmaysat}
    \end{pfsteps*}
    Assume $\csatisfyormay{e}{\xi}$. By rule induction over Rules (\ref{rules:satormay}) on it.
    \begin{byCases}
    \savelocalsteps{0}
    \item[\text{(\ref{rule:CSMSMay})}]
        \begin{pfsteps*}
        \item $\csatisfy{e}{\xi}$ \BY{assumption}
        \end{pfsteps*}
        Contradicts \pfref{notsatisfy}.
    \restorelocalsteps{0}
    \item[\text{(\ref{rule:CSMSSat})}]
        \begin{pfsteps*}
        \item $\cmaysatisfy{e}{\xi}$ \BY{assumption}
        \end{pfsteps*}
        Contradicts \pfref{notmaysat}.
    \end{byCases}
    Therefore, $\csatisfyormay{e}{\xi}$ is not derivable.
    \resetpfcounter
    \item Necessity:
    \begin{pfsteps*}
    \item $\cnotsatisfyormay{e}{\xi}$ \BY{assumption} \pflabel{notsatormay}
    \end{pfsteps*}
    Assume $\csatisfy{e}{\xi}$.
    \begin{pfsteps*}
    \item $\csatisfyormay{e}{\xi}$ \BY{\ruleref{rule:CSMSSat} on assumption}
    \end{pfsteps*}
    Contradicts \pfref{notsatormay}. Therefore, $\cnotsatisfy{e}{\xi}$.
    Assume $\cmaysatisfy{e}{\xi}$.
    \begin{pfsteps*}
    \item $\csatisfyormay{e}{\xi}$ \BY{\ruleref{rule:CSMSMay} on assumption}
    \end{pfsteps*}
    Contradicts \pfref{notsatormay}. Therefore, $\cnotmaysatisfy{e}{\xi}$.
\end{enumerate}
\end{proof}

\begin{theorem}[Exclusiveness of Satisfaction Judgment]
  \label{thrm:exclusive-constraint-satisfaction}
  If $\ctyp{\xi}{\tau}$ and $\hexptyp{\cdot}{\Delta}{e}{\tau}$ and $\isFinal{e}$ then exactly one of the following holds
  \begin{enumerate}
    \item $\csatisfy{e}{\xi}$
    \item $\cmaysatisfy{e}{\xi}$
    \item $\cnotsatisfyormay{e}{\xi}$
  \end{enumerate}
\end{theorem}
\begin{proof}
\begin{pfsteps*}
\item $\ctyp{\xi}{\tau}$ \BY{assumption} \pflabel{cTyp}
\item $\hexptyp{\cdot}{\Delta}{e}{\tau}$ \BY{assumption} \pflabel{eTyp}
\item $\isFinal{e}$ \BY{assumption} \pflabel{eFinal}
\end{pfsteps*}
By rule induction over Rules (\ref{rules:CTyp}) on \pfref{cTyp}, we would show one conclusion is derivable while the other two are not.
\begin{byCases}

\savelocalsteps{0}
\item[\text{(\ref{rule:CTTruth})}]
    \begin{pfsteps*}
    \item $\xi=\ctruth$ \BY{assumption}
    \item $\csatisfy{e}{\ctruth}$ \BY{Rule (\ref{rule:CSTruth})} \pflabel{[truth]satisfy}
    \item $\cnotmaysatisfy{e}{\ctruth}$ \BY{\autoref{lem:no-e-may-satisfy-truth}}
    \item $\csatisfyormay{e}{\ctruth}$ \BY{\ruleref{rule:CSMSSat} on \pfref{[truth]satisfy}}
    \end{pfsteps*}
    
\restorelocalsteps{0}
\item[\text{(\ref{rule:CTFalsity})}]
    \begin{pfsteps*}
    \item $\xi=\cfalsity$ \BY{assumption}
    \item $\cnotsatisfy{e}{\cfalsity}$ \BY{\autoref{lem:no-e-satisfy-falsity}} \pflabel{[falsity]notsatisfy}
    \item $\cnotmaysatisfy{e}{\cfalsity}$ \BY{\autoref{lem:no-e-may-satisfy-falsity}} \pflabel{[falsity]notmaysat}
    \item $\cnotsatisfyormay{e}{\cfalsity}$ \BY{\autoref{lem:not-satormay} on \pfref{[falsity]notsatisfy} and \pfref{[falsity]notmaysat}}
    \end{pfsteps*}
    
\restorelocalsteps{0}
\item[\text{(\ref{rule:CTUnknown})}]
    \begin{pfsteps*}
    \item $\xi=\cunknown$ \BY{assumption}
    \item $\cnotsatisfy{e}{\cunknown}$ \BY{\autoref{lem:no-e-satisfy-unknown}}
    \item $\cmaysatisfy{e}{\cunknown}$ \BY{Rule (\ref{rule:CMSUnknown})} \pflabel{[unknown]maysat}
    \item $\csatisfyormay{e}{\cunknown}$ \BY{\ruleref{rule:CSMSMay} on \pfref{[unknown]maysat}} 
    \end{pfsteps*}
    
\restorelocalsteps{0}
\item[\text{(\ref{rule:CTNum})}]
    \begin{pfsteps*}
    \item $\xi=\cnum{n_2}$ \BY{assumption}
    \item $\tau=\tnum$ \BY{assumption}
    \end{pfsteps*}
    By rule induction over Rules (\ref{rules:TExp}) on \pfref{eTyp}, the following cases apply.
    \begin{byCases}
    \savelocalsteps{1}
    \item[\text{(\ref{rule:TEHole}),(\ref{rule:THole}),(\ref{rule:TAp}),(\ref{rule:TPrl}),(\ref{rule:TPrr}),(\ref{rule:TMatchZPre}),(\ref{rule:TMatchNZPre})}]
        \begin{pfsteps*}
        \item $e=\hehole{u},\hhole{e_0}{u},\hap{e_1}{e_2},\hprl{e_0},\hprr{e_0},\hmatch{e_0}{\zrules}$ \BY{assumption}
        \item $\isntVal{e}$ \BY{Rule (\ref{rule:NVEHole}),(\ref{rule:NVHole}),(\ref{rule:NVAp}),(\ref{rule:NVMatch}),(\ref{rule:NVPrl}),(\ref{rule:NVPrr})} \pflabel{[num]notval}
        \end{pfsteps*}
        Assume $\csatisfy{e}{\cnum{n_2}}$. By rule induction over Rules (\ref{rules:Satisfy}) on it, no case applies due to syntactic contradiction on $\xi$.\\
        \begin{pfsteps*}
        \item $\cnotsatisfy{e}{\cnum{n_2}}$ \BY{contradiction}
        \item $\refutable{\cnum{n_2}}$ \BY{\ruleref{rule:RXNum}} \pflabel{[num]rft}
        \item $\cmaysatisfy{e}{\cnum{n_2}}$ \BY{Rule (\ref{rule:CMSNotVal}) on \pfref{[num]notval} and \pfref{[num]rft}} \pflabel{[num]maysat}
        \item $\csatisfyormay{e}{\cnum{n_2}}$ \BY{\ruleref{rule:CSMSMay} on \pfref{[num]maysat}} 
        \end{pfsteps*}
    \restorelocalsteps{1}
    \item[\text{(\ref{rule:TNum})}]
        \begin{pfsteps*}
        \item $e=\hnum{n_1}$ \BY{assumption}
        \end{pfsteps*}
        Assume $\cmaysatisfy{\hnum{n_1}}{\cnum{n_2}}$. By rule induction over Rules (\ref{rules:MaySatisfy}), only one case applies.
        \begin{byCases}
        \item[\text{(\ref{rule:CMSNotVal})}]
            \begin{pfsteps*}
            \item $\isntVal{\hnum{n_1}}$ \BY{assumption}
            \end{pfsteps*}
            Contradicts \autoref{lem:no-num-notval}.
        \end{byCases}
        \begin{pfsteps*}
        \item $\cnotmaysatisfy{\hnum{n_1}}{\cnum{n_2}}$ \BY{contradiction} \pflabel{[num]num-notmaysat}
        \end{pfsteps*}
        By case analysis on whether $n_1$ is equal to $n_2$.
        \begin{byCases}
        \savelocalsteps{2}
        \item[n_1=n_2]
            \begin{pfsteps*}
            \item $\fsatisfy{\hnum{n_1}}{\cnum{n_2}}=\true$ \BY{Definition \ref{defn:satisfy}} \pflabel{fsatisfy-num-num-true}
            \item $\csatisfy{\hnum{n_1}}{\cnum{n_2}}$ \BY{Lemma \ref{lem:sound-complete-satisfy} on \pfref{fsatisfy-num-num-true}} \pflabel{[num]num-satisfy}
            \item $\csatisfyormay{e}{\cnum{n_2}}$ \BY{\ruleref{rule:CSMSSat} on \pfref{[num]num-satisfy}} 
            \end{pfsteps*}
        \restorelocalsteps{2}
        \item[n_1\neq n_2]
            \begin{pfsteps*}
            \item $\fsatisfy{\hnum{n_1}}{\cnum{n_2}}=\false$ \BY{Definition \ref{defn:satisfy}} \pflabel{fsatisfy-num-num-false}
            \item $\cnotsatisfy{\hnum{n_1}}{\cnum{n_2}}$ \BY{Lemma \ref{lem:sound-complete-satisfy} on \pfref{fsatisfy-num-num-false}} \pflabel{[num]num-notsatisfy}
            \item $\cnotsatisfyormay{e}{\cnum{n_2}}$ \BY{\autoref{lem:not-satormay} on \pfref{[num]num-notmaysat} and \pfref{[num]num-notsatisfy}} 
            \end{pfsteps*}
        \end{byCases}
    \end{byCases}

\restorelocalsteps{0}
\item[\text{(\ref{rule:CTNotNum})}]
    \begin{pfsteps*}
    \item $\xi=\cnotnum{n_2}$ \BY{assumption}
    \item $\tau=\tnum$ \BY{assumption}
    \end{pfsteps*}
    By rule induction over Rules (\ref{rules:TExp}) on \pfref{eTyp}, the following cases apply.
    \begin{byCases}
    \savelocalsteps{1}
    \item[\text{(\ref{rule:TEHole}),(\ref{rule:THole}),(\ref{rule:TAp}),(\ref{rule:TPrl}),(\ref{rule:TPrr}),(\ref{rule:TMatchZPre}),(\ref{rule:TMatchNZPre})}]
        \begin{pfsteps*}
        \item $e=\hehole{u},\hhole{e_0}{u},\hap{e_1}{e_2},\hprl{e_0},\hprr{e_0},\hmatch{e_0}{\zrules}$ \BY{assumption}
         \item $\isntVal{e}$ \BY{Rule (\ref{rule:NVEHole}),(\ref{rule:NVHole}),(\ref{rule:NVAp}),(\ref{rule:NVMatch}),(\ref{rule:NVPrl}),(\ref{rule:NVPrr})} \pflabel{[notnum]notval}
        \end{pfsteps*}
        Assume $\csatisfy{e}{\cnotnum{n_2}}$. By rule induction over Rules (\ref{rules:Satisfy}) on it, no case applies due to syntactic contradiction on $\xi$.
        \begin{pfsteps*}
        \item $\cnotsatisfy{e}{\cnotnum{n_2}}$ \BY{contradiction}
        \item $\refutable{\cnotnum{n_2}}$ \BY{\ruleref{rule:RXNotNum}} \pflabel{[notnum]rft}
        \item $\cmaysatisfy{e}{\cnum{n_2}}$ \BY{Rule (\ref{rule:CMSNotVal}) on \pfref{[notnum]notval} and \pfref{[notnum]rft}} \pflabel{[notnum]maysat}
        \item $\csatisfyormay{e}{\cnum{n_2}}$ \BY{\ruleref{rule:CSMSMay} on \pfref{[notnum]maysat}} 
        \end{pfsteps*}
    \restorelocalsteps{1}
    \item[\text{(\ref{rule:TNum})}]
        \begin{pfsteps*}
        \item $e=\hnum{n_1}$ \BY{assumption}
        \end{pfsteps*}
        Assume $\cmaysatisfy{\hnum{n_1}}{\cnotnum{n_2}}$. By rule induction over Rules (\ref{rules:MaySatisfy}), only one case applies.
        \begin{byCases}
        \item[\text{(\ref{rule:CMSNotVal})}]
            \begin{pfsteps*}
            \item $\isntVal{\hnum{n_1}}$ \BY{assumption}
            \end{pfsteps*}
            Contradicts \autoref{lem:no-num-notval}.
        \end{byCases}
        \begin{pfsteps*}
        \item $\cnotmaysatisfy{\hnum{n_1}}{\cnotnum{n_2}}$ \BY{contradiction} \pflabel{[notnum]num-notmaysat}
        \end{pfsteps*}
        
        By case analysis on whether $n_1$ is equal to $n_2$.
        \begin{byCases}
        \savelocalsteps{2}
        \item[n_1=n_2]
            \begin{pfsteps*}
            \item $\fsatisfy{\hnum{n_1}}{\cnotnum{n_2}}=\false$ \BY{Definition \ref{defn:satisfy}} \pflabel{fsatisfy-num-notnum-false'}
            \item $\cnotsatisfy{\hnum{n_1}}{\cnotnum{n_2}}$ \BY{Lemma \ref{lem:sound-complete-satisfy} on \pfref{fsatisfy-num-notnum-false'}} \pflabel{[notnum]num-notsatisfy}
            \item $\cnotsatisfyormay{e}{\cnum{n_2}}$ \BY{\autoref{lem:not-satormay} on \pfref{[notnum]num-notmaysat} and \pfref{[notnum]num-notsatisfy}} 
            \end{pfsteps*}
        \restorelocalsteps{2}
        \item[n_1\neq n_2]
            \begin{pfsteps*}
            \item $\fsatisfy{\hnum{n_1}}{\cnotnum{n_2}}=\true$ \BY{Definition \ref{defn:satisfy}} \pflabel{fsatisfy-num-notnum-true'}
            \item $\csatisfy{\hnum{n_1}}{\cnotnum{n_2}}$ \BY{Lemma \ref{lem:sound-complete-satisfy} on \pfref{fsatisfy-num-notnum-true'}} \pflabel{[notnum]num-satisfy} 
            \item $\csatisfyormay{e}{\cnum{n_2}}$ \BY{\ruleref{rule:CSMSSat} on \pfref{[notnum]num-satisfy}} 
            \end{pfsteps*}
        \end{byCases}
    \end{byCases}

\restorelocalsteps{0}
\item[\text{(\ref{rule:CTAnd})}]
    \begin{pfsteps*}
    \item $\xi=\cand{\xi_1}{\xi_2}$ \BY{assumption}
    \end{pfsteps*}
    By inductive hypothesis on \pfref{eTyp} and \pfref{eFinal}, exactly one of $\csatisfy{e}{\xi_1}$, $\cmaysatisfy{e}{\xi_1}$, and $\cnotsatisfyormay{e}{\xi_1}$ holds. The same goes for $\xi_2$. By case analysis on which conclusion holds for $\xi_1$ and $\xi_2$.
    \begin{byCases}
    \savelocalsteps{1}
    \item[\csatisfy{e}{\xi_1},\csatisfy{e}{\xi_2}]
        \begin{pfsteps*}
        \item $\csatisfy{e}{\xi_1}$ \BY{assumption} \pflabel{[and]satisfy1}
        \item $\cnotmaysatisfy{e}{\xi_1}$ \BY{assumption} \pflabel{[and]notmaysat1}
        \item $\csatisfy{e}{\xi_2}$ \BY{assumption} \pflabel{[and]satisfy2}
        \item $\cnotmaysatisfy{e}{\xi_2}$ \BY{assumption} \pflabel{[and]notmaysat2}
        \item $\csatisfy{e}{\cand{\xi_1}{\xi_2}}$ \BY{Rule (\ref{rule:CSAnd}) on \pfref{[and]satisfy1} and \pfref{[and]satisfy2}} \pflabel{[and]satisfy-and}
        \item $\csatisfyormay{e}{\cand{\xi_1}{\xi_2}}$ \BY{\ruleref{rule:CSMSSat} on \pfref{[and]satisfy-and}}
        \end{pfsteps*}
        Assume $\cmaysatisfy{e}{\cand{\xi_1}{\xi_2}}$. By rule induction over Rules (\ref{rules:MaySatisfy}) on it, the following cases apply.
        \begin{byCases}
        \savelocalsteps{2}
        \item[\text{(\ref{rule:CMSNotVal})}]
            \begin{pfsteps*}
            \item $\refutable{\cand{\xi_1}{\xi_2}}$ \BY{assumption}
            \end{pfsteps*}
            Contradicts \autoref{lem:no-and-refutable}.
        \restorelocalsteps{2}
        \item[\text{(\ref{rule:CMSAnd1})}]
            \begin{pfsteps*}
            \item $\cmaysatisfy{e}{\xi_1}$ \BY{assumption}
            \end{pfsteps*}
            Contradicts \pfref{[and]notmaysat1}.
        \restorelocalsteps{2}
        \item[\text{(\ref{rule:CMSAnd2})}]
            \begin{pfsteps*}
            \item $\cmaysatisfy{e}{\xi_2}$ \BY{assumption}
            \end{pfsteps*}
            Contradicts \pfref{[and]notmaysat2}.
        \restorelocalsteps{2}
        \item[\text{(\ref{rule:CMSAnd3})}]
            \begin{pfsteps*}
            \item $\cmaysatisfy{e}{\xi_1}$ \BY{assumption}
            \end{pfsteps*}
            Contradicts \pfref{[and]notmaysat1}.
        \end{byCases}
        Therefore, $\cnotmaysatisfy{e}{\cand{\xi_1}{\xi_2}}$.
        
    \restorelocalsteps{1}
    \item[\csatisfy{e}{\xi_1},\cmaysatisfy{e}{\xi_2}]
        \begin{pfsteps*}
        \item $\csatisfy{e}{\xi_1}$ \BY{assumption} \pflabel{[and2]satisfy1}
        \item $\cnotmaysatisfy{e}{\xi_1}$ \BY{assumption} \pflabel{[and2]notmaysat1}
        \item $\cnotsatisfy{e}{\xi_2}$ \BY{assumption} \pflabel{[and2]notsatisfy2}
        \item $\cmaysatisfy{e}{\xi_2}$ \BY{assumption} \pflabel{[and2]maysat2}
        \item $\cmaysatisfy{e}{\cand{\xi_1}{\xi_2}}$ \BY{Rule (\ref{rule:CMSAnd2}) on \pfref{[and2]satisfy1} and \pfref{[and2]maysat2}} \pflabel{[and2]maysat-and}
        \item $\csatisfyormay{e}{\cand{\xi_1}{\xi_2}}$ \BY{\ruleref{rule:CSMSMay} on \pfref{[and2]maysat-and}}
        \end{pfsteps*}
        Assume $\csatisfy{e}{\cand{\xi_1}{\xi_2}}$. By rule induction over Rules (\ref{rules:Satisfy}), only one case applies.
        \begin{byCases}
        \item[\text{(\ref{rule:CSAnd})}]
            \begin{pfsteps*}
            \item $\csatisfy{e}{\xi_2}$ \BY{assumption}
            \end{pfsteps*}
            Contradicts \pfref{[and2]notsatisfy2}.
        \end{byCases}
        \begin{pfsteps*}
        \item $\cnotsatisfy{e}{\cand{\xi_1}{\xi_2}}$ \BY{contradiction}
        \end{pfsteps*}
    \restorelocalsteps{1}
    \item[\csatisfy{e}{\xi_1},\cnotsatisfyormay{e}{\xi_2}]
        \begin{pfsteps*}
        \item $\csatisfy{e}{\xi_1}$ \BY{assumption} \pflabel{[and3]satisfy1}
        \item $\cnotmaysatisfy{e}{\xi_1}$ \BY{assumption} \pflabel{[and3]notmaysat1}
        \item $\cnotsatisfy{e}{\xi_2}$ \BY{assumption} \pflabel{[and3]notsatisfy2}
        \item $\cnotmaysatisfy{e}{\xi_2}$ \BY{assumption} \pflabel{[and3]notmaysat2}
        \end{pfsteps*}
        Assume $\csatisfy{e}{\cand{\xi_1}{\xi_2}}$. By rule induction over Rules (\ref{rules:Satisfy}), only one case applies.
        \begin{byCases}
        \item[\text{(\ref{rule:CSAnd})}]
            \begin{pfsteps*}
            \item $\csatisfy{e}{\xi_2}$ \BY{assumption}
            \end{pfsteps*}
            Contradicts \pfref{[and3]notsatisfy2}.
        \end{byCases}
        \begin{pfsteps*}
        \item $\cnotsatisfy{e}{\cand{\xi_1}{\xi_2}}$ \BY{contradiction} \pflabel{[and3]notsatisfy}
        \end{pfsteps*}
        Assume $\cmaysatisfy{e}{\cand{\xi_1}{\xi_2}}$. By rule induction over Rules (\ref{rules:MaySatisfy}) on it, the following cases apply.
        \begin{byCases}
        \savelocalsteps{2}
        \item[\text{(\ref{rule:CMSNotVal})}]
            \begin{pfsteps*}
            \item $\refutable{\cand{\xi_1}{\xi_2}}$ \BY{assumption}
            \end{pfsteps*}
            Contradicts \autoref{lem:no-and-refutable}.
        \restorelocalsteps{2}
        \item[\text{(\ref{rule:CMSAnd1})}]
            \begin{pfsteps*}
            \item $\cmaysatisfy{e}{\xi_1}$ \BY{assumption}
            \end{pfsteps*}
            Contradicts \pfref{[and3]notmaysat1}.
        \restorelocalsteps{2}
        \item[\text{(\ref{rule:CMSAnd2})}]
            \begin{pfsteps*}
            \item $\cmaysatisfy{e}{\xi_2}$ \BY{assumption}
            \end{pfsteps*}
            Contradicts \pfref{[and3]notmaysat2}.
        \restorelocalsteps{2}
        \item[\text{(\ref{rule:CMSAnd3})}]
            \begin{pfsteps*}
            \item $\cmaysatisfy{e}{\xi_1}$ \BY{assumption}
            \end{pfsteps*}
            Contradicts \pfref{[and3]notmaysat1}.
        \end{byCases}
        \begin{pfsteps*}
        \item $\cnotmaysatisfy{e}{\cand{\xi_1}{\xi_2}}$ \BY{contradiction} \pflabel{[and3]notmaysat}
        \item $\cnotsatisfyormay{e}{\cand{\xi_1}{\xi_2}}$ \BY{\autoref{lem:not-satormay} on \pfref{[and3]notsatisfy} and \pfref{[and3]notmaysat}}
        \end{pfsteps*}
    \restorelocalsteps{1}
    \item[\cmaysatisfy{e}{\xi_1},\csatisfy{e}{\xi_2}]
        \begin{pfsteps*}
        \item $\cnotsatisfy{e}{\xi_1}$ \BY{assumption} \pflabel{[and4]notsatisfy1}
        \item $\cmaysatisfy{e}{\xi_1}$ \BY{assumption} \pflabel{[and4]maysat1}
        \item $\csatisfy{e}{\xi_2}$ \BY{assumption} \pflabel{[and4]satisfy2}
        \item $\cnotmaysatisfy{e}{\xi_2}$ \BY{assumption} \pflabel{[and4]notmaysat2}
        \item $\cmaysatisfy{e}{\cand{\xi_1}{\xi_2}}$ \BY{Rule (\ref{rule:CMSAnd1}) on \pfref{[and4]maysat1} and \pfref{[and4]satisfy2}} \pflabel{[and4]maysat-and}
        \item $\csatisfyormay{e}{\cand{\xi_1}{\xi_2}}$ \BY{\ruleref{rule:CSMSMay} on \pfref{[and4]maysat-and}}
        \end{pfsteps*}
        Assume $\csatisfy{e}{\cand{\xi_1}{\xi_2}}$. By rule induction over Rules (\ref{rules:Satisfy}), only one case applies.
        \begin{byCases}
        \item[\text{(\ref{rule:CSAnd})}]
            \begin{pfsteps*}
            \item $\csatisfy{e}{\xi_1}$ \BY{assumption}
            \end{pfsteps*}
            Contradicts \pfref{[and4]notsatisfy1}.
        \end{byCases}
        \begin{pfsteps*}
        \item $\cnotsatisfy{e}{\cand{\xi_1}{\xi_2}}$ \BY{contradiction}
        \end{pfsteps*}
        
    \restorelocalsteps{1}
    \item[\cmaysatisfy{e}{\xi_1},\cmaysatisfy{e}{\xi_2}]
        \begin{pfsteps*}
        \item $\cnotsatisfy{e}{\xi_1}$ \BY{assumption} \pflabel{[and5]notsatisfy1}
        \item $\cmaysatisfy{e}{\xi_1}$ \BY{assumption} \pflabel{[and5]maysat1}
        \item $\cnotsatisfy{e}{\xi_2}$ \BY{assumption} \pflabel{[and5]notsatisfy2}
        \item $\cmaysatisfy{e}{\xi_2}$ \BY{assumption} \pflabel{[and5]maysat2}
        \item $\cmaysatisfy{e}{\cand{\xi_1}{\xi_2}}$ \BY{Rule (\ref{rule:CMSAnd3}) on \pfref{[and5]maysat1} and \pfref{[and5]maysat2}} \pflabel{[and5]maysat-and}
        \item $\csatisfyormay{e}{\cand{\xi_1}{\xi_2}}$ \BY{\ruleref{rule:CSMSMay} on \pfref{[and5]maysat-and}}
        \end{pfsteps*}
        Assume $\csatisfy{e}{\cand{\xi_1}{\xi_2}}$. By rule induction over Rules (\ref{rules:Satisfy}), only one case applies.
        \begin{byCases}
        \item[\text{(\ref{rule:CSAnd})}]
            \begin{pfsteps*}
            \item $\csatisfy{e}{\xi_1}$ \BY{assumption}
            \end{pfsteps*}
            Contradicts \pfref{[and5]notsatisfy1}.
        \end{byCases}
        \begin{pfsteps*}
        \item $\cnotsatisfy{e}{\cand{\xi_1}{\xi_2}}$ \BY{contradiction}
        \end{pfsteps*}
        
    \restorelocalsteps{1}
    \item[\cmaysatisfy{e}{\xi_1},\cnotsatisfyormay{e}{\xi_2}]
        \begin{pfsteps*}
        \item $\cnotsatisfy{e}{\xi_1}$ \BY{assumption} \pflabel{[and6]notsatisfy1}
        \item $\cmaysatisfy{e}{\xi_1}$ \BY{assumption} \pflabel{[and6]maysat1}
        \item $\cnotsatisfy{e}{\xi_2}$ \BY{assumption} \pflabel{[and6]notsatisfy2}
        \item $\cnotmaysatisfy{e}{\xi_2}$ \BY{assumption} \pflabel{[and6]notmaysat2}
        \end{pfsteps*}
        Assume $\csatisfy{e}{\cand{\xi_1}{\xi_2}}$. By rule induction over Rules (\ref{rules:Satisfy}), only one case applies.
        \begin{byCases}
        \item[\text{(\ref{rule:CSAnd})}]
            \begin{pfsteps*}
            \item $\csatisfy{e}{\xi_2}$ \BY{assumption}
            \end{pfsteps*}
            Contradicts \pfref{[and6]notsatisfy2}.
        \end{byCases}
        \begin{pfsteps*}
        \item $\cnotsatisfy{e}{\cand{\xi_1}{\xi_2}}$ \BY{contradiction} \pflabel{[and6]notsatisfy}
        \end{pfsteps*}
        Assume $\cmaysatisfy{e}{\cand{\xi_1}{\xi_2}}$. By rule induction over Rules (\ref{rules:MaySatisfy}) on it, the following cases apply.
        \begin{byCases}
        \savelocalsteps{2}
        \item[\text{(\ref{rule:CMSNotVal})}]
            \begin{pfsteps*}
            \item $\refutable{\cand{\xi_1}{\xi_2}}$ \BY{assumption}
            \end{pfsteps*}
            Contradicts \autoref{lem:no-and-refutable}.
        \restorelocalsteps{2}
        \item[\text{(\ref{rule:CMSAnd1})}]
            \begin{pfsteps*}
            \item $\csatisfy{e}{\xi_2}$ \BY{assumption}
            \end{pfsteps*}
            Contradicts \pfref{[and6]notsatisfy2}.
        \restorelocalsteps{2}
        \item[\text{(\ref{rule:CMSAnd2})}]
            \begin{pfsteps*}
            \item $\cmaysatisfy{e}{\xi_2}$ \BY{assumption}
            \end{pfsteps*}
            Contradicts \pfref{[and6]notmaysat2}.
        \restorelocalsteps{2}
        \item[\text{(\ref{rule:CMSAnd3})}]
            \begin{pfsteps*}
            \item $\cmaysatisfy{e}{\xi_2}$ \BY{assumption}
            \end{pfsteps*}
            Contradicts \pfref{[and6]notmaysat2}.
        \end{byCases}
        \begin{pfsteps*}
        \item $\cnotmaysatisfy{e}{\cand{\xi_1}{\xi_2}}$ \BY{contradiction} \pflabel{[and6]notmaysat}
        \item $\cnotsatisfyormay{e}{\cand{\xi_1}{\xi_2}}$ \BY{\autoref{lem:not-satormay} on \pfref{[and6]notsatisfy} and \pfref{[and3]notmaysat}}
        \end{pfsteps*}
    \restorelocalsteps{1}
    \item[\cnotsatisfyormay{e}{\xi_1},\csatisfy{e}{\xi_2}]
        \begin{pfsteps*}
        \item $\cnotsatisfy{e}{\xi_1}$ \BY{assumption} \pflabel{[and7]notsatisfy1}
        \item $\cnotmaysatisfy{e}{\xi_1}$ \BY{assumption} \pflabel{[and7]notmaysat1}
        \item $\csatisfy{e}{\xi_2}$ \BY{assumption} \pflabel{[and7]satisfy2}
        \item $\cnotmaysatisfy{e}{\xi_2}$ \BY{assumption} \pflabel{[and7]notmaysat2}
        \end{pfsteps*}
        Assume $\csatisfy{e}{\cand{\xi_1}{\xi_2}}$. By rule induction over Rules (\ref{rules:Satisfy}), only one case applies.
        \begin{byCases}
        \item[\text{(\ref{rule:CSAnd})}]
            \begin{pfsteps*}
            \item $\csatisfy{e}{\xi_1}$ \BY{assumption}
            \end{pfsteps*}
            Contradicts \pfref{[and7]notsatisfy1}.
        \end{byCases}
        \begin{pfsteps*}
        \item $\cnotsatisfy{e}{\cand{\xi_1}{\xi_2}}$ \BY{contradiction} \pflabel{[and7]notsatisfy}
        \end{pfsteps*}
        Assume $\cmaysatisfy{e}{\cand{\xi_1}{\xi_2}}$. By rule induction over Rules (\ref{rules:MaySatisfy}) on it, the following cases apply.
        \begin{byCases}
        \savelocalsteps{2}
        \item[\text{(\ref{rule:CMSNotVal})}]
            \begin{pfsteps*}
            \item $\refutable{\cand{\xi_1}{\xi_2}}$ \BY{assumption}
            \end{pfsteps*}
            Contradicts \autoref{lem:no-and-refutable}.
        \restorelocalsteps{2}
        \item[\text{(\ref{rule:CMSAnd1})}]
            \begin{pfsteps*}
            \item $\cmaysatisfy{e}{\xi_1}$ \BY{assumption}
            \end{pfsteps*}
            Contradicts \pfref{[and7]notmaysat1}.
        \restorelocalsteps{2}
        \item[\text{(\ref{rule:CMSAnd2})}]
            \begin{pfsteps*}
            \item $\cmaysatisfy{e}{\xi_2}$ \BY{assumption}
            \end{pfsteps*}
            Contradicts \pfref{[and7]notmaysat2}.
        \restorelocalsteps{2}
        \item[\text{(\ref{rule:CMSAnd3})}]
            \begin{pfsteps*}
            \item $\cmaysatisfy{e}{\xi_1}$ \BY{assumption}
            \end{pfsteps*}
            Contradicts \pfref{[and7]notmaysat1}.
        \end{byCases}
        \begin{pfsteps*}
        \item $\cnotmaysatisfy{e}{\cand{\xi_1}{\xi_2}}$ \BY{contradiction} \pflabel{[and7]notmaysat}
        \item $\cnotsatisfyormay{e}{\cand{\xi_1}{\xi_2}}$ \BY{\autoref{lem:not-satormay} on \pfref{[and7]notsatisfy} and \pfref{[and7]notmaysat}}
        \end{pfsteps*}
    \restorelocalsteps{1}
    \item[\cnotsatisfyormay{e}{\xi_1},\cmaysatisfy{e}{\xi_2}]
        \begin{pfsteps*}
        \item $\cnotsatisfy{e}{\xi_1}$ \BY{assumption} \pflabel{[and8]notsatisfy1}
        \item $\cnotmaysatisfy{e}{\xi_1}$ \BY{assumption} \pflabel{[and8]notmaysat1}
        \item $\cnotsatisfy{e}{\xi_2}$ \BY{assumption} \pflabel{[and8]notsatisfy2}
        \item $\cmaysatisfy{e}{\xi_2}$ \BY{assumption} \pflabel{[and8]maysat2}
        \end{pfsteps*}
        Assume $\csatisfy{e}{\cand{\xi_1}{\xi_2}}$. By rule induction over Rules (\ref{rules:Satisfy}), only one case applies.
        \begin{byCases}
        \item[\text{(\ref{rule:CSAnd})}]
            \begin{pfsteps*}
            \item $\csatisfy{e}{\xi_1}$ \BY{assumption}
            \end{pfsteps*}
            Contradicts \pfref{[and8]notsatisfy1}.
        \end{byCases}
        \begin{pfsteps*}
        \item $\cnotsatisfy{e}{\cand{\xi_1}{\xi_2}}$ \BY{contradiction} \pflabel{[and8]notsatisfy}
        \end{pfsteps*}
        Assume $\cmaysatisfy{e}{\cand{\xi_1}{\xi_2}}$. By rule induction over Rules (\ref{rules:MaySatisfy}) on it, the following cases apply.
        \begin{byCases}
        \savelocalsteps{2}
        \item[\text{(\ref{rule:CMSNotVal})}]
            \begin{pfsteps*}
            \item $\refutable{\cand{\xi_1}{\xi_2}}$ \BY{assumption}
            \end{pfsteps*}
            Contradicts \autoref{lem:no-and-refutable}.
        \restorelocalsteps{2}
        \item[\text{(\ref{rule:CMSAnd1})}]
            \begin{pfsteps*}
            \item $\cmaysatisfy{e}{\xi_1}$ \BY{assumption}
            \end{pfsteps*}
            Contradicts \pfref{[and8]notmaysat1}.
        \restorelocalsteps{2}
        \item[\text{(\ref{rule:CMSAnd2})}]
            \begin{pfsteps*}
            \item $\csatisfy{e}{\xi_1}$ \BY{assumption}
            \end{pfsteps*}
            Contradicts \pfref{[and8]notsatisfy1}.
        \restorelocalsteps{2}
        \item[\text{(\ref{rule:CMSAnd3})}]
            \begin{pfsteps*}
            \item $\cmaysatisfy{e}{\xi_1}$ \BY{assumption}
            \end{pfsteps*}
            Contradicts \pfref{[and8]notmaysat1}.
        \end{byCases}
        \begin{pfsteps*}
        \item $\cnotmaysatisfy{e}{\cand{\xi_1}{\xi_2}}$ \BY{contradiction} \pflabel{[and8]notmaysat}
        \item $\cnotsatisfyormay{e}{\cand{\xi_1}{\xi_2}}$ \BY{\autoref{lem:not-satormay} on \pfref{[and8]notsatisfy} and \pfref{[and8]notmaysat}}
        \end{pfsteps*}
    \restorelocalsteps{1}
    \item[\cnotsatisfyormay{e}{\xi_1},\cnotsatisfyormay{e}{\xi_2}]
        \begin{pfsteps*}
        \item $\cnotsatisfy{e}{\xi_1}$ \BY{assumption} \pflabel{[and9]notsatisfy1}
        \item $\cnotmaysatisfy{e}{\xi_1}$ \BY{assumption} \pflabel{[and9]notmaysat1}
        \item $\cnotsatisfy{e}{\xi_2}$ \BY{assumption} \pflabel{[and9]notsatisfy2}
        \item $\cnotmaysatisfy{e}{\xi_2}$ \BY{assumption} \pflabel{[and9]notmaysat2}
        \end{pfsteps*}
        Assume $\csatisfy{e}{\cand{\xi_1}{\xi_2}}$. By rule induction over Rules (\ref{rules:Satisfy}) on it, only one case apply.
        \begin{byCases}
        \item[\text{(\ref{rule:CSAnd})}]
            \begin{pfsteps*}
            \item $\csatisfy{e}{\xi_1}$ \BY{assumption}
            \end{pfsteps*}
            Contradicts \pfref{[and9]notsatisfy1}.
        \end{byCases}
        \begin{pfsteps*}
        \item $\cnotsatisfy{e}{\cand{\xi_1}{\xi_2}}$ \BY{contradiction} \pflabel{[and9]notsatisfy}
        \end{pfsteps*}
        Assume $\cmaysatisfy{e}{\cand{\xi_1}{\xi_2}}$. By rule induction over Rules (\ref{rules:MaySatisfy}) on it, the following cases apply.
        \begin{byCases}
        \savelocalsteps{2}
        \item[\text{(\ref{rule:CMSNotVal})}]
            \begin{pfsteps*}
            \item $\refutable{\cand{\xi_1}{\xi_2}}$ \BY{assumption}
            \end{pfsteps*}
            Contradicts \autoref{lem:no-and-refutable}.
        \restorelocalsteps{2}
        \item[\text{(\ref{rule:CMSAnd1})}]
            \begin{pfsteps*}
            \item $\cmaysatisfy{e}{\xi_1}$ \BY{assumption}
            \end{pfsteps*}
            Contradicts \pfref{[and9]notmaysat1}.
        \restorelocalsteps{2}
        \item[\text{(\ref{rule:CMSAnd2})}]
            \begin{pfsteps*}
            \item $\cmaysatisfy{e}{\xi_2}$ \BY{assumption}
            \end{pfsteps*}
            Contradicts \pfref{[and9]notmaysat2}.
        \restorelocalsteps{2}
        \item[\text{(\ref{rule:CMSAnd3})}]
            \begin{pfsteps*}
            \item $\cmaysatisfy{e}{\xi_1}$ \BY{assumption}
            \end{pfsteps*}
            Contradicts \pfref{[and9]notmaysat1}.
        \end{byCases}
        \begin{pfsteps*}
        \item $\cnotmaysatisfy{e}{\cand{\xi_1}{\xi_2}}$ \BY{contradiction} \pflabel{[and9]notmaysat}
        \item $\cnotsatisfyormay{e}{\cand{\xi_1}{\xi_2}}$ \BY{\autoref{lem:not-satormay} on \pfref{[and9]notsatisfy} and \pfref{[and9]notmaysat}}
        \end{pfsteps*}
    \end{byCases}
    
\restorelocalsteps{0}
\item[\text{(\ref{rule:CTOr})}]
    \begin{pfsteps*}
    \item $\xi=\cor{\xi_1}{\xi_2}$ \BY{assumption}
    \end{pfsteps*}
    By inductive hypothesis on \pfref{eTyp} and \pfref{eFinal}, exactly one of $\csatisfy{e}{\xi_1}$, $\cmaysatisfy{e}{\xi_1}$, and $\cnotsatisfyormay{e}{\xi_1}$ holds. The same goes for $\xi_2$. By case analysis on which conclusion holds for $\xi_1$ and $\xi_2$.
    \begin{byCases}
    \savelocalsteps{1}
    \item[\csatisfy{e}{\xi_1},\csatisfy{e}{\xi_2}]
        \begin{pfsteps*}
        \item $\csatisfy{e}{\xi_1}$ \BY{assumption} \pflabel{[or1]satisfy1}
        \item $\cnotmaysatisfy{e}{\xi_1}$ \BY{assumption} \pflabel{[or1]notmaysat1}
        \item $\csatisfy{e}{\xi_2}$ \BY{assumption} \pflabel{[or1]satisfy2}
        \item $\cnotmaysatisfy{e}{\xi_2}$ \BY{assumption} \pflabel{[or1]notmaysat2}
        \item $\csatisfy{e}{\cor{\xi_1}{\xi_2}}$ \BY{Rule (\ref{rule:CSOr1}) on \pfref{[or1]satisfy1}} \pflabel{[or1]satisfy-or}
        \item $\csatisfyormay{e}{\cor{\xi_1}{\xi_2}}$ \BY{\ruleref{rule:CSMSSat} on \pfref{[or1]satisfy-or}}
        \end{pfsteps*}
        Assume $\cmaysatisfy{e}{\cor{\xi_1}{\xi_2}}$. By rule induction over Rules (\ref{rules:MaySatisfy}) on it, the following cases apply.
        \begin{byCases}
        \savelocalsteps{2}
        \item[\text{(\ref{rule:CMSNotVal})}]
            \begin{pfsteps*}
            \item $\refutable{\cor{\xi_1}{\xi_2}}$ \BY{assumption}
            \end{pfsteps*}
            Contradicts \autoref{lem:no-or-refutable}.
        \restorelocalsteps{2}
        \item[\text{(\ref{rule:CMSOr1})}]
            \begin{pfsteps*}
            \item $\cmaysatisfy{e}{\xi_1}$ \BY{assumption}
            \end{pfsteps*}
            Contradicts \pfref{[or1]notmaysat1}.
        \restorelocalsteps{2}
        \item[\text{(\ref{rule:CMSOr2})}]
            \begin{pfsteps*}
            \item $\cmaysatisfy{e}{\xi_2}$ \BY{assumption}
            \end{pfsteps*}
            Contradicts \pfref{[or1]notmaysat2}.
        \end{byCases}
        \begin{pfsteps*}
        \item $\cnotmaysatisfy{e}{\cor{\xi_1}{\xi_2}}$ \BY{contradiction}
        \end{pfsteps*}
        
    \restorelocalsteps{1}
    \item[\csatisfy{e}{\xi_1},\cmaysatisfy{e}{\xi_2}]
        \begin{pfsteps*}
        \item $\csatisfy{e}{\xi_1}$ \BY{assumption} \pflabel{[or2]satisfy1}
        \item $\cnotmaysatisfy{e}{\xi_1}$ \BY{assumption} \pflabel{[or2]notmaysat1}
        \item $\cnotsatisfy{e}{\xi_2}$ \BY{assumption} \pflabel{[or2]notsatisfy2}
        \item $\cmaysatisfy{e}{\xi_2}$ \BY{assumption} \pflabel{[or2]maysat2}
        \item $\csatisfy{e}{\cor{\xi_1}{\xi_2}}$ \BY{Rule (\ref{rule:CSOr1}) on \pfref{[or2]satisfy1}} \pflabel{[or2]satisfy-or}
        \item $\csatisfyormay{e}{\cor{\xi_1}{\xi_2}}$ \BY{\ruleref{rule:CSMSSat} on \pfref{[or2]satisfy-or}}
        \end{pfsteps*}
        Assume $\cmaysatisfy{e}{\cor{\xi_1}{\xi_2}}$. By rule induction over Rules (\ref{rules:MaySatisfy}) on it, the following cases apply.
        \begin{byCases}
        \savelocalsteps{2}
        \item[\text{(\ref{rule:CMSNotVal})}]
            \begin{pfsteps*}
            \item $\refutable{\cor{\xi_1}{\xi_2}}$ \BY{assumption}
            \end{pfsteps*}
            Contradicts \autoref{lem:no-or-refutable}.
        \restorelocalsteps{2}
        \item[\text{(\ref{rule:CMSOr1})}]
            \begin{pfsteps*}
            \item $\cmaysatisfy{e}{\xi_1}$ \BY{assumption}
            \end{pfsteps*}
            Contradicts \pfref{[or2]notmaysat1}.
        \restorelocalsteps{2}
        \item[\text{(\ref{rule:CMSOr2})}]
            \begin{pfsteps*}
            \item $\cnotsatisfy{e}{\xi_1}$ \BY{assumption}
            \end{pfsteps*}
            Contradicts \pfref{[or2]satisfy1}.
        \end{byCases}
        \begin{pfsteps*}
        \item $\cnotmaysatisfy{e}{\cor{\xi_1}{\xi_2}}$ \BY{contradiction}
        \end{pfsteps*}
    \restorelocalsteps{1}
    \item[\csatisfy{e}{\xi_1},\cnotsatisfyormay{e}{\xi_2}]
        \begin{pfsteps*}
        \item $\csatisfy{e}{\xi_1}$ \BY{assumption} \pflabel{[or3]satisfy1}
        \item $\cnotmaysatisfy{e}{\xi_1}$ \BY{assumption} \pflabel{[or3]notmaysat1}
        \item $\cnotsatisfy{e}{\xi_2}$ \BY{assumption} \pflabel{[or3]notsatisfy2}
        \item $\cnotmaysatisfy{e}{\xi_2}$ \BY{assumption} \pflabel{[or3]notmaysat2}
        \item $\csatisfy{e}{\cor{\xi_1}{\xi_2}}$ \BY{Rule (\ref{rule:CSOr1}) on \pfref{[or3]satisfy1}} \pflabel{[or3]satisfy-or}
        \item $\csatisfyormay{e}{\cor{\xi_1}{\xi_2}}$ \BY{\ruleref{rule:CSMSSat} on \pfref{[or3]satisfy-or}}
        \end{pfsteps*}
        Assume $\cmaysatisfy{e}{\cor{\xi_1}{\xi_2}}$. By rule induction over Rules (\ref{rules:MaySatisfy}) on it, the following cases apply.
        \begin{byCases}
        \savelocalsteps{2}
        \item[\text{(\ref{rule:CMSNotVal})}]
            \begin{pfsteps*}
            \item $\refutable{\cor{\xi_1}{\xi_2}}$ \BY{assumption}
            \end{pfsteps*}
            Contradicts \autoref{lem:no-or-refutable}.
        \restorelocalsteps{2}
        \item[\text{(\ref{rule:CMSOr1})}]
            \begin{pfsteps*}
            \item $\cmaysatisfy{e}{\xi_1}$ \BY{assumption}
            \end{pfsteps*}
            Contradicts \pfref{[or3]notmaysat1}.
        \restorelocalsteps{2}
        \item[\text{(\ref{rule:CMSOr2})}]
            \begin{pfsteps*}
            \item $\cnotsatisfy{e}{\xi_1}$ \BY{assumption}
            \end{pfsteps*}
            Contradicts \pfref{[or3]satisfy1}.
        \end{byCases}
        \begin{pfsteps*}
        \item $\cnotmaysatisfy{e}{\cor{\xi_1}{\xi_2}}$ \BY{contradiction}
        \end{pfsteps*}
    \restorelocalsteps{1}
    \item[\cmaysatisfy{e}{\xi_1},\csatisfy{e}{\xi_2}]
        \begin{pfsteps*}
        \item $\cnotsatisfy{e}{\xi_1}$ \BY{assumption} \pflabel{[or4]notsatisfy1}
        \item $\cmaysatisfy{e}{\xi_1}$ \BY{assumption} \pflabel{[or4]maysat1}
        \item $\csatisfy{e}{\xi_2}$ \BY{assumption} \pflabel{[or4]satisfy2}
        \item $\cnotmaysatisfy{e}{\xi_2}$ \BY{assumption} \pflabel{[or4]notmaysat2}
        \item $\csatisfy{e}{\cor{\xi_1}{\xi_2}}$ \BY{Rule (\ref{rule:CSOr2}) on \pfref{[or4]satisfy2}} \pflabel{[or4]satisfy-or}
        \item $\csatisfyormay{e}{\cor{\xi_1}{\xi_2}}$ \BY{\ruleref{rule:CSMSSat} on \pfref{[or4]satisfy-or}}
        \end{pfsteps*}
        Assume $\cmaysatisfy{e}{\cor{\xi_1}{\xi_2}}$. By rule induction over Rules (\ref{rules:MaySatisfy}) on it, the following cases apply.
        \begin{byCases}
        \savelocalsteps{2}
        \item[\text{(\ref{rule:CMSNotVal})}]
            \begin{pfsteps*}
            \item $\refutable{\cor{\xi_1}{\xi_2}}$ \BY{assumption}
            \end{pfsteps*}
            Contradicts \autoref{lem:no-or-refutable}.
        \restorelocalsteps{2}
        \item[\text{(\ref{rule:CMSOr1})}]
            \begin{pfsteps*}
            \item $\cnotsatisfy{e}{\xi_2}$ \BY{assumption}
            \end{pfsteps*}
            Contradicts \pfref{[or4]satisfy2}.
        \restorelocalsteps{2}
        \item[\text{(\ref{rule:CMSOr2})}]
            \begin{pfsteps*}
            \item $\cmaysatisfy{e}{\xi_2}$ \BY{assumption}
            \end{pfsteps*}
            Contradicts \pfref{[or4]notmaysat2}.
        \end{byCases}
        \begin{pfsteps*}
        \item $\cnotmaysatisfy{e}{\cor{\xi_1}{\xi_2}}$ \BY{contradiction}
        \end{pfsteps*}
        
    \restorelocalsteps{1}
    \item[\cmaysatisfy{e}{\xi_1},\cmaysatisfy{e}{\xi_2}]
        \begin{pfsteps*}
        \item $\cnotsatisfy{e}{\xi_1}$ \BY{assumption} \pflabel{[or5]notsatisfy1}
        \item $\cmaysatisfy{e}{\xi_1}$ \BY{assumption} \pflabel{[or5]maysat1}
        \item $\cnotsatisfy{e}{\xi_2}$ \BY{assumption} \pflabel{[or5]notsatisfy2}
        \item $\cmaysatisfy{e}{\xi_2}$ \BY{assumption} \pflabel{[or5]maysat2}
        \item $\cmaysatisfy{e}{\cor{\xi_1}{\xi_2}}$ \BY{Rule (\ref{rule:CMSOr1}) on \pfref{[or5]maysat1} and \pfref{[or5]notsatisfy2}} \pflabel{[or5]maysat-or}
        \item $\csatisfyormay{e}{\cor{\xi_1}{\xi_2}}$ \BY{\ruleref{rule:CSMSMay} on \pfref{[or5]maysat-or}}
        \end{pfsteps*}
        Assume $\csatisfy{e}{\cor{\xi_1}{\xi_2}}$. By rule induction over Rules (\ref{rules:Satisfy}), only two cases apply.
        \begin{byCases}
        \savelocalsteps{2}
        \item[\text{(\ref{rule:CSOr1})}]
            \begin{pfsteps*}
            \item $\csatisfy{e}{\xi_1}$ \BY{assumption}
            \end{pfsteps*}
            Contradicts $\pfref{[or5]notsatisfy1}$
        \restorelocalsteps{2}
        \item[\text{(\ref{rule:CSOr2})}]
            \begin{pfsteps*}
            \item $\csatisfy{e}{\xi_2}$ \BY{assumption}
            \end{pfsteps*}
            Contradicts $\pfref{[or5]notsatisfy2}$
        \end{byCases}
        \begin{pfsteps*}
        \item $\cnotsatisfy{e}{\cor{\xi_1}{\xi_2}}$ \BY{contradiction}
        \end{pfsteps*}
    \restorelocalsteps{1}
    \item[\cmaysatisfy{e}{\xi_1},\cnotsatisfyormay{e}{\xi_2}]
        \begin{pfsteps*}
        \item $\cnotsatisfy{e}{\xi_1}$ \BY{assumption} \pflabel{[or6]notsatisfy1}
        \item $\cmaysatisfy{e}{\xi_1}$ \BY{assumption} \pflabel{[or6]maysat1}
        \item $\cnotsatisfy{e}{\xi_2}$ \BY{assumption} \pflabel{[or6]notsatisfy2}
        \item $\cnotmaysatisfy{e}{\xi_2}$ \BY{assumption} \pflabel{[or6]notmaysat2}
        \item $\cmaysatisfy{e}{\cor{\xi_1}{\xi_2}}$ \BY{Rule (\ref{rule:CMSOr1}) on \pfref{[or6]maysat1} and \pfref{[or6]notsatisfy2}} \pflabel{[or6]maysat-or}
        \item $\csatisfyormay{e}{\cor{\xi_1}{\xi_2}}$ \BY{\ruleref{rule:CSMSMay} on \pfref{[or6]maysat-or}}
        \end{pfsteps*}
        Assume $\csatisfy{e}{\cor{\xi_1}{\xi_2}}$. By rule induction over Rules (\ref{rules:Satisfy}), only two cases apply.
        \begin{byCases}
        \savelocalsteps{2}
        \item[\text{(\ref{rule:CSOr1})}]
            \begin{pfsteps*}
            \item $\csatisfy{e}{\xi_1}$ \BY{assumption}
            \end{pfsteps*}
            Contradicts \pfref{[or6]notsatisfy1}.
        \restorelocalsteps{2}
        \item[\text{(\ref{rule:CSOr2})}]
            \begin{pfsteps*}
            \item $\csatisfy{e}{\xi_2}$ \BY{assumption}
            \end{pfsteps*}
            Contradicts \pfref{[or6]notsatisfy2}.
        \end{byCases}
        \begin{pfsteps*}
        \item $\cnotsatisfy{e}{\cor{\xi_1}{\xi_2}}$ \BY{contradiction}
        \end{pfsteps*}
    \restorelocalsteps{1}
    \item[\cnotsatisfyormay{e}{\xi_1},\csatisfy{e}{\xi_2}]
        \begin{pfsteps*}
        \item $\cnotsatisfy{e}{\xi_1}$ \BY{assumption} \pflabel{[or7]notsatisfy1}
        \item $\cnotmaysatisfy{e}{\xi_1}$ \BY{assumption} \pflabel{[or7]notmaysat1}
        \item $\csatisfy{e}{\xi_2}$ \BY{assumption} \pflabel{[or7]satisfy2}
        \item $\cnotmaysatisfy{e}{\xi_2}$ \BY{assumption} \pflabel{[or7]notmaysat2}
        \item $\csatisfy{e}{\cor{\xi_1}{\xi_2}}$ \BY{Rule (\ref{rule:CSOr2}) on \pfref{[or7]satisfy2}} \pflabel{[or7]satisfy-or}
        \item $\csatisfyormay{e}{\cor{\xi_1}{\xi_2}}$ \BY{\ruleref{rule:CSMSSat} on \pfref{[or7]satisfy-or}}
        \end{pfsteps*}
        Assume $\cmaysatisfy{e}{\cor{\xi_1}{\xi_2}}$. By rule induction over Rules (\ref{rules:MaySatisfy}) on it, the following cases apply.
        \begin{byCases}
        \savelocalsteps{2}
        \item[\text{(\ref{rule:CMSNotVal})}]
            \begin{pfsteps*}
            \item $\refutable{\cor{\xi_1}{\xi_2}}$ \BY{assumption}
            \end{pfsteps*}
            Contradicts \autoref{lem:no-or-refutable}.
        \restorelocalsteps{2}
        \item[\text{(\ref{rule:CMSOr1})}]
            \begin{pfsteps*}
            \item $\cnotsatisfy{e}{\xi_2}$ \BY{assumption}
            \end{pfsteps*}
            Contradicts \pfref{[or7]satisfy2}.
        \restorelocalsteps{2}
        \item[\text{(\ref{rule:CMSOr2})}]
            \begin{pfsteps*}
            \item $\cmaysatisfy{e}{\xi_2}$ \BY{assumption}
            \end{pfsteps*}
            Contradicts \pfref{[or7]notmaysat2}.
        \end{byCases}
        \begin{pfsteps*}
        \item $\cnotmaysatisfy{e}{\cor{\xi_1}{\xi_2}}$ \BY{contradiction}
        \end{pfsteps*}
        
    \restorelocalsteps{1}
    \item[\cnotsatisfyormay{e}{\xi_1},\cmaysatisfy{e}{\xi_2}]
        \begin{pfsteps*}
        \item $\cnotsatisfy{e}{\xi_1}$ \BY{assumption} \pflabel{[or8]notsatisfy1}
        \item $\cnotmaysatisfy{e}{\xi_1}$ \BY{assumption} \pflabel{[or8]notmaysat1}
        \item $\cnotsatisfy{e}{\xi_2}$ \BY{assumption} \pflabel{[or8]notsatisfy2}
        \item $\cmaysatisfy{e}{\xi_2}$ \BY{assumption} \pflabel{[or8]maysat2}
        \item $\cmaysatisfy{e}{\cor{\xi_1}{\xi_2}}$ \BY{Rule (\ref{rule:CMSOr2}) on \pfref{[or8]maysat2} and \pfref{[or8]notsatisfy1}} \pflabel{[or8]maysat-or}
        \item $\csatisfyormay{e}{\cor{\xi_1}{\xi_2}}$ \BY{\ruleref{rule:CSMSMay} on \pfref{[or8]maysat-or}}
        \end{pfsteps*}
        Assume $\csatisfy{e}{\cor{\xi_1}{\xi_2}}$. By rule induction over Rules (\ref{rules:Satisfy}), only two cases apply.
        \begin{byCases}
        \savelocalsteps{2}
        \item[\text{(\ref{rule:CSOr1})}]
            \begin{pfsteps*}
            \item $\csatisfy{e}{\xi_1}$ \BY{assumption}
            \end{pfsteps*}
            Contradicts $\pfref{[or8]notsatisfy1}$
        \restorelocalsteps{2}
        \item[\text{(\ref{rule:CSOr2})}]
            \begin{pfsteps*}
            \item $\csatisfy{e}{\xi_2}$ \BY{assumption}
            \end{pfsteps*}
            Contradicts $\pfref{[or8]notsatisfy2}$
        \end{byCases}
        \begin{pfsteps*}
        \item $\cnotsatisfy{e}{\cor{\xi_1}{\xi_2}}$ \BY{contradiction}
        \end{pfsteps*}
    \restorelocalsteps{1}
    \item[\cnotsatisfyormay{e}{\xi_1},\cnotsatisfyormay{e}{\xi_2}]
        \begin{pfsteps*}
        \item $\cnotsatisfy{e}{\xi_1}$ \BY{assumption} \pflabel{[or9]notsatisfy1}
        \item $\cnotmaysatisfy{e}{\xi_1}$ \BY{assumption} \pflabel{[or9]notmaysat1}
        \item $\cnotsatisfy{e}{\xi_2}$ \BY{assumption} \pflabel{[or9]notsatisfy2}
        \item $\cnotmaysatisfy{e}{\xi_2}$ \BY{assumption} \pflabel{[or9]notmaysat2}
        \end{pfsteps*}
        Assume $\csatisfy{e}{\cor{\xi_1}{\xi_2}}$. By rule induction over Rules (\ref{rules:Satisfy}) on it, only two cases apply.
        \begin{byCases}
        \savelocalsteps{2}
        \item[\text{(\ref{rule:CSOr1})}]
            \begin{pfsteps*}
            \item $\csatisfy{e}{\xi_1}$ \BY{assumption}
            \end{pfsteps*}
            Contradicts \pfref{[or9]notsatisfy1}.
        \restorelocalsteps{2}
        \item[\text{(\ref{rule:CSOr2})}]
            \begin{pfsteps*}
            \item $\csatisfy{e}{\xi_2}$ \BY{assumption}
            \end{pfsteps*}
            Contradicts \pfref{[or9]notsatisfy2}.
        \end{byCases}
        \begin{pfsteps*}
        \item $\cnotsatisfy{e}{\cor{\xi_1}{\xi_2}}$ \BY{contradiction} \pflabel{[or9]notsatisfy}
        \end{pfsteps*}
        Assume $\cmaysatisfy{e}{\cor{\xi_1}{\xi_2}}$. By rule induction over Rules (\ref{rules:MaySatisfy}) on it, the following cases apply.
        \begin{byCases}
        \savelocalsteps{2}
        \item[\text{(\ref{rule:CMSNotVal})}]
            \begin{pfsteps*}
            \item $\refutable{\cor{\xi_1}{\xi_2}}$ \BY{assumption}
            \end{pfsteps*}
            Contradicts \autoref{lem:no-or-refutable}.
        \restorelocalsteps{2}
        \item[\text{(\ref{rule:CMSOr1})}]
            \begin{pfsteps*}
            \item $\cmaysatisfy{e}{\xi_1}$ \BY{assumption}
            \end{pfsteps*}
            Contradicts \pfref{[or9]notmaysat1}.
        \restorelocalsteps{2}
        \item[\text{(\ref{rule:CMSOr2})}]
            \begin{pfsteps*}
            \item $\cmaysatisfy{e}{\xi_2}$ \BY{assumption}
            \end{pfsteps*}
            Contradicts \pfref{[or9]notmaysat2}.
        \end{byCases}
        \begin{pfsteps*}
        \item $\cnotmaysatisfy{e}{\cor{\xi_1}{\xi_2}}$ \BY{contradiction} \pflabel{[or9]notmaysat}
        \item $\cnotsatisfyormay{e}{\cor{\xi_1}{\xi_2}}$ \BY{\autoref{lem:not-satormay} on \pfref{[or9]notsatisfy} and \pfref{[or9]notmaysat}}
        \end{pfsteps*}
        
        
    \end{byCases}
\restorelocalsteps{0}
\item[\text{(\ref{rule:CTInl})}]
    \begin{pfsteps*}
    \item $\xi=\cinl{\xi_1}$ \BY{assumption}
    \item $\tau=\tsum{\tau_1}{\tau_2}$ \BY{assumption}
    \item $\ctyp{\xi_1}{\tau_1}$ \BY{assumption} \pflabel{c1Typ}
    \end{pfsteps*}
    By rule induction over Rules (\ref{rules:TExp}) on \pfref{eTyp}, the following cases apply.
    \begin{byCases}
    \savelocalsteps{1}
    \item[\text{(\ref{rule:TEHole}),(\ref{rule:THole}),(\ref{rule:TAp}),(\ref{rule:TPrl}),(\ref{rule:TPrr}),(\ref{rule:TMatchZPre}),(\ref{rule:TMatchNZPre})}]
        \begin{pfsteps*}
        \item $e=\hehole{u},\hhole{e_0}{u},\hap{e_1}{e_2},\hprl{e_0},\hprr{e_0},\hmatch{e_0}{\zrules}$ \BY{assumption}
        \item $\isntVal{e}$ \BY{Rule (\ref{rule:NVEHole}),(\ref{rule:NVHole}),(\ref{rule:NVAp}),(\ref{rule:NVMatch}),(\ref{rule:NVPrl}),(\ref{rule:NVPrr})} \pflabel{[inl]notval}
        \end{pfsteps*}
        Assume $\csatisfy{e}{\cinl{\xi_1}}$. By rule induction over Rules (\ref{rules:Satisfy}) on it, no case applies due to syntactic contradiction.
        \begin{pfsteps*}
        \item $\cnotsatisfy{e}{\cinl{\xi_1}}$ \BY{contradiction} \pflabel{[inl]notsat-inl}
        \end{pfsteps*}
        By case analysis on the value of $\frefutable{\cinl{\xi_1}}$.
        \begin{byCases}
        \savelocalsteps{2}
        \item[\frefutable{\cinl{\xi_1}}=\true]
            \begin{pfsteps*}
            \item $\frefutable{\cinl{\xi_1}}=\true$ \BY{assumption} \pflabel{[inl]frft-true}
            \item $\refutable{\cinl{\xi_1}}$ \BY{\autoref{lem:sound-complete-xi-refutable} on \pfref{[inl]frft-true}} \pflabel{[inl]rft-true}
            \item $\cmaysatisfy{e}{\cinl{\xi_1}}$ \BY{\ruleref{rule:CMSNotVal} on \pfref{[inl]notval} and \pfref{[inl]rft-true}} \pflabel{[inl]maysat}
            \item $\csatisfyormay{e}{\cinl{\xi_1}}$ \BY{\ruleref{rule:CSMSMay} on \pfref{[inl]maysat}}
            \end{pfsteps*}
        \restorelocalsteps{2}
        \item[\frefutable{\cinl{\xi_1}}=\false]
            \begin{pfsteps*}
            \item $\frefutable{\cinl{\xi_1}}=\false$ \BY{assumption} \pflabel{[inl]frft-false}
            \item $\cancel{\refutable{\cinl{\xi_1}}}$ \BY{\autoref{lem:sound-complete-xi-refutable} on \pfref{[inl]frft-false}} \pflabel{[inl]rft-false}
            \end{pfsteps*}
            Assume $\cmaysatisfy{e}{\cinl{\xi_1}}$. By rule induction over \rulesref{rules:MaySatisfy} on it, only one case applies.
            \begin{byCases}
            \item[\text{(\ref{rule:CMSNotVal})}]
                \begin{pfsteps*}
                \item $\refutable{\cinl{\xi_1}}$ \BY{assumption}
                \end{pfsteps*}
                Contradicts \pfref{[inl]rft-false}.
            \end{byCases}
            \begin{pfsteps*}
            \item $\cnotmaysatisfy{e}{\cinl{\xi_1}}$ \BY{contradiction} \pflabel{[inl]notmaysat-inl}
            \item $\cnotsatisfyormay{e}{\cinl{\xi_1}}$ \BY{\autoref{lem:not-satormay} on \pfref{[inl]notsat-inl} and \pfref{[inl]notmaysat-inl}}
            \end{pfsteps*}
        \end{byCases}
    \restorelocalsteps{1}
    \item[\text{(\ref{rule:TInl})}]
        \begin{pfsteps*}
        \item $e=\hinl{\tau_2}{e_1}$ \BY{assumption}
        \item $\hexptyp{\cdot}{\Delta}{e_1}{\tau_1}$ \BY{assumption} \pflabel{e1Typ}
        \item $\isFinal{e_1}$ \BY{\autoref{lem:inl-final} on \pfref{eFinal}} \pflabel{e1Final}
        \end{pfsteps*}
        By inductive hypothesis on \pfref{c1Typ} and \pfref{e1Typ} and \pfref{e1Final}, exactly one of $\csatisfy{e_1}{\xi_1}$, $\cmaysatisfy{e_1}{\xi_1}$, and $\cnotsatisfyormay{e_1}{\xi_1}$ holds. By case analysis on which one holds.
        \begin{byCases}
        \savelocalsteps{2}
        \item[\csatisfy{e_1}{\xi_1}]
            \begin{pfsteps*}
            \item $\csatisfy{e_1}{\xi_1}$ \BY{assumption} \pflabel{[inl1]satisfy1}
            \item $\cnotmaysatisfy{e_1}{\xi_1}$ \BY{assumption} \pflabel{[inl1]notmaysat1}
            \item $\csatisfy{\hinl{\tau_2}{e_1}}{\cinl{\xi_1}}$ \BY{Rule (\ref{rule:CSInl}) on \pfref{[inl1]satisfy1}} \pflabel{[inl1]satisfy}
            \item $\csatisfyormay{\hinl{\tau_2}{e_1}}{\cinl{\xi_1}}$ \BY{\ruleref{rule:CSMSSat} on \pfref{[inl1]satisfy}}
            \end{pfsteps*}
            Assume $\cmaysatisfy{\hinl{\tau_2}{e_1}}{\cinl{\xi_1}}$. By rule induction over Rules (\ref{rules:MaySatisfy}) on it, only two cases apply.
            \begin{byCases}
            \savelocalsteps{3}
            \item[\text{(\ref{rule:CMSNotVal})}]
                \begin{pfsteps*}
                \item $\isntVal{\hinl{\tau_2}{e_1}}$ \BY{assumption} \pflabel{[inl1]notval-inl}
                \end{pfsteps*}
                By rule induction over Rules (\ref{rules:notval}) on \pfref{[inl1]notval-inl}, no case applies due to syntactic contradiction.
            \restorelocalsteps{3}
            \item[\text{(\ref{rule:CMSInl})}]
                \begin{pfsteps*}
                \item $\cmaysatisfy{e_1}{\xi_1}$
                \end{pfsteps*}
                Contradicts \pfref{[inl1]notmaysat1}.
            \end{byCases}
            \begin{pfsteps*}
            \item $\cnotmaysatisfy{\hinl{\tau_2}{e_1}}{\cinl{\xi_1}}$ \BY{contradiction}
            \end{pfsteps*}
            
        \restorelocalsteps{2}
        \item[\cmaysatisfy{e_1}{\xi_1}]
            \begin{pfsteps*}
            \item $\cnotsatisfy{e_1}{\xi_1}$ \BY{assumption} \pflabel{[inl2]notsatisfy1}
            \item $\cmaysatisfy{e_1}{\xi_1}$ \BY{assumption} \pflabel{[inl2]maysat1}
            \item $\cmaysatisfy{\hinl{\tau_2}{e_1}}{\cinl{\xi_1}}$ \BY{Rule (\ref{rule:CMSInl}) on \pfref{[inl2]maysat1}} \pflabel{[inl2]maysat}
            \item $\csatisfyormay{\hinl{\tau_2}{e_1}}{\cinl{\xi_1}}$ \BY{\ruleref{rule:CSMSMay} on \pfref{[inl2]maysat}}
            \end{pfsteps*}
            Assume $\csatisfy{\hinl{\tau_2}{e_1}}{\cinl{\xi_1}}$. By rule induction over Rules (\ref{rules:Satisfy}) on it, only one case applies.
            \begin{byCases}
            \item[\text{(\ref{rule:CSInl})}]
                \begin{pfsteps*}
                \item $\csatisfy{e_1}{\xi_1}$
                \end{pfsteps*}
                Contradicts \pfref{[inl2]notsatisfy1}.
            \end{byCases}
            \begin{pfsteps*}
            \item $\cnotsatisfy{\hinl{\tau_2}{e_1}}{\cinl{\xi_1}}$ \BY{contradiction}
            \end{pfsteps*}
           
        \restorelocalsteps{2}
        \item[\cnotsatisfyormay{e_1}{\xi_1}]
            \begin{pfsteps*}
            \item $\cnotsatisfy{e_1}{\xi_1}$ \BY{assumption} \pflabel{[inl3]notsatisfy1}
            \item $\cnotmaysatisfy{e_1}{\xi_1}$ \BY{assumption} \pflabel{[inl3]notmaysat1}
            \end{pfsteps*}
            Assume $\csatisfy{\hinl{\tau_2}{e_1}}{\cinl{\xi_1}}$. By rule induction over Rules (\ref{rules:Satisfy}) on it, only one case applies.
            \begin{byCases}
            \item[\text{(\ref{rule:CSInl})}]
                \begin{pfsteps*}
                \item $\csatisfy{e_1}{\xi_1}$
                \end{pfsteps*}
                Contradicts \pfref{[inl3]notsatisfy1}.
            \end{byCases}
            \begin{pfsteps*}
            \item $\cnotsatisfy{\hinl{\tau_2}{e_1}}{\cinl{\xi_1}}$ \BY{contradiction} \pflabel{[inl3]notsatisfy}
            \end{pfsteps*}
            Assume $\cmaysatisfy{\hinl{\tau_2}{e_1}}{\cinl{\xi_1}}$. By rule induction over Rules (\ref{rules:MaySatisfy}) on it, only one case applies.
            \begin{byCases}
            \savelocalsteps{3}
            \item[\text{(\ref{rule:CMSNotVal})}]
                \begin{pfsteps*}
                \item $\isntVal{\hinl{\tau_2}{e_1}}$ \BY{assumption} \pflabel{[inl3]notval-inl}
                \end{pfsteps*}
                By rule induction over Rules (\ref{rules:notval}) on \pfref{[inl3]notval-inl}, no case applies due to syntactic contradiction.
            \restorelocalsteps{3}
            \item[\text{(\ref{rule:CMSInl})}]
                \begin{pfsteps*}
                \item $\cmaysatisfy{e_1}{\xi_1}$
                \end{pfsteps*}
                Contradicts \pfref{[inl3]notmaysat1}.
            \end{byCases}
            \begin{pfsteps*}
            \item $\cnotmaysatisfy{\hinl{\tau_2}{e_1}}{\cinl{\xi_1}}$ \BY{contradiction} \pflabel{[inl3]notmaysatisfy}
            \item $\cnotsatisfyormay{\hinl{\tau_2}{e_1}}{\cinl{\xi_1}}$ \BY{\autoref{lem:not-satormay} on \pfref{[inl3]notsatisfy} and \pfref{[inl3]notmaysatisfy}}
            \end{pfsteps*}
        \end{byCases}
    \restorelocalsteps{1}
    \item[\text{(\ref{rule:TInr})}]
        \begin{pfsteps*}
        \item $e=\hinr{\tau_1}{e_2}$ \BY{assumption}
        \end{pfsteps*}
        Assume $\csatisfy{\hinr{\tau_1}{e_2}}{\cinl{\xi_1}}$. By rule induction over Rules (\ref{rules:Satisfy}) on it, no case applies due to syntactic contradiction.
        \begin{pfsteps*}
        \item $\cnotsatisfy{\hinr{\tau_1}{e_2}}{\cinl{\xi_1}}$ \BY{contradiction} \pflabel{[inl]notsat-conf}
        \end{pfsteps*}
        Assume $\cmaysatisfy{\hinr{\tau_1}{e_2}}{\cinl{\xi_1}}$. By rule induction over Rules (\ref{rules:MaySatisfy}) on it, only one case applies.
        \begin{byCases}
        \item[\text{(\ref{rule:CMSNotVal})}]
            \begin{pfsteps*}
            \item $\isntVal{\hinr{\tau_1}{e_2}}$ \BY{assumption} \pflabel{[inl]notval-inr}
            \end{pfsteps*}
            By rule induction over Rules (\ref{rules:notval}) on \pfref{[inl]notval-inr}, no case applies due to syntactic contradiction.
        \end{byCases}
        \begin{pfsteps*}
        \item $\cnotmaysatisfy{\hinr{\tau_1}{e_2}}{\cinl{\xi_1}}$ \BY{contradiction} \pflabel{[inl]notmaysat-conf}
        \item $\cnotsatisfyormay{\hinr{\tau_1}{e_2}}{\cinl{\xi_1}}$ \BY{\autoref{lem:not-satormay} on \pfref{[inl]notsat-conf} and \pfref{[inl]notmaysat-conf}}
        \end{pfsteps*}
    \end{byCases}
\restorelocalsteps{0}
\item[\text{(\ref{rule:CTInr})}]
    \begin{pfsteps*}
    \item $\xi=\cinr{\xi_2}$ \BY{assumption}
    \item $\tau=\tsum{\tau_1}{\tau_2}$ \BY{assumption}
    \item $\ctyp{\xi_2}{\tau_2}$ \BY{assumption} \pflabel{c2Typ}
    \end{pfsteps*}
    By rule induction over Rules (\ref{rules:TExp}) on \pfref{eTyp}, the following cases apply.
    \begin{byCases}
    \savelocalsteps{1}
    \item[\text{(\ref{rule:TEHole}),(\ref{rule:THole}),(\ref{rule:TAp}),(\ref{rule:TPrl}),(\ref{rule:TPrr}),(\ref{rule:TMatchZPre}),(\ref{rule:TMatchNZPre})}]
        \begin{pfsteps*}
        \item $e=\hehole{u},\hhole{e_0}{u},\hap{e_1}{e_2},\hprl{e_0},\hprr{e_0},\hmatch{e_0}{\zrules}$ \BY{assumption}
        \item $\isntVal{e}$ \BY{Rule (\ref{rule:NVEHole}),(\ref{rule:NVHole}),(\ref{rule:NVAp}),(\ref{rule:NVMatch}),(\ref{rule:NVPrl}),(\ref{rule:NVPrr})} \pflabel{[inr]notval}
        \end{pfsteps*}
        Assume $\csatisfy{e}{\cinr{\xi_2}}$. By rule induction over Rules (\ref{rules:Satisfy}) on it, no case applies due to syntactic contradiction.
        \begin{pfsteps*}
        \item $\cnotsatisfy{e}{\cinr{\xi_2}}$ \BY{contradiction} \pflabel{[inr]notsat-inr}
        \end{pfsteps*}
        By case analysis on the value of $\frefutable{\cinr{\xi_2}}$.
        \begin{byCases}
        \savelocalsteps{2}
        \item[\frefutable{\cinr{\xi_2}}=\true]
            \begin{pfsteps*}
            \item $\frefutable{\cinr{\xi_2}}=\true$ \BY{assumption} \pflabel{[inr]frft-true}
            \item $\refutable{\cinr{\xi_2}}$ \BY{\autoref{lem:sound-complete-xi-refutable} on \pfref{[inr]frft-true}} \pflabel{[inr]rft-true}
            \item $\cmaysatisfy{e}{\cinr{\xi_2}}$ \BY{\ruleref{rule:CMSNotVal} on \pfref{[inr]notval} and \pfref{[inr]rft-true}} \pflabel{[inr]maysat}
            \item $\csatisfyormay{e}{\cinr{\xi_2}}$ \BY{\ruleref{rule:CSMSMay} on \pfref{[inr]maysat}}
            \end{pfsteps*}
        \restorelocalsteps{2}
        \item[\frefutable{\cinr{\xi_2}}=\false]
            \begin{pfsteps*}
            \item $\frefutable{\cinr{\xi_2}}=\false$ \BY{assumption} \pflabel{[inr]frft-false}
            \item $\cancel{\refutable{\cinr{\xi_2}}}$ \BY{\autoref{lem:sound-complete-xi-refutable} on \pfref{[inr]frft-false}} \pflabel{[inr]rft-false}
            \end{pfsteps*}
            Assume $\cmaysatisfy{e}{\cinr{\xi_2}}$. By rule induction over \rulesref{rules:MaySatisfy} on it, only one case applies.
            \begin{byCases}
            \item[\text{(\ref{rule:CMSNotVal})}]
                \begin{pfsteps*}
                \item $\refutable{\cinr{\xi_2}}$ \BY{assumption}
                \end{pfsteps*}
                Contradicts \pfref{[inr]rft-false}.
            \end{byCases}
            \begin{pfsteps*}
            \item $\cnotmaysatisfy{e}{\cinr{\xi_2}}$ \BY{contradiction} \pflabel{[inr]notmaysat-inr}
            \item $\cnotsatisfyormay{e}{\cinr{\xi_2}}$ \BY{\autoref{lem:not-satormay} on \pfref{[inr]notsat-inr} and \pfref{[inr]notmaysat-inr}}
            \end{pfsteps*}
        \end{byCases}
    \restorelocalsteps{1}
    \item[\text{(\ref{rule:TInl})}]
        \begin{pfsteps*}
        \item $e=\hinl{\tau_2}{e_1}$ \BY{assumption}
        \end{pfsteps*}
        Assume $\csatisfy{\hinl{\tau_2}{e_1}}{\cinr{\xi_2}}$. By rule induction over Rules (\ref{rules:Satisfy}) on it, no case applies due to syntactic contradiction.
        \begin{pfsteps*}
        \item $\cnotsatisfy{\hinl{\tau_2}{e_1}}{\cinr{\xi_2}}$ \BY{contradiction} \pflabel{[inr]notsat-conf}
        \end{pfsteps*}
        Assume $\cmaysatisfy{\hinl{\tau_2}{e_1}}{\cinr{\xi_2}}$. By rule induction over Rules (\ref{rules:MaySatisfy}) on it, only one case applies.
        \begin{byCases}
        \item[\text{(\ref{rule:CMSNotVal})}]
            \begin{pfsteps*}
            \item $\isntVal{\hinl{\tau_2}{e_1}}$ \BY{assumption} \pflabel{[inr]notval-inl}
            \end{pfsteps*}
            By rule induction over Rules (\ref{rules:notval}) on \pfref{[inr]notval-inl}, no case applies due to syntactic contradiction.
        \end{byCases}
        \begin{pfsteps*}
        \item $\cnotmaysatisfy{\hinl{\tau_2}{e_1}}{\cinr{\xi_2}}$ \BY{contradiction} \pflabel{[inr]notmaysat-conf}
        \item $\cnotsatisfyormay{\hinl{\tau_2}{e_1}}{\cinr{\xi_2}}$ \BY{\autoref{lem:not-satormay} on \pfref{[inr]notsat-conf} and \pfref{[inr]notmaysat-conf}}
        \end{pfsteps*}
        
    \restorelocalsteps{1}
    \item[\text{(\ref{rule:TInr})}]
        \begin{pfsteps*}
        \item $e=\hinr{\tau_1}{e_2}$ \BY{assumption}
        \item $\hexptyp{\cdot}{\Delta}{e_2}{\tau_2}$ \BY{assumption} \pflabel{e2Typ}
        \item $\isFinal{e_2}$ \BY{\autoref{lem:inr-final} on \pfref{eFinal}} \pflabel{e2Final}
        \end{pfsteps*}
        By inductive hypothesis on \pfref{c2Typ} and \pfref{e2Typ} and \pfref{e2Final}, exactly one of $\csatisfy{e_2}{\xi_2}$, $\cmaysatisfy{e_2}{\xi_2}$, and $\cnotsatisfyormay{e_2}{\xi_2}$ holds. By case analysis on which one holds.
        \begin{byCases}
        \savelocalsteps{2}
        \item[\csatisfy{e_2}{\xi_2}]
            \begin{pfsteps*}
            \item $\csatisfy{e_2}{\xi_2}$ \BY{assumption} \pflabel{[inr1]satisfy2}
            \item $\cnotmaysatisfy{e_2}{\xi_2}$ \BY{assumption} \pflabel{[inr1]notmaysat2}
            \item $\csatisfy{\hinr{\tau_1}{e_2}}{\cinr{\xi_2}}$ \BY{Rule (\ref{rule:CSInl}) on \pfref{[inr1]satisfy2}} \pflabel{[inr1]satisfy}
            \item $\csatisfyormay{\hinr{\tau_1}{e_2}}{\cinr{\xi_2}}$ \BY{\ruleref{rule:CSMSSat} on \pfref{[inr1]satisfy}}
            \end{pfsteps*}
            Assume $\cmaysatisfy{\hinr{\tau_1}{e_2}}{\cinr{\xi_2}}$. By rule induction over Rules (\ref{rules:MaySatisfy}) on it, only two cases apply.
            \begin{byCases}
            \savelocalsteps{3}
            \item[\text{(\ref{rule:CMSNotVal})}]
                \begin{pfsteps*}
                \item $\isntVal{\hinr{\tau_1}{e_2}}$ \BY{assumption} \pflabel{[inr1]notval-inr}
                \end{pfsteps*}
                By rule induction over Rules (\ref{rules:notval}) on \pfref{[inr1]notval-inr}, no case applies due to syntactic contradiction.
            \restorelocalsteps{3}
            \item[\text{(\ref{rule:CMSInr})}]
                \begin{pfsteps*}
                \item $\cmaysatisfy{e_2}{\xi_2}$
                \end{pfsteps*}
                Contradicts \pfref{[inr1]notmaysat2}.
            \end{byCases}
            \begin{pfsteps*}
            \item $\cnotmaysatisfy{\hinr{\tau_1}{e_2}}{\cinr{\xi_2}}$ \BY{contradiction}
            \end{pfsteps*}
        \restorelocalsteps{2}
        \item[\cmaysatisfy{e_2}{\xi_2}]
            \begin{pfsteps*}
            \item $\cnotsatisfy{e_2}{\xi_2}$ \BY{assumption} \pflabel{[inr2]notsatisfy2}
            \item $\cmaysatisfy{e_2}{\xi_2}$ \BY{assumption} \pflabel{[inr2]maysat2}
            \item $\cmaysatisfy{\hinr{\tau_1}{e_2}}{\cinr{\xi_2}}$ \BY{Rule (\ref{rule:CMSInr}) on \pfref{[inr2]maysat2}} \pflabel{[inr2]maysat}
            \item $\csatisfyormay{\hinr{\tau_1}{e_2}}{\cinr{\xi_2}}$ \BY{\ruleref{rule:CSMSMay} on \pfref{[inr2]maysat}}
            \end{pfsteps*}
            Assume $\csatisfy{\hinr{\tau_1}{e_2}}{\cinr{\xi_2}}$. By rule induction over Rules (\ref{rules:Satisfy}) on it, only one case applies.
            \begin{byCases}
            \item[\text{(\ref{rule:CSInr})}]
                \begin{pfsteps*}
                \item $\csatisfy{e_2}{\xi_2}$
                \end{pfsteps*}
                Contradicts \pfref{[inr2]notsatisfy2}.
            \end{byCases}
            \begin{pfsteps*}
            \item $\cnotsatisfy{\hinr{\tau_1}{e_2}}{\cinr{\xi_2}}$ \BY{contradiction}
            \end{pfsteps*}
        \restorelocalsteps{2}
        \item[\cnotsatisfyormay{e_2}{\xi_2}]
            \begin{pfsteps*}
            \item $\cnotsatisfy{e_2}{\xi_2}$ \BY{assumption} \pflabel{[inr3]notsatisfy2}
            \item $\cnotmaysatisfy{e_2}{\xi_2}$ \BY{assumption} \pflabel{[inr3]notmaysat2}
            \end{pfsteps*}
            Assume $\csatisfy{\hinr{\tau_1}{e_2}}{\cinr{\xi_2}}$. By rule induction over Rules (\ref{rules:Satisfy}) on it, only one case applies.
            \begin{byCases}
            \item[\text{(\ref{rule:CSInr})}]
                \begin{pfsteps*}
                \item $\csatisfy{e_2}{\xi_2}$
                \end{pfsteps*}
                Contradicts \pfref{[inr3]notsatisfy2}.
            \end{byCases}
            \begin{pfsteps*}
            \item $\cnotsatisfy{\hinr{\tau_1}{e_2}}{\cinr{\xi_2}}$ \BY{contradiction} \pflabel{[inr3]notsatisfy}
            \end{pfsteps*}
            Assume $\cmaysatisfy{\hinr{\tau_1}{e_2}}{\cinr{\xi_2}}$. By rule induction over Rules (\ref{rules:MaySatisfy}) on it, only one case applies.
            \begin{byCases}
            \savelocalsteps{3}
            \item[\text{(\ref{rule:CMSNotVal})}]
                \begin{pfsteps*}
                \item $\isntVal{\hinr{\tau_1}{e_2}}$ \BY{assumption} \pflabel{[inr3]notval-inr}
                \end{pfsteps*}
                By rule induction over Rules (\ref{rules:notval}) on \pfref{[inr3]notval-inr}, no case applies due to syntactic contradiction.
            \restorelocalsteps{3}
            \item[\text{(\ref{rule:CMSInr})}]
                \begin{pfsteps*}
                \item $\cmaysatisfy{e_2}{\xi_2}$
                \end{pfsteps*}
                Contradicts \pfref{[inr3]notmaysat2}.
            \end{byCases}
            \begin{pfsteps*}
            \item $\cnotmaysatisfy{\hinr{\tau_1}{e_2}}{\cinr{\xi_2}}$ \BY{contradiction} \pflabel{[inr3]notmaysatisfy}
            \item $\cnotsatisfyormay{\hinl{\tau_2}{e_1}}{\cinl{\xi_1}}$ \BY{\autoref{lem:not-satormay} on \pfref{[inr3]notsatisfy} and \pfref{[inr3]notmaysatisfy}}
            \end{pfsteps*}
        \end{byCases}
    \end{byCases}

\restorelocalsteps{0}
\item[\text{(\ref{rule:CSPair})}]
\begin{pfsteps*}
    \item $\xi=\cpair{\xi_1}{\xi_2}$ \BY{assumption}
    \item $\tau=\tprod{\tau_1}{\tau_2}$ \BY{assumption}
    \item $\ctyp{\xi_1}{\tau_1}$ \BY{assumption} \pflabel{[pair]c1Typ}
    \item $\ctyp{\xi_2}{\tau_2}$ \BY{assumption} \pflabel{[pair]c2Typ}
    \end{pfsteps*}
    By rule induction over Rules (\ref{rules:TExp}) on \pfref{eTyp}, the following cases apply.
    \begin{byCases}
    \savelocalsteps{00}
    \item[\text{(\ref{rule:TEHole}),(\ref{rule:THole}),(\ref{rule:TAp}),(\ref{rule:TPrl}),(\ref{rule:TPrr}),(\ref{rule:TMatchZPre}),(\ref{rule:TMatchNZPre})}]
        \begin{pfsteps*}
        \item $e=\hehole{u},\hhole{e_0}{u},\hap{e_1}{e_2},\hprl{e_0},\hprr{e_0},\hmatch{e_0}{\zrules}$ \BY{assumption}
        \item $\isntVal{e}$ \BY{Rule (\ref{rule:NVEHole}),(\ref{rule:NVHole}),(\ref{rule:NVAp}),(\ref{rule:NVMatch}),(\ref{rule:NVPrl}),(\ref{rule:NVPrr})} \pflabel{[pair]notval}
        \item $\isIndet{e}$ \BY{\autoref{lem:final-notval-indet} on \pfref{eFinal} and \pfref{[pair]notval}} \pflabel{[pair]indet}
        \item $\isIndet{\hprl{e}}$ \BY{Rule (\ref{rule:IPrl}) on \pfref{[pair]indet}} \pflabel{[pair]prl-indet}
        \item $\isFinal{\hprl{e}}$ \BY{Rule (\ref{rule:FIndet}) on \pfref{[pair]prl-indet}} \pflabel{[pair]prl-final}
        \item $\isIndet{\hprr{e}}$ \BY{Rule (\ref{rule:IPrr}) on \pfref{[pair]indet}} \pflabel{[pair]prr-indet}
        \item $\isFinal{\hprr{e}}$ \BY{Rule (\ref{rule:FIndet}) on \pfref{[pair]prr-indet}} \pflabel{[pair]prr-final}
        \item $\hexptyp{\cdot}{\Delta}{\hprl{e}}{\tau_1}$ \BY{Rule (\ref{rule:TPrl}) on \pfref{eTyp}} \pflabel{[pair]prl-typ}
        \item $\hexptyp{\cdot}{\Delta}{\hprr{e}}{\tau_2}$ \BY{Rule (\ref{rule:TPrr}) on \pfref{eTyp}} \pflabel{[pair]prr-typ}
        \end{pfsteps*}
        By inductive hypothesis on \pfref{[pair]c1Typ} and \pfref{[pair]prl-typ} and \pfref{[pair]prl-final}, exactly one of $\csatisfy{\hprl{e}}{\xi_1}$, $\cmaysatisfy{\hprl{e}}{\xi_1}$, and $\cnotsatisfyormay{\hprl{e}}{\xi_1}$ holds. \\
        By inductive hypothesis on \pfref{[pair]c2Typ} and \pfref{[pair]prr-typ} and \pfref{[pair]prr-final}, exactly one of $\csatisfy{\hprr{e}}{\xi_2}$, $\cmaysatisfy{\hprr{e}}{\xi_2}$, and $\cnotsatisfyormay{\hprr{e}}{\xi_2}$ holds. \\
        By case analysis on which conclusion holds for $\xi_1$ and $\xi_2$.
        \begin{byCases}
        \savelocalsteps{1}
        \item[\csatisfy{\hprl{e}}{\xi_1},\csatisfy{\hprr{e}}{\xi_2}]
            \begin{pfsteps*}
            \item $\csatisfy{\hprl{e}}{\xi_1}$ \BY{assumption} \pflabel{[pair1]satisfy1}
            \item $\cnotmaysatisfy{\hprl{e}}{\xi_1}$ \BY{assumption} \pflabel{[pair1]notmaysat1}
            \item $\csatisfy{\hprr{e}}{\xi_2}$ \BY{assumption} \pflabel{[pair1]satisfy2}
            \item $\cnotmaysatisfy{\hprr{e}}{\xi_2}$ \BY{assumption} \pflabel{[pair1]notmaysat2}
            \item $\csatisfy{e}{\cpair{\xi_1}{\xi_2}}$ \BY{Rule (\ref{rule:CSNotValPair}) on \pfref{[pair]notval} and \pfref{[pair1]satisfy1} and \pfref{[pair1]satisfy2}} \pflabel{[pair1]satisfy-pair}
            \item $\csatisfyormay{e}{\cpair{\xi_1}{\xi_2}}$ \BY{\ruleref{rule:CSMSSat} on \pfref{[pair1]satisfy-pair}}
            \item $\cancel{\refutable{\cpair{\xi_1}{\xi_2}}}$ \BY{\autoref{lem:satisfy-not-refutable} on \pfref{[pair]notval} and \pfref{[pair1]satisfy-pair}} \pflabel{[pair1]not-rft-pair}
            \end{pfsteps*}
            Assume $\cmaysatisfy{e}{\cpair{\xi_1}{\xi_2}}$. By rule induction over Rules (\ref{rules:MaySatisfy}) on it, only one case applies.
            \begin{byCases}
            \item[\text{(\ref{rule:CMSNotVal})}]
                \begin{pfsteps*}
                \item $\refutable{\cpair{\xi_1}{\xi_2}}$ \BY{assumption}
                \end{pfsteps*}
                Contradicts \pfref{[pair1]not-rft-pair}.
            \end{byCases}
            \begin{pfsteps*}
            \item $\cnotmaysatisfy{e}{\cpair{\xi_1}{\xi_2}}$ \BY{contradiction}
            \end{pfsteps*}
            
        \restorelocalsteps{1}
        \item[\csatisfy{\hprl{e}}{\xi_1},\cmaysatisfy{\hprr{e}}{\xi_2}]
            \begin{pfsteps*}
            \item $\csatisfy{\hprl{e}}{\xi_1}$ \BY{assumption} \pflabel{[pair2]satisfy1}
            \item $\cnotmaysatisfy{\hprl{e}}{\xi_1}$ \BY{assumption} \pflabel{[pair2]notmaysat1}
            \item $\cnotsatisfy{\hprr{e}}{\xi_2}$ \BY{assumption} \pflabel{[pair2]notsatisfy2}
            \item $\cmaysatisfy{\hprr{e}}{\xi_2}$ \BY{assumption} \pflabel{[pair2]maysat2}
            \end{pfsteps*}
            Assume $\csatisfy{e}{\cpair{\xi_1}{\xi_2}}$. By rule induction over Rules (\ref{rules:Satisfy}), only one case applies.
            \begin{byCases}
            \item[\text{(\ref{rule:CSNotValPair})}]
                \begin{pfsteps*}
                \item $\csatisfy{\hprr{e}}{\xi_2}$ \BY{assumption}
                \end{pfsteps*}
                Contradicts \pfref{[pair2]notsatisfy2}
            \end{byCases}
            \begin{pfsteps*}
            \item $\cnotsatisfy{e}{\cpair{\xi_1}{\xi_2}}$ \BY{contradiction} \pflabel{[pair2]notsatisfy-pair}
            \end{pfsteps*}
            By rule induction over Rules (\ref{rules:MaySatisfy}) on \pfref{[pair2]maysat2}, only one case applies. \todo{assume no "or" and "and" in pair}
            \begin{byCases}
            \item[\text{(\ref{rule:CMSNotVal})}]
                \begin{pfsteps*}
                \item $\refutable{\xi_2}$ \BY{assumption} \pflabel{[pair2]rft2}
                \item $\refutable{\cpair{\xi_1}{\xi_2}}$ \BY{\ruleref{rule:RXPairR} on \pfref{[pair2]rft2}} \pflabel{[pair2]rft-pair}
                \item $\cmaysatisfy{e}{\cpair{\xi_1}{\xi_2}}$ \BY{\ruleref{rule:CMSNotVal} on \pfref{[pair]notval} and \pfref{[pair2]rft-pair}} \pflabel{[pair2]maysat-pair}
                \item $\csatisfyormay{e}{\cpair{\xi_1}{\xi_2}}$ \BY{\ruleref{rule:CSMSMay} on \pfref{[pair2]maysat-pair}}
                \end{pfsteps*}
            \end{byCases}
            
        \restorelocalsteps{1}
        \item[\csatisfy{\hprl{e}}{\xi_1},\cnotsatisfyormay{\hprr{e}}{\xi_2}]
            \begin{pfsteps*}
            \item $\csatisfy{\hprl{e}}{\xi_1}$ \BY{assumption} \pflabel{[pair3]satisfy1}
            \item $\cnotmaysatisfy{\hprl{e}}{\xi_1}$ \BY{assumption} \pflabel{[pair3]notmaysat1}
            \item $\cnotsatisfy{\hprr{e}}{\xi_2}$ \BY{assumption} \pflabel{[pair3]notsatisfy2}
            \item $\cnotmaysatisfy{\hprr{e}}{\xi_2}$ \BY{assumption} \pflabel{[pair3]notmaysat2}
            \end{pfsteps*}
            Assume $\csatisfy{e}{\cpair{\xi_1}{\xi_2}}$. By rule induction over Rules (\ref{rules:Satisfy}) on it, only one case applies.
            \begin{byCases}
            \item[\text{(\ref{rule:CSNotValPair})}]
                \begin{pfsteps*}
                \item $\csatisfy{\hprr{e}}{\xi_2}$ \BY{assumption}
                \end{pfsteps*}
                Contradicts \pfref{[pair3]notsatisfy2}.
            \end{byCases}
            \begin{pfsteps*}
            \item $\cnotsatisfy{e}{\cpair{\xi_1}{\xi_2}}$ \BY{contradiction} \pflabel{[pair3]notsat}
            \end{pfsteps*}
            Assume $\cmaysatisfy{e}{\cpair{\xi_1}{\xi_2}}$. By rule induction over Rules (\ref{rules:MaySatisfy}) on it, only one case applies.
            \begin{byCases}
            \item[\text{(\ref{rule:CMSNotVal})}]
                \begin{pfsteps*}
                \item $\refutable{\cpair{\xi_1}{\xi_2}}$ \BY{assumption} \pflabel{[pair3]rft-pair}
                \end{pfsteps*}
                By rule induction over \rulesref{rules:xi-refutable} on \pfref{[pair3]rft-pair}, only two cases apply.
                \begin{byCases}
                \savelocalsteps{2}
                \item[\text{(\ref{rule:RXPairL})}]
                    \begin{pfsteps*}
                    \item $\refutable{\xi_1}$ \BY{assumption} \pflabel{[pair3]rft1}
                    \item $\isntVal{\hprl{e}}$ \BY{\ruleref{rule:NVPrl}} \pflabel{[pair3]prl-notval}
                    \item $\cmaysatisfy{\hprl{e}}{\xi_1}$ \BY{\ruleref{rule:CMSNotVal} on \pfref{[pair3]prl-notval} and \pfref{[pair3]rft1}}
                    \end{pfsteps*}
                    Contradicts \pfref{[pair3]notmaysat1}.
                \restorelocalsteps{2}
                \item[\text{(\ref{rule:RXPairR})}]
                    \begin{pfsteps*}
                    \item $\refutable{\xi_2}$ \BY{assumption} \pflabel{[pair3]rft2}
                    \item $\isntVal{\hprr{e}}$ \BY{\ruleref{rule:NVPrr}} \pflabel{[pair3]prr-notval}
                    \item $\cmaysatisfy{\hprr{e}}{\xi_2}$ \BY{\ruleref{rule:CMSNotVal} on \pfref{[pair3]prr-notval} and \pfref{[pair3]rft2}}
                    \end{pfsteps*}
                    Contradicts \pfref{[pair3]notmaysat2}.
                \end{byCases}
            \end{byCases}
            \begin{pfsteps*}
            \item $\cnotmaysatisfy{e}{\cpair{\xi_1}{\xi_2}}$ \BY{contradiction} \pflabel{[pair3]notmaysat}
            \item $\cnotsatisfyormay{e}{\cpair{\xi_1}{\xi_2}}$ \BY{\autoref{lem:not-satormay} on \pfref{[pair3]notsat} and \pfref{[pair3]notmaysat}}
            \end{pfsteps*}
            
        \restorelocalsteps{1}
        \item[\cmaysatisfy{\hprl{e}}{\xi_1},\csatisfy{\hprr{e}}{\xi_2}]
            \begin{pfsteps*}
            \item $\cnotsatisfy{\hprl{e}}{\xi_1}$ \BY{assumption} \pflabel{[pair4]notsatisfy1}
            \item $\cmaysatisfy{\hprl{e}}{\xi_1}$ \BY{assumption} \pflabel{[pair4]maysat1}
            \item $\csatisfy{\hprr{e}}{\xi_2}$ \BY{assumption} \pflabel{[pair4]satisfy2}
            \item $\cnotmaysatisfy{\hprr{e}}{\xi_2}$ \BY{assumption} \pflabel{[pair4]notmaysat2}
            \end{pfsteps*}
            Assume $\csatisfy{e}{\cpair{\xi_1}{\xi_2}}$. By rule induction over Rules (\ref{rules:Satisfy}), only one case applies.
            \begin{byCases}
            \item[\text{(\ref{rule:CSNotValPair})}]
                \begin{pfsteps*}
                \item $\csatisfy{\hprl{e}}{\xi_1}$ \BY{assumption}
                \end{pfsteps*}
                Contradicts \pfref{[pair4]notsatisfy1}.
            \end{byCases}
            \begin{pfsteps*}
            \item $\cnotsatisfy{e}{\cpair{\xi_1}{\xi_2}}$ \BY{contradiction} \pflabel{[pair4]notsatisfy-pair}
            \end{pfsteps*}
            By rule induction over Rules (\ref{rules:MaySatisfy}) on \pfref{[pair4]maysat1}, only one case applies. \todo{assume no "or" and "and" in pair}
            \begin{byCases}
            \item[\text{(\ref{rule:CMSNotVal})}]
                \begin{pfsteps*}
                \item $\refutable{\xi_1}$ \BY{assumption} \pflabel{[pair4]rft1}
                \item $\refutable{\cpair{\xi_1}{\xi_2}}$ \BY{\ruleref{rule:RXPairR} on \pfref{[pair4]rft1}} \pflabel{[pair4]rft-pair}
                \item $\cmaysatisfy{e}{\cpair{\xi_1}{\xi_2}}$ \BY{\ruleref{rule:CMSNotVal} on \pfref{[pair]notval} and \pfref{[pair4]rft-pair}} \pflabel{[pair4]maysat-pair}
                \item $\csatisfyormay{e}{\cpair{\xi_1}{\xi_2}}$ \BY{\ruleref{rule:CSMSMay} on \pfref{[pair4]maysat-pair}}
                \end{pfsteps*}
            \end{byCases}
            
        \restorelocalsteps{1}
        \item[\cmaysatisfy{\hprl{e}}{\xi_1},\cmaysatisfy{\hprr{e}}{\xi_2}]
            \begin{pfsteps*}
            \item $\cnotsatisfy{\hprl{e}}{\xi_1}$ \BY{assumption} \pflabel{[pair5]notsatisfy1}
            \item $\cmaysatisfy{\hprl{e}}{\xi_1}$ \BY{assumption} \pflabel{[pair5]maysat1}
            \item $\cnotsatisfy{\hprr{e}}{\xi_2}$ \BY{assumption} \pflabel{[pair5]notsatisfy2}
            \item $\cmaysatisfy{\hprr{e}}{\xi_2}$ \BY{assumption} \pflabel{[pair5]maysat2}
            \end{pfsteps*}
            Assume $\csatisfy{e}{\cpair{\xi_1}{\xi_2}}$. By rule induction over Rules (\ref{rules:Satisfy}), only one case applies.
            \begin{byCases}
            \item[\text{(\ref{rule:CSNotValPair})}]
                \begin{pfsteps*}
                \item $\csatisfy{\hprl{e}}{\xi_1}$ \BY{assumption}
                \end{pfsteps*}
                Contradicts \pfref{[pair5]notsatisfy1}.
            \end{byCases}
            \begin{pfsteps*}
            \item $\cnotsatisfy{e}{\cpair{\xi_1}{\xi_2}}$ \BY{contradiction} \pflabel{[pair5]notsatisfy-pair}
            \end{pfsteps*}
            By rule induction over Rules (\ref{rules:MaySatisfy}) on \pfref{[pair5]maysat2}, only one case applies. \todo{assume no "or" and "and" in pair}
            \begin{byCases}
            \item[\text{(\ref{rule:CMSNotVal})}]
                \begin{pfsteps*}
                \item $\refutable{\xi_2}$ \BY{assumption} \pflabel{[pair5]rft2}
                \item $\refutable{\cpair{\xi_1}{\xi_2}}$ \BY{\ruleref{rule:RXPairR} on \pfref{[pair5]rft2}} \pflabel{[pair5]rft-pair}
                \item $\cmaysatisfy{e}{\cpair{\xi_1}{\xi_2}}$ \BY{\ruleref{rule:CMSNotVal} on \pfref{[pair]notval} and \pfref{[pair5]rft-pair}} \pflabel{[pair5]maysat-pair}
                \item $\csatisfyormay{e}{\cpair{\xi_1}{\xi_2}}$ \BY{\ruleref{rule:CSMSMay} on \pfref{[pair5]maysat-pair}}
                \end{pfsteps*}
            \end{byCases}
        \restorelocalsteps{1}
        \item[\cmaysatisfy{\hprl{e}}{\xi_1},\cnotsatisfyormay{\hprr{e}}{\xi_2}]
            \begin{pfsteps*}
            \item $\cnotsatisfy{\hprl{e}}{\xi_1}$ \BY{assumption} \pflabel{[pair6]notsatisfy1}
            \item $\cmaysatisfy{\hprl{e}}{\xi_1}$ \BY{assumption} \pflabel{[pair6]maysat1}
            \item $\cnotsatisfy{\hprr{e}}{\xi_2}$ \BY{assumption} \pflabel{[pair6]notsatisfy2}
            \item $\cnotmaysatisfy{\hprr{e}}{\xi_2}$ \BY{assumption} \pflabel{[pair6]notmaysat2}
            \end{pfsteps*}
            Assume $\csatisfy{e}{\cpair{\xi_1}{\xi_2}}$. By rule induction over Rules (\ref{rules:Satisfy}), only one case applies.
            \begin{byCases}
            \item[\text{(\ref{rule:CSNotValPair})}]
                \begin{pfsteps*}
                \item $\csatisfy{\hprl{e}}{\xi_1}$ \BY{assumption}
                \end{pfsteps*}
                Contradicts \pfref{[pair6]notsatisfy1}
            \end{byCases}
            \begin{pfsteps*}
            \item $\cnotsatisfy{e}{\cpair{\xi_1}{\xi_2}}$ \BY{contradiction} \pflabel{[pair6]notsatisfy-pair}
            \end{pfsteps*}
            By rule induction over Rules (\ref{rules:MaySatisfy}) on \pfref{[pair6]maysat1}, only one case applies. \todo{assume no "or" and "and" in pair}
            \begin{byCases}
            \item[\text{(\ref{rule:CMSNotVal})}]
                \begin{pfsteps*}
                \item $\refutable{\xi_1}$ \BY{assumption} \pflabel{[pair6]rft1}
                \item $\refutable{\cpair{\xi_1}{\xi_2}}$ \BY{\ruleref{rule:RXPairR} on \pfref{[pair6]rft1}} \pflabel{[pair6]rft-pair}
                \item $\cmaysatisfy{e}{\cpair{\xi_1}{\xi_2}}$ \BY{\ruleref{rule:CMSNotVal} on \pfref{[pair]notval} and \pfref{[pair6]rft-pair}} \pflabel{[pair6]maysat-pair}
                \item $\csatisfyormay{e}{\cpair{\xi_1}{\xi_2}}$ \BY{\ruleref{rule:CSMSMay} on \pfref{[pair6]maysat-pair}}
                \end{pfsteps*}
            \end{byCases}
        \restorelocalsteps{1}
        \item[\cnotsatisfyormay{\hprl{e}}{\xi_1},\csatisfy{\hprr{e}}{\xi_2}]
            \begin{pfsteps*}
            \item $\cnotsatisfy{\hprl{e}}{\xi_1}$ \BY{assumption} \pflabel{[pair7]notsatisfy1}
            \item $\cnotmaysatisfy{\hprl{e}}{\xi_1}$ \BY{assumption} \pflabel{[pair7]notmaysat1}
            \item $\csatisfy{\hprr{e}}{\xi_2}$ \BY{assumption} \pflabel{[pair7]satisfy2}
            \item $\cnotmaysatisfy{\hprr{e}}{\xi_2}$ \BY{assumption} \pflabel{[pair7]notmaysat2}
            \end{pfsteps*}
            Assume $\csatisfy{e}{\cpair{\xi_1}{\xi_2}}$. By rule induction over Rules (\ref{rules:Satisfy}) on it, only one case applies.
            \begin{byCases}
            \item[\text{(\ref{rule:CSNotValPair})}]
                \begin{pfsteps*}
                \item $\csatisfy{\hprl{e}}{\xi_1}$ \BY{assumption}
                \end{pfsteps*}
                Contradicts \pfref{[pair7]notsatisfy1}
            \end{byCases}
            \begin{pfsteps*}
            \item $\cnotsatisfy{e}{\cpair{\xi_1}{\xi_2}}$ \BY{contradiction} \pflabel{[pair7]notsat}
            \end{pfsteps*}
            Assume $\cmaysatisfy{e}{\cpair{\xi_1}{\xi_2}}$. By rule induction over Rules (\ref{rules:MaySatisfy}) on it, only one case applies.
            \begin{byCases}
            \item[\text{(\ref{rule:CMSNotVal})}]
                \begin{pfsteps*}
                \item $\refutable{\cpair{\xi_1}{\xi_2}}$ \BY{assumption} \pflabel{[pair7]rft-pair}
                \end{pfsteps*}
                By rule induction over \rulesref{rules:xi-refutable} on \pfref{[pair7]rft-pair}, only two cases apply.
                \begin{byCases}
                \savelocalsteps{2}
                \item[\text{(\ref{rule:RXPairL})}]
                    \begin{pfsteps*}
                    \item $\refutable{\xi_1}$ \BY{assumption} \pflabel{[pair7]rft1}
                    \item $\isntVal{\hprl{e}}$ \BY{\ruleref{rule:NVPrl}} \pflabel{[pair7]prl-notval}
                    \item $\cmaysatisfy{\hprl{e}}{\xi_1}$ \BY{\ruleref{rule:CMSNotVal} on \pfref{[pair7]prl-notval} and \pfref{[pair7]rft1}}
                    \end{pfsteps*}
                    Contradicts \pfref{[pair3]notmaysat1}.
                \restorelocalsteps{2}
                \item[\text{(\ref{rule:RXPairR})}]
                    \begin{pfsteps*}
                    \item $\refutable{\xi_2}$ \BY{assumption} \pflabel{[pair7]rft2}
                    \item $\isntVal{\hprr{e}}$ \BY{\ruleref{rule:NVPrr}} \pflabel{[pair7]prr-notval}
                    \item $\cmaysatisfy{\hprr{e}}{\xi_2}$ \BY{\ruleref{rule:CMSNotVal} on \pfref{[pair7]prr-notval} and \pfref{[pair7]rft2}}
                    \end{pfsteps*}
                    Contradicts \pfref{[pair7]notmaysat2}.
                \end{byCases}
            \end{byCases}
            \begin{pfsteps*}
            \item $\cnotmaysatisfy{e}{\cpair{\xi_1}{\xi_2}}$ \BY{contradiction} \pflabel{[pair7]notmaysat}
            \item $\cnotsatisfyormay{e}{\cpair{\xi_1}{\xi_2}}$ \BY{\autoref{lem:not-satormay} on \pfref{[pair7]notsat} and \pfref{[pair7]notmaysat}}
            \end{pfsteps*}
            
        \restorelocalsteps{1}
        \item[\cnotsatisfyormay{\hprl{e}}{\xi_1},\cmaysatisfy{\hprr{e}}{\xi_2}]
            \begin{pfsteps*}
            \item $\cnotsatisfy{\hprl{e}}{\xi_1}$ \BY{assumption} \pflabel{[pair8]notsatisfy1}
            \item $\cnotmaysatisfy{\hprl{e}}{\xi_1}$ \BY{assumption} \pflabel{[pair8]notmaysat1}
            \item $\cnotsatisfy{\hprr{e}}{\xi_2}$ \BY{assumption} \pflabel{[pair8]notsatisfy2}
            \item $\cmaysatisfy{\hprr{e}}{\xi_2}$ \BY{assumption} \pflabel{[pair8]maysat2}
            \end{pfsteps*}
            Assume $\csatisfy{e}{\cpair{\xi_1}{\xi_2}}$. By rule induction over Rules (\ref{rules:Satisfy}), only one case applies.
            \begin{byCases}
            \item[\text{(\ref{rule:CSNotValPair})}]
                \begin{pfsteps*}
                \item $\csatisfy{\hprl{e}}{\xi_1}$ \BY{assumption}
                \end{pfsteps*}
                Contradicts \pfref{[pair8]notsatisfy1}.
            \end{byCases}
            \begin{pfsteps*}
            \item $\cnotsatisfy{e}{\cpair{\xi_1}{\xi_2}}$ \BY{contradiction} \pflabel{[pair8]notsatisfy-pair}
            \end{pfsteps*}
            By rule induction over Rules (\ref{rules:MaySatisfy}) on \pfref{[pair8]maysat2}, only one case applies. \todo{assume no "or" and "and" in pair}
            \begin{byCases}
            \item[\text{(\ref{rule:CMSNotVal})}]
                \begin{pfsteps*}
                \item $\refutable{\xi_2}$ \BY{assumption} \pflabel{[pair8]rft2}
                \item $\refutable{\cpair{\xi_1}{\xi_2}}$ \BY{\ruleref{rule:RXPairR} on \pfref{[pair8]rft2}} \pflabel{[pair8]rft-pair}
                \item $\cmaysatisfy{e}{\cpair{\xi_1}{\xi_2}}$ \BY{\ruleref{rule:CMSNotVal} on \pfref{[pair]notval} and \pfref{[pair8]rft-pair}} \pflabel{[pair8]maysat-pair}
                \item $\csatisfyormay{e}{\cpair{\xi_1}{\xi_2}}$ \BY{\ruleref{rule:CSMSMay} on \pfref{[pair8]maysat-pair}}
                \end{pfsteps*}
            \end{byCases}
        \restorelocalsteps{1}
        \item[\cnotsatisfyormay{\hprl{e}}{\xi_1},\cnotsatisfyormay{\hprr{e}}{\xi_2}]
            \begin{pfsteps*}
            \item $\cnotsatisfy{\hprl{e}}{\xi_1}$ \BY{assumption} \pflabel{[pair9]notsatisfy1}
            \item $\cnotmaysatisfy{\hprl{e}}{\xi_1}$ \BY{assumption} \pflabel{[pair9]notmaysat1}
            \item $\cnotsatisfy{\hprr{e}}{\xi_2}$ \BY{assumption} \pflabel{[pair9]notsatisfy2}
            \item $\cnotmaysatisfy{\hprr{e}}{\xi_2}$ \BY{assumption} \pflabel{[pair9]notmaysat2}
            \end{pfsteps*}
            Assume $\csatisfy{e}{\cpair{\xi_1}{\xi_2}}$. By rule induction over Rules (\ref{rules:Satisfy}) on it, only one case applies.
            \begin{byCases}
            \item[\text{(\ref{rule:CSNotValPair})}]
                \begin{pfsteps*}
                \item $\csatisfy{\hprl{e}}{\xi_1}$ \BY{assumption}
                \end{pfsteps*}
                Contradicts \pfref{[pair9]notsatisfy1}
            \end{byCases}
            \begin{pfsteps*}
            \item $\cnotsatisfy{e}{\cpair{\xi_1}{\xi_2}}$ \BY{contradiction} \pflabel{[pair9]notsat}
            \end{pfsteps*}
            Assume $\cmaysatisfy{e}{\cpair{\xi_1}{\xi_2}}$. By rule induction over Rules (\ref{rules:MaySatisfy}) on it, only one case applies.
            \begin{byCases}
            \item[\text{(\ref{rule:CMSNotVal})}]
                \begin{pfsteps*}
                \item $\refutable{\cpair{\xi_1}{\xi_2}}$ \BY{assumption} \pflabel{[pair9]rft-pair}
                \end{pfsteps*}
                By rule induction over \rulesref{rules:xi-refutable} on \pfref{[pair9]rft-pair}, only two cases apply.
                \begin{byCases}
                \savelocalsteps{2}
                \item[\text{(\ref{rule:RXPairL})}]
                    \begin{pfsteps*}
                    \item $\refutable{\xi_1}$ \BY{assumption} \pflabel{[pair9]rft1}
                    \item $\isntVal{\hprl{e}}$ \BY{\ruleref{rule:NVPrl}} \pflabel{[pair9]prl-notval}
                    \item $\cmaysatisfy{\hprl{e}}{\xi_1}$ \BY{\ruleref{rule:CMSNotVal} on \pfref{[pair9]prl-notval} and \pfref{[pair9]rft1}}
                    \end{pfsteps*}
                    Contradicts \pfref{[pair9]notmaysat1}.
                \restorelocalsteps{2}
                \item[\text{(\ref{rule:RXPairR})}]
                    \begin{pfsteps*}
                    \item $\refutable{\xi_2}$ \BY{assumption} \pflabel{[pair9]rft2}
                    \item $\isntVal{\hprr{e}}$ \BY{\ruleref{rule:NVPrr}} \pflabel{[pair9]prr-notval}
                    \item $\cmaysatisfy{\hprr{e}}{\xi_2}$ \BY{\ruleref{rule:CMSNotVal} on \pfref{[pair9]prr-notval} and \pfref{[pair9]rft2}}
                    \end{pfsteps*}
                    Contradicts \pfref{[pair9]notmaysat2}.
                \end{byCases}
            \end{byCases}
            \begin{pfsteps*}
            \item $\cnotmaysatisfy{e}{\cpair{\xi_1}{\xi_2}}$ \BY{contradiction} \pflabel{[pair9]notmaysat}
            \item $\cnotsatisfyormay{e}{\cpair{\xi_1}{\xi_2}}$ \BY{\autoref{lem:not-satormay} on \pfref{[pair9]notsat} and \pfref{[pair9]notmaysat}}
            \end{pfsteps*}
            
        \end{byCases}
    \restorelocalsteps{00}
    \item[\text{(\ref{rule:TPair})}]
        \begin{pfsteps*}
        \item $e=\hpair{e_1}{e_2}$ \BY{assumption}
        \item $\hexptyp{\cdot}{\Delta}{e_1}{\tau_1}$ \BY{assumption} \pflabel{[epair]e1-typ}
        \item $\hexptyp{\cdot}{\Delta}{e_2}{\tau_2}$ \BY{assumption} \pflabel{[epair]e2-typ}
        \item $\isFinal{e_1}$ \BY{\autoref{lem:pair-final} on \pfref{eFinal}} \pflabel{[epair]e1-final}
        \item $\isFinal{e_2}$ \BY{\autoref{lem:pair-final} on \pfref{eFinal}} \pflabel{[epair]e2-final}
        \end{pfsteps*}
        By inductive hypothesis on \pfref{[pair]c1Typ} and \pfref{[epair]e1-typ} and \pfref{[epair]e1-final}, exactly one of $\csatisfy{e_1}{\xi_1}$, $\cmaysatisfy{e_1}{\xi_1}$, and $\csatisfy{e_1}{\cdual{\xi_1}}$ holds. \\
        By inductive hypothesis on \pfref{[pair]c2Typ} and \pfref{[epair]e2-typ} and \pfref{[epair]e2-final}, exactly one of $\csatisfy{e_2}{\xi_2}$, $\cmaysatisfy{e_2}{\xi_2}$, and $\csatisfy{e_2}{\cdual{\xi_2}}$ holds. \\
        By case analysis on which conclusion holds for $\xi_1$ and $\xi_2$.
        \begin{byCases}
        \savelocalsteps{1}
        \item[\csatisfy{e_1}{\xi_1},\csatisfy{e_2}{\xi_2}]
            \begin{pfsteps*}
            \item $\csatisfy{e_1}{\xi_1}$ \BY{assumption} \pflabel{[epair1]satisfy1}
            \item $\cnotmaysatisfy{e_1}{\xi_1}$ \BY{assumption} \pflabel{[epair1]notmaysat1}
            \item $\csatisfy{e_2}{\xi_2}$ \BY{assumption} \pflabel{[epair1]satisfy2}
            \item $\cnotmaysatisfy{e_2}{\xi_2}$ \BY{assumption} \pflabel{[epair1]notmaysat2}
            \item $\csatisfy{\hpair{e_1}{e_2}}{\cpair{\xi_1}{\xi_2}}$ \BY{Rule (\ref{rule:CSPair}) on \pfref{[epair1]satisfy1} and \pfref{[epair1]satisfy2}} \pflabel{[epair1]satisfy}
            \item $\csatisfyormay{\hpair{e_1}{e_2}}{\cpair{\xi_1}{\xi_2}}$ \BY{\ruleref{rule:CSMSSat} on \pfref{[epair1]satisfy}}
            \end{pfsteps*}
            Assume $\cmaysatisfy{\hpair{e_1}{e_2}}{\cpair{\xi_1}{\xi_2}}$. By rule induction over Rules (\ref{rules:MaySatisfy}) on it, the following cases apply.
            \begin{byCases}
            \savelocalsteps{2}
            \item[\text{(\ref{rule:CMSNotVal})}]
                \begin{pfsteps*}
                \item $\isntVal{\hpair{e_1}{e_2}}$ \BY{assumption}
                \end{pfsteps*}
                Contradicts \autoref{lem:no-pair-notval}.
            \restorelocalsteps{2}
            \item[\text{(\ref{rule:CMSPair1})}]
                \begin{pfsteps*}
                \item $\cmaysatisfy{e_1}{\xi_1}$ \BY{assumption}
                \end{pfsteps*}
                Contradicts \pfref{[epair1]notmaysat1}.
            \restorelocalsteps{2}
            \item[\text{(\ref{rule:CMSPair2})}]
                \begin{pfsteps*}
                \item $\cmaysatisfy{e_2}{\xi_2}$ \BY{assumption}
                \end{pfsteps*}
                Contradicts \pfref{[epair1]notmaysat2}.
            \restorelocalsteps{2}
            \item[\text{(\ref{rule:CMSPair3})}]
                \begin{pfsteps*}
                \item $\cmaysatisfy{e_1}{\xi_1}$ \BY{assumption}
                \end{pfsteps*}
                Contradicts \pfref{[epair1]notmaysat1}.
            \end{byCases}
            \begin{pfsteps*}
            \item $\cnotmaysatisfy{\hpair{e_1}{e_2}}{\cpair{\xi_1}{\xi_2}}$ \BY{contradiction}
            \end{pfsteps*}
            
        \restorelocalsteps{1}
        \item[\csatisfy{e_1}{\xi_1},\cmaysatisfy{e_2}{\xi_2}]
            \begin{pfsteps*}
            \item $\csatisfy{e_1}{\xi_1}$ \BY{assumption} \pflabel{[epair2]satisfy1}
            \item $\cnotmaysatisfy{e_1}{\xi_1}$ \BY{assumption} \pflabel{[epair2]notmaysat1}
            \item $\cnotsatisfy{e_2}{\xi_2}$ \BY{assumption} \pflabel{[epair2]notsatisfy2}
            \item $\cmaysatisfy{e_2}{\xi_2}$ \BY{assumption} \pflabel{[epair2]maysat2}
            \item $\cmaysatisfy{\hpair{e_1}{e_2}}{\cpair{\xi_1}{\xi_2}}$ \BY{Rule (\ref{rule:CMSPair2}) on \pfref{[epair2]satisfy1} and \pfref{[epair2]maysat2}} \pflabel{[epair2]maysat}
            \item $\csatisfyormay{\hpair{e_1}{e_2}}{\cpair{\xi_1}{\xi_2}}$ \BY{\ruleref{rule:CSMSMay} on \pfref{[epair2]maysat}}
            \end{pfsteps*}
            Assume $\csatisfy{\hpair{e_1}{e_2}}{\cpair{\xi_1}{\xi_2}}$. By rule induction over Rules (\ref{rules:Satisfy}) on it, only two cases apply. 
           \begin{byCases}
            \savelocalsteps{2}
            \item[\text{(\ref{rule:CSNotValPair})}]
                \begin{pfsteps*}
                \item $\isntVal{\hpair{e_1}{e_2}}$ \BY{assumption}
                \end{pfsteps*}
                Contradicts \autoref{lem:no-pair-notval}.
            \restorelocalsteps{2}
            \item[\text{(\ref{rule:CSPair})}]
                \begin{pfsteps*}
                \item $\csatisfy{e_2}{\xi_2}$ \BY{assumption}
                \end{pfsteps*}
                Contradicts \pfref{[epair2]notsatisfy2}.
            \end{byCases}
            \begin{pfsteps*}
            \item $\cnotsatisfy{\hpair{e_1}{e_2}}{\cpair{\xi_1}{\xi_2}}$ \BY{contradiction}
            \end{pfsteps*}
            
        \restorelocalsteps{1}
        \item[\csatisfy{e_1}{\xi_1},\cnotsatisfyormay{e_2}{\xi_2}]
            \begin{pfsteps*}
            \item $\csatisfy{e_1}{\xi_1}$ \BY{assumption} \pflabel{[epair3]satisfy1}
            \item $\cnotmaysatisfy{e_1}{\xi_1}$ \BY{assumption} \pflabel{[epair3]notmaysat1}
            \item $\cnotsatisfy{e_2}{\xi_2}$ \BY{assumption} \pflabel{[epair3]notsatisfy2}
            \item $\cnotmaysatisfy{e_2}{\xi_2}$ \BY{assumption} \pflabel{[epair3]notmaysat2}
            \end{pfsteps*}
            Assume $\csatisfy{\hpair{e_1}{e_2}}{\cpair{\xi_1}{\xi_2}}$. By rule induction over Rules (\ref{rules:Satisfy}) on it, only two cases apply. 
           \begin{byCases}
            \savelocalsteps{2}
            \item[\text{(\ref{rule:CSNotValPair})}]
                \begin{pfsteps*}
                \item $\isntVal{\hpair{e_1}{e_2}}$ \BY{assumption}
                \end{pfsteps*}
                Contradicts \autoref{lem:no-pair-notval}.
            \restorelocalsteps{2}
            \item[\text{(\ref{rule:CSPair})}]
                \begin{pfsteps*}
                \item $\csatisfy{e_2}{\xi_2}$ \BY{assumption}
                \end{pfsteps*}
                Contradicts \pfref{[epair3]notsatisfy2}.
            \end{byCases}
            \begin{pfsteps*}
            \item $\cnotsatisfy{\hpair{e_1}{e_2}}{\cpair{\xi_1}{\xi_2}}$ \BY{contradiction} \pflabel{[epair3]notsat}
            \end{pfsteps*}
            Assume $\cmaysatisfy{\hpair{e_1}{e_2}}{\cpair{\xi_1}{\xi_2}}$. By rule induction over Rules (\ref{rules:MaySatisfy}) on it, the following cases apply.
            \begin{byCases}
            \savelocalsteps{2}
            \item[\text{(\ref{rule:CMSNotVal})}]
                \begin{pfsteps*}
                \item $\isntVal{\hpair{e_1}{e_2}}$ \BY{assumption}
                \end{pfsteps*}
                Contradicts \autoref{lem:no-pair-notval}.
            \restorelocalsteps{2}
            \item[\text{(\ref{rule:CMSPair1})}]
                \begin{pfsteps*}
                \item $\cmaysatisfy{e_1}{\xi_1}$ \BY{assumption}
                \end{pfsteps*}
                Contradicts \pfref{[epair3]notmaysat1}.
            \restorelocalsteps{2}
            \item[\text{(\ref{rule:CMSPair2})}]
                \begin{pfsteps*}
                \item $\cmaysatisfy{e_2}{\xi_2}$ \BY{assumption}
                \end{pfsteps*}
                Contradicts \pfref{[epair3]notmaysat2}.
            \restorelocalsteps{2}
            \item[\text{(\ref{rule:CMSPair3})}]
                \begin{pfsteps*}
                \item $\cmaysatisfy{e_1}{\xi_1}$ \BY{assumption}
                \end{pfsteps*}
                Contradicts \pfref{[epair3]notmaysat1}.
            \end{byCases}
            \begin{pfsteps*}
            \item $\cnotmaysatisfy{\hpair{e_1}{e_2}}{\cpair{\xi_1}{\xi_2}}$ \BY{contradiction} \pflabel{[epair3]notmaysat}
            \item $\cnotsatisfyormay{\hpair{e_1}{e_2}}{\cpair{\xi_1}{\xi_2}}$ \BY{\autoref{lem:not-satormay} on \pfref{[epair3]notsat} and \pfref{[epair3]notmaysat}}
            \end{pfsteps*}
        \restorelocalsteps{1}
        \item[\cmaysatisfy{e_1}{\xi_1},\csatisfy{e_2}{\xi_2}]
            \begin{pfsteps*}
            \item $\cnotsatisfy{e_1}{\xi_1}$ \BY{assumption} \pflabel{[epair4]notsatisfy1}
            \item $\cmaysatisfy{e_1}{\xi_1}$ \BY{assumption} \pflabel{[epair4]maysat1}
            \item $\csatisfy{e_2}{\xi_2}$ \BY{assumption} \pflabel{[epair4]satisfy2}
            \item $\cnotmaysatisfy{e_2}{\xi_2}$ \BY{assumption} \pflabel{[epair4]notmaysat2}
            \item $\cmaysatisfy{\hpair{e_1}{e_2}}{\cpair{\xi_1}{\xi_2}}$ \BY{Rule (\ref{rule:CMSPair1}) on \pfref{[epair4]maysat1} and \pfref{[epair4]satisfy2}} \pflabel{[epair4]maysat}
            \item $\csatisfyormay{\hpair{e_1}{e_2}}{\cpair{\xi_1}{\xi_2}}$ \BY{\ruleref{rule:CSMSMay} on \pfref{[epair4]maysat}}
            \end{pfsteps*}
            Assume $\csatisfy{\hpair{e_1}{e_2}}{\cpair{\xi_1}{\xi_2}}$. By rule induction over Rules (\ref{rules:Satisfy}) on it, only two cases apply. 
           \begin{byCases}
            \savelocalsteps{2}
            \item[\text{(\ref{rule:CSNotValPair})}]
                \begin{pfsteps*}
                \item $\isntVal{\hpair{e_1}{e_2}}$ \BY{assumption}
                \end{pfsteps*}
                Contradicts \autoref{lem:no-pair-notval}.
            \restorelocalsteps{2}
            \item[\text{(\ref{rule:CSPair})}]
                \begin{pfsteps*}
                \item $\csatisfy{e_1}{\xi_1}$ \BY{assumption}
                \end{pfsteps*}
                Contradicts \pfref{[epair4]notsatisfy1}.
            \end{byCases}
            \begin{pfsteps*}
            \item $\cnotsatisfy{\hpair{e_1}{e_2}}{\cpair{\xi_1}{\xi_2}}$ \BY{contradiction} \pflabel{[epair4]notsat}
            \end{pfsteps*}
            
        \restorelocalsteps{1}
        \item[\cmaysatisfy{e_1}{\xi_1},\cmaysatisfy{e_2}{\xi_2}]
            \begin{pfsteps*}
            \item $\cnotsatisfy{e_1}{\xi_1}$ \BY{assumption} \pflabel{[epair5]notsatisfy1}
            \item $\cmaysatisfy{e_1}{\xi_1}$ \BY{assumption} \pflabel{[epair5]maysat1}
            \item $\cnotsatisfy{e_2}{\xi_2}$ \BY{assumption} \pflabel{[epair5]notsatisfy2}
            \item $\cmaysatisfy{e_2}{\xi_2}$ \BY{assumption} \pflabel{[epair5]maysat2}
            \item $\cmaysatisfy{\hpair{e_1}{e_2}}{\cpair{\xi_1}{\xi_2}}$ \BY{Rule (\ref{rule:CMSPair3}) on \pfref{[epair5]maysat1} and \pfref{[epair5]maysat2}} \pflabel{[epair5]maysat}
            \item $\csatisfyormay{\hpair{e_1}{e_2}}{\cpair{\xi_1}{\xi_2}}$ \BY{\ruleref{rule:CSMSMay} on \pfref{[epair5]maysat}}
            \end{pfsteps*}
            Assume $\csatisfy{\hpair{e_1}{e_2}}{\cpair{\xi_1}{\xi_2}}$. By rule induction over Rules (\ref{rules:Satisfy}) on it, only two cases apply. 
           \begin{byCases}
            \savelocalsteps{2}
            \item[\text{(\ref{rule:CSNotValPair})}]
                \begin{pfsteps*}
                \item $\isntVal{\hpair{e_1}{e_2}}$ \BY{assumption}
                \end{pfsteps*}
                Contradicts \autoref{lem:no-pair-notval}.
            \restorelocalsteps{2}
            \item[\text{(\ref{rule:CSPair})}]
                \begin{pfsteps*}
                \item $\csatisfy{e_1}{\xi_1}$ \BY{assumption}
                \end{pfsteps*}
                Contradicts \pfref{[epair5]notsatisfy1}.
            \end{byCases}
            \begin{pfsteps*}
            \item $\cnotsatisfy{\hpair{e_1}{e_2}}{\cpair{\xi_1}{\xi_2}}$ \BY{contradiction} \pflabel{[epair5]notsat}
            \end{pfsteps*}
            
        \restorelocalsteps{1}
        \item[\cmaysatisfy{e_1}{\xi_1},\cnotsatisfyormay{e_2}{\xi_2}]
            \begin{pfsteps*}
            \item $\cnotsatisfy{e_1}{\xi_1}$ \BY{assumption} \pflabel{[epair6]notsatisfy1}
            \item $\cmaysatisfy{e_1}{\xi_1}$ \BY{assumption} \pflabel{[epair6]maysat1}
            \item $\cnotsatisfy{e_2}{\xi_2}$ \BY{assumption} \pflabel{[epair6]notsatisfy2}
            \item $\cnotmaysatisfy{e_2}{\xi_2}$ \BY{assumption} \pflabel{[epair6]notmaysat2}
            \end{pfsteps*}
            Assume $\csatisfy{\hpair{e_1}{e_2}}{\cpair{\xi_1}{\xi_2}}$. By rule induction over Rules (\ref{rules:Satisfy}) on it, only two cases apply. 
           \begin{byCases}
            \savelocalsteps{2}
            \item[\text{(\ref{rule:CSNotValPair})}]
                \begin{pfsteps*}
                \item $\isntVal{\hpair{e_1}{e_2}}$ \BY{assumption}
                \end{pfsteps*}
                Contradicts \autoref{lem:no-pair-notval}.
            \restorelocalsteps{2}
            \item[\text{(\ref{rule:CSPair})}]
                \begin{pfsteps*}
                \item $\csatisfy{e_1}{\xi_1}$ \BY{assumption}
                \end{pfsteps*}
                Contradicts \pfref{[epair6]notsatisfy1}.
            \end{byCases}
            \begin{pfsteps*}
            \item $\cnotsatisfy{\hpair{e_1}{e_2}}{\cpair{\xi_1}{\xi_2}}$ \BY{contradiction} \pflabel{[epair6]notsat}
            \end{pfsteps*}
            Assume $\cmaysatisfy{\hpair{e_1}{e_2}}{\cpair{\xi_1}{\xi_2}}$. By rule induction over Rules (\ref{rules:MaySatisfy}) on it, the following cases apply.
            \begin{byCases}
            \savelocalsteps{2}
            \item[\text{(\ref{rule:CMSNotVal})}]
                \begin{pfsteps*}
                \item $\isntVal{\hpair{e_1}{e_2}}$ \BY{assumption}
                \end{pfsteps*}
                Contradicts \autoref{lem:no-pair-notval}.
            \restorelocalsteps{2}
            \item[\text{(\ref{rule:CMSPair1})}]
                \begin{pfsteps*}
                \item $\csatisfy{e_2}{\xi_2}$ \BY{assumption}
                \end{pfsteps*}
                Contradicts \pfref{[epair6]notsatisfy2}.
            \restorelocalsteps{2}
            \item[\text{(\ref{rule:CMSPair2})}]
                \begin{pfsteps*}
                \item $\cmaysatisfy{e_2}{\xi_2}$ \BY{assumption}
                \end{pfsteps*}
                Contradicts \pfref{[epair6]notmaysat2}.
            \restorelocalsteps{2}
            \item[\text{(\ref{rule:CMSPair3})}]
                \begin{pfsteps*}
                \item $\cmaysatisfy{e_2}{\xi_2}$ \BY{assumption}
                \end{pfsteps*}
                Contradicts \pfref{[epair6]notmaysat2}.
            \end{byCases}
            \begin{pfsteps*}
            \item $\cnotmaysatisfy{\hpair{e_1}{e_2}}{\cpair{\xi_1}{\xi_2}}$ \BY{contradiction} \pflabel{[epair6]notmaysat}
            \item $\cnotsatisfyormay{\hpair{e_1}{e_2}}{\cpair{\xi_1}{\xi_2}}$ \BY{\autoref{lem:not-satormay} on \pfref{[epair6]notsat} and \pfref{[epair6]notmaysat}}
            \end{pfsteps*}
            
        \restorelocalsteps{1}
        \item[\cnotsatisfyormay{e_1}{\xi_1},\csatisfy{e_2}{\xi_2}]
            \begin{pfsteps*}
            \item $\cnotsatisfy{e_1}{\xi_1}$ \BY{assumption} \pflabel{[epair7]notsatisfy1}
            \item $\cnotmaysatisfy{e_1}{\xi_1}$ \BY{assumption} \pflabel{[epair7]notmaysat1}
            \item $\csatisfy{e_2}{\xi_2}$ \BY{assumption} \pflabel{[epair7]satisfy2}
            \item $\cnotmaysatisfy{e_2}{\xi_2}$ \BY{assumption} \pflabel{[epair7]notmaysat2}
            \end{pfsteps*}
            Assume $\csatisfy{\hpair{e_1}{e_2}}{\cpair{\xi_1}{\xi_2}}$. By rule induction over Rules (\ref{rules:Satisfy}) on it, only two cases apply. 
           \begin{byCases}
            \savelocalsteps{2}
            \item[\text{(\ref{rule:CSNotValPair})}]
                \begin{pfsteps*}
                \item $\isntVal{\hpair{e_1}{e_2}}$ \BY{assumption}
                \end{pfsteps*}
                Contradicts \autoref{lem:no-pair-notval}.
            \restorelocalsteps{2}
            \item[\text{(\ref{rule:CSPair})}]
                \begin{pfsteps*}
                \item $\csatisfy{e_1}{\xi_1}$ \BY{assumption}
                \end{pfsteps*}
                Contradicts \pfref{[epair7]notsatisfy1}.
            \end{byCases}
            \begin{pfsteps*}
            \item $\cnotsatisfy{\hpair{e_1}{e_2}}{\cpair{\xi_1}{\xi_2}}$ \BY{contradiction} \pflabel{[epair7]notsat}
            \end{pfsteps*}
            Assume $\cmaysatisfy{\hpair{e_1}{e_2}}{\cpair{\xi_1}{\xi_2}}$. By rule induction over Rules (\ref{rules:MaySatisfy}) on it, the following cases apply.
            \begin{byCases}
            \savelocalsteps{2}
            \item[\text{(\ref{rule:CMSNotVal})}]
                \begin{pfsteps*}
                \item $\isntVal{\hpair{e_1}{e_2}}$ \BY{assumption}
                \end{pfsteps*}
                Contradicts \autoref{lem:no-pair-notval}.
            \restorelocalsteps{2}
            \item[\text{(\ref{rule:CMSPair1})}]
                \begin{pfsteps*}
                \item $\cmaysatisfy{e_1}{\xi_1}$ \BY{assumption}
                \end{pfsteps*}
                Contradicts \pfref{[epair7]notmaysat1}.
            \restorelocalsteps{2}
            \item[\text{(\ref{rule:CMSPair2})}]
                \begin{pfsteps*}
                \item $\cmaysatisfy{e_2}{\xi_2}$ \BY{assumption}
                \end{pfsteps*}
                Contradicts \pfref{[epair7]notmaysat2}.
            \restorelocalsteps{2}
            \item[\text{(\ref{rule:CMSPair3})}]
                \begin{pfsteps*}
                \item $\cmaysatisfy{e_1}{\xi_1}$ \BY{assumption}
                \end{pfsteps*}
                Contradicts \pfref{[epair7]notmaysat1}.
            \end{byCases}
            \begin{pfsteps*}
            \item $\cnotmaysatisfy{\hpair{e_1}{e_2}}{\cpair{\xi_1}{\xi_2}}$ \BY{contradiction} \pflabel{[epair7]notmaysat}
            \item $\cnotsatisfyormay{\hpair{e_1}{e_2}}{\cpair{\xi_1}{\xi_2}}$ \BY{\autoref{lem:not-satormay} on \pfref{[epair7]notsat} and \pfref{[epair7]notmaysat}}
            \end{pfsteps*}
            
        \restorelocalsteps{1}
        \item[\cnotsatisfyormay{e_1}{\xi_1},\cmaysatisfy{e_2}{\xi_2}]
            \begin{pfsteps*}
            \item $\cnotsatisfy{e_1}{\xi_1}$ \BY{assumption} \pflabel{[epair8]notsatisfy1}
            \item $\cnotmaysatisfy{e_1}{\xi_1}$ \BY{assumption} \pflabel{[epair8]notmaysat1}
            \item $\cnotsatisfy{e_2}{\xi_2}$ \BY{assumption} \pflabel{[epair8]notsatisfy2}
            \item $\cmaysatisfy{e_2}{\xi_2}$ \BY{assumption} \pflabel{[epair8]maysat2}
            \end{pfsteps*}
            Assume $\csatisfy{\hpair{e_1}{e_2}}{\cpair{\xi_1}{\xi_2}}$. By rule induction over Rules (\ref{rules:Satisfy}) on it, only two cases apply. 
           \begin{byCases}
            \savelocalsteps{2}
            \item[\text{(\ref{rule:CSNotValPair})}]
                \begin{pfsteps*}
                \item $\isntVal{\hpair{e_1}{e_2}}$ \BY{assumption}
                \end{pfsteps*}
                Contradicts \autoref{lem:no-pair-notval}.
            \restorelocalsteps{2}
            \item[\text{(\ref{rule:CSPair})}]
                \begin{pfsteps*}
                \item $\csatisfy{e_2}{\xi_2}$ \BY{assumption}
                \end{pfsteps*}
                Contradicts \pfref{[epair8]notsatisfy2}.
            \end{byCases}
            \begin{pfsteps*}
            \item $\cnotsatisfy{\hpair{e_1}{e_2}}{\cpair{\xi_1}{\xi_2}}$ \BY{contradiction} \pflabel{[epair8]notsat}
            \end{pfsteps*}
            Assume $\cmaysatisfy{\hpair{e_1}{e_2}}{\cpair{\xi_1}{\xi_2}}$. By rule induction over Rules (\ref{rules:MaySatisfy}) on it, the following cases apply.
            \begin{byCases}
            \savelocalsteps{2}
            \item[\text{(\ref{rule:CMSNotVal})}]
                \begin{pfsteps*}
                \item $\isntVal{\hpair{e_1}{e_2}}$ \BY{assumption}
                \end{pfsteps*}
                Contradicts \autoref{lem:no-pair-notval}.
            \restorelocalsteps{2}
            \item[\text{(\ref{rule:CMSPair1})}]
                \begin{pfsteps*}
                \item $\cmaysatisfy{e_1}{\xi_1}$ \BY{assumption}
                \end{pfsteps*}
                Contradicts \pfref{[epair8]notmaysat1}.
            \restorelocalsteps{2}
            \item[\text{(\ref{rule:CMSPair2})}]
                \begin{pfsteps*}
                \item $\csatisfy{e_1}{\xi_1}$ \BY{assumption}
                \end{pfsteps*}
                Contradicts \pfref{[epair8]notsatisfy1}.
            \restorelocalsteps{2}
            \item[\text{(\ref{rule:CMSPair3})}]
                \begin{pfsteps*}
                \item $\cmaysatisfy{e_1}{\xi_1}$ \BY{assumption}
                \end{pfsteps*}
                Contradicts \pfref{[epair8]notmaysat1}.
            \end{byCases}
            \begin{pfsteps*}
            \item $\cnotmaysatisfy{\hpair{e_1}{e_2}}{\cpair{\xi_1}{\xi_2}}$ \BY{contradiction} \pflabel{[epair8]notmaysat}
            \item $\cnotsatisfyormay{\hpair{e_1}{e_2}}{\cpair{\xi_1}{\xi_2}}$ \BY{\autoref{lem:not-satormay} on \pfref{[epair8]notsat} and \pfref{[epair8]notmaysat}}
            \end{pfsteps*}
            
        \restorelocalsteps{1}
        \item[\cnotsatisfyormay{e_1}{\xi_1},\cnotsatisfyormay{e_2}{\xi_2}]
            \begin{pfsteps*}
            \item $\cnotsatisfy{e_1}{\xi_1}$ \BY{assumption} \pflabel{[epair9]notsatisfy1}
            \item $\cnotmaysatisfy{e_1}{\xi_1}$ \BY{assumption} \pflabel{[epair9]notmaysat1}
            \item $\cnotsatisfy{e_2}{\xi_2}$ \BY{assumption} \pflabel{[epair9]notsatisfy2}
            \item $\cnotmaysatisfy{e_2}{\xi_2}$ \BY{assumption} \pflabel{[epair9]notmaysat2}
            \end{pfsteps*}
            Assume $\csatisfy{\hpair{e_1}{e_2}}{\cpair{\xi_1}{\xi_2}}$. By rule induction over Rules (\ref{rules:Satisfy}) on it, only two cases apply. 
           \begin{byCases}
            \savelocalsteps{2}
            \item[\text{(\ref{rule:CSNotValPair})}]
                \begin{pfsteps*}
                \item $\isntVal{\hpair{e_1}{e_2}}$ \BY{assumption}
                \end{pfsteps*}
                Contradicts \autoref{lem:no-pair-notval}.
            \restorelocalsteps{2}
            \item[\text{(\ref{rule:CSPair})}]
                \begin{pfsteps*}
                \item $\csatisfy{e_2}{\xi_2}$ \BY{assumption}
                \end{pfsteps*}
                Contradicts \pfref{[epair9]notsatisfy2}.
            \end{byCases}
            \begin{pfsteps*}
            \item $\cnotsatisfy{\hpair{e_1}{e_2}}{\cpair{\xi_1}{\xi_2}}$ \BY{contradiction} \pflabel{[epair9]notsat}
            \end{pfsteps*}
            Assume $\cmaysatisfy{\hpair{e_1}{e_2}}{\cpair{\xi_1}{\xi_2}}$. By rule induction over Rules (\ref{rules:MaySatisfy}) on it, the following cases apply.
            \begin{byCases}
            \savelocalsteps{2}
            \item[\text{(\ref{rule:CMSNotVal})}]
                \begin{pfsteps*}
                \item $\isntVal{\hpair{e_1}{e_2}}$ \BY{assumption}
                \end{pfsteps*}
                Contradicts \autoref{lem:no-pair-notval}.
            \restorelocalsteps{2}
            \item[\text{(\ref{rule:CMSPair1})}]
                \begin{pfsteps*}
                \item $\cmaysatisfy{e_1}{\xi_1}$ \BY{assumption}
                \end{pfsteps*}
                Contradicts \pfref{[epair9]notmaysat1}.
            \restorelocalsteps{2}
            \item[\text{(\ref{rule:CMSPair2})}]
                \begin{pfsteps*}
                \item $\cmaysatisfy{e_2}{\xi_2}$ \BY{assumption}
                \end{pfsteps*}
                Contradicts \pfref{[epair9]notmaysat2}.
            \restorelocalsteps{2}
            \item[\text{(\ref{rule:CMSPair3})}]
                \begin{pfsteps*}
                \item $\cmaysatisfy{e_1}{\xi_1}$ \BY{assumption}
                \end{pfsteps*}
                Contradicts \pfref{[epair9]notmaysat1}.
            \end{byCases}
            \begin{pfsteps*}
            \item $\cnotmaysatisfy{\hpair{e_1}{e_2}}{\cpair{\xi_1}{\xi_2}}$ \BY{contradiction} \pflabel{[epair9]notmaysat}
            \item $\cnotsatisfyormay{\hpair{e_1}{e_2}}{\cpair{\xi_1}{\xi_2}}$ \BY{\autoref{lem:not-satormay} on \pfref{[epair9]notsat} and \pfref{[epair9]notmaysat}}
            \end{pfsteps*}
        \end{byCases}
    \end{byCases}
\resetpfcounter
\end{byCases}
\end{proof}

\begin{definition}[Entailment of Constraints]
  \label{defn:const-entailment}
  Suppose that $\ctyp{\xi_1}{\tau}$ and $\ctyp{\xi_2}{\tau}$.
  Then $\csatisfy{\xi_1}{\xi_2}$ iff for all $e$ such that $\hexptyp{\cdot}{\Delta}{e}{\tau}$ and $\isVal{e}$ we have $\csatisfyormay{e}{\xi_1}$ implies $\csatisfy{e}{\xi_2}$
\end{definition}

\begin{definition}[Potential Entailment of Constraints]
  \label{defn:nn-entailment}
  Suppose that $\ctyp{\xi_1}{\tau}$ and $\ctyp{\xi_2}{\tau}$. Then $\csatisfyormay{\xi_1}{\xi_2}$ iff for all $e$ such that $\hexptyp{\cdot}{\Delta}{e}{\tau}$ and $\isFinal{e}$ we have $\csatisfyormay{e}{\xi_1}$ implies $\csatisfyormay{e}{\xi_2}$ 
\end{definition}

\begin{corollary}
  \label{corol:nn-exhaust}
  Suppose that $\ctyp{\xi}{\tau}$ and $\hexptyp{\cdot}{\Delta}{e}{\tau}$ and $\isFinal{e}$. Then $\csatisfyormay{\ctruth}{\xi}$ implies $\csatisfyormay{e}{\xi}$
\end{corollary}
\begin{proof}
  \begin{pfsteps*}
  \item $\ctyp{\xi}{\tau}$ \BY{assumption} \pflabel{CTyp}
  \item $\hexptyp{\cdot}{\Gamma}{e}{\tau}$ \BY{assumption} \pflabel{eTyp}
  \item $\isFinal{e}$ \BY{assumption} \pflabel{eFinal}
  \item $\csatisfyormay{\ctruth}{\xi}$ \BY{assumption} \pflabel{entailormay}
  \item $\csatisfy{e_1}{\ctruth}$ \BY{Rule (\ref{rule:CSTruth})} \pflabel{satisfytop}
  \item $\csatisfyormay{e_1}{\ctruth}$ \BY{Rule (\ref{rule:CSMSSat}) on \pfref{satisfytop}} \pflabel{satisfyormaytop}
  \item $\ctyp{\ctruth}{\tau}$ \BY{Rule (\ref{rule:CTTruth})} \pflabel{topCTyp}
  \item $\csatisfyormay{e_1}{\xi_r}$ \BY{Definition \ref{defn:nn-entailment} of \pfref{entailormay} on \pfref{topCTyp} and \pfref{CTyp} and \pfref{eTyp} and \pfref{eFinal} and \pfref{satisfyormaytop}}
  \end{pfsteps*}
  \resetpfcounter
\end{proof}
\section{Static Semantics}
$\arraycolsep=4pt\begin{array}{lll}
\tau & ::= &
  \tnum ~\vert~
  \tarr{\tau_1}{\tau_2} ~\vert~
  \tprod{\tau_1}{\tau_2} ~\vert~
  \tsum{\tau_1}{\tau_2} \\
e & ::= &
  x ~\vert~
  \hnum{n} \\
  & ~\vert~ &
  \hlam{x}{\tau}{e} ~\vert~
  \hap{e_1}{e_2} \\
  & ~\vert~ &
  \hpair{e_1}{e_2} \\
  & ~\vert~ &
  \hinl{\tau}{e} ~\vert~
  \hinr{\tau}{e} ~\vert~
  \hmatch{e}{\hat{rs}} \\
  & ~\vert~ &
  \hehole{u} ~\vert~
  \hhole{e}{u} \\
\hat{rs} & ::= &
  \inparens{\zruls{rs}{r}{rs}} \\
rs & ::= &
  \cdot ~\vert~ \hrulesP{r}{rs'} \\
r & ::= &
  \hrul{p}{e} \\
p & ::= &
  x ~\vert~
  \hnum{n} ~\vert~
  \_ ~\vert~
  \hpair{p_1}{p_2} ~\vert~
  \hinlp{p} ~\vert~
  \hinrp{p} ~\vert~
  \hehole{w} ~\vert~
  \hhole{p}{w}
\end{array}$

\judgboxa{\rmpointer{\zrules} = rs}
        {$rs$ can be obtained by erasing pointer from $\zrules$}
\begin{subequations}\label{defn:rmpointer}
\begin{align}
  \rmpointer{\zruls{\cdot}{r}{rs}} &= \hrules{r}{rs} \\
  \rmpointer{\zruls{\hrulesP{r'}{rs'}}{r}{rs}} &= \hrules{r'}{\rmpointer{\zruls{rs'}{r}{rs}}}
\end{align}
\end{subequations}

\judgboxa{
  \hexptyp{\Gamma}{\Delta}{e}{\tau}
}{
  $e$ is of type \(\tau\)
}
\begin{subequations}\label{rules:TExp}
\begin{equation}\label{rule:TVar}
\inferrule[TVar]{ }{
  \hexptyp{\Gamma, x : \tau}{\Delta}{x}{\tau}
}
\end{equation}
\begin{equation}\label{rule:TEHole}
\inferrule[TEHole]{ }{
  \hexptyp{\Gamma}{\Delta, u::\tau}{\hehole{u}}{\tau}
}
\end{equation}
\begin{equation}\label{rule:THole}
\inferrule[THole]{
  \hexptyp{\Gamma}{\Delta, u::\tau}{e}{\tau'}
}{
  \hexptyp{\Gamma}{\Delta, u::\tau}{\hhole{e}{u}}{\tau}
}
\end{equation}
\begin{equation}\label{rule:TNum}
\inferrule[TNum]{ }{
  \hexptyp{\Gamma}{\Delta}{\hnum{n}}{\tnum}
}
\end{equation}
\begin{equation}\label{rule:TLam}
\inferrule[TLam]{
  \hexptyp{\Gamma, x : \tau_1}{\Delta}{e}{\tau_2}
}{
  \hexptyp{\Gamma}{\Delta}{\hlam{x}{\tau_1}{e}}{\tarr{\tau_1}{\tau_2}}
}
\end{equation}
\begin{equation}\label{rule:TAp}
\inferrule[TAp]{
  \hexptyp{\Gamma}{\Delta}{e_1}{\tarr{\tau_2}{\tau}} \\
  \hexptyp{\Gamma}{\Delta}{e_2}{\tau_2}
}{
  \hexptyp{\Gamma}{\Delta}{\hap{e_1}{e_2}}{\tau}
}
\end{equation}
\begin{equation}\label{rule:TPair}
\inferrule[TPair]{
  \hexptyp{\Gamma}{\Delta}{e_1}{\tau_1} \\
  \hexptyp{\Gamma}{\Delta}{e_2}{\tau_2}
}{
  \hexptyp{\Gamma}{\Delta}{\hpair{e_1}{e_2}}{\tprod{\tau_1}{\tau_2}}
}
\end{equation}
\begin{equation}\label{rule:TPrl}
\inferrule[TPrl]{
    \hexptyp{\Gamma}{\Delta}{e}{\tprod{\tau_1}{\tau_2}}
}{
    \hexptyp{\Gamma}{\Delta}{\hprl{e}}{\tau_1}
} 
\end{equation}
\begin{equation}\label{rule:TPrr}
  \inferrule[TPrr]{
    \hexptyp{\Gamma}{\Delta}{e}{\tprod{\tau_1}{\tau_2}}
  }{
    \hexptyp{\Gamma}{\Delta}{\hprr{e}}{\tau_2}
  }
\end{equation}
\begin{equation}\label{rule:TInl}
\inferrule[TInl]{
  \hexptyp{\Gamma}{\Delta}{e}{\tau_1}
}{
  \hexptyp{\Gamma}{\Delta}{\hinl{\tau_2}{e}}{\tsum{\tau_1}{\tau_2}}
}
\end{equation}
\begin{equation}\label{rule:TInr}
\inferrule[TInr]{
  \hexptyp{\Gamma}{\Delta}{e}{\tau_2}
}{
  \hexptyp{\Gamma}{\Delta}{\hinr{\tau_1}{e}}{\tsum{\tau_1}{\tau_2}}
}
\end{equation}
\begin{equation}\label{rule:TMatchZPre}
\inferrule[TMatchZPre]{
  \hexptyp{\Gamma}{\Delta}{e}{\tau} \\
  \chrulstyp{\Gamma}{\Delta}{\cfalsity}{\hrules{r}{rs}}{\tau}{\xi}{\tau'} \\
  \csatisfyormay{\ctruth}{\xi}
}{
\hexptyp{\Gamma}{\Delta}{\hmatch{e}{\zruls{\cdot}{r}{rs}}}{\tau'}
}
\end{equation}
\begin{equation}\label{rule:TMatchNZPre}
\inferrule[TMatchNZPre]{
  \hexptyp{\Gamma}{\Delta}{e}{\tau} \\
  \isFinal{e} \\
  \chrulstyp{\Gamma}{\Delta}{\cfalsity}{rs_{pre}}{\tau}{\xi_{pre}}{\tau'} \\
  \chrulstyp{\Gamma}{\Delta}{\cor{\cfalsity}{\xi_{pre}}}{\hrules{r}{rs_{post}}}{\tau}{\xi_{rest}}{\tau'} \\
  \cnotsatisfyormay{e}{\xi_{pre}} \\
  \csatisfyormay{\ctruth}{\cor{\xi_{pre}}{\xi_{rest}}}
}{
  \hexptyp{\Gamma}{\Delta}{\hmatch{e}{\zruls{rs_{pre}}{r}{rs_{post}}}}{\tau'}
}
\end{equation}
\end{subequations}

\judgboxa{
    \chpattyp{p}{\tau}{\xi}{\Gamma}{\Delta}
  }{
    $p$ is assigned type $\tau$ and emits constraint $\xi$
  }
\begin{subequations}\label{rules:PatTyp}
\begin{equation}\label{rule:PTVar}
\inferrule[PTVar]{ }{
  \chpattyp{x}{\tau}{\ctruth}{\cdot}{x : \tau}
}
\end{equation}
\begin{equation}\label{rule:PTWild}
\inferrule[PTWild]{ }{
  \chpattyp{\_}{\tau}{\ctruth}{\cdot}{\cdot}
}
\end{equation}
\begin{equation}\label{rule:PTEHole}
\inferrule[PTEHole]{ }{
  \chpattyp{\hehole{w}}{\tau}{\cunknown}{\cdot}{w :: \tau}
}
\end{equation}
\begin{equation}\label{rule:PTHole}
\inferrule[PTHole]{
  \chpattyp{p}{\tau}{\xi}{\Gamma}{\Delta}
}{
  \chpattyp{\hhole{p}{w}}{\tau'}{\cunknown}
  {\Gamma}{\Delta , w :: \tau'}
}
\end{equation}
\begin{equation}\label{rule:PTNum}
\inferrule[PTNum]{ }{
  \chpattyp{\hnum{n}}{\tnum}{\cnum{n}}{\cdot}{\cdot}
}
\end{equation}
\begin{equation}\label{rule:PTInl}
\inferrule[PTInl]{
  \chpattyp{p}{\tau_1}{\xi}{\Gamma}{\Delta}
}{
  \chpattyp{\hinlp{p}}{\tsum{\tau_1}{\tau_2}}{\cinl{\xi}}{\Gamma}{\Delta}
}
\end{equation}
\begin{equation}\label{rule:PTInr}
\inferrule[PTInr]{
  \chpattyp{p}{\tau_2}{\xi}{\Gamma}{\Delta}
}{
  \chpattyp{\hinrp{p}}{\tsum{\tau_1}{\tau_2}}{\cinr{\xi}}{\Gamma}{\Delta}
}
\end{equation}
\begin{equation}\label{rule:PTPair}
\inferrule[PTPair]{
  \chpattyp{p_1}{\tau_1}{\xi_1}{\Gamma_1}{\Delta_1} \\
  \chpattyp{p_2}{\tau_2}{\xi_2}{\Gamma_2}{\Delta_2}
}{
  \chpattyp{\hpair{p_1}{p_2}}{\tprod{\tau_1}{\tau_2}}
  {\cpair{\xi_1}{\xi_2}}{\Gamma_1 \uplus \Gamma_2}{\Delta_1 \uplus \Delta_2}
}
\end{equation}
\end{subequations}

\judgboxa{
\chrultyp{\Gamma}{\Delta}{\hrulP{p}{e}}{\tau}{\xi}{\tau'}
}{$r$ transforms a final expression of type $\tau$ \\ to a final expression of type $\tau'$}
\begin{subequations}\label{rules:CTRule}
\begin{equation}\label{rule:CTRule}
\inferrule[CTRule]{
    \chpattyp{p}{\tau}{\xi}{\Gamma_p}{\Delta_p} \\
    \hexptyp{\Gamma \uplus \Gamma_p}{\Delta \uplus \Delta_p}{e}{\tau'}
}{
  \chrultyp{\Gamma}{\Delta}{\hrul{p}{e}}{\tau}{\xi}{\tau'}
}
\end{equation}
\end{subequations}

\judgboxa{\chrulstyp{\Gamma}{\Delta}{\xi_{pre}}{rs}{\tau}{\xi_{rs}}{\tau'}}
{$rs$ transforms a final expression of type $\tau$ \\ to a final expression of type $\tau'$}
\begin{subequations}\label{rules:CTRules}
\begin{equation}\label{rule:CTOneRules}
\inferrule[CTOneRules]{
  \chrultyp{\Gamma}{\Delta}{r}{\tau}{\xi_r}{\tau'} \\
  \cnotsatisfy{\xi_r}{\xi_{pre}}
}{
  \chrulstyp{\Gamma}{\Delta}{\xi_{pre}}{\hrulesP{r}{\cdot}}{\tau}{\xi_r}{\tau'}
}
\end{equation}
\begin{equation}\label{rule:CTRules}
\inferrule[CTRules]{
  \chrultyp{\Gamma}{\Delta}{r}{\tau}{\xi_r}{\tau'} \\
  \chrulstyp{\Gamma}{\Delta}{\cor{\xi_{pre}}{\xi_r}}{rs}
  {\tau}{\xi_{rs}}{\tau'} \\
  \cnotsatisfy{\xi_r}{\xi_{pre}}
}{
  \chrulstyp{\Gamma}{\Delta}{\xi_{pre}}{\hrules{r}{rs}}
  {\tau}{\cor{\xi_r}{\xi_{rs}}}{\tau'}
}
\end{equation}
\end{subequations}

\begin{lemma}
  \label{lem:pat-xi-type}
  If $\chpattyp{p}{\tau}{\xi}{\Gamma}{\Delta}$ then $\ctyp{\xi}{\tau}$.
\end{lemma}
\begin{proof}
By rule induction over \rulesref{rules:PatTyp}.
\end{proof}

\begin{lemma}
  \label{lem:rule-constraint-typ}
  If $\chrultyp{\cdot}{\Delta}{r}{\tau_1}{\xi_r}{\tau}$ then $\ctyp{\xi_r}{\tau_1}$.
\end{lemma}
\begin{proof}
By rule induction over \rulesref{rules:CTRule}.
\end{proof}

\begin{lemma}
  \label{lem:rules-constraint-typ}
  If $\chrulstyp{\cdot}{\Delta}{\xi_{pre}}{rs}{\tau_1}{\xi_{rs}}{\tau}$ then $\ctyp{\xi_{rs}}{\tau_1}$.
\end{lemma}
\begin{proof}
By rule induction over \rulesref{rules:CTRules}.
\end{proof}

\begin{lemma}
  \label{lem:rule-append}
  If $\chrulstyp{\Gamma}{\Delta}{\xi_{pre}}{rs}{\tau}{\xi_{rs}}{\tau'}$ and $\chrultyp{\Gamma}{\Delta}{r}{\tau}{\xi_r}{\tau'}$ and $\cnotsatisfy{\xi_r}{\cor{\xi_{pre}}{\xi_{rs}}}$ then $\chrulstyp{\Gamma}{\Delta}{\xi_{pre}}{\rmpointer{\zruls{rs}{r}{\cdot}}}{\tau}{\cor{\xi_{rs}}{\xi_r}}{\tau'}$
\end{lemma}
\begin{proof}
  \begin{pfsteps*}
  \item $\chrulstyp{\Gamma}{\Delta}{\xi_{pre}}{rs}{\tau}{\xi_{rs}}{\tau'}$ \BY{assumption} \pflabel{rsType}
  \item $\chrultyp{\Gamma}{\Delta}{r}{\tau}{\xi_r}{\tau'}$ \BY{assumption} \pflabel{rType}
  \item $\cnotsatisfy{\xi_r}{\cor{\xi_{pre}}{\xi_{rs}}}$ \BY{assumption} \pflabel{r|/=pre+rs}
  \end{pfsteps*}
  By rule induction over Rules (\ref{rules:CTRules}) on \pfref{rsType}.
  \begin{byCases}
    
  \savelocalsteps{lem:rule-append-1}
  \item[\text{(\ref{rule:CTOneRules})}]
    \begin{pfsteps*}
    \item $rs = \hrules{r'}{\cdot}$ \BY{assumption}
    \item $\xi_{rs} = \xi_r'$ \BY{assumption}
    \item $\chrultyp{\Gamma}{\Delta}{r'}{\tau}{\xi_r'}{\tau'}$ \BY{assumption} \pflabel{[one]r'Type}
    \item $\cnotsatisfy{\xi_r'}{\xi_{pre}}$ \BY{assumption} \pflabel{r'|/=pre}
    \item $\chrulstyp{\Gamma}{\Delta}{\cor{\xi_{pre}}{\xi_r'}}{\hrules{r}{\cdot}}{\tau}{\xi_r}{\tau'}$ \BY{Rule (\ref{rule:CTOneRules}) on \pfref{rType} and \pfref{r|/=pre+rs}} \pflabel{r+dotType}
    \item $\chrulstyp{\Gamma}{\Delta}{\xi_{pre}}{\hrulesP{r'}{\hrules{r}{\cdot}}}{\tau}{\cor{\xi_r'}{\xi_r}}{\tau'}$ \BY{Rule (\ref{rule:CTRules}) on \pfref{r'Type} and \pfref{r+dotType} and \pfref{r'|/=pre}} \pflabel{[one]conc}
    \item $\chrulstyp{\Gamma}{\Delta}{\xi_{pre}}{\rmpointer{\zruls{\hrulesP{r'}{\cdot}}{r}{\cdot}}}{\tau}{\cor{\xi_r'}{\xi_r}}{\tau'}$ \BY{Definition \ref{defn:rmpointer} on \pfref{[one]conc}}
    \end{pfsteps*}

  \restorelocalsteps{lem:rule-append-1}
  \item[\text{(\ref{rule:CTRules})}]
    \begin{pfsteps*}
    \item $rs = \hrules{r'}{rs'}$ \BY{assumption}
    \item $\xi_{rs} = \cor{\xi_r'}{\xi_{rs}'}$ \BY{assumption}
    \item $\chrultyp{\Gamma}{\Delta}{r'}{\tau}{\xi_r'}{\tau'}$ \BY{assumption} \pflabel{r'Type}
    \item $\chrulstyp{\Gamma}{\Delta}{\cor{\xi_{pre}}{\xi_r'}}{rs'}{\tau}{\xi_{rs}'}{\tau'}$ \BY{assumption} \pflabel{rs'Type}
    \item $\cnotsatisfy{\xi_r'}{\xi_{pre}}$ \BY{assumption} \pflabel{prenotredundant}
    \item $\chrulstyp{\Gamma}{\Delta}{\cor{\xi_{pre}}{\xi_r'}}{\rmpointer{\zruls{rs'}{r}{\cdot}}}{\tau}{\cor{\xi_{rs}'}{\xi_r}}{\tau'}$ \BY{IH on \pfref{rs'Type} and \pfref{rType} and \pfref{r|/=pre+rs}} \pflabel{rs'+rType}
    \item $\chrulstyp{\Gamma}{\Delta}{\xi_{pre}}{\hrulesP{r'}{\rmpointer{\zruls{rs'}{r}{\cdot}}}}{\tau}{\cor{\xi_r'}{\cor{\xi_{rs}'}{\xi_r}}}{\tau'}$ \BY{Rule (\ref{rule:CTRules}) on \pfref{r'Type} and \pfref{rs'+rType} and \pfref{prenotredundant}} \pflabel{conc}
    \item $\chrulstyp{\Gamma}{\Delta}{\xi_{pre}}{\rmpointer{\zruls{\hrulesP{r'}{rs'}}{r}{\cdot}}}{\tau}{\cor{\xi_r'}{\cor{\xi_{rs}'}{\xi_r}}}{\tau'}$ \BY{Definition \ref{defn:rmpointer} on \pfref{conc}}
    \end{pfsteps*}
  \resetpfcounter
  \end{byCases}
\end{proof}

\begin{lemma}[Substitution]
  \label{lem:substitution}
  If $\hexptyp{\Gamma, x : \tau}{\Delta}{e_0}{\tau_0}$ and $\hexptyp{\Gamma}{\Delta}{e}{\tau}$
  then $\hexptyp{\Gamma}{\Delta}{[e/x]e_0}{\tau_0}$
\end{lemma}

\begin{lemma}[Simultaneous Substitution]
  \label{lem:simult-substitution}
  If $\hexptyp{\Gamma \uplus \Gamma'}{\Delta}{e}{\tau}$ and $\hsubstyp{\theta}{\Gamma'}$
  then $\hexptyp{\Gamma}{\Delta}{[\theta]e}{\tau}$
\end{lemma}

\begin{lemma}[Substitution Typing]
  \label{lem:subs-typing}
  If $\hpatmatch{e}{p}{\theta}$ and $\hexptyp{\cdot}{\Delta_e}{e}{\tau}$ and $\chpattyp{p}{\tau}{\xi}{\Gamma}{\Delta}$
  then $\hsubstyp{\theta}{\Gamma}$
\end{lemma}
Proof by induction on the derivation of $\hpatmatch{e}{p}{\theta}$.

\begin{theorem}[Determinism]
  \label{thrm:determinism}
  If $\hexptyp{\cdot}{\Delta}{e}{\tau}$ then exactly one of the following holds
  \begin{enumerate}
    \item $\isVal{e}$
    \item $\isIndet{e}$
    \item $\htrans{e}{e'}$ for some unique $e'$
  \end{enumerate}
\end{theorem}

\section{Dynamic Semantics}
\judgboxa{\isVal{e}}{$e$ is a value}
\begin{subequations}\label{rules:Value}
\begin{equation}
\inferrule[VNum]{ }{
  \isVal{\hnum{n}}
}
\end{equation}
\begin{equation}
\inferrule[VLam]{ }{
  \isVal{\hlam{x}{\tau}{e}}
}
\end{equation}
\begin{equation}
\inferrule[VPair]{
  \isVal{e_1} \\
  \isVal{e_2}
}{\isVal{\hpair{e_1}{e_2}}}
\end{equation}
\begin{equation}
\inferrule[VInl]{
  \isVal{e}
}{
  \isVal{\hinl{\tau}{e}}
}
\end{equation}
\begin{equation}
\inferrule[Vinr]{
  \isVal{e}
}{
  \isVal{\hinr{\tau}{e}}
}
\end{equation}
\end{subequations}

\judgboxa{\isIndet{e}}{$e$ is indeterminate}
\begin{subequations}\label{rules:Indet}
\begin{equation}\label{rule:IEHole}
\inferrule[IEHole]{ }{
  \isIndet{\hehole{u}}
}
\end{equation}
\begin{equation}\label{rule:IHole}
\inferrule[IHole]{
  \isFinal{e}
}{
  \isIndet{\hhole{e}{u}}
}
\end{equation}
\begin{equation}\label{rule:IAp}
\inferrule[IAp]{
  \isIndet{e_1} \\ \isFinal{e_2}
}{
  \isIndet{\hap{e_1}{e_2}}
}
\end{equation}
\begin{equation}\label{rule:IPairL}
\inferrule[IPairL]{
  \isIndet{e_1} \\ \isVal{e_2}
}{
  \isIndet{\hpair{e_1}{e_2}}
}
\end{equation}
\begin{equation}\label{rule:IPairR}
\inferrule[IPairR]{
  \isVal{e_1} \\
  \isIndet{e_2}
}{
  \isIndet{\hpair{e_1}{e_2}}
}
\end{equation}
\begin{equation}
\inferrule[IPair]{
  \isIndet{e_1} \\ \isIndet{e_2}
}{
  \isIndet{\hpair{e_1}{e_2}}
}
\end{equation}
\begin{equation}
\inferrule[IPrl]{
  \isIndet{e}
}{
  \isIndet{\hprl{e}}
}
\end{equation}
\begin{equation}
\inferrule[IPrr]{
  \isIndet{e}
}{
  \isIndet{\hprr{e}}
}
\end{equation}
\begin{equation}\label{rule:IInl}
\inferrule[IInL]{
  \isIndet{e}
}{
  \isIndet{\hinl{\tau}{e}}
}
\end{equation}
\begin{equation}\label{rule:IInR}
\inferrule[IInR]{
  \isIndet{e}
}{
  \isIndet{\hinr{\tau}{e}}
}
\end{equation}
\begin{equation}\label{rule:IMatch}
\inferrule[IMatch]{
  \isFinal{e} \\
  \hmaymatch{e}{p_r}
}{
  \isIndet{
    \hmatch{e}{\zruls{rs_{pre}}{\hrulP{p_r}{e_r}}{rs_{post}}}
  }
}
\end{equation}
\end{subequations}

\judgboxa{\isFinal{e}}{$e$ is final}
\begin{subequations}\label{rules:Final}
  \begin{equation}\label{rule:FVal}
\inferrule[FVal]{
  \isVal{e}
}{
  \isFinal{e}
}
\end{equation}
\begin{equation}\label{rule:FIndet}
\inferrule[FIndet]{
  \isIndet{e}
}{
  \isFinal{e}
}
\end{equation}
\end{subequations}

\judgboxa{
  \isntVal{e}
}{
  $e$ cannot be a value syntactically
}
\begin{subequations}
\begin{equation}
\inferrule[NVEHole]{ }{
  \isntVal{\hehole{u}}
}
\end{equation}
\begin{equation}
\inferrule[NVHole]{ }{
  \isntVal{\hhole{e}{u}}
}
\end{equation}
\begin{equation}
\inferrule[NVAp]{ }{
  \isntVal{\hap{e_1}{e_2}}
}
\end{equation}
\begin{equation}
\inferrule[NVMatch]{ }{
  \isntVal{\hmatch{e}{\zrules}}
}
\end{equation}
\begin{equation}
\inferrule[NVPrl]{ }{
  \isntVal{\hprl{e}}
}
\end{equation}
\begin{equation}
\inferrule[NVPrr]{ }{
  \isntVal{\hprr{e}}
}
\end{equation}
\end{subequations}

\judgboxa{
  \hsubstyp{\theta}{\Gamma}
}{
  $\theta$ is of type $\Gamma$
}
\begin{subequations}
\begin{equation}
\inferrule[STEmpty]{ }{
  \hsubstyp{\emptyset}{\cdot}
}
\end{equation}
\begin{equation}
\inferrule[STExtend]{
  \hsubstyp{\theta}{\Gamma_\theta} \\
  \hexptyp{\Gamma}{\Delta}{e}{\tau}
}{
  \hsubstyp{\theta , x / e}{\Gamma_\theta , x : \tau}
}
\end{equation}
\end{subequations}

\judgboxa{
  \refutable{p}
}{$p$ is refutable}
\begin{subequations}
\begin{equation}
\inferrule[RNum]{ }{
  \refutable{\hnum{n}}
}
\end{equation}
\begin{equation}
\inferrule[REHole]{ }{
  \refutable{\hehole{w}}
}
\end{equation}
\begin{equation}
\inferrule[RHole]{ }{
  \refutable{\hhole{p}{w}}
}
\end{equation}
\begin{equation}
\inferrule[RInl]{ }{
  \refutable{\hinlp{p}}
}
\end{equation}
\begin{equation}
\inferrule[RInr]{ }{
  \refutable{\hinrp{p}}
}
\end{equation}
\begin{equation}
\inferrule[RPairL]{
  \refutable{p_1}
}{
  \refutable{\hpair{p_1}{p_2}}
}
\end{equation}
\begin{equation}
\inferrule[RPairR]{
  \refutable{p_2}
}{
  \refutable{\hpair{p_1}{p_2}}
}
\end{equation}
\end{subequations}

\judgboxa{
  \hpatmatch{e}{p}{\theta}
}{
  $e$ matches $p$, emitting $\theta$
}
\begin{subequations}\label{rules:match}
\begin{equation}\label{rule:MVar}
\inferrule[MVar]{ }{
  \hpatmatch{e}{x}{e / x}
}
\end{equation}
\begin{equation}\label{rule:MWild}
\inferrule[MWild]{ }{
  \hpatmatch{e}{\_}{\cdot}
}
\end{equation}
\begin{equation}\label{rule:MNum}
\inferrule[MNum]{ }{
  \hpatmatch{\hnum{n}}{\hnum{n}}{\cdot}
}
\end{equation}
\begin{equation}\label{rule:MPair}
\inferrule[MPair]{
  \hpatmatch{e_1}{p_1}{\theta_1} \\
  \hpatmatch{e_2}{p_2}{\theta_2}
}{
  \hpatmatch{\hpair{e_1}{e_2}}{\hpair{p_1}{p_2}}{\theta_1 \uplus \theta_2}
}
\end{equation}
\begin{equation}\label{rule:MInl}
\inferrule[MInl]{
  \hpatmatch{e}{p}{\theta}
}{
  \hpatmatch{\hinl{\tau}{e}}{\hinlp{p}}{\theta}
}
\end{equation}
\begin{equation}\label{rule:MInr}
\inferrule[MInr]{
  \hpatmatch{e}{p}{\theta}
}{
  \hpatmatch{\hinr{\tau}{e}}{\hinrp{p}}{\theta}
}
\end{equation}
\begin{equation}
\inferrule[MEHolePair]{
  \hpatmatch{\hprl{\hehole{u}}}{p_1}{\theta_1} \\
  \hpatmatch{\hprr{\hehole{u}}}{p_2}{\theta_2}
}{
  \hpatmatch{\hehole{u}}{\hpair{p_1}{p_2}}{\theta_1 \uplus \theta_2}
}
\end{equation}
\begin{equation}
\inferrule[MHolePair]{
  \hpatmatch{\hprl{\hhole{e}{u}}}{p_1}{\theta_1} \\
  \hpatmatch{\hprr{\hhole{e}{u}}}{p_2}{\theta_2}
}{
  \hpatmatch{\hhole{e}{u}}{\hpair{p_1}{p_2}}{\theta_1 \uplus \theta_2}
}
\end{equation}
\begin{equation}
\inferrule[MApPair]{
  \hpatmatch{\hprl{\hap{e_1}{e_2}}}{p_1}{\theta_1} \\
  \hpatmatch{\hprr{\hap{e_1}{e_2}}}{p_2}{\theta_2}
}{
  \hpatmatch{\hap{e_1}{e_2}}{\hpair{p_1}{p_2}}{\theta_1 \uplus \theta_2}
}
\end{equation}
\begin{equation}
\inferrule[MMatchPair]{
  \hpatmatch{\hprl{\hmatch{e}{\zrules}}}{p_1}{\theta_1} \\
  \hpatmatch{\hprr{\hmatch{e}{\zrules}}}{p_2}{\theta_2}
}{
  \hpatmatch{\hmatch{e}{\zrules}}{\hpair{p_1}{p_2}}{\theta_1 \uplus \theta_2}
}
\end{equation}
\begin{equation}
\inferrule[MPrlPair]{
  \hpatmatch{\hprl{\hprl{e}}}{p_1}{\theta_1} \\
  \hpatmatch{\hprr{\hprl{e}}}{p_2}{\theta_2}
}{
  \hpatmatch{\hprl{e}}{\hpair{p_1}{p_2}}{\theta_1 \uplus \theta_2}
}
\end{equation}
\begin{equation}
\inferrule[MPrrPair]{
  \hpatmatch{\hprl{\hprr{e}}}{p_1}{\theta_1} \\
  \hpatmatch{\hprr{\hprr{e}}}{p_2}{\theta_2}
}{
  \hpatmatch{\hprr{e}}{\hpair{p_1}{p_2}}{\theta_1 \uplus \theta_2}
}
\end{equation}
\end{subequations}

\judgboxa{
  \hmaymatch{e}{p}
}{
  $e$ may match $p$
}
\begin{subequations}\label{rules:maymatch}
\begin{equation}\label{rule:MMEHole}
\inferrule[MMEHole]{ }{
  \hmaymatch{e}{\hehole{w}}
}
\end{equation}
\begin{equation}\label{rule:MMHole}
\inferrule[MMHole]{ }{
  \hmaymatch{e}{\hhole{p}{w}}
}
\end{equation}
\begin{equation}\label{rule:MMExpEHole}
\inferrule[MMExpEHole]{
  \refutable{p}
}{
  \hmaymatch{\hehole{u}}{p}
}
\end{equation}
\begin{equation}\label{rule:MMExpHole}
\inferrule[MMExpHole]{
  \refutable{p}
}{
  \hmaymatch{\hhole{e}{u}}{p}
}
\end{equation}
\begin{equation}\label{rule:MMAp}
\inferrule[MMAp]{
  \refutable{p}
}{
  \hmaymatch{\hap{e_1}{e_2}}{p}
}
\end{equation}
\begin{equation}
\inferrule[MMMatch]{
  \refutable{p}
}{
  \hmaymatch{\hmatch{e}{\zrules}}{p}
}
\end{equation}
\begin{equation}
\inferrule[MMPrl]{
  \refutable{p}
}{
  \hmaymatch{\hprl{e}}{p}
}
\end{equation}
\begin{equation}
\inferrule[MMPrr]{
  \refutable{p}
}{
  \hmaymatch{\hprr{e}}{p}
}
\end{equation}
\begin{equation}\label{rule:MMPairL}
\inferrule[MMPairL]{
  \hmaymatch{e_1}{p_1} \\
  \hpatmatch{e_2}{p_2}{\theta_2}
}{
  \hmaymatch{\hpair{e_1}{e_2}}{\hpair{p_1}{p_2}}
}
\end{equation}
\begin{equation}\label{rule:MMPairR}
\inferrule[MMPairR]{
  \hpatmatch{e_1}{p_1}{\theta_1} \\
  \hmaymatch{e_2}{p_2}
}{
  \hmaymatch{\hpair{e_1}{e_2}}{\hpair{p_1}{p_2}}
}
\end{equation}
\begin{equation}\label{rule:MMPair}
\inferrule[MMPair]{
  \hmaymatch{e_1}{p_1} \\
  \hmaymatch{e_2}{p_2}
}{
  \hmaymatch{\hpair{e_1}{e_2}}{\hpair{p_1}{p_2}}
}
\end{equation}
\begin{equation}\label{rule:MMInl}
\inferrule[MMInl]{
  \hmaymatch{e}{p}
}{
  \hmaymatch{\hinl{\tau}{e}}{\hinlp{p}}
}
\end{equation}
\begin{equation}\label{rule:MMInr}
\inferrule[MMInr]{
  \hmaymatch{e}{p}
}{
  \hmaymatch{\hinr{\tau}{e}}{\hinrp{p}}
}
\end{equation}
\end{subequations}

\judgboxa{
  \hnotmatch{e}{p}
}{
  $e$ does not match $p$
}
\begin{subequations}\label{rules:notmatch}
\begin{equation}
\inferrule[NMNum]{
  n_1 \neq n_2
}{
  \hnotmatch{\hnum{n_1}}{\hnum{n_2}}
}
\end{equation}
\begin{equation}
\inferrule[NMPairL]{
  \hnotmatch{e_1}{p_1}
}{
  \hnotmatch{\hpair{e_1}{e_2}}{\hpair{p_1}{p_2}}
}
\end{equation}
\begin{equation}
\inferrule[NMPairR]{
  \hnotmatch{e_2}{p_2}
}{
  \hnotmatch{\hpair{e_1}{e_2}}{\hpair{p_1}{p_2}}
}
\end{equation}
\begin{equation}
\inferrule[NMConfL]{ }{
  \hnotmatch{\hinr{\tau}{e}}{\hinlp{p}}
}
\end{equation}
\begin{equation}
\inferrule[NMConfR]{ }{
  \hnotmatch{\hinl{\tau}{e}}{\hinrp{p}}
}
\end{equation}
\begin{equation}
\inferrule[NMInl]{
  \hnotmatch{e}{p}
}{
  \hnotmatch{\hinr{\tau}{e}}{\hinlp{p}}
}
\end{equation}
\begin{equation}
\inferrule[NMInr]{
  \hnotmatch{e}{p}
}{
  \hnotmatch{\hinl{\tau}{e}}{\hinrp{p}}
}
\end{equation}
\end{subequations}

\judgboxa{\htrans{e}{e'}}{$e$ takes a step to $e'$}
\begin{subequations}\label{rules:ITExp}
\begin{equation}
\inferrule[ITHole]{
  \htrans{e}{e'}
}{
  \htrans{\hhole{e}{u}}{\hhole{e'}{u}}
}
\end{equation}
\begin{equation}
\inferrule[ITApFun]{
  \htrans{e_1}{e_1'}
}{
  \htrans{\hap{e_1}{e_2}}{\hap{e_1'}{e_2}}
}
\end{equation}
\begin{equation}
\inferrule[ITApArg]{
  \isVal{e_1} \\
  \htrans{e_2}{e_2'}
}{
  \htrans{\hap{e_1}{e_2}}{\hap{e_1}{e_2'}}
}
\end{equation}
\begin{equation}
\inferrule[ITAP]{
  \isVal{e_2}
}{
  \hap{\hlam{x}{\tau}{e_1}}{e_2} \mapsto
    [e_2/x]e_1
}
\end{equation}
\begin{equation}
\inferrule[ITPairL]{
  \htrans{e_1}{e_1'}
}{
  \htrans{\hpair{e_1}{e_2}}{\hpair{e_1'}{e_2}}
}
\end{equation}
\begin{equation}
\inferrule[ITPairR]{
  \isVal{e_1} \\
  \htrans{e_2}{e_2'}
}{
  \htrans{\hpair{e_1}{e_2}}{\hpair{e_1}{e_2'}}
}
\end{equation}
\begin{equation}
\inferrule[ITPrl]{
  \isFinal{\hpair{e_1}{e_2}}
}{
  \htrans{\hprl{\hpair{e_1}{e_2}}}{e_1}
}
\end{equation}
\begin{equation}
\inferrule[ITPrr]{
  \isFinal{\hpair{e_1}{e_2}}
}{
  \htrans{\hprr{\hpair{e_1}{e_2}}}{e_2}
}
\end{equation}
\begin{equation}
\inferrule[ITInl]{
  \htrans{e}{e'}
}{
  \htrans{\hinl{\tau}{e}}{\hinl{\tau}{e'}}
}
\end{equation}
\begin{equation}
\inferrule[ITInr]{
  \htrans{e}{e'}
}{
  \htrans{\hinr{\tau}{e}}{\hinr{\tau}{e'}}
}
\end{equation}
\begin{equation}\label{rule:ITExpMatch}
\inferrule[ITExpMatch]{
  \htrans{e}{e'}
}{
  \htrans{\hmatch{e}{\zrules}}{\hmatch{e'}{\zrules}}
}
\end{equation}
\begin{equation}\label{rule:ITSuccMatch}
\inferrule[ITSuccMatch]{
  \isFinal{e} \\
  \hpatmatch{e}{p_r}{\theta}
}{
  \htrans{
    \hmatch{e}{\zruls{rs_{pre}}{\hrulP{p_r}{e_r}}{rs_{post}}}
  }{
    [\theta](e_r)
  }
}
\end{equation}
\begin{equation}\label{rule:ITFailMatch}
\inferrule[ITFailMatch]{
  \isFinal{e} \\
  \hnotmatch{e}{p_r}
}{
  \htrans{
    \hmatch{e}{\zruls{rs}{\hrulP{p_r}{e_r}}{\hrulesP{r'}{rs'}}}
  }{
    \hmatch{e}{
      \zruls{
        \rmpointer{\zruls{rs}{\hrulP{p_r}{e_r}}{\cdot}}
      }{r'}{rs'}
    }
  }
}
\end{equation}
\end{subequations}

\begin{lemma}[Matching Coherence of Constraint]
  \label{lem:const-matching-coherence}
  Suppose that $\hexptyp{\cdot}{\Delta_e}{e}{\tau}$ and $\isFinal{e}$ and $\chpattyp{p}{\tau}{\xi}{\Gamma}{\Delta}$. Then we have
  \begin{enumerate}
  \item $\csatisfy{e}{\xi}$ iff $\hpatmatch{e}{p}{\theta}$
  \item $\csatisfy{e}{\cdual{\xi}}$ iff $\hnotmatch{e}{p}$
  \item $\cmaysatisfy{e}{\xi}$ iff $\hmaymatch{e}{p}$
  \end{enumerate}
\end{lemma}
\begin{proof}
\begin{pfsteps*}
\item $\hexptyp{\cdot}{\Delta_e}{e}{\tau}$ \BY{assumption} \pflabel{eTyp}
\item $\isFinal{e}$ \BY{assumption} \pflabel{eFinal}
\item $\chpattyp{p}{\tau}{\xi}{\Gamma}{\Delta}$ \BY{assumption} \pflabel{patTyp}
\end{pfsteps*}
By rule induction over Rules (\ref{rules:PatTyp}) on \pfref{patTyp}.
\begin{byCases}
\savelocalsteps{0}
\item[\text{(\ref{rule:PTVar})}]
    \begin{pfsteps*}
    \item $p=x$ \BY{assumption}
    \item $\xi=\ctruth$ \BY{assumption}
    \end{pfsteps*}
    \begin{enumerate}
    \savelocalsteps{1}
    \item Prove $\csatisfy{e}{\ctruth}$ implies $\hpatmatch{e}{x}{\theta}$ for some $\theta$.
        \begin{pfsteps*}
        \item $\hpatmatch{e}{x}{e / x}$ \BY{Rule (\ref{rule:MVar})}
        \end{pfsteps*}
    \restorelocalsteps{1}
    \item Prove $\hpatmatch{e}{x}{\theta}$ implies $\csatisfy{e}{\ctruth}$.
        \begin{pfsteps*}
        \item $\csatisfy{e}{\ctruth}$ \BY{Rule (\ref{rule:CSTruth})}
        \end{pfsteps*}
    \restorelocalsteps{1}
    \item Prove $\csatisfy{e}{\cdual{\ctruth}}$ implies $\hnotmatch{e}{x}$.
        \begin{pfsteps*}
        \item $\cdual{\ctruth}=\cfalsity$ \BY{Definition \ref{defn:dual}}
        \item $\cnotsatisfy{e}{\cfalsity}$ \BY{\autoref{lem:no-e-satisfy-falsity}}
        \end{pfsteps*}
        Vacuously true.
    \restorelocalsteps{1}
    \item Prove $\hnotmatch{e}{x}$ implies $\csatisfy{e}{\cdual{\ctruth}}$.

        By rule induction over Rules (\ref{rules:notmatch}), we notice that $\hnotmatch{e}{x}$ is in syntactic contradiction with all the cases, hence not derivable. And thus vacuously true.
    \restorelocalsteps{1}
    \item Prove $\cmaysatisfy{e}{\ctruth}$ implies $\hmaymatch{e}{x}$.
        \begin{pfsteps*}
        \item $\cnotmaysatisfy{e}{\ctruth}$ \BY{\autoref{lem:no-e-may-satisfy-truth}}
        \end{pfsteps*}
        Vacuously true.
    \restorelocalsteps{1}
    \item Prove $\hmaymatch{e}{x}$ implies $\cmaysatisfy{e}{\xi}$.
    
        By rule induction over Rules (\ref{rules:maymatch}), we notice that either, $\hmaymatch{e}{x}$ is in syntactic contradiction with all the cases, or the premise $\refutable{x}$ is not derivable. Hence, $\hmaymatch{e}{x}$ are not derivable. And thus vacuously true.
    \end{enumerate}
    
\restorelocalsteps{0}
\item[\text{(\ref{rule:PTWild})}]
    \begin{pfsteps*}
    \item $p=\_$ \BY{assumption}
    \item $\xi=\ctruth$ \BY{assumption}
    \end{pfsteps*}
    \begin{enumerate}
    \savelocalsteps{1}
    \item Prove $\csatisfy{e}{\ctruth}$ implies $\hpatmatch{e}{\_}{\theta}$ for some $\theta$.
        \begin{pfsteps*}
        \item $\hpatmatch{e}{\_}{\cdot}$ \BY{Rule (\ref{rule:MVar})}
        \end{pfsteps*}
    \restorelocalsteps{1}
    \item Prove $\hpatmatch{e}{\_}{\theta}$ implies $\csatisfy{e}{\ctruth}$.
        \begin{pfsteps*}
        \item $\csatisfy{e}{\ctruth}$ \BY{Rule (\ref{rule:CSTruth})}
        \end{pfsteps*}
    \restorelocalsteps{1}
    \item Prove $\csatisfy{e}{\cdual{\ctruth}}$ implies $\hnotmatch{e}{\_}$.
        \begin{pfsteps*}
        \item $\cdual{\ctruth}=\cfalsity$ \BY{Definition \ref{defn:dual}}
        \item $\cnotsatisfy{e}{\cfalsity}$ \BY{\autoref{lem:no-e-satisfy-falsity}}
        \end{pfsteps*}
        Vacuously true.
    \restorelocalsteps{1}
    \item Prove $\hnotmatch{e}{\_}$ implies $\csatisfy{e}{\cdual{\ctruth}}$.

        By rule induction over Rules (\ref{rules:notmatch}), we notice that $\hnotmatch{e}{\_}$ is in syntactic contradiction with all the cases, hence not derivable. And thus vacuously true.
    \restorelocalsteps{1}
    \item Prove $\cmaysatisfy{e}{\ctruth}$ implies $\hmaymatch{e}{\_}$.
        \begin{pfsteps*}
        \item $\cnotmaysatisfy{e}{\ctruth}$ \BY{\autoref{lem:no-e-may-satisfy-truth}}
        \end{pfsteps*}
        Vacuously true.
    \restorelocalsteps{1}
    \item Prove $\hmaymatch{e}{\_}$ implies $\cmaysatisfy{e}{\xi}$.
    
        By rule induction over Rules (\ref{rules:maymatch}), we notice that either, $\hmaymatch{e}{\_}$ is in syntactic contradiction with all the cases, or the premise $\refutable{\_}$ is not derivable. Hence, $\hmaymatch{e}{\_}$ are not derivable. And thus vacuously true.
    \end{enumerate}
    
\restorelocalsteps{0}
\item[\text{(\ref{rule:PTEHole})}]
    \begin{pfsteps*}
    \item $p=\hehole{w}$ \BY{assumption}
    \item $\xi=\cunknown$ \BY{assumption}
    \item $\cdual{\xi}=\cunknown$ \BY{Definition \ref{defn:dual}}
    \end{pfsteps*}
    \begin{enumerate}
    \savelocalsteps{1}
    \item Prove $\csatisfy{e}{\cunknown}$ implies $\hpatmatch{e}{\hehole{w}}{\theta}$ for some $\theta$.
        \begin{pfsteps*}
        \item $\cnotsatisfy{e}{\cunknown}$ \BY{Rule (\ref{rule:MVar})}
        \end{pfsteps*}
        Vacuously true.
    \restorelocalsteps{1}
    \item Prove $\hpatmatch{e}{\hehole{w}}{\theta}$ implies $\csatisfy{e}{\cunknown}$.\\
        By rule induction over Rules (\ref{rules:match}), we notice that $\hpatmatch{e}{\hehole{w}}{\theta}$ is in syntactic contradiction with all the cases, hence not derivable. And thus vacuously true.
    \restorelocalsteps{1}
    \item Prove $\csatisfy{e}{\cunknown}$ implies $\hnotmatch{e}{\hehole{w}}$.
        \begin{pfsteps*}
        \item $\cnotsatisfy{e}{\cunknown}$ \BY{Rule (\ref{rule:MVar})}
        \end{pfsteps*}
        Vacuously true.
    \restorelocalsteps{1}
    \item Prove $\hnotmatch{e}{\hehole{w}}$ implies $\csatisfy{e}{\cunknown}$.\\
        By rule induction over Rules (\ref{rules:notmatch}), we notice that $\hnotmatch{e}{\hehole{w}}$ is in syntactic contradiction with all the cases, hence not derivable. And thus vacuously true.
    \restorelocalsteps{1}
    \item Prove $\cmaysatisfy{e}{\cunknown}$ implies $\hmaymatch{e}{\hehole{w}}$.
        \begin{pfsteps*}
        \item $\hmaymatch{e}{\hehole{w}}$ \BY{Rule (\ref{rule:MMEHole})}
        \end{pfsteps*}
    \restorelocalsteps{1}
    \item Prove $\hmaymatch{e}{\hehole{w}}$ implies $\cmaysatisfy{e}{\cunknown}$.
        \begin{pfsteps*}
        \item $\cmaysatisfy{e}{\cunknown}$ \BY{Rule (\ref{rule:CMSUnknown})}
        \end{pfsteps*}
    \end{enumerate}

\restorelocalsteps{0}
\item[\text{(\ref{rule:PTHole})}]
    \begin{pfsteps*}
    \item $p=\hhole{p_0}{w}$ \BY{assumption}
    \item $\xi=\cunknown$ \BY{assumption}
    \item $\cdual{\xi}=\cunknown$ \BY{Definition \ref{defn:dual}}
    \end{pfsteps*}
    \begin{enumerate}
    \savelocalsteps{1}
    \item Prove $\csatisfy{e}{\cunknown}$ implies $\hpatmatch{e}{\hhole{p_0}{w}}{\theta}$ for some $\theta$.
        \begin{pfsteps*}
        \item $\cnotsatisfy{e}{\cunknown}$ \BY{Rule (\ref{rule:MVar})}
        \end{pfsteps*}
        Vacuously true.
    \restorelocalsteps{1}
    \item Prove $\hpatmatch{e}{\hhole{p_0}{w}}{\theta}$ implies $\csatisfy{e}{\cunknown}$.\\
        By rule induction over Rules (\ref{rules:match}), we notice that $\hpatmatch{e}{\hhole{p_0}{w}}{\theta}$ is in syntactic contradiction with all the cases, hence not derivable. And thus vacuously true.
    \restorelocalsteps{1}
    \item Prove $\csatisfy{e}{\cunknown}$ implies $\hnotmatch{e}{\hhole{p_0}{w}}$.
        \begin{pfsteps*}
        \item $\cnotsatisfy{e}{\cunknown}$ \BY{Rule (\ref{rule:MVar})}
        \end{pfsteps*}
        Vacuously true.
    \restorelocalsteps{1}
    \item Prove $\hnotmatch{e}{\hhole{p_0}{w}}$ implies $\csatisfy{e}{\cunknown}$.\\
        By rule induction over Rules (\ref{rules:notmatch}), we notice that $\hnotmatch{e}{\hhole{p_0}{w}}$ is in syntactic contradiction with all the cases, hence not derivable. And thus vacuously true.
    \restorelocalsteps{1}
    \item Prove $\cmaysatisfy{e}{\cunknown}$ implies $\hmaymatch{e}{\hhole{p_0}{w}}$.
        \begin{pfsteps*}
        \item $\hmaymatch{e}{\hhole{p_0}{w}}$ \BY{Rule (\ref{rule:MMHole})}
        \end{pfsteps*}
    \restorelocalsteps{1}
    \item Prove $\hmaymatch{e}{\hhole{p_0}{w}}$ implies $\cmaysatisfy{e}{\cunknown}$.
        \begin{pfsteps*}
        \item $\cmaysatisfy{e}{\cunknown}$ \BY{Rule (\ref{rule:CMSUnknown})}
        \end{pfsteps*}
    \end{enumerate}
    
\restorelocalsteps{0}
\item[\text{(\ref{rule:PTNum})}]
    \begin{pfsteps*}
    \item $p=\hhole{p_0}{w}$ \BY{assumption}
    \item $\xi=\cunknown$ \BY{assumption}
    \item $\cdual{\xi}=\cunknown$ \BY{Definition \ref{defn:dual}}
    \end{pfsteps*}
    \begin{enumerate}
    \savelocalsteps{1}
    \item Prove $\csatisfy{e}{\cunknown}$ implies $\hpatmatch{e}{\hhole{p_0}{w}}{\theta}$ for some $\theta$.
        \begin{pfsteps*}
        \item $\cnotsatisfy{e}{\cunknown}$ \BY{Rule (\ref{rule:MVar})}
        \end{pfsteps*}
        Vacuously true.
    \restorelocalsteps{1}
    \item Prove $\hpatmatch{e}{\hhole{p_0}{w}}{\theta}$ implies $\csatisfy{e}{\cunknown}$.\\
        By rule induction over Rules (\ref{rules:match}), we notice that $\hpatmatch{e}{\hhole{p_0}{w}}{\theta}$ is in syntactic contradiction with all the cases, hence not derivable. And thus vacuously true.
    \restorelocalsteps{1}
    \item Prove $\csatisfy{e}{\cunknown}$ implies $\hnotmatch{e}{\hhole{p_0}{w}}$.
        \begin{pfsteps*}
        \item $\cnotsatisfy{e}{\cunknown}$ \BY{Rule (\ref{rule:MVar})}
        \end{pfsteps*}
        Vacuously true.
    \restorelocalsteps{1}
    \item Prove $\hnotmatch{e}{\hhole{p_0}{w}}$ implies $\csatisfy{e}{\cunknown}$.\\
        By rule induction over Rules (\ref{rules:notmatch}), we notice that $\hnotmatch{e}{\hhole{p_0}{w}}$ is in syntactic contradiction with all the cases, hence not derivable. And thus vacuously true.
    \restorelocalsteps{1}
    \item Prove $\cmaysatisfy{e}{\cunknown}$ implies $\hmaymatch{e}{\hhole{p_0}{w}}$.
        \begin{pfsteps*}
        \item $\hmaymatch{e}{\hhole{p_0}{w}}$ \BY{Rule (\ref{rule:MMHole})}
        \end{pfsteps*}
    \restorelocalsteps{1}
    \item Prove $\hmaymatch{e}{\hhole{p_0}{w}}$ implies $\cmaysatisfy{e}{\cunknown}$.
        \begin{pfsteps*}
        \item $\cmaysatisfy{e}{\cunknown}$ \BY{Rule (\ref{rule:CMSUnknown})}
        \end{pfsteps*}
    \end{enumerate}
\end{byCases}
\end{proof}
\section{Preservation and Progress}

\begin{theorem}[Preservation]
  \label{thrm:preservation}
  If $\hexptyp{\cdot}{\Delta}{e}{\tau}$ and $\htrans{e}{e'}$
  then $\hexptyp{\cdot}{\Delta}{e'}{\tau}$
\end{theorem}
\begin{proof}
By rule induction over Rules (\ref{rules:TExp}) on typing judgment of $e$.
For simplicity, we only consider two cases for match expressions here.
\begin{byCases}
\item[\text{(\ref{rule:TMatchZPre})}]
  \begin{pfsteps*}
  \item $\hexptyp{\cdot}{\Delta}{\hmatch{e_1}{\zruls{\cdot}{r}{rs}}}{\tau}$ \BY{assumption} \pflabel{expType}
  \item $\htrans{\hmatch{e_1}{\zruls{\cdot}{r}{rs}}}{e'}$ \BY{assumption} \pflabel{expTrans}
  \item $\hexptyp{\cdot}{\Delta}{e_1}{\tau_1}$ \BY{assumption} \pflabel{scrutType}
  \item $\chrulstyp{\cdot}{\Delta}{\cfalsity}{\hrulesP{r}{rs}}{\tau_1}{\xi}{\tau}$ \BY{assumption} \pflabel{rulesType}
  \item $\csatisfyormay{\ctruth}{\xi}$ \BY{assumption} \pflabel{exhaust}
  \end{pfsteps*}
  By rule induction over Rules (\ref{rules:ITExp}) on \pfref{expTrans}.
  \begin{byCases}
    
  \savelocalsteps{thrm:preservation-1}
  \item[\text{(\ref{rule:ITExpMatch})}]
    \begin{pfsteps*}
    \item $e' = \hmatch{e_1'}{\zruls{\cdot}{r}{rs}}$ \BY{assumption}
    \item $\htrans{e_1}{e_1'}$ \BY{assumption} \pflabel{scrutTrans}
    \item $\hexptyp{\cdot}{\Delta}{e_1'}{\tau_1}$ \BY{IH on \pfref{scrutType} and \pfref{scrutTrans}} \pflabel{newscrutType}
    \item $\hexptyp{\cdot}{\Delta}{\hmatch{e_1'}{\zruls{\cdot}{r}{rs}}}{\tau}$ \BY{Rule (\ref{rule:TMatchZPre}) on \pfref{newscrutType} and \pfref{rulesType} and \pfref{exhaust}}
    \end{pfsteps*}

  \restorelocalsteps{thrm:preservation-1}
  \item[\text{(\ref{rule:ITSuccMatch})}]
    \begin{pfsteps*}
    \item $r = \hrul{p_r}{e_r}$ \BY{assumption}
    \item $e' = [\theta](e_r)$ \BY{assumption}
    \item $\hpatmatch{e_1}{p_r}{\theta}$ \BY{assumption} \pflabel{patMatch}
    \end{pfsteps*}
    By rule induction over Rules (\ref{rules:CTRules}) on \pfref{rulesType}.
    \begin{byCases}

    \savelocalsteps{thrm:preservation-2}
    \item[\text{(\ref{rule:CTOneRules})}]
      \begin{pfsteps*}
      \item $\xi = \xi_r$ \BY{assumption}
      \item $\chrultyp{\cdot}{\Delta}{\hrulP{p_r}{e_r}}{\tau_1}{\xi_r}{\tau}$ \BY{assumption} \pflabel{[one]ruleType}
      \item $\chpattyp{p_r}{\tau_1}{\xi_r}{\Gamma_r}{\Delta_r}$ \BY{Inversion of Rule (\ref{rule:CTRule}) on \pfref{[one]ruleType}} \pflabel{[one]patType}
      \item $\hexptyp{\Gamma_r}{\Delta \uplus \Delta_r}{e_r}{\tau}$ \BY{Inversion of Rule (\ref{rule:CTRule}) on \pfref{[one]ruleType}} \pflabel{[one]expinruleType}
      \item $\hsubstyp{\theta}{\Gamma_r}$ \BY{Lemma \ref{lem:subs-typing} on \pfref{scrutType} and \pfref{[one]patType} and \pfref{patMatch}} \pflabel{[one]substType}
      \item $\hexptyp{\cdot}{\Delta \uplus \Delta_r}{[\theta](e_r)}{\tau}$ \BY{Lemma \ref{lem:simult-substitution} on \pfref{[one]expinruleType} and \pfref{[one]substType}}
      \end{pfsteps*}

    \restorelocalsteps{thrm:preservation-2}
    \item[\text{(\ref{rule:CTRules})}]
      \begin{pfsteps*}
      \item $\xi = \cor{\xi_r}{\xi_{rs}}$ \BY{assumption}
      \item $\chrultyp{\cdot}{\Delta}{\hrulP{p_r}{e_r}}{\tau_1}{\xi_r}{\tau}$ \BY{assumption} \pflabel{ruleType}
      \item $\chpattyp{p_r}{\tau_1}{\xi_r}{\Gamma_r}{\Delta_r}$ \BY{Inversion of Rule (\ref{rule:CTRule}) on \pfref{ruleType}} \pflabel{patType}
      \item $\hexptyp{\Gamma_r}{\Delta \uplus \Delta_r}{e_r}{\tau}$ \BY{Inversion of Rule (\ref{rule:CTRule}) on \pfref{ruleType}} \pflabel{expinruleType}
      \item $\hsubstyp{\theta}{\Gamma_r}$ \BY{Lemma \ref{lem:subs-typing} on \pfref{scrutType} and \pfref{patType} and \pfref{patMatch}} \pflabel{substType}
      \item $\hexptyp{\cdot}{\Delta \uplus \Delta_r}{[\theta](e_r)}{\tau}$ \BY{Lemma \ref{lem:simult-substitution} on \pfref{expinruleType} and \pfref{substType}}
      \end{pfsteps*}
    \end{byCases}

  \restorelocalsteps{thrm:preservation-1}
  \item[\text{(\ref{rule:ITFailMatch})}]
    \begin{pfsteps*}
    \item $rs = \hrules{r'}{rs'}$ \BY{assumption}
    \item $e' = \hmatch{e_1}{\zruls{\hrulesP{\hrul{p_r}{e_r}}{\cdot}}{r'}{rs'}}$ \BY{assumption}
    \item $\isFinal{e_1}$ \BY{assumption} \pflabel{scrutFinal}
    \item $\hnotmatch{e_1}{p_r}$ \BY{assumption} \pflabel{patnotmatch}
    \end{pfsteps*}

    By rule induction over Rules (\ref{rules:CTRules}) on \pfref{rulesType}.
    \begin{byCases}
    \item[\text{(\ref{rule:CTOneRules})}]
      Syntactic contradiction of $rs$.
    \item[\text{(\ref{rule:CTRules})}]
      \begin{pfsteps*}
      \item $\xi = \cor{\xi_r}{\xi_{rs}}$ \BY{assumption}
      \item $\chrultyp{\cdot}{\Delta}{\hrulP{p_r}{e_r}}{\tau_1}{\xi_r}{\tau}$ \BY{assumption} \pflabel{[fail]ruleType}
      \item $\chrulstyp{\cdot}{\Delta}{\cor{\cfalsity}{\xi_r}}{\hrulesP{r'}{rs'}}{\tau_1}{\xi_{rs}}{\tau}$ \BY{assumption} \pflabel{[fail]rulesType}
      \item $\cnotsatisfy{\xi_r}{\cfalsity}$ \BY{assumption} \pflabel{r|/=bot}
      \item $\chpattyp{p_r}{\tau_1}{\xi_r}{\Gamma_r}{\Delta_r}$ \BY{Inversion of Rule (\ref{rule:CTRule}) on \pfref{[fail]ruleType}} \pflabel{[fail]patType}
      \item $\hexptyp{\Gamma_r}{\Delta \uplus \Delta_r}{e_r}{\tau}$ \BY{Inversion of Rule (\ref{rule:CTRule}) on \pfref{[fail]ruleType}} \pflabel{[fail]expinruleType}
      \item $\chrulstyp{\cdot}{\Delta}{\cfalsity}{\hrulesP{\hrul{p_r}{e_r}}{\cdot}}{\tau_1}{\xi_r}{\tau}$ \BY{Rule (\ref{rule:CTOneRules}) on \pfref{[fail]ruleType} and \pfref{r|/=bot}} \pflabel{r+emptyType}
      \item $\cnotsatisfyormay{e_1}{\xi_r}$ \BY{Lemma \ref{lem:const-matching-coherence} on \pfref{scrutType} and \pfref{scrutFinal} and \pfref{[fail]patType} and \pfref{patnotmatch}} \pflabel{notrConst}
      \item $\hexptyp{\cdot}{\Delta}{\hmatch{e_1}{\zruls{\hrulesP{\hrul{p_r}{e_r}}{\cdot}}{r'}{rs'}}}{\tau}$ \BY{Rule (\ref{rule:TMatchNZPre}) on \pfref{scrutType} and \pfref{scrutFinal} and \pfref{r+emptyType} and \pfref{[fail]rulesType} and \pfref{notrConst} and \pfref{exhaust}}
      \end{pfsteps*}
    \end{byCases}
  \end{byCases}

  \resetpfcounter
\item[\text{(\ref{rule:TMatchNZPre})}]
  \begin{pfsteps*}
  \item $rs_{pre} = \hrules{r_{pre}}{rs_{pre}'}$ \BY{assumption}
  \item $\hexptyp{\cdot}{\Delta}{\hmatch{e_1}{\zruls{rs_{pre}}{r}{rs_{post}}}}{\tau}$ \BY{assumption} \pflabel{expType}
  \item $\htrans{\hmatch{e_1}{\zruls{rs_{pre}}{r}{rs_{post}}}}{e'}$ \BY{assumption} \pflabel{expTrans}
  \item $\hexptyp{\cdot}{\Delta}{e_1}{\tau_1}$ \BY{assumption} \pflabel{scrutType}
  \item $\isFinal{e_1}$ \BY{assumption} \pflabel{scrutFinal}
  \item $\chrulstyp{\cdot}{\Delta}{\cfalsity}{rs_{pre}}{\tau_1}{\xi_{pre}}{\tau}$ \BY{assumption} \pflabel{prerulesType}
  \item $\chrulstyp{\cdot}{\Delta}{\cor{\cfalsity}{\xi_{pre}}}{\hrulesP{r}{rs_{post}}}{\tau_1}{\xi_{rest}}{\tau}$ \BY{assumption} \pflabel{restrulesType}
  \item $\cnotsatisfyormay{e_1}{\xi_{pre}}$ \BY{assumption} \pflabel{notsatisfypre}
  \item $\csatisfyormay{\ctruth}{\cor{\xi_{pre}}{\xi_{rest}}}$ \BY{assumption} \pflabel{exhaust}
  \end{pfsteps*}
  By rule induction over Rules (\ref{rules:ITExp}) on \pfref{expTrans}.
  \begin{byCases}

  \savelocalsteps{thrm:preservation-1}
  \item[\text{(\ref{rule:ITExpMatch})}]
    \begin{pfsteps*}
    \item $e' = \hmatch{e_1'}{\zruls{rs_{pre}}{r}{rs_{post}}}$ \BY{assumption}
    \item $\htrans{e_1}{e_1'}$ \BY{assumption} \pflabel{scrutTrans}
    \end{pfsteps*}
    By Lemma \ref{lem:finality},  \pfref{scrutTrans} contradicts \pfref{scrutFinal}.
  
  \restorelocalsteps{thrm:preservation-1}
  \item[\text{(\ref{rule:ITSuccMatch})}]
    \begin{pfsteps*}
    \item $r = \hrul{p_r}{e_r}$ \BY{assumption}
    \item $e' = [\theta](e_r)$ \BY{assumption}
    \item $\hpatmatch{e_1}{p_r}{\theta}$ \BY{assumption} \pflabel{patMatch}
    \end{pfsteps*}
    By rule induction over Rules (\ref{rules:CTRules}) on \pfref{restrulesType}.
    \begin{byCases}

    \savelocalsteps{thrm:preservation-2}
    \item[\text{(\ref{rule:CTOneRules})}]
      \begin{pfsteps*}
      \item $\xi_{rest} = \xi_r$ \BY{assumption}
      \item $\chrultyp{\cdot}{\Delta}{\hrulP{p_r}{e_r}}{\tau_1}{\xi_r}{\tau}$ \BY{assumption} \pflabel{[one]ruleType}
      \item $\chpattyp{p_r}{\tau_1}{\xi_r}{\Gamma_r}{\Delta_r}$ \BY{Inversion of Rule (\ref{rule:CTRule}) on \pfref{[one]ruleType}} \pflabel{[one]patType}
      \item $\hexptyp{\Gamma_r}{\Delta \uplus \Delta_r}{e_r}{\tau}$ \BY{Inversion of Rule (\ref{rule:CTRule}) on \pfref{[one]ruleType}} \pflabel{[one]expinruleType}
      \item $\hsubstyp{\theta}{\Gamma_r}$ \BY{Lemma \ref{lem:subs-typing} on \pfref{scrutType} and \pfref{[one]patType} and \pfref{patMatch}} \pflabel{[one]substType}
      \item $\hexptyp{\cdot}{\Delta \uplus \Delta_r}{[\theta](e_r)}{\tau}$ \BY{Lemma \ref{lem:simult-substitution} on \pfref{[one]expinruleType} and \pfref{[one]substType}}
      \end{pfsteps*}

    \restorelocalsteps{thrm:preservation-2}
    \item[\text{(\ref{rule:CTRules})}]
      \begin{pfsteps*}
      \item $\xi_{rest} = \cor{\xi_r}{\xi_{rs}}$ \BY{assumption}
      \item $\chrultyp{\cdot}{\Delta}{\hrulP{p_r}{e_r}}{\tau_1}{\xi_r}{\tau}$ \BY{assumption} \pflabel{ruleType}
      \item $\chpattyp{p_r}{\tau_1}{\xi_r}{\Gamma_r}{\Delta_r}$ \BY{assumption} \pflabel{patType}
      \item $\hexptyp{\Gamma_r}{\Delta \uplus \Delta_r}{e_r}{\tau}$ \BY{assumption} \pflabel{expinruleType}
      \item $\hsubstyp{\theta}{\Gamma_r}$ \BY{Lemma \ref{lem:subs-typing} on \pfref{scrutType} and \pfref{patType} and \pfref{patMatch}} \pflabel{substType}
      \item $\hexptyp{\cdot}{\Delta \uplus \Delta_r}{[\theta](e_r)}{\tau}$ \BY{Lemma \ref{lem:simult-substitution} on \pfref{expinruleType} and \pfref{substType}}
      \end{pfsteps*}
    \end{byCases}
    
  
  \restorelocalsteps{thrm:preservation-1}
  \item[\text{(\ref{rule:ITFailMatch})}]
    \begin{pfsteps*}
    \item $r = \hrul{p_r}{e_r}$ \BY{assumption}
    \item $rs_{post} = \hrules{r'}{rs'}$ \BY{assumption}
    \item $e' = \hmatch{e_1}{\zruls{\rmpointer{\zruls{rs_{pre}}{\hrul{p_r}{e_r}}{\cdot}}}{r'}{rs'}}$ \BY{assumption}
    \item $\hnotmatch{e_1}{p_r}$ \BY{assumption} \pflabel{patnotmatch}
    \end{pfsteps*}
    By rule induction over Rules (\ref{rules:CTRules}) on \pfref{restrulesType}.
    \begin{byCases}
    \item[\text{(\ref{rule:CTOneRules})}]
      Syntactic contradiction of $rs_{post}$.
    \item[\text{(\ref{rule:CTRules})}]
      \begin{pfsteps*}
      \item $\xi_{rest} = \cor{\xi_r}{\xi_{post}}$ \BY{assumption}
      \item $\chrultyp{\cdot}{\Delta}{\hrulP{p_r}{e_r}}{\tau_1}{\xi_r}{\tau}$ \BY{assumption} \pflabel{[fail]ruleType}
      \item $\chrulstyp{\cdot}{\Delta}{\cor{\cfalsity}{\cor{\xi_{pre}}}{\xi_r}}{\hrulesP{r'}{rs'}}{\tau_1}{\xi_{post}}{\tau}$ \BY{assumption} \pflabel{[fail]rulesType}
      \item $\cnotsatisfy{\xi_r}{\xi_{pre}}$ \BY{assumption} \pflabel{r|/=pre}
      \item $\chpattyp{p_r}{\tau_1}{\xi_r}{\Gamma_r}{\Delta_r}$ \BY{Inversion of Rule (\ref{rule:CTRule}) on \pfref{[fail]ruleType}} \pflabel{[fail]patType}
      \item $\hexptyp{\Gamma_r}{\Delta \uplus \Delta_r}{e_r}{\tau}$ \BY{Inversion of Rule (\ref{rule:CTRule}) on \pfref{[fail]ruleType}} \pflabel{[fail]expinruleType}
      \item $\ctyp{\xi_r}{\tau_1}$ \BY{Lemma \ref{lem:rule-constraint-typ} on \pfref{[fail]ruleType}} \pflabel{xirType}
      \item $\ctyp{\xi_{pre}}{\tau_1}$ \BY{Lemma \ref{lem:rules-constraint-typ} on \pfref{prerulesType}} \pflabel{xipreType}
      \item $\cnotsatisfy{\xi_r}{\cor{\cfalsity}{\xi_{pre}}}$ \BY{Lemma \ref{lem:relax-not-redundant} on \pfref{xirType} and \pfref{xipreType} and \pfref{r|/=pre}} \pflabel{r|/=bot+pre}
      \item $\chrulstyp{\cdot}{\Delta}{\cfalsity}{\rmpointer{\zruls{rs_{pre}}{\hrul{p_r}{e_r}}{\cdot}}}{\tau_1}{\cor{\xi_{pre}}{\xi_r}}{\tau}$ \BY{Lemma \ref{lem:rule-append} on \pfref{prerulesType} and \pfref{[fail]ruleType} and \pfref{r|/=bot+pre}} \pflabel{pre+rrulesType}
      \item $\cnotsatisfyormay{e_1}{\xi_r}$ \BY{Lemma \ref{lem:const-matching-coherence} on \pfref{scrutType} and \pfref{scrutFinal} and \pfref{[fail]patType} and \pfref{patnotmatch}} \pflabel{notsatormayr}
    \item $\cnotsatisfyormay{e_1}{\cor{\xi_{pre}}{\xi_r}}$ \BY{Lemma \ref{lem:or-nn-satisfy} on \pfref{notsatisfypre} and \pfref{notsatormayr}} \pflabel{notsatormaypre+r}
      \item $\hexptyp{\cdot}{\Delta}{\hmatch{e_1}{\zruls{\rmpointer{\zruls{rs_{pre}}{\hrul{p_r}{e_r}}{\cdot}}}{r'}{rs'}}}{\tau}$ \BY{Rule (\ref{rule:TMatchNZPre}) on \pfref{scrutType} and \pfref{scrutFinal} and \pfref{pre+rrulesType} and \pfref{[fail]rulesType} and \pfref{notsatormaypre+r} and \pfref{exhaust}}
      \end{pfsteps*}
    \end{byCases}
  \end{byCases}
\resetpfcounter
\end{byCases}
\end{proof}

\begin{theorem}[Progress]
 \label{thrm:progress}
 If $\hexptyp{\cdot}{\Delta}{e}{\tau}$ then either $\isFinal{e}$ or $\htrans{e}{e'}$ for some $e'$.
\end{theorem}

\begin{proof}
By rule induction over Rules (\ref{rules:TExp}) on typing judgment of $e$. For simplicity, we only consider two cases for match expressions here.
\begin{byCases}
\item[\text{(\ref{rule:TMatchZPre})}]
  \begin{pfsteps*}
  \item $\hexptyp{\cdot}{\Delta}{\hmatch{e_1}{\zruls{\cdot}{r}{rs}}}{\tau}$ \BY{assumption} \pflabel{expType}
  \item $\hexptyp{\cdot}{\Delta}{e_1}{\tau_1}$ \BY{assumption} \pflabel{scrutType}
  \item $\chrulstyp{\cdot}{\Delta}{\cfalsity}{\hrulesP{r}{rs}}{\tau_1}{\xi}{\tau}$ \BY{assumption} \pflabel{rulesType}
  \item $\csatisfyormay{\ctruth}{\xi}$ \BY{assumption} \pflabel{exhaust}
  \end{pfsteps*}
  By IH on \pfref{scrutType}.
  \begin{byCases}

  \savelocalsteps{thrm:progress-1}
  \item[\text{Scrutinee takes a step}]
    \begin{pfsteps*}
    \item $\htrans{e_1}{e_1'}$ \BY{assumption} \pflabel{scrutStep}
    \item $\htrans{\hmatch{e_1}{\zruls{\cdot}{r}{rs}}}{\hmatch{e_1'}{\zruls{\cdot}{r}{rs}}}$ \BY{Rule (\ref{rule:ITExpMatch}) on \pfref{scrutStep}}
    \end{pfsteps*}

  \restorelocalsteps{thrm:progress-1}
  \item[\text{Scrutinee is final}]
    \begin{pfsteps*}
    \item $\isFinal{e_1}$ \BY{assumption} \pflabel{scrutFinal}
    \end{pfsteps*}
    By rule induction over Rules (\ref{rules:CTRules}) on \pfref{rulesType}.
    \begin{byCases}

    \savelocalsteps{thrm:progress-2}
    \item[\text{(\ref{rule:CTOneRules})}]
      \begin{pfsteps*}
      \item $rs = \cdot$ \BY{assumption}
      \item $\xi = \xi_r$ \BY{assumption}
      \item $\chrultyp{\cdot}{\Delta}{r}{\tau_1}{\xi_r}{\tau}$ \BY{assumption} \pflabel{ruleType}
      \item $r = \hrul{p_r}{e_r}$ \BY{Inversion of Rule (\ref{rule:CTRule}) on \pfref{ruleType}}
      \item $\chpattyp{p_r}{\tau_1}{\xi_r}{\Gamma_r}{\Delta_r}$ \BY{Inversion of Rule (\ref{rule:CTRule}) on \pfref{ruleType}} \pflabel{patType}
      \item $\csatisfyormay{e_1}{\xi_r}$ \BY{Corollary \ref{corol:nn-exhaust} on \pfref{scrutFinal} and \pfref{exhaust}} \pflabel{satormayr}
      \end{pfsteps*}
      By rule induction over Rules (\ref{rules:satormay}) on \pfref{satormayr}.
      \begin{byCases}

      \savelocalsteps{thrm:progress-3}
      \item[\text{(\ref{rule:CSMSMay})}]
        \begin{pfsteps*}
        \item $\cmaysatisfy{e_1}{\xi_r}$ \BY{assumption} \pflabel{maysatr}
        \item $\hmaymatch{e_1}{p_r}$ \BY{Lemma \ref{lem:const-matching-coherence} on \pfref{scrutType} and \pfref{scrutFinal} and \pfref{patType} and \pfref{maysatr}} \pflabel{maymatchr}
        \item $\isIndet{\hmatch{e_1}{\zruls{\cdot}{\hrul{p_r}{e_r}}{\cdot}}}$ \BY{Rule (\ref{rule:IMatch}) on \pfref{scrutFinal} and \pfref{maymatchr}} \pflabel{matchIndet}
        \item $\isFinal{\hmatch{e_1}{\zruls{\cdot}{\hrul{p_r}{e_r}}{\cdot}}}$ \BY{Rule (\ref{rule:FIndet}) on \pfref{matchIndet}}
        \end{pfsteps*}

      \restorelocalsteps{thrm:progress-3}
      \item[\text{(\ref{rule:CSMSSat})}]
        \begin{pfsteps*}
        \item $\csatisfy{e_1}{\xi_r}$ \BY{assumption} \pflabel{satisfyr}
        \item $\hpatmatch{e_1}{p_r}{\theta}$ \BY{Lemma \ref{lem:const-matching-coherence} on \pfref{scrutType} and \pfref{scrutFinal} and \pfref{patType} and \pfref{satisfyr}} \pflabel{matchr}
        \item $\htrans{\hmatch{e_1}{\zruls{\cdot}{\hrul{p_r}{e_r}}{\cdot}}}{[\theta](e_r)}$ \BY{Rule (\ref{rule:ITSuccMatch}) on \pfref{scrutFinal} and \pfref{matchr}}
        \end{pfsteps*}
      \end{byCases}

    \restorelocalsteps{thrm:progress-2}
    \item[\text{(\ref{rule:CTRules})}]
      \begin{pfsteps*}
      \item $rs = \hrules{r'}{rs'}$ \BY{assumption}
      \item $\xi = \cor{\xi_r}{\xi_{rs}}$ \BY{assumption}
      \item $\chrultyp{\cdot}{\Delta}{r}{\tau_1}{\xi_r}{\tau}$ \BY{assumption} \pflabel{[rs]ruleType}
      \item $r = \hrul{p_r}{e_r}$ \BY{Inversion of Rule (\ref{rule:CTRule}) on \pfref{[rs]ruleType}}
      \item $\chpattyp{p_r}{\tau_1}{\xi_r}{\Gamma_r}{\Delta_r}$ \BY{Inversion of Rule (\ref{rule:CTRule}) on \pfref{[rs]ruleType}} \pflabel{[rs]patType}
      \end{pfsteps*}
      By Lemma \ref{lem:match-determinism} on \pfref{scrutType} and \pfref{scrutFinal} and \pfref{[rs]patType}.
      \begin{byCases}

      \savelocalsteps{thrm:progress-3}
      \item[\text{Scrutinee matches pattern}]
        \begin{pfsteps*}
        \item $\hpatmatch{e_1}{p_r}{\theta}$ \BY{assumption} \pflabel{succmatch}
        \item $\htrans{\hmatch{e_1}{\zruls{\cdot}{\hrul{p_r}{e_r}}{rs}}}{[\theta](e_r)}$ \BY{Rule (\ref{rule:ITSuccMatch}) on \pfref{scrutFinal} and \pfref{succmatch}}
        \end{pfsteps*}

      \restorelocalsteps{thrm:progress-3}
      \item[\text{Scrutinee may matches pattern}]
        \begin{pfsteps*}
        \item $\hmaymatch{e_1}{p_r}$ \BY{assumption} \pflabel{maymatch}
        \item $\isIndet{\hmatch{e_1}{\zruls{\cdot}{\hrul{p_r}{e_r}}{rs}}}$ \BY{Rule (\ref{rule:IMatch}) on \pfref{scrutFinal} and \pfref{maymatch}} \pflabel{indetmatch}
        \item $\isFinal{\hmatch{e_1}{\zruls{\cdot}{\hrul{p_r}{e_r}}{rs}}}$ \BY{Rule (\ref{rule:FIndet}) on \pfref{indetmatch}}
        \end{pfsteps*}

      \restorelocalsteps{thrm:progress-3}
      \item[\text{Scrutinee doesn't matche pattern}]
        \begin{pfsteps*}
        \item $\hnotmatch{e_1}{p_r}$ \BY{assumption} \pflabel{notmatch}
        \item $\hlongtrans{\hmatch{e_1}{\zruls{\cdot}{\hrul{p_r}{e_r}}{\hrulesP{r'}{rs'}}}}{\hmatch{e_1}{\zruls{\hrulesP{\hrul{p_r}{e_r}}{\cdot}}{r'}{rs'}}}$ \BY{Rule (\ref{rule:ITFailMatch}) on \pfref{scrutFinal} and \pfref{notmatch}}
        \end{pfsteps*}
      \end{byCases}
    \end{byCases}
  \end{byCases}
    
\resetpfcounter
\item[\text{(\ref{rule:TMatchNZPre})}]
  \begin{pfsteps*}
  \item $rs_{pre} = \hrules{r_{pre}}{rs_{pre}'}$ \BY{assumption}
  \item $\hexptyp{\cdot}{\Delta}{\hmatch{e_1}{\zruls{rs_{pre}}{r}{rs_{post}}}}{\tau}$ \BY{assumption} \pflabel{expType}
  \item $\hexptyp{\cdot}{\Delta}{e_1}{\tau_1}$ \BY{assumption} \pflabel{scrutType}
  \item $\isFinal{e_1}$ \BY{assumption} \pflabel{scrutFinal}
  \item $\chrulstyp{\cdot}{\Delta}{\xi_{pre}}{\hrulesP{r}{rs_{post}}}{\tau_1}{\xi_{rest}}{\tau}$ \BY{assumption} \pflabel{restrulesType}
  \item $\cnotsatisfyormay{e_1}{\xi_{pre}}$ \BY{assumption} \pflabel{notsatisfypre}
  \item $\csatisfyormay{\ctruth}{\cor{\xi_{pre}}{\xi_{rest}}}$ \BY{assumption} \pflabel{exhaust}
  \end{pfsteps*}
  By rule induction over Rules (\ref{rules:CTRules}) on \pfref{restrulesType}.
  \begin{byCases}

  \restorelocalsteps{thrm:progress-1}
  \item[\text{(\ref{rule:CTOneRules})}]
    \begin{pfsteps*}
    \item $rs_{post} = \cdot$ \BY{assumption}
    \item $\xi_{rest} = \xi_r$ \BY{assumption}
    \item $\chrultyp{\cdot}{\Delta}{r}{\tau_1}{\xi_r}{\tau}$ \BY{assumption} \pflabel{ruleType}
    \item $r = \hrul{p_r}{e_r}$ \BY{Inversion of Rule (\ref{rule:CTRule}) on \pfref{ruleType}}
    \item $\chpattyp{p_r}{\tau_1}{\xi_r}{\Gamma_r}{\Delta_r}$ \BY{Inversion of Rule (\ref{rule:CTRule}) on \pfref{ruleType}} \pflabel{patType}
    \item $\csatisfyormay{e_1}{\cor{\xi_{pre}}{\xi_r}}$ \BY{Corollary \ref{corol:nn-exhaust} on \pfref{scrutFinal} and \pfref{exhaust}} \pflabel{satisfypre+r}
    \item $\csatisfyormay{e_1}{\xi_r}$ \BY{Lemma \ref{lem:satisfy-substraction} on \pfref{satisfypre+r} and \pfref{notsatisfypre}} \pflabel{satormayr}
    \end{pfsteps*}
    By rule induction over Rules (\ref{rules:satormay}) on \pfref{satormayr}.
    \begin{byCases}

    \savelocalsteps{thrm:progress-2}
    \item[\text{(\ref{rule:CSMSMay})}]
      \begin{pfsteps*}
      \item $\cmaysatisfy{e_1}{\xi_r}$ \BY{assumption} \pflabel{maysatr}
      \item $\hmaymatch{e_1}{p_r}$ \BY{Lemma \ref{lem:const-matching-coherence} on \pfref{scrutType} and \pfref{scrutFinal} and \pfref{patType} and \pfref{maysatr}} \pflabel{maymatchr}
      \item $\isIndet{\hmatch{e_1}{\zruls{\cdot}{\hrul{p_r}{e_r}}{\cdot}}}$ \BY{Rule (\ref{rule:IMatch}) on \pfref{scrutFinal} and \pfref{maymatchr}} \pflabel{matchIndet}
      \item $\isFinal{\hmatch{e_1}{\zruls{\cdot}{\hrul{p_r}{e_r}}{\cdot}}}$ \BY{Rule (\ref{rule:FIndet}) on \pfref{matchIndet}}
      \end{pfsteps*}

    \restorelocalsteps{thrm:progress-2}
    \item[\text{(\ref{rule:CSMSSat})}]
      \begin{pfsteps*}
      \item $\csatisfy{e_1}{\xi_r}$ \BY{assumption} \pflabel{satisfyr}
      \item $\hpatmatch{e_1}{p_r}{\theta}$ \BY{Lemma \ref{lem:const-matching-coherence} on \pfref{scrutType} and \pfref{scrutFinal} and \pfref{patType} and \pfref{satisfyr}} \pflabel{matchr}
      \item $\htrans{\hmatch{e_1}{\zruls{\cdot}{\hrul{p_r}{e_r}}{\cdot}}}{[\theta](e_r)}$ \BY{Rule (\ref{rule:ITSuccMatch}) on \pfref{scrutFinal} and \pfref{matchr}}
      \end{pfsteps*}
    \end{byCases}

  \restorelocalsteps{thrm:progress-1}
  \item[\text{(\ref{rule:CTRules})}]
    \begin{pfsteps*}
    \item $rs_{post} = \hrules{r'}{rs_{post}'}$ \BY{assumption}
    \item $\chrultyp{\cdot}{\Delta}{r}{\tau_1}{\xi_r}{\tau}$ \BY{assumption} \pflabel{[rules]ruleType}
    \item $r = \hrul{p_r}{e_r}$ \BY{Inversion of Rule (\ref{rule:CTRule}) on \pfref{[rules]ruleType}}
    \item $\chpattyp{p_r}{\tau_1}{\xi_r}{\Gamma_r}{\Delta_r}$ \BY{Inversion of Rule (\ref{rule:CTRule}) on \pfref{[rules]ruleType}} \pflabel{[rules]patType}
    \end{pfsteps*}
    By Lemma \ref{lem:match-determinism} on \pfref{scrutType} and \pfref{scrutFinal} and \pfref{[rules]patType}.
    \begin{byCases}

    \savelocalsteps{thrm:progress-2}
    \item[\text{Scrutinee matches pattern}]
      \begin{pfsteps*}
      \item $\hpatmatch{e_1}{p_r}{\theta}$ \BY{assumption} \pflabel{succmatch}
      \item $\htrans{\hmatch{e_1}{\zruls{\cdot}{\hrul{p_r}{e_r}}{rs_{post}}}}{[\theta](e_r)}$ \BY{Rule (\ref{rule:ITSuccMatch}) on \pfref{scrutFinal} and \pfref{succmatch}}
      \end{pfsteps*}

    \restorelocalsteps{thrm:progress-2}
    \item[\text{Scrutinee may matches pattern}]
      \begin{pfsteps*}
      \item $\hmaymatch{e_1}{p_r}$ \BY{assumption} \pflabel{maymatch}
      \item $\isIndet{\hmatch{e_1}{\zruls{\cdot}{\hrul{p_r}{e_r}}{rs_{post}}}}$ \BY{Rule (\ref{rule:IMatch}) on \pfref{scrutFinal} and \pfref{maymatch}} \pflabel{indetmatch}
      \item $\isFinal{\hmatch{e_1}{\zruls{\cdot}{\hrul{p_r}{e_r}}{rs_{post}}}}$ \BY{Rule (\ref{rule:FIndet}) on \pfref{indetmatch}}
      \end{pfsteps*}

    \restorelocalsteps{thrm:progress-2}
    \item[\text{Scrutinee doesn't matche pattern}]
      \begin{pfsteps*}
      \item $\hnotmatch{e_1}{p_r}$ \BY{assumption} \pflabel{notmatch}
      \item $\hlongtrans{\hmatch{e_1}{\zruls{\cdot}{\hrul{p_r}{e_r}}{\hrulesP{r'}{rs_{post}'}}}}{\hmatch{e_1}{\zruls{\hrulesP{\hrul{p_r}{e_r}}{\cdot}}{r'}{rs_{post}'}}}$ \BY{Rule (\ref{rule:ITFailMatch}) on \pfref{scrutFinal} and \pfref{notmatch}}
      \end{pfsteps*}
      \end{byCases}
  \end{byCases}
\end{byCases}
\end{proof}

\section{Decidability}

\judgboxa{\cincon{\Xi}}{A finite set of constraints, $\Xi$, is inconsistent}
\begin{subequations}
\begin{equation}
\inferrule[CINCTruth]{
  \cincon{\Xi}
}{
  \cincon{\Xi, \ctruth}
}
\end{equation}
\begin{equation}
\inferrule[CINCFalsity]{ }{
  \cincon{\Xi, \cfalsity}
}
\end{equation}
\begin{equation}
\inferrule[CINCNum]{
  n_1 \neq n_2
}{
  \cincon{\Xi, \cnum{n_1}, \cnum{n_2}}
}
\end{equation}
\begin{equation}
\inferrule[CINCNotNum]{ }{
  \cincon{\Xi, \cnum{n}, \cnotnum{n}}
}
\end{equation}
\begin{equation}
\inferrule[CINCAnd]{
  \cincon{\Xi, \xi_1, \xi_2}
}{
  \cincon{\Xi, \cand{\xi_1}{\xi_2}}
}
\end{equation}
\begin{equation}
\inferrule[CINCOr]{
  \cincon{\Xi, \xi_1} \\
  \cincon{\Xi, \xi_2}
}{
  \cincon{\Xi, \cor{\xi_1}{\xi_2}}
}
\end{equation}
\begin{equation}
\inferrule[CINCInj]{ }{
  \cincon{\Xi, \cinl{\xi_1}, \cinr{\xi_2}}
}
\end{equation}
\begin{equation}
\inferrule[CINCInl]{
  \cincon{\Xi}
}{
  \cincon{\cinl{\Xi}}
}
\end{equation}
\begin{equation}
\inferrule[CINCInr]{
  \cincon{\Xi}
}{
  \cincon{\cinr{\Xi}}
}
\end{equation}
\begin{equation}
\inferrule[CINCPairL]{
  \cincon{\Xi_1}
}{
  \cincon{\cpair{\Xi_1}{\Xi_2}}
}
\end{equation}
\begin{equation}
\inferrule[CINCPairR]{
  \cincon{\Xi_2}
}{
  \cincon{\cpair{\Xi_1}{\Xi_2}}
}
\end{equation}
\end{subequations}

\begin{lemma}[Decidability of Inconsistency]
  \label{lem:inconsistency-decidability}
  Suppose $\ctruify{\xi}=\xi$. It is decidable whether $\cincon{\xi}$.
\end{lemma}

\begin{lemma}[Inconsistency and Entailment of Constraint]
  \label{lem:inconsistency-and-entailment}
  Suppose that $\ctruify{\xi} = \xi$. Then $\cincon{\cdual{\xi}}$ iff $\csatisfy{\ctruth}{\xi}$
\end{lemma}

\begin{lemma}
  \label{lem:satisfy-truify}
  If $\csatisfy{e}{\xi}$ then $\csatisfy{e}{\ctruify{\xi}}$
\end{lemma}
\begin{proof}
  By rule induction over Rules (\ref{rules:Satisfy}), it is obvious to see that $\ctruify{\xi} = \xi$.
\end{proof}

\begin{lemma}
  \label{lem:maysat-satormay-truify}
  If $\cmaysatisfy{e}{\xi}$ then $\csatisfyormay{e}{\ctruify{\xi}}$.
\end{lemma}
\begin{proof}
  \begin{pfsteps*}
  \item $\cmaysatisfy{e}{\xi}$ \BY{assumption} \pflabel{maysat}
  \end{pfsteps*}
  By Rule Induction over Rules (\ref{rules:MaySatisfy}) on \pfref{maysat}.
  \begin{byCases}

  \savelocalsteps{lem:maysat-satormay-truify-1}
  \item[\text{(\ref{rule:CMSUnknown})}]
      \begin{pfsteps*}
      \item $\xi = \cunknown$ \BY{assumption}
      \item $\csatisfy{e}{\ctruth}$ \BY{Rule (\ref{rule:CSTruth})} \pflabel{satisfyT}
      \item $\csatisfyormay{e}{\ctruth}$ \BY{Rule (\ref{rule:CSMSSat}) on \pfref{satisfyT}} \pflabel{satormayT}
      \end{pfsteps*}
      
  \restorelocalsteps{lem:maysat-satormay-truify-1}
  \item[\text{(\ref{rule:CMSNotVal})}]
    \begin{pfsteps*}
    \item $\isntVal{e}$ \BY{assumption} \pflabel{notval}
    \item $\refutable{\xi}$ \BY{assumption} \pflabel{rft}
    \end{pfsteps*}
    By \autoref{lem:notval-refutable} on \pfref{notval} and \pfref{rft} and case analysis on its conclusion.
    By rule induction over \rulesref{rules:xi-refutable}.
    \begin{byCases}
    \savelocalsteps{lem:maysat-satormay-truify-2}
    \item[\refutable{\ctruify{\xi}}]
      \begin{pfsteps*}
      \item $\refutable{\ctruify{\xi}}$ \BY{assumption} \pflabel{rft-truify}
      \item $\cmaysatisfy{e}{\ctruify{\xi}}$ \BY{\ruleref{rule:CMSNotVal} on \pfref{notval} and \pfref{rft-truify}} \pflabel{[notval]maysat}
      \item $\csatisfyormay{e}{\ctruify{\xi}}$ \BY{Rule (\ref{rule:CSMSSat}) on \pfref{[notval]maysat}}
      \end{pfsteps*}
    \restorelocalsteps{lem:maysat-satormay-truify-2}
    \item[\csatisfy{e}{\ctruify{\xi}}]
      \begin{pfsteps*}
      \item $\csatisfy{e}{\ctruify{\xi}}$ \BY{assumption} \pflabel{[notval]satisfy}
      \item $\csatisfyormay{e}{\ctruth}$ \BY{Rule (\ref{rule:CSMSSat}) on \pfref{[notval]satisfy}}
      \end{pfsteps*}
    \end{byCases}

  \restorelocalsteps{lem:maysat-satormay-truify-1}
  \item[\text{(\ref{rule:CMSAnd1})}]
    \begin{pfsteps*}
    \item $\xi = \cand{\xi_1}{\xi_2}$ \BY{assumption}
    \item $\cmaysatisfy{e}{\xi_1}$ \BY{assumption} \pflabel{[and1]maysat1}
    \item $\csatisfy{e}{\xi_2}$ \BY{assumption} \pflabel{[and1]satisfy2}
    \item $\csatisfyormay{e}{\ctruify{\xi_1}}$ \BY{IH on \pfref{[and1]maysat1}} \pflabel{[and1]satormay1}
    \item $\csatisfy{e}{\ctruify{\xi_2}}$ \BY{Lemma \ref{lem:satisfy-truify} on \pfref{[and1]satisfy2}} \pflabel{[and1]satisfy-truify2}
    \item $\csatisfyormay{e}{\ctruify{\xi_2}}$ \BY{Rule (\ref{rule:CSMSSat}) on \pfref{[and1]satisfy-truify2}} \pflabel{[and1]satormay2}
    \item $\csatisfyormay{e}{\cand{\ctruify{\xi_1}}{\ctruify{\xi_2}}}$ \BY{Lemma \ref{lem:satormay-and} on \pfref{[and1]satormay1} and \pfref{[and1]satormay2}}
    \end{pfsteps*}

  \restorelocalsteps{lem:maysat-satormay-truify-1}
  \item[\text{(\ref{rule:CMSAnd2})}]
    \begin{pfsteps*}
    \item $\xi = \cand{\xi_1}{\xi_2}$ \BY{assumption}
    \item $\csatisfy{e}{\xi_1}$ \BY{assumption} \pflabel{[and2]satisfy1}
    \item $\cmaysatisfy{e}{\xi_2}$ \BY{assumption} \pflabel{[and2]maysat2}
    \item $\csatisfy{e}{\ctruify{\xi_1}}$ \BY{Lemma \ref{lem:satisfy-truify} on \pfref{[and2]satisfy1}} \pflabel{[and2]satisfy-truify1}
    \item $\csatisfyormay{e}{\ctruify{\xi_1}}$ \BY{Rule (\ref{rule:CSMSSat}) on \pfref{[and2]satisfy-truify1}} \pflabel{[and2]satormay1}
    \item $\csatisfyormay{e}{\ctruify{\xi_2}}$ \BY{IH on \pfref{[and2]maysat2}} \pflabel{[and2]satormay2}
    \item $\csatisfyormay{e}{\cand{\ctruify{\xi_1}}{\ctruify{\xi_2}}}$ \BY{Lemma \ref{lem:satormay-and} on \pfref{[and2]satormay1} and \pfref{[and2]satormay2}}
    \end{pfsteps*}

  \restorelocalsteps{lem:maysat-satormay-truify-1}
  \item[\text{(\ref{rule:CMSAnd3})}]
    \begin{pfsteps*}
    \item $\xi = \cand{\xi_1}{\xi_2}$ \BY{assumption}
    \item $\cmaysatisfy{e}{\xi_1}$ \BY{assumption} \pflabel{[and3]maysat1}
    \item $\cmaysatisfy{e}{\xi_2}$ \BY{assumption} \pflabel{[and3]maysat2}
    \item $\csatisfyormay{e}{\ctruify{\xi_1}}$ \BY{IH on \pfref{[and3]maysat1}} \pflabel{[and3]satormay1}
    \item $\csatisfyormay{e}{\ctruify{\xi_2}}$ \BY{IH on \pfref{[and3]maysat2}} \pflabel{[and3]satormay2}
    \item $\csatisfyormay{e}{\cand{\ctruify{\xi_1}}{\ctruify{\xi_2}}}$ \BY{Lemma \ref{lem:satormay-and} on \pfref{[and3]satormay1} and \pfref{[and3]satormay2}}
    \end{pfsteps*}

  \restorelocalsteps{lem:maysat-satormay-truify-1}
  \item[\text{(\ref{rule:CMSOr1})}]
    \begin{pfsteps*}
    \item $\xi = \cor{\xi_1}{\xi_2}$ \BY{assumption}
    \item $\cmaysatisfy{e}{\xi_1}$ \BY{assumption} \pflabel{[or1]maysat1}
    \item $\csatisfyormay{e}{\ctruify{\xi_1}}$ \BY{IH on \pfref{[or1]maysat1}} \pflabel{[or1]satormayTruify1}
    \item $\csatisfyormay{e}{\cor{\ctruify{\xi_1}}{\ctruify{\xi_2}}}$ \BY{Lemma \ref{lem:satormay-or} on \pfref{[or1]satormayTruify1}}
    \end{pfsteps*}

  \restorelocalsteps{lem:maysat-satormay-truify-1}
  \item[\text{(\ref{rule:CMSOr2})}]
    \begin{pfsteps*}
    \item $\xi = \cor{\xi_1}{\xi_2}$ \BY{assumption}
    \item $\cmaysatisfy{e}{\xi_2}$ \BY{assumption} \pflabel{[or2]maysat2}
    \item $\csatisfyormay{e}{\ctruify{\xi_2}}$ \BY{IH on \pfref{[or2]maysat2}} \pflabel{[or2]satormayTruify2}
    \item $\csatisfyormay{e}{\cor{\ctruify{\xi_1}}{\ctruify{\xi_2}}}$ \BY{Lemma \ref{lem:satormay-or} on \pfref{[or2]satormayTruify2}}
    \end{pfsteps*}

  \restorelocalsteps{lem:maysat-satormay-truify-1}
  \item[\text{(\ref{rule:CMSInl})}]
    \begin{pfsteps*}
    \item $e = \hinl{\tau_2}{e_1}$ \BY{assumption}
    \item $\xi = \cinl{\xi_1}$ \BY{assumption}
    \item $\cmaysatisfy{e_1}{\xi_1}$ \BY{assumption} \pflabel{[inl]maysat1}
    \item $\csatisfyormay{e_1}{\ctruify{\xi_1}}$ \BY{IH on \pfref{[inl]maysat1}} \pflabel{[inl]satormayTruify1}
    \item $\csatisfyormay{\hinl{\tau_2}{e_1}}{\cinl{\ctruify{\xi_1}}}$ \BY{Lemma \ref{lem:satormay-inl} on \pfref{[inl]satormayTruify1}}
    \end{pfsteps*}

  \restorelocalsteps{lem:maysat-satormay-truify-1}
  \item[\text{(\ref{rule:CMSInr})}]
    \begin{pfsteps*}
    \item $e = \hinr{\tau_1}{e_2}$ \BY{assumption}
    \item $\xi = \cinr{\xi_2}$ \BY{assumption}
    \item $\cmaysatisfy{e_2}{\xi_2}$ \BY{assumption} \pflabel{[inr]maysat2}
    \item $\csatisfyormay{e_2}{\ctruify{\xi_2}}$ \BY{IH on \pfref{[inr]maysat2}} \pflabel{[inr]satormayTruify2}
    \item $\csatisfyormay{\hinr{\tau_1}{e_2}}{\cinr{\ctruify{\xi_2}}}$ \BY{Lemma \ref{lem:satormay-inr} on \pfref{[inr]satormayTruify2}}
    \end{pfsteps*}

  \restorelocalsteps{lem:maysat-satormay-truify-1}
  \item[\text{(\ref{rule:CMSPair1})}]
    \begin{pfsteps*}
    \item $e = \hpair{e_1}{e_2}$ \BY{assumption}
    \item $\xi = \cpair{\xi_1}{\xi_2}$ \BY{assumption}
    \item $\cmaysatisfy{e_1}{\xi_1}$ \BY{assumption} \pflabel{[pair1]maysat1}
    \item $\csatisfy{e_2}{\xi_2}$ \BY{assumption} \pflabel{[pair1]satisfy2}
    \item $\csatisfyormay{e_1}{\ctruify{\xi_1}}$ \BY{IH on \pfref{[pair1]maysat1}} \pflabel{[pair1]satormay1}
    \item $\csatisfy{e_2}{\ctruify{\xi_2}}$ \BY{Lemma \ref{lem:satisfy-truify} on \pfref{[pair1]satisfy2}} \pflabel{[pair1]satisfy-truify2}
    \item $\csatisfyormay{e_2}{\ctruify{\xi_2}}$ \BY{Rule (\ref{rule:CSMSSat}) on \pfref{[pair1]satisfy-truify2}} \pflabel{[pair1]satormay2}
    \item $\csatisfyormay{\hpair{e_1}{e_2}}{\cpair{\ctruify{\xi_1}}{\ctruify{\xi_2}}}$ \BY{Lemma \ref{lem:satormay-pair} on \pfref{[pair1]satormay1} and \pfref{[pair1]satormay2}}
    \end{pfsteps*}

  \restorelocalsteps{lem:maysat-satormay-truify-1}
  \item[\text{(\ref{rule:CMSPair2})}]
    \begin{pfsteps*}
    \item $e = \hpair{e_1}{e_2}$ \BY{assumption}
    \item $\xi = \cpair{\xi_1}{\xi_2}$ \BY{assumption}
    \item $\csatisfy{e_1}{\xi_1}$ \BY{assumption} \pflabel{[pair2]satisfy1}
    \item $\cmaysatisfy{e_2}{\xi_2}$ \BY{assumption} \pflabel{[pair2]maysat2}
    \item $\csatisfy{e_1}{\ctruify{\xi_1}}$ \BY{Lemma \ref{lem:satisfy-truify} on \pfref{[pair2]satisfy1}} \pflabel{[pair2]satisfy-truify1}
    \item $\csatisfyormay{e_1}{\ctruify{\xi_1}}$ \BY{Rule (\ref{rule:CSMSSat}) on \pfref{[pair2]satisfy-truify1}} \pflabel{[pair2]satormay1}
    \item $\csatisfyormay{e_2}{\ctruify{\xi_2}}$ \BY{IH on \pfref{[pair2]maysat2}} \pflabel{[pair2]satormay2}
    \item $\csatisfyormay{\hpair{e_1}{e_2}}{\cpair{\ctruify{\xi_1}}{\ctruify{\xi_2}}}$ \BY{Lemma \ref{lem:satormay-pair} on \pfref{[pair2]satormay1} and \pfref{[pair2]satormay2}}
    \end{pfsteps*}

  \restorelocalsteps{lem:maysat-satormay-truify-1}
  \item[\text{(\ref{rule:CMSPair3})}]
    \begin{pfsteps*}
    \item $e = \hpair{e_1}{e_2}$ \BY{assumption}
    \item $\xi = \cpair{\xi_1}{\xi_2}$ \BY{assumption}
    \item $\cmaysatisfy{e_1}{\xi_1}$ \BY{assumption} \pflabel{[Pair3]maysat1}
    \item $\cmaysatisfy{e_2}{\xi_2}$ \BY{assumption} \pflabel{[Pair3]maysat2}
    \item $\csatisfyormay{e_1}{\ctruify{\xi_1}}$ \BY{IH on \pfref{[Pair3]maysat1}} \pflabel{[Pair3]satormay1}
    \item $\csatisfyormay{e_2}{\ctruify{\xi_2}}$ \BY{IH on \pfref{[Pair3]maysat2}} \pflabel{[Pair3]satormay2}
    \item $\csatisfyormay{\hpair{e_1}{e_2}}{\cpair{\ctruify{\xi_1}}{\ctruify{\xi_2}}}$ \BY{Lemma \ref{lem:satormay-pair} on \pfref{[Pair3]satormay1} and \pfref{[Pair3]satormay2}}
    \end{pfsteps*}

  \end{byCases}

  \resetpfcounter
\end{proof}

\begin{lemma}
  \label{lem:satormay-truify}
  $\csatisfyormay{e}{\xi}$ iff $\csatisfyormay{e}{\ctruify{\xi}}$
\end{lemma}
\begin{proof}
  \begin{enumerate}
    \item Sufficiency:
      \begin{pfsteps*}
      \item $\csatisfyormay{e}{\xi}$ \BY{assumption} \pflabel{satormay}
      \end{pfsteps*}
      By rule induction over Rules (\ref{rules:satormay}) on \pfref{satormay}
      \savelocalsteps{lem:satormay-truify-suff-1}
      \begin{byCases}

      \item[\text{(\ref{rule:CSMSSat})}]
        \begin{pfsteps*}
        \item $\csatisfy{e}{\xi}$ \BY{assumption} \pflabel{sat}
        \item $\csatisfy{e}{\ctruify{\xi}}$ \BY{Lemma \ref{lem:satisfy-truify} on \pfref{sat}} \pflabel{[sat]sat-truify}
        \item $\csatisfyormay{e}{\ctruify{\xi}}$ \BY{Rule (\ref{rule:CSMSSat}) on \pfref{[sat]sat-truify}}
        \end{pfsteps*}

      \restorelocalsteps{lem:satormay-truify-suff-1}
      \item[\text{(\ref{rule:CSMSMay})}]
        \begin{pfsteps*}
        \item $\cmaysatisfy{e}{\xi}$ \BY{assumption} \pflabel{maysat}
        \item $\csatisfyormay{e}{\ctruify{\xi}}$ \BY{Lemma \ref{lem:maysat-satormay-truify} on \pfref{maysat}}
        \end{pfsteps*}

      \end{byCases}

    \resetpfcounter

    \item Necessity:
    \begin{pfsteps*}
    \item $\csatisfyormay{e}{\ctruify{\xi}}$ \BY{assumption} \pflabel{satormayTruify}
    \end{pfsteps*}
    By structural induction on $\xi$,
    \begin{byCases}

    \savelocalsteps{lem:satormay-truify-necs-1}
    \item[\xi=\ctruth, \cfalsity, \cnum{n}, \cnotnum{n}]
      \begin{pfsteps*}
      \item $\csatisfyormay{e}{\xi}$ \BY{\pfref{satormayTruify} and Definition \ref{defn:truify}}
      \end{pfsteps*}

    \restorelocalsteps{lem:satormay-truify-necs-1}
    \item[\xi=\cunknown]
      \begin{pfsteps*}
      \item $\cmaysatisfy{e}{\cunknown}$ \BY{Rule (\ref{rule:CMSUnknown})} \pflabel{[unknown]maysatUnknown}
      \item $\csatisfyormay{e}{\cunknown}$ \BY{Rule (\ref{rule:CSMSMay}) on \pfref{[unknown]maysatUnknown}}
      \end{pfsteps*}
    
    \restorelocalsteps{lem:satormay-truify-necs-1}
    \item[\xi=\cand{\xi_1}{\xi_2}]
      \begin{pfsteps*}
      \item $\ctruify{\xi} = \cand{\ctruify{\xi_1}}{\ctruify{\xi_2}}$ \BY{Definition \ref{defn:truify}}
      \end{pfsteps*}
      By rule induction over Rules (\ref{rules:satormay}) on \pfref{satormayTruify},
      \savelocalsteps{lem:satormay-truify-necs-2}
      \begin{byCases}

      \item[\text{(\ref{rule:CSMSSat})}]
        \begin{pfsteps*}
        \item $\csatisfy{e}{\cand{\ctruify{\xi_1}}{\ctruify{\xi_2}}}$ \BY{assumption} \pflabel{satisfyAndTruify}
        \end{pfsteps*}
        By rule induction over Rules (\ref{rules:Satisfy}) on \pfref{satisfyAndTruify} and only one case applies,
        \begin{byCases}
        \item[\text{(\ref{rule:CSAnd})}]
          \begin{pfsteps*}
          \item $\csatisfy{e}{\ctruify{\xi_1}}$ \BY{assumption} \pflabel{[and]satisfyTruify1}
          \item $\csatisfy{e}{\ctruify{\xi_2}}$ \BY{assumption} \pflabel{[and]satisfyTruify2}
          \item $\csatisfyormay{e}{\ctruify{\xi_1}}$ \BY{Rule (\ref{rule:CSMSSat}) on \pfref{[and]satisfyTruify1}} \pflabel{[and]satormayTruify1}
          \item $\csatisfyormay{e}{\ctruify{\xi_2}}$ \BY{Rule (\ref{rule:CSMSSat}) on \pfref{[and]satisfyTruify2}} \pflabel{[and]satormayTruify2}
          \item $\csatisfyormay{e}{\xi_1}$ \BY{IH on \pfref{[and]satormayTruify1}} \pflabel{[and]satormay1}
          \item $\csatisfyormay{e}{\xi_2}$ \BY{IH on \pfref{[and]satormayTruify2}} \pflabel{[and]satormay2}
          \item $\csatisfyormay{e}{\cand{\xi_1}{\xi_2}}$ \BY{Lemma \ref{lem:satormay-and} on \pfref{[and]satormay1} and \pfref{[and]satormay2}}
          \end{pfsteps*}
        \end{byCases}

      \restorelocalsteps{lem:satormay-truify-necs-2}
      \item[\text{(\ref{rule:CSMSMay})}]
        \begin{pfsteps*}
        \item $\cmaysatisfy{e}{\cand{\ctruify{\xi_1}}{\ctruify{\xi_2}}}$ \BY{assumption} \pflabel{maysatAndTruify}
        \end{pfsteps*}
        By rule induction over Rules (\ref{rules:MaySatisfy}) on \pfref{maysatAndTruify} and three cases apply,
        \savelocalsteps{lem:satormay-truify-necs-3}
        \begin{byCases}

        \item[\text{(\ref{rule:CMSAnd1})}]
          \begin{pfsteps*}
          \item $\cmaysatisfy{e}{\ctruify{\xi_1}}$ \BY{assumption} \pflabel{[and1]maysatTruify1}
          \item $\csatisfy{e}{\ctruify{\xi_2}}$ \BY{assumption} \pflabel{[and1]satisfyTruify2}
          \item $\csatisfyormay{e}{\ctruify{\xi_1}}$ \BY{Rule (\ref{rule:CSMSMay}) on \pfref{[and1]maysatTruify1}} \pflabel{[and1]satormayTruify1}
          \item $\csatisfyormay{e}{\ctruify{\xi_2}}$ \BY{Rule (\ref{rule:CSMSSat}) on \pfref{[and1]satisfyTruify2}} \pflabel{[and1]satormayTruify2}
          \item $\csatisfyormay{e}{\xi_1}$ \BY{IH on \pfref{[and1]satormayTruify1}} \pflabel{[and1]satormay1}
          \item $\csatisfyormay{e}{\xi_2}$ \BY{IH on \pfref{[and1]satormayTruify2}} \pflabel{[and1]satormay2}
          \item $\csatisfyormay{e}{\cand{\xi_1}{\xi_2}}$ \BY{Lemma \ref{lem:satormay-and} on \pfref{[and1]satormay1} and \pfref{[and1]satormay2}}
          \end{pfsteps*}

        \restorelocalsteps{lem:satormay-truify-necs-3}
        \item[\text{(\ref{rule:CMSAnd2})}]
          \begin{pfsteps*}
          \item $\csatisfy{e}{\ctruify{\xi_1}}$ \BY{assumption} \pflabel{[and2]satisfyTruify1}
          \item $\cmaysatisfy{e}{\ctruify{\xi_2}}$ \BY{assumption} \pflabel{[and2]maysatTruify2}
          \item $\csatisfyormay{e}{\ctruify{\xi_1}}$ \BY{Rule (\ref{rule:CSMSSat}) on \pfref{[and2]satisfyTruify1}} \pflabel{[and2]satormayTruify1}
          \item $\csatisfyormay{e}{\ctruify{\xi_2}}$ \BY{Rule (\ref{rule:CSMSMay}) on \pfref{[and2]maysatTruify2}} \pflabel{[and2]satormayTruify2}
          \item $\csatisfyormay{e}{\xi_1}$ \BY{IH on \pfref{[and2]satormayTruify1}} \pflabel{[and2]satormay1}
          \item $\csatisfyormay{e}{\xi_2}$ \BY{IH on \pfref{[and2]satormayTruify2}} \pflabel{[and2]satormay2}
          \item $\csatisfyormay{e}{\cand{\xi_1}{\xi_2}}$ \BY{Lemma \ref{lem:satormay-and} on \pfref{[and2]satormay1} and \pfref{[and2]satormay2}}
          \end{pfsteps*}

        \restorelocalsteps{lem:satormay-truify-necs-3}
        \item[\text{(\ref{rule:CMSAnd3})}]
          \begin{pfsteps*}
          \item $\cmaysatisfy{e}{\ctruify{\xi_1}}$ \BY{assumption} \pflabel{[and3]maysatTruify1}
          \item $\cmaysatisfy{e}{\ctruify{\xi_2}}$ \BY{assumption} \pflabel{[and3]maysatTruify2}
          \item $\csatisfyormay{e}{\ctruify{\xi_1}}$ \BY{Rule (\ref{rule:CSMSMay}) on \pfref{[and3]maysatTruify1}} \pflabel{[and3]satormayTruify1}
          \item $\csatisfyormay{e}{\ctruify{\xi_2}}$ \BY{Rule (\ref{rule:CSMSMay}) on \pfref{[and3]maysatTruify2}} \pflabel{[and3]satormayTruify2}
          \item $\csatisfyormay{e}{\xi_1}$ \BY{IH on \pfref{[and3]satormayTruify1}} \pflabel{[and3]satormay1}
          \item $\csatisfyormay{e}{\xi_2}$ \BY{IH on \pfref{[and3]satormayTruify2}} \pflabel{[and3]satormay2}
          \item $\csatisfyormay{e}{\cand{\xi_1}{\xi_2}}$ \BY{Lemma \ref{lem:satormay-and} on \pfref{[and3]satormay1} and \pfref{[and3]satormay2}}
          \end{pfsteps*}
        \end{byCases}
      \end{byCases}

    \restorelocalsteps{lem:satormay-truify-necs-1}
    \item[\xi=\cor{\xi_1}{\xi_2}]
      \begin{pfsteps*}
      \item $\ctruify{\cor{\xi_1}{\xi_2}} = \cor{\ctruify{\xi_1}}{\ctruify{\xi_2}}$ \BY{Definition \ref{defn:truify}}
      \end{pfsteps*}
      By rule induction over Rules (\ref{rules:satormay}) on \pfref{satormayTruify},
      \savelocalsteps{lem:satormay-truify-necs-2}
      \begin{byCases}

      \item[\text{(\ref{rule:CSMSSat})}]
        \begin{pfsteps*}
        \item $\csatisfy{e}{\cor{\ctruify{\xi_1}}{\ctruify{\xi_2}}}$ \BY{assumption} \pflabel{satisfyOrTruify}
        \end{pfsteps*}
        By rule induction over Rules (\ref{rules:Satisfy}) on \pfref{satisfyOrTruify} and two cases apply,
        \begin{byCases}

        \savelocalsteps{lem:satormay-truify-necs-3}
        \item[\text(\ref{rule:CSOr1})]
          \begin{pfsteps*}
          \item $\csatisfy{e}{\ctruify{\xi_1}}$ \BY{assumption} \pflabel{[or1]satisfyTruify1}
          \item $\csatisfyormay{e}{\ctruify{\xi_1}}$ \BY{Rule (\ref{rule:CSMSSat}) on \pfref{[or1]satisfyTruify1}} \pflabel{[or1]satormayTruify1}
          \item $\csatisfyormay{e}{\xi_1}$ \BY{IH on \pfref{[or1]satormayTruify1}} \pflabel{[or1]satormay1}
          \item $\csatisfyormay{e}{\cor{\xi_1}{\xi_2}}$ \BY{Lemma \ref{lem:satormay-or} on \pfref{[or1]satormay1}}
          \end{pfsteps*}

        \restorelocalsteps{lem:satormay-truify-necs-3}
        \item[\text(\ref{rule:CSOr2})]
          \begin{pfsteps*}
          \item $\csatisfy{e}{\ctruify{\xi_2}}$ \BY{assumption} \pflabel{[or2]satisfyTruify2}
          \item $\csatisfyormay{e}{\ctruify{\xi_2}}$ \BY{Rule (\ref{rule:CSMSSat}) on \pfref{[or2]satisfyTruify2}} \pflabel{[or2]satormayTruify2}
          \item $\csatisfyormay{e}{\xi_2}$ \BY{IH on \pfref{[or2]satormayTruify2}} \pflabel{[or2]satormay2}
          \item $\csatisfyormay{e}{\cor{\xi_1}{\xi_2}}$ \BY{Lemma \ref{lem:satormay-or} on \pfref{[or2]satormay2}}
          \end{pfsteps*}

        \end{byCases}

      \restorelocalsteps{lem:satormay-truify-necs-2}
      \item[\text{(\ref{rule:CSMSMay})}]
        \begin{pfsteps*}
        \item $\cmaysatisfy{e}{\cor{\ctruify{\xi_1}}{\ctruify{\xi_2}}}$ \BY{assumption} \pflabel{maysatOrTruify}
        \end{pfsteps*}
        By rule induction over Rules (\ref{rules:MaySatisfy}) on \pfref{maysatOrTruify} and two cases apply,
        \begin{byCases}

        \savelocalsteps{lem:satormay-truify-necs-3}
        \item[\text{(\ref{rule:CMSOr1})}]
          \begin{pfsteps*}
          \item $\cmaysatisfy{e}{\ctruify{\xi_1}}$ \BY{assumption} \pflabel{[or3]maysatTruify1}
          \item $\csatisfyormay{e}{\ctruify{\xi_1}}$ \BY{Rule (\ref{rule:CSMSMay}) on \pfref{[or3]maysatTruify1}} \pflabel{[or3]satormayTruify1}
          \item $\csatisfyormay{e}{\xi_1}$ \BY{IH on \pfref{[or3]satormayTruify1}} \pflabel{[or3]satormay1}
          \item $\csatisfyormay{e}{\cor{\xi_1}{\xi_2}}$ \BY{Lemma \ref{lem:satormay-or} on \pfref{[or3]satormay1}}
          \end{pfsteps*}
          
        \restorelocalsteps{lem:satormay-truify-necs-3}
        \item[\text{(\ref{rule:CMSOr2})}]
          \begin{pfsteps*}
          \item $\cmaysatisfy{e}{\ctruify{\xi_2}}$ \BY{assumption} \pflabel{[or4]maysatTruify2}
          \item $\csatisfyormay{e}{\ctruify{\xi_2}}$ \BY{Rule (\ref{rule:CSMSMay}) on \pfref{[or4]maysatTruify2}} \pflabel{[or4]satormayTruify2}
          \item $\csatisfyormay{e}{\xi_2}$ \BY{IH on \pfref{[or4]satormayTruify2}} \pflabel{[or4]satormay2}
          \item $\csatisfyormay{e}{\cor{\xi_1}{\xi_2}}$ \BY{Lemma \ref{lem:satormay-or} on \pfref{[or4]satormay2}}
          \end{pfsteps*}
          
        \end{byCases}
      \end{byCases}

    \restorelocalsteps{lem:satormay-truify-necs-1}
    \item[\xi=\cinl{\xi_1}]
      \begin{pfsteps*}
      \item $e = \hinl{\tau_2}{e_1}$ \BY{assumption}
      \item $\ctruify{\xi} = \cinl{\ctruify{\xi_1}}$ \BY{assumption}
      \end{pfsteps*}
      By rule induction over Rules (\ref{rules:satormay}) on \pfref{satormayTruify},
      \begin{byCases}

      \savelocalsteps{lem:satormay-truify-necs-2}
      \item[\text{(\ref{rule:CSMSSat})}]
        \begin{pfsteps*}
        \item $\csatisfy{\hinl{\tau_2}{e_1}}{\cinl{\ctruify{\xi_1}}}$ \BY{assumption} \pflabel{satisfyInlTruify1}
        \end{pfsteps*}
        By rule induction over Rules (\ref{rules:Satisfy}) and only one case applies,
        \begin{byCases}
        \item[\text{(\ref{rule:CSInl})}]
          \begin{pfsteps*}
          \item $\csatisfy{e_1}{\ctruify{\xi_1}}$ \BY{assumption} \pflabel{satisfyTruify1}
          \item $\csatisfyormay{e_1}{\ctruify{\xi_1}}$ \BY{Rule (\ref{rule:CSMSSat}) on \pfref{satisfyTruify1}} \pflabel{satormayTruify1}
          \item $\csatisfyormay{e_1}{\xi_1}$ \BY{IH on \pfref{satormayTruify1}} \pflabel{satormay1}
          \item $\csatisfyormay{\hinl{\tau_2}{e_1}}{\cinl{\xi_1}}$ \BY{Lemma \ref{lem:satormay-inl} on \pfref{satormay1}}
          \end{pfsteps*}
        \end{byCases}

      \restorelocalsteps{lem:satormay-truify-necs-2}
      \item[\text{(\ref{rule:CSMSMay})}]
        \begin{pfsteps*}
        \item $\cmaysatisfy{\hinl{\tau_2}{e_1}}{\cinl{\ctruify{\xi_1}}}$ \BY{assumption} \pflabel{maysatInlTruify1}
        \end{pfsteps*}
        By rule induction over Rules (\ref{rules:MaySatisfy}) and only one case applies,
        \begin{byCases}
        \item[\text{(\ref{rule:CMSInl})}]
          \begin{pfsteps*}
          \item $\cmaysatisfy{e_1}{\ctruify{\xi_1}}$ \BY{assumption} \pflabel{maysatTruify1}
          \item $\csatisfyormay{e_1}{\ctruify{\xi_1}}$ \BY{Rule (\ref{rule:CSMSMay}) on \pfref{maysatTruify1}} \pflabel{[may]satormayTruify1}
          \item $\csatisfyormay{e_1}{\xi_1}$ \BY{IH on \pfref{[may]satormayTruify1}} \pflabel{[may]satormay1}
          \item $\csatisfyormay{\hinl{\tau_2}{e_1}}{\cinl{\xi_1}}$ \BY{Lemma \ref{lem:satormay-inl} on \pfref{[may]satormay1}}
          \end{pfsteps*}
        \end{byCases}
      \end{byCases}

    \restorelocalsteps{lem:satormay-truify-necs-1}
    \item[\xi=\cinr{\xi_2}]
      \begin{pfsteps*}
      \item $e = \hinr{\tau_1}{e_2}$ \BY{assumption}
      \item $\ctruify{\xi} = \cinr{\ctruify{\xi_2}}$ \BY{assumption}
      \end{pfsteps*}
      By rule induction over Rules (\ref{rules:satormay}) on \pfref{satormayTruify},
      \begin{byCases}

      \savelocalsteps{lem:satormay-truify-necs-2}
      \item[\text{(\ref{rule:CSMSSat})}]
        \begin{pfsteps*}
        \item $\csatisfy{\hinr{\tau_1}{e_2}}{\cinr{\ctruify{\xi_2}}}$ \BY{assumption} \pflabel{satisfyInrTruify2}
        \end{pfsteps*}
        By rule induction over Rules (\ref{rules:Satisfy}) and only one case applies,
        \begin{byCases}
        \item[\text{(\ref{rule:CSInr})}]
          \begin{pfsteps*}
          \item $\csatisfy{e_2}{\ctruify{\xi_2}}$ \BY{assumption} \pflabel{satisfyTruify2}
          \item $\csatisfyormay{e_2}{\ctruify{\xi_2}}$ \BY{Rule (\ref{rule:CSMSSat}) on \pfref{satisfyTruify2}} \pflabel{satormayTruify2}
          \item $\csatisfyormay{e_2}{\xi_2}$ \BY{IH on \pfref{satormayTruify2}} \pflabel{satormay2}
          \item $\csatisfyormay{\hinr{\tau_1}{e_2}}{\cinr{\xi_2}}$ \BY{Lemma \ref{lem:satormay-inr} on \pfref{satormay2}}
          \end{pfsteps*}
        \end{byCases}

      \restorelocalsteps{lem:satormay-truify-necs-2}
      \item[\text{(\ref{rule:CSMSMay})}]
        \begin{pfsteps*}
        \item $\cmaysatisfy{\hinr{\tau_1}{e_2}}{\cinr{\ctruify{\xi_2}}}$ \BY{assumption} \pflabel{maysatInrTruify2}
        \end{pfsteps*}
        By rule induction over Rules (\ref{rules:MaySatisfy}) and only one case applies,
        \begin{byCases}
        \item[\text{(\ref{rule:CMSInr})}]
          \begin{pfsteps*}
          \item $\cmaysatisfy{e_2}{\ctruify{\xi_2}}$ \BY{assumption} \pflabel{maysatTruify2}
          \item $\csatisfyormay{e_2}{\ctruify{\xi_2}}$ \BY{Rule (\ref{rule:CSMSMay}) on \pfref{maysatTruify2}} \pflabel{[may]satormayTruify2}
          \item $\csatisfyormay{e_2}{\xi_2}$ \BY{IH on \pfref{[may]satormayTruify2}} \pflabel{[may]satormay2}
          \item $\csatisfyormay{\hinr{\tau_1}{e_2}}{\cinr{\xi_2}}$ \BY{Lemma \ref{lem:satormay-inr} on \pfref{[may]satormay2}}
          \end{pfsteps*}
        \end{byCases}
      \end{byCases}

    \restorelocalsteps{lem:satormay-truify-necs-1}
    \item[\xi=\cpair{\xi_1}{\xi_2}]
      \begin{pfsteps*}
      \item $e = \hpair{e_1}{e_2}$ \BY{assumption}
      \item $\ctruify{\xi} = \cand{\ctruify{\xi_1}}{\ctruify{\xi_2}}$ \BY{Definition \ref{defn:truify}}
      \end{pfsteps*}
      By rule induction over Rules (\ref{rules:satormay}) on \pfref{satormayTruify},
      \begin{byCases}

      \savelocalsteps{lem:satormay-truify-necs-2}
      \item[\text{(\ref{rule:CSMSSat})}]
        \begin{pfsteps*}
        \item $\csatisfy{\hpair{e_1}{e_2}}{\cpair{\ctruify{\xi_1}}{\ctruify{\xi_2}}}$ \BY{assumption} \pflabel{satisfyPairTruify}
        \end{pfsteps*}
        By rule induction over Rules (\ref{rules:Satisfy}) on \pfref{satisfyPairTruify} and only one case applies,
        \begin{byCases}
        \item[\text{(\ref{rule:CSPair})}]
          \begin{pfsteps*}
          \item $\csatisfy{e_1}{\ctruify{\xi_1}}$ \BY{assumption} \pflabel{[pair]satisfyTruify1}
          \item $\csatisfy{e_2}{\ctruify{\xi_2}}$ \BY{assumption} \pflabel{[pair]satisfyTruify2}
          \item $\csatisfyormay{e_1}{\ctruify{\xi_1}}$ \BY{Rule (\ref{rule:CSMSSat}) on \pfref{[pair]satisfyTruify1}} \pflabel{[pair]satormayTruify1}
          \item $\csatisfyormay{e_2}{\ctruify{\xi_2}}$ \BY{Rule (\ref{rule:CSMSSat}) on \pfref{[pair]satisfyTruify2}} \pflabel{[pair]satormayTruify2}
          \item $\csatisfyormay{e_1}{\xi_1}$ \BY{IH on \pfref{[pair]satormayTruify1}} \pflabel{[pair]satormay1}
          \item $\csatisfyormay{e_2}{\xi_2}$ \BY{IH on \pfref{[pair]satormayTruify2}} \pflabel{[pair]satormay2}
          \item $\csatisfyormay{\hpair{e_1}{e_2}}{\cpair{\xi_1}{\xi_2}}$ \BY{Lemma \ref{lem:satormay-pair} on \pfref{[pair]satormay1} and \pfref{[pair]satormay2}}
          \end{pfsteps*}
        \end{byCases}

      \restorelocalsteps{lem:satormay-truify-necs-2}
      \item[\text{(\ref{rule:CSMSMay})}]
        \begin{pfsteps*}
        \item $\cmaysatisfy{\hpair{e_1}{e_2}}{\cpair{\ctruify{\xi_1}}{\ctruify{\xi_2}}}$ \BY{assumption} \pflabel{maysatPairTruify}
        \end{pfsteps*}
        By rule induction over Rules (\ref{rules:MaySatisfy}) on \pfref{maysatPairTruify} and three cases apply,
        \begin{byCases}

        \savelocalsteps{lem:satormay-truify-necs-3}
        \item[\text{(\ref{rule:CMSPair1})}]
          \begin{pfsteps*}
          \item $\cmaysatisfy{e_1}{\ctruify{\xi_1}}$ \BY{assumption} \pflabel{[pair1]maysatTruify1}
          \item $\csatisfy{e_2}{\ctruify{\xi_2}}$ \BY{assumption} \pflabel{[pair1]satisfyTruify2}
          \item $\csatisfyormay{e_1}{\ctruify{\xi_1}}$ \BY{Rule (\ref{rule:CSMSMay}) on \pfref{[pair1]maysatTruify1}} \pflabel{[pair1]satormayTruify1}
          \item $\csatisfyormay{e_2}{\ctruify{\xi_2}}$ \BY{Rule (\ref{rule:CSMSSat}) on \pfref{[pair1]satisfyTruify2}} \pflabel{[pair1]satormayTruify2}
          \item $\csatisfyormay{e_1}{\xi_1}$ \BY{IH on \pfref{[pair1]satormayTruify1}} \pflabel{[pair1]satormay1}
          \item $\csatisfyormay{e_2}{\xi_2}$ \BY{IH on \pfref{[pair1]satormayTruify2}} \pflabel{[pair1]satormay2}
          \item $\csatisfyormay{\hpair{e_1}{e_2}}{\cpair{\xi_1}{\xi_2}}$ \BY{Lemma \ref{lem:satormay-pair} on \pfref{[pair1]satormay1} and \pfref{[pair1]satormay2}}
          \end{pfsteps*}

        \restorelocalsteps{lem:satormay-truify-necs-3}
        \item[\text{(\ref{rule:CMSPair2})}]
          \begin{pfsteps*}
          \item $\csatisfy{e_1}{\ctruify{\xi_1}}$ \BY{assumption} \pflabel{[pair2]satisfyTruify1}
          \item $\cmaysatisfy{e_2}{\ctruify{\xi_2}}$ \BY{assumption} \pflabel{[pair2]maysatTruify2}
          \item $\csatisfyormay{e_1}{\ctruify{\xi_1}}$ \BY{Rule (\ref{rule:CSMSSat}) on \pfref{[pair2]satisfyTruify1}} \pflabel{[pair2]satormayTruify1}
          \item $\csatisfyormay{e_2}{\ctruify{\xi_2}}$ \BY{Rule (\ref{rule:CSMSMay}) on \pfref{[pair2]maysatTruify2}} \pflabel{[pair2]satormayTruify2}
          \item $\csatisfyormay{e_1}{\xi_1}$ \BY{IH on \pfref{[pair2]satormayTruify1}} \pflabel{[pair2]satormay1}
          \item $\csatisfyormay{e_2}{\xi_2}$ \BY{IH on \pfref{[pair2]satormayTruify2}} \pflabel{[pair2]satormay2}
          \item $\csatisfyormay{\hpair{e_1}{e_2}}{\cpair{\xi_1}{\xi_2}}$ \BY{Lemma \ref{lem:satormay-pair} on \pfref{[pair2]satormay1} and \pfref{[pair2]satormay2}}
          \end{pfsteps*}

        \restorelocalsteps{lem:satormay-truify-necs-3}
        \item[\text{(\ref{rule:CMSAnd3})}]
          \begin{pfsteps*}
          \item $\cmaysatisfy{e_1}{\ctruify{\xi_1}}$ \BY{assumption} \pflabel{[pair3]maysatTruify1}
          \item $\cmaysatisfy{e_2}{\ctruify{\xi_2}}$ \BY{assumption} \pflabel{[pair3]maysatTruify2}
          \item $\csatisfyormay{e_1}{\ctruify{\xi_1}}$ \BY{Rule (\ref{rule:CSMSMay}) on \pfref{[pair3]maysatTruify1}} \pflabel{[pair3]satormayTruify1}
          \item $\csatisfyormay{e_2}{\ctruify{\xi_2}}$ \BY{Rule (\ref{rule:CSMSMay}) on \pfref{[pair3]maysatTruify2}} \pflabel{[pair3]satormayTruify2}
          \item $\csatisfyormay{e_1}{\xi_1}$ \BY{IH on \pfref{[pair3]satormayTruify1}} \pflabel{[pair3]satormay1}
          \item $\csatisfyormay{e_2}{\xi_2}$ \BY{IH on \pfref{[pair3]satormayTruify2}} \pflabel{[pair3]satormay2}
          \item $\csatisfyormay{\hpair{e_1}{e_2}}{\cpair{\xi_1}{\xi_2}}$ \BY{Lemma \ref{lem:satormay-pair} on \pfref{[pair3]satormay1} and \pfref{[pair3]satormay2}}
          \end{pfsteps*}
        \end{byCases}
      \end{byCases}

    \end{byCases}
      
    \resetpfcounter
  \end{enumerate}
\end{proof}

\begin{lemma}
\label{lem:truify-satormay-satisfy}
  Assume $\ctruify{\xi}=\xi$. Then $\csatisfyormay{\ctruth}{\xi}$ iff $\csatisfy{\ctruth}{\xi}$.
\end{lemma}
\begin{proof}
We prove sufficiency and necessity separately.
\begin{enumerate}
\item Sufficiency:
    
\item Necessity:
    
\end{enumerate}
\end{proof}

\begin{theorem}
\label{thrm:exhaustive-truify}
  $\csatisfyormay{\ctruth}{\xi}$ iff $\csatisfy{\ctruth}{\ctruify{\xi}}$.
\end{theorem}

\begin{lemma}
  \label{lem:val-satisfy-truify}
  Assume that $\isVal{e}$. Then $\csatisfyormay{e}{\xi}$ iff $\csatisfy{e}{\ctruify{\xi}}$
\end{lemma}
\begin{proof}
  \begin{pfsteps*}
  \item $\isVal{e}$ \BY{assumption} \pflabel{val}
  \end{pfsteps*}
  We prove sufficiency and necessity separately.
  \begin{enumerate}

    \savelocalsteps{0}
    \item Sufficiency:
      \begin{pfsteps*}
      \item $\csatisfyormay{e}{\xi}$ \BY{assumption} \pflabel{satormay}
      \end{pfsteps*}
      By rule induction over Rules (\ref{rules:satormay}) on \pfref{satormay}.
      \begin{byCases}

      \savelocalsteps{1}
      \item[\text{(\ref{rule:CSMSSat})}]
        \begin{pfsteps*}
        \item $\csatisfy{e}{\xi}$ \BY{assumption} \pflabel{satisfy}
        \item $\csatisfy{e}{\ctruify{\xi}}$ \BY{Lemma \ref{lem:satisfy-truify} on \pfref{satisfy}}
        \end{pfsteps*}
      
      \restorelocalsteps{1}
      \item[\text{(\ref{rule:CSMSMay})}]
        \begin{pfsteps*}
        \item $\cmaysatisfy{e}{\xi}$ \BY{assumption} \pflabel{maysat}
        \end{pfsteps*}
        By rule induction over Rules (\ref{rules:MaySatisfy}) on \pfref{maysat}.
        \begin{byCases}

        \savelocalsteps{2}
        \item[\text{(\ref{rule:CMSUnknown})}]
          \begin{pfsteps*}
          \item $\xi=\cunknown$ \BY{assumption}
          \item $\csatisfy{e}{\ctruify{\xi}}$ \BY{Rule (\ref{rule:CSTruth}) and Definition \ref{defn:truify}}
          \end{pfsteps*}
          
        \restorelocalsteps{2} 
        \item[\text{(\ref{rule:CMSNotVal})}] 
          \begin{pfsteps*}
          \item $\isntVal{e}$ \BY{assumption} \pflabel{notval}
          \end{pfsteps*}
          By rule induction over \rulesref{rules:notval} on \pfref{notval}, for each case,
          by rule induction over Rules (\ref{rules:Value}) on \pfref{val}, no case applies due to syntactic contradiction.
          
        \restorelocalsteps{2} 
        \item[\text{(\ref{rule:CMSAnd1})}] 
          \begin{pfsteps*}
          \item $\xi=\cand{\xi_1}{\xi_2}$ \BY{assumption}
          \item $\cmaysatisfy{e}{\xi_1}$ \BY{assumption} \pflabel{[and1]maysat1}
          \item $\csatisfy{e}{\xi_2}$ \BY{assumption} \pflabel{[and1]satisfy2}
          \item $\ctruify{\xi}=\cand{\ctruify{\xi_1}}{\ctruify{\xi_2}}$ \BY{\autoref{defn:truify}}
          \item $\csatisfyormay{e}{\xi_1}$ \BY{\ruleref{rule:CSMSMay} on \pfref{[and1]maysat1}} \pflabel{[and1]satormay1}
          \item $\csatisfyormay{e}{\xi_2}$ \BY{\ruleref{rule:CSMSSat} on \pfref{[and1]satisfy2}} \pflabel{[and1]satormay2}
          \item $\csatisfy{e}{\ctruify{\xi_1}}$ \BY{IH on \pfref{[and1]satormay1}} \pflabel{[and1]sat-truify1}
          \item $\csatisfy{e}{\ctruify{\xi_2}}$ \BY{IH on \pfref{[and1]satormay2}} \pflabel{[and1]sat-truify2}
          \item $\csatisfy{e}{\cand{\ctruify{\xi_1}}{\ctruify{\xi_2}}}$ \BY{\ruleref{rule:CSAnd} on \pfref{[and1]sat-truify1} and \pfref{[and1]sat-truify2}}
          \end{pfsteps*}
        
        \restorelocalsteps{2} 
        \item[\text{(\ref{rule:CMSAnd2})}] 
          \begin{pfsteps*}
          \item $\xi=\cand{\xi_1}{\xi_2}$ \BY{assumption}
          \item $\csatisfy{e}{\xi_1}$ \BY{assumption} \pflabel{[and2]satisfy1}
          \item $\cmaysatisfy{e}{\xi_2}$ \BY{assumption} \pflabel{[and2]maysat2}
          \item $\ctruify{\xi}=\cand{\ctruify{\xi_1}}{\ctruify{\xi_2}}$ \BY{\autoref{defn:truify}}
          \item $\csatisfyormay{e}{\xi_1}$ \BY{\ruleref{rule:CSMSSat} on \pfref{[and2]satisfy1}} \pflabel{[and2]satormay1}
          \item $\csatisfyormay{e}{\xi_2}$ \BY{\ruleref{rule:CSMSMay} on \pfref{[and2]maysat2}} \pflabel{[and2]satormay2}
          \item $\csatisfy{e}{\ctruify{\xi_1}}$ \BY{IH on \pfref{[and2]satormay1}} \pflabel{[and2]sat-truify1}
          \item $\csatisfy{e}{\ctruify{\xi_2}}$ \BY{IH on \pfref{[and2]satormay2}} \pflabel{[and2]sat-truify2}
          \item $\csatisfy{e}{\cand{\ctruify{\xi_1}}{\ctruify{\xi_2}}}$ \BY{\ruleref{rule:CSAnd} on \pfref{[and2]sat-truify1} and \pfref{[and2]sat-truify2}}
          \end{pfsteps*}
        
        \restorelocalsteps{2} 
        \item[\text{(\ref{rule:CMSAnd3})}] 
          \begin{pfsteps*}
          \item $\xi=\cand{\xi_1}{\xi_2}$ \BY{assumption}
          \item $\cmaysatisfy{e}{\xi_1}$ \BY{assumption} \pflabel{[and3]maysat1}
          \item $\cmaysatisfy{e}{\xi_2}$ \BY{assumption} \pflabel{[and3]maysat2}
          \item $\ctruify{\xi}=\cand{\ctruify{\xi_1}}{\ctruify{\xi_2}}$ \BY{\autoref{defn:truify}}
          \item $\csatisfyormay{e}{\xi_1}$ \BY{\ruleref{rule:CSMSMay} on \pfref{[and3]maysat1}} \pflabel{[and3]satormay1}
          \item $\csatisfyormay{e}{\xi_2}$ \BY{\ruleref{rule:CSMSMay} on \pfref{[and3]maysat2}} \pflabel{[and3]satormay2}
          \item $\csatisfy{e}{\ctruify{\xi_1}}$ \BY{IH on \pfref{[and3]satormay1}} \pflabel{[and3]sat-truify1}
          \item $\csatisfy{e}{\ctruify{\xi_2}}$ \BY{IH on \pfref{[and3]satormay2}} \pflabel{[and3]sat-truify2}
          \item $\csatisfy{e}{\cand{\ctruify{\xi_1}}{\ctruify{\xi_2}}}$ \BY{\ruleref{rule:CSAnd} on \pfref{[and3]sat-truify1} and \pfref{[and3]sat-truify2}}
          \end{pfsteps*}
        
        \restorelocalsteps{2} 
        \item[\text{(\ref{rule:CMSOr1})}] 
          \begin{pfsteps*}
          \item $\xi=\cor{\xi_1}{\xi_2}$ \BY{assumption}
          \item $\cmaysatisfy{e}{\xi_1}$ \BY{assumption} \pflabel{[or1]maysat}
          \item $\ctruify{\xi}=\cor{\ctruify{\xi_1}}{\ctruify{\xi_2}}$ \BY{\autoref{defn:truify}}
          \item $\csatisfyormay{e}{\xi_1}$ \BY{\ruleref{rule:CSMSMay} on \pfref{[or1]maysat}} \pflabel{[or1]satormay}
          \item $\csatisfy{e}{\ctruify{\xi_1}}$ \BY{IH on \pfref{[or1]satormay}} \pflabel{[or1]sat-truify}
          \item $\csatisfy{e}{\cor{\ctruify{\xi_1}}{\ctruify{\xi_2}}}$ \BY{\ruleref{rule:CSOr1} on \pfref{[or1]sat-truify}}
          \end{pfsteps*}
        
        \restorelocalsteps{2} 
        \item[\text{(\ref{rule:CMSOr2})}] 
          \begin{pfsteps*}
          \item $\xi=\cor{\xi_1}{\xi_2}$ \BY{assumption}
          \item $\cmaysatisfy{e}{\xi_2}$ \BY{assumption} \pflabel{[or2]maysat}
          \item $\ctruify{\xi}=\cor{\ctruify{\xi_1}}{\ctruify{\xi_2}}$ \BY{\autoref{defn:truify}}
          \item $\csatisfyormay{e}{\xi_2}$ \BY{\ruleref{rule:CSMSMay} on \pfref{[or2]maysat}} \pflabel{[or2]satormay}
          \item $\csatisfy{e}{\ctruify{\xi_2}}$ \BY{IH on \pfref{[or2]satormay}} \pflabel{[or2]sat-truify}
          \item $\csatisfy{e}{\cor{\ctruify{\xi_1}}{\ctruify{\xi_2}}}$ \BY{\ruleref{rule:CSOr2} on \pfref{[or2]sat-truify}}
          \end{pfsteps*}
        
        \restorelocalsteps{2} 
        \item[\text{(\ref{rule:CMSInl})}] 
          \begin{pfsteps*}
          \item $\xi=\cinl{\xi_1}$ \BY{assumption}
          \item $\cmaysatisfy{e}{\xi_1}$ \BY{assumption} \pflabel{[inl1]maysat}
          \item $\ctruify{\xi}=\cinl{\ctruify{\xi_1}}$ \BY{\autoref{defn:truify}}
          \item $\csatisfyormay{e}{\xi_1}$ \BY{\ruleref{rule:CSMSMay} on \pfref{[inl1]maysat}} \pflabel{[inl1]satormay}
          \item $\csatisfy{e}{\ctruify{\xi_1}}$ \BY{IH on \pfref{[inl1]satormay}} \pflabel{[inl1]sat-truify}
          \item $\csatisfy{e}{\cinl{\ctruify{\xi_1}}}$ \BY{\ruleref{rule:CSInl} on \pfref{[inl1]sat-truify}}
          \end{pfsteps*}
        
        \restorelocalsteps{2} 
        \item[\text{(\ref{rule:CMSInr})}] 
          \begin{pfsteps*}
          \item $\xi=\cinr{\xi_2}$ \BY{assumption}
          \item $\cmaysatisfy{e}{\xi_2}$ \BY{assumption} \pflabel{[inr]maysat}
          \item $\ctruify{\xi}=\cinr{\ctruify{\xi_2}}$ \BY{\autoref{defn:truify}}
          \item $\csatisfyormay{e}{\xi_2}$ \BY{\ruleref{rule:CSMSMay} on \pfref{[inr]maysat}} \pflabel{[inr]satormay}
          \item $\csatisfy{e}{\ctruify{\xi_2}}$ \BY{IH on \pfref{[inr]satormay}} \pflabel{[inr]sat-truify}
          \item $\csatisfy{e}{\cinr{\ctruify{\xi_2}}}$ \BY{\ruleref{rule:CSInr} on \pfref{[inr]sat-truify}}
          \end{pfsteps*}
        
        \restorelocalsteps{2} 
        \item[\text{(\ref{rule:CMSPair1})}] 
          \begin{pfsteps*}
          \item $e=\hpair{e_1}{e_2}$ \BY{assumption}
          \item $\xi=\cpair{\xi_1}{\xi_2}$ \BY{assumption}
          \item $\cmaysatisfy{e_1}{\xi_1}$ \BY{assumption} \pflabel{[pair1]maysat1}
          \item $\csatisfy{e_2}{\xi_2}$ \BY{assumption} \pflabel{[pair1]satisfy2}
          \item $\ctruify{\xi}=\cpair{\ctruify{\xi_1}}{\ctruify{\xi_2}}$ \BY{\autoref{defn:truify}}
          \item $\csatisfyormay{e_1}{\xi_1}$ \BY{\ruleref{rule:CSMSMay} on \pfref{[pair1]maysat1}} \pflabel{[pair1]satormay1}
          \item $\csatisfyormay{e_2}{\xi_2}$ \BY{\ruleref{rule:CSMSSat} on \pfref{[pair1]satisfy2}} \pflabel{[pair1]satormay2}
          \item $\csatisfy{e_1}{\ctruify{\xi_1}}$ \BY{IH on \pfref{[pair1]satormay1}} \pflabel{[pair1]sat-truify1}
          \item $\csatisfy{e_2}{\ctruify{\xi_2}}$ \BY{IH on \pfref{[pair1]satormay2}} \pflabel{[pair1]sat-truify2}
          \item $\csatisfy{\hpair{e_1}{e_2}}{\cpair{\ctruify{\xi_1}}{\ctruify{\xi_2}}}$ \BY{\ruleref{rule:CSPair} on \pfref{[pair1]sat-truify1} and \pfref{[pair1]sat-truify2}}
          \end{pfsteps*}
        
        \restorelocalsteps{2} 
        \item[\text{(\ref{rule:CMSPair2})}] 
          \begin{pfsteps*}
          \item $e=\hpair{e_1}{e_2}$ \BY{assumption}
          \item $\xi=\cpair{\xi_1}{\xi_2}$ \BY{assumption}
          \item $\csatisfy{e_1}{\xi_1}$ \BY{assumption} \pflabel{[pair2]satisfy1}
          \item $\cmaysatisfy{e_2}{\xi_2}$ \BY{assumption} \pflabel{[pair2]maysat2}
          \item $\ctruify{\xi}=\cpair{\ctruify{\xi_1}}{\ctruify{\xi_2}}$ \BY{\autoref{defn:truify}}
          \item $\csatisfyormay{e_1}{\xi_1}$ \BY{\ruleref{rule:CSMSSat} on \pfref{[pair2]satisfy1}} \pflabel{[pair2]satormay1}
          \item $\csatisfyormay{e_2}{\xi_2}$ \BY{\ruleref{rule:CSMSMay} on \pfref{[pair2]maysat2}} \pflabel{[pair2]satormay2}
          \item $\csatisfy{e_1}{\ctruify{\xi_1}}$ \BY{IH on \pfref{[pair2]satormay1}} \pflabel{[pair2]sat-truify1}
          \item $\csatisfy{e_2}{\ctruify{\xi_2}}$ \BY{IH on \pfref{[pair2]satormay2}} \pflabel{[pair2]sat-truify2}
          \item $\csatisfy{\hpair{e_1}{e_2}}{\cpair{\ctruify{\xi_1}}{\ctruify{\xi_2}}}$ \BY{\ruleref{rule:CSPair} on \pfref{[pair2]sat-truify1} and \pfref{[pair2]sat-truify2}}
          \end{pfsteps*}
        
        \restorelocalsteps{2} 
        \item[\text{(\ref{rule:CMSPair3})}] 
          \begin{pfsteps*}
          \item $e=\hpair{e_1}{e_2}$ \BY{assumption}
          \item $\xi=\cpair{\xi_1}{\xi_2}$ \BY{assumption}
          \item $\cmaysatisfy{e_1}{\xi_1}$ \BY{assumption} \pflabel{[pair3]maysat1}
          \item $\cmaysatisfy{e_2}{\xi_2}$ \BY{assumption} \pflabel{[pair3]maysat2}
          \item $\ctruify{\xi}=\cpair{\ctruify{\xi_1}}{\ctruify{\xi_2}}$ \BY{\autoref{defn:truify}}
          \item $\csatisfyormay{e_1}{\xi_1}$ \BY{\ruleref{rule:CSMSMay} on \pfref{[pair3]maysat1}} \pflabel{[pair3]satormay1}
          \item $\csatisfyormay{e_2}{\xi_2}$ \BY{\ruleref{rule:CSMSMay} on \pfref{[pair3]maysat2}} \pflabel{[pair3]satormay2}
          \item $\csatisfy{e_1}{\ctruify{\xi_1}}$ \BY{IH on \pfref{[pair3]satormay1}} \pflabel{[pair3]sat-truify1}
          \item $\csatisfy{e_2}{\ctruify{\xi_2}}$ \BY{IH on \pfref{[pair3]satormay2}} \pflabel{[pair3]sat-truify2}
          \item $\csatisfy{\hpair{e_1}{e_2}}{\cpair{\ctruify{\xi_1}}{\ctruify{\xi_2}}}$ \BY{\ruleref{rule:CSPair} on \pfref{[pair3]sat-truify1} and \pfref{[pair3]sat-truify2}}
          \end{pfsteps*}
        
        \end{byCases}
      \end{byCases}
    \restorelocalsteps{0}
    \item Necessity:
    \begin{pfsteps*}
    \item $\csatisfy{e}{\ctruify{\xi}}$ \BY{assumption} \pflabel{satisfy-truify}
    \end{pfsteps*}
    By structural induction on $\xi$.
    \begin{byCases}

      \savelocalsteps{1}
      \item[\xi=\ctruth, \cfalsity, \cnum{n}, \cnotnum{n}]
        \begin{pfsteps*}
        \item $\xi=\ctruify{\xi}$ \BY{\autoref{defn:truify}}
        \item $\csatisfyormay{e}{\xi}$ \BY{\ruleref{rule:CSMSSat} on \pfref{satisfy-truify}}
        \end{pfsteps*}

      \restorelocalsteps{1}
      \item[\xi=\cunknown]
        \begin{pfsteps*}
        \item $\cmaysatisfy{e}{\cunknown}$ \BY{\ruleref{rule:CMSUnknown}} \pflabel{[unknown]maysat}
        \item $\csatisfyormay{e}{\cunknown}$ \BY{\ruleref{rule:CSMSMay} on \pfref{[unknown]maysat}}
        \end{pfsteps*}

      \restorelocalsteps{1}
      \item[\xi=\cand{\xi_1}{\xi_2}]
        \begin{pfsteps*}
        \item $\ctruify{\xi}=\cand{\ctruify{\xi_1}}{\ctruify{\xi_2}}$ \BY{\autoref{defn:truify}}
        \end{pfsteps*}
        By rule induction over Rules (\ref{rules:Satisfy}) on \pfref{satisfy-truify}, only one case applies.
        \begin{byCases}
          \item[\text{(\ref{rule:CSAnd})}]
            \begin{pfsteps*}
            \item $\csatisfy{e}{\ctruify{\xi_1}}$ \BY{assumption} \pflabel{[and]satisfy-falsify1}
            \item $\csatisfy{e}{\ctruify{\xi_2}}$ \BY{assumption} \pflabel{[and]satisfy-falsify2}
            \item $\csatisfyormay{e}{\xi_1}$ \BY{IH on \pfref{[and]satisfy-falsify1}} \pflabel{[and]satisfy1}
            \item $\csatisfyormay{e}{\xi_2}$ \BY{IH on \pfref{[and]satisfy-falsify2}} \pflabel{[and]satisfy2}
            \item $\csatisfy{e}{\cand{\xi_1}{\xi_2}}$ \BY{\autoref{lem:satormay-and} on \pfref{[and]satisfy1} and \pfref{[and]satisfy2}}
            \end{pfsteps*}
        \end{byCases}

      \restorelocalsteps{1}
      \item[\xi = \cor{\xi_1}{\xi_2}]
        \begin{pfsteps*}
        \item $\ctruify{\xi}=\cor{\ctruify{\xi_1}}{\ctruify{\xi_2}}$ \BY{\autoref{defn:truify}}
        \end{pfsteps*}
        By rule induction over Rules (\ref{rules:Satisfy}) on \pfref{satisfy-truify} and only two cases apply.
        \begin{byCases}

          \savelocalsteps{2}
          \item[\text{(\ref{rule:CSOr1})}]
            \begin{pfsteps*}
            \item $\csatisfy{e}{\ctruify{\xi_1}}$ \BY{assumption} \pflabel{[or]satisfy-falsify1}
            \item $\csatisfyormay{e}{\xi_1}$ \BY{IH on \pfref{[or]satisfy-falsify1}} \pflabel{[or]satisfy1}
            \item $\csatisfyormay{e}{\cor{\xi_1}{\xi_2}}$ \BY{\autoref{lem:satormay-or} on \pfref{[or]satisfy1}}
            \end{pfsteps*}

          \restorelocalsteps{2}
          \item[\text{(\ref{rule:CSOr2})}]
            \begin{pfsteps*}
            \item $\csatisfy{e}{\ctruify{\xi_2}}$ \BY{assumption} \pflabel{[or]satisfy-falsify2}
            \item $\csatisfyormay{e}{\xi_2}$ \BY{IH on \pfref{[or]satisfy-falsify2}} \pflabel{[or]satisfy2}
            \item $\csatisfyormay{e}{\cor{\xi_1}{\xi_2}}$ \BY{\autoref{lem:satormay-or} on \pfref{[or]satisfy2}}
            \end{pfsteps*}
        \end{byCases}
      
      \restorelocalsteps{1}
      \item[\xi=\cinl{\xi_1}]
        \begin{pfsteps*}
        \item $\ctruify{\xi}=\cinl{\ctruify{\xi_1}}$ \BY{\autoref{defn:truify}}
        \end{pfsteps*}
        By rule induction over Rules (\ref{rules:Satisfy}) on \pfref{satisfy-truify} and only one case applies.
        \begin{byCases}
          \item[\text{(\ref{rule:CSInl})}]
            \begin{pfsteps*}
            \item $e = \hinl{\tau_2}{e_1}$ \BY{assumption}
            \item $\csatisfy{e_1}{\ctruify{\xi_1}}$ \BY{assumption} \pflabel{[inl]satisfy-falsify1}
            \item $\csatisfyormay{e_1}{\xi_1}$ \BY{IH on \pfref{[inl]satisfy-falsify1}} \pflabel{[inl]satisfy1}
            \item $\csatisfyormay{\hinl{\tau_2}{e_1}}{\cinl{\xi_1}}$ \BY{\autoref{lem:satormay-inl} on \pfref{[inl]satisfy1}}
            \end{pfsteps*} 
        \end{byCases}

      \restorelocalsteps{1}
      \item[\xi=\cinr{\xi_2}]
        \begin{pfsteps*}
        \item $\ctruify{\xi}=\cinr{\ctruify{\xi_2}}$ \BY{\autoref{defn:truify}}
        \end{pfsteps*}
        By rule induction over Rules (\ref{rules:Satisfy}) on \pfref{satisfy-truify} and only one case applies.
        \begin{byCases}
          \item[\text{(\ref{rule:CSInr})}]
            \begin{pfsteps*}
            \item $e = \hinr{\tau_1}{e_2}$ \BY{assumption}
            \item $\csatisfy{e_2}{\ctruify{\xi_2}}$ \BY{assumption} \pflabel{[inr]satisfy-falsify2}
            \item $\csatisfyormay{e_2}{\xi_2}$ \BY{IH on \pfref{[inr]satisfy-falsify2}} \pflabel{[inr]satisfy2}
            \item $\csatisfyormay{\hinr{\tau_1}{e_2}}{\cinr{\xi_2}}$ \BY{\autoref{lem:satormay-inr} on \pfref{[inr]satisfy2}}
            \end{pfsteps*} 
        \end{byCases}
      
      \restorelocalsteps{1}
      \item[\xi=\cpair{\xi_1}{\xi_2}]
        \begin{pfsteps*}
        \item $\ctruify{\xi}=\cpair{\ctruify{\xi_1}}{\ctruify{\xi_2}}$ \BY{\autoref{defn:truify}}
        \end{pfsteps*}
        By rule induction over Rules (\ref{rules:Satisfy}) on \pfref{satisfy-truify} and only one case applies.
        \begin{byCases}
          \item[\text{(\ref{rule:CSPair})}]
            \begin{pfsteps*}
            \item $e=\hpair{e_1}{e_2}$ \BY{assumption}
            \item $\csatisfy{e_1}{\cfalsify{\xi_1}}$ \BY{assumption} \pflabel{[pair]satisfy-falsify1}
            \item $\csatisfy{e_2}{\cfalsify{\xi_2}}$ \BY{assumption} \pflabel{[pair]satisfy-falsify2}
            \item $\csatisfyormay{e_1}{\xi_1}$ \BY{IH on \pfref{[pair]satisfy-falsify1}} \pflabel{[pair]satisfy1}
            \item $\csatisfyormay{e_2}{\xi_2}$ \BY{IH on \pfref{[pair]satisfy-falsify2}} \pflabel{[pair]satisfy2}
            \item $\csatisfyormay{\hpair{e_1}{e_2}}{\cpair{\xi_1}{\xi_2}}$ \BY{\autoref{lem:satormay-pair} on \pfref{[pair]satisfy1} and \pfref{[pair]satisfy2}}
            \end{pfsteps*}
        \end{byCases}
    \end{byCases}
  \end{enumerate}
  \resetpfcounter
\end{proof}

\begin{lemma}
  \label{lem:satisfy-falsify}
  $\csatisfy{e}{\xi}$ iff $\csatisfy{e}{\cfalsify{\xi}}$
\end{lemma}
\begin{proof}
  We prove sufficiency and necessity separately.
  \begin{enumerate}
    \item Sufficiency:
    \begin{pfsteps*}
    \item $\csatisfy{e}{\xi}$ \BY{assumption} \pflabel{satisfy}
    \end{pfsteps*}
    By rule induction over Rules (\ref{rules:Satisfy}) on \pfref{satisfy}.
    \begin{byCases}
      
      \savelocalsteps{1}
      \item[\text{(\ref{rule:CSTruth})}]
        \begin{pfsteps*}
        \item $\xi = \ctruth$ \BY{assumption}
        \item $\csatisfy{e}{\cfalsify{\ctruth}}$ \BY{\pfref{satisfy} and Definition \ref{defn:falsify}}
        \end{pfsteps*}

      \restorelocalsteps{1}
      \item[\text{(\ref{rule:CSNum})}]
        \begin{pfsteps*}
        \item $\xi = \cnum{n}$ \BY{assumption}
        \item $\csatisfy{e}{\cfalsify{\cnum{n}}}$ \BY{\pfref{satisfy} and Definition \ref{defn:falsify}}
        \end{pfsteps*}

      \restorelocalsteps{1}
      \item[\text{(\ref{rule:CSNotNum})}]
        \begin{pfsteps*}
        \item $\xi = \cnotnum{n}$ \BY{assumption}
        \item $\csatisfy{e}{\cfalsify{\cnotnum{n}}}$ \BY{\pfref{satisfy} and Definition \ref{defn:falsify}}
        \end{pfsteps*}

      \restorelocalsteps{1}
      \item[\text{(\ref{rule:CSAnd})}]
        \begin{pfsteps*}
        \item $\xi = \cand{\xi_1}{\xi_2}$ \BY{assumption}
        \item $\csatisfy{e}{\xi_1}$ \BY{assumption} \pflabel{[and]satisfy1}
        \item $\csatisfy{e}{\xi_2}$ \BY{assumption} \pflabel{[and]satisfy2}
        \item $\csatisfy{e}{\cfalsify{\xi_1}}$ \BY{IH on \pfref{[and]satisfy1}} \pflabel{[and]satisfy-falsify1}
        \item $\csatisfy{e}{\cfalsify{\xi_2}}$ \BY{IH on \pfref{[and]satisfy2}} \pflabel{[and]satisfy-falsify2}
        \item $\csatisfy{e}{\cand{\cfalsify{\xi_1}}{\cfalsify{\xi_2}}}$ \BY{Rule (\ref{rule:CSAnd}) on \pfref{[and]satisfy-falsify1} and \pfref{[and]satisfy-falsify2}} \pflabel{[and]satisfy-and-falsify}
        \item $\csatisfy{e}{\cfalsify{\cand{\xi_1}{\xi_2}}}$ \BY{\pfref{[and]satisfy-and-falsify} and Definition \ref{defn:falsify}}
        \end{pfsteps*}

      \restorelocalsteps{1}
      \item[\text{(\ref{rule:CSOr1})}]
        \begin{pfsteps*}
        \item $\xi = \cor{\xi_1}{\xi_2}$ \BY{assumption}
        \item $\csatisfy{e}{\xi_1}$ \BY{assumption} \pflabel{[or]satisfy1}
        \item $\csatisfy{e}{\cfalsify{\xi_1}}$ \BY{IH on \pfref{[or]satisfy1}} \pflabel{[or]satisfy-falsify1}
        \item $\csatisfy{e}{\cor{\cfalsify{\xi_1}}{\cfalsify{\xi_2}}}$ \BY{Rule (\ref{rule:CSOr1}) on \pfref{[or]satisfy-falsify1}} \pflabel{[or]satisfy-or-falsify}
        \item $\csatisfy{e}{\cfalsify{\cor{\xi_1}{\xi_2}}}$ \BY{\pfref{[or]satisfy-or-falsify} and Definition \ref{defn:falsify}}
        \end{pfsteps*}
      
      \restorelocalsteps{1}
      \item[\text{(\ref{rule:CSOr2})}]
        \begin{pfsteps*}
        \item $\xi = \cor{\xi_1}{\xi_2}$ \BY{assumption}
        \item $\csatisfy{e}{\xi_2}$ \BY{assumption} \pflabel{[or]satisfy2}
        \item $\csatisfy{e}{\cfalsify{\xi_2}}$ \BY{IH on \pfref{[or]satisfy2}} \pflabel{[or]satisfy-falsify2}
        \item $\csatisfy{e}{\cor{\cfalsify{\xi_1}}{\cfalsify{\xi_2}}}$ \BY{Rule (\ref{rule:CSOr2}) on \pfref{[or]satisfy-falsify2}} \pflabel{[or]satisfy-or-falsify'}
        \item $\csatisfy{e}{\cfalsify{\cor{\xi_1}{\xi_2}}}$ \BY{\pfref{[or]satisfy-or-falsify'} and Definition \ref{defn:falsify}}
        \end{pfsteps*}

      \restorelocalsteps{1}
      \item[\text{(\ref{rule:CSInl})}]
        \begin{pfsteps*}
        \item $e = \hinl{\tau_2}{e_1}$ \BY{assumption}
        \item $\xi = \cinl{\xi_1}$ \BY{assumption}
        \item $\csatisfy{e_1}{\xi_1}$ \BY{assumption} \pflabel{[inl]satisfy1}
        \item $\csatisfy{e_1}{\cfalsify{\xi_1}}$ \BY{IH on \pfref{[inl]satisfy1}} \pflabel{[inl]satisfy-falsify1}
        \item $\csatisfy{\hinl{\tau_2}{e_1}}{\cinl{\cfalsify{\xi_1}}}$ \BY{Rule (\ref{rule:CSInl}) on \pfref{[inl]satisfy-falsify1}} \pflabel{[inl]satisfy-inl-falsify}
        \item $\csatisfy{\hinl{\tau_2}{e_1}}{\cfalsify{\cinl{\xi_1}}}$ \BY{\pfref{[inl]satisfy-inl-falsify} and Definition \ref{defn:falsify}}
        \end{pfsteps*}

      \restorelocalsteps{1}
      \item[\text{(\ref{rule:CSInr})}]
        \begin{pfsteps*}
        \item $e = \hinr{\tau_1}{e_2}$ \BY{assumption}
        \item $\xi = \cinr{\xi_2}$ \BY{assumption}
        \item $\csatisfy{e_2}{\xi_2}$ \BY{assumption} \pflabel{[inr]satisfy2}
        \item $\csatisfy{e_2}{\cfalsify{\xi_2}}$ \BY{IH on \pfref{[inr]satisfy2}} \pflabel{[inr]satisfy-falsify2}
        \item $\csatisfy{\hinr{\tau_1}{e_2}}{\cinr{\cfalsify{\xi_2}}}$ \BY{Rule (\ref{rule:CSInr}) on \pfref{[inr]satisfy-falsify2}} \pflabel{[inr]satisfy-inr-falsify}
        \item $\csatisfy{\hinr{\tau_1}{e_2}}{\cfalsify{\cinr{\xi_2}}}$ \BY{\pfref{[inr]satisfy-inr-falsify} and Definition \ref{defn:falsify}}
        \end{pfsteps*}
      
      \restorelocalsteps{1}
      \item[\text{(\ref{rule:CSPair})}]
        \begin{pfsteps*}
        \item $e = \hpair{e_1}{e_2}$ \BY{assumption}
        \item $\xi = \cpair{\xi_1}{\xi_2}$ \BY{assumption}
        \item $\csatisfy{e_1}{\xi_1}$ \BY{assumption} \pflabel{[pair]satisfy1}
        \item $\csatisfy{e_2}{\xi_2}$ \BY{assumption} \pflabel{[pair]satisfy2}
        \item $\csatisfy{e_1}{\cfalsify{\xi_1}}$ \BY{IH on \pfref{[pair]satisfy1}} \pflabel{[pair]satisfy-falsify1}
        \item $\csatisfy{e_2}{\cfalsify{\xi_2}}$ \BY{IH on \pfref{[pair]satisfy2}} \pflabel{[pair]satisfy-falsify2}
        \item $\csatisfy{\hpair{e_1}{e_2}}{\cpair{\cfalsify{\xi_1}}{\cfalsify{\xi_2}}}$ \BY{Rule (\ref{rule:CSPair}) on \pfref{[pair]satisfy-falsify1} and \pfref{[pair]satisfy-falsify2}} \pflabel{[pair]satisfy-pair-falsify}
        \item $\csatisfy{\hpair{e_1}{e_2}}{\cfalsify{\cpair{\xi_1}{\xi_2}}}$ \BY{\pfref{[pair]satisfy-pair-falsify} and Definition \ref{defn:falsify}}
        \end{pfsteps*}
    \end{byCases}

    \resetpfcounter

    \item Necessity:
    \begin{pfsteps*}
    \item $\csatisfy{e}{\cfalsify{\xi}}$ \BY{assumption} \pflabel{satisfy-falsify}
    \end{pfsteps*}
    By structural induction on $\xi$.
    \begin{byCases}

      \savelocalsteps{1}
      \item[\xi=\ctruth, \cfalsity, \cnum{n}, \cnotnum{n}]
        \begin{pfsteps*}
        \item $\csatisfy{e}{\xi}$ \BY{\pfref{satisfy-falsify} and Definition \ref{defn:falsify}}
        \end{pfsteps*}

      \restorelocalsteps{1}
      \item[\xi=\cunknown]
        \begin{pfsteps*}
        \item $\csatisfy{e}{\cfalsity}$ \BY{\pfref{satisfy-falsify} and Definition \ref{defn:falsify}} \pflabel{satisfy-falsity}
        \item $\cnotsatisfy{e}{\cfalsity}$ \BY{Lemma \ref{lem:no-e-satisfy-falsity}} \pflabel{no-satisfy-falsity}
        \end{pfsteps*}
        \pfref{no-satisfy-falsity} contradicts \pfref{satisfy-falsity}.

      \restorelocalsteps{1}
      \item[\xi=\cand{\xi_1}{\xi_2}]
        \begin{pfsteps*}
        \item $\csatisfy{e}{\cand{\cfalsify{\xi_1}}{\cfalsify{\xi_2}}}$ \BY{\pfref{satisfy-falsify} and Definition \ref{defn:falsify}} \pflabel{satisfy-and-falsify}
        \end{pfsteps*}
        By rule induction over Rules (\ref{rules:Satisfy}) on \pfref{satisfy-and-falsify} and only case applies.
        \begin{byCases}
          \item[\text{(\ref{rule:CSAnd})}]
            \begin{pfsteps*}
            \item $\csatisfy{e}{\cfalsify{\xi_1}}$ \BY{assumption} \pflabel{[and]satisfy-falsify1}
            \item $\csatisfy{e}{\cfalsify{\xi_2}}$ \BY{assumption} \pflabel{[and]satisfy-falsify2}
            \item $\csatisfy{e}{\xi_1}$ \BY{IH on \pfref{[and]satisfy-falsify1}} \pflabel{[and]satisfy1}
            \item $\csatisfy{e}{\xi_2}$ \BY{IH on \pfref{[and]satisfy-falsify2}} \pflabel{[and]satisfy2}
            \item $\csatisfy{e}{\cand{\xi_1}{\xi_2}}$ \BY{Rule (\ref{rule:CSAnd}) on \pfref{[and]satisfy1} and \pfref{[and]satisfy2}}
            \end{pfsteps*}
        \end{byCases}

      \restorelocalsteps{1}
      \item[\xi = \cor{\xi_1}{\xi_2}]
        \begin{pfsteps*}
        \item $\csatisfy{e}{\cor{\cfalsify{\xi_1}}{\cfalsify{\xi_2}}}$ \BY{\pfref{satisfy-falsify} and Definition \ref{defn:falsify}} \pflabel{satisfy-or-falsify}
        \end{pfsteps*}
        By rule induction over Rules (\ref{rules:Satisfy}) on \pfref{satisfy-or-falsify} and only two cases apply.
        \begin{byCases}

          \savelocalsteps{2}
          \item[\text{(\ref{rule:CSOr1})}]
            \begin{pfsteps*}
            \item $\csatisfy{e}{\cfalsify{\xi_1}}$ \BY{assumption} \pflabel{[or]satisfy-falsify1}
            \item $\csatisfy{e}{\xi_1}$ \BY{IH on \pfref{[or]satisfy-falsify1}} \pflabel{[or]satisfy1}
            \item $\csatisfy{e}{\cor{\xi_1}{\xi_2}}$ \BY{Rule (\ref{rule:CSOr1}) on \pfref{[or]satisfy1}}
            \end{pfsteps*}

          \restorelocalsteps{2}
          \item[\text{(\ref{rule:CSOr2})}]
            \begin{pfsteps*}
            \item $\csatisfy{e}{\cfalsify{\xi_2}}$ \BY{assumption} \pflabel{[or]satisfy-falsify2}
            \item $\csatisfy{e}{\xi_2}$ \BY{IH on \pfref{[or]satisfy-falsify2}} \pflabel{[or]satisfy2}
            \item $\csatisfy{e}{\cor{\xi_1}{\xi_2}}$ \BY{Rule (\ref{rule:CSOr2}) on \pfref{[or]satisfy2}}
            \end{pfsteps*}
        \end{byCases}
      
      \restorelocalsteps{1}
      \item[\xi=\cinl{\xi_1}]
        \begin{pfsteps*}
        \item $\csatisfy{e}{\cinl{\cfalsify{\xi_1}}}$ \BY{\pfref{satisfy-falsify} and Definition \ref{defn:falsify}} \pflabel{satisfy-inl-falsify}
        \end{pfsteps*}
        By rule induction over Rules (\ref{rules:Satisfy}) on \pfref{satisfy-inl-falsify} and only one case applies.
        \begin{byCases}
          \item[\text{(\ref{rule:CSInl})}]
            \begin{pfsteps*}
            \item $e = \hinl{\tau_2}{e_1}$ \BY{assumption}
            \item $\csatisfy{e_1}{\cfalsify{\xi_1}}$ \BY{assumption} \pflabel{[inl]satisfy-falsify1}
            \item $\csatisfy{e_1}{\xi_1}$ \BY{IH on \pfref{[inl]satisfy-falsify1}} \pflabel{[inl]satisfy1}
            \item $\csatisfy{e}{\cinl{\xi_1}}$ \BY{Rule (\ref{rule:CSInl}) on \pfref{[inl]satisfy1}}
            \end{pfsteps*} 
        \end{byCases}

      \restorelocalsteps{1}
      \item[\xi=\cinr{\xi_2}]
        \begin{pfsteps*}
        \item $\csatisfy{e}{\cinr{\cfalsify{\xi_2}}}$ \BY        {\pfref{satisfy-falsify} and Definition \ref{defn:falsify}} \pflabel{satisfy-inr-falsify}
        \end{pfsteps*}
        By rule induction over Rules (\ref{rules:Satisfy}) on \pfref{satisfy-inr-falsify} and only one case applies.
        \begin{byCases}
          \item[\text{(\ref{rule:CSInr})}]
            \begin{pfsteps*}
            \item $e = \hinr{\tau_1}{e_2}$ \BY{assumption}
            \item $\csatisfy{e_2}{\cfalsify{\xi_2}}$ \BY{assumption} \pflabel{[inr]satisfy-falsify2}
            \item $\csatisfy{e_2}{\xi_2}$ \BY{IH on \pfref{[inr]satisfy-falsify2}} \pflabel{[inr]satisfy2}
            \item $\csatisfy{e}{\cinr{\xi_2}}$ \BY{Rule (\ref{rule:CSInr}) on \pfref{[inr]satisfy2}}
            \end{pfsteps*} 
        \end{byCases}
      
      \restorelocalsteps{1}
      \item[\xi=\cpair{\xi_1}{\xi_2}]
        \begin{pfsteps*}
        \item $\csatisfy{e}{\cpair{\cfalsify{\xi_1}}{\cfalsify{\xi_2}}}$ \BY{\pfref{satisfy-falsify} and Definition \ref{defn:falsify}} \pflabel{satisfy-pair-falsify}
        \end{pfsteps*}
        By rule induction over Rules (\ref{rules:Satisfy}) on \pfref{satisfy-pair-falsify} and only case applies.
        \begin{byCases}
          \item[\text{(\ref{rule:CSPair})}]
            \begin{pfsteps*}
            \item $e=\hpair{e_1}{e_2}$ \BY{assumption}
            \item $\csatisfy{e_1}{\cfalsify{\xi_1}}$ \BY{assumption} \pflabel{[pair]satisfy-falsify1}
            \item $\csatisfy{e_2}{\cfalsify{\xi_2}}$ \BY{assumption} \pflabel{[pair]satisfy-falsify2}
            \item $\csatisfy{e_1}{\xi_1}$ \BY{IH on \pfref{[pair]satisfy-falsify1}} \pflabel{[pair]satisfy1}
            \item $\csatisfy{e_2}{\xi_2}$ \BY{IH on \pfref{[pair]satisfy-falsify2}} \pflabel{[pair]satisfy2}
            \item $\csatisfy{e}{\cpair{\xi_1}{\xi_2}}$ \BY{Rule (\ref{rule:CSPair}) on \pfref{[pair]satisfy1} and \pfref{[pair]satisfy2}}
            \end{pfsteps*}
        \end{byCases} 
    \end{byCases}
    \resetpfcounter
  \end{enumerate}
\end{proof}

\begin{lemma}
  Assume $\isVal{e}$ and $\ctruify{\xi}=\xi$. Then $\cnotsatisfy{e}{\xi}$ iff $\csatisfy{e}{\cdual{\xi}}$.
\end{lemma}

\begin{theorem}
\label{thrm:redundant-truify-falsify}
  $\csatisfy{\xi_r}{\xi_{rs}}$ iff $\csatisfy{\ctruth}{\cor{\cdual{\ctruify{\xi_r}}}{\cfalsify{\xi_{rs}}}}$.
\end{theorem}
\end{document}
