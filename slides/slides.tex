\documentclass[notheorems]{beamer}
\usepackage{beamerthemeshadow}
\usepackage[absolute,overlay]{textpos}

\usepackage{amsmath,amsthm}

\usepackage{stmaryrd} % llparenthesis
\usepackage{anyfontsize} % workaround for font size difference warning

\usepackage{cancel} % slash over symbol

\newcommand{\highlight}[1]{\colorbox{yellow}{$\displaystyle #1$}}

\newcommand{\todo}[1]{{\color{red} TODO: #1}}

%% Joshua Dunfield macros
\def\OPTIONConf{1}
\usepackage{jdunfield}
\usepackage{rulelinks} % hyperlink of rule name

\newtheoremstyle{slplain}% name
  {.15\baselineskip\@plus.1\baselineskip\@minus.1\baselineskip}% Space above
  {.15\baselineskip\@plus.1\baselineskip\@minus.1\baselineskip}% Space below
  {\slshape}% Body font
  {\parindent}%Indent amount (empty = no indent, \parindent = para indent)
  {\bfseries}%  Thm head font
  {.}%       Punctuation after thm head
  { }%      Space after thm head: " " = normal interword space;
       %       \newline = linebreak
  {}%       Thm head spec
\theoremstyle{slplain}
\newtheorem{thm}{Theorem}  % Numbered with the equation counter
\numberwithin{thm}{section}
\newtheorem{defn}[thm]{Definition}
\newtheorem{lem}[thm]{Lemma}
\newtheorem{prop}[thm]{Proposition}
\newtheorem{corol}[thm]{Corollary}

% BEAMER
\definecolor{hazelgreen}{RGB}{7,63,36}
\definecolor{hazellightgreen}{RGB}{103,138,97}
\definecolor{hazelyellow}{RGB}{245,222,179}
\definecolor{hazellightyellow}{RGB}{254,254,234}

\definecolor{green}{RGB}{132, 194, 70}
\definecolor{yellow}{RGB}{227, 202, 43}
\definecolor{red}{RGB}{249,130,147}

\useoutertheme{infolines}
\useinnertheme{rectangles}

\definecolor{myblue1}{RGB}{35,119,189}
\definecolor{myblue2}{RGB}{95,179,238}
\definecolor{myblue3}{RGB}{129,168,207}
\definecolor{myblue4}{RGB}{26,89,142}

\setbeamercolor*{structure}{fg=hazellightgreen,bg=blue}
\setbeamercolor*{palette primary}{use=structure,fg=white,bg=structure.fg}
\setbeamercolor*{palette secondary}{use=structure,fg=white,bg=structure.fg!75!black}
\setbeamercolor*{palette tertiary}{use=structure,fg=white,bg=structure.fg!50!black}
\setbeamercolor*{palette quaternary}{fg=black,bg=white}

\setbeamercolor{background canvas}{bg=hazellightyellow}
\setbeamercolor{itemize item}{fg=hazelgreen}
\setbeamercolor{frame title}{fg=white,bg=hazellightgreen}
\setbeamercolor{block body}{fg=black,bg=white}

\newlength{\sepwid}
\newlength{\onecolwid}
\newlength{\twocolwid}
\setlength{\sepwid}{0.025\paperwidth}
\setlength{\onecolwid}{0.3\paperwidth}
\setlength{\twocolwid}{0.625\paperwidth}

\usepackage{xcolor}
\usepackage{graphicx}
\usepackage{listings}
\usepackage{tikz}
\usetikzlibrary{calc,fit,tikzmark,plotmarks,arrows.meta,positioning,overlay-beamer-styles}
\usetikzmarklibrary{listings}

\lstset{
  language=caml,
  breaklines=true,
       showspaces=false,                          % show spaces (with underscores)
     showstringspaces=false,            % underline spaces within strings
     showtabs=false,                            % show tabs using underscores
     tabsize=4,                     % default tabsize
     columns=fullflexible,
     breakautoindent=false,
     framerule=1pt,
     xleftmargin=0pt,
     xrightmargin=0pt,
     breakindent=0pt,
     resetmargins=true
  }

\tikzset{
  next page=below,
%  every highlighter/.style={rounded corners},
  line/.style={
    draw,
    rounded corners=3pt,
    -latex
  },
  balloon/.style={
    draw,
    fill=red!20,
    opacity=0.4,
    inner sep=4pt,
    inner ysep=0pt,
    rounded corners=2pt,
    shift={(0,-0.14)}
  },
  comment/.style={
    draw,
    fill=blue!70,
    text=white,
%    text width=3cm,
%    minimum height=1cm,
    rounded corners,
    drop shadow,
    align=left,
%    font=\scriptsize
  },
}

\newcommand\balloon[4]{%
  \pgfmathtruncatemacro\firstline{%
    #3-1
  }%
  \iftikzmark{line-#2-\firstline-start}{%
    \iftikzmark{line-#2-#3-first}{%
      \xdef\blines{({pic cs:line-#2-\firstline-start} -| {pic           cs:line-#2-#3-first})}%
    }{%
      \iftikzmark{line-#2-#3-start}{%
        \xdef\blines{({pic cs:line-#2-\firstline-start} -| {pic             cs:line-#2-#3-start})}%
      }{%
        \xdef\blines{(pic cs:line-#2-\firstline-start)}%
      }%
    }%
  }{%
    \xdef\blines{}%
  }%
  \foreach \k in {#3,...,#4} {%
    \iftikzmark{line-#2-\k-first}{%
      \xdef\blines{\blines (pic cs:line-#2-\k-first) }
    }{}
    \iftikzmark{line-#2-\k-end}{%
      \xdef\blines{\blines (pic cs:line-#2-\k-end) }
    }{}
  }%
  \ifx\blines\empty
  \else
  \edef\temp{\noexpand\tikz[remember picture,overlay]\noexpand\node[fit={\blines},balloon] (#1) {};}%
\temp
  \fi
}


% !TEX root = ./patterns-paper.tex

% reverse Vdash
\newcommand{\dashV}{\mathbin{\rotatebox[origin=c]{180}{$\Vdash$}}}

% Violet hotdogs; highlight color helps distinguish them
\newcommand{\llparenthesiscolor}{\textcolor{violet}{\llparenthesis}}
\newcommand{\rrparenthesiscolor}{\textcolor{violet}{\rrparenthesis}}

% HTyp and HExp
\newcommand{\hcomplete}[1]{#1~\mathsf{complete}}

% HTyp
\newcommand{\htau}{\dot{\tau}}
\newcommand{\tarr}[2]{\inparens{#1 \rightarrow #2}}
\newcommand{\tarrnp}[2]{#1 \rightarrow #2}
\newcommand{\trul}[2]{\inparens{#1 \Rightarrow #2}}
\newcommand{\tnum}{\mathtt{num}}
\newcommand{\tehole}{\llparenthesiscolor\rrparenthesiscolor}
\newcommand{\tsum}[2]{\inparens{{#1} + {#2}}}
\newcommand{\tprod}[2]{\inparens{{#1} \times {#2}}}
\newcommand{\tunit}{\mathtt{1}}
\newcommand{\tvoid}{\mathtt{0}}

\newcommand{\tcompat}[2]{#1 \sim #2}
\newcommand{\tincompat}[2]{#1 \nsim #2}

% HExp
\newcommand{\hexp}{\dot{e}}
\newcommand{\hlam}[3]{\inparens{\lambda #1:#2.#3}}
\newcommand{\hap}[2]{#1(#2)}
\newcommand{\hapP}[2]{(#1)~(#2)} % Extra paren around function term
\newcommand{\hnum}[1]{\underline{#1}}
\newcommand{\hadd}[2]{\inparens{#1 + #2}}
\newcommand{\hpair}[2]{\inparens{#1 , #2}}
\newcommand{\htriv}{()}
\newcommand{\hehole}{\llparenthesiscolor\rrparenthesiscolor}
\newcommand{\hhole}[1]{\llparenthesiscolor#1\rrparenthesiscolor}
\newcommand{\hindet}[1]{\lceil#1\rceil}
\newcommand{\hinj}[2]{\mathtt{inj}_{#1}({#2})}
\newcommand{\hinl}[2]{\mathtt{inl}_{#1}({#2})}
\newcommand{\hinr}[2]{\mathtt{inr}_{#1}({#2})}
\newcommand{\hinlp}[1]{\mathtt{inl}(#1)}
\newcommand{\hinrp}[1]{\mathtt{inr}(#1)}
\newcommand{\hmatch}[2]{\mathtt{match}(#1) \{#2\}}
\newcommand{\hcase}[5]{\mathtt{case}({#1},{#2}.{#3},{#4}.{#5})}
\newcommand{\hrules}[2]{\inparens{#1 \mid #2}}
\newcommand{\hrul}[2]{#1 \Rightarrow #2}

\newcommand{\hGamma}{\dot{\Gamma}}
\newcommand{\domof}[1]{\text{dom}(#1)}
\newcommand{\hsyn}[3]{#1 \vdash #2 \Rightarrow #3}
\newcommand{\hana}[3]{#1 \vdash #2 \Leftarrow #3}
\newcommand{\hexptyp}[3]{#1 \vdash #2 : #3}
\newcommand{\hpattyp}[3]{#1 : #2 \dashV #3}
\newcommand{\hpatmatch}[3]{#1 \vartriangleright #2 \dashV #3}
\newcommand{\hval}[1]{#1 ~\mathtt{val}}
\newcommand{\herr}[1]{#1 ~\mathtt{err}}

% ZTyp and ZExp
\newcommand{\zlsel}[1]{{\bowtie}{#1}}
\newcommand{\zrsel}[1]{{#1}{\bowtie}}
\newcommand{\zwsel}[1]{
  \setlength{\fboxsep}{0pt}
  \colorbox{green!10!white!100}{
    \ensuremath{{{\textcolor{Green}{{\hspace{-2px}\triangleright}}}}{#1}{\textcolor{Green}{\triangleleft{\vphantom{\tehole}}}}}}
}

\newcommand{\removeSel}[1]{#1^{\diamond}}

% ZTyp
\newcommand{\ztau}{\hat{\tau}}

% ZExp
\newcommand{\zexp}{\hat{e}}

% Direction
\newcommand{\dParent}{\mathtt{parent}}
\newcommand{\dChildn}[1]{\mathtt{child}~\mathtt{{#1}}}
\newcommand{\dChildnm}[1]{\mathtt{child}~{#1}}

% Action
\newcommand{\aMove}[1]{\mathtt{move}~#1}
	\newcommand{\zrightmost}[1]{\mathsf{rightmost}(#1)}
	\newcommand{\zleftmost}[1]{\mathsf{leftmost}(#1)}
\newcommand{\aSelect}[1]{\mathtt{sel}~#1}
\newcommand{\aDel}{\mathtt{del}}
\newcommand{\aReplace}[1]{\mathtt{replace}~#1}
\newcommand{\aConstruct}[1]{\mathtt{construct}~#1}
\newcommand{\aConstructx}[1]{#1}
\newcommand{\aFinish}{\mathtt{finish}}

\newcommand{\performAna}[5]{#1 \vdash #2 \xlongrightarrow{#4} #5 \Leftarrow #3}
\newcommand{\performAnaI}[5]{#1 \vdash #2 \xlongrightarrow{#4}\hspace{-3px}{}^{*}~ #5 \Leftarrow #3}
\newcommand{\performSyn}[6]{#1 \vdash #2 \Rightarrow #3 \xlongrightarrow{#4} #5 \Rightarrow #6}
\newcommand{\performSynI}[6]{#1 \vdash #2 \Rightarrow #3 \xlongrightarrow{#4}\hspace{-3px}{}^{*}~ #5 \Rightarrow #6}
\newcommand{\performTyp}[3]{#1 \xlongrightarrow{#2} #3}
\newcommand{\performTypI}[3]{#1 \xlongrightarrow{#2}\hspace{-3px}{}^{*}~#3}

\newcommand{\performMove}[3]{#1 \xlongrightarrow{#2} #3}
\newcommand{\performDel}[2]{#1 \xlongrightarrow{\aDel} #2}

% Form
\newcommand{\farr}{\mathtt{arrow}}
\newcommand{\fnum}{\mathtt{num}}
\newcommand{\fsum}{\mathtt{sum}}

\newcommand{\fasc}{\mathtt{asc}}
\newcommand{\fvar}[1]{\mathtt{var}~#1}
\newcommand{\flam}[1]{\mathtt{lam}~#1}
\newcommand{\fap}{\mathtt{ap}}
% \newcommand{\farg}{\mathtt{arg}}
\newcommand{\fnumlit}[1]{\mathtt{lit}~#1}
\newcommand{\fplus}{\mathtt{plus}}
\newcommand{\fhole}{\mathtt{hole}}
\newcommand{\fnehole}{\mathtt{nehole}}

\newcommand{\finj}[1]{\mathtt{inj}~#1}
\newcommand{\fcase}[2]{\mathtt{case}~#1~#2}

% Talk about formal rules in example
\newcommand{\refrule}[1]{\textrm{Rule~(#1)}}

\newcommand{\herase}[1]{\left|#1\right|_\textsf{erase}}

\newcommand{\arrmatch}[2]{#1 \blacktriangleright_{\rightarrow} #2}


\newcommand{\TABperformAna}[5]{#1 \vdash & #2                & \xlongrightarrow{#4} & #5 & \Leftarrow #3}
\newcommand{\TABperformSyn}[6]{#1 \vdash & #2 \Rightarrow #3 & \xlongrightarrow{#4} & #5 \Rightarrow #6}
\newcommand{\TABperformTyp}[3]{& #1 & \xlongrightarrow{#2} & #3}

\newcommand{\TABperformMove}[3]{#1 & \xlongrightarrow{#2} & #3}
\newcommand{\TABperformDel}[2]{#1 \xlongrightarrow{\aDel} #2}

\newcommand{\sumhasmatched}[2]{#1 \mathrel{\textcolor{black}{\blacktriangleright_{+}}} #2}

\newcommand{\subminsyn}[1]{\mathsf{submin}_{\Rightarrow}(#1)}
\newcommand{\subminana}[1]{\mathsf{submin}_{\Leftarrow}(#1)}


\newcommand{\inparens}[1]{{\color{gray}(}#1{\color{gray})}}

%% rule names for appendix
\newcommand{\rname}[1]{\textsc{#1}}
\newcommand{\gap}{\vspace{7pt}}


% !TEX root = pattern-paper.tex
%% Statics

% typing of internal expressions
\newrulecommand{TVar}{TVar}
\newrulecommand{TNum}{TNum}
\newrulecommand{TLam}{TLam}
\newrulecommand{TAp}{TAp}
\newrulecommand{TPair}{TPair}
\newrulecommand{TPrl}{TPrl}
\newrulecommand{TPrr}{TPrr}
\newrulecommand{TInl}{TInl}
\newrulecommand{TInr}{TInr}
\newrulecommand{TMatchZPre}{TMatchZPre}
\newrulecommand{TMatchNZPre}{TMatchNZPre}
\newrulecommand{TEHole}{TEHole}
\newrulecommand{THole}{THole}

% typing of rule and rules
\newrulecommand{TRule}{TRule}
\newrulecommand{TOneRules}{TOneRules}
\newrulecommand{TRules}{TRules}

% typing of pattern
\newrulecommand{PTVar}{PTVar}
\newrulecommand{PTNum}{PTNum}
\newrulecommand{PTWild}{PTWild}
\newrulecommand{PTPair}{PTPair}
\newrulecommand{PTInl}{PTInl}
\newrulecommand{PTInr}{PTInr}
\newrulecommand{PTEHole}{PTEHole}
\newrulecommand{PTHole}{PTHole}

%% Dynamics

% typing of substitution
\newrulecommand{STEmpty}{STEmpty}
\newrulecommand{STExtend}{STExtend}

% refutable pattern
\newrulecommand{RNum}{RNum}
\newrulecommand{REHole}{REHole}
\newrulecommand{RHole}{RHole}
\newrulecommand{RInl}{RInl}
\newrulecommand{RInr}{RInr}
\newrulecommand{RPairL}{RPairL}
\newrulecommand{RPairR}{RPairR}

% match
\newrulecommand{MVar}{MVar}
\newrulecommand{MNum}{MNum}
\newrulecommand{MWild}{MWild}
\newrulecommand{MPair}{MPair}
\newrulecommand{MEHolePair}{MEHolePair}
\newrulecommand{MHolePair}{MHolePair}
\newrulecommand{MApPair}{MApPair}
\newrulecommand{MMatchPair}{MMatchPair}
\newrulecommand{MPrlPair}{MPrlPair}
\newrulecommand{MPrrPair}{MPrrPair}
\newrulecommand{MInl}{MInl}
\newrulecommand{MInr}{MInr}

% may match
\newrulecommand{MMEHole}{MMEHole}
\newrulecommand{MMHole}{MMHole}
\newrulecommand{MMExpEHole}{MMExpEHole}
\newrulecommand{MMExpHole}{MMExpHole}
\newrulecommand{MMAp}{MMAp}
\newrulecommand{MMMatch}{MMMatch}
\newrulecommand{MMPrl}{MMPrl}
\newrulecommand{MMPrr}{MMPrr}
\newrulecommand{MMPairL}{MMPairL}
\newrulecommand{MMPairR}{MMPairR}
\newrulecommand{MMPair}{MMPair}
\newrulecommand{MMInl}{MMInl}
\newrulecommand{MMInr}{MMInr}

% not match
\newrulecommand{NMNum}{NMNum}
\newrulecommand{NMPairL}{NMPairL}
\newrulecommand{NMPairR}{NMPairR}
\newrulecommand{NMConfL}{NMConfL}
\newrulecommand{NMConfR}{NMConfR}
\newrulecommand{NMInl}{NMInl}
\newrulecommand{NMInr}{NMInr}

% value
\newrulecommand{VNum}{VNum}
\newrulecommand{VLam}{VLam}
\newrulecommand{VPair}{VPair}
\newrulecommand{VInl}{VInl}
\newrulecommand{VInr}{VInr}

% indeterminate
\newrulecommand{IEHole}{IEHole}
\newrulecommand{IHole}{IHole}
\newrulecommand{IAp}{IAp}
\newrulecommand{IPairL}{IPairL}
\newrulecommand{IPairR}{IPairR}
\newrulecommand{IPair}{IPair}
\newrulecommand{IPrl}{IPrl}
\newrulecommand{IPrr}{IPrr}
\newrulecommand{IInl}{IInl}
\newrulecommand{IInr}{IInr}
\newrulecommand{IMatch}{IMatch}

% final
\newrulecommand{FVal}{FVal}
\newrulecommand{FIndet}{FIndet}

% step
\newrulecommand{ITHole}{ITHole}
\newrulecommand{ITApFun}{ITApFun}
\newrulecommand{ITApArg}{ITApArg}
\newrulecommand{ITAp}{ITAp}
\newrulecommand{ITPairL}{ITPairL}
\newrulecommand{ITPairR}{ITPairR}
\newrulecommand{ITPrl}{ITPrl}
\newrulecommand{ITPrr}{ITPrr}
\newrulecommand{ITInl}{ITInl}
\newrulecommand{ITInr}{ITInr}
\newrulecommand{ITExpMatch}{ITExpMatch}
\newrulecommand{ITSuccMatch}{ITSuccMatch}
\newrulecommand{ITFailMatch}{ITFailMatch}

% typing of constraints
\newrulecommand{CTTruth}{CTTruth}
\newrulecommand{CTFalsity}{CTFalsity}
\newrulecommand{CTUnknown}{CTUnknown}
\newrulecommand{CTNum}{CTNum}
\newrulecommand{CTNotNum}{CTNotNum}
\newrulecommand{CTAnd}{CTAnd}
\newrulecommand{CTOr}{CTOr}
\newrulecommand{CTInl}{CTInl}
\newrulecommand{CTInr}{CTInr}
\newrulecommand{CTPair}{CTPair}

% satisfaction judgment
\newrulecommand{CSTruth}{CSTruth}
\newrulecommand{CSNum}{CSNum}
\newrulecommand{CSNotNum}{CSNotNum}
\newrulecommand{CSAnd}{CSAnd}
\newrulecommandONE{CSOrL}{CSOrL}
\newrulecommandONE{CSOrR}{CSOrR}
\newrulecommand{CSInl}{CSInl}
\newrulecommand{CSInr}{CSInr}
\newrulecommand{CSPair}{CSPair}
\newrulecommand{CSEHolePair}{CSEHolePair}
\newrulecommand{CSHolePair}{CSHolePair}
\newrulecommand{CSApPair}{CSApPair}
\newrulecommand{CSMatchPair}{CSMatchPair}
\newrulecommand{CSPrlPair}{CSPrlPair}
\newrulecommand{CSPrrPair}{CSPrrPair}

% maybe satisfaction judgment
\newrulecommand{CMSUnknown}{CMSUnknown}
\newrulecommand{CMSExpEHole}{CMSExpEHole}
\newrulecommand{CMSExpHole}{CMSExpHole}
\newrulecommand{CMSAp}{CMSAp}
\newrulecommand{CMSMatch}{CMSMatch}
\newrulecommand{CMSPrl}{CMSPrl}
\newrulecommand{CMSPrr}{CMSPrr}
\newrulecommand{CMSAndL}{CMSAndL}
\newrulecommand{CMSAndR}{CMSAndR}
\newrulecommand{CMSAnd}{CMSAnd}
\newrulecommand{CMSOrL}{CMSOrL}
\newrulecommand{CMSOrR}{CMSOrR}
\newrulecommand{CMSInl}{CMSInl}
\newrulecommand{CMSInr}{CMSInr}
\newrulecommand{CMSPairL}{CMSPairL}
\newrulecommand{CMSPairR}{CMSPairR}
\newrulecommand{CMSPair}{CMSPair}

% must or maybe satisfaction judgment
\newrulecommand{CSMSMay}{CSMSMay}
\newrulecommand{CSMSSat}{CSMSSat}

% incon
\newrulecommand{CINCTruth}{CINCTruth}
\newrulecommand{CINCFalsity}{CINCFalsity}
\newrulecommand{CINCNum}{CINCNum}
\newrulecommand{CINCNotNum}{CINCNotNum}
\newrulecommand{CINCAnd}{CINCAnd}
\newrulecommand{CINCOr}{CINCOr}
\newrulecommand{CINCInj}{CINCInj}
\newrulecommand{CINCInl}{CINCInl}
\newrulecommand{CINCInr}{CINCInr}
\newrulecommand{CINCPairL}{CINCPairL}
\newrulecommand{CINCPairR}{CINCPairR}

\title{Pattern Matching with Typed Holes}
\author{Yongwei Yuan}
\institute{University of Michigan}

\begin{document}
\addtobeamertemplate{block end}{}{\vspace*{2ex}} % White space under blocks
\addtobeamertemplate{block example end}{}{\vspace*{2ex}} % White space under example blocks
\addtobeamertemplate{block alerted end}{}{\vspace*{2ex}} % White space under highlighted (alert) blocks

\setlength{\belowcaptionskip}{2ex} % White space under figures
\setlength\belowdisplayshortskip{2ex} % White space under equations
%\begin{darkframes} % Uncomment for dark theme, don't forget to \end{darkframes}


\frame{
  \titlepage
  \let\thefootnote\relax\footnote{Supervised by Cyrus Omar at Future of Programming Lab}
}

\section{Motivation and Background}
\begin{frame}[fragile]
  \frametitle{Pattern Matching}
\begin{center}
\begin{tabular}{c}
  \begin{lstlisting}[basicstyle=\small,name=pm no hole,escapechar=!,mathescape]
    match (1::[ ]) {
    !\tikzmark{pm no hole rule1}!| [ ] -> "empty"
    !\tikzmark{pm no hole rule2}!| 1::xs -> "1,..."
    | 2::xs -> "2,..."
    }
  \end{lstlisting}
\end{tabular}
\\
\vspace{1cm}
\only<1>{
  The scrutinee \textsf{does not match} the pattern in the first branch.
}
\only<2>{
  The scrutinee \textsf{matches} the pattern in the second branch.
}
\end{center}

\only<1>{
\begin{tikzpicture}[remember picture, overlay]
  \node[rotate=-90,mark size=5pt,color=hazellightgreen,shift={(-0.1,-0.2)}] at (pic cs:pm no hole rule1) {\pgfuseplotmark{triangle*}};
\end{tikzpicture}
}
\only<2>{
\begin{tikzpicture}[remember picture, overlay]
  \node[rotate=-90,mark size=5pt,color=hazellightgreen,shift={(-0.1,-0.2)}] at (pic cs:pm no hole rule2) {\pgfuseplotmark{triangle*}};
\end{tikzpicture}
}
\end{frame}

\frame[containsverbatim]{\frametitle{Exhaustiveness and Redundancy}
  \begin{columns}
    \begin{column}{5cm}
      \begin{lstlisting}[name=not exhaustive,title={Not Exhaustive},captionpos=b,escapechar=!]
        match (3::[ ]) {
        | [ ] -> "empty"
        | 1::xs -> "1,..."
        | 2::xs -> "2,..."
        }
      \end{lstlisting}
      \balloon{comment}{not exhaustive}{2}{4}
    \end{column}

    \begin{column}{5cm}
      \begin{lstlisting}[name=redundant,title={Redundant Branch},captionpos=b,escapechar=!]
        match (1::2::[ ]) {
        | [ ] -> "empty"
        | 1::xs -> "1,..."
        | 1::2::xs -> "1,2,..."
        }
      \end{lstlisting}
      \balloon{comment}{redundant}{4}{4}
    \end{column}
  \end{columns}
Chapter \textit{Pattern Matching} in \textsf{PFPL} \cite{Harper2012} introduces a match constraint language and the algorithm to check exhaustiveness of branches and redundancy of a single branch with respect to its preceding branches.
}

\begin{frame}[fragile]
  \frametitle{What if Match Expression is Incomplete?}
  \only<1>{
    \begin{itemize}
       \item \textsf{Agda} and \textsf{Haskell} allow programmers to use typed holes to represent missing parts in the program.
       \item \textsf{Hazel} is a live programming environment featuring typed holes \cite{DBLP:conf/popl/OmarVHAH17,DBLP:journals/pacmpl/OmarVCH19}, but it only supports simple case analysis on binary sum types.
      
    \end{itemize}
  }
  \begin{columns}
    \begin{column}{5cm}
      \begin{onlyenv}<2->
      \begin{lstlisting}[name=not exhaustive,title={Expression hole},captionpos=b,escapechar=!]
        match (1::!\textcolor{green}{?}!) {
        | [ ] -> "empty"
        | 1::xs -> "1,..."
        | 2::xs -> "2,..."
        }
      \end{lstlisting}
    \end{onlyenv}
    \end{column}

    \begin{column}{5cm}
      \begin{onlyenv}<3->
      \begin{lstlisting}[name=redundant,title={Pattern hole},captionpos=b,escapechar=!]
        match (1::[ ]) {
        | [ ] -> "empty"
        | !\textcolor{green}{?}!::xs -> "1,..."
        | 2::xs -> "2,..."
        }
      \end{lstlisting}
      \end{onlyenv}
    \end{column}
  \end{columns}
  \begin{columns}
    \begin{column}{5cm}
      \begin{onlyenv}<4->
      \begin{lstlisting}[title={Exhaustive?},captionpos=b,escapechar=!]
        match (3::[ ]) {
        | [ ] -> "empty"
        | 1::xs -> "1,..."
        | !\textcolor{green}{?}!::xs -> "2,..."
        }
      \end{lstlisting}
      \end{onlyenv}
    \end{column}

    \begin{column}{5cm}
      \begin{onlyenv}<5->
      \begin{lstlisting}[title={Redundant?},captionpos=b,escapechar=!]
        match (1::2::[ ]) {
        | [ ] -> "empty"
        | !\textcolor{green}{?}!::xs -> "1,..."
        | 1::2::xs -> "1,2,..."
        }
      \end{lstlisting}
      \end{onlyenv}
    \end{column}
  \end{columns}
\end{frame}

\section{Our Approach}
\frame{
  The key point behind reasoning about incomplete programs is to give feedback that is always correct no matter how programmers fill these holes at the end.
}
\frame{
  \frametitle{Early Evaluation}
  We can evaluate the expression even if there are holes \\
  as long as the evaluation is correct regardless of how holes are filled.
}
\subsection{Early Evaluation}

\begin{frame}[fragile]
  \frametitle{Expression Hole doesn't have to Stop Evaluation}
\begin{center}
\begin{tabular}{c}
  \begin{lstlisting}[basicstyle=\small,name=exp hole,escapechar=!,mathescape]
    match (1::!\textcolor{green}{?}!) {
    !\tikzmark{exp hole rule1}!| [ ] -> "empty"
    !\tikzmark{exp hole rule2}!| 1::xs -> "1,..."
    | 2::xs -> "2,..."
    }
  \end{lstlisting}
\end{tabular}
\\
\vspace{1cm}
\only<1>{
  The scrutinee \textsf{does not match} the pattern in the first branch.
}
\only<2>{
  The scrutinee \textsf{matches} the pattern in the second branch. \\
  The value of the match expression is a string "1,...".
}
\end{center}

\only<1>{
\begin{tikzpicture}[remember picture, overlay]
  \node[rotate=-90,mark size=5pt,color=hazellightgreen,shift={(-0.1,-0.2)}] at (pic cs:exp hole rule1) {\pgfuseplotmark{triangle*}};
\end{tikzpicture}
}
\only<2>{
\begin{tikzpicture}[remember picture, overlay]
  \node[rotate=-90,mark size=5pt,color=hazellightgreen,shift={(-0.1,-0.2)}] at (pic cs:exp hole rule2) {\pgfuseplotmark{triangle*}};
\end{tikzpicture}
}
\end{frame}

\begin{frame}[fragile]
  \frametitle{Pattern Hole May be Matched}
\begin{center}
\begin{tabular}{c}
  \begin{lstlisting}[basicstyle=\small,name=pat hole,escapechar=!,mathescape]
    match (1::[ ]) {
    !\tikzmark{pat hole rule1}!| [ ] -> "empty"
    !\tikzmark{pat hole rule2}!| !\textcolor{green}{?}!::xs -> "1,..."
    | 2::xs -> "2,..."
    }
  \end{lstlisting}
\end{tabular}
\\
\vspace{1cm}
\only<1>{
  The scrutinee \textsf{does not match} the pattern in the first branch.
}
\only<2>{
  The scrutinee \textbf{may match} the pattern in the second branch.\\
  Because $1$ may match the hole $?$ depending on what is filled. \\
  We say that the match expression is \textbf{indeterminate}.
}
\end{center}

\only<1>{
\begin{tikzpicture}[remember picture, overlay]
  \node[rotate=-90,mark size=5pt,color=hazellightgreen,shift={(-0.1,-0.2)}] at (pic cs:pat hole rule1) {\pgfuseplotmark{triangle*}};
\end{tikzpicture}
}
\only<2>{
\begin{tikzpicture}[remember picture, overlay]
  \node[rotate=-90,mark size=5pt,color=hazellightgreen,shift={(-0.1,-0.2)}] at (pic cs:pat hole rule2) {\pgfuseplotmark{triangle*}};
\end{tikzpicture}
}
\end{frame}

\begin{frame}
  \frametitle{Evaluate as further as possible}
  We keep evaluating the expression until it is either a \textbf{value} or \textbf{indeterminate}. \\
  We regard such an expression as \textbf{final}.
\end{frame}

\subsection{Best-Case Error Reporting}
\frame{
  \frametitle{Best-Case Error Reporting}
  Report error only when it can't be avoided.
}

\begin{frame}[fragile]
  \frametitle{No Error Messsage when the Branches May be Exhaustive}
\begin{center}
\begin{tabular}{c}
  \begin{lstlisting}[escapechar=!]
    match (3::[ ]) {
    | [ ] -> "empty"
    | 1::xs -> "1,..."
    | !
    \textcolor{green}{
    \begin{onlyenv}<1>
      ?
    \end{onlyenv}
    \begin{onlyenv}<2>
      x
    \end{onlyenv}
    }
    !::xs -> "2,..."
    }
  \end{lstlisting}
\end{tabular}
\\
\vspace{1cm}
\only<1>{
  Any pattern of Num type can be filled in the hole.
}
\only<2>{
  By filling the hole with some variable, the branches can be exhaustive.
}
\end{center}
\end{frame}

\begin{frame}[fragile]
  \frametitle{Prompt Error Messsage \\ only when the Branches Mustn't be Exhaustive}
\begin{center}
\begin{tabular}{c}
  \begin{lstlisting}[escapechar=!]
    match (3::[ ]) {
    | [ ] -> "empty"
    | 1::xs -> "1,..."
    | 2::!
    \textcolor{green}{
    \begin{onlyenv}<1>
      ?
    \end{onlyenv}
    \begin{onlyenv}<2>
      xs
    \end{onlyenv}
    }
    ! -> "2,..."
    }
  \end{lstlisting}
\end{tabular}
\\
\vspace{1cm}
\only<1>{
  Any pattern of List of Num type can be filled in the hole.
}
\only<2>{
  No matter what pattern we fill in the hole, \\
  the branches can't be exhaustive.
}
\end{center}
\end{frame}

\begin{frame}[fragile]
  \frametitle{No Error Messsage when the Branch may be Redundant}
\begin{center}
\begin{tabular}{c}
  \begin{lstlisting}[escapechar=!]
    match (2::[ ]) {
    | [ ] -> "empty"
    | !
    \textcolor{green}{
    \begin{onlyenv}<1>
      ?
    \end{onlyenv}
    \begin{onlyenv}<2>
      2
    \end{onlyenv}
    \begin{onlyenv}<3>
      x
    \end{onlyenv}
    \begin{onlyenv}<4>
      3
    \end{onlyenv}
    }
    !::xs -> "1,..."
    | 2::xs -> "2,..."
    }
  \end{lstlisting}
\end{tabular}
\\
\vspace{1cm}
\only<1>{
  Any pattern of Num type can be filled in the hole.
}
\only<2>{
  By filling the hole with number 2, the third branch can be redundant.
}
\only<3>{
  By filling the hole with a variable, the third branch is also redundant.
}
\only<4>{
  By filling the hole with a number other than 2, \\
  the third branch is not redundant.
}
\end{center}
\end{frame}

\begin{frame}[fragile]
  \frametitle{Prompt Error Messsage when the Branch must be Redundant}
\begin{center}
\begin{tabular}{c}
  \begin{lstlisting}[escapechar=!]
    match (3::[ ]) {
    | [ ] -> "empty"
    | x::xs -> "..."
    | !
    \textcolor{green}{
    \begin{onlyenv}<1>
      ?
    \end{onlyenv}
    \begin{onlyenv}<2>
      1
    \end{onlyenv}
    \begin{onlyenv}<3>
      2
    \end{onlyenv}
    \begin{onlyenv}<4>
      y
    \end{onlyenv}
    }
    !::xs -> "?,..."
    }
  \end{lstlisting}
\end{tabular}
\\
\vspace{1cm}
\only<1>{
  Any pattern of Num type can be filled in the hole.
}
\only<2->{
  No matter what pattern we fill in the hole, \\
  the third branch must be exhaustive.
}
\end{center}
\end{frame}

\begin{frame}
  \frametitle{Prompt Error Message or Not}
  Yes
  \begin{itemize}
    \item Branches mustn't be exhaustive
    \item Some branch must be redundant
  \end{itemize}
  No
  \begin{itemize}
    \item Branches must be exhaustive
    \item Branches may be exhaustive
    \item Every branch either
      \begin{itemize}
        \item mustn't be redundant
        \item may be redundant
      \end{itemize}
  \end{itemize}
\end{frame}


\section{Formalism}
\subsection{Syntax}
\frame[containsverbatim]{
  \frametitle{A Match Expression in Our Internal Language}
  \begin{columns}
    \begin{column}{.8\textwidth}
    \begin{lstlisting}[name=syntax,escapechar=!,mathescape]
      match (!\tikzmark{scrut-left}!inr((1, $\hehole{u}$))!\tikzmark{scrut-right}!) {
      | !\tikzmark{empty-left}!inl(())!\tikzmark{empty-right}! -> "empty"
      | !\tikzmark{hole-xs-left}!inr(($\hehole{w}$, xs))!\tikzmark{hole-xs-right}! -> "1,..."
      | !\tikzmark{1-2-xs-left}!inr((1, inr((2, xs))))!\tikzmark{1-2-xs-right}! -> "1,2,..."
      }
    \end{lstlisting}
    \end{column}
    \begin{column}{.2\textwidth}
      \begin{itemize}
        \item[\tikzmark{scrut}] 1::?
        \item[\tikzmark{empty}] [\,]
        \item[\tikzmark{hole-xs}] ?::xs
        \item[\tikzmark{1-2-xs}] 1::2::xs
      \end{itemize}
    \end{column}
      \begin{tikzpicture}[remember picture, overlay]
        \draw[hazellightgreen,rounded corners]
        ([shift={(0pt,2ex)}]pic cs:scrut-left) 
          rectangle 
        ([shift={(0pt,-0.65ex)}]pic cs:scrut-right);
        \draw[line width=1,hazellightgreen,->]
        (pic cs:scrut-right) -> ($(pic cs:scrut)+(0,0.1)$);

        \draw[hazellightgreen,rounded corners]
        ([shift={(-3pt,2ex)}]pic cs:empty-left) 
          rectangle 
        ([shift={(3pt,-0.65ex)}]pic cs:empty-right);
        \draw[line width=1,hazellightgreen,->]
        (pic cs:empty-right) -> ($(pic cs:empty)+(0,0.1)$);

        \draw[hazellightgreen,rounded corners]
        ([shift={(-3pt,2ex)}]pic cs:hole-xs-left) 
          rectangle 
        ([shift={(3pt,-0.65ex)}]pic cs:hole-xs-right);
        \draw[line width=1,hazellightgreen,->]
        (pic cs:hole-xs-right) -> ($(pic cs:hole-xs)+(0,0.1)$);

        \draw[hazellightgreen,rounded corners]
        ([shift={(-3pt,2ex)}]pic cs:1-2-xs-left) 
          rectangle 
        ([shift={(3pt,-0.65ex)}]pic cs:1-2-xs-right);
        \draw[line width=1,hazellightgreen,->]
          (pic cs:1-2-xs-right) -> ($(pic cs:1-2-xs)+(0,0.1)$);
      \end{tikzpicture}
  \end{columns}
}

\subsection{Constraint Language: how to "predict" pattern matching}
\frame{
  \frametitle{Constraint Emitting and Relationship Triangle}
%  % !TEX root= extended-abstract.tex

\begin{figure}[h]
$\arraycolsep=4pt\begin{array}{lll}
e & ::= &
  x ~\vert~
  \htriv ~\vert~
  \hnum{n} ~\vert~
  \hpair{e_1}{e_2} ~\vert~
  \hinl{\tau}{e} ~\vert~
  \hinr{\tau}{e} ~\vert~
  \hehole{u} ~\vert~
  \hhole{e}{u} \\
  & ~\vert~ &
  \hlam{x}{\tau}{e} ~\vert~
  \hap{e_1}{e_2} ~\vert~
  \hmatch{e}{\hat{rs}} \\
p & ::= &
  x ~\vert~
  \_ ~\vert~
  \htriv ~\vert~
  \hnum{n} ~\vert~
  \hpair{p_1}{p_2} ~\vert~
  \hinlp{p} ~\vert~
  \hinrp{p} ~\vert~
  \hehole{w} ~\vert~
  \hhole{p}{w} \\
\xi & ::= &
  \ctruth ~\vert~
  \cfalsity ~\vert~
  \cunit ~\vert~
  \cnum{n} ~\vert~
  \cnotnum{n} ~\vert~
  \cpair{\xi_1}{\xi_2} ~\vert~
  \cinl{\xi} ~\vert~
  \cinr{\xi} ~\vert~
  \cunknown ~\vert~
  \cand{\xi_1}{\xi_2} ~\vert~
  \cor{\xi_1}{\xi_2}
\end{array}$
\label{fig:exp-pat-constraint}
\end{figure}

  \begin{tikzpicture}
    \node (E) at (0,0) {expression e};
    \node (P) [below left=5cm and 3cm of E] {pattern p};
    \node (Xi) [below right=5cm and 3cm of E] {constraint $\xi$};

    \draw [line width=2, ->] (E) -- (P) node (Match) [midway, above, sloped] {\begin{tabular}{c} \textcolor{green}{match} \\ \textcolor{yellow}{may match} \\ \textcolor{red}{not match} \end{tabular}};
    \draw [line width=2, ->] (E) -- (Xi) node (Satisfy) [midway, above, sloped] {\begin{tabular}{c} \textcolor{green}{satisfy $\csatisfy{e}{\xi}$} \\ \textcolor{yellow}{may satisfy $\cmaysatisfy{e}{\xi}$} \\ \textcolor{red}{not satisfy $\csatisfy{e}{\cdual{\xi}}$} \end{tabular}};
    \draw [color=hazellightgreen, line width=2, ->] (P) -- (Xi) node [midway, below, sloped] {emit};
    \draw [color=hazellightgreen, line width=1, {Latex[length=0.2cm]}-{Latex[length=0.2cm]},double] ($(Match)+(5,-0.5)$) -- ($(Satisfy)-(5,0.5)$) node [midway, below, sloped] {one-to-one equivalence};
  \end{tikzpicture}
}

%\frame{
%  \frametitle{Pattern Hole and Unknown Constraint}
%%  % !TEX root= extended-abstract.tex

\begin{figure}[h]
$\arraycolsep=4pt\begin{array}{lll}
e & ::= &
  x ~\vert~
  \htriv ~\vert~
  \hnum{n} ~\vert~
  \hpair{e_1}{e_2} ~\vert~
  \hinl{\tau}{e} ~\vert~
  \hinr{\tau}{e} ~\vert~
  \hehole{u} ~\vert~
  \hhole{e}{u} \\
  & ~\vert~ &
  \hlam{x}{\tau}{e} ~\vert~
  \hap{e_1}{e_2} ~\vert~
  \hmatch{e}{\hat{rs}} \\
p & ::= &
  x ~\vert~
  \_ ~\vert~
  \htriv ~\vert~
  \hnum{n} ~\vert~
  \hpair{p_1}{p_2} ~\vert~
  \hinlp{p} ~\vert~
  \hinrp{p} ~\vert~
  \hehole{w} ~\vert~
  \hhole{p}{w} \\
\xi & ::= &
  \ctruth ~\vert~
  \cfalsity ~\vert~
  \cunit ~\vert~
  \cnum{n} ~\vert~
  \cnotnum{n} ~\vert~
  \cpair{\xi_1}{\xi_2} ~\vert~
  \cinl{\xi} ~\vert~
  \cinr{\xi} ~\vert~
  \cunknown ~\vert~
  \cand{\xi_1}{\xi_2} ~\vert~
  \cor{\xi_1}{\xi_2}
\end{array}$
\label{fig:exp-pat-constraint}
\end{figure}

%  \begin{tikzpicture}
%    \node (E) at (0,0) {any expression e};
%    \node (P) [below left=5cm and 2cm of E] {pattern hole $\hehole{w}$};
%    \node (Xi) [below right=5cm and 2cm of E] {constraint $\cunknown$};
%
%    \draw [line width=2, ->] (E) -- (P) node (Match) [midway, above, sloped] {\textcolor{yellow}{may match}};
%    \draw [line width=2, ->] (E) -- (Xi) node (Satisfy) [midway, above, sloped] {\textcolor{yellow}{may satisfy $\cmaysatisfy{e}{\cunknown}$}};
%    \draw [color=hazellightgreen, line width=2, ->] (P) -- (Xi) node [midway, below, sloped] {emit};
%    \draw [color=hazellightgreen, line width=1, {Latex[length=0.2cm]}-{Latex[length=0.2cm]},double] ($(Match)+(5,-0.5)$) -- ($(Satisfy)-(5,0.5)$) node [midway, below, sloped] {equivalence};
%  \end{tikzpicture}
%}
%
%\frame{
%  \frametitle{Expression Hole and Truth Constraint}
%%  % !TEX root= extended-abstract.tex

\begin{figure}[h]
$\arraycolsep=4pt\begin{array}{lll}
e & ::= &
  x ~\vert~
  \htriv ~\vert~
  \hnum{n} ~\vert~
  \hpair{e_1}{e_2} ~\vert~
  \hinl{\tau}{e} ~\vert~
  \hinr{\tau}{e} ~\vert~
  \hehole{u} ~\vert~
  \hhole{e}{u} \\
  & ~\vert~ &
  \hlam{x}{\tau}{e} ~\vert~
  \hap{e_1}{e_2} ~\vert~
  \hmatch{e}{\hat{rs}} \\
p & ::= &
  x ~\vert~
  \_ ~\vert~
  \htriv ~\vert~
  \hnum{n} ~\vert~
  \hpair{p_1}{p_2} ~\vert~
  \hinlp{p} ~\vert~
  \hinrp{p} ~\vert~
  \hehole{w} ~\vert~
  \hhole{p}{w} \\
\xi & ::= &
  \ctruth ~\vert~
  \cfalsity ~\vert~
  \cunit ~\vert~
  \cnum{n} ~\vert~
  \cnotnum{n} ~\vert~
  \cpair{\xi_1}{\xi_2} ~\vert~
  \cinl{\xi} ~\vert~
  \cinr{\xi} ~\vert~
  \cunknown ~\vert~
  \cand{\xi_1}{\xi_2} ~\vert~
  \cor{\xi_1}{\xi_2}
\end{array}$
\label{fig:exp-pat-constraint}
\end{figure}

%  \begin{tikzpicture}
%    \node (E) at (0,0) {expression hole $\hehole{u}$};
%    \node (P) [below left=5cm and 2cm of E] {pattern $xs$};
%    \node (Xi) [below right=5cm and 2cm of E] {constraint $\ctruth$};
%
%    \draw [line width=2, ->] (E) -- (P) node (Match) [midway, above, sloped] {\textcolor{yellow}{match}};
%    \draw [line width=2, ->] (E) -- (Xi) node (Satisfy) [midway, above, sloped] {\textcolor{yellow}{satisfy $\csatisfy{\hehole{u}}{\ctruth}$}};
%    \draw [color=hazellightgreen, line width=2, ->] (P) -- (Xi) node [midway, below, sloped] {emit};
%    \draw [color=hazellightgreen, line width=1, {Latex[length=0.2cm]}-{Latex[length=0.2cm]},double] ($(Match)+(5,-0.5)$) -- ($(Satisfy)-(5,0.5)$) node [midway, below, sloped] {equivalence};
%  \end{tikzpicture}
%}

\subsection{Redundancy and Exhaustiveness Checking}
\frame{
  \frametitle{Branches and $\vee$ constraint}
  \begin{align*}
    & \text{Branch } r & \text{Constraint } \xi\\
    ~\vert~ &
    \hrul{\hinlp{\htriv}}{\text{"empty"}} & \hinlp{\cunit} \\
    ~\vert~ &
    \hrul{\hinrp{\hpair{\hehole{w}}{xs}}}{\text{"empty"}} & \hinrp{\cpair{\cunknown}{\ctruth}} \\
    ~\vert~ &
    \hrul{\hinrp{\hpair{\hnum{1}}{\hinrp{\hpair{\hnum{2}}{xs}}}}}{\text{"empty"}} & \cinr{\cpair{\cnum{1}}{\cinr{\cpair{\cnum{2}}{\ctruth}}}} \\
  \end{align*}
  And multiple branches correspond to their constraints connected by $\vee$.
}
\frame{
  \frametitle{Entailment of Constraints}
  \begin{defn}["Must" or "May" Entailment]
  \label{defn:exhaustiveness}
  $\csatisfyormay{\xi_1}{\xi_2}$ iff $\ctyp{\xi_1}{\tau}$ and $\ctyp{\xi_2}{\tau}$ and for all $e$ such that $\hexptyp{\cdot}{\Gamma}{e}{\tau}$ and $\isFinal{e}$ we have \textcolor{green}{$\csatisfy{e}{\xi_1}$} or \textcolor{yellow}{$\cmaysatisfy{e}{\xi_1}$} implies \textcolor{green}{$\csatisfy{e}{\xi_2}$} or \textcolor{yellow}{$\cmaysatisfy{e}{\xi_2}$}.
\end{defn}

\begin{defn}["Must" Entailment]
  \label{defn:redundancy}
  $\csatisfy{\xi_1}{\xi_2}$ iff $\ctyp{\xi_1}{\tau}$ and $\ctyp{\xi_2}{\tau}$ and for all $e$ such that $\hexptyp{\cdot}{\Gamma}{e}{\tau}$ and $\isVal{e}$ we have \textcolor{green}{$\csatisfy{e}{\xi_1}$} or \textcolor{yellow}{$\cmaysatisfy{e}{\xi_1}$} implies \textcolor{green}{$\csatisfy{e}{\xi_2}$}.
\end{defn}
}

\frame{
  \frametitle{Redundancy(must) Checking in Statics}
  \begin{align*}
    & \text{Branch } r & \text{Constraint } \xi\\
    ~\vert~ &
    \hrul{\hinlp{\htriv}}{\text{"empty"}} & \hinlp{\cunit} \\
    ~\vert~ &
    \hrul{\hinrp{\hpair{\hehole{w}}{xs}}}{\text{"empty"}} & \hinrp{\cpair{\cunknown}{\ctruth}} \\
    ~\vert~ &
    \hrul{\hinrp{\hpair{\hnum{1}}{\hinrp{\hpair{\hnum{2}}{xs}}}}}{\text{"empty"}} & \cinr{\cpair{\cnum{1}}{\cinr{\cpair{\cnum{2}}{\ctruth}}}} \\
  \end{align*}
\only<1>{
  \begin{itemize}
    \item the second branch is not redundant, $\cnotsatisfy{\hinrp{\cpair{\cunknown}{\ctruth}}}{\hinlp{\cunit}}$
  \item the third branch is not redundant, $\cnotsatisfy{\cinr{\cpair{\cnum{1}}{\cinr{\cpair{\cnum{2}}{\ctruth}}}}}{\cor{\hinlp{\cunit}}{\hinrp{\cpair{\cunknown}{\ctruth}}}}$
  \end{itemize}
}
\only<2>{
\begin{mathpar}
\Infer{\TRules}{
  \chrultyp{\Gamma}{\Delta}{r}{\tau_1}{\xi_r}{\tau_2} \\
  \chrulstyp{\Gamma}{\Delta}{\cor{\xi_{pre}}{\xi_r}}{rs}
  {\tau_1}{\xi_{rs}}{\tau_2} \\
  \boxed{\cnotsatisfy{\xi_r}{\xi_{pre}}}
}{
  \chrulstyp{\Gamma}{\Delta}{\xi_{pre}}{\hrules{r}{rs}}
  {\tau_1}{\cor{\xi_r}{\xi_{rs}}}{\tau_2}
}
\end{mathpar}
}
}

\frame{
  \frametitle{Exhaustiveness(must or maybe) Checking in Statics}
  \begin{align*}
    & \text{Branch } r & \text{Constraint } \xi\\
    ~\vert~ &
    \hrul{\hinlp{\htriv}}{\text{"empty"}} & \hinlp{\cunit} \\
    ~\vert~ &
    \hrul{\hinrp{\hpair{\hehole{w}}{xs}}}{\text{"empty"}} & \hinrp{\cpair{\cunknown}{\ctruth}} \\
    ~\vert~ &
    \hrul{\hinrp{\hpair{\hnum{1}}{\hinrp{\hpair{\hnum{2}}{xs}}}}}{\text{"empty"}} & \cinr{\cpair{\cnum{1}}{\cinr{\cpair{\cnum{2}}{\ctruth}}}} \\
  \end{align*}
\only<1>{
  \[
  \csatisfyormay{\ctruth}{
    \cor{\hinlp{\cunit}}{\cor{\hinrp{\cpair{\cunknown}{\ctruth}}}{\cinr{\cpair{\cnum{1}}{\cinr{\cpair{\cnum{2}}{\ctruth}}}}}}
  }
\]
}
\only<2>{
\begin{mathpar}
\Infer{\TMatchZPre}{
  \hexptyp{\Gamma}{\Delta}{e}{\tau_1} \\
  \chrulstyp{\Gamma}{\Delta}{\cfalsity}{\hrules{r}{rs}}{\tau_1}{\xi}{\tau_2} \\
  \boxed{\csatisfyormay{\ctruth}{\xi}}
}{
  \hexptyp{\Gamma}{\Delta}{\hmatch{e}{\zruls{\cdot}{r}{rs}}}{\tau_2}
}
\end{mathpar}
}
}

\subsection{Checking Algorithm}
\frame{
  \frametitle{De-unknown in Exhaustiveness Checking}
  \begin{center}
  \begin{tikzpicture}
    \node (Exh) at (0,-3) {$\csatisfyormay{\ctruth}{\cor{\hinlp{\cunit}}{\cor{\hinrp{\cpair{\cunknown}{\ctruth}}}{\cinr{\cpair{\cnum{1}}{\cinr{\cpair{\cnum{2}}{\ctruth}}}}}}}$};
    \node (De-Exh) at (0,-5) {$\csatisfy{\ctruth}{\cor{\hinlp{\cunit}}{\cor{\hinrp{\cpair{\ctruth}{\ctruth}}}{\cinr{\cpair{\cnum{1}}{\cinr{\cpair{\cnum{2}}{\ctruth}}}}}}}$};
    \node (De) at (0,-4) {$?::xs \longrightarrow x::xs$};

    \draw [color=hazellightgreen,line width=1,Latex-Latex,double] ($(Exh.west)+(0,0)$) to [out=200,in=160] (De-Exh.west);
  \end{tikzpicture}
  \end{center}
}

\frame{
  \frametitle{De-unknown in Redundancy Checking}
  Second branch:
  \begin{center}
  \begin{tikzpicture}
    \node (Red2) at (0,-3) {$\csatisfy{\hinrp{\cpair{\cunknown}{\ctruth}}}{\hinlp{\cunit}}$};
    \node (De-Red2) at (0,-4.5) {$\csatisfy{\hinrp{\cpair{\ctruth}{\ctruth}}}{\hinlp{\cunit}}$};
    \node (De) at (0,-3.75) {$?::xs \longrightarrow x::xs$};

    \draw [color=hazellightgreen,line width=1,Latex-Latex,double] ($(Red2.west)+(0,0)$) to [out=200,in=160] (De-Red2.west);
  \end{tikzpicture}
  \end{center}
  Third Branch:
  \begin{center}
  \begin{tikzpicture}
    \node (Red3) at (0,-6) {$\csatisfy{\cinr{\cpair{\cnum{1}}{\cinr{\cpair{\cnum{2}}{\ctruth}}}}}{\cor{\hinlp{\cunit}}{\hinrp{\cpair{\cunknown}{\ctruth}}}}$};
    \node (De-Red3) at (0,-7.5) {$\csatisfy{\cinr{\cpair{\cnum{1}}{\cinr{\cpair{\cnum{2}}{\ctruth}}}}}{\cor{\hinlp{\cunit}}{\hinrp{\cpair{\cfalsity}{\ctruth}}}}$};
    \node (De) at (0,-6.75) {$?::xs \longrightarrow 2::xs \text{, or } 3::xs \text{, or } ...$};
    \draw [color=hazellightgreen,line width=1,Latex-Latex,double] ($(Red3.west)+(0,0)$) to [out=200,in=160] (De-Red3.west);
  \end{tikzpicture}
  \end{center}
  Then, we can apply similar checking algorithm as described in Chapter $\textit{Pattern Matching}$ of PFPL \cite{Harper2012}.
}

\section{Results and Contribution}
\frame{
  \frametitle{Conclusion}
  \begin{itemize}
    \item Formalize pattern matching with typed holes in a type system
    \item Develop a theoretical foundation for constant feedback on match expressions
    \item Implement the type system in a toy programming language
    (\url{https://github.com/fplab/pattern-paper/tree/master/src})
  \end{itemize}
  \pause
  What is next?
  \begin{itemize}
    \item Prove the correctness of checking algorithm
    \item Integrate it into \textsf{Hazel}
  \end{itemize}
}

\section{Bibliography}
\frame{
  \frametitle{References}

\small{\bibliographystyle{apalike}
\bibliography{references}\vspace{1cm}}
}


%\end{darkframes} % Uncomment for dark theme

\end{document}
