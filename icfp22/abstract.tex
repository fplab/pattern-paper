\begin{abstract}
    Several programming systems, including GHC Haskell, Agda, Idris, and Hazel, now support \emph{typed holes}. 
    By assigning static and, to varying degree, dynamic meaning to programs with holes, editors and other tools can offer more meaningful  feedback and suggestions throughout the editing process, i.e. in a \emph{live} manner.
    % In these languages, empty holes serve as placeholders for missing program terms, while
    % non-empty holes operate as semantic membranes around erroneous terms, 
    % isolating them so that the remainder of the program remains statically meaningful,
    % and, in certain cases, even dynamically meaningful.
    Prior work, however, has considered only holes appearing in expressions and types. 
    This paper considers, from type theoretic and logical first principles,
    the problem of typed holes that appear in patterns.
    We confront two main difficulties, (1) statically reasoning about exhaustiveness and irredundancy 
    when patterns are not fully known, and (2) live evaluation of expressions
    containing both pattern and expression holes. 
    In both cases, this requires reasoning conservatively about all 
    possible hole fillings.
    We develop a typed lambda calculus, Peanut, 
    where reasoning about exhaustiveness and redundancy is mapped to the
    problem of deriving first order entailments between constraints. 
    We also develop an operational semantics in the style of Hazelnut Live that allows us 
    to evaluate around holes in both expressions and patterns.
    We mechanize the metatheory of Peanut in Agda and formalize a procedure capable of 
    deciding the necessary entailments.
    Finally, we scale up the core calculus to support finite labeled sums, adding support for \emph{datatype constructor holes} in the process, and integrate the mechanism into Hazel, a live functional 
    programming environment for a dialect of Elm that automatically inserts holes during editing to provide rich static and dynamic feedback to the programmer without gaps in service, i.e. for every possible editor state. 
    % Hazel is the first gap-free environment for a language supporting pattern matching. 
  \keywords{pattern matching \and typed holes \and live programming}
\end{abstract}