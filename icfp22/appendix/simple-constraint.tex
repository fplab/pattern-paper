\section{Incomplete Constraint Language}\label{sec:incompleteconstraint}

$\arraycolsep=4pt\begin{array}{lll}
\hxi & ::= &
  \ctruth ~\vert~
  \cfalsity ~\vert~
  \cunknown ~\vert~
  \cnum{n} ~\vert~
  \cinl{\hxi} ~\vert~
  \cinr{\hxi} ~\vert~
  \cpair{\hxi}{\hxi} ~\vert~
  \cor{\hxi}{\hxi}
\end{array}$

\judgboxa{\ctyp{\hxi}{\tau}}{$\hxi$ constrains final expressions of type $\tau$}
\begin{subequations}\label{rules:CTyp}
\begin{equation}\label{rule:CTTruth}
\inferrule[CTTruth]{ }{
  \ctyp{\ctruth}{\tau}
}
\end{equation}
\begin{equation}\label{rule:CTFalsity}
\inferrule[CTFalsity]{ }{
  \ctyp{\cfalsity}{\tau}
}
\end{equation}
\begin{equation}\label{rule:CTUnknown}
\inferrule[CTUnknown]{ }{
  \ctyp{\cunknown}{\tau}
}
\end{equation}
\begin{equation}\label{rule:CTNum}
\inferrule[CTNum]{ }{
  \ctyp{\cnum{n}}{\tnum}
}
\end{equation}
\begin{equation}\label{rule:CTInl}
\inferrule[CTInl]{
  \ctyp{\hxi_1}{\tau_1}
}{
  \ctyp{\cinl{\hxi_1}}{\tsum{\tau_1}{\tau_2}}
}
\end{equation}
\begin{equation}\label{rule:CTInr}
\inferrule[CTInr]{
  \ctyp{\hxi_2}{\tau_2}
}{
  \ctyp{\cinr{\hxi_2}}{\tsum{\tau_1}{\tau_2}}
}
\end{equation}
\begin{equation}\label{rule:CTPair}
\inferrule[CTPair]{
  \ctyp{\hxi_1}{\tau_1} \\ \ctyp{\hxi_2}{\tau_2}
}{
  \ctyp{\cpair{\hxi_1}{\hxi_2}}{\tprod{\tau_1}{\tau_2}}
}
\end{equation}
\begin{equation}\label{rule:CTOr}
\inferrule[CTOr]{
  \ctyp{\hxi_1}{\tau} \\ \ctyp{\hxi_2}{\tau}
}{
  \ctyp{\cor{\hxi_1}{\hxi_2}}{\tau}
}
\end{equation}
\end{subequations}

\judgboxa{
  \refutable{\hxi}
}{$\hxi$ is refutable}

\begin{subequations}\label{rules:xi-refutable}
\begin{equation}\label{rule:RXFalsity}
\inferrule[RXFalsity]{ }{
  \refutable{\cfalsity}
}
\end{equation}
\begin{equation}\label{rule:RXUnknown}
\inferrule[RXUnknown]{ }{
  \refutable{\cunknown}
}
\end{equation}
\begin{equation}\label{rule:RXNum}
\inferrule[RXNum]{ }{
  \refutable{\cnum{n}}
}
\end{equation}
\begin{equation}\label{rule:RXInl}
\inferrule[RXInl]{ }{
  \refutable{\cinl{\hxi}}
}
\end{equation}
\begin{equation}\label{rule:RXInr}
\inferrule[RXInr]{ }{
  \refutable{\cinr{\hxi}}
}
\end{equation}
\begin{equation}\label{rule:RXPairL}
\inferrule[RXPairL]{
  \refutable{\hxi_1}
}{
  \refutable{\cpair{\hxi_1}{\hxi_2}}
}
\end{equation}
\begin{equation}\label{rule:RXPairR}
\inferrule[RXPairR]{
  \refutable{\hxi_2}
}{
  \refutable{\cpair{\hxi_1}{\hxi_2}}
}
\end{equation}
\begin{equation}\label{rule:RXOr}
  \inferrule[RXOr]{
    \refutable{\hxi_1} \\
    \refutable{\hxi_2}
  }{
    \refutable{\cor{\hxi_1}{\hxi_2}}
  }
  \end{equation}
\end{subequations}

% \judgboxa{\frefutable{\hxi}}{}
% \begin{subequations}\label{defn:xi-refutable}
% \begin{align}
%     \frefutable{\ctruth} &= \false \\
%     \frefutable{\cfalsity} &= \true \\
%     \frefutable{\cunknown} &= \true \\
%     \frefutable{\cnum{n}} &= \true \\
%     \frefutable{\cinl{\hxi}} &= \true \\
%     \frefutable{\cinr{\hxi}} &= \true \\
%     \frefutable{\cpair{\hxi_1}{\hxi_2}} &= \frefutable{\hxi_1} \textor \frefutable{\hxi_2} \\
%     \frefutable{\cor{\hxi_1}{\hxi_2}} &= \frefutable{\hxi_1} \textand \frefutable{\hxi_2}
% \end{align}
% \end{subequations}

% \begin{lemma}[Soundness and Completeness of Refutable Constraints]
%   \label{lem:sound-complete-xi-refutable}
%   $\refutable{\hxi}$ iff $\frefutable{\hxi} = \true$.
% \end{lemma}

\judgboxa{
  \possible{\hxi}
}{$\hxi$ is possible}

\begin{subequations}\label{rules:xi-possible}
\begin{equation}\label{rule:PTruth}
\inferrule[PTruth]{ }{
  \possible{\ctruth}
}
\end{equation}
\begin{equation}\label{rule:PUnknown}
\inferrule[PUnknown]{ }{
  \possible{\cunknown}
}
\end{equation}
\begin{equation}\label{rule:PNum}
\inferrule[PNum]{ }{
  \possible{\cnum{n}}
}
\end{equation}
\begin{equation}\label{rule:PInl}
\inferrule[PInl]{ 
  \possible{\hxi} 
}{
  \possible{\cinl{\hxi}}
}
\end{equation}
\begin{equation}\label{rule:PInr}
\inferrule[PInr]{ 
  \possible{\hxi}
}{
  \possible{\cinr{\hxi}}
}
\end{equation}
\begin{equation}\label{rule:PPair}
\inferrule[PPair]{
  \possible{\hxi_1} \\ 
  \possible{\hxi_2}
}{
  \possible{\cpair{\hxi_1}{\hxi_2}}
}
\end{equation}
\begin{equation}\label{rule:POrL}
\inferrule[POrL]{
  \possible{\hxi_1}
}{
  \possible{\cor{\hxi_1}{\hxi_2}}
}
\end{equation}
\begin{equation}\label{rule:POrR}
\inferrule[POrR]{
  \possible{\hxi_2}
}{
  \possible{\cor{\hxi_1}{\hxi_2}}
}
\end{equation}
\end{subequations}

% \judgboxa{\fpossible{\hxi}}{}
% \begin{subequations}\label{defn:xi-possible}
% \begin{align}
%     \fpossible{\ctruth} &= \true \\
%     \fpossible{\cfalsity} &= \false \\
%     \fpossible{\cunknown} &= \true \\
%     \fpossible{\cnum{n}} &= \true \\
%     \fpossible{\cinl{\hxi}} &= \fpossible{\hxi} \\
%     \fpossible{\cinr{\hxi}} &= \fpossible{\hxi} \\
%     \fpossible{\cpair{\hxi_1}{\hxi_2}} &= \fpossible{\hxi_1} \textand \fpossible{\hxi_2} \\
%     \fpossible{\cor{\hxi_1}{\hxi_2}} &= \fpossible{\hxi_1} \textor \fpossible{\hxi_2}
% \end{align}
% \end{subequations}

% \begin{lemma}[Soundness and Completeness of Possible Constraints]
%   \label{lem:sound-complete-xi-possible}
%   $\possible{\hxi}$ iff $\fpossible{\hxi} = \true$.
% \end{lemma}

\judgboxa{\csatisfy{e}{\hxi}}{$e$ satisfies $\hxi$}
\begin{subequations}\label{rules:Satisfy}
\begin{equation}\label{rule:CSTruth}
\inferrule[CSTruth]{ }{
  \csatisfy{e}{\ctruth}
}
\end{equation}
\begin{equation}\label{rule:CSNum}
\inferrule[CSNum]{ }{
  \csatisfy{\hnum{n}}{\cnum{n}}
}
\end{equation}
\begin{equation}\label{rule:CSInl}
\inferrule[CSInl]{
  \csatisfy{e_1}{\hxi_1}
}{
  \csatisfy{
    \hinl{\tau_2}{e_1}
  }{
    \cinl{\hxi_1}
  }
}
\end{equation}
\begin{equation}\label{rule:CSInr}
\inferrule[CSInr]{
  \csatisfy{e_2}{\hxi_2}
}{
  \csatisfy{
    \hinr{\tau_1}{e_2}
  }{
    \cinr{\hxi_2}
  }
}
\end{equation}
\begin{equation}\label{rule:CSPair}
\inferrule[CSPair]{
  \csatisfy{e_1}{\hxi_1} \\
  \csatisfy{e_2}{\hxi_2}
}{
\csatisfy{\hpair{e_1}{e_2}}{\cpair{\hxi_1}{\hxi_2}}
}
\end{equation}
\begin{equation}\label{rule:CSNotIntroPair}
\inferrule[CSNotIntroPair]{
  \notIntro{e} \\
  \csatisfy{\hfst{e}}{\hxi_1} \\
  \csatisfy{\hsnd{e}}{\hxi_2}
}{
  \csatisfy{e}{\cpair{\hxi_1}{\hxi_2}}
}
\end{equation}
\begin{equation}\label{rule:CSOr1}
\inferrule[CSOrL]{
  \csatisfy{e}{\hxi_1}
}{
  \csatisfy{e}{\cor{\hxi_1}{\hxi_2}}
}
\end{equation}
\begin{equation}\label{rule:CSOr2}
\inferrule[CSOrR]{
  \csatisfy{e}{\hxi_2}
}{
  \csatisfy{e}{\cor{\hxi_1}{\hxi_2}}
}
\end{equation}
\end{subequations}

% \judgboxa{\fsatisfy{e}{\hxi}}{}
% \begin{subequations}\label{defn:satisfy}
% \begin{align}
%   \fsatisfy{e}{\ctruth} ={}& \true \label{defn:satisfy-truth}\\
%   \fsatisfy{\hnum{n_1}}{\cnum{n_2}} ={}& (n_1 = n_2) \label{defn:num-satisfy-num}\\
%   \fsatisfy{e}{\cor{\hxi_1}{\hxi_2}} ={}& \fsatisfy{e}{\hxi_1} \textor \fsatisfy{e}{\hxi_2} \label{defn:satisfy-or}\\
%   \fsatisfy{\hinl{\tau_2}{e_1}}{\cinl{\hxi_1}} ={}& \fsatisfy{e_1}{\hxi_1} \label{defn:inl-satisfy-inl}\\
%   \fsatisfy{\hinr{\tau_1}{e_2}}{\cinr{\hxi_2}} ={}& \fsatisfy{e_2}{\hxi_2} \label{defn:inr-satisfy-inr}\\
%   \fsatisfy{\hpair{e_1}{e_2}}{\cpair{\hxi_1}{\hxi_2}} ={}& \fsatisfy{e_1}{\hxi_1} \textand \fsatisfy{e_2}{\hxi_2} \label{defn:pair-satisfy-pair}\\
%   \text{Otherwise}\quad \fsatisfy{e}{\cpair{\hxi_1}{\hxi_2}} ={}& \fnotIntro{e} \textand \fsatisfy{\hfst{e}}{\hxi_1} \notag\\
%   &\textand \fsatisfy{\hsnd{e}}{\hxi_2} \label{defn:notintro-satisfy-pair}\\
%   \fsatisfy{e}{\hxi} ={}& \false \label{defn:not-satisfy}
% \end{align}
% \end{subequations}

% \begin{lemma}[Soundness and Completeness of Satisfaction Judgment]
%   \label{lem:sound-complete-satisfy}
%   $\csatisfy{e}{\hxi}$ iff $\fsatisfy{e}{\hxi} = \true$.
% \end{lemma}

\judgboxa{\cmaysatisfy{e}{\hxi}}{$e$ may satisfy $\hxi$}
\begin{subequations}\label{rules:MaySatisfy}
\begin{equation}\label{rule:CMSUnknown}
\inferrule[CMSUnknown]{ }{
  \cmaysatisfy{e}{\cunknown}
}
\end{equation}
\begin{equation}\label{rule:CMSInl}
\inferrule[CMSInl]{
  \cmaysatisfy{e_1}{\hxi_1}
}{
  \cmaysatisfy{
    \hinl{\tau_2}{e_1}
  }{
    \cinl{\hxi_1}
  }
}
\end{equation}
\begin{equation}\label{rule:CMSInr}
\inferrule[CMSInr]{
  \cmaysatisfy{e_2}{\hxi_2}
}{
  \cmaysatisfy{
    \hinr{\tau_1}{e_2}
  }{
    \cinr{\hxi_2}
  }
}
\end{equation}
\begin{equation}\label{rule:CMSPair1}
\inferrule[CMSPairL]{
  \cmaysatisfy{e_1}{\hxi_1} \\
  \csatisfy{e_2}{\hxi_2}
}{
  \cmaysatisfy{\hpair{e_1}{e_2}}{\cpair{\hxi_1}{\hxi_2}}
}
\end{equation}
\begin{equation}\label{rule:CMSPair2}
\inferrule[CMSPairR]{
  \csatisfy{e_1}{\hxi_1} \\
  \cmaysatisfy{e_2}{\hxi_2}
}{
  \cmaysatisfy{\hpair{e_1}{e_2}}{\cpair{\hxi_1}{\hxi_2}}
}
\end{equation}
\begin{equation}\label{rule:CMSPair3}
\inferrule[CMSPair]{
  \cmaysatisfy{e_1}{\hxi_1} \\
  \cmaysatisfy{e_2}{\hxi_2}
}{
  \cmaysatisfy{\hpair{e_1}{e_2}}{\cpair{\hxi_1}{\hxi_2}}
}
\end{equation}
\begin{equation}\label{rule:CMSOr1}
\inferrule[CMSOrL]{
  \cmaysatisfy{e}{\hxi_1} \\
  \cnotsatisfy{e}{\hxi_2}
}{
  \cmaysatisfy{e}{\cor{\hxi_1}{\hxi_2}}
}
\end{equation}
\begin{equation}\label{rule:CMSOr2}
\inferrule[CMSOrR]{
  \cnotsatisfy{e}{\hxi_1} \\
  \cmaysatisfy{e}{\hxi_2}
}{
  \cmaysatisfy{e}{\cor{\hxi_1}{\hxi_2}}
}
\end{equation}
\begin{equation}\label{rule:CMSNotIntro}
\inferrule[CMSNotIntro]{
  \notIntro{e} \\
  \refutable{\hxi} \\
  \possible{\hxi}
}{
  \cmaysatisfy{e}{\hxi}
}
\end{equation}
\end{subequations}

% \judgboxa{\fmaysatisfy{e}{\hxi}}{}
% \begin{subequations}\label{defn:maysatisfy}
%   \begin{align}
%     \fmaysatisfy{e}{\cunknown} ={}& \true \label{defn:maysat-unknown}\\
%     \fmaysatisfy{\hinl{\tau_2}{e_1}}{\cinl{\hxi_1}} ={}& \fmaysatisfy{e_1}{\hxi_1} \label{defn:maysat-inl}\\
%     \fmaysatisfy{\hinr{\tau_1}{e_2}}{\cinr{\hxi_2}} ={}& \fmaysatisfy{e_2}{\hxi_2} \label{defn:maysat-inr}\\
%     \fmaysatisfy{\hinl{\tau_2}{e_1}}{\cinr{\hxi_2}} ={}& \false \label{defn:not-inl-maysat-inr}\\
%     \fmaysatisfy{\hinr{\tau_1}{e_2}}{\cinl{\hxi_1}} ={}& \false \label{defn:not-inr-maysat-inl}\\
%     \fmaysatisfy{\hpair{e_1}{e_2}}{\cpair{\hxi_1}{\hxi_2}} ={}& \left(\fmaysatisfy{e_1}{\hxi_1} \textand \fsatisfy{e_2}{\hxi_2}\right) \notag\\
%     &\textor \left(\fsatisfy{e_1}{\hxi_1} \textand \fmaysatisfy{e_2}{\hxi_2}\right) \notag\\
%     &\textor \left(\fmaysatisfy{e_1}{\hxi_1} \textand \fmaysatisfy{e_2}{\hxi_2}\right) \label{defn:maysat-pair}\\
%     \fmaysatisfy{e}{\cor{\hxi_1}{\hxi_2}} ={}& \left(\fmaysatisfy{e}{\hxi_1} \textand \left(\textnot \fsatisfy{e}{\hxi_2}\right)\right) \notag\\
%     &\textor \left(\left(\textnot \fsatisfy{e}{\hxi_1}\right) \textand \fmaysatisfy{e}{\hxi_2}\right) \label{defn:maysat-or}\\
%     \fmaysatisfy{e}{\hxi} =& \fnotIntro{e} \textand \frefutable{\hxi} \textand \fpossible{\hxi} \label{defn:notintro-maysat}
%   \end{align}
% \end{subequations}

% \begin{lemma}[Soundness and Completeness of Maybe Satisfaction Judgment]
%   \label{lem:sound-complete-maysatisfy}
%   $\cmaysatisfy{e}{\hxi}$ iff $\fmaysatisfy{e}{\hxi} = \true$.
% \end{lemma}

\judgboxa{\csatisfyormay{e}{\hxi}}{$e$ satisfies or may satisfy $\hxi$}
\begin{subequations}\label{rules:satormay}
\begin{equation}\label{rule:CSMSMay}
\inferrule[CSMSMay]{
  \cmaysatisfy{e}{\hxi}
}{
  \csatisfyormay{e}{\hxi}
}
\end{equation}
\begin{equation}\label{rule:CSMSSat}
\inferrule[CSMSSat]{
  \csatisfy{e}{\hxi}
}{
  \csatisfyormay{e}{\hxi}
}
\end{equation}
\end{subequations}

% \judgboxa{\fsatisfyormay{e}{\hxi}}{}
% \begin{equation}\label{defn:satormay}
%   \fsatisfyormay{e}{\hxi} = \fsatisfy{e}{\hxi} \textor \fmaysatisfy{e}{\hxi}
% \end{equation}

% \begin{lemma}[Soundness and Completeness of Satisfaction or Maybe Satisfaction]
%   \label{lem:sound-complete-satormay}
%   $\csatisfyormay{e}{\hxi}$ iff $\fsatisfyormay{e}{\hxi}$.
% \end{lemma}

\begin{lemma}
  \label{lem:notintro-maysat-or-notsat-refutable}
  Assume $\notIntro{e}$. If $\cmaysatisfy{e}{\hxi}$ or $\cnotsatisfy{e}{\hxi}$ then $\refutable{\hxi}$.
\end{lemma}

\begin{lemma}
\label{lem:satisfy-not-refutable}
If $\notIntro{e}$ and $\csatisfy{e}{\hxi}$ then $\cancel{\refutable{\hxi}}$.
\end{lemma}

\begin{theorem}[Exclusiveness of Satisfaction Judgment]
  \label{thrm:exclusive-constraint-satisfaction}
  If $\ctyp{\hxi}{\tau}$ and $\hexptyp{\cdot}{\Delta}{e}{\tau}$ and $\isFinal{e}$ then exactly one of the following holds
  \begin{enumerate}
    \item $\csatisfy{e}{\hxi}$
    \item $\cmaysatisfy{e}{\hxi}$
    \item $\cnotsatisfyormay{e}{\hxi}$
  \end{enumerate}
\end{theorem}

\begin{definition}[Entailment of Constraints]
  \label{defn:const-entailment}
  Suppose that $\ctyp{\hxi_1}{\tau}$ and $\ctyp{\hxi_2}{\tau}$.
  Then $\csatisfy{\hxi_1}{\hxi_2}$ iff for all $e$ such that $\hexptyp{\cdot}{\Delta}{e}{\tau}$ and $\isVal{e}$ we have $\csatisfyormay{e}{\hxi_1}$ implies $\csatisfy{e}{\hxi_2}$
\end{definition}

\begin{definition}[Potential Entailment of Constraints]
  \label{defn:nn-entailment}
  Suppose that $\ctyp{\hxi_1}{\tau}$ and $\ctyp{\hxi_2}{\tau}$. Then $\csatisfyormay{\hxi_1}{\hxi_2}$ iff for all $e$ such that $\hexptyp{\cdot}{\Delta}{e}{\tau}$ and $\isFinal{e}$ we have $\csatisfyormay{e}{\hxi_1}$ implies $\csatisfyormay{e}{\hxi_2}$ 
\end{definition}

\begin{corollary}
  \label{corol:nn-exhaust}
  Suppose that $\ctyp{\hxi}{\tau}$ and $\hexptyp{\cdot}{\Delta}{e}{\tau}$ and $\isFinal{e}$. Then $\csatisfyormay{\ctruth}{\hxi}$ implies $\csatisfyormay{e}{\hxi}$
\end{corollary}

%%% Local Variables:
%%% mode: latex
%%% TeX-master: "appendix"
%%% TeX-master: "appendix"
%%% End:
