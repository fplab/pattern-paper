\RequirePackage{etex}
\documentclass{article}

\usepackage[T1]{fontenc} % fix missing font cmtt
\usepackage{amsmath}
\usepackage{amssymb} % Vdash
\usepackage{amsthm} % proof
\usepackage{graphicx} % rotatebox
\usepackage{stmaryrd} % llparenthesis
\usepackage{anyfontsize} % workaround for font size difference warning
\usepackage{todonotes}
\usepackage{listings}
\usepackage{tikz}
\usetikzlibrary{calc,fit,tikzmark,plotmarks,arrows.meta,positioning,overlay-beamer-styles}
\usepackage[caption=false]{subfig}


\usepackage{cancel} % slash over symbol
\usepackage{hyperref}
\renewcommand\UrlFont{\color{blue}\rmfamily}
\let\figureautorefname\figurename
\def\sectionautorefname{Sec.}
\def\lemmaautorefname{Lemma}
\def\equationautorefname{Definition}
\def\definitionautorefname{Definition}
\def\theoremautorefname{Theorem}
\def\corollaryautorefname{Corollary}
\let\subsectionautorefname\sectionautorefname
\let\subsubsectionautorefname\sectionautorefname
\newcommand{\rulesref}[1]{Rules (\ref{#1})}
\newcommand{\ruleref}[1]{Rule (\ref{#1})}

\usepackage{xcolor}
\definecolor{hazelgreen}{RGB}{7,63,36}
\definecolor{hazellightgreen}{RGB}{103,138,97}
\definecolor{hazelyellow}{RGB}{245,222,179}
\definecolor{hazellightyellow}{RGB}{254,254,234}

\newcommand{\highlight}[1]{\colorbox{yellow}{$\displaystyle #1$}}

%% Joshua Dunfield macros
\def\OPTIONConf{1}
\usepackage{jdunfield}
\usepackage{pfsteps}
\makeatletter
\newcommand{\savelocalsteps}[1]{
  \@ifundefined{c@#1}
    {% the counter doesn't exist
     \newcounter{#1}
   }{}
  \setcounter{#1}{\value{pfsteps@pfc@local}}
}
\makeatother
\newcommand{\restorelocalsteps}[1]{\setcounter{pfsteps@pfc@local}{\value{#1}}}

\newtheorem{theorem}{Theorem}[section]
\newtheorem{corollary}{Corollary}[theorem]
\newtheorem{lemma}{Lemma}[theorem]
\newtheorem{definition}{Definition}[theorem]

% !TEX root = ./patterns-paper.tex

% reverse Vdash
\newcommand{\dashV}{\mathbin{\rotatebox[origin=c]{180}{$\Vdash$}}}

% Violet hotdogs; highlight color helps distinguish them
\newcommand{\llparenthesiscolor}{\textcolor{violet}{\llparenthesis}}
\newcommand{\rrparenthesiscolor}{\textcolor{violet}{\rrparenthesis}}

% HTyp and HExp
\newcommand{\hcomplete}[1]{#1~\mathsf{complete}}

% HTyp
\newcommand{\htau}{\dot{\tau}}
\newcommand{\tarr}[2]{\inparens{#1 \rightarrow #2}}
\newcommand{\tarrnp}[2]{#1 \rightarrow #2}
\newcommand{\trul}[2]{\inparens{#1 \Rightarrow #2}}
\newcommand{\tnum}{\mathtt{num}}
\newcommand{\tehole}{\llparenthesiscolor\rrparenthesiscolor}
\newcommand{\tsum}[2]{\inparens{{#1} + {#2}}}
\newcommand{\tprod}[2]{\inparens{{#1} \times {#2}}}
\newcommand{\tunit}{\mathtt{1}}
\newcommand{\tvoid}{\mathtt{0}}

\newcommand{\tcompat}[2]{#1 \sim #2}
\newcommand{\tincompat}[2]{#1 \nsim #2}

% HExp
\newcommand{\hexp}{\dot{e}}
\newcommand{\hlam}[3]{\inparens{\lambda #1:#2.#3}}
\newcommand{\hap}[2]{#1(#2)}
\newcommand{\hapP}[2]{(#1)~(#2)} % Extra paren around function term
\newcommand{\hnum}[1]{\underline{#1}}
\newcommand{\hadd}[2]{\inparens{#1 + #2}}
\newcommand{\hpair}[2]{\inparens{#1 , #2}}
\newcommand{\htriv}{()}
\newcommand{\hehole}{\llparenthesiscolor\rrparenthesiscolor}
\newcommand{\hhole}[1]{\llparenthesiscolor#1\rrparenthesiscolor}
\newcommand{\hindet}[1]{\lceil#1\rceil}
\newcommand{\hinj}[2]{\mathtt{inj}_{#1}({#2})}
\newcommand{\hinl}[2]{\mathtt{inl}_{#1}({#2})}
\newcommand{\hinr}[2]{\mathtt{inr}_{#1}({#2})}
\newcommand{\hinlp}[1]{\mathtt{inl}(#1)}
\newcommand{\hinrp}[1]{\mathtt{inr}(#1)}
\newcommand{\hmatch}[2]{\mathtt{match}(#1) \{#2\}}
\newcommand{\hcase}[5]{\mathtt{case}({#1},{#2}.{#3},{#4}.{#5})}
\newcommand{\hrules}[2]{\inparens{#1 \mid #2}}
\newcommand{\hrul}[2]{#1 \Rightarrow #2}

\newcommand{\hGamma}{\dot{\Gamma}}
\newcommand{\domof}[1]{\text{dom}(#1)}
\newcommand{\hsyn}[3]{#1 \vdash #2 \Rightarrow #3}
\newcommand{\hana}[3]{#1 \vdash #2 \Leftarrow #3}
\newcommand{\hexptyp}[3]{#1 \vdash #2 : #3}
\newcommand{\hpattyp}[3]{#1 : #2 \dashV #3}
\newcommand{\hpatmatch}[3]{#1 \vartriangleright #2 \dashV #3}
\newcommand{\hval}[1]{#1 ~\mathtt{val}}
\newcommand{\herr}[1]{#1 ~\mathtt{err}}

% ZTyp and ZExp
\newcommand{\zlsel}[1]{{\bowtie}{#1}}
\newcommand{\zrsel}[1]{{#1}{\bowtie}}
\newcommand{\zwsel}[1]{
  \setlength{\fboxsep}{0pt}
  \colorbox{green!10!white!100}{
    \ensuremath{{{\textcolor{Green}{{\hspace{-2px}\triangleright}}}}{#1}{\textcolor{Green}{\triangleleft{\vphantom{\tehole}}}}}}
}

\newcommand{\removeSel}[1]{#1^{\diamond}}

% ZTyp
\newcommand{\ztau}{\hat{\tau}}

% ZExp
\newcommand{\zexp}{\hat{e}}

% Direction
\newcommand{\dParent}{\mathtt{parent}}
\newcommand{\dChildn}[1]{\mathtt{child}~\mathtt{{#1}}}
\newcommand{\dChildnm}[1]{\mathtt{child}~{#1}}

% Action
\newcommand{\aMove}[1]{\mathtt{move}~#1}
	\newcommand{\zrightmost}[1]{\mathsf{rightmost}(#1)}
	\newcommand{\zleftmost}[1]{\mathsf{leftmost}(#1)}
\newcommand{\aSelect}[1]{\mathtt{sel}~#1}
\newcommand{\aDel}{\mathtt{del}}
\newcommand{\aReplace}[1]{\mathtt{replace}~#1}
\newcommand{\aConstruct}[1]{\mathtt{construct}~#1}
\newcommand{\aConstructx}[1]{#1}
\newcommand{\aFinish}{\mathtt{finish}}

\newcommand{\performAna}[5]{#1 \vdash #2 \xlongrightarrow{#4} #5 \Leftarrow #3}
\newcommand{\performAnaI}[5]{#1 \vdash #2 \xlongrightarrow{#4}\hspace{-3px}{}^{*}~ #5 \Leftarrow #3}
\newcommand{\performSyn}[6]{#1 \vdash #2 \Rightarrow #3 \xlongrightarrow{#4} #5 \Rightarrow #6}
\newcommand{\performSynI}[6]{#1 \vdash #2 \Rightarrow #3 \xlongrightarrow{#4}\hspace{-3px}{}^{*}~ #5 \Rightarrow #6}
\newcommand{\performTyp}[3]{#1 \xlongrightarrow{#2} #3}
\newcommand{\performTypI}[3]{#1 \xlongrightarrow{#2}\hspace{-3px}{}^{*}~#3}

\newcommand{\performMove}[3]{#1 \xlongrightarrow{#2} #3}
\newcommand{\performDel}[2]{#1 \xlongrightarrow{\aDel} #2}

% Form
\newcommand{\farr}{\mathtt{arrow}}
\newcommand{\fnum}{\mathtt{num}}
\newcommand{\fsum}{\mathtt{sum}}

\newcommand{\fasc}{\mathtt{asc}}
\newcommand{\fvar}[1]{\mathtt{var}~#1}
\newcommand{\flam}[1]{\mathtt{lam}~#1}
\newcommand{\fap}{\mathtt{ap}}
% \newcommand{\farg}{\mathtt{arg}}
\newcommand{\fnumlit}[1]{\mathtt{lit}~#1}
\newcommand{\fplus}{\mathtt{plus}}
\newcommand{\fhole}{\mathtt{hole}}
\newcommand{\fnehole}{\mathtt{nehole}}

\newcommand{\finj}[1]{\mathtt{inj}~#1}
\newcommand{\fcase}[2]{\mathtt{case}~#1~#2}

% Talk about formal rules in example
\newcommand{\refrule}[1]{\textrm{Rule~(#1)}}

\newcommand{\herase}[1]{\left|#1\right|_\textsf{erase}}

\newcommand{\arrmatch}[2]{#1 \blacktriangleright_{\rightarrow} #2}


\newcommand{\TABperformAna}[5]{#1 \vdash & #2                & \xlongrightarrow{#4} & #5 & \Leftarrow #3}
\newcommand{\TABperformSyn}[6]{#1 \vdash & #2 \Rightarrow #3 & \xlongrightarrow{#4} & #5 \Rightarrow #6}
\newcommand{\TABperformTyp}[3]{& #1 & \xlongrightarrow{#2} & #3}

\newcommand{\TABperformMove}[3]{#1 & \xlongrightarrow{#2} & #3}
\newcommand{\TABperformDel}[2]{#1 \xlongrightarrow{\aDel} #2}

\newcommand{\sumhasmatched}[2]{#1 \mathrel{\textcolor{black}{\blacktriangleright_{+}}} #2}

\newcommand{\subminsyn}[1]{\mathsf{submin}_{\Rightarrow}(#1)}
\newcommand{\subminana}[1]{\mathsf{submin}_{\Leftarrow}(#1)}


\newcommand{\inparens}[1]{{\color{gray}(}#1{\color{gray})}}

%% rule names for appendix
\newcommand{\rname}[1]{\textsc{#1}}
\newcommand{\gap}{\vspace{7pt}}


\usepackage{comment}
\excludecomment{proof}
\begin{document}

In the main paper, we present only a single constraint language. However, conceptually, we work with this language in two distinct stages: first, the constraints directly emitted by lists of rules, then, for use in redundancy and exhaustiveness checking, the constraints which are in the image of the truify and falsify functions and their duals. While irrelevant to the overall theory, to simplify some proofs, it is useful to make this distinction explicit.

In \autoref{sec:incompleteconstraint}, we present the first stage of constraints, called the \emph{incomplete constraint language}. This consists of any constraint emitted by a pattern, and in particular, includes the $\cunknown$ constraint. In order to define the constraint emitted by a list of rules, we also include $\cfalsity$ and allow taking the $\vee$ of incomplete constraints. At this stage, we often require two versions of each judgement: one describing a determinate result, and one describing a result which is indeterminate due to the presence of the $\cunknown$ constraint.

In turn, in \autoref{sec:completeconstraint}, we discuss those constraints in the image of the truify and falsify functions, as well as their duals. We call this the \emph{complete constraint language}, and it includes almost all of the incomplete language, but excludes the $\cunknown$ constraint. To support the dual operation, we also may take the $\wedge$ of complete constraints, and we add a $\cnotnum{n}$ constraint. Due to the absence of $\cunknown$, judgements related to the complete language do not have to consider indeterminacy, and thus are often simpler than their counterparts in the incomplete language. This is the primary motivation for distinguishing these languages at all.

\section{Incomplete Constraint Language}\label{sec:incompleteconstraint}

$\arraycolsep=4pt\begin{array}{lll}
\hxi & ::= &
  \ctruth ~\vert~
  \cfalsity ~\vert~
  \cunknown ~\vert~
  \cnum{n} ~\vert~
  \cinl{\hxi} ~\vert~
  \cinr{\hxi} ~\vert~
  \cpair{\hxi}{\hxi} ~\vert~
  \cor{\hxi}{\hxi}
\end{array}$

\judgboxa{\ctyp{\hxi}{\tau}}{$\hxi$ constrains final expressions of type $\tau$}
\begin{subequations}\label{rules:CTyp}
\begin{equation}\label{rule:CTTruth}
\inferrule[CTTruth]{ }{
  \ctyp{\ctruth}{\tau}
}
\end{equation}
\begin{equation}\label{rule:CTFalsity}
\inferrule[CTFalsity]{ }{
  \ctyp{\cfalsity}{\tau}
}
\end{equation}
\begin{equation}\label{rule:CTUnknown}
\inferrule[CTUnknown]{ }{
  \ctyp{\cunknown}{\tau}
}
\end{equation}
\begin{equation}\label{rule:CTNum}
\inferrule[CTNum]{ }{
  \ctyp{\cnum{n}}{\tnum}
}
\end{equation}
\begin{equation}\label{rule:CTInl}
\inferrule[CTInl]{
  \ctyp{\hxi_1}{\tau_1}
}{
  \ctyp{\cinl{\hxi_1}}{\tsum{\tau_1}{\tau_2}}
}
\end{equation}
\begin{equation}\label{rule:CTInr}
\inferrule[CTInr]{
  \ctyp{\hxi_2}{\tau_2}
}{
  \ctyp{\cinr{\hxi_2}}{\tsum{\tau_1}{\tau_2}}
}
\end{equation}
\begin{equation}\label{rule:CTPair}
\inferrule[CTPair]{
  \ctyp{\hxi_1}{\tau_1} \\ \ctyp{\hxi_2}{\tau_2}
}{
  \ctyp{\cpair{\hxi_1}{\hxi_2}}{\tprod{\tau_1}{\tau_2}}
}
\end{equation}
\begin{equation}\label{rule:CTOr}
\inferrule[CTOr]{
  \ctyp{\hxi_1}{\tau} \\ \ctyp{\hxi_2}{\tau}
}{
  \ctyp{\cor{\hxi_1}{\hxi_2}}{\tau}
}
\end{equation}
\end{subequations}

\judgboxa{
  \refutable{\hxi}
}{$\hxi$ is refutable}

\begin{subequations}\label{rules:xi-refutable}
\begin{equation}\label{rule:RXFalsity}
\inferrule[RXFalsity]{ }{
  \refutable{\cfalsity}
}
\end{equation}
\begin{equation}\label{rule:RXUnknown}
\inferrule[RXUnknown]{ }{
  \refutable{\cunknown}
}
\end{equation}
\begin{equation}\label{rule:RXNum}
\inferrule[RXNum]{ }{
  \refutable{\cnum{n}}
}
\end{equation}
\begin{equation}\label{rule:RXInl}
\inferrule[RXInl]{ }{
  \refutable{\cinl{\hxi}}
}
\end{equation}
\begin{equation}\label{rule:RXInr}
\inferrule[RXInr]{ }{
  \refutable{\cinr{\hxi}}
}
\end{equation}
\begin{equation}\label{rule:RXPairL}
\inferrule[RXPairL]{
  \refutable{\hxi_1}
}{
  \refutable{\cpair{\hxi_1}{\hxi_2}}
}
\end{equation}
\begin{equation}\label{rule:RXPairR}
\inferrule[RXPairR]{
  \refutable{\hxi_2}
}{
  \refutable{\cpair{\hxi_1}{\hxi_2}}
}
\end{equation}
\begin{equation}\label{rule:RXOr}
  \inferrule[RXOr]{
    \refutable{\hxi_1} \\
    \refutable{\hxi_2}
  }{
    \refutable{\cor{\hxi_1}{\hxi_2}}
  }
  \end{equation}
\end{subequations}

% \judgboxa{\frefutable{\hxi}}{}
% \begin{subequations}\label{defn:xi-refutable}
% \begin{align}
%     \frefutable{\ctruth} &= \false \\
%     \frefutable{\cfalsity} &= \true \\
%     \frefutable{\cunknown} &= \true \\
%     \frefutable{\cnum{n}} &= \true \\
%     \frefutable{\cinl{\hxi}} &= \true \\
%     \frefutable{\cinr{\hxi}} &= \true \\
%     \frefutable{\cpair{\hxi_1}{\hxi_2}} &= \frefutable{\hxi_1} \textor \frefutable{\hxi_2} \\
%     \frefutable{\cor{\hxi_1}{\hxi_2}} &= \frefutable{\hxi_1} \textand \frefutable{\hxi_2}
% \end{align}
% \end{subequations}

% \begin{lemma}[Soundness and Completeness of Refutable Constraints]
%   \label{lem:sound-complete-xi-refutable}
%   $\refutable{\hxi}$ iff $\frefutable{\hxi} = \true$.
% \end{lemma}

\judgboxa{
  \possible{\hxi}
}{$\hxi$ is possible}

\begin{subequations}\label{rules:xi-possible}
\begin{equation}\label{rule:PTruth}
\inferrule[PTruth]{ }{
  \possible{\ctruth}
}
\end{equation}
\begin{equation}\label{rule:PUnknown}
\inferrule[PUnknown]{ }{
  \possible{\cunknown}
}
\end{equation}
\begin{equation}\label{rule:PNum}
\inferrule[PNum]{ }{
  \possible{\cnum{n}}
}
\end{equation}
\begin{equation}\label{rule:PInl}
\inferrule[PInl]{ 
  \possible{\hxi} 
}{
  \possible{\cinl{\hxi}}
}
\end{equation}
\begin{equation}\label{rule:PInr}
\inferrule[PInr]{ 
  \possible{\hxi}
}{
  \possible{\cinr{\hxi}}
}
\end{equation}
\begin{equation}\label{rule:PPair}
\inferrule[PPair]{
  \possible{\hxi_1} \\ 
  \possible{\hxi_2}
}{
  \possible{\cpair{\hxi_1}{\hxi_2}}
}
\end{equation}
\begin{equation}\label{rule:POrL}
\inferrule[POrL]{
  \possible{\hxi_1}
}{
  \possible{\cor{\hxi_1}{\hxi_2}}
}
\end{equation}
\begin{equation}\label{rule:POrR}
\inferrule[POrR]{
  \possible{\hxi_2}
}{
  \possible{\cor{\hxi_1}{\hxi_2}}
}
\end{equation}
\end{subequations}

% \judgboxa{\fpossible{\hxi}}{}
% \begin{subequations}\label{defn:xi-possible}
% \begin{align}
%     \fpossible{\ctruth} &= \true \\
%     \fpossible{\cfalsity} &= \false \\
%     \fpossible{\cunknown} &= \true \\
%     \fpossible{\cnum{n}} &= \true \\
%     \fpossible{\cinl{\hxi}} &= \fpossible{\hxi} \\
%     \fpossible{\cinr{\hxi}} &= \fpossible{\hxi} \\
%     \fpossible{\cpair{\hxi_1}{\hxi_2}} &= \fpossible{\hxi_1} \textand \fpossible{\hxi_2} \\
%     \fpossible{\cor{\hxi_1}{\hxi_2}} &= \fpossible{\hxi_1} \textor \fpossible{\hxi_2}
% \end{align}
% \end{subequations}

% \begin{lemma}[Soundness and Completeness of Possible Constraints]
%   \label{lem:sound-complete-xi-possible}
%   $\possible{\hxi}$ iff $\fpossible{\hxi} = \true$.
% \end{lemma}

\judgboxa{\csatisfy{e}{\hxi}}{$e$ satisfies $\hxi$}
\begin{subequations}\label{rules:Satisfy}
\begin{equation}\label{rule:CSTruth}
\inferrule[CSTruth]{ }{
  \csatisfy{e}{\ctruth}
}
\end{equation}
\begin{equation}\label{rule:CSNum}
\inferrule[CSNum]{ }{
  \csatisfy{\hnum{n}}{\cnum{n}}
}
\end{equation}
\begin{equation}\label{rule:CSInl}
\inferrule[CSInl]{
  \csatisfy{e_1}{\hxi_1}
}{
  \csatisfy{
    \hinl{\tau_2}{e_1}
  }{
    \cinl{\hxi_1}
  }
}
\end{equation}
\begin{equation}\label{rule:CSInr}
\inferrule[CSInr]{
  \csatisfy{e_2}{\hxi_2}
}{
  \csatisfy{
    \hinr{\tau_1}{e_2}
  }{
    \cinr{\hxi_2}
  }
}
\end{equation}
\begin{equation}\label{rule:CSPair}
\inferrule[CSPair]{
  \csatisfy{e_1}{\hxi_1} \\
  \csatisfy{e_2}{\hxi_2}
}{
\csatisfy{\hpair{e_1}{e_2}}{\cpair{\hxi_1}{\hxi_2}}
}
\end{equation}
\begin{equation}\label{rule:CSNotIntroPair}
\inferrule[CSNotIntroPair]{
  \notIntro{e} \\
  \csatisfy{\hfst{e}}{\hxi_1} \\
  \csatisfy{\hsnd{e}}{\hxi_2}
}{
  \csatisfy{e}{\cpair{\hxi_1}{\hxi_2}}
}
\end{equation}
\begin{equation}\label{rule:CSOr1}
\inferrule[CSOrL]{
  \csatisfy{e}{\hxi_1}
}{
  \csatisfy{e}{\cor{\hxi_1}{\hxi_2}}
}
\end{equation}
\begin{equation}\label{rule:CSOr2}
\inferrule[CSOrR]{
  \csatisfy{e}{\hxi_2}
}{
  \csatisfy{e}{\cor{\hxi_1}{\hxi_2}}
}
\end{equation}
\end{subequations}

% \judgboxa{\fsatisfy{e}{\hxi}}{}
% \begin{subequations}\label{defn:satisfy}
% \begin{align}
%   \fsatisfy{e}{\ctruth} ={}& \true \label{defn:satisfy-truth}\\
%   \fsatisfy{\hnum{n_1}}{\cnum{n_2}} ={}& (n_1 = n_2) \label{defn:num-satisfy-num}\\
%   \fsatisfy{e}{\cor{\hxi_1}{\hxi_2}} ={}& \fsatisfy{e}{\hxi_1} \textor \fsatisfy{e}{\hxi_2} \label{defn:satisfy-or}\\
%   \fsatisfy{\hinl{\tau_2}{e_1}}{\cinl{\hxi_1}} ={}& \fsatisfy{e_1}{\hxi_1} \label{defn:inl-satisfy-inl}\\
%   \fsatisfy{\hinr{\tau_1}{e_2}}{\cinr{\hxi_2}} ={}& \fsatisfy{e_2}{\hxi_2} \label{defn:inr-satisfy-inr}\\
%   \fsatisfy{\hpair{e_1}{e_2}}{\cpair{\hxi_1}{\hxi_2}} ={}& \fsatisfy{e_1}{\hxi_1} \textand \fsatisfy{e_2}{\hxi_2} \label{defn:pair-satisfy-pair}\\
%   \text{Otherwise}\quad \fsatisfy{e}{\cpair{\hxi_1}{\hxi_2}} ={}& \fnotIntro{e} \textand \fsatisfy{\hfst{e}}{\hxi_1} \notag\\
%   &\textand \fsatisfy{\hsnd{e}}{\hxi_2} \label{defn:notintro-satisfy-pair}\\
%   \fsatisfy{e}{\hxi} ={}& \false \label{defn:not-satisfy}
% \end{align}
% \end{subequations}

% \begin{lemma}[Soundness and Completeness of Satisfaction Judgment]
%   \label{lem:sound-complete-satisfy}
%   $\csatisfy{e}{\hxi}$ iff $\fsatisfy{e}{\hxi} = \true$.
% \end{lemma}

\judgboxa{\cmaysatisfy{e}{\hxi}}{$e$ may satisfy $\hxi$}
\begin{subequations}\label{rules:MaySatisfy}
\begin{equation}\label{rule:CMSUnknown}
\inferrule[CMSUnknown]{ }{
  \cmaysatisfy{e}{\cunknown}
}
\end{equation}
\begin{equation}\label{rule:CMSInl}
\inferrule[CMSInl]{
  \cmaysatisfy{e_1}{\hxi_1}
}{
  \cmaysatisfy{
    \hinl{\tau_2}{e_1}
  }{
    \cinl{\hxi_1}
  }
}
\end{equation}
\begin{equation}\label{rule:CMSInr}
\inferrule[CMSInr]{
  \cmaysatisfy{e_2}{\hxi_2}
}{
  \cmaysatisfy{
    \hinr{\tau_1}{e_2}
  }{
    \cinr{\hxi_2}
  }
}
\end{equation}
\begin{equation}\label{rule:CMSPair1}
\inferrule[CMSPairL]{
  \cmaysatisfy{e_1}{\hxi_1} \\
  \csatisfy{e_2}{\hxi_2}
}{
  \cmaysatisfy{\hpair{e_1}{e_2}}{\cpair{\hxi_1}{\hxi_2}}
}
\end{equation}
\begin{equation}\label{rule:CMSPair2}
\inferrule[CMSPairR]{
  \csatisfy{e_1}{\hxi_1} \\
  \cmaysatisfy{e_2}{\hxi_2}
}{
  \cmaysatisfy{\hpair{e_1}{e_2}}{\cpair{\hxi_1}{\hxi_2}}
}
\end{equation}
\begin{equation}\label{rule:CMSPair3}
\inferrule[CMSPair]{
  \cmaysatisfy{e_1}{\hxi_1} \\
  \cmaysatisfy{e_2}{\hxi_2}
}{
  \cmaysatisfy{\hpair{e_1}{e_2}}{\cpair{\hxi_1}{\hxi_2}}
}
\end{equation}
\begin{equation}\label{rule:CMSOr1}
\inferrule[CMSOrL]{
  \cmaysatisfy{e}{\hxi_1} \\
  \cnotsatisfy{e}{\hxi_2}
}{
  \cmaysatisfy{e}{\cor{\hxi_1}{\hxi_2}}
}
\end{equation}
\begin{equation}\label{rule:CMSOr2}
\inferrule[CMSOrR]{
  \cnotsatisfy{e}{\hxi_1} \\
  \cmaysatisfy{e}{\hxi_2}
}{
  \cmaysatisfy{e}{\cor{\hxi_1}{\hxi_2}}
}
\end{equation}
\begin{equation}\label{rule:CMSNotIntro}
\inferrule[CMSNotIntro]{
  \notIntro{e} \\
  \refutable{\hxi} \\
  \possible{\hxi}
}{
  \cmaysatisfy{e}{\hxi}
}
\end{equation}
\end{subequations}

% \judgboxa{\fmaysatisfy{e}{\hxi}}{}
% \begin{subequations}\label{defn:maysatisfy}
%   \begin{align}
%     \fmaysatisfy{e}{\cunknown} ={}& \true \label{defn:maysat-unknown}\\
%     \fmaysatisfy{\hinl{\tau_2}{e_1}}{\cinl{\hxi_1}} ={}& \fmaysatisfy{e_1}{\hxi_1} \label{defn:maysat-inl}\\
%     \fmaysatisfy{\hinr{\tau_1}{e_2}}{\cinr{\hxi_2}} ={}& \fmaysatisfy{e_2}{\hxi_2} \label{defn:maysat-inr}\\
%     \fmaysatisfy{\hinl{\tau_2}{e_1}}{\cinr{\hxi_2}} ={}& \false \label{defn:not-inl-maysat-inr}\\
%     \fmaysatisfy{\hinr{\tau_1}{e_2}}{\cinl{\hxi_1}} ={}& \false \label{defn:not-inr-maysat-inl}\\
%     \fmaysatisfy{\hpair{e_1}{e_2}}{\cpair{\hxi_1}{\hxi_2}} ={}& \left(\fmaysatisfy{e_1}{\hxi_1} \textand \fsatisfy{e_2}{\hxi_2}\right) \notag\\
%     &\textor \left(\fsatisfy{e_1}{\hxi_1} \textand \fmaysatisfy{e_2}{\hxi_2}\right) \notag\\
%     &\textor \left(\fmaysatisfy{e_1}{\hxi_1} \textand \fmaysatisfy{e_2}{\hxi_2}\right) \label{defn:maysat-pair}\\
%     \fmaysatisfy{e}{\cor{\hxi_1}{\hxi_2}} ={}& \left(\fmaysatisfy{e}{\hxi_1} \textand \left(\textnot \fsatisfy{e}{\hxi_2}\right)\right) \notag\\
%     &\textor \left(\left(\textnot \fsatisfy{e}{\hxi_1}\right) \textand \fmaysatisfy{e}{\hxi_2}\right) \label{defn:maysat-or}\\
%     \fmaysatisfy{e}{\hxi} =& \fnotIntro{e} \textand \frefutable{\hxi} \textand \fpossible{\hxi} \label{defn:notintro-maysat}
%   \end{align}
% \end{subequations}

% \begin{lemma}[Soundness and Completeness of Maybe Satisfaction Judgment]
%   \label{lem:sound-complete-maysatisfy}
%   $\cmaysatisfy{e}{\hxi}$ iff $\fmaysatisfy{e}{\hxi} = \true$.
% \end{lemma}

\judgboxa{\csatisfyormay{e}{\hxi}}{$e$ satisfies or may satisfy $\hxi$}
\begin{subequations}\label{rules:satormay}
\begin{equation}\label{rule:CSMSMay}
\inferrule[CSMSMay]{
  \cmaysatisfy{e}{\hxi}
}{
  \csatisfyormay{e}{\hxi}
}
\end{equation}
\begin{equation}\label{rule:CSMSSat}
\inferrule[CSMSSat]{
  \csatisfy{e}{\hxi}
}{
  \csatisfyormay{e}{\hxi}
}
\end{equation}
\end{subequations}

% \judgboxa{\fsatisfyormay{e}{\hxi}}{}
% \begin{equation}\label{defn:satormay}
%   \fsatisfyormay{e}{\hxi} = \fsatisfy{e}{\hxi} \textor \fmaysatisfy{e}{\hxi}
% \end{equation}

% \begin{lemma}[Soundness and Completeness of Satisfaction or Maybe Satisfaction]
%   \label{lem:sound-complete-satormay}
%   $\csatisfyormay{e}{\hxi}$ iff $\fsatisfyormay{e}{\hxi}$.
% \end{lemma}

\begin{lemma}
  \label{lem:notintro-maysat-or-notsat-refutable}
  Assume $\notIntro{e}$. If $\cmaysatisfy{e}{\hxi}$ or $\cnotsatisfy{e}{\hxi}$ then $\refutable{\hxi}$.
\end{lemma}

\begin{lemma}
\label{lem:satisfy-not-refutable}
If $\notIntro{e}$ and $\csatisfy{e}{\hxi}$ then $\cancel{\refutable{\hxi}}$.
\end{lemma}

\begin{theorem}[Exclusiveness of Satisfaction Judgment]
  \label{thrm:exclusive-constraint-satisfaction}
  If $\ctyp{\hxi}{\tau}$ and $\hexptyp{\cdot}{\Delta}{e}{\tau}$ and $\isFinal{e}$ then exactly one of the following holds
  \begin{enumerate}
    \item $\csatisfy{e}{\hxi}$
    \item $\cmaysatisfy{e}{\hxi}$
    \item $\cnotsatisfyormay{e}{\hxi}$
  \end{enumerate}
\end{theorem}

\begin{definition}[Entailment of Constraints]
  \label{defn:const-entailment}
  Suppose that $\ctyp{\hxi_1}{\tau}$ and $\ctyp{\hxi_2}{\tau}$.
  Then $\csatisfy{\hxi_1}{\hxi_2}$ iff for all $e$ such that $\hexptyp{\cdot}{\Delta}{e}{\tau}$ and $\isVal{e}$ we have $\csatisfyormay{e}{\hxi_1}$ implies $\csatisfy{e}{\hxi_2}$
\end{definition}

\begin{definition}[Potential Entailment of Constraints]
  \label{defn:nn-entailment}
  Suppose that $\ctyp{\hxi_1}{\tau}$ and $\ctyp{\hxi_2}{\tau}$. Then $\csatisfyormay{\hxi_1}{\hxi_2}$ iff for all $e$ such that $\hexptyp{\cdot}{\Delta}{e}{\tau}$ and $\isFinal{e}$ we have $\csatisfyormay{e}{\hxi_1}$ implies $\csatisfyormay{e}{\hxi_2}$ 
\end{definition}

\begin{corollary}
  \label{corol:nn-exhaust}
  Suppose that $\ctyp{\hxi}{\tau}$ and $\hexptyp{\cdot}{\Delta}{e}{\tau}$ and $\isFinal{e}$. Then $\csatisfyormay{\ctruth}{\hxi}$ implies $\csatisfyormay{e}{\hxi}$
\end{corollary}

%%% Local Variables:
%%% mode: latex
%%% TeX-master: "appendix"
%%% TeX-master: "appendix"
%%% End:

\section{Normal Match Constraint Language}
$\arraycolsep=4pt\begin{array}{lll}
\xi & ::= &
  \ctruth ~\vert~
  \cfalsity ~\vert~
  \cnum{n} ~\vert~
  \cnotnum{n} ~\vert~
  \cand{\xi_1}{\xi_2} ~\vert~
  \cor{\xi_1}{\xi_2} ~\vert~
  \cinl{\xi} ~\vert~
  \cinr{\xi} ~\vert~
  \cpair{\xi_1}{\xi_2}
\end{array}$

\judgboxa{\ctyp{\xi}{\tau}}{$\xi$ constrains final expressions of type $\tau$}
\begin{subequations}\label{rules:CCTyp}
\begin{equation}\label{rule:CCTTruth}
\inferrule[CTTruth]{ }{
  \ctyp{\ctruth}{\tau}
}
\end{equation}
\begin{equation}\label{rule:CCTFalsity}
\inferrule[CTFalsity]{ }{
  \ctyp{\cfalsity}{\tau}
}
\end{equation}
\begin{equation}\label{rule:CCTNum}
\inferrule[CTNum]{ }{
  \ctyp{\cnum{n}}{\tnum}
}
\end{equation}
\begin{equation}\label{rule:CCTNotNum}
\inferrule[CTNotNum]{ }{
  \ctyp{\cnotnum{n}}{\tnum}
}
\end{equation}
\begin{equation}\label{rule:CCTAnd}
\inferrule[CTAnd]{
  \ctyp{\xi_1}{\tau} \\ \ctyp{\xi_2}{\tau}
}{
  \ctyp{\cand{\xi_1}{\xi_2}}{\tau}
}
\end{equation}
\begin{equation}\label{rule:CCTOr}
\inferrule[CTOr]{
  \ctyp{\xi_1}{\tau} \\ \ctyp{\xi_2}{\tau}
}{
  \ctyp{\cor{\xi_1}{\xi_2}}{\tau}
}
\end{equation}
\begin{equation}\label{rule:CCTInl}
\inferrule[CTInl]{
  \ctyp{\xi_1}{\tau_1}
}{
  \ctyp{\cinl{\xi_1}}{\tsum{\tau_1}{\tau_2}}
}
\end{equation}
\begin{equation}\label{rule:CCTInr}
\inferrule[CTInr]{
  \ctyp{\xi_2}{\tau_2}
}{
  \ctyp{\cinr{\xi_2}}{\tsum{\tau_1}{\tau_2}}
}
\end{equation}
\begin{equation}\label{rule:CCTPair}
\inferrule[CTPair]{
  \ctyp{\xi_1}{\tau_1} \\ \ctyp{\xi_2}{\tau_2}
}{
  \ctyp{\cpair{\xi_1}{\xi_2}}{\tprod{\tau_1}{\tau_2}}
}
\end{equation}
\end{subequations}

\judgboxa{\cdual{\xi_1} = \xi_2}{dual of $\xi_1$ is $\xi_2$}
\begin{subequations}\label{defn:dual}
\begin{align}
  \cdual{\ctruth} &= \cfalsity \\
  \cdual{\cfalsity} &= \ctruth \\
  \cdual{\cnum{n}} &= \cnotnum{n} \\
  \cdual{\cnotnum{n}} &= \cnum{n} \\
  \cdual{\cand{\xi_1}{\xi_2}} &= \cor{\cdual{\xi_1}}{\cdual{\xi_2}} \\
  \cdual{\cor{\xi_1}{\xi_2}} &= \cand{\cdual{\xi_1}}{\cdual{\xi_2}} \\
  \cdual{\cinl{\xi_1}} &= \cor{ \cinl{\cdual{\xi_1}} }{ \cinr{\ctruth} } \\
  \cdual{\cinr{\xi_2}} &= \cor{ \cinr{\cdual{\xi_2}} }{ \cinl{\ctruth} } \\
  \cdual{\cpair{\xi_1}{\xi_2}} &=
  \cor{ \cor{ 
    \cpair{\xi_1}{\cdual{\xi_2}}
  }{
    \cpair{\cdual{\xi_1}}{\xi_2}
  }}{
    \cpair{\cdual{\xi_1}}{\cdual{\xi_2}}
  }
\end{align}
\end{subequations}

\judgboxa{\ccsatisfy{e}{\xi}}{$e$ satisfies $\xi$}
\begin{subequations}\label{rules:cSatisfy}
\begin{equation}\label{rule:CCSTruth}
\inferrule[CSTruth]{ }{
  \ccsatisfy{e}{\ctruth}
}
\end{equation}
\begin{equation}\label{rule:CCSNum}
\inferrule[CSNum]{ }{
  \ccsatisfy{\hnum{n}}{\cnum{n}}
}
\end{equation}
\begin{equation}\label{rule:CCSNotNum}
\inferrule[CSNotNum]{
  n_1 \neq n_2
}{
  \ccsatisfy{\hnum{n_1}}{\cnotnum{n_2}}
}
\end{equation}
\begin{equation}\label{rule:CCSAnd}
\inferrule[CSAnd]{
  \ccsatisfy{e}{\xi_1} \\
  \ccsatisfy{e}{\xi_2}
}{
  \ccsatisfy{e}{\cand{\xi_1}{\xi_2}}
}
\end{equation}
\begin{equation}\label{rule:CCSOr1}
\inferrule[CSOrL]{
  \ccsatisfy{e}{\xi_1}
}{
  \ccsatisfy{e}{\cor{\xi_1}{\xi_2}}
}
\end{equation}
\begin{equation}\label{rule:CCSOr2}
\inferrule[CSOrR]{
  \ccsatisfy{e}{\xi_2}
}{
  \ccsatisfy{e}{\cor{\xi_1}{\xi_2}}
}
\end{equation}
\begin{equation}\label{rule:CCSInl}
\inferrule[CSInl]{
  \ccsatisfy{e_1}{\xi_1}
}{
  \ccsatisfy{
    \hinl{\tau_2}{e_1}
  }{
    \cinl{\xi_1}
  }
}
\end{equation}
\begin{equation}\label{rule:CCSInr}
\inferrule[CSInr]{
  \ccsatisfy{e_2}{\xi_2}
}{
  \ccsatisfy{
    \hinr{\tau_1}{e_2}
  }{
    \cinr{\xi_2}
  }
}
\end{equation}
\begin{equation}\label{rule:CCSPair}
\inferrule[CSPair]{
  \ccsatisfy{e_1}{\xi_1} \\
  \ccsatisfy{e_2}{\xi_2}
}{
\ccsatisfy{\hpair{e_1}{e_2}}{\cpair{\xi_1}{\xi_2}}
}
\end{equation}
\end{subequations}

\begin{lemma}
  \label{lem:notsatisfy-dual}
  Assume $\isVal{e}$. Then $\ccnotsatisfy{e}{\xi}$ iff $\ccsatisfy{e}{\cdual{\xi}}$.
\end{lemma}

\begin{theorem}[Exclusiveness of Satisfaction Judgment]
  \label{thrm:exclusive-complete-constraint-satisfaction}
  If $\ctyp{\xi}{\tau}$ and $\hexptyp{\cdot}{\Delta}{e}{\tau}$ and $\isVal{e}$ then exactly one of the following holds
  \begin{enumerate}
    \item $\csatisfy{e}{\xi}$
    \item $\csatisfy{e}{\cdual{\xi}}$
  \end{enumerate}
\end{theorem}
\begin{proof}
\end{proof}

\begin{definition}[Entailment of Constraints]
  \label{defn:complete-constraint-entailment}
  Suppose that $\ctyp{\xi_1}{\tau}$ and $\ctyp{\xi_2}{\tau}$.
  Then $\csatisfy{\xi_1}{\xi_2}$ iff for all $e$ such that $\hexptyp{\cdot}{\Delta}{e}{\tau}$ and $\isVal{e}$ we have $\csatisfy{e}{\xi_1}$ implies $\csatisfy{e}{\xi_2}$
\end{definition}

\subsection{Relationship with Incomplete Constraint Language}

\begin{theorem}
\label{thrm:demystify-exhaustiveness}
$\csatisfyormay{\ctruth}{\hxi}$ iff $\ccsatisfy{\ctruth}{\ctruify{\hxi}}$.
\end{theorem}

\begin{theorem}
\label{thrm:demystify-redundancy}
$\csatisfy{\hxi_1}{\hxi_2}$ iff $\ccsatisfy{\ctruify{\hxi_1}}{\cfalsify{\hxi_2}}$.
\end{theorem}


\begin{lemma}
  \label{lem:val-satisfy-truify}
  Assume that $\isVal{e}$. Then $\csatisfyormay{e}{\hxi}$ iff $\csatisfy{e}{\ctruify{\hxi}}$
\end{lemma}
\begin{proof}\mbox{}\\
  We prove sufficiency and necessity separately.
  \begin{enumerate}
    \item Sufficiency:
      \begin{pfsteps}
      \item \isVal{e} \BY{assumption} \pflabel{val}
      \item \csatisfyormay{e}{\hxi} \BY{assumption} \pflabel{satormay}
      \end{pfsteps}
      By rule induction over Rules (\ref{rules:satormay}) on \pfref{satormay}.
      \begin{byCases}

      \savelocalsteps{1}
      \item[\text{(\ref{rule:CSMSSat})}]
        \begin{pfsteps*}
        \item $\csatisfy{e}{\hxi}$ \BY{assumption} \pflabel{satisfy}
        \end{pfsteps*}
        By rule induction over \rulesref{rules:Satisfy} on \pfref{satisfy}.
        \begin{byCases}
          \savelocalsteps{2}
          \item[\text{(\ref{rule:CSTruth})}]
          \begin{pfsteps}
          \item \hxi = \ctruth \BY{assumption}
          \item \ctruify{\hxi} = \ctruth \BY{\autoref{defn:truify}}
          \item \ccsatisfy{e}{\ctruth} \BY{\ruleref{rule:CCSTruth}}
          \end{pfsteps} 
          \restorelocalsteps{2} 
          \item[\text{(\ref{rule:CSNum})}] 
          \begin{pfsteps}
          \item e = \hnum{n} \BY{assumption}
          \item \hxi = \cnum{n} \BY{assumption}
          \item \ctruify{\cnum{n}} = \cnum{n} \BY{\autoref{defn:truify}}
          \item \ccsatisfy{e}{\cnum{n}} \BY{\ruleref{rule:CCSNum}}
          \end{pfsteps} 
          \restorelocalsteps{2} 
          \item[\text{(\ref{rule:CSInl})}]
          \begin{pfsteps}
          \item e = \hinl{\tau_2}{e_1} \BY{assumption}
          \item \hxi = \cinl{\hxi_1} \BY{assumption}
          \item \csatisfy{e_1}{\hxi_1} \BY{assumption} \pflabel{[inl]satisfy1}
          \item \csatisfyormay{e_1}{\hxi_1} \BY{\ruleref{rule:CSMSSat} on \pfref{[inl]satisfy1}} \pflabel{[inl]satormay1}
          \item \ctruify{\cinl{\hxi_1}} = \cinl{\ctruify{\hxi_1}} \BY{\autoref{defn:truify}}
          \end{pfsteps}
          By rule induction over \rulesref{rules:Value} on \pfref{val}, only one rule applies.
          \begin{byCases}
            \item[\text{(\ref{rule:VInl})}]
            \begin{pfsteps}
            \item \isVal{e_1} \BY{assumption} \pflabel{[inl]val1}
            \item \ccsatisfy{e_1}{\ctruify{\hxi_1}} \BY{IH on \pfref{[inl]val1} and \pfref{[inl]satormay1}} \pflabel{[inl]ccsat1}
            \item \ccsatisfy{\hinl{\tau_2}{e_1}}{\cinl{\ctruify{\hxi_1}}} \BY{\ruleref{rule:CCSInl} on \pfref{[inl]ccsat1}}
            \end{pfsteps}    
          \end{byCases}
          \restorelocalsteps{2} 
          \item[\text{(\ref{rule:CSInr})}]
          \begin{pfsteps}
          \item e = \hinr{\tau_1}{e_2} \BY{assumption}
          \item \hxi = \cinr{\hxi_2} \BY{assumption}
          \item \csatisfy{e_2}{\hxi_2} \BY{assumption} \pflabel{[inr]satisfy2}
          \item \csatisfyormay{e_2}{\hxi_2} \BY{\ruleref{rule:CSMSSat} on \pfref{[inr]satisfy2}} \pflabel{[inr]satormay2}
          \item \ctruify{\cinr{\hxi_2}} = \cinr{\ctruify{\hxi_2}} \BY{\autoref{defn:truify}}
          \end{pfsteps}
          By rule induction over \rulesref{rules:Value} on \pfref{val}, only one rule applies.
          \begin{byCases}
            \item[\text{(\ref{rule:VInr})}]
            \begin{pfsteps}
            \item \isVal{e_2} \BY{assumption} \pflabel{[inr]val2}
            \item \ccsatisfy{e_2}{\ctruify{\hxi_2}} \BY{IH on \pfref{[inr]val2} and \pfref{[inr]satormay2}} \pflabel{[inr]ccsat2}
            \item \ccsatisfy{\hinr{\tau_1}{e_2}}{\cinr{\ctruify{\hxi_2}}} \BY{\ruleref{rule:CCSInr} on \pfref{[inr]ccsat2}}
            \end{pfsteps}    
          \end{byCases}
          \restorelocalsteps{2} 
          \item[\text{(\ref{rule:CSPair})}]
          \begin{pfsteps}
          \item e = \hpair{e_1}{e_2} \BY{assumption}
          \item \hxi = \cpair{\hxi_1}{\hxi_2} \BY{assumption}
          \item \csatisfy{e_1}{\hxi_1} \BY{assumption} \pflabel{[pair]satisfy1}
          \item \csatisfy{e_2}{\hxi_2} \BY{assumption} \pflabel{[pair]satisfy2}
          \item \csatisfyormay{e_1}{\hxi_1} \BY{\ruleref{rule:CSMSSat} on \pfref{[pair]satisfy1}} \pflabel{[pair]satormay1}
          \item \csatisfyormay{e_2}{\hxi_2} \BY{\ruleref{rule:CSMSSat} on \pfref{[pair]satisfy2}} \pflabel{[pair]satormay2}
          \item \ctruify{\cpair{\hxi_1}{\hxi_2}} = \cpair{\ctruify{\hxi_1}}{\ctruify{\hxi_2}} \BY{\autoref{defn:truify}}
          \end{pfsteps}
          By rule induction over \rulesref{rules:Value} on \pfref{val}, only one rule applies.
          \begin{byCases}
            \item[\text{(\ref{rule:VPair})}]
            \begin{pfsteps}
            \item \isVal{e_1} \BY{assumption} \pflabel{[pair]val1}
            \item \isVal{e_2} \BY{assumption} \pflabel{[pair]val2}
            \item \ccsatisfy{e_1}{\ctruify{\hxi_1}} \BY{IH on \pfref{[pair]val1} and \pfref{[pair]satormay1}} \pflabel{[pair]ccsat1}
            \item \ccsatisfy{e_2}{\ctruify{\hxi_2}} \BY{IH on \pfref{[pair]val2} and \pfref{[pair]satormay2}} \pflabel{[pair]ccsat2}
            \item \ccsatisfy{\hpair{e_1}{e_2}}{\cpair{\ctruify{\hxi_1}}{\ctruify{\hxi_2}}} \BY{\ruleref{rule:CCSPair} on \pfref{[pair]ccsat1} and \pfref{[pair]ccsat2}}
            \end{pfsteps}    
          \end{byCases}
          \restorelocalsteps{2} 
          \item[\text{(\ref{rule:CSNotIntroPair})}]
          \begin{pfsteps}
          \item \notIntro{e} \BY{assumption}
          \end{pfsteps} 
          Contradicts \pfref{val} by \autoref{lem:val-not-notintro}.
          \restorelocalsteps{2} 
          \item[\text{(\ref{rule:CSOr1})}]
          \begin{pfsteps}
          \item \hxi = \cor{\hxi_1}{\hxi_2} \BY{assumption}
          \item \csatisfy{e}{\hxi_1} \BY{assumption} \pflabel{[or1]sat1}
          \item \csatisfyormay{e}{\hxi_1} \BY{\ruleref{rule:CSMSSat} on \pfref{[or1]sat1}} \pflabel{[or1]satormay1}
          \item \ctruify{\cor{\hxi_1}{\hxi_2}} = \cor{\ctruify{\hxi_1}}{\ctruify{\hxi_2}} \BY{\autoref{defn:truify}}
          \item \ccsatisfy{e}{\ctruify{\hxi_1}} \BY{IH on \pfref{val} and \pfref{[or1]satormay1}} \pflabel{[or1]ccsat1}
          \item \ccsatisfy{e}{\cor{\ctruify{\hxi_1}}{\ctruify{\hxi_2}}} \BY{\ruleref{rule:CCSOr1} on \pfref{[or1]ccsat1}}
          \end{pfsteps} 
          \restorelocalsteps{2} 
          \item[\text{(\ref{rule:CSOr2})}]
          \begin{pfsteps}
          \item \hxi = \cor{\hxi_1}{\hxi_2} \BY{assumption}
          \item \csatisfy{e}{\hxi_2} \BY{assumption} \pflabel{[or2]sat2}
          \item \csatisfyormay{e}{\hxi_2} \BY{\ruleref{rule:CSMSSat} on \pfref{[or2]sat2}} \pflabel{[or2]satormay2}
          \item \ctruify{\cor{\hxi_1}{\hxi_2}} = \cor{\ctruify{\hxi_1}}{\ctruify{\hxi_2}} \BY{\autoref{defn:truify}}
          \item \ccsatisfy{e}{\ctruify{\hxi_2}} \BY{IH on \pfref{val} and \pfref{[or2]satormay2}} \pflabel{[or2]ccsat2}
          \item \ccsatisfy{e}{\cor{\ctruify{\hxi_1}}{\ctruify{\hxi_2}}} \BY{\ruleref{rule:CCSOr2} on \pfref{[or2]ccsat2}}
          \end{pfsteps} 
        \end{byCases}
      
      \restorelocalsteps{1}
      \item[\text{(\ref{rule:CSMSMay})}]
        \begin{pfsteps*}
        \item $\cmaysatisfy{e}{\hxi}$ \BY{assumption} \pflabel{maysat}
        \end{pfsteps*}
        By rule induction over Rules (\ref{rules:MaySatisfy}) on \pfref{maysat}.
        \begin{byCases}

        \savelocalsteps{2}
        \item[\text{(\ref{rule:CMSUnknown})}]
          \begin{pfsteps}
          \item \hxi=\cunknown \BY{assumption}
          \item \ctruify{\cunknown} = \ctruth \BY{\autoref{defn:truify}} 
          \item \ccsatisfy{e}{\ctruth} \BY{Rule (\ref{rule:CCSTruth})}
          \end{pfsteps}
        
        \restorelocalsteps{2} 
        \item[\text{(\ref{rule:CMSInl})}] 
          \begin{pfsteps}
          \item e = \hinl{\tau_2}{e_1} \BY{assumption}
          \item \hxi=\cinl{\hxi_1} \BY{assumption}
          \item \cmaysatisfy{e_1}{\hxi_1} \BY{assumption} \pflabel{[inl1]maysat}
          \item \csatisfyormay{e_1}{\hxi_1} \BY{\ruleref{rule:CSMSMay} on \pfref{[inl1]maysat}} \pflabel{[inl1]satormay}
          \item \ctruify{\cinl{\hxi_1}}=\cinl{\ctruify{\xi_1}} \BY{\autoref{defn:truify}}
          \end{pfsteps}
          By rule induction over \rulesref{rules:Value} on \pfref{val}, only one rule applies.
          \begin{byCases}
            \item[\text{(\ref{rule:VInl})}]
            \begin{pfsteps}
            \item \isVal{e_1} \BY{assumption} \pflabel{[may][inl]val1}
            \item \ccsatisfy{e_1}{\ctruify{\hxi_1}} \BY{IH on \pfref{[may][inl]val1} and \pfref{[inl1]satormay}} \pflabel{[inl1]sat-truify}
            \item \ccsatisfy{\hinl{\tau_2}{e_1}}{\cinl{\ctruify{\hxi_1}}} \BY{\ruleref{rule:CSInl} on \pfref{[inl1]sat-truify}}
            \end{pfsteps}
          \end{byCases}
        
        \restorelocalsteps{2} 
        \item[\text{(\ref{rule:CMSInr})}] 
          \begin{pfsteps}
          \item e = \hinr{\tau_1}{e_2} \BY{assumption}
          \item \hxi=\cinr{\hxi_2} \BY{assumption}
          \item \cmaysatisfy{e_2}{\hxi_2} \BY{assumption} \pflabel{[inr]maysat}
          \item \csatisfyormay{e_2}{\hxi_2} \BY{\ruleref{rule:CSMSMay} on \pfref{[inr]maysat}} \pflabel{[inr]satormay}
          \item \ctruify{\cinr{\hxi_2}}=\cinr{\ctruify{\hxi_2}} \BY{\autoref{defn:truify}}
          \end{pfsteps}
          By rule induction over \rulesref{rules:Value} on \pfref{val}, only one rule applies.
          \begin{byCases}
            \item[\text{(\ref{rule:VInr})}]
            \begin{pfsteps}
            \item \isVal{e_2} \BY{assumption} \pflabel{[may][inr]val2}
            \item \ccsatisfy{e_2}{\ctruify{\hxi_2}} \BY{IH on \pfref{[may][inr]val2} and \pfref{[inr]satormay}} \pflabel{[inr]sat-truify}
            \item \ccsatisfy{\hinr{\tau_1}{e_2}}{\cinr{\ctruify{\hxi_2}}} \BY{\ruleref{rule:CSInr} on \pfref{[inr]sat-truify}}
            \end{pfsteps}
          \end{byCases}
        
        \restorelocalsteps{2} 
        \item[\text{(\ref{rule:CMSPair1})}] 
          \begin{pfsteps}
          \item e=\hpair{e_1}{e_2} \BY{assumption}
          \item \hxi=\cpair{\hxi_1}{\hxi_2} \BY{assumption}
          \item \cmaysatisfy{e_1}{\hxi_1} \BY{assumption} \pflabel{[pair1]maysat1}
          \item \csatisfy{e_2}{\hxi_2} \BY{assumption} \pflabel{[pair1]satisfy2}
          \item \ctruify{\hxi}=\cpair{\ctruify{\hxi_1}}{\ctruify{\hxi_2}} \BY{\autoref{defn:truify}}
          \item \csatisfyormay{e_1}{\hxi_1} \BY{\ruleref{rule:CSMSMay} on \pfref{[pair1]maysat1}} \pflabel{[pair1]satormay1}
          \item \csatisfyormay{e_2}{\hxi_2} \BY{\ruleref{rule:CSMSSat} on \pfref{[pair1]satisfy2}} \pflabel{[pair1]satormay2}
          \end{pfsteps}
          By rule induction over \rulesref{rules:Value} on \pfref{val}, only one rule applies.
          \begin{byCases}
            \item[\text{(\ref{rule:VPair})}]
            \begin{pfsteps}
            \item \isVal{e_1} \BY{assumption} \pflabel{[pair1]val1}
            \item \isVal{e_2} \BY{assumption} \pflabel{[pair1]val2}
            \item \ccsatisfy{e_1}{\ctruify{\hxi_1}} \BY{IH on \pfref{[pair1]val1} and \pfref{[pair1]satormay1}} \pflabel{[pair1]sat-truify1}
            \item \ccsatisfy{e_2}{\ctruify{\hxi_2}} \BY{IH on \pfref{[pair1]val2} and \pfref{[pair1]satormay2}} \pflabel{[pair1]sat-truify2}
            \item \ccsatisfy{\hpair{e_1}{e_2}}{\cpair{\ctruify{\hxi_1}}{\ctruify{\hxi_2}}} \BY{\ruleref{rule:CSPair} on \pfref{[pair1]sat-truify1} and \pfref{[pair1]sat-truify2}}
            \end{pfsteps}
          \end{byCases}
        
        \restorelocalsteps{2} 
        \item[\text{(\ref{rule:CMSPair2})}] 
          \begin{pfsteps}
          \item e=\hpair{e_1}{e_2} \BY{assumption}
          \item \hxi=\cpair{\hxi_1}{\hxi_2} \BY{assumption}
          \item \csatisfy{e_1}{\hxi_1} \BY{assumption} \pflabel{[pair2]satisfy1}
          \item \cmaysatisfy{e_2}{\hxi_2} \BY{assumption} \pflabel{[pair2]maysat2}
          \item \ctruify{\cpair{\hxi_1}{\hxi_2}}=\cpair{\ctruify{\hxi_1}}{\ctruify{\xi_2}} \BY{\autoref{defn:truify}}
          \item \csatisfyormay{e_1}{\hxi_1} \BY{\ruleref{rule:CSMSSat} on \pfref{[pair2]satisfy1}} \pflabel{[pair2]satormay1}
          \item \csatisfyormay{e_2}{\hxi_2} \BY{\ruleref{rule:CSMSMay} on \pfref{[pair2]maysat2}} \pflabel{[pair2]satormay2}
          \end{pfsteps}
          By rule induction over \rulesref{rules:Value} on \pfref{val}, only one rule applies.
          \begin{byCases}
            \item[\text{(\ref{rule:VPair})}]
            \begin{pfsteps}
            \item \isVal{e_1} \BY{assumption} \pflabel{[pair2]val1}
            \item \isVal{e_2} \BY{assumption} \pflabel{[pair2]val2}
            \item \ccsatisfy{e_1}{\ctruify{\hxi_1}} \BY{IH on \pfref{[pair2]val1} and \pfref{[pair2]satormay1}} \pflabel{[pair2]sat-truify1}
            \item \ccsatisfy{e_2}{\ctruify{\hxi_2}} \BY{IH on \pfref{[pair2]val2} and \pfref{[pair2]satormay2}} \pflabel{[pair2]sat-truify2}
            \item \ccsatisfy{\hpair{e_1}{e_2}}{\cpair{\ctruify{\hxi_1}}{\ctruify{\hxi_2}}} \BY{\ruleref{rule:CSPair} on \pfref{[pair2]sat-truify1} and \pfref{[pair2]sat-truify2}}
            \end{pfsteps}
          \end{byCases}
        
        \restorelocalsteps{2} 
        \item[\text{(\ref{rule:CMSPair3})}] 
          \begin{pfsteps}
          \item e=\hpair{e_1}{e_2} \BY{assumption}
          \item \hxi=\cpair{\hxi_1}{\hxi_2} \BY{assumption}
          \item \cmaysatisfy{e_1}{\hxi_1} \BY{assumption} \pflabel{[pair3]maysat1}
          \item \cmaysatisfy{e_2}{\hxi_2} \BY{assumption} \pflabel{[pair3]maysat2}
          \item \ctruify{\cpair{\hxi_1}{\hxi_2}}=\cpair{\ctruify{\hxi_1}}{\ctruify{\hxi_2}} \BY{\autoref{defn:truify}}
          \item \csatisfyormay{e_1}{\hxi_1} \BY{\ruleref{rule:CSMSMay} on \pfref{[pair3]maysat1}} \pflabel{[pair3]satormay1}
          \item \csatisfyormay{e_2}{\hxi_2} \BY{\ruleref{rule:CSMSMay} on \pfref{[pair3]maysat2}} \pflabel{[pair3]satormay2}
          \item \ccsatisfy{e_1}{\ctruify{\hxi_1}} \BY{IH on \pfref{[pair3]satormay1}} \pflabel{[pair3]sat-truify1}
          \item \ccsatisfy{e_2}{\ctruify{\hxi_2}} \BY{IH on \pfref{[pair3]satormay2}} \pflabel{[pair3]sat-truify2}
          \item \ccsatisfy{\hpair{e_1}{e_2}}{\cpair{\ctruify{\hxi_1}}{\ctruify{\hxi_2}}} \BY{\ruleref{rule:CSPair} on \pfref{[pair3]sat-truify1} and \pfref{[pair3]sat-truify2}}
          \end{pfsteps}
        
        \restorelocalsteps{2} 
        \item[\text{(\ref{rule:CMSOr1})}] 
          \begin{pfsteps}
          \item \hxi=\cor{\hxi_1}{\hxi_2} \BY{assumption}
          \item \cmaysatisfy{e}{\hxi_1} \BY{assumption} \pflabel{[or1]maysat}
          \item \ctruify{\cor{\hxi_1}{\hxi_2}}=\cor{\ctruify{\hxi_1}}{\ctruify{\hxi_2}} \BY{\autoref{defn:truify}}
          \item \csatisfyormay{e}{\hxi_1} \BY{\ruleref{rule:CSMSMay} on \pfref{[or1]maysat}} \pflabel{[or1]satormay}
          \item \ccsatisfy{e}{\ctruify{\hxi_1}} \BY{IH on \pfref{val} and \pfref{[or1]satormay}} \pflabel{[or1]sat-truify}
          \item \ccsatisfy{e}{\cor{\ctruify{\hxi_1}}{\ctruify{\hxi_2}}} \BY{\ruleref{rule:CCSOr1} on \pfref{[or1]sat-truify}}
          \end{pfsteps}
        
        \restorelocalsteps{2} 
        \item[\text{(\ref{rule:CMSOr2})}] 
          \begin{pfsteps}
          \item \hxi=\cor{\hxi_1}{\hxi_2} \BY{assumption}
          \item \cmaysatisfy{e}{\hxi_2} \BY{assumption} \pflabel{[or2]maysat}
          \item \ctruify{\hxi}=\cor{\ctruify{\hxi_1}}{\ctruify{\hxi_2}} \BY{\autoref{defn:truify}}
          \item \csatisfyormay{e}{\hxi_2} \BY{\ruleref{rule:CSMSMay} on \pfref{[or2]maysat}} \pflabel{[or2]satormay}
          \item \ccsatisfy{e}{\ctruify{\hxi_2}} \BY{IH on \pfref{val} and \pfref{[or2]satormay}} \pflabel{[or2]sat-truify}
          \item \ccsatisfy{e}{\cor{\ctruify{\hxi_1}}{\ctruify{\hxi_2}}} \BY{\ruleref{rule:CSOr2} on \pfref{[or2]sat-truify}}
          \end{pfsteps}
          
        \restorelocalsteps{2} 
        \item[\text{(\ref{rule:CMSNotIntro})}] 
          \begin{pfsteps}
          \item \notIntro{e} \BY{assumption} \pflabel{notintro}
          \end{pfsteps}
          Contradicts \pfref{val} by \autoref{lem:val-not-notintro}.
        
        \end{byCases}
      \end{byCases}

    \resetpfcounter

    \item Necessity:
    \begin{pfsteps}
    \item \isVal{e} \BY{assumption} \pflabel{val}
    \item \ccsatisfy{e}{\ctruify{\hxi}} \BY{assumption} \pflabel{ccsatisfy}
    \end{pfsteps}
    By structural induction on $\hxi$.
    \begin{byCases}

      \savelocalsteps{1}
      \item[\hxi=\ctruth]
      \begin{pfsteps}
      \item \csatisfy{e}{\ctruth} \BY{\ruleref{rule:CSTruth}} \pflabel{[truth]satisfy}
      \item \csatisfyormay{e}{\ctruth} \BY{\ruleref{rule:CSMSSat} on \pfref{[truth]satisfy}}
      \end{pfsteps}
      
      \restorelocalsteps{1}
      \item[\hxi=\cnum{n}]
      \begin{pfsteps}
      \item \ctruify{\cnum{n}} = \cnum{n} \BY{assumption}
      \end{pfsteps}
      By rule induction over \rulesref{rules:cSatisfy} on \pfref{ccsatisfy}, only one rule applies.
      \begin{byCases}
        \item[\text{(\ref{rule:CCSNum})}]
        \begin{pfsteps}
        \item e = \hnum{n} \BY{assumption}
        \item \csatisfy{\hnum{n}}{\cnum{n}} \BY{\ruleref{rule:CSNum}} \pflabel{[num]satisfy}
        \item \csatisfyormay{\hnum{n}}{\cnum{n}} \BY{\ruleref{rule:CSMSSat} on \pfref{[num]satisfy}}
        \end{pfsteps}
      \end{byCases}

      \restorelocalsteps{1}
      \item[\hxi=\cunknown]
        \begin{pfsteps}
        \item \cmaysatisfy{e}{\cunknown} \BY{\ruleref{rule:CMSUnknown}} \pflabel{[unknown]maysat}
        \item \csatisfyormay{e}{\cunknown} \BY{\ruleref{rule:CSMSMay} on \pfref{[unknown]maysat}}
        \end{pfsteps}

      \restorelocalsteps{1}
      \item[\hxi = \cor{\hxi_1}{\hxi_2}]
        \begin{pfsteps*}
        \item $\ctruify{\cor{\hxi_1}{\hxi_2}}=\cor{\ctruify{\hxi_1}}{\ctruify{\hxi_2}}$ \BY{\autoref{defn:truify}}
        \end{pfsteps*}
        By rule induction over \rulesref{rules:cSatisfy} on \pfref{ccsatisfy}, only two rules apply.
        \begin{byCases}

          \savelocalsteps{2}
          \item[\text{(\ref{rule:CCSOr1})}]
            \begin{pfsteps*}
            \item $\ccsatisfy{e}{\ctruify{\hxi_1}}$ \BY{assumption} \pflabel{[or]satisfy-falsify1}
            \item $\csatisfyormay{e}{\hxi_1}$ \BY{IH on \pfref{val} and \pfref{[or]satisfy-falsify1}} \pflabel{[or]satisfy1}
            \item $\csatisfyormay{e}{\cor{\hxi_1}{\hxi_2}}$ \BY{\autoref{lem:satormay-or} on \pfref{[or]satisfy1}}
            \end{pfsteps*}

          \restorelocalsteps{2}
          \item[\text{(\ref{rule:CCSOr2})}]
            \begin{pfsteps*}
            \item $\ccsatisfy{e}{\ctruify{\hxi_2}}$ \BY{assumption} \pflabel{[or]satisfy-falsify2}
            \item $\csatisfyormay{e}{\hxi_2}$ \BY{IH on \pfref{val} and \pfref{[or]satisfy-falsify2}} \pflabel{[or]satisfy2}
            \item $\csatisfyormay{e}{\cor{\hxi_1}{\hxi_2}}$ \BY{\autoref{lem:satormay-or} on \pfref{[or]satisfy2}}
            \end{pfsteps*}
        \end{byCases}
      
      \restorelocalsteps{1}
      \item[\hxi=\cinl{\hxi_1}]
        \begin{pfsteps*}
        \item $\ctruify{\cinl{\hxi_1}}=\cinl{\ctruify{\hxi_1}}$ \BY{\autoref{defn:truify}}
        \end{pfsteps*}
        By rule induction over \rulesref{rules:cSatisfy} on \pfref{ccsatisfy}, only one rule applies.
        \begin{byCases}
          \item[\text{(\ref{rule:CCSInl})}]
            \begin{pfsteps*}
            \item $e = \hinl{\tau_2}{e_1}$ \BY{assumption}
            \item $\ccsatisfy{e_1}{\ctruify{\hxi_1}}$ \BY{assumption} \pflabel{[inl]satisfy-falsify1}
            \end{pfsteps*}
            By rule induction over \rulesref{rules:Value} on \pfref{val}, only one rule applies.
            \begin{byCases}
              \item[\text{(\ref{rule:VInl})}]
              \begin{pfsteps*}
              \item $\isVal{e_1}$ \BY{assumption} \pflabel{[inl]val1}
              \item $\csatisfyormay{e_1}{\hxi_1}$ \BY{IH on \pfref{[inl]val1} and \pfref{[inl]satisfy-falsify1}} \pflabel{[inl]satisfy1}
              \item $\csatisfyormay{\hinl{\tau_2}{e_1}}{\cinl{\hxi_1}}$ \BY{\autoref{lem:satormay-inl} on \pfref{[inl]satisfy1}}
              \end{pfsteps*} 
            \end{byCases}
        \end{byCases}

      \restorelocalsteps{1}
      \item[\hxi=\cinr{\hxi_2}]
        \begin{pfsteps*}
        \item $\ctruify{\cinr{\hxi_2}}=\cinr{\ctruify{\hxi_2}}$ \BY{\autoref{defn:truify}}
        \end{pfsteps*}
        By rule induction over \rulesref{rules:cSatisfy} on \pfref{ccsatisfy}, only one rule applies.
        \begin{byCases}
          \item[\text{(\ref{rule:CSInr})}]
            \begin{pfsteps*}
            \item $e = \hinr{\tau_1}{e_2}$ \BY{assumption}
            \item $\ccsatisfy{e_2}{\ctruify{\hxi_2}}$ \BY{assumption} \pflabel{[inr]satisfy-falsify2}
            \end{pfsteps*}
            By rule induction over \rulesref{rules:Value} on \pfref{val}, only one rule applies.
            \begin{byCases}
              \item[\text{(\ref{rule:VInr})}]
              \begin{pfsteps*}
              \item $\isVal{e_2}$ \BY{assumption} \pflabel{[inr]val2}
              \item $\csatisfyormay{e_2}{\hxi_2}$ \BY{IH on \pfref{[inr]val2} and \pfref{[inr]satisfy-falsify2}} \pflabel{[inr]satisfy2}
              \item $\csatisfyormay{\hinr{\tau_1}{e_2}}{\cinr{\hxi_2}}$ \BY{\autoref{lem:satormay-inr} on \pfref{[inr]satisfy2}}
              \end{pfsteps*} 
            \end{byCases}
        \end{byCases}
      
      \restorelocalsteps{1}
      \item[\hxi=\cpair{\hxi_1}{\hxi_2}]
        \begin{pfsteps*}
        \item $\ctruify{\cpair{\hxi_1}{\hxi_2}}=\cpair{\ctruify{\hxi_1}}{\ctruify{\hxi_2}}$ \BY{\autoref{defn:truify}}
        \end{pfsteps*}
        By rule induction over \rulesref{rules:cSatisfy} on \pfref{ccsatisfy}, only one rule applies.
        \begin{byCases}
          \item[\text{(\ref{rule:CSPair})}]
            \begin{pfsteps*}
            \item $e=\hpair{e_1}{e_2}$ \BY{assumption}
            \item $\ccsatisfy{e_1}{\cfalsify{\hxi_1}}$ \BY{assumption} \pflabel{[pair]satisfy-falsify1}
            \item $\ccsatisfy{e_2}{\cfalsify{\hxi_2}}$ \BY{assumption} \pflabel{[pair]satisfy-falsify2}
            \end{pfsteps*}
            By rule induction over \rulesref{rules:Value} on \pfref{val}, only one rule applies.
            \begin{byCases}
              \item[\text{(\ref{rule:VPair})}]
              \begin{pfsteps*}
              \item $\isVal{e_1}$ \BY{assumption} \pflabel{[pair]val1}
              \item $\isVal{e_2}$ \BY{assumption} \pflabel{[pair]val2}
              \item $\csatisfyormay{e_1}{\hxi_1}$ \BY{IH on \pfref{[pair]val1} and \pfref{[pair]satisfy-falsify1}} \pflabel{[pair]satisfy1}
              \item $\csatisfyormay{e_2}{\hxi_2}$ \BY{IH on \pfref{[pair]val2} and \pfref{[pair]satisfy-falsify2}} \pflabel{[pair]satisfy2}
              \item $\csatisfyormay{\hpair{e_1}{e_2}}{\cpair{\hxi_1}{\hxi_2}}$ \BY{\autoref{lem:satormay-pair} on \pfref{[pair]satisfy1} and \pfref{[pair]satisfy2}}
              \end{pfsteps*}
            \end{byCases}
        \end{byCases}
    \end{byCases}
  \end{enumerate}
  \resetpfcounter
\end{proof}

\begin{lemma}
  \label{lem:satisfy-falsify}
  $\csatisfy{e}{\xi}$ iff $\csatisfy{e}{\cfalsify{\xi}}$
\end{lemma}
\begin{proof}
  We prove sufficiency and necessity separately.
  \begin{enumerate}
    \item Sufficiency:
    \begin{pfsteps*}
    \item $\csatisfy{e}{\xi}$ \BY{assumption} \pflabel{satisfy}
    \end{pfsteps*}
    By rule induction over Rules (\ref{rules:Satisfy}) on \pfref{satisfy}.
    \begin{byCases}
      
      \savelocalsteps{1}
      \item[\text{(\ref{rule:CSTruth})}]
        \begin{pfsteps*}
        \item $\xi = \ctruth$ \BY{assumption}
        \item $\csatisfy{e}{\cfalsify{\ctruth}}$ \BY{\pfref{satisfy} and Definition \ref{defn:falsify}}
        \end{pfsteps*}

      \restorelocalsteps{1}
      \item[\text{(\ref{rule:CSNum})}]
        \begin{pfsteps*}
        \item $\xi = \cnum{n}$ \BY{assumption}
        \item $\csatisfy{e}{\cfalsify{\cnum{n}}}$ \BY{\pfref{satisfy} and Definition \ref{defn:falsify}}
        \end{pfsteps*}

      \restorelocalsteps{1}
      \item[\text{(\ref{rule:CSNotNum})}]
        \begin{pfsteps*}
        \item $\xi = \cnotnum{n}$ \BY{assumption}
        \item $\csatisfy{e}{\cfalsify{\cnotnum{n}}}$ \BY{\pfref{satisfy} and Definition \ref{defn:falsify}}
        \end{pfsteps*}

      \restorelocalsteps{1}
      \item[\text{(\ref{rule:CSAnd})}]
        \begin{pfsteps*}
        \item $\xi = \cand{\xi_1}{\xi_2}$ \BY{assumption}
        \item $\csatisfy{e}{\xi_1}$ \BY{assumption} \pflabel{[and]satisfy1}
        \item $\csatisfy{e}{\xi_2}$ \BY{assumption} \pflabel{[and]satisfy2}
        \item $\csatisfy{e}{\cfalsify{\xi_1}}$ \BY{IH on \pfref{[and]satisfy1}} \pflabel{[and]satisfy-falsify1}
        \item $\csatisfy{e}{\cfalsify{\xi_2}}$ \BY{IH on \pfref{[and]satisfy2}} \pflabel{[and]satisfy-falsify2}
        \item $\csatisfy{e}{\cand{\cfalsify{\xi_1}}{\cfalsify{\xi_2}}}$ \BY{Rule (\ref{rule:CSAnd}) on \pfref{[and]satisfy-falsify1} and \pfref{[and]satisfy-falsify2}} \pflabel{[and]satisfy-and-falsify}
        \item $\csatisfy{e}{\cfalsify{\cand{\xi_1}{\xi_2}}}$ \BY{\pfref{[and]satisfy-and-falsify} and Definition \ref{defn:falsify}}
        \end{pfsteps*}

      \restorelocalsteps{1}
      \item[\text{(\ref{rule:CSOr1})}]
        \begin{pfsteps*}
        \item $\xi = \cor{\xi_1}{\xi_2}$ \BY{assumption}
        \item $\csatisfy{e}{\xi_1}$ \BY{assumption} \pflabel{[or]satisfy1}
        \item $\csatisfy{e}{\cfalsify{\xi_1}}$ \BY{IH on \pfref{[or]satisfy1}} \pflabel{[or]satisfy-falsify1}
        \item $\csatisfy{e}{\cor{\cfalsify{\xi_1}}{\cfalsify{\xi_2}}}$ \BY{Rule (\ref{rule:CSOr1}) on \pfref{[or]satisfy-falsify1}} \pflabel{[or]satisfy-or-falsify}
        \item $\csatisfy{e}{\cfalsify{\cor{\xi_1}{\xi_2}}}$ \BY{\pfref{[or]satisfy-or-falsify} and Definition \ref{defn:falsify}}
        \end{pfsteps*}
      
      \restorelocalsteps{1}
      \item[\text{(\ref{rule:CSOr2})}]
        \begin{pfsteps*}
        \item $\xi = \cor{\xi_1}{\xi_2}$ \BY{assumption}
        \item $\csatisfy{e}{\xi_2}$ \BY{assumption} \pflabel{[or]satisfy2}
        \item $\csatisfy{e}{\cfalsify{\xi_2}}$ \BY{IH on \pfref{[or]satisfy2}} \pflabel{[or]satisfy-falsify2}
        \item $\csatisfy{e}{\cor{\cfalsify{\xi_1}}{\cfalsify{\xi_2}}}$ \BY{Rule (\ref{rule:CSOr2}) on \pfref{[or]satisfy-falsify2}} \pflabel{[or]satisfy-or-falsify'}
        \item $\csatisfy{e}{\cfalsify{\cor{\xi_1}{\xi_2}}}$ \BY{\pfref{[or]satisfy-or-falsify'} and Definition \ref{defn:falsify}}
        \end{pfsteps*}

      \restorelocalsteps{1}
      \item[\text{(\ref{rule:CSInl})}]
        \begin{pfsteps*}
        \item $e = \hinl{\tau_2}{e_1}$ \BY{assumption}
        \item $\xi = \cinl{\xi_1}$ \BY{assumption}
        \item $\csatisfy{e_1}{\xi_1}$ \BY{assumption} \pflabel{[inl]satisfy1}
        \item $\csatisfy{e_1}{\cfalsify{\xi_1}}$ \BY{IH on \pfref{[inl]satisfy1}} \pflabel{[inl]satisfy-falsify1}
        \item $\csatisfy{\hinl{\tau_2}{e_1}}{\cinl{\cfalsify{\xi_1}}}$ \BY{Rule (\ref{rule:CSInl}) on \pfref{[inl]satisfy-falsify1}} \pflabel{[inl]satisfy-inl-falsify}
        \item $\csatisfy{\hinl{\tau_2}{e_1}}{\cfalsify{\cinl{\xi_1}}}$ \BY{\pfref{[inl]satisfy-inl-falsify} and Definition \ref{defn:falsify}}
        \end{pfsteps*}

      \restorelocalsteps{1}
      \item[\text{(\ref{rule:CSInr})}]
        \begin{pfsteps*}
        \item $e = \hinr{\tau_1}{e_2}$ \BY{assumption}
        \item $\xi = \cinr{\xi_2}$ \BY{assumption}
        \item $\csatisfy{e_2}{\xi_2}$ \BY{assumption} \pflabel{[inr]satisfy2}
        \item $\csatisfy{e_2}{\cfalsify{\xi_2}}$ \BY{IH on \pfref{[inr]satisfy2}} \pflabel{[inr]satisfy-falsify2}
        \item $\csatisfy{\hinr{\tau_1}{e_2}}{\cinr{\cfalsify{\xi_2}}}$ \BY{Rule (\ref{rule:CSInr}) on \pfref{[inr]satisfy-falsify2}} \pflabel{[inr]satisfy-inr-falsify}
        \item $\csatisfy{\hinr{\tau_1}{e_2}}{\cfalsify{\cinr{\xi_2}}}$ \BY{\pfref{[inr]satisfy-inr-falsify} and Definition \ref{defn:falsify}}
        \end{pfsteps*}
      
      \restorelocalsteps{1}
      \item[\text{(\ref{rule:CSPair})}]
        \begin{pfsteps*}
        \item $e = \hpair{e_1}{e_2}$ \BY{assumption}
        \item $\xi = \cpair{\xi_1}{\xi_2}$ \BY{assumption}
        \item $\csatisfy{e_1}{\xi_1}$ \BY{assumption} \pflabel{[pair]satisfy1}
        \item $\csatisfy{e_2}{\xi_2}$ \BY{assumption} \pflabel{[pair]satisfy2}
        \item $\csatisfy{e_1}{\cfalsify{\xi_1}}$ \BY{IH on \pfref{[pair]satisfy1}} \pflabel{[pair]satisfy-falsify1}
        \item $\csatisfy{e_2}{\cfalsify{\xi_2}}$ \BY{IH on \pfref{[pair]satisfy2}} \pflabel{[pair]satisfy-falsify2}
        \item $\csatisfy{\hpair{e_1}{e_2}}{\cpair{\cfalsify{\xi_1}}{\cfalsify{\xi_2}}}$ \BY{Rule (\ref{rule:CSPair}) on \pfref{[pair]satisfy-falsify1} and \pfref{[pair]satisfy-falsify2}} \pflabel{[pair]satisfy-pair-falsify}
        \item $\csatisfy{\hpair{e_1}{e_2}}{\cfalsify{\cpair{\xi_1}{\xi_2}}}$ \BY{\pfref{[pair]satisfy-pair-falsify} and Definition \ref{defn:falsify}}
        \end{pfsteps*}
    \end{byCases}

    \resetpfcounter

    \item Necessity:
    \begin{pfsteps*}
    \item $\csatisfy{e}{\cfalsify{\xi}}$ \BY{assumption} \pflabel{satisfy-falsify}
    \end{pfsteps*}
    By structural induction on $\xi$.
    \begin{byCases}

      \savelocalsteps{1}
      \item[\xi=\ctruth, \cfalsity, \cnum{n}, \cnotnum{n}]
        \begin{pfsteps*}
        \item $\csatisfy{e}{\xi}$ \BY{\pfref{satisfy-falsify} and Definition \ref{defn:falsify}}
        \end{pfsteps*}

      \restorelocalsteps{1}
      \item[\xi=\cunknown]
        \begin{pfsteps*}
        \item $\csatisfy{e}{\cfalsity}$ \BY{\pfref{satisfy-falsify} and Definition \ref{defn:falsify}} \pflabel{satisfy-falsity}
        \item $\cnotsatisfy{e}{\cfalsity}$ \BY{Lemma \ref{lem:no-e-satisfy-falsity}} \pflabel{no-satisfy-falsity}
        \end{pfsteps*}
        \pfref{no-satisfy-falsity} contradicts \pfref{satisfy-falsity}.

      \restorelocalsteps{1}
      \item[\xi=\cand{\xi_1}{\xi_2}]
        \begin{pfsteps*}
        \item $\csatisfy{e}{\cand{\cfalsify{\xi_1}}{\cfalsify{\xi_2}}}$ \BY{\pfref{satisfy-falsify} and Definition \ref{defn:falsify}} \pflabel{satisfy-and-falsify}
        \end{pfsteps*}
        By rule induction over Rules (\ref{rules:Satisfy}) on \pfref{satisfy-and-falsify} and only case applies.
        \begin{byCases}
          \item[\text{(\ref{rule:CSAnd})}]
            \begin{pfsteps*}
            \item $\csatisfy{e}{\cfalsify{\xi_1}}$ \BY{assumption} \pflabel{[and]satisfy-falsify1}
            \item $\csatisfy{e}{\cfalsify{\xi_2}}$ \BY{assumption} \pflabel{[and]satisfy-falsify2}
            \item $\csatisfy{e}{\xi_1}$ \BY{IH on \pfref{[and]satisfy-falsify1}} \pflabel{[and]satisfy1}
            \item $\csatisfy{e}{\xi_2}$ \BY{IH on \pfref{[and]satisfy-falsify2}} \pflabel{[and]satisfy2}
            \item $\csatisfy{e}{\cand{\xi_1}{\xi_2}}$ \BY{Rule (\ref{rule:CSAnd}) on \pfref{[and]satisfy1} and \pfref{[and]satisfy2}}
            \end{pfsteps*}
        \end{byCases}

      \restorelocalsteps{1}
      \item[\xi = \cor{\xi_1}{\xi_2}]
        \begin{pfsteps*}
        \item $\csatisfy{e}{\cor{\cfalsify{\xi_1}}{\cfalsify{\xi_2}}}$ \BY{\pfref{satisfy-falsify} and Definition \ref{defn:falsify}} \pflabel{satisfy-or-falsify}
        \end{pfsteps*}
        By rule induction over Rules (\ref{rules:Satisfy}) on \pfref{satisfy-or-falsify} and only two cases apply.
        \begin{byCases}

          \savelocalsteps{2}
          \item[\text{(\ref{rule:CSOr1})}]
            \begin{pfsteps*}
            \item $\csatisfy{e}{\cfalsify{\xi_1}}$ \BY{assumption} \pflabel{[or]satisfy-falsify1}
            \item $\csatisfy{e}{\xi_1}$ \BY{IH on \pfref{[or]satisfy-falsify1}} \pflabel{[or]satisfy1}
            \item $\csatisfy{e}{\cor{\xi_1}{\xi_2}}$ \BY{Rule (\ref{rule:CSOr1}) on \pfref{[or]satisfy1}}
            \end{pfsteps*}

          \restorelocalsteps{2}
          \item[\text{(\ref{rule:CSOr2})}]
            \begin{pfsteps*}
            \item $\csatisfy{e}{\cfalsify{\xi_2}}$ \BY{assumption} \pflabel{[or]satisfy-falsify2}
            \item $\csatisfy{e}{\xi_2}$ \BY{IH on \pfref{[or]satisfy-falsify2}} \pflabel{[or]satisfy2}
            \item $\csatisfy{e}{\cor{\xi_1}{\xi_2}}$ \BY{Rule (\ref{rule:CSOr2}) on \pfref{[or]satisfy2}}
            \end{pfsteps*}
        \end{byCases}
      
      \restorelocalsteps{1}
      \item[\xi=\cinl{\xi_1}]
        \begin{pfsteps*}
        \item $\csatisfy{e}{\cinl{\cfalsify{\xi_1}}}$ \BY{\pfref{satisfy-falsify} and Definition \ref{defn:falsify}} \pflabel{satisfy-inl-falsify}
        \end{pfsteps*}
        By rule induction over Rules (\ref{rules:Satisfy}) on \pfref{satisfy-inl-falsify} and only one case applies.
        \begin{byCases}
          \item[\text{(\ref{rule:CSInl})}]
            \begin{pfsteps*}
            \item $e = \hinl{\tau_2}{e_1}$ \BY{assumption}
            \item $\csatisfy{e_1}{\cfalsify{\xi_1}}$ \BY{assumption} \pflabel{[inl]satisfy-falsify1}
            \item $\csatisfy{e_1}{\xi_1}$ \BY{IH on \pfref{[inl]satisfy-falsify1}} \pflabel{[inl]satisfy1}
            \item $\csatisfy{e}{\cinl{\xi_1}}$ \BY{Rule (\ref{rule:CSInl}) on \pfref{[inl]satisfy1}}
            \end{pfsteps*} 
        \end{byCases}

      \restorelocalsteps{1}
      \item[\xi=\cinr{\xi_2}]
        \begin{pfsteps*}
        \item $\csatisfy{e}{\cinr{\cfalsify{\xi_2}}}$ \BY        {\pfref{satisfy-falsify} and Definition \ref{defn:falsify}} \pflabel{satisfy-inr-falsify}
        \end{pfsteps*}
        By rule induction over Rules (\ref{rules:Satisfy}) on \pfref{satisfy-inr-falsify} and only one case applies.
        \begin{byCases}
          \item[\text{(\ref{rule:CSInr})}]
            \begin{pfsteps*}
            \item $e = \hinr{\tau_1}{e_2}$ \BY{assumption}
            \item $\csatisfy{e_2}{\cfalsify{\xi_2}}$ \BY{assumption} \pflabel{[inr]satisfy-falsify2}
            \item $\csatisfy{e_2}{\xi_2}$ \BY{IH on \pfref{[inr]satisfy-falsify2}} \pflabel{[inr]satisfy2}
            \item $\csatisfy{e}{\cinr{\xi_2}}$ \BY{Rule (\ref{rule:CSInr}) on \pfref{[inr]satisfy2}}
            \end{pfsteps*} 
        \end{byCases}
      
      \restorelocalsteps{1}
      \item[\xi=\cpair{\xi_1}{\xi_2}]
        \begin{pfsteps*}
        \item $\csatisfy{e}{\cpair{\cfalsify{\xi_1}}{\cfalsify{\xi_2}}}$ \BY{\pfref{satisfy-falsify} and Definition \ref{defn:falsify}} \pflabel{satisfy-pair-falsify}
        \end{pfsteps*}
        By rule induction over Rules (\ref{rules:Satisfy}) on \pfref{satisfy-pair-falsify} and only case applies.
        \begin{byCases}
          \item[\text{(\ref{rule:CSPair})}]
            \begin{pfsteps*}
            \item $e=\hpair{e_1}{e_2}$ \BY{assumption}
            \item $\csatisfy{e_1}{\cfalsify{\xi_1}}$ \BY{assumption} \pflabel{[pair]satisfy-falsify1}
            \item $\csatisfy{e_2}{\cfalsify{\xi_2}}$ \BY{assumption} \pflabel{[pair]satisfy-falsify2}
            \item $\csatisfy{e_1}{\xi_1}$ \BY{IH on \pfref{[pair]satisfy-falsify1}} \pflabel{[pair]satisfy1}
            \item $\csatisfy{e_2}{\xi_2}$ \BY{IH on \pfref{[pair]satisfy-falsify2}} \pflabel{[pair]satisfy2}
            \item $\csatisfy{e}{\cpair{\xi_1}{\xi_2}}$ \BY{Rule (\ref{rule:CSPair}) on \pfref{[pair]satisfy1} and \pfref{[pair]satisfy2}}
            \end{pfsteps*}
        \end{byCases} 
    \end{byCases}
    \resetpfcounter
  \end{enumerate}
\end{proof}
\pagebreak
\section{Dynamic Semantics}
\judgboxa{\isVal{e}}{$e$ is a value}
\begin{subequations}\label{rules:Value}
\begin{equation}\label{rule:VNum}
\inferrule[VNum]{ }{
  \isVal{\hnum{n}}
}
\end{equation}
\begin{equation}\label{rule:VLam}
\inferrule[VLam]{ }{
  \isVal{\hlam{x}{\tau}{e}}
}
\end{equation}
\begin{equation}\label{rule:VPair}
\inferrule[VPair]{
  \isVal{e_1} \\
  \isVal{e_2}
}{\isVal{\hpair{e_1}{e_2}}}
\end{equation}
\begin{equation}\label{rule:VInl}
\inferrule[VInl]{
  \isVal{e}
}{
  \isVal{\hinl{\tau}{e}}
}
\end{equation}
\begin{equation}\label{rule:VInr}
\inferrule[VInr]{
  \isVal{e}
}{
  \isVal{\hinr{\tau}{e}}
}
\end{equation}
\end{subequations}

\judgboxa{\isIndet{e}}{$e$ is indeterminate}
\begin{subequations}\label{rules:Indet}
\begin{equation}\label{rule:IEHole}
\inferrule[IEHole]{ }{
  \isIndet{\hehole{u}}
}
\end{equation}
\begin{equation}\label{rule:IHole}
\inferrule[IHole]{
  \isFinal{e}
}{
  \isIndet{\hhole{e}{u}}
}
\end{equation}
\begin{equation}\label{rule:IAp}
\inferrule[IAp]{
  \isIndet{e_1} \\ \isFinal{e_2}
}{
  \isIndet{\hap{e_1}{e_2}}
}
\end{equation}
\begin{equation}\label{rule:IPairL}
\inferrule[IPairL]{
  \isIndet{e_1} \\ \isVal{e_2}
}{
  \isIndet{\hpair{e_1}{e_2}}
}
\end{equation}
\begin{equation}\label{rule:IPairR}
\inferrule[IPairR]{
  \isVal{e_1} \\
  \isIndet{e_2}
}{
  \isIndet{\hpair{e_1}{e_2}}
}
\end{equation}
\begin{equation}\label{rule:IPair}
\inferrule[IPair]{
  \isIndet{e_1} \\ \isIndet{e_2}
}{
  \isIndet{\hpair{e_1}{e_2}}
}
\end{equation}
\begin{equation}\label{rule:IPrl}
\inferrule[IPrl]{
  \isIndet{e}
}{
  \isIndet{\hprl{e}}
}
\end{equation}
\begin{equation}\label{rule:IPrr}
\inferrule[IPrr]{
  \isIndet{e}
}{
  \isIndet{\hprr{e}}
}
\end{equation}
\begin{equation}\label{rule:IInl}
\inferrule[IInL]{
  \isIndet{e}
}{
  \isIndet{\hinl{\tau}{e}}
}
\end{equation}
\begin{equation}\label{rule:IInr}
\inferrule[IInR]{
  \isIndet{e}
}{
  \isIndet{\hinr{\tau}{e}}
}
\end{equation}
\begin{equation}\label{rule:IMatch}
\inferrule[IMatch]{
  \isFinal{e} \\
  \hmaymatch{e}{p_r}
}{
  \isIndet{
    \hmatch{e}{\zruls{rs_{pre}}{\hrulP{p_r}{e_r}}{rs_{post}}}
  }
}
\end{equation}
\end{subequations}

\judgboxa{\isFinal{e}}{$e$ is final}
\begin{subequations}\label{rules:Final}
  \begin{equation}\label{rule:FVal}
\inferrule[FVal]{
  \isVal{e}
}{
  \isFinal{e}
}
\end{equation}
\begin{equation}\label{rule:FIndet}
\inferrule[FIndet]{
  \isIndet{e}
}{
  \isFinal{e}
}
\end{equation}
\end{subequations}

\judgboxa{
  \notIntro{e}
}{
  $e$ cannot be a value syntactically
}
\begin{subequations}\label{rules:notintro}
\begin{equation}\label{rule:NVEHole}
\inferrule[NVEHole]{ }{
  \notIntro{\hehole{u}}
}
\end{equation}
\begin{equation}\label{rule:NVHole}
\inferrule[NVHole]{ }{
  \notIntro{\hhole{e}{u}}
}
\end{equation}
\begin{equation}\label{rule:NVAp}
\inferrule[NVAp]{ }{
  \notIntro{\hap{e_1}{e_2}}
}
\end{equation}
\begin{equation}\label{rule:NVMatch}
\inferrule[NVMatch]{ }{
  \notIntro{\hmatch{e}{\zrules}}
}
\end{equation}
\begin{equation}\label{rule:NVPrl}
\inferrule[NVPrl]{ }{
  \notIntro{\hprl{e}}
}
\end{equation}
\begin{equation}\label{rule:NVPrr}
\inferrule[NVPrr]{ }{
  \notIntro{\hprr{e}}
}
\end{equation}
\end{subequations}

\judgboxa{\fnotIntro{e}}{}
\begin{subequations}\label{defn:notintro}
  \begin{align}
    \fnotIntro{\hehole{u}} ={}& \true \label{defn:ehole-notintro}\\
    \fnotIntro{\hhole{e}{u}} ={}& \true \label{defn:hole-notintro}\\
    \fnotIntro{\hap{e_1}{e_2}} ={}& \true \label{defn:ap-notintro}\\
    \fnotIntro{\hmatch{e}{\zrules}} ={}& \true \label{defn:match-notintro}\\
    \fnotIntro{\hprl{e}} ={}& \true \label{defn:prl-notintro}\\
    \fnotIntro{\hprr{e}} ={}& \true \label{defn:prr-notintro}\\
    \text{Otherwise}\quad \fnotIntro{e} ={}& \false \label{defn:not-notintro}
  \end{align}
\end{subequations}

\begin{lemma}[Soundness and Completeness of NotIntro Judgment]
  \label{lem:sound-complete-notintro}
  $\notIntro{e}$ iff $\fnotIntro{e}$.
\end{lemma}
\begin{proof}
  TODO
\end{proof}

\judgboxa{
  \inValues{e'}{e}
}{
  $e'$ is one of the possible values of $e$
}
\begin{subequations}\label{rules:inValues}
  \begin{equation}\label{rule:IVVal}
    \inferrule[IVVal]{
      \isVal{e} \\
      \hexptyp{\cdot}{\Delta}{e}{\tau}
    }{
      \inValues{e}{e}
    }
  \end{equation}
  \begin{equation}\label{rule:IVIndet}
    \inferrule[IVIndet]{
      \notIntro{e} \\
      \hexptyp{\cdot}{\Delta}{e}{\tau} \\
      \isVal{e'} \\
      \hexptyp{\cdot}{\Delta}{e'}{\tau}
    }{
      \inValues{e'}{e}
    }
  \end{equation}
  \begin{equation}\label{rule:IVInl}
    \inferrule[IVInl]{
      \isIndet{\hinl{\tau_2}{e_1}} \\
      \hexptyp{\cdot}{\Delta}{\hinl{\tau_2}{e_1}}{\tau} \\
      \inValues{e_1'}{e_1} \\
    }{ 
      \inValues{\hinl{\tau_2}{e_1'}}{\hinl{\tau_2}{e_1}}
    }
  \end{equation}
  \begin{equation}\label{rule:IVInr}
    \inferrule[IVInr]{
      \isIndet{\hinr{\tau_1}{e_2}} \\
      \hexptyp{\cdot}{\Delta}{\hinr{\tau_1}{e_2}}{\tau} \\
      \inValues{e_2'}{e_2} \\
    }{ 
      \inValues{\hinr{\tau_1}{e_2'}}{\hinr{\tau_1}{e_2}}
    }
  \end{equation}
  \begin{equation}\label{rule:IVPair}
    \inferrule[IVPair]{
      \isIndet{\hpair{e_1}{e_2}} \\
      \hexptyp{\cdot}{\Delta}{\hpair{e_1}{e_2}}{\tau} \\
      \inValues{e_1'}{e_1} \\
      \inValues{e_2'}{e_2}
    }{
      \inValues{\hpair{e_1'}{e_2'}}{\hpair{e_1}{e_2}}
    }
  \end{equation}
\end{subequations}

\begin{lemma}
  \label{lem:invalues-typ}
  If $\inValues{e'}{e}$ and $\hexptyp{\cdot}{\Delta}{e}{\tau}$ then $\hexptyp{\cdot}{\Delta}{e'}{\tau}$.
\end{lemma}

\begin{lemma}
  \label{lem:invalues-val}
  If $\inValues{e'}{e}$ then $\isVal{e'}$.
\end{lemma}

\begin{lemma}
  \label{lem:invalues-derivable}
  If $\isIndet{e}$ then there exists $e'$ such that $\inValues{e'}{e}$.
\end{lemma}

\begin{lemma}
  \label{lem:complete-not-satormay}
  If $\isIndet{e}$ and $\hexptyp{\cdot}{\Delta}{e}{\tau}$ and $\ctyp{\hxi}{\tau}$ and $\cnotsatisfyormay{e}{\hxi}$
  then $\cnotsatisfyormay{e'}{\hxi}$ whenever $\inValues{e'}{e}$.
\end{lemma}
\begin{proof}
  \begin{pfsteps}
  \item \isIndet{e} \BY{assumption} \pflabel{eindet}
  \item \hexptyp{\cdot}{\Delta}{e}{\tau} \BY{assumption} \pflabel{etyp}
  \item \ctyp{\hxi}{\tau} \BY{assumption} \pflabel{ctyp}
  \item \cnotsatisfyormay{e}{\hxi} \BY{assumption} \pflabel{not-satormay}
  \end{pfsteps}
  By rule induction over \rulesref{rules:CTyp} on \pfref{ctyp}.
  \begin{byCases}
    \savelocalsteps{0}
    \item[\text{(\ref{rule:CTTruth})}]
    \begin{pfsteps*}
    \item $\hxi = \ctruth$ \BY{assumption}
    \item $\csatisfy{e}{\ctruth}$ \BY{\ruleref{rule:CSTruth}} \pflabel{[truth]sat}
    \item $\csatisfyormay{e}{\ctruth}$ \BY{\ruleref{rule:CSMSSat} on \pfref{[truth]sat}}
    \end{pfsteps*}
    Contradicts \pfref{not-satormay}.
    \restorelocalsteps{0}
    \item[\text{(\ref{rule:CTUnknown})}]
    \begin{pfsteps*}
    \item $\hxi = \cunknown$ \BY{assumption}
    \item $\cmaysatisfy{e}{\cunknown}$ \BY{\ruleref{rule:CMSUnknown}} \pflabel{[unknown]maysat}
    \item $\csatisfyormay{e}{\cunknown}$ \BY{\ruleref{rule:CSMSMay} on \pfref{[unknown]maysat}}
    \end{pfsteps*}
    Contradicts \pfref{not-satormay}.
    \restorelocalsteps{0}
    \item[\text{(\ref{rule:CTNum})}]
    \begin{pfsteps*}
    \item $\hxi = \cnum{n}$ \BY{assumption}
    \item $\tau = \tnum$ \BY{assumption}
    \item $\refutable{\cnum{n}}$ \BY{\ruleref{rule:RXNum}} \pflabel{[num]refutable}
    \end{pfsteps*}
    By rule induction over \rulesref{rules:Indet} on \pfref{eindet}.
    \begin{byCases}
      \savelocalsteps{1}
      \item[\text{(\ref{rule:IEHole})}]
      \begin{pfsteps*}
      \item $e=\hehole{u}$ \BY{assumption}
      \item $\notIntro{\hehole{u}}$ \BY{\ruleref{rule:NVEHole}} \pflabel{[ehole]notintro}
      \item $\cmaysatisfy{\hehole{u}}{\cnum{n}}$ \BY{\ruleref{rule:CMSNotIntro} on \pfref{[ehole]notintro} and \pfref{[num]refutable}} \pflabel{[ehole][num]maysat}
      \item $\csatisfyormay{\hehole{u}}{\cnum{n}}$ \BY{\ruleref{rule:CSMSMay} on \pfref{[ehole][num]maysat}}
      \end{pfsteps*}
      Contradicts \pfref{not-satormay}.
      \restorelocalsteps{1}
      \item[\text{(\ref{rule:IHole})}]
      \begin{pfsteps*}
      \item $e=\hhole{e_1}{u}$ \BY{assumption}
      \item $\notIntro{\hhole{e_1}{u}}$ \BY{\ruleref{rule:NVHole}} \pflabel{[hole]notintro}
      \item $\cmaysatisfy{\hhole{e_1}{u}}{\cnum{n}}$ \BY{\ruleref{rule:CMSNotIntro} on \pfref{[hole]notintro} and \pfref{[num]refutable}} \pflabel{[hole][num]maysat}
      \item $\csatisfyormay{\hhole{e_1}{u}}{\cnum{n}}$ \BY{\ruleref{rule:CSMSMay} on \pfref{[hole][num]maysat}}
      \end{pfsteps*} 
      Contradicts \pfref{not-satormay}.
      \restorelocalsteps{1}
      \item[\text{(\ref{rule:IAp})}]
      \begin{pfsteps*}
      \item $e=\hap{e_1}{e_2}$ \BY{assumption}
      \item $\notIntro{\hap{e_1}{e_2}}$ \BY{\ruleref{rule:NVAp}} \pflabel{[ap]notintro}
      \item $\cmaysatisfy{\hap{e_1}{e_2}}{\cnum{n}}$ \BY{\ruleref{rule:CMSNotIntro} on \pfref{[ap]notintro} and \pfref{[num]refutable}} \pflabel{[ap][num]maysat}
      \item $\csatisfyormay{\hap{e_1}{e_2}}{\cnum{n}}$ \BY{\ruleref{rule:CSMSMay} on \pfref{[ap][num]maysat}}
      \end{pfsteps*} 
      Contradicts \pfref{not-satormay}.
      \restorelocalsteps{1}
      \item[\text{(\ref{rule:IPrl})}]
      \begin{pfsteps*}
      \item $e=\hprl{e_1}$ \BY{assumption}
      \item $\notIntro{\hprl{e_1}}$ \BY{\ruleref{rule:NVPrl}} \pflabel{[prl]notintro}
      \item $\cmaysatisfy{\hprl{e_1}}{\cnum{n}}$ \BY{\ruleref{rule:CMSNotIntro} on \pfref{[prl]notintro} and \pfref{[num]refutable}} \pflabel{[prl][num]maysat}
      \item $\csatisfyormay{\hprl{e_1}}{\cnum{n}}$ \BY{\ruleref{rule:CSMSMay} on \pfref{[prl][num]maysat}}
      \end{pfsteps*} 
      Contradicts \pfref{not-satormay}.
      \restorelocalsteps{1}
      \item[\text{(\ref{rule:IPrr})}]
      \begin{pfsteps*}
      \item $e=\hprr{e_1}$ \BY{assumption}
      \item $\notIntro{\hprr{e_1}}$ \BY{\ruleref{rule:NVPrr}} \pflabel{[prr]notintro}
      \item $\cmaysatisfy{\hprr{e_1}}{\cnum{n}}$ \BY{\ruleref{rule:CMSNotIntro} on \pfref{[prr]notintro} and \pfref{[num]refutable}} \pflabel{[prr][num]maysat}
      \item $\csatisfyormay{\hprr{e_1}}{\cnum{n}}$ \BY{\ruleref{rule:CSMSMay} on \pfref{[prr][num]maysat}}
      \end{pfsteps*} 
      Contradicts \pfref{not-satormay}.
      \restorelocalsteps{1}
      \item[\text{(\ref{rule:IMatch})}]
      \begin{pfsteps*}
      \item $e=\hmatch{e_1}{\zrules}$ \BY{assumption}
      \item $\notIntro{\hmatch{e_1}{\zrules}}$ \BY{\ruleref{rule:NVMatch}} \pflabel{[match]notintro}
      \item $\cmaysatisfy{\hmatch{e_1}{\zrules}}{\cnum{n}}$ \BY{\ruleref{rule:CMSNotIntro} on \pfref{[match]notintro} and \pfref{[num]refutable}} \pflabel{[match][num]maysat}
      \item $\csatisfyormay{\hmatch{e_1}{\zrules}}{\cnum{n}}$ \BY{\ruleref{rule:CSMSMay} on \pfref{[match][num]maysat}}
      \end{pfsteps*} 
      Contradicts \pfref{not-satormay}.
      \restorelocalsteps{1}
      \item[\text{(\ref{rule:IPairL}), (\ref{rule:IPairR}), (\ref{rule:IPair})}] 
      \begin{pfsteps*}
      \item $e=\hpair{e_1}{e_2}$ \BY{assumption}
      \end{pfsteps*} 
      By rule induction over \rulesref{rules:TExp} on \pfref{etyp}, no rule applies due to syntactic contradiction.
      \restorelocalsteps{1}
      \item[\text{(\ref{rule:IInl})}] 
      \begin{pfsteps*}
      \item $e=\hinl{\tau_2}{e_1}$ \BY{assumption}
      \end{pfsteps*} 
      By rule induction over \rulesref{rules:TExp} on \pfref{etyp}, no rule applies due to syntactic contradiction.
      \restorelocalsteps{1}
      \item[\text{(\ref{rule:IInr})}] 
      \begin{pfsteps*}
      \item $e=\hinr{\tau_1}{e_2}$ \BY{assumption}
      \end{pfsteps*} 
      By rule induction over \rulesref{rules:TExp} on \pfref{etyp}, no rule applies due to syntactic contradiction.
    \end{byCases}
    \restorelocalsteps{0}
    \item[\text{(\ref{rule:CTInl})}]
    \begin{pfsteps*}
      \item $\hxi = \cinl{\hxi_1}$ \BY{assumption}
      \item $\tau = \tsum{\tau_1}{\tau_2}$ \BY{assumption}
      \item $\ctyp{\hxi_1}{\tau_1}$ \BY{assumption} \pflabel{[inl]c1typ}
      \item $\refutable{\cinl{\hxi_1}}$ \BY{\ruleref{rule:RXInl}} \pflabel{[inl]refutable}
      \end{pfsteps*}
      By rule induction over \rulesref{rules:Indet} on \pfref{eindet}.
      \begin{byCases}
        \savelocalsteps{1}
        \item[\text{(\ref{rule:IEHole})}]
        \begin{pfsteps*}
        \item $e=\hehole{u}$ \BY{assumption}
        \item $\notIntro{\hehole{u}}$ \BY{\ruleref{rule:NVEHole}} \pflabel{[ehole][inl]notintro}
        \item $\cmaysatisfy{\hehole{u}}{\cinl{\hxi_1}}$ \BY{\ruleref{rule:CMSNotIntro} on \pfref{[ehole][inl]notintro} and \pfref{[inl]refutable}} \pflabel{[ehole][inl]maysat}
        \item $\csatisfyormay{\hehole{u}}{\cinl{\hxi_1}}$ \BY{\ruleref{rule:CSMSMay} on \pfref{[ehole][inl]maysat}}
        \end{pfsteps*}
        Contradicts \pfref{not-satormay}.
        \restorelocalsteps{1}
        \item[\text{(\ref{rule:IHole})}]
        \begin{pfsteps*}
        \item $e=\hhole{e_1}{u}$ \BY{assumption}
        \item $\notIntro{\hhole{e_1}{u}}$ \BY{\ruleref{rule:NVHole}} \pflabel{[hole][inl]notintro}
        \item $\cmaysatisfy{\hhole{e_1}{u}}{\cinl{\hxi_1}}$ \BY{\ruleref{rule:CMSNotIntro} on \pfref{[hole][inl]notintro} and \pfref{[inl]refutable}} \pflabel{[hole][inl]maysat}
        \item $\csatisfyormay{\hhole{e_1}{u}}{\cinl{\hxi_1}}$ \BY{\ruleref{rule:CSMSMay} on \pfref{[hole][inl]maysat}}
        \end{pfsteps*} 
        Contradicts \pfref{not-satormay}.
        \restorelocalsteps{1}
        \item[\text{(\ref{rule:IAp})}]
        \begin{pfsteps*}
        \item $e=\hap{e_1}{e_2}$ \BY{assumption}
        \item $\notIntro{\hap{e_1}{e_2}}$ \BY{\ruleref{rule:NVAp}} \pflabel{[ap][inl]notintro}
        \item $\cmaysatisfy{\hap{e_1}{e_2}}{\cinl{\hxi_1}}$ \BY{\ruleref{rule:CMSNotIntro} on \pfref{[ap][inl]notintro} and \pfref{[inl]refutable}} \pflabel{[ap][inl]maysat}
        \item $\csatisfyormay{\hap{e_1}{e_2}}{\cinl{\hxi_1}}$ \BY{\ruleref{rule:CSMSMay} on \pfref{[ap][inl]maysat}}
        \end{pfsteps*} 
        Contradicts \pfref{not-satormay}.
        \restorelocalsteps{1}
        \item[\text{(\ref{rule:IPrl})}]
        \begin{pfsteps*}
        \item $e=\hprl{e_1}$ \BY{assumption}
        \item $\notIntro{\hprl{e_1}}$ \BY{\ruleref{rule:NVPrl}} \pflabel{[prl][inl]notintro}
        \item $\cmaysatisfy{\hprl{e_1}}{\cinl{\hxi_1}}$ \BY{\ruleref{rule:CMSNotIntro} on \pfref{[prl][inl]notintro} and \pfref{[inl]refutable}} \pflabel{[prl][inl]maysat}
        \item $\csatisfyormay{\hprl{e_1}}{\cinl{\hxi_1}}$ \BY{\ruleref{rule:CSMSMay} on \pfref{[prl][inl]maysat}}
        \end{pfsteps*} 
        Contradicts \pfref{not-satormay}.
        \restorelocalsteps{1}
        \item[\text{(\ref{rule:IPrr})}]
        \begin{pfsteps*}
        \item $e=\hprr{e_1}$ \BY{assumption}
        \item $\notIntro{\hprr{e_1}}$ \BY{\ruleref{rule:NVPrr}} \pflabel{[prr][inl]notintro}
        \item $\cmaysatisfy{\hprr{e_1}}{\cinl{\hxi_1}}$ \BY{\ruleref{rule:CMSNotIntro} on \pfref{[prr][inl]notintro} and \pfref{[inl]refutable}} \pflabel{[prr][inl]maysat}
        \item $\csatisfyormay{\hprr{e_1}}{\cinl{\hxi_1}}$ \BY{\ruleref{rule:CSMSMay} on \pfref{[prr][inl]maysat}}
        \end{pfsteps*} 
        Contradicts \pfref{not-satormay}.
        \restorelocalsteps{1}
        \item[\text{(\ref{rule:IMatch})}]
        \begin{pfsteps*}
        \item $e=\hmatch{e_1}{\zrules}$ \BY{assumption}
        \item $\notIntro{\hmatch{e_1}{\zrules}}$ \BY{\ruleref{rule:NVMatch}} \pflabel{[match][inl]notintro}
        \item $\cmaysatisfy{\hmatch{e_1}{\zrules}}{\cinl{\hxi_1}}$ \BY{\ruleref{rule:CMSNotIntro} on \pfref{[match][inl]notintro} and \pfref{[inl]refutable}} \pflabel{[match][inl]maysat}
        \item $\csatisfyormay{\hmatch{e_1}{\zrules}}{\cinl{\hxi_1}}$ \BY{\ruleref{rule:CSMSMay} on \pfref{[match][inl]maysat}}
        \end{pfsteps*} 
        Contradicts \pfref{not-satormay}.
        \restorelocalsteps{1}
        \item[\text{(\ref{rule:IPairL}), (\ref{rule:IPairR}), (\ref{rule:IPair})}] 
        \begin{pfsteps*}
        \item $e=\hpair{e_1}{e_2}$ \BY{assumption}
        \end{pfsteps*} 
        By rule induction over \rulesref{rules:Indet} on \pfref{eindet}, no rule applies due to syntactic contradiction.
        \restorelocalsteps{1}
        \item[\text{(\ref{rule:IInl})}] 
        \begin{pfsteps*}
        \item $e=\hinl{\tau_2'}{e_1}$ \BY{assumption}
        \item $\isIndet{e_1}$ \BY{assumption} \pflabel{[inl]e1indet}
        \end{pfsteps*} 
        By rule induction over \rulesref{rules:TExp} on \pfref{etyp}, only one rule applies.
        \begin{byCases}
          \item[\text{(\ref{rule:TInl})}]
          \begin{pfsteps*}
          \item $\tau_2'=\tau_2$ \BY{assumption}
          \item $\hexptyp{\cdot}{\Delta}{e_1}{\tau_1}$ \BY{assumption} \pflabel{[inl]e1typ}
          \item $\cnotsatisfyormay{e_1}{\hxi_1}$ \BY{\autoref{lem:satormay-inl} on \pfref{not-satormay}} \pflabel{[inl]not-satormay1}
          \item if $\inValues{e_1'}{e_1}$ then $\cnotsatisfyormay{e_1'}{\hxi_1}$ \BY{IH on \pfref{[inl]e1indet} and \pfref{[inl]e1typ} and \pfref{[inl]c1typ} and \pfref{[inl]not-satormay1}} \pflabel{[inl]forall1}
          \end{pfsteps*} 
          To show if $\inValues{e'}{\hinl{\tau_2}{e_1}}$ then $\cnotsatisfyormay{e'}{\cinl{\hxi_1}}$, we assume $\inValues{e'}{\hinl{\tau_2}{e_1}}$.
          \begin{pfsteps*}
          \item $\inValues{e'}{\hinl{\tau_2}{e_1}}$ \BY{assumption} \pflabel{[inl]invalues}
          \end{pfsteps*}
          By rule induction over \rulesref{rules:inValues} on \pfref{[inl]invalues}.
          \begin{byCases}
            \savelocalsteps{2}
            \item[\text{(\ref{rule:IVVal})}]
            \begin{pfsteps*}
            \item $\isVal{\hinl{\tau_2}{e_1}}$ \BY{assumption}
            \end{pfsteps*} 
            Contradicts \pfref{eindet} by \autoref{lem:val-not-indet}.
            \restorelocalsteps{2} 
            \item[\text{(\ref{rule:IVIndet})}] 
            \begin{pfsteps*}
            \item $\notIntro{\hinl{\tau_2}{e_1}}$ \BY{assumption}
            \end{pfsteps*} 
            Contradicts \autoref{lem:no-inl-notintro}
            \restorelocalsteps{2}
            \item[\text{(\ref{rule:IVInl})}] 
            \begin{pfsteps*}
            \item $e'=\hinl{\tau_2}{e_1'}$ \BY{assumption}
            \item $\inValues{e_1'}{e_1}$ \BY{assumption} \pflabel{[inl]invalues1}
            \item $\cnotsatisfyormay{e_1'}{\hxi_1}$ \BY{\pfref{[inl]forall1} on \pfref{[inl]invalues1}} \pflabel{[inl]not-satormay1'}
            \item $\cnotsatisfyormay{\hinl{\tau_2}{e_1'}}{\cinl{\hxi_1}}$ \BY{\autoref{lem:satormay-inl} on \pfref{[inl]not-satormay1'}}
            \end{pfsteps*} 
          \end{byCases}
        \end{byCases}
        \restorelocalsteps{1}
        \item[\text{(\ref{rule:IInr})}] 
        \begin{pfsteps*}
        \item $e=\hinr{\tau_1}{e_2}$ \BY{assumption}
        \end{pfsteps*}
        To show if $\inValues{e'}{\hinr{\tau_1}{e_2}}$ then $\cnotsatisfyormay{e'}{\cinl{\hxi_1}}$, we assume $\inValues{e'}{\hinr{\tau_1}{e_2}}$.
        \begin{pfsteps*}
        \item $\inValues{e'}{\hinr{\tau_1}{e_2}}$ \BY{assumption} \pflabel{[inr]invalues}
        \end{pfsteps*}
        By rule induction over \rulesref{rules:inValues} on \pfref{[inr]invalues}.
        \begin{byCases}
          \savelocalsteps{2}
          \item[\text{(\ref{rule:IVVal})}]
          \begin{pfsteps*}
          \item $\isVal{\hinr{\tau_1}{e_2}}$ \BY{assumption}
          \end{pfsteps*} 
          Contradicts \pfref{eindet} by \autoref{lem:val-not-indet}.
          \restorelocalsteps{2} 
          \item[\text{(\ref{rule:IVIndet})}] 
          \begin{pfsteps*}
          \item $\notIntro{\hinr{\tau_1}{e_2}}$ \BY{assumption}
          \end{pfsteps*} 
          Contradicts \autoref{lem:no-inr-notintro}
          \restorelocalsteps{2}
          \item[\text{(\ref{rule:IVInr})}] 
          \begin{pfsteps*}
          \item $e'=\hinr{\tau_1}{e_2'}$ \BY{assumption}
          \item $\cnotsatisfyormay{\hinr{\tau_1}{e_2'}}{\cinl{\hxi_1}}$ \BY{\autoref{lem:inr-notsatormay-inl}}
          \end{pfsteps*} 
        \end{byCases}
      \end{byCases} 
    \restorelocalsteps{0}
    \item[\text{(\ref{rule:CTInr})}]
    \begin{pfsteps*}
      \item $\hxi = \cinr{\hxi_2}$ \BY{assumption}
      \item $\tau = \tsum{\tau_1}{\tau_2}$ \BY{assumption}
      \item $\ctyp{\hxi_2}{\tau_2}$ \BY{assumption} \pflabel{[inr]c2typ}
      \item $\refutable{\cinr{\hxi_2}}$ \BY{\ruleref{rule:RXInr}} \pflabel{[inr]refutable}
      \end{pfsteps*}
      By rule induction over \rulesref{rules:Indet} on \pfref{eindet}.
      \begin{byCases}
        \savelocalsteps{1}
        \item[\text{(\ref{rule:IEHole})}]
        \begin{pfsteps*}
        \item $e=\hehole{u}$ \BY{assumption}
        \item $\notIntro{\hehole{u}}$ \BY{\ruleref{rule:NVEHole}} \pflabel{[ehole][inr]notintro}
        \item $\cmaysatisfy{\hehole{u}}{\cinr{\hxi_2}}$ \BY{\ruleref{rule:CMSNotIntro} on \pfref{[ehole][inr]notintro} and \pfref{[inr]refutable}} \pflabel{[ehole][inr]maysat}
        \item $\csatisfyormay{\hehole{u}}{\cinr{\hxi_2}}$ \BY{\ruleref{rule:CSMSMay} on \pfref{[ehole][inr]maysat}}
        \end{pfsteps*}
        Contradicts \pfref{not-satormay}.
        \restorelocalsteps{1}
        \item[\text{(\ref{rule:IHole})}]
        \begin{pfsteps*}
        \item $e=\hhole{e_1}{u}$ \BY{assumption}
        \item $\notIntro{\hhole{e_1}{u}}$ \BY{\ruleref{rule:NVHole}} \pflabel{[hole][inr]notintro}
        \item $\cmaysatisfy{\hhole{e_1}{u}}{\cinr{\hxi_2}}$ \BY{\ruleref{rule:CMSNotIntro} on \pfref{[hole][inr]notintro} and \pfref{[inr]refutable}} \pflabel{[hole][inr]maysat}
        \item $\csatisfyormay{\hhole{e_1}{u}}{\cinr{\hxi_2}}$ \BY{\ruleref{rule:CSMSMay} on \pfref{[hole][inr]maysat}}
        \end{pfsteps*} 
        Contradicts \pfref{not-satormay}.
        \restorelocalsteps{1}
        \item[\text{(\ref{rule:IAp})}]
        \begin{pfsteps*}
        \item $e=\hap{e_1}{e_2}$ \BY{assumption}
        \item $\notIntro{\hap{e_1}{e_2}}$ \BY{\ruleref{rule:NVAp}} \pflabel{[ap][inr]notintro}
        \item $\cmaysatisfy{\hap{e_1}{e_2}}{\cinr{\hxi_2}}$ \BY{\ruleref{rule:CMSNotIntro} on \pfref{[ap][inr]notintro} and \pfref{[inr]refutable}} \pflabel{[ap][inr]maysat}
        \item $\csatisfyormay{\hap{e_1}{e_2}}{\cinr{\hxi_2}}$ \BY{\ruleref{rule:CSMSMay} on \pfref{[ap][inr]maysat}}
        \end{pfsteps*} 
        Contradicts \pfref{not-satormay}.
        \restorelocalsteps{1}
        \item[\text{(\ref{rule:IPrl})}]
        \begin{pfsteps*}
        \item $e=\hprl{e_1}$ \BY{assumption}
        \item $\notIntro{\hprl{e_1}}$ \BY{\ruleref{rule:NVPrl}} \pflabel{[prl][inr]notintro}
        \item $\cmaysatisfy{\hprl{e_1}}{\cinr{\hxi_2}}$ \BY{\ruleref{rule:CMSNotIntro} on \pfref{[prl][inr]notintro} and \pfref{[inr]refutable}} \pflabel{[prl][inr]maysat}
        \item $\csatisfyormay{\hprl{e_1}}{\cinr{\hxi_2}}$ \BY{\ruleref{rule:CSMSMay} on \pfref{[prl][inr]maysat}}
        \end{pfsteps*} 
        Contradicts \pfref{not-satormay}.
        \restorelocalsteps{1}
        \item[\text{(\ref{rule:IPrr})}]
        \begin{pfsteps*}
        \item $e=\hprr{e_1}$ \BY{assumption}
        \item $\notIntro{\hprr{e_1}}$ \BY{\ruleref{rule:NVPrr}} \pflabel{[prr][inr]notintro}
        \item $\cmaysatisfy{\hprr{e_1}}{\cinr{\hxi_2}}$ \BY{\ruleref{rule:CMSNotIntro} on \pfref{[prr][inr]notintro} and \pfref{[inr]refutable}} \pflabel{[prr][inr]maysat}
        \item $\csatisfyormay{\hprr{e_1}}{\cinr{\hxi_2}}$ \BY{\ruleref{rule:CSMSMay} on \pfref{[prr][inr]maysat}}
        \end{pfsteps*} 
        Contradicts \pfref{not-satormay}.
        \restorelocalsteps{1}
        \item[\text{(\ref{rule:IMatch})}]
        \begin{pfsteps*}
        \item $e=\hmatch{e_1}{\zrules}$ \BY{assumption}
        \item $\notIntro{\hmatch{e_1}{\zrules}}$ \BY{\ruleref{rule:NVMatch}} \pflabel{[match][inr]notintro}
        \item $\cmaysatisfy{\hmatch{e_1}{\zrules}}{\cinr{\hxi_2}}$ \BY{\ruleref{rule:CMSNotIntro} on \pfref{[match][inr]notintro} and \pfref{[inr]refutable}} \pflabel{[match][inr]maysat}
        \item $\csatisfyormay{\hmatch{e_1}{\zrules}}{\cinr{\hxi_2}}$ \BY{\ruleref{rule:CSMSMay} on \pfref{[match][inr]maysat}}
        \end{pfsteps*} 
        Contradicts \pfref{not-satormay}.
        \restorelocalsteps{1}
        \item[\text{(\ref{rule:IPairL}), (\ref{rule:IPairR}), (\ref{rule:IPair})}] 
        \begin{pfsteps*}
        \item $e=\hpair{e_1}{e_2}$ \BY{assumption}
        \end{pfsteps*} 
        By rule induction over \rulesref{rules:Indet} on \pfref{eindet}, no rule applies due to syntactic contradiction.
        \restorelocalsteps{1}
        \item[\text{(\ref{rule:IInl})}] 
        \begin{pfsteps*}
        \item $e=\hinl{\tau_2}{e_1}$ \BY{assumption}
        \end{pfsteps*}
        To show if $\inValues{e'}{\hinl{\tau_2}{e_1}}$ then $\cnotsatisfyormay{e'}{\cinr{\hxi_2}}$, we assume $\inValues{e'}{\hinl{\tau_2}{e_1}}$.
        \begin{pfsteps*}
        \item $\inValues{e'}{\hinl{\tau_2}{e_1}}$ \BY{assumption} \pflabel{[inl][inr]invalues}
        \end{pfsteps*}
        By rule induction over \rulesref{rules:inValues} on \pfref{[inl][inr]invalues}.
        \begin{byCases}
          \savelocalsteps{2}
          \item[\text{(\ref{rule:IVVal})}]
          \begin{pfsteps*}
          \item $\isVal{\hinl{\tau_2}{e_1}}$ \BY{assumption}
          \end{pfsteps*} 
          Contradicts \pfref{eindet} by \autoref{lem:val-not-indet}.
          \restorelocalsteps{2} 
          \item[\text{(\ref{rule:IVIndet})}] 
          \begin{pfsteps*}
          \item $\notIntro{\hinl{\tau_2}{e_1}}$ \BY{assumption}
          \end{pfsteps*} 
          Contradicts \autoref{lem:no-inl-notintro}
          \restorelocalsteps{2}
          \item[\text{(\ref{rule:IVInl})}] 
          \begin{pfsteps*}
          \item $e'=\hinl{\tau_2}{e_1'}$ \BY{assumption}
          \item $\cnotsatisfyormay{\hinl{\tau_2}{e_1'}}{\cinr{\hxi_2}}$ \BY{\autoref{lem:inl-notsatormay-inr}}
          \end{pfsteps*} 
        \end{byCases}
        \restorelocalsteps{1}
        \item[\text{(\ref{rule:IInr})}] 
        \begin{pfsteps*}
        \item $e=\hinr{\tau_1'}{e_2}$ \BY{assumption}
        \item $\isIndet{e_2}$ \BY{assumption} \pflabel{[inr]e2indet}
        \end{pfsteps*} 
        By rule induction over \rulesref{rules:TExp} on \pfref{etyp}, only one rule applies.
        \begin{byCases}
          \item[\text{(\ref{rule:TInr})}]
          \begin{pfsteps*}
          \item $\tau_1'=\tau_1$ \BY{assumption}
          \item $\hexptyp{\cdot}{\Delta}{e_2}{\tau_2}$ \BY{assumption} \pflabel{[inr]e2typ}
          \item $\cnotsatisfyormay{e_2}{\hxi_2}$ \BY{\autoref{lem:satormay-inl} on \pfref{not-satormay}} \pflabel{[inr]not-satormay2}
          \item if $\inValues{e_2'}{e_2}$ then $\cnotsatisfyormay{e_2'}{\hxi_2}$ \BY{IH on \pfref{[inr]e2indet} and \pfref{[inr]e2typ} and \pfref{[inr]c2typ} and \pfref{[inr]not-satormay2}} \pflabel{[inr]forall2}
          \end{pfsteps*} 
          To show if $\inValues{e'}{\hinr{\tau_1}{e_2}}$ then $\cnotsatisfyormay{e'}{\cinr{\hxi_2}}$, we assume $\inValues{e'}{\hinr{\tau_1}{e_2}}$.
          \begin{pfsteps*}
          \item $\inValues{e'}{\hinr{\tau_1}{e_2}}$ \BY{assumption} \pflabel{[inr][inr]invalues}
          \end{pfsteps*}
          By rule induction over \rulesref{rules:inValues} on \pfref{[inr][inr]invalues}.
          \begin{byCases}
            \savelocalsteps{2}
            \item[\text{(\ref{rule:IVVal})}]
            \begin{pfsteps*}
            \item $\isVal{\hinr{\tau_1}{e_2}}$ \BY{assumption}
            \end{pfsteps*} 
            Contradicts \pfref{eindet} by \autoref{lem:val-not-indet}.
            \restorelocalsteps{2} 
            \item[\text{(\ref{rule:IVIndet})}] 
            \begin{pfsteps*}
            \item $\notIntro{\hinr{\tau_1}{e_2}}$ \BY{assumption}
            \end{pfsteps*} 
            Contradicts \autoref{lem:no-inr-notintro}
            \restorelocalsteps{2}
            \item[\text{(\ref{rule:IVInr})}] 
            \begin{pfsteps*}
            \item $e'=\hinr{\tau_1}{e_2'}$ \BY{assumption}
            \item $\inValues{e_2'}{e_2}$ \BY{assumption} \pflabel{[inr]invalues2}
            \item $\cnotsatisfyormay{e_2'}{\hxi_2}$ \BY{\pfref{[inr]forall2} on \pfref{[inr]invalues2}} \pflabel{[inr]not-satormay2'}
            \item $\cnotsatisfyormay{\hinr{\tau_1}{e_2'}}{\cinr{\hxi_2}}$ \BY{\autoref{lem:satormay-inr} on \pfref{[inr]not-satormay2'}}
            \end{pfsteps*} 
          \end{byCases}
        \end{byCases}
      \end{byCases} 
    \restorelocalsteps{0}
    \item[\text{(\ref{rule:CTPair})}]
    \begin{pfsteps*}
    \item $\hxi=\cpair{\hxi_1}{\hxi_2}$ \BY{assumption}
    \item $\tau=\tprod{\tau_1}{\tau_2}$ \BY{assumption}
    \item $\ctyp{\hxi_1}{\tau_1}$ \BY{assumption} \pflabel{[pair]c1typ}
    \item $\ctyp{\hxi_2}{\tau_2}$ \BY{assumption} \pflabel{[pair]c2typ}
    \end{pfsteps*} 
    By rule induction over \rulesref{rules:Indet} on \pfref{eindet}.
    \begin{byCases}
      \savelocalsteps{1}
      \item[\text{(\ref{rule:IEHole}), (\ref{rule:IHole}), (\ref{rule:IAp}), (\ref{rule:IPrl}), (\ref{rule:IPrr}), (\ref{rule:IMatch})}]
      \begin{pfsteps*}
      \item $e=\hehole{u}, \hhole{e_1}{u}, \hap{e_1}{e_2}, \hprl{e_1}, \hprr{e_1}, \hmatch{e_1}{\zrules}$ \BY{assumption}
      \item $\notIntro{e}$ \BY{\rulesref{rules:notintro}} \pflabel{[pair]notintro}
      \item $\notIntro{\hprl{e}}$ \BY{\ruleref{rule:NVPrl}} \pflabel{[pair]prlnotintro}
      \item $\notIntro{\hprr{e}}$ \BY{\ruleref{rule:NVPrr}} \pflabel{[pair]prrnotintro}
      \item $\isIndet{\hprl{e}}$ \BY{\ruleref{rule:IPrl} on \pfref{eindet}} \pflabel{[pair]prlindet}
      \item $\isIndet{\hprr{e}}$ \BY{\ruleref{rule:IPrr} on \pfref{eindet}} \pflabel{[pair]prrindet}
      \item $\hexptyp{\cdot}{\Delta}{\hprl{e}}{\tau_1}$ \BY{\ruleref{rule:TPrl} on \pfref{etyp}} \pflabel{[pair]prltyp}
      \item $\hexptyp{\cdot}{\Delta}{\hprr{e}}{\tau_2}$ \BY{\ruleref{rule:TPrr} on \pfref{etyp}} \pflabel{[pair]prrtyp}
      \end{pfsteps*}
      By case analysis on the result of $\fsatisfyormay{\hprl{e}}{\hxi_1}$.
      \begin{byCases}
        \savelocalsteps{2}
        \item[\true]
        \begin{pfsteps*}
        \item $\fsatisfyormay{\hprl{e}}{\hxi_1}=\true$ \BY{assumption} \pflabel{[pair]fsatormay1}
        \item $\csatisfyormay{\hprl{e}}{\hxi_1}$ \BY{\autoref{lem:sound-complete-satormay} on \pfref{[pair]fsatormay1}} \pflabel{[pair]satormay1}
        \end{pfsteps*} 
        By case analysis on the result of $\fsatisfyormay{\hprr{e}}{\hxi_2}$.
        \begin{byCases}
          \savelocalsteps{3}
          \item[\true]
          \begin{pfsteps*}
          \item $\fsatisfyormay{\hprr{e}}{\hxi_2}=\true$ \BY{assumption} \pflabel{[pair]fsatormay2}
          \item $\csatisfyormay{\hprr{e}}{\hxi_2}$ \BY{\autoref{lem:sound-complete-satormay} on \pfref{[pair]fsatormay2}} \pflabel{[pair]satormay2}
          \end{pfsteps*}  
          By rule induction over \rulesref{rules:satormay} on \pfref{[pair]satormay1}.
          \begin{byCases}
            \savelocalsteps{4}
            \item[\text{(\ref{rule:CSMSSat})}]
            \begin{pfsteps*}
            \item $\csatisfy{\hprl{e}}{\hxi_1}$ \BY{asssumption} \pflabel{[pair]satisfy1}
            \end{pfsteps*} 
            By rule induction over \rulesref{rules:satormay} on \pfref{[pair]satormay2}.
            \begin{byCases}
              \savelocalsteps{5}
              \item[\text{(\ref{rule:CSMSSat})}]
              \begin{pfsteps*}
              \item $\csatisfy{\hprr{e}}{\hxi_2}$ \BY{asssumption} \pflabel{[pair]satisfy2}
              \item $\csatisfy{e}{\cpair{\hxi_1}{\hxi_2}}$ \BY{\ruleref{rule:CSNotIntroPair} on \pfref{[pair]notintro} and \pfref{[pair]satisfy1} and \pfref{[pair]satisfy2}} \pflabel{[pair]satisfy}
              \item $\csatisfyormay{e}{\cpair{\hxi_1}{\hxi_2}}$ \BY{\ruleref{rule:CSMSSat} on \pfref{[pair]satisfy}}
              \end{pfsteps*} 
              Contradicts \pfref{not-satormay}.
              \restorelocalsteps{5}
              \item[\text{(\ref{rule:CSMSMay})}]
              \begin{pfsteps*}
              \item $\cmaysatisfy{\hprr{e}}{\hxi_2}$ \BY{assumption} \pflabel{[pair]maysat2}
              \item $\refutable{\hxi_2}$ \BY{\autoref{lem:notintro-maysat-refutable} on \pfref{[pair]prrnotintro} and \pfref{[pair]maysat2}} \pflabel{[pair]refutable2}
              \item $\refutable{\cpair{\hxi_1}{\hxi_2}}$ \BY{\ruleref{rule:RXPairR} on \pfref{[pair]refutable2}} \pflabel{[pair]refutable}
              \item $\cmaysatisfy{e}{\cpair{\hxi_1}{\hxi_2}}$ \BY{\ruleref{rule:CMSNotIntro} on \pfref{[pair]notintro} and \pfref{[pair]refutable}} \pflabel{[pair]maysat}
              \item $\csatisfyormay{e}{\cpair{\hxi_1}{\hxi_2}}$ \BY{\ruleref{rule:CSMSMay} on \pfref{[pair]maysat}}
              \end{pfsteps*} 
            \end{byCases}
            \restorelocalsteps{4} 
            \item[\text{(\ref{rule:CSMSMay})}]
            \begin{pfsteps*}
            \item $\cmaysatisfy{\hprl{e}}{\hxi_1}$ \BY{assumption} \pflabel{[pair]maysat1}
            \item $\refutable{\hxi_1}$ \BY{\autoref{lem:notintro-maysat-refutable} on \pfref{[pair]prlnotintro} and \pfref{[pair]maysat1}} \pflabel{[pair]refutable1}
            \item $\refutable{\cpair{\hxi_1}{\hxi_2}}$ \BY{\ruleref{rule:RXPairL} on \pfref{[pair]refutable1}} \pflabel{[pair]refutable'}
            \item $\cmaysatisfy{e}{\cpair{\hxi_1}{\hxi_2}}$ \BY{\ruleref{rule:CMSNotIntro} on \pfref{[pair]notintro} and \pfref{[pair]refutable'}} \pflabel{[pair]maysat'}
            \item $\csatisfyormay{e}{\cpair{\hxi_1}{\hxi_2}}$ \BY{\ruleref{rule:CSMSMay} on \pfref{[pair]maysat'}}
            \end{pfsteps*} 
          \end{byCases} 
          \restorelocalsteps{3} 
          \item[\false]
          \begin{pfsteps*}
          \item $\fsatisfyormay{\hprr{e}}{\hxi_2} = \false$ \BY{assumption} \pflabel{[pair]not-fsatormay2}
          \item $\csatisfyormay{\hprr{e}}{\hxi_2}$ \BY{\autoref{lem:sound-complete-satormay} on \pfref{[pair]not-fsatormay2}} \pflabel{[pair]not-satormay2}
          \item if $\inValues{e_2'}{\hprr{e}}$ then $\cnotsatisfyormay{e_2'}{\hxi_2}$ \BY{IH on \pfref{[pair]prrindet} and \pfref{[pair]prrtyp} and \pfref{[pair]c2typ} and \pfref{[pair]not-satormay2}} \pflabel{[pair]complete2}
          \end{pfsteps*} 
          To show if $\inValues{e'}{e}$ then $\cnotsatisfyormay{e'}{\cpair{\hxi_1}{\hxi_2}}$, we assume $\inValues{e'}{e}$.
          \begin{pfsteps*}
          \item $\inValues{e'}{e}$ \BY{assumption} \pflabel{e'invalues'}
          \end{pfsteps*}
          By rule induction over \rulesref{rules:inValues} on \pfref{e'invalues'}, only two rules apply.
          \begin{byCases}
            \savelocalsteps{4}
            \item[\text{(\ref{rule:IVVal})}]
            \begin{pfsteps*}
            \item $\isVal{e}$ \BY{assumption}
            \end{pfsteps*} 
            Contradicts \pfref{eindet} by \autoref{lem:val-not-indet}.
            \restorelocalsteps{4} 
            \item[\text{(\ref{rule:IVIndet})}] 
            \begin{pfsteps*}
            \item $\isVal{e'}$ \BY{assumption} \pflabel{e'val'}
            \item $\hexptyp{\cdot}{\Delta}{e'}{\tprod{\tau_1}{\tau_2}}$ \BY{assumption} \pflabel{e'typ'}
            \end{pfsteps*}
            By rule induction over \rulesref{rules:Value} on \pfref{e'val'}.
            \begin{byCases}
              \savelocalsteps{5}
              \item[\text{(\ref{rule:VNum})}]
              \begin{pfsteps*}
              \item $e'=\hnum{n}$ \BY{assumption}
              \end{pfsteps*} 
              By rule induction over \rulesref{rules:TExp} on \pfref{e'typ'}, no rule applies due to syntactic contradiction.
              \restorelocalsteps{5}
              \item[\text{(\ref{rule:VLam})}]
              \begin{pfsteps*} 
              \item $e'=\hlam{x}{\tau'}{e_1'}$ \BY{assumption}
              \end{pfsteps*}
              By rule induction over \rulesref{rules:TExp} on \pfref{e'typ'}, no rule applies due to syntactic contradiction.
              \restorelocalsteps{5}
              \item[\text{(\ref{rule:VPair})}] 
              \begin{pfsteps*} 
              \item $e'=\hpair{e_1'}{e_2'}$ \BY{assumption}
              \item $\isVal{e_2'}$ \BY{assumption} \pflabel{e2'val}
              \end{pfsteps*}
              By rule induction over \rulesref{rules:TExp} on \pfref{e'typ'}, only one rule applies. \\
                \textbf{Case} \text{(\ref{rule:TPair})}\textbf{.}
                \begin{pfsteps*}
                \item $\hexptyp{\cdot}{\Delta}{e_2'}{\tau_2}$ \BY{assumption} \pflabel{e2'typ}
                \item $\inValues{e_2'}{\hprr{e}}$ \BY{\ruleref{rule:IVIndet} on \pfref{[pair]prrnotintro} and \pfref{[pair]prrtyp} and \pfref{e2'val} and \pfref{e2'typ}} \pflabel{e2'invalues}
                \item $\cnotsatisfyormay{e_2'}{\hxi_2}$ \BY{\pfref{[pair]complete2} on \pfref{e2'invalues}} \pflabel{[pair]not-satormay2'}
                \item $\cnotsatisfyormay{\hpair{e_1'}{e_2'}}{\cpair{\hxi_1}{\hxi_2}}$ \BY{\autoref{lem:satormay-pair} on \pfref{[pair]not-satormay1'}}
                \end{pfsteps*}
              \restorelocalsteps{5}
              \item[\text{(\ref{rule:VInl})}] 
              \begin{pfsteps*} 
              \item $e'=\hinl{\tau_2}{e_1'}$ \BY{assumption}
              \end{pfsteps*}
              By rule induction over \rulesref{rules:TExp} on \pfref{e'typ'}, no rule applies due to syntactic contradiction.
              \restorelocalsteps{5}
              \item[\text{(\ref{rule:VInr})}] 
              \begin{pfsteps*} 
              \item $e'=\hinr{\tau_1}{e_2'}$ \BY{assumption}
              \end{pfsteps*}
              By rule induction over \rulesref{rules:TExp} on \pfref{e'typ'}, no rule applies due to syntactic contradiction.
            \end{byCases}
          \end{byCases}
        \end{byCases}
        
        \restorelocalsteps{2}
        \item[\false] 
        \begin{pfsteps*}
        \item $\fsatisfyormay{\hprl{e}}{\hxi_1}=\false$ \BY{assumption} \pflabel{[pair]not-fsatormay1}
        \item $\cnotsatisfyormay{\hprl{e}}{\hxi_1}$ \BY{\autoref{lem:sound-complete-satormay} on \pfref{[pair]not-fsatormay1}} \pflabel{[pair]not-satormay1}
        \item if $\inValues{e_1'}{\hprl{e}}$ then $\cnotsatisfyormay{e_1'}{\hxi_1}$ \BY{IH on \pfref{[pair]prlindet} and \pfref{[pair]prltyp} and \pfref{[pair]c1typ} and \pfref{[pair]not-satormay1}} \pflabel{[pair]complete1}
        \end{pfsteps*} 
        To show if $\inValues{e'}{e}$ then $\cnotsatisfyormay{e'}{\cpair{\hxi_1}{\hxi_2}}$, we assume $\inValues{e'}{e}$.
        \begin{pfsteps*}
        \item $\inValues{e'}{e}$ \BY{assumption} \pflabel{e'invalues}
        \end{pfsteps*}
        By rule induction over \rulesref{rules:inValues} on \pfref{e'invalues}, only two rules apply.
        \begin{byCases}
          \savelocalsteps{3}
          \item[\text{(\ref{rule:IVVal})}]
          \begin{pfsteps*}
          \item $\isVal{e}$ \BY{assumption}
          \end{pfsteps*} 
          Contradicts \pfref{eindet} by \autoref{lem:val-not-indet}.
          \restorelocalsteps{3} 
          \item[\text{(\ref{rule:IVIndet})}] 
          \begin{pfsteps*}
          \item $\isVal{e'}$ \BY{assumption} \pflabel{e'val}
          \item $\hexptyp{\cdot}{\Delta}{e'}{\tprod{\tau_1}{\tau_2}}$ \BY{assumption} \pflabel{e'typ}
          \end{pfsteps*}
          By rule induction over \rulesref{rules:Value} on \pfref{e'val}.
          \begin{byCases}
            \savelocalsteps{3}
            \item[\text{(\ref{rule:VNum})}]
            \begin{pfsteps*}
            \item $e'=\hnum{n}$ \BY{assumption}
            \end{pfsteps*} 
            By rule induction over \rulesref{rules:TExp} on \pfref{e'typ}, no rule applies due to syntactic contradiction.
            \restorelocalsteps{3}
            \item[\text{(\ref{rule:VLam})}]
            \begin{pfsteps*} 
            \item $e'=\hlam{x}{\tau'}{e_1'}$ \BY{assumption}
            \end{pfsteps*}
            By rule induction over \rulesref{rules:TExp} on \pfref{e'typ}, no rule applies due to syntactic contradiction.
            \restorelocalsteps{3}
            \item[\text{(\ref{rule:VPair})}] 
            \begin{pfsteps*} 
            \item $e'=\hpair{e_1'}{e_2'}$ \BY{assumption}
            \item $\isVal{e_1'}$ \BY{assumption} \pflabel{e1'val}
            \end{pfsteps*}
            By rule induction over \rulesref{rules:TExp} on \pfref{e'typ}, only one rule applies.
            \begin{byCases}
              \item[\text{(\ref{rule:TPair})}]
              \begin{pfsteps*}
              \item $\hexptyp{\cdot}{\Delta}{e_1'}{\tau_1}$ \BY{assumption} \pflabel{e1'typ}
              \item $\inValues{e_1'}{\hprl{e}}$ \BY{\ruleref{rule:IVIndet} on \pfref{[pair]prlnotintro} and \pfref{[pair]prltyp} and \pfref{e1'val} and \pfref{e1'typ}} \pflabel{e1'invalues}
              \item $\cnotsatisfyormay{e_1'}{\hxi_1}$ \BY{\pfref{[pair]complete1} on \pfref{e1'invalues}} \pflabel{[pair]not-satormay1'}
              \item $\cnotsatisfyormay{\hpair{e_1'}{e_2'}}{\cpair{\hxi_1}{\hxi_2}}$ \BY{\autoref{lem:satormay-pair} on \pfref{[pair]not-satormay1'}}
              \end{pfsteps*} 
            \end{byCases}
            \restorelocalsteps{3}
            \item[\text{(\ref{rule:VInl})}] 
            \begin{pfsteps*} 
            \item $e'=\hinl{\tau_2}{e_1'}$ \BY{assumption}
            \end{pfsteps*}
            By rule induction over \rulesref{rules:TExp} on \pfref{e'typ}, no rule applies due to syntactic contradiction.
            \restorelocalsteps{3}
            \item[\text{(\ref{rule:VInr})}] 
            \begin{pfsteps*} 
            \item $e'=\hinr{\tau_1}{e_2'}$ \BY{assumption}
            \end{pfsteps*}
            By rule induction over \rulesref{rules:TExp} on \pfref{e'typ}, no rule applies due to syntactic contradiction.
          \end{byCases}
        \end{byCases}
      \end{byCases}
      \restorelocalsteps{1}
      \item[\text{(\ref{rule:IPairL})}] 
      \begin{pfsteps*}
      \item $e=\hpair{e_1}{e_2}$ \BY{assumption}
      \item $\isIndet{e_1}$ \BY{assumption} \pflabel{[pair1]e1indet}
      \item $\isVal{e_2}$ \BY{assumption} \pflabel{[pair1]e2val}
      \item $\cnotsatisfyormay{e_1}{\hxi_1}$ or $\cnotsatisfyormay{e_2}{\hxi_2}$ \BY{\autoref{lem:satormay-pair} on \pfref{not-satormay}} \pflabel{[pair1]not-satormay-1or2}
      \end{pfsteps*}
      By case analysis on the disjunction in \pfref{[pair1]not-satormay-1or2}.
      \begin{byCases}
        \savelocalsteps{2}
        \item[\cnotsatisfyormay{e_1}{\hxi_1}]
        \begin{pfsteps*}
        \item $\cnotsatisfyormay{e_1}{\hxi_1}$ \BY{assumption} \pflabel{[pair1]not-satormay-1}
        \end{pfsteps*}
        By rule induction over \rulesref{rules:TExp} on \pfref{etyp}, only one rule applies.
        \begin{byCases}
          \item[\text{(\ref{rule:TPair})}]
          \begin{pfsteps*}
          \item $\hexptyp{\cdot}{\Delta}{e_1}{\tau_1}$ \BY{assumption} \pflabel{[pair1]e1typ}
          \item if $\inValues{e_1'}{e_1}$ then $\cnotsatisfyormay{e_1'}{\hxi_1}$ \BY{IH on \pfref{[pair1]e1indet} and \pfref{[pair1]e1typ} and \pfref{[pair]c1typ} and \pfref{[pair1]not-satormay-1}} \pflabel{[pair1]complete1}
          \end{pfsteps*} 
          To show that if $\inValues{e'}{\hpair{e_1}{e_2}}$ then $\cnotsatisfyormay{\hpair{e_1}{e_2}}{\cpair{\hxi_1}{\hxi_2}}$, we assume $\inValues{e'}{\hpair{e_1}{e_2}}$.
          \begin{pfsteps*}
          \item $\inValues{e'}{\hpair{e_1}{e_2}}$ \BY{assumption} \pflabel{[pair1]invalues}
          \end{pfsteps*}
          By rule induction over \rulesref{rules:inValues} on \pfref{[pair1]invalues}.
          \begin{byCases}
            \savelocalsteps{3}
            \item[\text{(\ref{rule:IVVal})}]
            \begin{pfsteps*}
            \item $\isVal{\hpair{e_1}{e_2}}$ \BY{assumption}
            \end{pfsteps*} 
            Contradicts \pfref{eindet} by \autoref{lem:val-not-indet}.
            \restorelocalsteps{3} 
            \item[\text{(\ref{rule:IVIndet})}] 
            \begin{pfsteps*}
            \item $\notIntro{\hpair{e_1}{e_2}}$ \BY{assumption}
            \end{pfsteps*}
            Contradicts \autoref{lem:no-pair-notintro}.
            \restorelocalsteps{3}
            \item[\text{(\ref{rule:IVPair})}]
            \begin{pfsteps*}
            \item $e'=\hpair{e_1'}{e_2'}$ \BY{assumption}
            \item $\inValues{e_1'}{e_1}$ \BY{assumption} \pflabel{[pair1]invalues1}
            \item $\cnotsatisfyormay{e_1'}{\hxi_1}$ \BY{\pfref{[pair1]complete1} on \pfref{[pair1]invalues1}} \pflabel{[pair1]not-satormay1}
            \item $\cnotsatisfyormay{\hpair{e_1'}{e_2'}}{\cpair{\hxi_1}{\hxi_2}}$ \BY{\autoref{lem:satormay-pair} on \pfref{[pair1]not-satormay1}}
            \end{pfsteps*} 
          \end{byCases}
        \end{byCases}
        \restorelocalsteps{2}
        \item[\cnotsatisfyormay{e_2}{\hxi_2}]
        \begin{pfsteps*}
        \item $\cnotsatisfyormay{e_2}{\hxi_2}$ \BY{assumption} \pflabel{[pair1]not-satormay-2}
        \end{pfsteps*}
        To show that if $\inValues{e'}{\hpair{e_1}{e_2}}$ then $\cnotsatisfyormay{\hpair{e_1}{e_2}}{\cpair{\hxi_1}{\hxi_2}}$, we assume $\inValues{e'}{\hpair{e_1}{e_2}}$.
        \begin{pfsteps*}
        \item $\inValues{e'}{\hpair{e_1}{e_2}}$ \BY{assumption} \pflabel{[pair1]invalues'}
        \end{pfsteps*}
        By rule induction over \rulesref{rules:inValues} on \pfref{[pair1]invalues'}.
        \begin{byCases}
          \savelocalsteps{3}
          \item[\text{(\ref{rule:IVVal})}]
          \begin{pfsteps*}
          \item $\isVal{\hpair{e_1}{e_2}}$ \BY{assumption}
          \end{pfsteps*} 
          Contradicts \pfref{eindet} by \autoref{lem:val-not-indet}.
          \restorelocalsteps{3} 
          \item[\text{(\ref{rule:IVIndet})}] 
          \begin{pfsteps*}
          \item $\notIntro{\hpair{e_1}{e_2}}$ \BY{assumption}
          \end{pfsteps*}
          Contradicts \autoref{lem:no-pair-notintro}.
          \restorelocalsteps{3}
          \item[\text{(\ref{rule:IVPair})}]
          \begin{pfsteps*}
          \item $e'=\hpair{e_1'}{e_2'}$ \BY{assumption}
          \item $\inValues{e_2'}{e_2}$ \BY{assumption} \pflabel{[pair1]invalues1'}
          \end{pfsteps*}
          By rule induction over \rulesref{rules:inValues} on \pfref{[pair1]invalues1'}.
          \begin{byCases}
            \savelocalsteps{4}
            \item[\text{(\ref{rule:IVVal})}]
            \begin{pfsteps*}
            \item $e_2'=e_2$ \BY{assumption} \pflabel{subst}
            \item $\cnotsatisfyormay{e_2'}{\hxi_2}$ \BY{\pfref{subst} and \pfref{[pair1]not-satormay-2}} \pflabel{[pair1]not-satormay-2'}
            \item $\cnotsatisfyormay{\hpair{e_1'}{e_2'}}{\cpair{\hxi_1}{\hxi_2}}$ \BY{\autoref{lem:satormay-pair} on \pfref{[pair1]not-satormay-2'}}
            \end{pfsteps*} 
            \restorelocalsteps{4} 
            \item[\text{(\ref{rule:IVIndet})}]
            \begin{pfsteps*}
            \item $\notIntro{e_2}$ \BY{assumption}
            \end{pfsteps*} 
            Contradicts \pfref{[pair1]e2val} by \autoref{lem:val-not-notintro}.
            \restorelocalsteps{4} 
            \item[\text{(\ref{rule:IVInl}), (\ref{rule:IVInr}), (\ref{rule:IVPair})}]
            \begin{pfsteps*}
            \item $\isIndet{e_2}$ \BY{assumption}
            \end{pfsteps*} 
            Contradicts \pfref{[pair1]e2val} by \autoref{lem:val-not-indet}.
          \end{byCases}
        \end{byCases}
      \end{byCases}
      \restorelocalsteps{1}
      \item[\text{(\ref{rule:IPairR})}] 
      \begin{pfsteps*}
      \item $e=\hpair{e_1}{e_2}$ \BY{assumption}
      \item $\isVal{e_1}$ \BY{assumption} \pflabel{[pair2]e1val}
      \item $\isIndet{e_2}$ \BY{assumption} \pflabel{[pair2]e2indet}
      \item $\cnotsatisfyormay{e_1}{\hxi_1}$ or $\cnotsatisfyormay{e_2}{\hxi_2}$ \BY{\autoref{lem:satormay-pair} on \pfref{not-satormay}} \pflabel{[pair2]not-satormay-1or2}
      \end{pfsteps*}
      By case analysis on the disjunction in \pfref{[pair2]not-satormay-1or2}.
      \begin{byCases}
        \savelocalsteps{2}
        \item[\cnotsatisfyormay{e_1}{\hxi_1}]
        \begin{pfsteps*}
        \item $\cnotsatisfyormay{e_1}{\hxi_1}$ \BY{assumption} \pflabel{[pair2]not-satormay-1}
        \end{pfsteps*}
        To show that if $\inValues{e'}{\hpair{e_1}{e_2}}$ then $\cnotsatisfyormay{\hpair{e_1}{e_2}}{\cpair{\hxi_1}{\hxi_2}}$, we assume $\inValues{e'}{\hpair{e_1}{e_2}}$.
        \begin{pfsteps*}
        \item $\inValues{e'}{\hpair{e_1}{e_2}}$ \BY{assumption} \pflabel{[pair2]invalues'}
        \end{pfsteps*}
        By rule induction over \rulesref{rules:inValues} on \pfref{[pair2]invalues'}.
        \begin{byCases}
          \savelocalsteps{3}
          \item[\text{(\ref{rule:IVVal})}]
          \begin{pfsteps*}
          \item $\isVal{\hpair{e_1}{e_2}}$ \BY{assumption}
          \end{pfsteps*} 
          Contradicts \pfref{eindet} by \autoref{lem:val-not-indet}.
          \restorelocalsteps{3} 
          \item[\text{(\ref{rule:IVIndet})}] 
          \begin{pfsteps*}
          \item $\notIntro{\hpair{e_1}{e_2}}$ \BY{assumption}
          \end{pfsteps*}
          Contradicts \autoref{lem:no-pair-notintro}.
          \restorelocalsteps{3}
          \item[\text{(\ref{rule:IVPair})}]
          \begin{pfsteps*}
          \item $e'=\hpair{e_1'}{e_2'}$ \BY{assumption}
          \item $\inValues{e_1'}{e_1}$ \BY{assumption} \pflabel{[pair2]invalues1'}
          \end{pfsteps*}
          By rule induction over \rulesref{rules:inValues} on \pfref{[pair2]invalues1'}.
          \begin{byCases}
            \savelocalsteps{4}
            \item[\text{(\ref{rule:IVVal})}]
            \begin{pfsteps*}
            \item $e_1'=e_1$ \BY{assumption} \pflabel{[pair2]subst}
            \item $\cnotsatisfyormay{e_1'}{\hxi_1}$ \BY{\pfref{[pair2]subst} and \pfref{[pair2]not-satormay-1}} \pflabel{[pair2]not-satormay-1'}
            \item $\cnotsatisfyormay{\hpair{e_1'}{e_2'}}{\cpair{\hxi_1}{\hxi_2}}$ \BY{\autoref{lem:satormay-pair} on \pfref{[pair2]not-satormay-1'}}
            \end{pfsteps*} 
            \restorelocalsteps{4} 
            \item[\text{(\ref{rule:IVIndet})}]
            \begin{pfsteps*}
            \item $\notIntro{e_1}$ \BY{assumption}
            \end{pfsteps*} 
            Contradicts \pfref{[pair2]e1val} by \autoref{lem:val-not-notintro}.
            \restorelocalsteps{4} 
            \item[\text{(\ref{rule:IVInl}), (\ref{rule:IVInr}), (\ref{rule:IVPair})}]
            \begin{pfsteps*}
            \item $\isIndet{e_1}$ \BY{assumption}
            \end{pfsteps*} 
            Contradicts \pfref{[pair2]e1val} by \autoref{lem:val-not-indet}.
          \end{byCases}
        \end{byCases}
        \restorelocalsteps{2}
        \item[\cnotsatisfyormay{e_2}{\hxi_2}]
        \begin{pfsteps*}
        \item $\cnotsatisfyormay{e_2}{\hxi_2}$ \BY{assumption} \pflabel{[pair2]not-satormay-2}
        \end{pfsteps*}
        By rule induction over \rulesref{rules:TExp} on \pfref{etyp}, only one rule applies.
        \begin{byCases}
          \item[\text{(\ref{rule:TPair})}]
          \begin{pfsteps*}
          \item $\hexptyp{\cdot}{\Delta}{e_2}{\tau_2}$ \BY{assumption} \pflabel{[pair2]e2typ}
          \item if $\inValues{e_2'}{e_2}$ then $\cnotsatisfyormay{e_2'}{\hxi_2}$ \BY{IH on \pfref{[pair2]e2indet} and \pfref{[pair2]e2typ} and \pfref{[pair]c2typ} and \pfref{[pair2]not-satormay-2}} \pflabel{[pair2]complete2}
          \end{pfsteps*} 
          To show that if $\inValues{e'}{\hpair{e_1}{e_2}}$ then $\cnotsatisfyormay{\hpair{e_1}{e_2}}{\cpair{\hxi_1}{\hxi_2}}$, we assume $\inValues{e'}{\hpair{e_1}{e_2}}$.
          \begin{pfsteps*}
          \item $\inValues{e'}{\hpair{e_1}{e_2}}$ \BY{assumption} \pflabel{[pair2]invalues}
          \end{pfsteps*}
          By rule induction over \rulesref{rules:inValues} on \pfref{[pair2]invalues}.
          \begin{byCases}
            \savelocalsteps{3}
            \item[\text{(\ref{rule:IVVal})}]
            \begin{pfsteps*}
            \item $\isVal{\hpair{e_1}{e_2}}$ \BY{assumption}
            \end{pfsteps*} 
            Contradicts \pfref{eindet} by \autoref{lem:val-not-indet}.
            \restorelocalsteps{3} 
            \item[\text{(\ref{rule:IVIndet})}] 
            \begin{pfsteps*}
            \item $\notIntro{\hpair{e_1}{e_2}}$ \BY{assumption}
            \end{pfsteps*}
            Contradicts \autoref{lem:no-pair-notintro}.
            \restorelocalsteps{3}
            \item[\text{(\ref{rule:IVPair})}]
            \begin{pfsteps*}
            \item $e'=\hpair{e_1'}{e_2'}$ \BY{assumption}
            \item $\inValues{e_2'}{e_2}$ \BY{assumption} \pflabel{[pair2]invalues2}
            \item $\cnotsatisfyormay{e_2'}{\hxi_2}$ \BY{\pfref{[pair2]complete2} on \pfref{[pair2]invalues2}} \pflabel{[pair2]not-satormay2}
            \item $\cnotsatisfyormay{\hpair{e_1'}{e_2'}}{\cpair{\hxi_1}{\hxi_2}}$ \BY{\autoref{lem:satormay-pair} on \pfref{[pair2]not-satormay2}}
            \end{pfsteps*} 
          \end{byCases}
        \end{byCases}
      \end{byCases}
      \restorelocalsteps{1}
      \item[\text{(\ref{rule:IPair})}] 
      \begin{pfsteps*}
      \item $e=\hpair{e_1}{e_2}$ \BY{assumption}
      \item $\isIndet{e_1}$ \BY{assumption} \pflabel{[pair3]e1indet}
      \item $\isIndet{e_2}$ \BY{assumption} \pflabel{[pair3]e2indet}
      \item $\cnotsatisfyormay{e_1}{\hxi_1}$ or $\cnotsatisfyormay{e_2}{\hxi_2}$ \BY{\autoref{lem:satormay-pair} on \pfref{not-satormay}} \pflabel{[pair3]not-satormay-1or2}
      \end{pfsteps*}
      By rule induction over \rulesref{rules:TExp} on \pfref{etyp}, only one rule applies.
      \begin{byCases}
        \item[\text{(\ref{rule:TPair})}]
        \begin{pfsteps*}
        \item $\hexptyp{\cdot}{\Delta}{e_1}{\tau_1}$ \BY{assumption} \pflabel{[pair3]e1typ}
        \item $\hexptyp{\cdot}{\Delta}{e_2}{\tau_2}$ \BY{assumption} \pflabel{[pair3]e2typ}
        \end{pfsteps*} 
        By case analysis on the disjunction in \pfref{[pair3]not-satormay-1or2}.
        \begin{byCases}
          \savelocalsteps{2}
          \item[\cnotsatisfyormay{e_1}{\hxi_1}]
          \begin{pfsteps*}
          \item $\cnotsatisfyormay{e_1}{\hxi_1}$ \BY{assumption} \pflabel{[pair3]not-satormay-1}
          \item if $\inValues{e_1'}{e_1}$ then $\cnotsatisfyormay{e_1'}{\hxi_1}$ \BY{IH on \pfref{[pair3]e1indet} and \pfref{[pair3]e1typ} and \pfref{[pair]c1typ} and \pfref{[pair3]not-satormay-1}} \pflabel{[pair3]complete1}
          \end{pfsteps*}
          To show that if $\inValues{e'}{\hpair{e_1}{e_2}}$ then $\cnotsatisfyormay{\hpair{e_1}{e_2}}{\cpair{\hxi_1}{\hxi_2}}$, we assume $\inValues{e'}{\hpair{e_1}{e_2}}$.
          \begin{pfsteps*}
          \item $\inValues{e'}{\hpair{e_1}{e_2}}$ \BY{assumption} \pflabel{[pair3]invalues}
          \end{pfsteps*}
          By rule induction over \rulesref{rules:inValues} on \pfref{[pair3]invalues}.
          \begin{byCases}
            \savelocalsteps{3}
            \item[\text{(\ref{rule:IVVal})}]
            \begin{pfsteps*}
            \item $\isVal{\hpair{e_1}{e_2}}$ \BY{assumption}
            \end{pfsteps*} 
            Contradicts \pfref{eindet} by \autoref{lem:val-not-indet}.
            \restorelocalsteps{3} 
            \item[\text{(\ref{rule:IVIndet})}] 
            \begin{pfsteps*}
            \item $\notIntro{\hpair{e_1}{e_2}}$ \BY{assumption}
            \end{pfsteps*}
            Contradicts \autoref{lem:no-pair-notintro}.
            \restorelocalsteps{3}
            \item[\text{(\ref{rule:IVPair})}]
            \begin{pfsteps*}
            \item $e'=\hpair{e_1'}{e_2'}$ \BY{assumption}
            \item $\inValues{e_1'}{e_1}$ \BY{assumption} \pflabel{[pair3]invalues1}
            \item $\cnotsatisfyormay{e_1'}{\hxi_1}$ \BY{\pfref{[pair3]complete1} on \pfref{[pair3]invalues1}} \pflabel{[pair3]not-satormay1}
            \item $\cnotsatisfyormay{\hpair{e_1'}{e_2'}}{\cpair{\hxi_1}{\hxi_2}}$ \BY{\autoref{lem:satormay-pair} on \pfref{[pair3]not-satormay1}}
            \end{pfsteps*} 
          \end{byCases}
          \restorelocalsteps{2}
          \item[\cnotsatisfyormay{e_2}{\hxi_2}]
          \begin{pfsteps*}
          \item $\cnotsatisfyormay{e_2}{\hxi_2}$ \BY{assumption} \pflabel{[pair3]not-satormay-2}
          \item if $\inValues{e_2'}{e_2}$ then $\cnotsatisfyormay{e_2'}{\hxi_2}$ \BY{IH on \pfref{[pair3]e2indet} and \pfref{[pair3]e2typ} and \pfref{[pair]c2typ} and \pfref{[pair3]not-satormay-2}} \pflabel{[pair3]complete2}
          \end{pfsteps*}
          To show that if $\inValues{e'}{\hpair{e_1}{e_2}}$ then $\cnotsatisfyormay{\hpair{e_1}{e_2}}{\cpair{\hxi_1}{\hxi_2}}$, we assume $\inValues{e'}{\hpair{e_1}{e_2}}$.
          \begin{pfsteps*}
          \item $\inValues{e'}{\hpair{e_1}{e_2}}$ \BY{assumption} \pflabel{[pair3]invalues'}
          \end{pfsteps*}
          By rule induction over \rulesref{rules:inValues} on \pfref{[pair3]invalues'}.
          \begin{byCases}
            \savelocalsteps{3}
            \item[\text{(\ref{rule:IVVal})}]
            \begin{pfsteps*}
            \item $\isVal{\hpair{e_1}{e_2}}$ \BY{assumption}
            \end{pfsteps*} 
            Contradicts \pfref{eindet} by \autoref{lem:val-not-indet}.
            \restorelocalsteps{3} 
            \item[\text{(\ref{rule:IVIndet})}] 
            \begin{pfsteps*}
            \item $\notIntro{\hpair{e_1}{e_2}}$ \BY{assumption}
            \end{pfsteps*}
            Contradicts \autoref{lem:no-pair-notintro}.
            \restorelocalsteps{3}
            \item[\text{(\ref{rule:IVPair})}]
            \begin{pfsteps*}
            \item $e'=\hpair{e_1'}{e_2'}$ \BY{assumption}
            \item $\inValues{e_2'}{e_2}$ \BY{assumption} \pflabel{[pair3]invalues2}
            \item $\cnotsatisfyormay{e_2'}{\hxi_2}$ \BY{\pfref{[pair3]complete2} on \pfref{[pair3]invalues2}} \pflabel{[pair3]not-satormay2}
            \item $\cnotsatisfyormay{\hpair{e_1'}{e_2'}}{\cpair{\hxi_1}{\hxi_2}}$ \BY{\autoref{lem:satormay-pair} on \pfref{[pair3]not-satormay2}}
            \end{pfsteps*} 
          \end{byCases}
        \end{byCases}
      \end{byCases}
      \restorelocalsteps{1}
      \item[\text{(\ref{rule:IInl})}] 
      \begin{pfsteps*}
      \item $e=\hinl{\tau_2}{e_1}$ \BY{assumption}
      \end{pfsteps*} 
      By rule induction over \rulesref{rules:TExp} on \pfref{etyp}, no rule applies due to syntactic contradiction.
      \restorelocalsteps{1}
      \item[\text{(\ref{rule:IInr})}] 
      \begin{pfsteps*}
      \item $e=\hinr{\tau_1'}{e_2}$ \BY{assumption}
      \end{pfsteps*} 
      By rule induction over \rulesref{rules:TExp} on \pfref{etyp}, no rule applies due to syntactic contradiction.
    \end{byCases}
    \restorelocalsteps{0}
    \item[\text{(\ref{rule:CTOr})}]
    \begin{pfsteps*}
    \item $\hxi=\cor{\hxi_1}{\hxi_2}$ \BY{assumption}
    \item $\ctyp{\hxi_1}{\tau_1}$ \BY{assumption} \pflabel{[or]c1typ}
    \item $\ctyp{\hxi_2}{\tau_2}$ \BY{assumption} \pflabel{[or]c2typ}
    \item $\cnotsatisfyormay{e}{\cor{\hxi_1}{\hxi_2}}$ \BY{assumption} \pflabel{[or]not-satormay}
    \item $\cnotsatisfyormay{e}{\hxi_1}$ \BY{\autoref{lem:satormay-or} on \pfref{[or]not-satormay}} \pflabel{[or]not-satormay-1}
    \item $\cnotsatisfyormay{e}{\hxi_2}$ \BY{\autoref{lem:satormay-or} on \pfref{[or]not-satormay}} \pflabel{[or]not-satormay-2}
    \item if $\inValues{e'}{e}$ then $\cnotsatisfyormay{e'}{\hxi_1}$ \BY{IH on \pfref{eindet} and \pfref{etyp} and \pfref{[or]c1typ} and \pfref{[or]not-satormay-1}} \pflabel{[or]complete1}
    \item if $\inValues{e'}{e}$ then $\cnotsatisfyormay{e'}{\hxi_2}$ \BY{IH on \pfref{eindet} and \pfref{etyp} and \pfref{[or]c2typ} and \pfref{[or]not-satormay-2}} \pflabel{[or]complete2}
    \end{pfsteps*}
    To show that if $\inValues{e'}{e}$ then $\cnotsatisfyormay{e'}{\cor{\hxi_1}{\hxi_2}}$, we assume $\inValues{e'}{e}$.
    \begin{pfsteps*}
    \item $\inValues{e'}{e}$ \BY{assumption} \pflabel{[or]invalues}
    \item $\cnotsatisfyormay{e'}{\hxi_1}$ \BY{\pfref{[or]complete1} on \pfref{[or]invalues}} \pflabel{[or]not-satormay-1'}
    \item $\cnotsatisfyormay{e'}{\hxi_2}$ \BY{\pfref{[or]complete2} on \pfref{[or]invalues}} \pflabel{[or]not-satormay-2'}
    \item $\cnotsatisfyormay{e'}{\cor{\hxi_1}{\hxi_2}}$ \BY{\autoref{lem:satormay-or} on \pfref{[or]not-satormay-1'} and \pfref{[or]not-satormay-2'}}
    \end{pfsteps*}
  \end{byCases}
\end{proof}

\judgboxa{
  \hsubstyp{\theta}{\Gamma}
}{
  $\theta$ is of type $\Gamma$
}
\begin{subequations}
\begin{equation}
\inferrule[STEmpty]{ }{
  \hsubstyp{\emptyset}{\cdot}
}
\end{equation}
\begin{equation}
\inferrule[STExtend]{
  \hsubstyp{\theta}{\Gamma_\theta} \\
  \hexptyp{\Gamma}{\Delta}{e}{\tau}
}{
  \hsubstyp{\theta , x / e}{\Gamma_\theta , x : \tau}
}
\end{equation}
\end{subequations}

\judgboxa{
  \refutable{p}
}{$p$ is refutable}
\begin{subequations}\label{rules:p-refutable}
\begin{equation}\label{rule:RNum}
\inferrule[RNum]{ }{
  \refutable{\hnum{n}}
}
\end{equation}
\begin{equation}\label{rule:REHole}
\inferrule[REHole]{ }{
  \refutable{\heholep{w}}
}
\end{equation}
\begin{equation}\label{rule:RHole}
\inferrule[RHole]{ }{
  \refutable{\hholep{p}{w}{\tau}}
}
\end{equation}
\begin{equation}\label{rule:RInl}
\inferrule[RInl]{ }{
  \refutable{\hinlp{p}}
}
\end{equation}
\begin{equation}\label{rule:RInr}
\inferrule[RInr]{ }{
  \refutable{\hinrp{p}}
}
\end{equation}
\begin{equation}\label{rule:RPairL}
\inferrule[RPairL]{
  \refutable{p_1}
}{
  \refutable{\hpair{p_1}{p_2}}
}
\end{equation}
\begin{equation}\label{rule:RPairR}
\inferrule[RPairR]{
  \refutable{p_2}
}{
  \refutable{\hpair{p_1}{p_2}}
}
\end{equation}
\end{subequations}

\judgboxa{
  \hpatmatch{e}{p}{\theta}
}{
  $e$ matches $p$, emitting $\theta$
}
\begin{subequations}\label{rules:match}
\begin{equation}\label{rule:MVar}
\inferrule[MVar]{ }{
  \hpatmatch{e}{x}{e / x}
}
\end{equation}
\begin{equation}\label{rule:MWild}
\inferrule[MWild]{ }{
  \hpatmatch{e}{\_}{\cdot}
}
\end{equation}
\begin{equation}\label{rule:MNum}
\inferrule[MNum]{ }{
  \hpatmatch{\hnum{n}}{\hnum{n}}{\cdot}
}
\end{equation}
\begin{equation}\label{rule:MPair}
\inferrule[MPair]{
  \hpatmatch{e_1}{p_1}{\theta_1} \\
  \hpatmatch{e_2}{p_2}{\theta_2}
}{
  \hpatmatch{\hpair{e_1}{e_2}}{\hpair{p_1}{p_2}}{\theta_1 \uplus \theta_2}
}
\end{equation}
\begin{equation}\label{rule:MInl}
\inferrule[MInl]{
  \hpatmatch{e}{p}{\theta}
}{
  \hpatmatch{\hinl{\tau}{e}}{\hinlp{p}}{\theta}
}
\end{equation}
\begin{equation}\label{rule:MInr}
\inferrule[MInr]{
  \hpatmatch{e}{p}{\theta}
}{
  \hpatmatch{\hinr{\tau}{e}}{\hinrp{p}}{\theta}
}
\end{equation}
\begin{equation}\label{rule:MNotIntroPair}
\inferrule[MNotIntroPair]{
  \notIntro{e} \\
  \hpatmatch{\hprl{e}}{p_1}{\theta_1} \\
  \hpatmatch{\hprr{e}}{p_2}{\theta_2}
}{
  \hpatmatch{e}{\hpair{p_1}{p_2}}{\theta_1 \uplus \theta_2}
}
\end{equation}
\end{subequations}

\judgboxa{
  \hmaymatch{e}{p}
}{
  $e$ may match $p$
}
\begin{subequations}\label{rules:maymatch}
\begin{equation}\label{rule:MMEHole}
\inferrule[MMEHole]{ }{
  \hmaymatch{e}{\heholep{w}}
}
\end{equation}
\begin{equation}\label{rule:MMHole}
\inferrule[MMHole]{ }{
  \hmaymatch{e}{\hholep{p}{w}{\tau}}
}
\end{equation}
\begin{equation}\label{rule:MMNotIntro}
\inferrule[MMNotIntro]{
  \notIntro{e} \\
  \refutable{p}
}{
  \hmaymatch{e}{p}
}
\end{equation}
\begin{equation}\label{rule:MMPairL}
\inferrule[MMPairL]{
  \hmaymatch{e_1}{p_1} \\
  \hpatmatch{e_2}{p_2}{\theta_2}
}{
  \hmaymatch{\hpair{e_1}{e_2}}{\hpair{p_1}{p_2}}
}
\end{equation}
\begin{equation}\label{rule:MMPairR}
\inferrule[MMPairR]{
  \hpatmatch{e_1}{p_1}{\theta_1} \\
  \hmaymatch{e_2}{p_2}
}{
  \hmaymatch{\hpair{e_1}{e_2}}{\hpair{p_1}{p_2}}
}
\end{equation}
\begin{equation}\label{rule:MMPair}
\inferrule[MMPair]{
  \hmaymatch{e_1}{p_1} \\
  \hmaymatch{e_2}{p_2}
}{
  \hmaymatch{\hpair{e_1}{e_2}}{\hpair{p_1}{p_2}}
}
\end{equation}
\begin{equation}\label{rule:MMInl}
\inferrule[MMInl]{
  \hmaymatch{e}{p}
}{
  \hmaymatch{\hinl{\tau}{e}}{\hinlp{p}}
}
\end{equation}
\begin{equation}\label{rule:MMInr}
\inferrule[MMInr]{
  \hmaymatch{e}{p}
}{
  \hmaymatch{\hinr{\tau}{e}}{\hinrp{p}}
}
\end{equation}
\end{subequations}

\judgboxa{
  \hnotmatch{e}{p}
}{
  $e$ does not match $p$
}
\begin{subequations}\label{rules:notmatch}
\begin{equation}\label{rule:NMNum}
\inferrule[NMNum]{
  n_1 \neq n_2
}{
  \hnotmatch{\hnum{n_1}}{\hnum{n_2}}
}
\end{equation}
\begin{equation}\label{rule:NMPairL}
\inferrule[NMPairL]{
  \hnotmatch{e_1}{p_1}
}{
  \hnotmatch{\hpair{e_1}{e_2}}{\hpair{p_1}{p_2}}
}
\end{equation}
\begin{equation}\label{rule:NMPairR}
\inferrule[NMPairR]{
  \hnotmatch{e_2}{p_2}
}{
  \hnotmatch{\hpair{e_1}{e_2}}{\hpair{p_1}{p_2}}
}
\end{equation}
\begin{equation}\label{rule:NMConfL}
\inferrule[NMConfL]{ }{
  \hnotmatch{\hinr{\tau}{e}}{\hinlp{p}}
}
\end{equation}
\begin{equation}\label{rule:NMConfR}
\inferrule[NMConfR]{ }{
  \hnotmatch{\hinl{\tau}{e}}{\hinrp{p}}
}
\end{equation}
\begin{equation}\label{rule:NMInl}
\inferrule[NMInl]{
  \hnotmatch{e}{p}
}{
  \hnotmatch{\hinr{\tau}{e}}{\hinlp{p}}
}
\end{equation}
\begin{equation}\label{rule:NMInr}
\inferrule[NMInr]{
  \hnotmatch{e}{p}
}{
  \hnotmatch{\hinl{\tau}{e}}{\hinrp{p}}
}
\end{equation}
\end{subequations}

\judgboxa{\htrans{e}{e'}}{$e$ takes a step to $e'$}
\begin{subequations}\label{rules:ITExp}
\begin{equation}
\inferrule[ITHole]{
  \htrans{e}{e'}
}{
  \htrans{\hhole{e}{u}}{\hhole{e'}{u}}
}
\end{equation}
\begin{equation}
\inferrule[ITApFun]{
  \htrans{e_1}{e_1'}
}{
  \htrans{\hap{e_1}{e_2}}{\hap{e_1'}{e_2}}
}
\end{equation}
\begin{equation}
\inferrule[ITApArg]{
  \isVal{e_1} \\
  \htrans{e_2}{e_2'}
}{
  \htrans{\hap{e_1}{e_2}}{\hap{e_1}{e_2'}}
}
\end{equation}
\begin{equation}
\inferrule[ITAP]{
  \isVal{e_2}
}{
  \hap{\hlam{x}{\tau}{e_1}}{e_2} \mapsto
    [e_2/x]e_1
}
\end{equation}
\begin{equation}
\inferrule[ITPairL]{
  \htrans{e_1}{e_1'}
}{
  \htrans{\hpair{e_1}{e_2}}{\hpair{e_1'}{e_2}}
}
\end{equation}
\begin{equation}
\inferrule[ITPairR]{
  \isVal{e_1} \\
  \htrans{e_2}{e_2'}
}{
  \htrans{\hpair{e_1}{e_2}}{\hpair{e_1}{e_2'}}
}
\end{equation}
\begin{equation}
\inferrule[ITPrl]{
  \isFinal{\hpair{e_1}{e_2}}
}{
  \htrans{\hprl{\hpair{e_1}{e_2}}}{e_1}
}
\end{equation}
\begin{equation}
\inferrule[ITPrr]{
  \isFinal{\hpair{e_1}{e_2}}
}{
  \htrans{\hprr{\hpair{e_1}{e_2}}}{e_2}
}
\end{equation}
\begin{equation}
\inferrule[ITInl]{
  \htrans{e}{e'}
}{
  \htrans{\hinl{\tau}{e}}{\hinl{\tau}{e'}}
}
\end{equation}
\begin{equation}
\inferrule[ITInr]{
  \htrans{e}{e'}
}{
  \htrans{\hinr{\tau}{e}}{\hinr{\tau}{e'}}
}
\end{equation}
\begin{equation}\label{rule:ITExpMatch}
\inferrule[ITExpMatch]{
  \htrans{e}{e'}
}{
  \htrans{\hmatch{e}{\zrules}}{\hmatch{e'}{\zrules}}
}
\end{equation}
\begin{equation}\label{rule:ITSuccMatch}
\inferrule[ITSuccMatch]{
  \isFinal{e} \\
  \hpatmatch{e}{p_r}{\theta}
}{
  \htrans{
    \hmatch{e}{\zruls{rs_{pre}}{\hrulP{p_r}{e_r}}{rs_{post}}}
  }{
    [\theta](e_r)
  }
}
\end{equation}
\begin{equation}\label{rule:ITFailMatch}
\inferrule[ITFailMatch]{
  \isFinal{e} \\
  \hnotmatch{e}{p_r}
}{
  \htrans{
    \hmatch{e}{\zruls{rs}{\hrulP{p_r}{e_r}}{\hrulesP{r'}{rs'}}}
  }{
    \hmatch{e}{
      \zruls{
        \rmpointer{\zruls{rs}{\hrulP{p_r}{e_r}}{\cdot}}
      }{r'}{rs'}
    }
  }
}
\end{equation}
\end{subequations}

\begin{lemma}
  \label{lem:inl-final}
  If $\isFinal{\hinl{\tau_2}{e_1}}$ then $\isFinal{e_1}$.
\end{lemma}
\begin{proof}
By rule induction over Rules (\ref{rules:Final}) on $\isFinal{\hinl{\tau_2}{e_1}}$.
\begin{byCases}
\savelocalsteps{0}
\item[\text{(\ref{rule:FVal})}]
    \begin{pfsteps*}
    \item $\isVal{\hinl{\tau_2}{e_1}}$ \BY{assumption} \pflabel{inl-val}
    \end{pfsteps*}
    By rule induction over Rules (\ref{rules:Value}) on \pfref{inl-val}, only one case applies.
    \begin{byCases}
    \item[\text{(\ref{rule:VInl})}]
        \begin{pfsteps*}
        \item $\isVal{e_1}$ \BY{assumption} \pflabel{val}
        \item $\isFinal{e_1}$ \BY{Rule (\ref{rule:FVal}) on \pfref{val}}
        \end{pfsteps*}
    \end{byCases}
\restorelocalsteps{0}
\item[\text{(\ref{rule:FIndet})}]
    \begin{pfsteps*}
    \item $\isIndet{\hinl{\tau_2}{e_1}}$ \BY{assumption} \pflabel{inl-indet}
    \end{pfsteps*}
    By rule induction over Rules (\ref{rules:Indet}) on \pfref{inl-indet}, only one case applies.
    \begin{byCases}
    \item[\text{(\ref{rule:IInl})}]
        \begin{pfsteps*}
        \item $\isIndet{e_1}$ \BY{assumption} \pflabel{indet}
        \item $\isFinal{e_1}$ \BY{Rule (\ref{rule:FIndet}) on \pfref{indet}}
        \end{pfsteps*}
    \end{byCases}
\end{byCases}
\resetpfcounter
\end{proof}

\begin{lemma}
  \label{lem:inr-final}
  If $\isFinal{\hinr{\tau_1}{e_2}}$ then $\isFinal{e_2}$.
\end{lemma}
\begin{proof}
By rule induction over Rules (\ref{rules:Final}) on $\isFinal{\hinr{\tau_1}{e_2}}$.
\begin{byCases}
\savelocalsteps{0}
\item[\text{(\ref{rule:FVal})}]
    \begin{pfsteps*}
    \item $\isVal{\hinr{\tau_1}{e_2}}$ \BY{assumption} \pflabel{inr-val}
    \end{pfsteps*}
    By rule induction over Rules (\ref{rules:Value}) on \pfref{inr-val}, only one case applies.
    \begin{byCases}
    \item[\text{(\ref{rule:VInl})}]
        \begin{pfsteps*}
        \item $\isVal{e_2}$ \BY{assumption} \pflabel{val}
        \item $\isFinal{e_2}$ \BY{Rule (\ref{rule:FVal}) on \pfref{val}}
        \end{pfsteps*}
    \end{byCases}
\restorelocalsteps{0}
\item[\text{(\ref{rule:FIndet})}]
    \begin{pfsteps*}
    \item $\isIndet{\hinr{\tau_1}{e_2}}$ \BY{assumption} \pflabel{inr-indet}
    \end{pfsteps*}
    By rule induction over Rules (\ref{rules:Indet}) on \pfref{inr-indet}, only one case applies.
    \begin{byCases}
    \item[\text{(\ref{rule:IInl})}]
        \begin{pfsteps*}
        \item $\isIndet{e_2}$ \BY{assumption} \pflabel{indet}
        \item $\isFinal{e_2}$ \BY{Rule (\ref{rule:FIndet}) on \pfref{indet}}
        \end{pfsteps*}
    \end{byCases}
\end{byCases}
\resetpfcounter
\end{proof}

\begin{lemma}
  \label{lem:pair-final}
  If $\isFinal{\hpair{e_1}{e_2}}$ then $\isFinal{e_1}$ and $\isFinal{e_2}$.
\end{lemma}
\begin{proof}
By rule induction over Rules (\ref{rules:Final}) on $\isFinal{\hpair{e_1}{e_2}}$.
\begin{byCases}
\savelocalsteps{0}
\item[\text{(\ref{rule:FVal})}]
    \begin{pfsteps*}
    \item $\isVal{\hpair{e_1}{e_2}}$ \BY{assumption} \pflabel{pair-val}
    \end{pfsteps*}
    By rule induction over Rules (\ref{rules:Value}) on \pfref{pair-val}, only one case applies.
    \begin{byCases}
    \item[\text{(\ref{rule:VPair})}]
        \begin{pfsteps*}
        \item $\isVal{e_1}$ \BY{assumption} \pflabel{e1-val}
        \item $\isVal{e_2}$ \BY{assumption} \pflabel{e2-val}
        \item $\isFinal{e_1}$ \BY{Rule (\ref{rule:FVal}) on \pfref{e1-val}}
        \item $\isFinal{e_2}$ \BY{Rule (\ref{rule:FVal}) on \pfref{e2-val}}
        \end{pfsteps*}
    \end{byCases}
\restorelocalsteps{0}
\item[\text{(\ref{rule:FIndet})}]
    \begin{pfsteps*}
    \item $\isIndet{\hpair{e_1}{e_2}}$ \BY{assumption} \pflabel{pair-indet}
    \end{pfsteps*}
    By rule induction over Rules (\ref{rules:Indet}) on \pfref{pair-indet}, only three cases apply.
    \begin{byCases}
    \savelocalsteps{1}
    \item[\text{(\ref{rule:IPairL})}]
        \begin{pfsteps*}
        \item $\isIndet{e_1}$ \BY{assumption} \pflabel{[indet1]e1-indet}
        \item $\isVal{e_2}$ \BY{assumption} \pflabel{[indet1]e2-val}
        \item $\isFinal{e_1}$ \BY{Rule (\ref{rule:FIndet}) on \pfref{[indet1]e1-indet}}
        \item $\isFinal{e_1}$ \BY{Rule (\ref{rule:FVal}) on \pfref{[indet1]e2-val}}
        \end{pfsteps*}
    \restorelocalsteps{1}
    \item[\text{(\ref{rule:IPairR})}]
        \begin{pfsteps*}
        \item $\isVal{e_1}$ \BY{assumption} \pflabel{[indet2]e1-val}
        \item $\isIndet{e_2}$ \BY{assumption} \pflabel{[indet2]e2-indet}
        \item $\isFinal{e_1}$ \BY{Rule (\ref{rule:FVal}) on \pfref{[indet2]e1-val}}
        \item $\isFinal{e_1}$ \BY{Rule (\ref{rule:FIndet}) on \pfref{[indet2]e2-indet}}
        \end{pfsteps*}
    \restorelocalsteps{1}
    \item[\text{(\ref{rule:IPair})}]
        \begin{pfsteps*}
        \item $\isIndet{e_1}$ \BY{assumption} \pflabel{[indet3]e1-indet}
        \item $\isIndet{e_2}$ \BY{assumption} \pflabel{[indet3]e2-indet}
        \item $\isFinal{e_1}$ \BY{Rule (\ref{rule:FIndet}) on \pfref{[indet3]e1-indet}}
        \item $\isFinal{e_1}$ \BY{Rule (\ref{rule:FIndet}) on \pfref{[indet3]e2-indet}}
        \end{pfsteps*}
    \end{byCases}
\end{byCases}
\resetpfcounter
\end{proof}

\begin{lemma}
  \label{lem:no-num-notintro}
  There doesn't exist $\hnum{n}$ such that $\notIntro{\hnum{n}}$.
\end{lemma}
\begin{proof}
    By rule induction over Rules (\ref{rules:notintro}) on $\notIntro{\hnum{n}}$, no case applies due to syntactic contradiction.
\end{proof}

\begin{lemma}
  \label{lem:no-inl-notintro}
  There doesn't exist $\hinl{\tau}{e}$ such that $\notIntro{\hinl{\tau}{e}}$.
\end{lemma}
\begin{proof}
    By rule induction over Rules (\ref{rules:notintro}) on $\notIntro{\hinl{\tau}{e}}$, no case applies due to syntactic contradiction.
\end{proof}

\begin{lemma}
  \label{lem:no-inr-notintro}
  There doesn't exist $\hinr{\tau}{e}$ such that $\notIntro{\hinr{\tau}{e}}$.
\end{lemma}
\begin{proof}
    By rule induction over Rules (\ref{rules:notintro}) on $\notIntro{\hinr{\tau}{e}}$, no case applies due to syntactic contradiction.
\end{proof}

\begin{lemma}
  \label{lem:no-pair-notintro}
  There doesn't exist $\hpair{e_1}{e_2}$ such that $\notIntro{\hpair{e_1}{e_2}}$.
\end{lemma}
\begin{proof}
    By rule induction over Rules (\ref{rules:notintro}) on $\notIntro{\hpair{e_1}{e_2}}$, no case applies due to syntactic contradiction.
\end{proof}

\begin{lemma}
  \label{lem:final-notintro-indet}
  If $\isFinal{e}$ and $\notIntro{e}$ then $\isIndet{e}$.
\end{lemma}
\begin{proof}[Proof Sketch]
By rule induction over Rules (\ref{rules:notintro}) on  $\notIntro{e}$, for each case, by rule induction over Rules (\ref{rules:Value}) on $\isVal{e}$ and we notice that $\isVal{e}$ is not derivable. By rule induction over Rules (\ref{rules:Final}) on $\isFinal{e}$, Rule (\ref{rule:FVal}) result in a contradiction with the fact that $\isVal{e}$ is not derivable while Rule (\ref{rule:FIndet}) tells us $\isIndet{e}$.
\end{proof}

\begin{lemma}
  \label{lem:val-not-indet}
  There doesn't exist such an expression $e$ such that both $\isVal{e}$ and $\isIndet{e}$.
\end{lemma}

\begin{lemma}
  \label{lem:val-not-notintro}
  There doesn't exist such an expression $e$ such that both $\isVal{e}$ and $\notIntro{e}$.
\end{lemma}

\begin{lemma}[Finality]
  \label{lem:finality}
  There doesn't exist such an expression $e$ such that both $\isFinal{e}$ and $\htrans{e}{e'}$ for some $e'$
\end{lemma}
\begin{proof}Assume there exists such an $e$ such that both $\isFinal{e}$ and $\htrans{e}{e'}$ for some $e'$ then proof by contradiction.
 
  By rule induction over Rules (\ref{rules:Final}) and Rules (\ref{rules:ITExp}), \textit{i.e.}, over Rules (\ref{rules:Value}) and Rules (\ref{rules:ITExp}) and over Rules (\ref{rules:Indet}) and Rules (\ref{rules:ITExp}) respectively. The proof can be done by straightforward observation of syntactic contradictions.
\end{proof}

\begin{lemma}[Matching Determinism]
  \label{lem:match-determinism}
  If $\isFinal{e}$ and $\hexptyp{\cdot}{\Delta_e}{e}{\tau}$ and $\chpattyp{p}{\tau}{\xi}{\Gamma}{\Delta}$ then exactly one of the following holds
  \begin{enumerate}
    \item $\hpatmatch{e}{p}{\theta}$ for some $\theta$
    \item $\hmaymatch{e}{p}$
    \item $\hnotmatch{e}{p}$
  \end{enumerate}
\end{lemma}
\begin{proof}
\begin{pfsteps*}
\item $\isFinal{e}$ \BY{assumption} \pflabel{e-final}
\item $\hexptyp{\cdot}{\Delta_e}{e}{\tau}$ \BY{assumption} \pflabel{e-typ}
\item $\chpattyp{p}{\tau}{\xi}{\Gamma}{\Delta}$ \BY{assumption} \pflabel{p-typ}
\end{pfsteps*}
By rule induction over Rules (\ref{rules:PatTyp}) on \pfref{p-typ}, we would show one conclusion is derivable while the other two are not.
\begin{byCases}
\savelocalsteps{0}
\item[\text{(\ref{rule:PTVar})}]
    \begin{pfsteps*}
    \item $p=x$ \BY{assumption}
    \item $\hpatmatch{e}{x}{e / x}$ \BY{\ruleref{rule:MVar}}
    \end{pfsteps*}
    Assume $\hmaymatch{e}{x}$. By rule induction over \rulesref{rules:maymatch} on it, only one case applies.
    \begin{byCases}
    \item[\text{(\ref{rule:MMNotIntro})}]
        \begin{pfsteps*}
        \item $\refutable{x}$ \BY{assumption} \pflabel{[var]rft}
        \end{pfsteps*}
        By rule induction over \rulesref{rules:p-refutable} on \pfref{[var]rft}, no case applies due to syntactic contradiction.
    \end{byCases}
    \begin{pfsteps*}
    \item $\cancel{\hmaymatch{e}{x}}$ \BY{contradiction}
    \end{pfsteps*}
    Assume $\hnotmatch{e}{x}$. By rule induction over \rulesref{rules:notmatch} on it, no case applies due to syntactic contradiction.
    \begin{pfsteps*}
    \item $\cancel{\hnotmatch{e}{x}}$ \BY{contradiction}
    \end{pfsteps*}
\restorelocalsteps{0}
\item[\text{(\ref{rule:PTWild})}]
    \begin{pfsteps*}
    \item $p=\_$ \BY{assumption}
    \item $\hpatmatch{e}{\_}{\cdot}$ \BY{\ruleref{rule:MWild}}
    \end{pfsteps*}
    Assume $\hmaymatch{e}{\_}$. By rule induction over \rulesref{rules:maymatch} on it, only one case applies.
    \begin{byCases}
    \item[\text{(\ref{rule:MMNotIntro})}]
        \begin{pfsteps*}
        \item $\refutable{\_}$ \BY{assumption} \pflabel{[wild]rft}
        \end{pfsteps*}
        By rule induction over \rulesref{rules:p-refutable} on \pfref{[wild]rft}, no case applies due to syntactic contradiction.
    \end{byCases}
    \begin{pfsteps*}
    \item $\cancel{\hmaymatch{e}{\_}}$ \BY{contradiction}
    \end{pfsteps*}
    Assume $\hnotmatch{e}{\_}$. By rule induction over \rulesref{rules:notmatch} on it, no case applies due to syntactic contradiction.
    \begin{pfsteps*}
    \item $\cancel{\hnotmatch{e}{\_}}$ \BY{contradiction}
    \end{pfsteps*}
\restorelocalsteps{0}
\item[\text{(\ref{rule:PTEHole})}]
    \begin{pfsteps*}
    \item $p=\heholep{w}$ \BY{assumption}
    \item $\hmaymatch{e}{\heholep{w}}$ \BY{\ruleref{rule:MMEHole}}
    \end{pfsteps*}
    Assume $\hpatmatch{e}{\heholep{w}}{\theta}$ for some $\theta$. By rule induction over \rulesref{rules:maymatch} on it, no case applies due to syntactic contradiction.
    \begin{pfsteps*}
    \item $\cancel{\hpatmatch{e}{\heholep{w}}{\theta}}$ \BY{contradiction}
    \end{pfsteps*}
    Assume $\hnotmatch{e}{\heholep{w}}$. By rule induction over \rulesref{rules:notmatch} on it, no case applies due to syntactic contradiction.
    \begin{pfsteps*}
    \item $\cancel{\hnotmatch{e}{\heholep{w}}}$ \BY{contradiction}
    \end{pfsteps*}
\restorelocalsteps{0}
\item[\text{(\ref{rule:PTHole})}]
    \begin{pfsteps*}
    \item $p=\hholep{p_0}{w}{\tau'}$ \BY{assumption}
    \item $\hmaymatch{e}{\hholep{p_0}{w}{\tau'}}$ \BY{\ruleref{rule:MMHole}}
    \end{pfsteps*}
    Assume $\hpatmatch{e}{\hholep{p_0}{w}{\tau'}}{\theta}$ for some $\theta$. By rule induction over \rulesref{rules:maymatch} on it, no case applies due to syntactic contradiction.
    \begin{pfsteps*}
    \item $\cancel{\hpatmatch{e}{\hholep{p_0}{w}{\tau'}}{\theta}}$ \BY{contradiction}
    \end{pfsteps*}
    Assume $\hnotmatch{e}{\hholep{p_0}{w}{\tau'}}$. By rule induction over \rulesref{rules:notmatch} on it, no case applies due to syntactic contradiction.
    \begin{pfsteps*}
    \item $\cancel{\hnotmatch{e}{\hholep{p_0}{w}{\tau'}}}$ \BY{contradiction}
    \end{pfsteps*}
\restorelocalsteps{0}
\item[\text{(\ref{rule:PTNum})}]
    \begin{pfsteps*}
    \item $p=\hnum{n_2}$ \BY{assumption}
    \item $\tau=\tnum$ \BY{assumption}
    \item $\xi=\cnum{n_2}$ \BY{assumption}
    \item $\refutable{\hnum{n_2}}$ \BY{\ruleref{rule:RNum}} \pflabel{[num]rft}
    \end{pfsteps*}
    By rule induction over Rules (\ref{rules:TExp}) on \pfref{e-typ}, the following cases apply.
    \begin{byCases}
    \savelocalsteps{1}
    \item[\text{(\ref{rule:TEHole}),(\ref{rule:THole}),(\ref{rule:TAp}),(\ref{rule:TPrl}),(\ref{rule:TPrr}),(\ref{rule:TMatchZPre}),(\ref{rule:TMatchNZPre})}]
        \begin{pfsteps*}
        \item $e=\hehole{u},\hhole{e_0}{u},\hap{e_1}{e_2},\hprl{e_0},\hprr{e_0},\hmatch{e_0}{\zrules}$ \BY{assumption}
        \item $\notIntro{e}$ \BY{Rule (\ref{rule:NVEHole}),(\ref{rule:NVHole}),(\ref{rule:NVAp}),(\ref{rule:NVMatch}),(\ref{rule:NVPrl}),(\ref{rule:NVPrr})} \pflabel{[num]notintro}
        \item $\hmaymatch{e}{\hnum{n_2}}$ \BY{\ruleref{rule:CMSNotIntro} on \pfref{[num]rft} and \pfref{[num]notintro}}
        \end{pfsteps*}
        Assume $\hpatmatch{e}{\hnum{n_2}}{\theta}$ for some $\theta$. By rule induction over it, no case applies due to syntactic contradiction.
        \begin{pfsteps*}
        \item $\cancel{\hpatmatch{e}{\hnum{n_2}}{\theta}}$ \BY{contradiction}
        \end{pfsteps*}
        Assume $\hnotmatch{e}{\hnum{n_2}}$. By rule induction over it, no case applies due to syntactic contradiction.
        \begin{pfsteps*}
        \item $\cancel{\hnotmatch{e}{\hnum{n_2}}}$ \BY{contradiction}
        \end{pfsteps*}
    \restorelocalsteps{1}
    \item[\text{(\ref{rule:TNum})}]
        \begin{pfsteps*}
        \item $e=\hnum{n_1}$
        \end{pfsteps*}
        Assume $\hmaymatch{\hnum{n_1}}{\hnum{n_2}}$. By rule induction over \rulesref{rules:maymatch} on it, only two cases apply.
        \begin{byCases}
        \item[\text{(\ref{rule:MMNotIntro})}]
            \begin{pfsteps*}
            \item $\notIntro{\hnum{n_1}}$ \BY{assumption}
            \end{pfsteps*}
            Contradicts \autoref{lem:no-num-notintro}.
        \end{byCases}
        \begin{pfsteps*}
        \item $\cancel{\hmaymatch{\hnum{n_1}}{\hnum{n_2}}}$ \BY{contradiction}
        \end{pfsteps*}
        By case analysis on whether $n_1=n_2$.
        \begin{byCases}
        \savelocalsteps{2}
        \item[n_1=n_2]
            \begin{pfsteps*}
            \item $n_1=n_2$ \BY{assumption} \pflabel{[num]eq}
            \item $\hpatmatch{\hnum{n_1}}{\hnum{n_2}}{\cdot}$ \BY{\ruleref{rule:MNum}}
            \end{pfsteps*}
            Assume $\hnotmatch{\hnum{n_1}}{\hnum{n_2}}$. By rule induction over \rulesref{rules:notmatch} on it, only one case applies.
            \begin{byCases}
            \item[\text{(\ref{rule:NMNum})}]
                \begin{pfsteps*}
                \item $n_1\neq n_2$ \BY{assumption}
                \end{pfsteps*}
                Contradicts \pfref{[num]eq}.
            \end{byCases}
            \begin{pfsteps*}
            \item $\cancel{\hnotmatch{\hnum{n_1}}{\hnum{n_2}}}$ \BY{contradiction}
            \end{pfsteps*}
        \restorelocalsteps{2}
        \item[n_1\neq n_2]
            \begin{pfsteps*}
            \item $n_1\neq n_2$ \BY{assumption} \pflabel{[num]neq}
            \item $\hnotmatch{\hnum{n_1}}{\hnum{n_2}}$ \BY{\ruleref{rule:NMNum} on \pfref{[num]neq}}
            \end{pfsteps*}
            Assume $\hpatmatch{\hnum{n_1}}{\hnum{n_2}}{\theta}$ for some $\theta$. By rule induction over \rulesref{rules:match} on it, no case applies due to syntactic contradiction.
            \begin{pfsteps*}
            \item $\cancel{\hpatmatch{\hnum{n_1}}{\hnum{n_2}}{\theta}}$ \BY{contradiction}
            \end{pfsteps*}
        \end{byCases}
    \end{byCases}
\restorelocalsteps{0}
\item[\text{(\ref{rule:PTInl})}]
    \begin{pfsteps*}
    \item $p=\hinlp{p_1}$ \BY{assumption}
    \item $\tau=\tsum{\tau_1}{\tau_2}$ \BY{assumption}
    \item $\xi=\cinl{\xi_1}$ \BY{assumption}
    \item $\chpattyp{p_1}{\tau_1}{\xi_1}{\Gamma}{\Delta}$ \BY{assumption} \pflabel{[inl]p1-typ}
    \item $\refutable{\hinlp{p_1}}$ \BY{\ruleref{rule:RInl}} \pflabel{[inl]rft}
    \end{pfsteps*}
    By rule induction over Rules (\ref{rules:TExp}) on \pfref{e-typ}, the following cases apply.
    \begin{byCases}
    \savelocalsteps{1}
    \item[\text{(\ref{rule:TEHole}),(\ref{rule:THole}),(\ref{rule:TAp}),(\ref{rule:TPrl}),(\ref{rule:TPrr}),(\ref{rule:TMatchZPre}),(\ref{rule:TMatchNZPre})}]
        \begin{pfsteps*}
        \item $e=\hehole{u},\hhole{e_0}{u},\hap{e_1}{e_2},\hprl{e_0},\hprr{e_0},\hmatch{e_0}{\zrules}$ \BY{assumption}
        \item $\notIntro{e}$ \BY{Rule (\ref{rule:NVEHole}),(\ref{rule:NVHole}),(\ref{rule:NVAp}),(\ref{rule:NVMatch}),(\ref{rule:NVPrl}),(\ref{rule:NVPrr})} \pflabel{[inl]notintro}
        \item $\hmaymatch{e}{\hinlp{p_1}}$ \BY{\ruleref{rule:CMSNotIntro} on \pfref{[inl]rft} and \pfref{[inl]notintro}}
        \end{pfsteps*}
        Assume $\hpatmatch{e}{\hinlp{p_1}}{\theta_1}$ for some $\theta_1$. By rule induction over it, no case applies due to syntactic contradiction.
        \begin{pfsteps*}
        \item $\cancel{\hpatmatch{e}{\hinlp{p_1}}{\theta_1}}$ \BY{contradiction}
        \end{pfsteps*}
        Assume $\hnotmatch{e}{\hinlp{p_1}}$. By rule induction over it, no case applies due to syntactic contradiction.
        \begin{pfsteps*}
        \item $\cancel{\hnotmatch{e}{\hinlp{p_1}}}$ \BY{contradiction}
        \end{pfsteps*}
    \restorelocalsteps{1}
    \item[\text{(\ref{rule:TInl})}]
        \begin{pfsteps*}
        \item $e=\hinl{\tau_2}{e_1}$ \BY{assumption}
        \item $\hexptyp{\cdot}{\Delta_e}{e_1}{\tau_1}$ \BY{assumption} \pflabel{[inl]e1-typ}
        \item $\isFinal{e_1}$ \BY{\autoref{lem:inl-final} on \pfref{e-final}} \pflabel{[inl]e1-final}
        \end{pfsteps*}
        By inductive hypothesis on \pfref{[inl]e1-typ} and \pfref{[inl]e1-final} and \pfref{[inl]p1-typ}, exactly one of $\hpatmatch{e_1}{p_1}{\theta_1}$ for some $\theta_1$, $\hmaymatch{e_1}{p_1}$, and $\hnotmatch{e_1}{p_1}$ holds.\\
        By case analysis on which one holds.
        \begin{byCases}
        \savelocalsteps{2}
        \item[\hpatmatch{e_1}{p_1}{\theta_1}]
            \begin{pfsteps*}
            \item $\hpatmatch{e_1}{p_1}{\theta_1}$ \BY{assumption} \pflabel{[inl1]match}
            \item $\cancel{\hmaymatch{e_1}{p_1}}$ \BY{assumption} \pflabel{[inl1]not-maymatch}
            \item $\cancel{\hnotmatch{e_1}{p_1}}$ \BY{assumption} \pflabel{[inl1]not-notmatch}
            \item $\hpatmatch{\hinl{\tau_2}{e_1}}{\hinlp{p_1}}{\theta_1}$ \BY{\ruleref{rule:MInl} on \pfref{[inl1]match}}
            \end{pfsteps*}
            Assume $\hmaymatch{\hinl{\tau_2}{e_1}}{\hinlp{p_1}}$. By rule induction over \rulesref{rules:maymatch} on it, only two cases apply.
            \begin{byCases}
            \savelocalsteps{3}
            \item[\text{(\ref{rule:MMNotIntro})}]
                \begin{pfsteps*}
                \item $\notIntro{\hinl{\tau_2}{e_1}}$ \BY{assumption}
                \end{pfsteps*}
                Contradicts \autoref{lem:no-inl-notintro}.
            \restorelocalsteps{3}
            \item[\text{(\ref{rule:MMInl})}]
                \begin{pfsteps*}
                \item $\hmaymatch{e_1}{p_1}$ \BY{assumption}
                \end{pfsteps*}
                Contradicts \pfref{[inl1]not-maymatch}.
            \end{byCases}
            \begin{pfsteps*}
            \item $\cancel{\hmaymatch{\hinl{\tau_2}{e_1}}{\hinlp{p_1}}}$ \BY{contradiction}
            \end{pfsteps*}
            Assume $\hnotmatch{\hinl{\tau_2}{e_1}}{\hinlp{p_1}}$. By rule induction over \rulesref{rules:notmatch} on it, only one case applies.
            \begin{byCases}
            \item[\text{(\ref{rule:NMInl})}]
                \begin{pfsteps*}
                \item $\hnotmatch{e_1}{p_1}$ \BY{assumption}
                \end{pfsteps*}
                Contradicts \pfref{[inl1]not-notmatch}.
            \end{byCases}
            \begin{pfsteps*}
            \item $\cancel{\hnotmatch{\hinl{\tau_2}{e_1}}{\hinlp{p_1}}}$ \BY{contradiction}
            \end{pfsteps*}
        \restorelocalsteps{2}
        \item[\hmaymatch{e_1}{p_1}]
            \begin{pfsteps*}
            \item $\cancel{\hpatmatch{e_1}{p_1}{\theta_1}}$ \BY{assumption} \pflabel{[inl2]not-match}
            \item $\hmaymatch{e_1}{p_1}$ \BY{assumption} \pflabel{[inl2]maymatch}
            \item $\cancel{\hnotmatch{e_1}{p_1}}$ \BY{assumption} \pflabel{[inl2]not-notmatch}
            \item $\hmaymatch{\hinl{\tau_2}{e_1}}{\hinlp{p_1}}$ \BY{\ruleref{rule:MMInl} on \pfref{[inl2]maymatch}}
            \end{pfsteps*}
            Assume $\hpatmatch{\hinl{\tau_2}{e_1}}{\hinlp{p_1}}{\theta}$ for some $\theta$. By rule induction over \rulesref{rules:match} on it, only one case applies.
            \begin{byCases}
            \item[\text{(\ref{rule:MInl})}]
                \begin{pfsteps*}
                \item $\hpatmatch{e_1}{p_1}{\theta}$ \BY{assumption}
                \end{pfsteps*}
                Contradicts \pfref{[inl2]not-match}.
            \end{byCases}
            \begin{pfsteps*}
            \item $\cancel{\hpatmatch{\hinl{\tau_2}{e_1}}{\hinlp{p_1}}{\theta}}$ \BY{contradiction}
            \end{pfsteps*}
            Assume $\hnotmatch{\hinl{\tau_2}{e_1}}{\hinlp{p_1}}$. By rule induction over \rulesref{rules:notmatch} on it, only one case applies.
            \begin{byCases}
            \item[\text{(\ref{rule:NMInl})}]
                \begin{pfsteps*}
                \item $\hnotmatch{e_1}{p_1}$ \BY{assumption}
                \end{pfsteps*}
                Contradicts \pfref{[inl2]not-notmatch}.
            \end{byCases}
            \begin{pfsteps*}
            \item $\cancel{\hnotmatch{\hinl{\tau_2}{e_1}}{\hinlp{p_1}}}$ \BY{contradiction}
            \end{pfsteps*}
        \restorelocalsteps{2}
        \item[\hnotmatch{e_1}{p_1}]
            \begin{pfsteps*}
            \item $\cancel{\hpatmatch{e_1}{p_1}{\theta_1}}$ \BY{assumption} \pflabel{[inl3]not-match}
            \item $\cancel{\hmaymatch{e_1}{p_1}}$ \BY{assumption} \pflabel{[inl3]not-maymatch}
            \item $\hnotmatch{e_1}{p_1}$ \BY{assumption} \pflabel{[inl3]notmatch}
            \item $\hnotmatch{\hinl{\tau_2}{e_1}}{\hinlp{p_1}}$ \BY{\ruleref{rule:NMInl} on \pfref{[inl3]notmatch}}
            \end{pfsteps*}
            Assume $\hpatmatch{\hinl{\tau_2}{e_1}}{\hinlp{p_1}}{\theta}$ for some $\theta$. By rule induction over \rulesref{rules:match} on it, only one case applies.
            \begin{byCases}
            \item[\text{(\ref{rule:MInl})}]
                \begin{pfsteps*}
                \item $\hpatmatch{e_1}{p_1}{\theta}$ \BY{assumption}
                \end{pfsteps*}
                Contradicts \pfref{[inl3]not-match}.
            \end{byCases}
            \begin{pfsteps*}
            \item $\cancel{\hpatmatch{\hinl{\tau_2}{e_1}}{\hinlp{p_1}}{\theta}}$ \BY{contradiction}
            \end{pfsteps*}
            Assume $\hmaymatch{\hinl{\tau_2}{e_1}}{\hinlp{p_1}}$. By rule induction over \rulesref{rules:maymatch} on it, only two cases apply.
            \begin{byCases}
            \savelocalsteps{3}
            \item[\text{(\ref{rule:MMNotIntro})}]
                \begin{pfsteps*}
                \item $\notIntro{\hinl{\tau_2}{e_1}}$ \BY{assumption}
                \end{pfsteps*}
                Contradicts \autoref{lem:no-inl-notintro}.
            \restorelocalsteps{3}
            \item[\text{(\ref{rule:MMInl})}]
                \begin{pfsteps*}
                \item $\hmaymatch{e_1}{p_1}$ \BY{assumption}
                \end{pfsteps*}
                Contradicts \pfref{[inl3]not-maymatch}.
            \end{byCases}
            \begin{pfsteps*}
            \item $\cancel{\hmaymatch{\hinl{\tau_2}{e_1}}{\hinlp{p_1}}}$ \BY{contradiction}
            \end{pfsteps*}
        \end{byCases}
    \end{byCases}
\restorelocalsteps{0}
\item[\text{(\ref{rule:PTInr})}]
    \begin{pfsteps*}
    \item $p=\hinrp{p_2}$ \BY{assumption}
    \item $\tau=\tsum{\tau_1}{\tau_2}$ \BY{assumption}
    \item $\xi=\cinr{\xi_2}$ \BY{assumption}
    \item $\chpattyp{p_2}{\tau_2}{\xi_2}{\Gamma}{\Delta}$ \BY{assumption} \pflabel{[inr]p2-typ}
    \item $\refutable{\hinrp{p_2}}$ \BY{\ruleref{rule:RInr}} \pflabel{[inr]rft}
    \end{pfsteps*}
    By rule induction over Rules (\ref{rules:TExp}) on \pfref{e-typ}, the following cases apply.
    \begin{byCases}
    \savelocalsteps{1}
    \item[\text{(\ref{rule:TEHole}),(\ref{rule:THole}),(\ref{rule:TAp}),(\ref{rule:TPrl}),(\ref{rule:TPrr}),(\ref{rule:TMatchZPre}),(\ref{rule:TMatchNZPre})}]
        \begin{pfsteps*}
        \item $e=\hehole{u},\hhole{e_0}{u},\hap{e_1}{e_2},\hprl{e_0},\hprr{e_0},\hmatch{e_0}{\zrules}$ \BY{assumption}
        \item $\notIntro{e}$ \BY{Rule (\ref{rule:NVEHole}),(\ref{rule:NVHole}),(\ref{rule:NVAp}),(\ref{rule:NVMatch}),(\ref{rule:NVPrl}),(\ref{rule:NVPrr})} \pflabel{[inr]notintro}
        \item $\hmaymatch{e}{\hinrp{p_2}}$ \BY{\ruleref{rule:CMSNotIntro} on \pfref{[inr]rft} and \pfref{[inr]notintro}}
        \end{pfsteps*}
        Assume $\hpatmatch{e}{\hinrp{p_2}}{\theta_2}$ for some $\theta_2$. By rule induction over it, no case applies due to syntactic contradiction.
        \begin{pfsteps*}
        \item $\cancel{\hpatmatch{e}{\hinrp{p_2}}{\theta_2}}$ \BY{contradiction}
        \end{pfsteps*}
        Assume $\hnotmatch{e}{\hinrp{p_2}}$. By rule induction over it, no case applies due to syntactic contradiction.
        \begin{pfsteps*}
        \item $\cancel{\hnotmatch{e}{\hinrp{p_2}}}$ \BY{contradiction}
        \end{pfsteps*}
    \restorelocalsteps{1}
    \item[\text{(\ref{rule:TInr})}]
        \begin{pfsteps*}
        \item $e=\hinr{\tau_1}{e_2}$ \BY{assumption}
        \item $\hexptyp{\cdot}{\Delta_e}{e_2}{\tau_2}$ \BY{assumption} \pflabel{[inr]e2-typ}
        \item $\isFinal{e_2}$ \BY{\autoref{lem:inr-final} on \pfref{e-final}} \pflabel{[inr]e2-final}
        \end{pfsteps*}
        By inductive hypothesis on \pfref{[inr]e2-typ} and \pfref{[inr]e2-final} and \pfref{[inr]p2-typ}, exactly one of $\hpatmatch{e_2}{p_2}{\theta_2}$ for some $\theta_2$, $\hmaymatch{e_2}{p_2}$, and $\hnotmatch{e_2}{p_2}$ holds.\\
        By case analysis on which one holds.
        \begin{byCases}
        \savelocalsteps{2}
        \item[\hpatmatch{e_2}{p_2}{\theta_2}]
            \begin{pfsteps*}
            \item $\hpatmatch{e_2}{p_2}{\theta_2}$ \BY{assumption} \pflabel{[inr1]match}
            \item $\cancel{\hmaymatch{e_2}{p_2}}$ \BY{assumption} \pflabel{[inr1]not-maymatch}
            \item $\cancel{\hnotmatch{e_2}{p_2}}$ \BY{assumption} \pflabel{[inr1]not-notmatch}
            \item $\hpatmatch{\hinr{\tau_1}{e_2}}{\hinrp{p_2}}{\theta_2}$ \BY{\ruleref{rule:MInr} on \pfref{[inr1]match}}
            \end{pfsteps*}
            Assume $\hmaymatch{\hinr{\tau_1}{e_2}}{\hinrp{p_2}}$. By rule induction over \rulesref{rules:maymatch} on it, only two cases apply.
            \begin{byCases}
            \savelocalsteps{3}
            \item[\text{(\ref{rule:MMNotIntro})}]
                \begin{pfsteps*}
                \item $\notIntro{\hinr{\tau_1}{e_2}}$ \BY{assumption}
                \end{pfsteps*}
                Contradicts \autoref{lem:no-inr-notintro}.
            \restorelocalsteps{3}
            \item[\text{(\ref{rule:MMInr})}]
                \begin{pfsteps*}
                \item $\hmaymatch{e_2}{p_2}$ \BY{assumption}
                \end{pfsteps*}
                Contradicts \pfref{[inr1]not-maymatch}.
            \end{byCases}
            \begin{pfsteps*}
            \item $\cancel{\hmaymatch{\hinr{\tau_1}{e_2}}{\hinrp{p_2}}}$ \BY{contradiction}
            \end{pfsteps*}
            Assume $\hnotmatch{\hinr{\tau_1}{e_2}}{\hinrp{p_2}}$. By rule induction over \rulesref{rules:notmatch} on it, only one case applies.
            \begin{byCases}
            \item[\text{(\ref{rule:NMInr})}]
                \begin{pfsteps*}
                \item $\hnotmatch{e_2}{p_2}$ \BY{assumption}
                \end{pfsteps*}
                Contradicts \pfref{[inr1]not-notmatch}.
            \end{byCases}
            \begin{pfsteps*}
            \item $\cancel{\hnotmatch{\hinr{\tau_1}{e_2}}{\hinrp{p_2}}}$ \BY{contradiction}
            \end{pfsteps*}
        \restorelocalsteps{2}
        \item[\hmaymatch{e_2}{p_2}]
            \begin{pfsteps*}
            \item $\cancel{\hpatmatch{e_2}{p_2}{\theta}}$ \BY{assumption} \pflabel{[inr2]not-match}
            \item $\hmaymatch{e_2}{p_2}$ \BY{assumption} \pflabel{[inr2]maymatch}
            \item $\cancel{\hnotmatch{e_2}{p_2}}$ \BY{assumption} \pflabel{[inr2]not-notmatch}
            \item $\hmaymatch{\hinr{\tau_1}{e_2}}{\hinrp{p_2}}$ \BY{\ruleref{rule:MMInr} on \pfref{[inr2]maymatch}}
            \end{pfsteps*}
            Assume $\hpatmatch{\hinr{\tau_1}{e_2}}{\hinrp{p_2}}{\theta}$ for some $\theta$. By rule induction over \rulesref{rules:match} on it, only one case applies.
            \begin{byCases}
            \item[\text{(\ref{rule:MInr})}]
                \begin{pfsteps*}
                \item $\hpatmatch{e_2}{p_2}{\theta}$ \BY{assumption}
                \end{pfsteps*}
                Contradicts \pfref{[inr2]not-match}.
            \end{byCases}
            \begin{pfsteps*}
            \item $\cancel{\hpatmatch{\hinr{\tau_1}{e_2}}{\hinrp{p_2}}{\theta}}$ \BY{contradiction}
            \end{pfsteps*}
            Assume $\hnotmatch{\hinr{\tau_1}{e_2}}{\hinrp{p_2}}$. By rule induction over \rulesref{rules:notmatch} on it, only one case applies.
            \begin{byCases}
            \item[\text{(\ref{rule:NMInr})}]
                \begin{pfsteps*}
                \item $\hnotmatch{e_2}{p_2}$ \BY{assumption}
                \end{pfsteps*}
                Contradicts \pfref{[inr2]not-notmatch}.
            \end{byCases}
            \begin{pfsteps*}
            \item $\cancel{\hnotmatch{\hinr{\tau_1}{e_2}}{\hinrp{p_2}}}$ \BY{contradiction}
            \end{pfsteps*}
        \restorelocalsteps{2}
        \item[\hnotmatch{e_2}{p_2}]
            \begin{pfsteps*}
            \item $\cancel{\hpatmatch{e_2}{p_2}{\theta}}$ \BY{assumption} \pflabel{[inr3]not-match}
            \item $\cancel{\hmaymatch{e_2}{p_2}}$ \BY{assumption} \pflabel{[inr3]not-maymatch}
            \item $\hnotmatch{e_2}{p_2}$ \BY{assumption} \pflabel{[inr3]notmatch}
            \item $\hnotmatch{\hinr{\tau_1}{e_2}}{\hinrp{p_2}}$ \BY{\ruleref{rule:NMInr} on \pfref{[inr3]notmatch}}
            \end{pfsteps*}
            Assume $\hpatmatch{\hinr{\tau_1}{e_2}}{\hinrp{p_2}}{\theta}$ for some $\theta$. By rule induction over \rulesref{rules:match} on it, only one case applies.
            \begin{byCases}
            \item[\text{(\ref{rule:MInr})}]
                \begin{pfsteps*}
                \item $\hpatmatch{e_2}{p_2}{\theta}$ \BY{assumption}
                \end{pfsteps*}
                Contradicts \pfref{[inr3]not-match}.
            \end{byCases}
            \begin{pfsteps*}
            \item $\cancel{\hpatmatch{\hinr{\tau_1}{e_2}}{\hinrp{p_2}}{\theta}}$ \BY{contradiction}
            \end{pfsteps*}
            Assume $\hmaymatch{\hinr{\tau_1}{e_2}}{\hinrp{p_2}}$. By rule induction over \rulesref{rules:maymatch} on it, only two cases apply.
            \begin{byCases}
            \savelocalsteps{3}
            \item[\text{(\ref{rule:MMNotIntro})}]
                \begin{pfsteps*}
                \item $\notIntro{\hinr{\tau_1}{e_2}}$ \BY{assumption}
                \end{pfsteps*}
                Contradicts \autoref{lem:no-inr-notintro}.
            \restorelocalsteps{3}
            \item[\text{(\ref{rule:MMInr})}]
                \begin{pfsteps*}
                \item $\hmaymatch{e_2}{p_2}$ \BY{assumption}
                \end{pfsteps*}
                Contradicts \pfref{[inr3]not-maymatch}.
            \end{byCases}
            \begin{pfsteps*}
            \item $\cancel{\hmaymatch{\hinr{\tau_1}{e_2}}{\hinrp{p_2}}}$ \BY{contradiction}
            \end{pfsteps*}
        \end{byCases}
    \end{byCases}
\restorelocalsteps{0}
\item[\text{(\ref{rule:PTPair})}]
    \begin{pfsteps*}
    \item $p=\hpair{p_1}{p_2}$ \BY{assumption}
    \item $\tau=\tprod{\tau_1}{\tau_2}$ \BY{assumption}
    \item $\xi=\cpair{\xi_1}{\xi_2}$ \BY{assumption}
    \item $\Gamma=\Gamma_1 \uplus \Gamma_2$ \BY{assumption}
    \item $\Delta=\Delta_1 \uplus \Delta_2$ \BY{assumption}
    \item $\chpattyp{p_1}{\tau_1}{\xi_1}{\Gamma_1}{\Delta_1}$ \BY{assumption} \pflabel{[pair]p1-typ}
    \item $\chpattyp{p_2}{\tau_2}{\xi_2}{\Gamma_2}{\Delta_2}$ \BY{assumption} \pflabel{[pair]p2-typ}
    \end{pfsteps*}
    By rule induction over Rules (\ref{rules:TExp}) on \pfref{e-typ}, the following cases apply.
    \begin{byCases}
    \savelocalsteps{1}
    \item[\text{(\ref{rule:TEHole}),(\ref{rule:THole}),(\ref{rule:TAp}),(\ref{rule:TPrl}),(\ref{rule:TPrr}),(\ref{rule:TMatchZPre}),(\ref{rule:TMatchNZPre})}]
        \begin{pfsteps*}
        \item $e=\hehole{u},\hhole{e_0}{u},\hap{e_1}{e_2},\hprl{e_0},\hprr{e_0},\hmatch{e_0}{\zrules}$ \BY{assumption}
        \item $\notIntro{e}$ \BY{Rule (\ref{rule:NVEHole}),(\ref{rule:NVHole}),(\ref{rule:NVAp}),(\ref{rule:NVMatch}),(\ref{rule:NVPrl}),(\ref{rule:NVPrr})} \pflabel{[epair]notintro}
        \item $\isIndet{e}$ \BY{\autoref{lem:final-notintro-indet} on \pfref{e-final} and \pfref{[epair]notintro}} \pflabel{[epair]indet}
        \item $\isIndet{\hprl{e}}$ \BY{Rule (\ref{rule:IPrl}) on \pfref{[epair]indet}} \pflabel{[epair]prl-indet}
        \item $\isFinal{\hprl{e}}$ \BY{Rule (\ref{rule:FIndet}) on \pfref{[epair]prl-indet}} \pflabel{[epair]prl-final}
        \item $\isIndet{\hprr{e}}$ \BY{Rule (\ref{rule:IPrr}) on \pfref{[epair]indet}} \pflabel{[epair]prr-indet}
        \item $\isFinal{\hprr{e}}$ \BY{Rule (\ref{rule:FIndet}) on \pfref{[epair]prr-indet}} \pflabel{[epair]prr-final}
        \item $\hexptyp{\cdot}{\Delta}{\hprl{e}}{\tau_1}$ \BY{Rule (\ref{rule:TPrl}) on \pfref{e-typ}} \pflabel{[epair]prl-typ}
        \item $\hexptyp{\cdot}{\Delta}{\hprr{e}}{\tau_2}$ \BY{Rule (\ref{rule:TPrr}) on \pfref{e-typ}} \pflabel{[epair]prr-typ}
        \end{pfsteps*}
        Assume $\hnotmatch{e}{\hpair{p_1}{p_2}}$. By rule induction on it, no case applies due to syntactic contradiction.
        \begin{pfsteps*}
        \item $\cancel{\hnotmatch{e}{\hpair{p_1}{p_2}}}$ \BY{contradiction}
        \end{pfsteps*}
        By inductive hypothesis on \pfref{[epair]prl-final} and \pfref{[epair]prl-typ} and \pfref{[pair]p1-typ}, exactly one of $\hpatmatch{\hprl{e}}{p_1}{\theta_1}$, $\hmaymatch{\hprl{e}}{p_1}$, and $\hnotmatch{\hprl{e}}{p_1}$ holds. \\
        By inductive hypothesis on \pfref{[epair]prr-final} and \pfref{[epair]prr-typ} and \pfref{[pair]p2-typ}, exactly one of $\hpatmatch{\hprr{e}}{p_2}{\theta_2}$, $\hmaymatch{\hprr{e}}{p_2}$, and $\hnotmatch{\hprr{e}}{p_2}$ holds. \\
        By case analysis on which conclusion holds for $p_1$ and $p_2$. Note that we have already shown $\cancel{\hnotmatch{e}{\hpair{p_1}{p_2}}}$.
        \begin{byCases}
        \savelocalsteps{2}
        \item[\hpatmatch{\hprl{e}}{p_1}{\theta_1},\hpatmatch{\hprr{e}}{p_2}{\theta_2}]
            \begin{pfsteps*}
            \item $\hpatmatch{\hprl{e}}{p_1}{\theta_1}$ \BY{assumption} \pflabel{[epair1]match1}
            \item $\cancel{\hmaymatch{\hprl{e}}{p_1}}$ \BY{assumption} \pflabel{[epair1]not-maymatch1}
            \item $\cancel{\hnotmatch{\hprl{e}}{p_1}}$ \BY{assumption} \pflabel{[epair1]not-notmatch1}
            \item $\hpatmatch{\hprr{e}}{p_2}{\theta_2}$ \BY{assumption} \pflabel{[epair1]match2}
            \item $\cancel{\hmaymatch{\hprr{e}}{p_2}}$ \BY{assumption} \pflabel{[epair1]not-maymatch2}
            \item $\cancel{\hnotmatch{\hprr{e}}{p_2}}$ \BY{assumption} \pflabel{[epair1]not-notmatch2}
            \item $\hpatmatch{e}{\hpair{p_1}{p_2}}{\theta_1 \uplus \theta_2}$ \BY{\ruleref{rule:MNotIntroPair} on \pfref{[epair]notintro} and \pfref{[epair1]match1} and \pfref{[epair1]match2}}
            \end{pfsteps*}
            Assume $\hmaymatch{e}{\hpair{p_1}{p_2}}$. By rule induction over \rulesref{rules:maymatch} on it, only one case applies.
            \begin{byCases}
            \item[\text{(\ref{rule:MMNotIntro})}]
                \begin{pfsteps*}
                \item $\refutable{\hpair{p_1}{p_2}}$ \BY{assumption} \pflabel{[epair1]rft-pair}
                \end{pfsteps*}
                By rule induction over \rulesref{rules:p-refutable}, only two cases apply.
                \begin{byCases}
                \savelocalsteps{3}
                \item[\text{(\ref{rule:RPairL})}]
                    \begin{pfsteps*}
                    \item $\refutable{p_1}$ \BY{assumption} \pflabel{[epair1]rft-p1}
                    \item $\notIntro{\hprl{e}}$ \BY{\ruleref{rule:NVPrl}} \pflabel{[epair1]notintro-prl}
                    \item $\hmaymatch{\hprl{e}}{p_1}$ \BY{\ruleref{rule:MMNotIntro} on \pfref{[epair1]rft-p1} and \pfref{[epair1]notintro-prl}}
                    \end{pfsteps*}
                    Contradicts \pfref{[epair1]not-maymatch1}.
                \restorelocalsteps{3}
                \item[\text{(\ref{rule:RPairR})}]
                    \begin{pfsteps*}
                    \item $\refutable{p_2}$ \BY{assumption} \pflabel{[epair1]rft-p2}
                    \item $\notIntro{\hprr{e}}$ \BY{\ruleref{rule:NVPrr}} \pflabel{[epair1]notintro-prr}
                    \item $\hmaymatch{\hprl{e}}{p_1}$ \BY{\ruleref{rule:MMNotIntro} on \pfref{[epair1]rft-p2} and \pfref{[epair1]notintro-prr}}
                    \end{pfsteps*}
                    Contradicts \pfref{[epair1]not-maymatch1}.
                \end{byCases}
            \end{byCases}
            \begin{pfsteps*}
            \item $\cancel{\hmaymatch{e}{\hpair{p_1}{p_2}}}$ \BY{contradiction}
            \end{pfsteps*}
        \restorelocalsteps{2}
        \item[\hpatmatch{\hprl{e}}{p_1}{\theta_1},\hmaymatch{\hprr{e}}{p_2}]
            \begin{pfsteps*}
            \item $\hpatmatch{\hprl{e}}{p_1}{\theta_1}$ \BY{assumption} \pflabel{[epair2]match1}
            \item $\cancel{\hmaymatch{\hprl{e}}{p_1}}$ \BY{assumption} \pflabel{[epair2]not-maymatch1}
            \item $\cancel{\hnotmatch{\hprl{e}}{p_1}}$ \BY{assumption} \pflabel{[epair2]not-notmatch1}
            \item $\cancel{\hpatmatch{\hprr{e}}{p_2}{\theta_2}}$ \BY{assumption} \pflabel{[epair2]not-match2}
            \item $\hmaymatch{\hprr{e}}{p_2}$ \BY{assumption} \pflabel{[epair2]maymatch2}
            \item $\cancel{\hnotmatch{\hprr{e}}{p_2}}$ \BY{assumption} \pflabel{[epair2]not-notmatch2}
            \end{pfsteps*}
            Assume $\hpatmatch{e}{\hpair{p_1}{p_2}}{\theta}$. By rule induction over \rulesref{rules:match}, only one case applies.
            \begin{byCases}
            \item[\text{(\ref{rule:MNotIntroPair})}]
                \begin{pfsteps*}
                \item $\theta=\theta_1\uplus\theta_2$ \BY{assumption}
                \item $\hpatmatch{\hprr{e}}{p_2}{\theta_2}$ \BY{assumption} \pflabel{[epair2]match2}
                \end{pfsteps*}
                Contradicts \pfref{[epair2]not-match2}.
            \end{byCases}
            \begin{pfsteps*}
            \item $\cancel{\hpatmatch{e}{\hpair{p_1}{p_2}}{\theta}}$ \BY{contradiction}
            \end{pfsteps*}
            By rule induction over \rulesref{rules:maymatch} on \pfref{[epair2]maymatch2}, the following cases apply.
            \begin{byCases}
            \savelocalsteps{3}
            \item[\text{(\ref{rule:MMEHole}),(\ref{rule:MMHole})}]
                \begin{pfsteps*}
                \item $p_2=\heholep{w},\hholep{p}{w}{\tau'}$ \BY{assumption}
                \item $\refutable{p_2}$ \BY{\ruleref{rule:REHole} and \ruleref{rule:RHole}} \pflabel{[epair2]rft-p2}
                \item $\refutable{\hpair{p_1}{p_2}}$ \BY{\ruleref{rule:RPairR} on \pfref{[epair2]rft-p2}} \pflabel{[epair2]rft-pair}
                \item $\hmaymatch{e}{\hpair{p_1}{p_2}}$ \BY{\ruleref{rule:MMNotIntro} on \pfref{[epair]notintro} and \pfref{[epair2]rft-pair}}
                \end{pfsteps*}
            \restorelocalsteps{3}
            \item[\text{(\ref{rule:MMNotIntro})}]
                \begin{pfsteps*}
                \item $\refutable{p_2}$ \BY{assumption} \pflabel{[epair2]rft-p2'}
                \item $\refutable{\hpair{p_1}{p_2}}$ \BY{\ruleref{rule:RPairR} on \pfref{[epair2]rft-p2'}} \pflabel{[epair2]rft-pair'}
                \item $\hmaymatch{e}{\hpair{p_1}{p_2}}$ \BY{\ruleref{rule:MMNotIntro} on \pfref{[epair]notintro} and \pfref{[epair2]rft-pair'}}
                \end{pfsteps*}
            \end{byCases}
        \restorelocalsteps{2}
        \item[\hpatmatch{\hprl{e}}{p_1}{\theta_1},\hnotmatch{\hprr{e}}{p_2}]
            \begin{pfsteps*}
            \item $\hpatmatch{\hprl{e}}{p_1}{\theta_1}$ \BY{assumption} \pflabel{[epair3]match1}
            \item $\cancel{\hmaymatch{\hprl{e}}{p_1}}$ \BY{assumption} \pflabel{[epair3]not-maymatch1}
            \item $\cancel{\hnotmatch{\hprl{e}}{p_1}}$ \BY{assumption} \pflabel{[epair3]not-notmatch1}
            \item $\cancel{\hpatmatch{\hprr{e}}{p_2}{\theta_2}}$ \BY{assumption} \pflabel{[epair3]not-match2}
            \item $\cancel{\hmaymatch{\hprr{e}}{p_2}}$ \BY{assumption} \pflabel{[epair3]not-maymatch2}
            \item $\hnotmatch{\hprr{e}}{p_2}$ \BY{assumption} \pflabel{[epair3]notmatch2}
            \end{pfsteps*}
            By rule induction over \rulesref{rules:notmatch} on \pfref{[epair3]notmatch2}, no case applies due to syntactic contradiction.\\
            Therefore, vacuously true.
        \restorelocalsteps{2}
        \item[\hmaymatch{\hprl{e}}{p_1},\hpatmatch{\hprr{e}}{p_2}{\theta_2}]
            \begin{pfsteps*}
            \item $\cancel{\hpatmatch{\hprl{e}}{p_1}{\theta_1}}$ \BY{assumption} \pflabel{[epair4]not-match1}
            \item $\hmaymatch{\hprl{e}}{p_1}$ \BY{assumption} \pflabel{[epair4]maymatch1}
            \item $\cancel{\hnotmatch{\hprl{e}}{p_1}}$ \BY{assumption} \pflabel{[epair4]not-notmatch1}
            \item $\hpatmatch{\hprr{e}}{p_2}{\theta_2}$ \BY{assumption} \pflabel{[epair4]match2}
            \item $\cancel{\hmaymatch{\hprr{e}}{p_2}}$ \BY{assumption} \pflabel{[epair4]not-maymatch2}
            \item $\cancel{\hnotmatch{\hprr{e}}{p_2}}$ \BY{assumption} \pflabel{[epair4]not-notmatch2}
            \end{pfsteps*}
            Assume $\hpatmatch{e}{\hpair{p_1}{p_2}}{\theta}$. By rule induction over \rulesref{rules:match}, only one case applies.
            \begin{byCases}
            \item[\text{(\ref{rule:MNotIntroPair})}]
                \begin{pfsteps*}
                \item $\theta=\theta_1\uplus\theta_2$ \BY{assumption}
                \item $\hpatmatch{\hprl{e}}{p_1}{\theta_1}$ \BY{assumption} \pflabel{[epair4]match1}
                \end{pfsteps*}
                Contradicts \pfref{[epair4]not-match1}.
            \end{byCases}
            \begin{pfsteps*}
            \item $\cancel{\hpatmatch{e}{\hpair{p_1}{p_2}}{\theta}}$ \BY{contradiction}
            \end{pfsteps*}
            By rule induction over \rulesref{rules:maymatch} on \pfref{[epair4]maymatch1}, the following cases apply.
            \begin{byCases}
            \savelocalsteps{3}
            \item[\text{(\ref{rule:MMEHole}),(\ref{rule:MMHole})}]
                \begin{pfsteps*}
                \item $p_1=\heholep{w},\hholep{p}{w}{\tau'}$ \BY{assumption}
                \item $\refutable{p_1}$ \BY{\ruleref{rule:REHole} and \ruleref{rule:RHole}} \pflabel{[epair4]rft-p1}
                \item $\refutable{\hpair{p_1}{p_2}}$ \BY{\ruleref{rule:RPairR} on \pfref{[epair4]rft-p1}} \pflabel{[epair4]rft-pair}
                \item $\hmaymatch{e}{\hpair{p_1}{p_2}}$ \BY{\ruleref{rule:MMNotIntro} on \pfref{[epair]notintro} and \pfref{[epair4]rft-pair}}
                \end{pfsteps*}
            \restorelocalsteps{3}
            \item[\text{(\ref{rule:MMNotIntro})}]
                \begin{pfsteps*}
                \item $\refutable{p_1}$ \BY{assumption} \pflabel{[epair4]rft-p1'}
                \item $\refutable{\hpair{p_1}{p_2}}$ \BY{\ruleref{rule:RPairR} on \pfref{[epair4]rft-p1'}} \pflabel{[epair4]rft-pair'}
                \item $\hmaymatch{e}{\hpair{p_1}{p_2}}$ \BY{\ruleref{rule:MMNotIntro} on \pfref{[epair]notintro} and \pfref{[epair4]rft-pair'}}
                \end{pfsteps*}
            \end{byCases}
        \restorelocalsteps{2}
        \item[\hmaymatch{\hprl{e}}{p_1},\hmaymatch{\hprr{e}}{p_2}]
            \begin{pfsteps*}
            \item $\cancel{\hpatmatch{\hprl{e}}{p_1}{\theta_1}}$ \BY{assumption} \pflabel{[epair5]not-match1}
            \item $\hmaymatch{\hprl{e}}{p_1}$ \BY{assumption} \pflabel{[epair5]maymatch1}
            \item $\cancel{\hnotmatch{\hprl{e}}{p_1}}$ \BY{assumption} \pflabel{[epair5]not-notmatch1}
            \item $\cancel{\hpatmatch{\hprr{e}}{p_2}{\theta_2}}$ \BY{assumption} \pflabel{[epair5]not-match2}
            \item $\hmaymatch{\hprr{e}}{p_2}$ \BY{assumption} \pflabel{[epair5]maymatch2}
            \item $\cancel{\hnotmatch{\hprr{e}}{p_2}}$ \BY{assumption} \pflabel{[epair5]not-notmatch2}
            \end{pfsteps*}
            Assume $\hpatmatch{e}{\hpair{p_1}{p_2}}{\theta}$. By rule induction over \rulesref{rules:match}, only one case applies.
            \begin{byCases}
            \item[\text{(\ref{rule:MNotIntroPair})}]
                \begin{pfsteps*}
                \item $\theta=\theta_1\uplus\theta_2$ \BY{assumption}
                \item $\hpatmatch{\hprl{e}}{p_1}{\theta_1}$ \BY{assumption} \pflabel{[epair5]match1}
                \end{pfsteps*}
                Contradicts \pfref{[epair5]not-match1}.
            \end{byCases}
            \begin{pfsteps*}
            \item $\cancel{\hpatmatch{e}{\hpair{p_1}{p_2}}{\theta}}$ \BY{contradiction}
            \end{pfsteps*}
            By rule induction over \rulesref{rules:maymatch} on \pfref{[epair5]maymatch1}, the following cases apply.
            \begin{byCases}
            \savelocalsteps{3}
            \item[\text{(\ref{rule:MMEHole}),(\ref{rule:MMHole})}]
                \begin{pfsteps*}
                \item $p_1=\heholep{w},\hholep{p}{w}{\tau'}$ \BY{assumption}
                \item $\refutable{p_1}$ \BY{\ruleref{rule:REHole} and \ruleref{rule:RHole}} \pflabel{[epair5]rft-p1}
                \item $\refutable{\hpair{p_1}{p_2}}$ \BY{\ruleref{rule:RPairR} on \pfref{[epair5]rft-p1}} \pflabel{[epair5]rft-pair}
                \item $\hmaymatch{e}{\hpair{p_1}{p_2}}$ \BY{\ruleref{rule:MMNotIntro} on \pfref{[epair]notintro} and \pfref{[epair5]rft-pair}}
                \end{pfsteps*}
            \restorelocalsteps{3}
            \item[\text{(\ref{rule:MMNotIntro})}]
                \begin{pfsteps*}
                \item $\refutable{p_1}$ \BY{assumption} \pflabel{[epair5]rft-p1'}
                \item $\refutable{\hpair{p_1}{p_2}}$ \BY{\ruleref{rule:RPairR} on \pfref{[epair5]rft-p1'}} \pflabel{[epair5]rft-pair'}
                \item $\hmaymatch{e}{\hpair{p_1}{p_2}}$ \BY{\ruleref{rule:MMNotIntro} on \pfref{[epair]notintro} and \pfref{[epair5]rft-pair'}}
                \end{pfsteps*}
            \end{byCases}
        \restorelocalsteps{2}
        \item[\hmaymatch{\hprl{e}}{p_1},\hnotmatch{\hprr{e}}{p_2}]
            \begin{pfsteps*}
            \item $\cancel{\hpatmatch{\hprl{e}}{p_1}{\theta_1}}$ \BY{assumption} \pflabel{[epair6]not-match1}
            \item $\hmaymatch{\hprl{e}}{p_1}$ \BY{assumption} \pflabel{[epair6]maymatch1}
            \item $\cancel{\hnotmatch{\hprl{e}}{p_1}}$ \BY{assumption} \pflabel{[epair6]not-notmatch1}
            \item $\cancel{\hpatmatch{\hprr{e}}{p_2}{\theta_2}}$ \BY{assumption} \pflabel{[epair6]not-match2}
            \item $\cancel{\hmaymatch{\hprr{e}}{p_2}}$ \BY{assumption} \pflabel{[epair6]not-maymatch2}
            \item $\hnotmatch{\hprr{e}}{p_2}$ \BY{assumption} \pflabel{[epair6]notmatch2}
            \end{pfsteps*}
            By rule induction over \rulesref{rules:notmatch} on \pfref{[epair6]notmatch2}, no case applies due to syntactic contradiction.\\
            Therefore, vacuously true.
        \restorelocalsteps{2}
        \item[\hnotmatch{\hprl{e}}{p_1},\hpatmatch{\hprr{e}}{p_2}{\theta_2}]
            \begin{pfsteps*}
            \item $\cancel{\hpatmatch{\hprl{e}}{p_1}{\theta_1}}$ \BY{assumption} \pflabel{[epair7]not-match1}
            \item $\cancel{\hmaymatch{\hprl{e}}{p_1}}$ \BY{assumption} \pflabel{[epair7]not-maymatch1}
            \item $\hnotmatch{\hprl{e}}{p_1}$ \BY{assumption} \pflabel{[epair7]notmatch1}
            \item $\hpatmatch{\hprr{e}}{p_2}{\theta_2}$ \BY{assumption} \pflabel{[epair7]match2}
            \item $\cancel{\hmaymatch{\hprr{e}}{p_2}}$ \BY{assumption} \pflabel{[epair7]not-maymatch2}
            \item $\cancel{\hnotmatch{\hprr{e}}{p_2}}$ \BY{assumption} \pflabel{[epair7]not-notmatch2}
            \end{pfsteps*}
            By rule induction over \rulesref{rules:notmatch} on \pfref{[epair7]notmatch1}, no case applies due to syntactic contradiction.\\
            Therefore, vacuously true.
        \restorelocalsteps{2}
        \item[\hnotmatch{\hprl{e}}{p_1},\hmaymatch{\hprr{e}}{p_2}]
            \begin{pfsteps*}
            \item $\cancel{\hpatmatch{\hprl{e}}{p_1}{\theta_1}}$ \BY{assumption} \pflabel{[epair8]not-match1}
            \item $\cancel{\hmaymatch{\hprl{e}}{p_1}}$ \BY{assumption} \pflabel{[epair8]not-maymatch1}
            \item $\hnotmatch{\hprl{e}}{p_1}$ \BY{assumption} \pflabel{[epair8]notmatch1}
            \item $\cancel{\hpatmatch{\hprr{e}}{p_2}{\theta_2}}$ \BY{assumption} \pflabel{[epair8]not-match2}
            \item $\hmaymatch{\hprr{e}}{p_2}$ \BY{assumption} \pflabel{[epair8]maymatch2}
            \item $\cancel{\hnotmatch{\hprr{e}}{p_2}}$ \BY{assumption} \pflabel{[epair8]not-notmatch2}
            \end{pfsteps*}
            By rule induction over \rulesref{rules:notmatch} on \pfref{[epair8]notmatch1}, no case applies due to syntactic contradiction.\\
            Therefore, vacuously true.
        \restorelocalsteps{2}
        \item[\hnotmatch{\hprl{e}}{p_1},\hnotmatch{\hprr{e}}{p_2}]
            \begin{pfsteps*}
            \item $\cancel{\hpatmatch{\hprl{e}}{p_1}{\theta_1}}$ \BY{assumption} \pflabel{[epair9]not-match1}
            \item $\cancel{\hmaymatch{\hprl{e}}{p_1}}$ \BY{assumption} \pflabel{[epair9]not-maymatch1}
            \item $\hnotmatch{\hprl{e}}{p_1}$ \BY{assumption} \pflabel{[epair9]notmatch1}
            \item $\cancel{\hpatmatch{\hprr{e}}{p_2}{\theta_2}}$ \BY{assumption} \pflabel{[epair9]not-match2}
            \item $\hmaymatch{\hprr{e}}{p_2}$ \BY{assumption} \pflabel{[epair9]maymatch2}
            \item $\cancel{\hnotmatch{\hprr{e}}{p_2}}$ \BY{assumption} \pflabel{[epair9]not-notmatch2}
            \end{pfsteps*}
            By rule induction over \rulesref{rules:notmatch} on \pfref{[epair9]notmatch1}, no case applies due to syntactic contradiction.\\
            Therefore, vacuously true.
        \end{byCases}
    \restorelocalsteps{1}
    \item[\text{(\ref{rule:TPair})}]
        \begin{pfsteps*}
        \item $e=\hpair{e_1}{e_2}$ \BY{assumption}
        \item $\hexptyp{\cdot}{\Delta}{e_1}{\tau_1}$ \BY{assumption} \pflabel{[pair]e1-typ}
        \item $\hexptyp{\cdot}{\Delta}{e_2}{\tau_2}$ \BY{assumption} \pflabel{[pair]e2-typ}
        \item $\isFinal{e_1}$ \BY{\autoref{lem:pair-final} on \pfref{e-final}} \pflabel{[pair]e1-final}
        \item $\isFinal{e_2}$ \BY{\autoref{lem:pair-final} on \pfref{e-final}} \pflabel{[pair]e2-final}
        \end{pfsteps*}
        By inductive hypothesis on \pfref{[pair]p1-typ} and \pfref{[pair]e1-typ} and \pfref{[pair]e1-final}, exactly one of $\hpatmatch{e_1}{p_1}{\theta_1}$, $\hmaymatch{e_1}{p_1}$, and $\hnotmatch{e_1}{p_1}$ holds. \\
        By inductive hypothesis on \pfref{[pair]p2-typ} and \pfref{[pair]e2-typ} and \pfref{[pair]e2-final}, exactly one of $\hpatmatch{e_2}{p_2}{\theta_2}$, $\hmaymatch{e_2}{p_2}$, and $\hnotmatch{e_2}{p_2}$ holds. \\
        By case analysis on which conclusion holds for $p_1$ and $p_2$.
        \begin{byCases}
        \savelocalsteps{2}
        \item[\hpatmatch{e_1}{p_1}{\theta_1},\hpatmatch{e_2}{p_2}{\theta_2}]
            \begin{pfsteps*}
            \item $\hpatmatch{e_1}{p_1}{\theta_1}$ \BY{assumption} \pflabel{[pair1]match1}
            \item $\cancel{\hmaymatch{e_1}{p_1}}$ \BY{assumption} \pflabel{[pair1]not-maymatch1}
            \item $\cancel{\hnotmatch{e_1}{p_1}}$ \BY{assumption} \pflabel{[pair1]not-notmatch1}
            \item $\hpatmatch{e_2}{p_2}{\theta_2}$ \BY{assumption} \pflabel{[pair1]match2}
            \item $\cancel{\hmaymatch{e_2}{p_2}}$ \BY{assumption} \pflabel{[pair1]not-maymatch2}
            \item $\cancel{\hnotmatch{e_2}{p_2}}$ \BY{assumption} \pflabel{[pair1]not-notmatch2}
            \item $\hpatmatch{\hpair{e_1}{e_2}}{\hpair{p_1}{p_2}}{\theta_1 \uplus \theta_2}$ \BY{\ruleref{rule:MPair} on \pfref{[pair1]match1} and \pfref{[pair1]match2}}
            \end{pfsteps*}
            Assume $\hmaymatch{\hpair{e_1}{e_2}}{\hpair{p_1}{p_2}}$. By rule induction over \rulesref{rules:maymatch} on it, only four cases apply.
            \begin{byCases}
            \savelocalsteps{3}
            \item[\text{(\ref{rule:MMNotIntro})}]
                \begin{pfsteps*}
                \item $\notIntro{\hpair{e_1}{e_2}}$ \BY{assumption}
                \end{pfsteps*}
                Contradicts \autoref{lem:no-pair-notintro}.
            \restorelocalsteps{3}
            \item[\text{(\ref{rule:MMPairL})}]
                \begin{pfsteps*}
                \item $\hmaymatch{e_1}{p_1}$ \BY{assumption}
                \end{pfsteps*}
                Contradicts \pfref{[pair1]not-maymatch1}.
            \restorelocalsteps{3}
            \item[\text{(\ref{rule:MMPairR})}]
                \begin{pfsteps*}
                \item $\hmaymatch{e_2}{p_2}$ \BY{assumption}
                \end{pfsteps*}
                Contradicts \pfref{[pair1]not-maymatch2}.
            \restorelocalsteps{3}
            \item[\text{(\ref{rule:MMPair})}]
                \begin{pfsteps*}
                \item $\hmaymatch{e_1}{p_1}$ \BY{assumption}
                \end{pfsteps*}
                Contradicts \pfref{[pair1]not-maymatch1}.
            \end{byCases}
            \begin{pfsteps*}
            \item $\cancel{\hmaymatch{\hpair{e_1}{e_2}}{\hpair{p_1}{p_2}}}$ \BY{contradiction}
            \end{pfsteps*}
            Assume $\hnotmatch{\hpair{e_1}{e_2}}{\hpair{p_1}{p_2}}$. By rule induction over \rulesref{rules:notmatch} on it, only two cases apply.
            \begin{byCases}
            \savelocalsteps{3}
            \item[\text{(\ref{rule:NMPairL})}]
                \begin{pfsteps*}
                \item $\hnotmatch{e_1}{p_1}$ \BY{assumption}
                \end{pfsteps*}
                Contradicts \pfref{[pair1]not-notmatch1}.
            \restorelocalsteps{3}
            \item[\text{(\ref{rule:NMPairR})}]
                \begin{pfsteps*}
                \item $\hnotmatch{e_2}{p_2}$ \BY{assumption}
                \end{pfsteps*}
                Contradicts \pfref{[pair1]not-notmatch2}.
            \end{byCases}
            \begin{pfsteps*}
            \item $\cancel{\hnotmatch{\hpair{e_1}{e_2}}{\hpair{p_1}{p_2}}}$ \BY{contradiction}
            \end{pfsteps*}
        \restorelocalsteps{2}
        \item[\hpatmatch{e_1}{p_1}{\theta_1},\hmaymatch{e_2}{p_2}]
            \begin{pfsteps*}
            \item $\hpatmatch{e_1}{p_1}{\theta_1}$ \BY{assumption} \pflabel{[pair2]match1}
            \item $\cancel{\hmaymatch{e_1}{p_1}}$ \BY{assumption} \pflabel{[pair2]not-maymatch1}
            \item $\cancel{\hnotmatch{e_1}{p_1}}$ \BY{assumption} \pflabel{[pair2]not-notmatch1}
            \item $\cancel{\hpatmatch{e_2}{p_2}{\theta_2}}$ \BY{assumption} \pflabel{[pair2]not-match2}
            \item $\hmaymatch{e_2}{p_2}$ \BY{assumption} \pflabel{[pair2]maymatch2}
            \item $\cancel{\hnotmatch{e_2}{p_2}}$ \BY{assumption} \pflabel{[pair2]not-notmatch2}
            \item $\hmaymatch{\hpair{e_1}{e_2}}{\hpair{p_1}{p_2}}$ \BY{\ruleref{rule:MMPairR} on \pfref{[pair2]match1} and \pfref{[pair2]maymatch2}}
            \end{pfsteps*}
            Assume $\hpatmatch{\hpair{e_1}{e_2}}{\hpair{p_1}{p_2}}{\theta}$. By rule induction over \rulesref{rules:match} on it, only one case applies.
            \begin{byCases}
            \item[\text{(\ref{rule:MPair})}]
                \begin{pfsteps*}
                \item $\theta=\theta_1\uplus\theta_2$
                \item $\hpatmatch{e_2}{p_2}{\theta_2}$ \BY{assumption}
                \end{pfsteps*}
                Contradicts \pfref{[pair2]not-match2}.
            \end{byCases}
            \begin{pfsteps*}
            \item $\cancel{\hpatmatch{\hpair{e_1}{e_2}}{\hpair{p_1}{p_2}}{\theta}}$ \BY{contradiction}
            \end{pfsteps*}
            Assume $\hnotmatch{\hpair{e_1}{e_2}}{\hpair{p_1}{p_2}}$. By rule induction over \rulesref{rules:notmatch} on it, only two cases apply.
            \begin{byCases}
            \savelocalsteps{3}
            \item[\text{(\ref{rule:NMPairL})}]
                \begin{pfsteps*}
                \item $\hnotmatch{e_1}{p_1}$ \BY{assumption}
                \end{pfsteps*}
                Contradicts \pfref{[pair2]not-notmatch1}.
            \restorelocalsteps{3}
            \item[\text{(\ref{rule:NMPairR})}]
                \begin{pfsteps*}
                \item $\hnotmatch{e_2}{p_2}$ \BY{assumption}
                \end{pfsteps*}
                Contradicts \pfref{[pair2]not-notmatch2}.
            \end{byCases}
            \begin{pfsteps*}
            \item $\cancel{\hnotmatch{\hpair{e_1}{e_2}}{\hpair{p_1}{p_2}}}$ \BY{contradiction}
            \end{pfsteps*}
        \restorelocalsteps{2}
        \item[\hpatmatch{e_1}{p_1}{\theta_1},\hnotmatch{e_2}{p_2}]
            \begin{pfsteps*}
            \item $\hpatmatch{e_1}{p_1}{\theta_1}$ \BY{assumption} \pflabel{[pair3]match1}
            \item $\cancel{\hmaymatch{e_1}{p_1}}$ \BY{assumption} \pflabel{[pair3]not-maymatch1}
            \item $\cancel{\hnotmatch{e_1}{p_1}}$ \BY{assumption} \pflabel{[pair3]not-notmatch1}
            \item $\cancel{\hpatmatch{e_2}{p_2}{\theta_2}}$ \BY{assumption} \pflabel{[pair3]not-match2}
            \item $\cancel{\hmaymatch{e_2}{p_2}}$ \BY{assumption} \pflabel{[pair3]not-maymatch2}
            \item $\hnotmatch{e_2}{p_2}$ \BY{assumption} \pflabel{[pair3]notmatch2}
            \item $\hnotmatch{\hpair{e_1}{e_2}}{\hpair{p_1}{p_2}}$ \BY{\ruleref{rule:NMPairR} on \pfref{[pair3]notmatch2}}
            \end{pfsteps*}
            Assume $\hpatmatch{\hpair{e_1}{e_2}}{\hpair{p_1}{p_2}}{\theta}$. By rule induction over \rulesref{rules:match} on it, only one case applies.
            \begin{byCases}
            \item[\text{(\ref{rule:MPair})}]
                \begin{pfsteps*}
                \item $\theta=\theta_1\uplus\theta_2$
                \item $\hpatmatch{e_2}{p_2}{\theta_2}$ \BY{assumption}
                \end{pfsteps*}
                Contradicts \pfref{[pair3]not-match2}.
            \end{byCases}
            \begin{pfsteps*}
            \item $\cancel{\hpatmatch{\hpair{e_1}{e_2}}{\hpair{p_1}{p_2}}{\theta}}$ \BY{contradiction}
            \end{pfsteps*}
            Assume $\hmaymatch{\hpair{e_1}{e_2}}{\hpair{p_1}{p_2}}$. By rule induction over \rulesref{rules:maymatch} on it, only four cases apply.
            \begin{byCases}
            \savelocalsteps{3}
            \item[\text{(\ref{rule:MMNotIntro})}]
                \begin{pfsteps*}
                \item $\notIntro{\hpair{e_1}{e_2}}$ \BY{assumption}
                \end{pfsteps*}
                Contradicts \autoref{lem:no-pair-notintro}.
            \restorelocalsteps{3}
            \item[\text{(\ref{rule:MMPairL})}]
                \begin{pfsteps*}
                \item $\hmaymatch{e_1}{p_1}$ \BY{assumption}
                \end{pfsteps*}
                Contradicts \pfref{[pair3]not-maymatch1}.
            \restorelocalsteps{3}
            \item[\text{(\ref{rule:MMPairR})}]
                \begin{pfsteps*}
                \item $\hmaymatch{e_2}{p_2}$ \BY{assumption}
                \end{pfsteps*}
                Contradicts \pfref{[pair3]not-maymatch2}.
            \restorelocalsteps{3}
            \item[\text{(\ref{rule:MMPair})}]
                \begin{pfsteps*}
                \item $\hmaymatch{e_1}{p_1}$ \BY{assumption}
                \end{pfsteps*}
                Contradicts \pfref{[pair3]not-maymatch1}.
            \end{byCases}
            \begin{pfsteps*}
            \item $\cancel{\hmaymatch{\hpair{e_1}{e_2}}{\hpair{p_1}{p_2}}}$ \BY{contradiction}
            \end{pfsteps*}
        \restorelocalsteps{2}
        \item[\hmaymatch{e_1}{p_1},\hpatmatch{e_2}{p_2}{\theta_2}]
            \begin{pfsteps*}
            \item $\cancel{\hpatmatch{e_1}{p_1}{\theta_1}}$ \BY{assumption} \pflabel{[pair4]not-match1}
            \item $\hmaymatch{e_1}{p_1}$ \BY{assumption} \pflabel{[pair4]maymatch1}
            \item $\cancel{\hnotmatch{e_1}{p_1}}$ \BY{assumption} \pflabel{[pair4]not-notmatch1}
            \item $\hpatmatch{e_2}{p_2}{\theta_2}$ \BY{assumption} \pflabel{[pair4]match2}
            \item $\cancel{\hmaymatch{e_2}{p_2}}$ \BY{assumption} \pflabel{[pair4]not-maymatch2}
            \item $\cancel{\hnotmatch{e_2}{p_2}}$ \BY{assumption} \pflabel{[pair4]not-notmatch2}
            \item $\hmaymatch{\hpair{e_1}{e_2}}{\hpair{p_1}{p_2}}$ \BY{\ruleref{rule:MMPairL} on \pfref{[pair4]maymatch1} and \pfref{[pair4]match2}}
            \end{pfsteps*}
            Assume $\hpatmatch{\hpair{e_1}{e_2}}{\hpair{p_1}{p_2}}{\theta}$. By rule induction over \rulesref{rules:match} on it, only one case applies.
            \begin{byCases}
            \item[\text{(\ref{rule:MPair})}]
                \begin{pfsteps*}
                \item $\theta=\theta_1\uplus\theta_2$
                \item $\hpatmatch{e_1}{p_1}{\theta_1}$ \BY{assumption}
                \end{pfsteps*}
                Contradicts \pfref{[pair4]not-match1}.
            \end{byCases}
            \begin{pfsteps*}
            \item $\cancel{\hpatmatch{\hpair{e_1}{e_2}}{\hpair{p_1}{p_2}}{\theta}}$ \BY{contradiction}
            \end{pfsteps*}
            Assume $\hnotmatch{\hpair{e_1}{e_2}}{\hpair{p_1}{p_2}}$. By rule induction over \rulesref{rules:notmatch} on it, only two cases apply.
            \begin{byCases}
            \savelocalsteps{3}
            \item[\text{(\ref{rule:NMPairL})}]
                \begin{pfsteps*}
                \item $\hnotmatch{e_1}{p_1}$ \BY{assumption}
                \end{pfsteps*}
                Contradicts \pfref{[pair4]not-notmatch1}.
            \restorelocalsteps{3}
            \item[\text{(\ref{rule:NMPairR})}]
                \begin{pfsteps*}
                \item $\hnotmatch{e_2}{p_2}$ \BY{assumption}
                \end{pfsteps*}
                Contradicts \pfref{[pair4]not-notmatch2}.
            \end{byCases}
            \begin{pfsteps*}
            \item $\cancel{\hnotmatch{\hpair{e_1}{e_2}}{\hpair{p_1}{p_2}}}$ \BY{contradiction}
            \end{pfsteps*}
        \restorelocalsteps{2}
        \item[\hmaymatch{e_1}{p_1},\hmaymatch{e_2}{p_2}]
            \begin{pfsteps*}
            \item $\cancel{\hpatmatch{e_1}{p_1}{\theta_1}}$ \BY{assumption} \pflabel{[pair5]not-match1}
            \item $\hmaymatch{e_1}{p_1}$ \BY{assumption} \pflabel{[pair5]maymatch1}
            \item $\cancel{\hnotmatch{e_1}{p_1}}$ \BY{assumption} \pflabel{[pair5]not-notmatch1}
            \item $\cancel{\hpatmatch{e_2}{p_2}{\theta_2}}$ \BY{assumption} \pflabel{[pair5]not-match2}
            \item $\hmaymatch{e_2}{p_2}$ \BY{assumption} \pflabel{[pair5]maymatch2}
            \item $\cancel{\hnotmatch{e_2}{p_2}}$ \BY{assumption} \pflabel{[pair5]not-notmatch2}
            \item $\hmaymatch{\hpair{e_1}{e_2}}{\hpair{p_1}{p_2}}$ \BY{\ruleref{rule:MMPair} on \pfref{[pair5]maymatch1} and \pfref{[pair5]maymatch2}}
            \end{pfsteps*}
            Assume $\hpatmatch{\hpair{e_1}{e_2}}{\hpair{p_1}{p_2}}{\theta}$. By rule induction over \rulesref{rules:match} on it, only one case applies.
            \begin{byCases}
            \item[\text{(\ref{rule:MPair})}]
                \begin{pfsteps*}
                \item $\theta=\theta_1\uplus\theta_2$
                \item $\hpatmatch{e_2}{p_2}{\theta_2}$ \BY{assumption}
                \end{pfsteps*}
                Contradicts \pfref{[pair5]not-match2}.
            \end{byCases}
            \begin{pfsteps*}
            \item $\cancel{\hpatmatch{\hpair{e_1}{e_2}}{\hpair{p_1}{p_2}}{\theta}}$ \BY{contradiction}
            \end{pfsteps*}
            Assume $\hnotmatch{\hpair{e_1}{e_2}}{\hpair{p_1}{p_2}}$. By rule induction over \rulesref{rules:notmatch} on it, only two cases apply.
            \begin{byCases}
            \savelocalsteps{3}
            \item[\text{(\ref{rule:NMPairL})}]
                \begin{pfsteps*}
                \item $\hnotmatch{e_1}{p_1}$ \BY{assumption}
                \end{pfsteps*}
                Contradicts \pfref{[pair5]not-notmatch1}.
            \restorelocalsteps{3}
            \item[\text{(\ref{rule:NMPairR})}]
                \begin{pfsteps*}
                \item $\hnotmatch{e_2}{p_2}$ \BY{assumption}
                \end{pfsteps*}
                Contradicts \pfref{[pair5]not-notmatch2}.
            \end{byCases}
            \begin{pfsteps*}
            \item $\cancel{\hnotmatch{\hpair{e_1}{e_2}}{\hpair{p_1}{p_2}}}$ \BY{contradiction}
            \end{pfsteps*}
        \restorelocalsteps{2}
        \item[\hmaymatch{e_1}{p_1},\hnotmatch{e_2}{p_2}]
            \begin{pfsteps*}
            \item $\cancel{\hpatmatch{e_1}{p_1}{\theta_1}}$ \BY{assumption} \pflabel{[pair6]not-match1}
            \item $\hmaymatch{e_1}{p_1}$ \BY{assumption} \pflabel{[pair6]maymatch1}
            \item $\cancel{\hnotmatch{e_1}{p_1}}$ \BY{assumption} \pflabel{[pair6]not-notmatch1}
            \item $\cancel{\hpatmatch{e_2}{p_2}{\theta_2}}$ \BY{assumption} \pflabel{[pair6]not-match2}
            \item $\cancel{\hmaymatch{e_2}{p_2}}$ \BY{assumption} \pflabel{[pair6]not-maymatch2}
            \item $\hnotmatch{e_2}{p_2}$ \BY{assumption} \pflabel{[pair6]notmatch2}
            \item $\hnotmatch{\hpair{e_1}{e_2}}{\hpair{p_1}{p_2}}$ \BY{\ruleref{rule:NMPairR} on \pfref{[pair6]notmatch2}}
            \end{pfsteps*}
            Assume $\hpatmatch{\hpair{e_1}{e_2}}{\hpair{p_1}{p_2}}{\theta}$. By rule induction over \rulesref{rules:match} on it, only one case applies.
            \begin{byCases}
            \item[\text{(\ref{rule:MPair})}]
                \begin{pfsteps*}
                \item $\theta=\theta_1\uplus\theta_2$
                \item $\hpatmatch{e_2}{p_2}{\theta_2}$ \BY{assumption}
                \end{pfsteps*}
                Contradicts \pfref{[pair6]not-match2}.
            \end{byCases}
            \begin{pfsteps*}
            \item $\cancel{\hpatmatch{\hpair{e_1}{e_2}}{\hpair{p_1}{p_2}}{\theta}}$ \BY{contradiction}
            \end{pfsteps*}
            Assume $\hmaymatch{\hpair{e_1}{e_2}}{\hpair{p_1}{p_2}}$. By rule induction over \rulesref{rules:maymatch} on it, only four cases apply.
            \begin{byCases}
            \savelocalsteps{3}
            \item[\text{(\ref{rule:MMNotIntro})}]
                \begin{pfsteps*}
                \item $\notIntro{\hpair{e_1}{e_2}}$ \BY{assumption}
                \end{pfsteps*}
                Contradicts \autoref{lem:no-pair-notintro}.
            \restorelocalsteps{3}
            \item[\text{(\ref{rule:MMPairL})}]
                \begin{pfsteps*}
                \item $\hpatmatch{e_2}{p_2}{\theta_2}$ \BY{assumption}
                \end{pfsteps*}
                Contradicts \pfref{[pair6]not-match2}.
            \restorelocalsteps{3}
            \item[\text{(\ref{rule:MMPairR})}]
                \begin{pfsteps*}
                \item $\hmaymatch{e_2}{p_2}$ \BY{assumption}
                \end{pfsteps*}
                Contradicts \pfref{[pair6]not-maymatch2}.
            \restorelocalsteps{3}
            \item[\text{(\ref{rule:MMPair})}]
                \begin{pfsteps*}
                \item $\hmaymatch{e_2}{p_2}$ \BY{assumption}
                \end{pfsteps*}
                Contradicts \pfref{[pair6]not-maymatch2}.
            \end{byCases}
            \begin{pfsteps*}
            \item $\cancel{\hmaymatch{\hpair{e_1}{e_2}}{\hpair{p_1}{p_2}}}$ \BY{contradiction}
            \end{pfsteps*}
        \restorelocalsteps{2}
        \item[\hnotmatch{e_1}{p_1},\hpatmatch{e_2}{p_2}{\theta_2}]
            \begin{pfsteps*}
            \item $\cancel{\hpatmatch{e_1}{p_1}{\theta_1}}$ \BY{assumption} \pflabel{[pair7]not-match1}
            \item $\cancel{\hmaymatch{e_1}{p_1}}$ \BY{assumption} \pflabel{[pair7]not-maymatch1}
            \item $\hnotmatch{e_1}{p_1}$ \BY{assumption} \pflabel{[pair7]notmatch1}
            \item $\hpatmatch{e_2}{p_2}{\theta_2}$ \BY{assumption} \pflabel{[pair7]match2}
            \item $\cancel{\hmaymatch{e_2}{p_2}}$ \BY{assumption} \pflabel{[pair7]not-maymatch2}
            \item $\cancel{\hnotmatch{e_2}{p_2}}$ \BY{assumption} \pflabel{[pair7]not-notmatch2}
            \item $\hnotmatch{\hpair{e_1}{e_2}}{\hpair{p_1}{p_2}}$ \BY{\ruleref{rule:NMPairL} on \pfref{[pair7]notmatch1}}
            \end{pfsteps*}
            Assume $\hpatmatch{\hpair{e_1}{e_2}}{\hpair{p_1}{p_2}}{\theta}$. By rule induction over \rulesref{rules:match} on it, only one case applies.
            \begin{byCases}
            \item[\text{(\ref{rule:MPair})}]
                \begin{pfsteps*}
                \item $\theta=\theta_1\uplus\theta_2$
                \item $\hpatmatch{e_1}{p_1}{\theta_1}$ \BY{assumption}
                \end{pfsteps*}
                Contradicts \pfref{[pair7]not-match1}.
            \end{byCases}
            \begin{pfsteps*}
            \item $\cancel{\hpatmatch{\hpair{e_1}{e_2}}{\hpair{p_1}{p_2}}{\theta}}$ \BY{contradiction}
            \end{pfsteps*}
            Assume $\hmaymatch{\hpair{e_1}{e_2}}{\hpair{p_1}{p_2}}$. By rule induction over \rulesref{rules:maymatch} on it, only four cases apply.
            \begin{byCases}
            \savelocalsteps{3}
            \item[\text{(\ref{rule:MMNotIntro})}]
                \begin{pfsteps*}
                \item $\notIntro{\hpair{e_1}{e_2}}$ \BY{assumption}
                \end{pfsteps*}
                Contradicts \autoref{lem:no-pair-notintro}.
            \restorelocalsteps{3}
            \item[\text{(\ref{rule:MMPairL})}]
                \begin{pfsteps*}
                \item $\hmaymatch{e_1}{p_1}$ \BY{assumption}
                \end{pfsteps*}
                Contradicts \pfref{[pair7]not-maymatch1}.
            \restorelocalsteps{3}
            \item[\text{(\ref{rule:MMPairR})}]
                \begin{pfsteps*}
                \item $\hmaymatch{e_2}{p_2}$ \BY{assumption}
                \end{pfsteps*}
                Contradicts \pfref{[pair7]not-maymatch2}.
            \restorelocalsteps{3}
            \item[\text{(\ref{rule:MMPair})}]
                \begin{pfsteps*}
                \item $\hmaymatch{e_1}{p_1}$ \BY{assumption}
                \end{pfsteps*}
                Contradicts \pfref{[pair7]not-maymatch1}.
            \end{byCases}
            \begin{pfsteps*}
            \item $\cancel{\hmaymatch{\hpair{e_1}{e_2}}{\hpair{p_1}{p_2}}}$ \BY{contradiction}
            \end{pfsteps*}
        \restorelocalsteps{2}
        \item[\hnotmatch{e_1}{p_1},\hmaymatch{e_2}{p_2}]
            \begin{pfsteps*}
            \item $\cancel{\hpatmatch{e_1}{p_1}{\theta_1}}$ \BY{assumption} \pflabel{[pair8]not-match1}
            \item $\cancel{\hmaymatch{e_1}{p_1}}$ \BY{assumption} \pflabel{[pair8]not-maymatch1}
            \item $\hnotmatch{e_1}{p_1}$ \BY{assumption} \pflabel{[pair8]notmatch1}
            \item $\cancel{\hpatmatch{e_2}{p_2}{\theta_2}}$ \BY{assumption} \pflabel{[pair8]not-match2}
            \item $\hmaymatch{e_2}{p_2}$ \BY{assumption} \pflabel{[pair8]maymatch2}
            \item $\cancel{\hnotmatch{e_2}{p_2}}$ \BY{assumption} \pflabel{[pair8]not-notmatch2}
            \item $\hnotmatch{\hpair{e_1}{e_2}}{\hpair{p_1}{p_2}}$ \BY{\ruleref{rule:NMPairL} on \pfref{[pair8]notmatch1}}
            \end{pfsteps*}
            Assume $\hpatmatch{\hpair{e_1}{e_2}}{\hpair{p_1}{p_2}}{\theta}$. By rule induction over \rulesref{rules:match} on it, only one case applies.
            \begin{byCases}
            \item[\text{(\ref{rule:MPair})}]
                \begin{pfsteps*}
                \item $\theta=\theta_1\uplus\theta_2$
                \item $\hpatmatch{e_2}{p_2}{\theta_2}$ \BY{assumption}
                \end{pfsteps*}
                Contradicts \pfref{[pair8]not-match2}.
            \end{byCases}
            \begin{pfsteps*}
            \item $\cancel{\hpatmatch{\hpair{e_1}{e_2}}{\hpair{p_1}{p_2}}{\theta}}$ \BY{contradiction}
            \end{pfsteps*}
            Assume $\hmaymatch{\hpair{e_1}{e_2}}{\hpair{p_1}{p_2}}$. By rule induction over \rulesref{rules:maymatch} on it, only four cases apply.
            \begin{byCases}
            \savelocalsteps{3}
            \item[\text{(\ref{rule:MMNotIntro})}]
                \begin{pfsteps*}
                \item $\notIntro{\hpair{e_1}{e_2}}$ \BY{assumption}
                \end{pfsteps*}
                Contradicts \autoref{lem:no-pair-notintro}.
            \restorelocalsteps{3}
            \item[\text{(\ref{rule:MMPairL})}]
                \begin{pfsteps*}
                \item $\hpatmatch{e_2}{p_2}{\theta_2}$ \BY{assumption}
                \end{pfsteps*}
                Contradicts \pfref{[pair8]not-match2}.
            \restorelocalsteps{3}
            \item[\text{(\ref{rule:MMPairR})}]
                \begin{pfsteps*}
                \item $\hpatmatch{e_1}{p_1}{\theta_1}$ \BY{assumption}
                \end{pfsteps*}
                Contradicts \pfref{[pair8]not-match1}.
            \restorelocalsteps{3}
            \item[\text{(\ref{rule:MMPair})}]
                \begin{pfsteps*}
                \item $\hmaymatch{e_1}{p_1}$ \BY{assumption}
                \end{pfsteps*}
                Contradicts \pfref{[pair8]not-maymatch1}.
            \end{byCases}
            \begin{pfsteps*}
            \item $\cancel{\hmaymatch{\hpair{e_1}{e_2}}{\hpair{p_1}{p_2}}}$ \BY{contradiction}
            \end{pfsteps*}
        \restorelocalsteps{2}
        \item[\hnotmatch{e_1}{p_1},\hnotmatch{e_2}{p_2}]
            \begin{pfsteps*}
            \item $\cancel{\hpatmatch{e_1}{p_1}{\theta_1}}$ \BY{assumption} \pflabel{[pair9]not-match1}
            \item $\cancel{\hmaymatch{e_1}{p_1}}$ \BY{assumption} \pflabel{[pair9]not-maymatch1}
            \item $\hnotmatch{e_1}{p_1}$ \BY{assumption} \pflabel{[pair9]notmatch1}
            \item $\cancel{\hpatmatch{e_2}{p_2}{\theta_2}}$ \BY{assumption} \pflabel{[pair9]not-match2}
            \item $\hmaymatch{e_2}{p_2}$ \BY{assumption} \pflabel{[pair9]maymatch2}
            \item $\cancel{\hnotmatch{e_2}{p_2}}$ \BY{assumption} \pflabel{[pair9]not-notmatch2}
            \item $\hnotmatch{\hpair{e_1}{e_2}}{\hpair{p_1}{p_2}}$ \BY{\ruleref{rule:NMPairL} on \pfref{[pair9]notmatch1}}
            \end{pfsteps*}
            Assume $\hpatmatch{\hpair{e_1}{e_2}}{\hpair{p_1}{p_2}}{\theta}$. By rule induction over \rulesref{rules:match} on it, only one case applies.
            \begin{byCases}
            \item[\text{(\ref{rule:MPair})}]
                \begin{pfsteps*}
                \item $\theta=\theta_1\uplus\theta_2$
                \item $\hpatmatch{e_2}{p_2}{\theta_2}$ \BY{assumption}
                \end{pfsteps*}
                Contradicts \pfref{[pair9]not-match2}.
            \end{byCases}
            \begin{pfsteps*}
            \item $\cancel{\hpatmatch{\hpair{e_1}{e_2}}{\hpair{p_1}{p_2}}{\theta}}$ \BY{contradiction}
            \end{pfsteps*}
            Assume $\hmaymatch{\hpair{e_1}{e_2}}{\hpair{p_1}{p_2}}$. By rule induction over \rulesref{rules:maymatch} on it, only four cases apply.
            \begin{byCases}
            \savelocalsteps{3}
            \item[\text{(\ref{rule:MMNotIntro})}]
                \begin{pfsteps*}
                \item $\notIntro{\hpair{e_1}{e_2}}$ \BY{assumption}
                \end{pfsteps*}
                Contradicts \autoref{lem:no-pair-notintro}.
            \restorelocalsteps{3}
            \item[\text{(\ref{rule:MMPairL})}]
                \begin{pfsteps*}
                \item $\hpatmatch{e_2}{p_2}{\theta_2}$ \BY{assumption}
                \end{pfsteps*}
                Contradicts \pfref{[pair9]not-match2}.
            \restorelocalsteps{3}
            \item[\text{(\ref{rule:MMPairR})}]
                \begin{pfsteps*}
                \item $\hpatmatch{e_1}{p_1}{\theta_1}$ \BY{assumption}
                \end{pfsteps*}
                Contradicts \pfref{[pair9]not-match1}.
            \restorelocalsteps{3}
            \item[\text{(\ref{rule:MMPair})}]
                \begin{pfsteps*}
                \item $\hmaymatch{e_1}{p_1}$ \BY{assumption}
                \end{pfsteps*}
                Contradicts \pfref{[pair9]not-maymatch1}.
            \end{byCases}
            \begin{pfsteps*}
            \item $\cancel{\hmaymatch{\hpair{e_1}{e_2}}{\hpair{p_1}{p_2}}}$ \BY{contradiction}
            \end{pfsteps*}
        \end{byCases}
    \end{byCases}
    
\end{byCases}
\resetpfcounter
\end{proof}

\begin{lemma}[Matching Coherence of Constraint]
  \label{lem:const-matching-coherence}
  Suppose that $\hexptyp{\cdot}{\Delta_e}{e}{\tau}$ and $\isFinal{e}$ and $\chpattyp{p}{\tau}{\xi}{\Gamma}{\Delta}$. Then we have
  \begin{enumerate}
  \item $\csatisfy{e}{\xi}$ iff $\hpatmatch{e}{p}{\theta}$
  \item $\cmaysatisfy{e}{\xi}$ iff $\hmaymatch{e}{p}$
  \item $\cnotsatisfyormay{e}{\xi}$ iff $\hnotmatch{e}{p}$
  \end{enumerate}
\end{lemma}
\begin{proof}
\begin{pfsteps*}
\item $\hexptyp{\cdot}{\Delta_e}{e}{\tau}$ \BY{assumption} \pflabel{eTyp}
\item $\isFinal{e}$ \BY{assumption} \pflabel{eFinal}
\item $\chpattyp{p}{\tau}{\xi}{\Gamma}{\Delta}$ \BY{assumption} \pflabel{patTyp}
\end{pfsteps*}
Given \autoref{lem:pat-xi-type}, \autoref{thrm:exclusive-constraint-satisfaction}, and \autoref{lem:match-determinism}, it is sufficient to prove
\begin{enumerate}
  \item $\csatisfy{e}{\xi}$ iff $\hpatmatch{e}{p}{\theta}$
  \item $\cmaysatisfy{e}{\xi}$ iff $\hmaymatch{e}{p}$
\end{enumerate}
By rule induction over Rules (\ref{rules:PatTyp}) on \pfref{patTyp}.
\begin{byCases}
\savelocalsteps{0}
\item[\text{(\ref{rule:PTVar})}]
    \begin{pfsteps*}
    \item $p=x$ \BY{assumption}
    \item $\xi=\ctruth$ \BY{assumption}
    \end{pfsteps*}
    \begin{enumerate}
    \savelocalsteps{1}
    \item Prove $\csatisfy{e}{\ctruth}$ implies $\hpatmatch{e}{x}{\theta}$ for some $\theta$.
        \begin{pfsteps*}
        \item $\hpatmatch{e}{x}{e / x}$ \BY{Rule (\ref{rule:MVar})}
        \end{pfsteps*}
    \restorelocalsteps{1}
    \item Prove $\hpatmatch{e}{x}{\theta}$ implies $\csatisfy{e}{\ctruth}$.
        \begin{pfsteps*}
        \item $\csatisfy{e}{\ctruth}$ \BY{Rule (\ref{rule:CSTruth})}
        \end{pfsteps*}
    \restorelocalsteps{1}
    \item Prove $\cmaysatisfy{e}{\ctruth}$ implies $\hmaymatch{e}{x}$.
        \begin{pfsteps*}
        \item $\cnotmaysatisfy{e}{\ctruth}$ \BY{\autoref{lem:no-e-may-satisfy-truth}}
        \end{pfsteps*}
        Vacuously true.
    \restorelocalsteps{1}
    \item Prove $\hmaymatch{e}{x}$ implies $\cmaysatisfy{e}{\ctruth}$.
    
        By rule induction over Rules (\ref{rules:maymatch}), we notice that either, $\hmaymatch{e}{x}$ is in syntactic contradiction with all the cases, or the premise $\refutable{x}$ is not derivable. Hence, $\hmaymatch{e}{x}$ are not derivable. And thus vacuously true.
    \end{enumerate}
    
\restorelocalsteps{0}
\item[\text{(\ref{rule:PTWild})}]
    \begin{pfsteps*}
    \item $p=\_$ \BY{assumption}
    \item $\xi=\ctruth$ \BY{assumption}
    \end{pfsteps*}
    \begin{enumerate}
    \savelocalsteps{1}
    \item Prove $\csatisfy{e}{\ctruth}$ implies $\hpatmatch{e}{\_}{\theta}$ for some $\theta$.
        \begin{pfsteps*}
        \item $\hpatmatch{e}{\_}{\cdot}$ \BY{Rule (\ref{rule:MVar})}
        \end{pfsteps*}
    \restorelocalsteps{1}
    \item Prove $\hpatmatch{e}{\_}{\theta}$ implies $\csatisfy{e}{\ctruth}$.
        \begin{pfsteps*}
        \item $\csatisfy{e}{\ctruth}$ \BY{Rule (\ref{rule:CSTruth})}
        \end{pfsteps*}
    \restorelocalsteps{1}
    \item Prove $\cmaysatisfy{e}{\ctruth}$ implies $\hmaymatch{e}{\_}$.
        \begin{pfsteps*}
        \item $\cnotmaysatisfy{e}{\ctruth}$ \BY{\autoref{lem:no-e-may-satisfy-truth}}
        \end{pfsteps*}
        Vacuously true.
    \restorelocalsteps{1}
    \item Prove $\hmaymatch{e}{\_}$ implies $\cmaysatisfy{e}{\xi}$.
    
        By rule induction over Rules (\ref{rules:maymatch}), we notice that either, $\hmaymatch{e}{\_}$ is in syntactic contradiction with all the cases, or the premise $\refutable{\_}$ is not derivable. Hence, $\hmaymatch{e}{\_}$ are not derivable. And thus vacuously true.
    \end{enumerate}
    
\restorelocalsteps{0}
\item[\text{(\ref{rule:PTEHole})}]
    \begin{pfsteps*}
    \item $p=\heholep{w}$ \BY{assumption}
    \item $\xi=\cunknown$ \BY{assumption}
    \item $\cdual{\xi}=\cunknown$ \BY{Definition \ref{defn:dual}}
    \end{pfsteps*}
    \begin{enumerate}
    \savelocalsteps{1}
    \item Prove $\csatisfy{e}{\cunknown}$ implies $\hpatmatch{e}{\heholep{w}}{\theta}$ for some $\theta$.
        \begin{pfsteps*}
        \item $\cnotsatisfy{e}{\cunknown}$ \BY{Rule (\ref{rule:MVar})}
        \end{pfsteps*}
        Vacuously true.
    \restorelocalsteps{1}
    \item Prove $\hpatmatch{e}{\heholep{w}}{\theta}$ implies $\csatisfy{e}{\cunknown}$.\\
        By rule induction over Rules (\ref{rules:match}), we notice that $\hpatmatch{e}{\heholep{w}}{\theta}$ is in syntactic contradiction with all the cases, hence not derivable. And thus vacuously true.
    \restorelocalsteps{1}
    \item Prove $\cmaysatisfy{e}{\cunknown}$ implies $\hmaymatch{e}{\heholep{w}}$.
        \begin{pfsteps*}
        \item $\hmaymatch{e}{\heholep{w}}$ \BY{Rule (\ref{rule:MMEHole})}
        \end{pfsteps*}
    \restorelocalsteps{1}
    \item Prove $\hmaymatch{e}{\heholep{w}}$ implies $\cmaysatisfy{e}{\cunknown}$.
        \begin{pfsteps*}
        \item $\cmaysatisfy{e}{\cunknown}$ \BY{Rule (\ref{rule:CMSUnknown})}
        \end{pfsteps*}
    \end{enumerate}

\restorelocalsteps{0}
\item[\text{(\ref{rule:PTHole})}]
    \begin{pfsteps*}
    \item $p=\hholep{p_0}{w}{\tau'}$ \BY{assumption}
    \item $\xi=\cunknown$ \BY{assumption}
    \end{pfsteps*}
    \begin{enumerate}
    \savelocalsteps{1}
    \item Prove $\csatisfy{e}{\cunknown}$ implies $\hpatmatch{e}{\hholep{p_0}{w}{\tau'}}{\theta}$ for some $\theta$.
        \begin{pfsteps*}
        \item $\cnotsatisfy{e}{\cunknown}$ \BY{Rule (\ref{rule:MVar})}
        \end{pfsteps*}
        Vacuously true.
    \restorelocalsteps{1}
    \item Prove $\hpatmatch{e}{\hholep{p_0}{w}{\tau'}}{\theta}$ implies $\csatisfy{e}{\cunknown}$.\\
        By rule induction over Rules (\ref{rules:match}), we notice that $\hpatmatch{e}{\hholep{p_0}{w}{\tau'}}{\theta}$ is in syntactic contradiction with all the cases, hence not derivable. And thus vacuously true.
    \restorelocalsteps{1}
    \item Prove $\cmaysatisfy{e}{\cunknown}$ implies $\hmaymatch{e}{\hholep{p_0}{w}{\tau'}}$.
        \begin{pfsteps*}
        \item $\hmaymatch{e}{\hholep{p_0}{w}{\tau'}}$ \BY{Rule (\ref{rule:MMHole})}
        \end{pfsteps*}
    \restorelocalsteps{1}
    \item Prove $\hmaymatch{e}{\hholep{p_0}{w}{\tau'}}$ implies $\cmaysatisfy{e}{\cunknown}$.
        \begin{pfsteps*}
        \item $\cmaysatisfy{e}{\cunknown}$ \BY{Rule (\ref{rule:CMSUnknown})}
        \end{pfsteps*}
    \end{enumerate}
    
\restorelocalsteps{0}
\item[\text{(\ref{rule:PTNum})}]
    \begin{pfsteps*}
    \item $p=\hnum{n}$ \BY{assumption}
    \item $\xi=\cnum{n}$ \BY{assumption}
    \end{pfsteps*}
    \begin{enumerate}
    \savelocalsteps{1}
    \item Prove $\csatisfy{e}{\cnum{n}}$ implies $\hpatmatch{e}{\hnum{n}}{\theta}$ for some $\theta$.
        \begin{pfsteps*}
        \item $\csatisfy{e}{\cnum{n}}$ \BY{assumption} \pflabel{[num]satisfy}
        \end{pfsteps*}
        By rule induction over \rulesref{rules:Satisfy} on \pfref{[num]satisfy}, only one case applies.
        \begin{byCases}
        \item[\text{(\ref{rule:CSNum})}]
            \begin{pfsteps*}
            \item $e=\hnum{n}$ \BY{assumption}
            \item $\hpatmatch{\hnum{n}}{\hnum{n}}{\cdot}$ \BY{\ruleref{rule:MNum}}
            \end{pfsteps*}
        \end{byCases}
    \restorelocalsteps{1}
    \item Prove $\hpatmatch{e}{\hnum{n}}{\theta}$ implies $\csatisfy{e}{\cnum{n}}$.
        \begin{pfsteps*}
        \item $\hpatmatch{e}{\hnum{n}}{\theta}$ \BY{assumption} \pflabel{[num]match}
        \end{pfsteps*}
        By rule induction over Rules (\ref{rules:match}) on \pfref{[num]match}, only one case applies.
        \begin{byCases}
        \item[\text{(\ref{rule:MNum})}]
            \begin{pfsteps*}
            \item $e=\hnum{n}$ \BY{assumption}
            \item $\theta=\cdot$ \BY{assumption}
            \item $\csatisfy{\hnum{n}}{\cnum{n}}$ \BY{\ruleref{rule:CSNum}}
            \end{pfsteps*}
        \end{byCases}
    \restorelocalsteps{1}
    \item Prove $\cmaysatisfy{e}{\cnum{n}}$ implies $\hmaymatch{e}{\hnum{n}}$.
        \begin{pfsteps*}
        \item $\cmaysatisfy{e}{\cnum{n}}$ \BY{assumption} \pflabel{[num]maysat}
        \end{pfsteps*}
        By rule induction over \rulesref{rules:MaySatisfy} on \pfref{[num]maysat}, only one case applies.
        \begin{byCases}
        \item[\text{(\ref{rule:CMSNotIntro})}]
            \begin{pfsteps*}
            \item $\notIntro{e}$ \BY{assumption} \pflabel{[num]notnum}
            \item $\refutable{\hnum{n}}$ \BY{\ruleref{rule:RNum}} \pflabel{[num]rft}
            \item $\hmaymatch{e}{\hnum{n}}$ \BY{\ruleref{rule:MMNotIntro} on \pfref{[num]notnum} and \pfref{[num]rft}}
            \end{pfsteps*}
        \end{byCases}
    \restorelocalsteps{1}
    \item Prove $\hmaymatch{e}{\hnum{n}}$ implies $\cmaysatisfy{e}{\cnum{n}}$.
        \begin{pfsteps*}
        \item $\hmaymatch{e}{\hnum{n}}$ \BY{assumption} \pflabel{[num]maymatch}
        \end{pfsteps*}
        By rule induction over \rulesref{rules:maymatch} on \pfref{[num]maymatch}, only one case applies.
        \begin{byCases}
        \item[\text{(\ref{rule:MMNotIntro})}]
            \begin{pfsteps*}
            \item $\notIntro{e}$ \BY{assumption} \pflabel{[num]notnum'}
            \item $\refutable{\cnum{n}}$ \BY{\ruleref{rule:RXNum}} \pflabel{[num]xi-rft}
            \item $\cmaysatisfy{e}{\cnum{n}}$ \BY{\ruleref{rules:MaySatisfy} on \pfref{[num]notnum'} and \pfref{[num]xi-rft}}
            \end{pfsteps*}
        \end{byCases}
    \end{enumerate}
\restorelocalsteps{0}
\item[\text{(\ref{rule:PTInl})}]
    \begin{pfsteps*}
    \item $p=\hinlp{p_1}$ \BY{assumption}
    \item $\xi=\cinl{\xi_1}$ \BY{assumption}
    \item $\tau=\tsum{\tau_1}{\tau_2}$ \BY{assumption}
    \item $\chpattyp{p_1}{\tau_1}{\xi_1}{\Gamma}{\Delta}$ \BY{assumption} \pflabel{[inl]p1-typ}
    \end{pfsteps*}
    By rule induction over Rules (\ref{rules:TExp}) on \pfref{eTyp}, the following cases apply.
    \begin{byCases}
    \savelocalsteps{1}
    \item[\text{(\ref{rule:TEHole}),(\ref{rule:THole}),(\ref{rule:TAp}),(\ref{rule:TPrl}),(\ref{rule:TPrr}),(\ref{rule:TMatchZPre}),(\ref{rule:TMatchNZPre})}]
        \begin{pfsteps*}
        \item $e=\hehole{u},\hhole{e_0}{u},\hap{e_1}{e_2},\hprl{e_0},\hprr{e_0},\hmatch{e_0}{\zrules}$ \BY{assumption}
        \item $\notIntro{e}$ \BY{Rule (\ref{rule:NVEHole}),(\ref{rule:NVHole}),(\ref{rule:NVAp}),(\ref{rule:NVMatch}),(\ref{rule:NVPrl}),(\ref{rule:NVPrr})} \pflabel{[inl]notintro}
        \end{pfsteps*}
        \begin{enumerate}
        \savelocalsteps{2}
        \item Prove $\csatisfy{e}{\cinl{\xi_1}}$ implies $\hpatmatch{e}{\hinlp{p_1}}{\theta}$ for some $\theta$.
        By rule induction over \rulesref{rules:Satisfy} on $\csatisfy{e}{\cinl{\xi_1}}$, no case applies due to syntactic contradiction. \\ Therefore, vacuously true.
        \restorelocalsteps{2}
        \item Prove $\hpatmatch{e}{\hinlp{p_1}}{\theta}$ implies $\csatisfy{e}{\cinl{\xi_1}}$.
        By rule induction over \rulesref{rules:match} on $\hpatmatch{e}{\hinlp{p_1}}{\theta}$, no case applies due to syntactic contradiction. \\ Therefore, vacuously true.
        \restorelocalsteps{2}
        \item Prove $\cmaysatisfy{e}{\cinl{\xi_1}}$ implies $\hmaymatch{e}{\hinlp{p_1}}$.
            \begin{pfsteps*}
            \item $\refutable{\hinlp{p_1}}$ \BY{\ruleref{rule:RInl}} \pflabel{[inl]rft}
            \item $\hmaymatch{e}{\hinlp{p_1}}$ \BY{\ruleref{rule:MMNotIntro} on \pfref{[inl]notintro} and \pfref{[inl]rft}}
            \end{pfsteps*}
        \restorelocalsteps{2}
        \item Prove $\hmaymatch{e}{\hinlp{p_1}}$ implies $\cmaysatisfy{e}{\cinl{\xi_1}}$.
            \begin{pfsteps*}
            \item $\refutable{\hinlp{\xi_1}}$ \BY{\ruleref{rule:RXInl}} \pflabel{[inl]xi-rft}
            \item $\cmaysatisfy{e}{\cinl{\xi_1}}$ \BY{\ruleref{rule:CMSNotIntro} on \pfref{[inl]notintro} and \pfref{[inl]xi-rft}}
            \end{pfsteps*}
        \end{enumerate}
    \restorelocalsteps{1}
    \item[\text{(\ref{rule:TInl})}]
        \begin{pfsteps*}
        \item $e=\hinl{\tau_2}{e_1}$ \BY{assumption}
        \item $\hexptyp{\cdot}{\Delta_e}{e_1}{\tau_1}$ \BY{assumption} \pflabel{[inl]e1-typ}
        \item $\isFinal{e_1}$ \BY{\autoref{lem:inl-final} on \pfref{eFinal}} \pflabel{[inl]e1-final}
        \end{pfsteps*}
        By inductive hypothesis on \pfref{[inl]e1-final} and \pfref{[inl]e1-typ} and \pfref{[inl]p1-typ}.
        \begin{pfsteps*}
        \item $\csatisfy{e_1}{\xi_1}$ iff $\hpatmatch{e_1}{p_1}{\theta}$ for some $\theta$ \pflabel{[inl]true-equiv}
        \item $\cmaysatisfy{e_1}{\xi_1}$ iff $\hmaymatch{e_1}{p_1}$ \pflabel{[inl]may-equiv}
        \end{pfsteps*}
        \begin{enumerate}
        \savelocalsteps{2}
        \item Prove $\csatisfy{\hinl{\tau_2}{e_1}}{\cinl{\xi_1}}$ implies $\hpatmatch{\hinl{\tau_2}{e_1}}{\hinlp{p_1}}{\theta}$ for some $\theta$.
            \begin{pfsteps*}
            \item $\csatisfy{\hinl{\tau_2}{e_1}}{\cinl{\xi_1}}$ \BY{assumption} \pflabel{[inl1]satisfy}
            \end{pfsteps*}
            By rule induction over \rulesref{rules:Satisfy} on \pfref{[inl1]satisfy}, only one case applies.
            \begin{byCases}
            \item[\text{(\ref{rule:CSInl})}]
                \begin{pfsteps*}
                \item $\csatisfy{e_1}{\xi_1}$ \BY{assumption} \pflabel{[inl1]satisfy1}
                \item $\hpatmatch{e_1}{p_1}{\theta_1}$ for some $\theta_1$ \BY{\pfref{[inl]true-equiv} on \pfref{[inl1]satisfy1}} \pflabel{[inl1]match1}
                \item $\hpatmatch{\hinl{\tau_2}{e_1}}{\hinlp{p_1}}{\theta_1}$ \BY{\ruleref{rule:MInl} on \pfref{[inl1]match1}}
                \end{pfsteps*}
            \end{byCases}
        \restorelocalsteps{2}
        \item Prove $\hpatmatch{\hinl{\tau_2}{e_1}}{\hinlp{p_1}}{\theta}$ implies $\csatisfy{\hinl{\tau_2}{e_1}}{\cinl{\xi_1}}$.
            \begin{pfsteps*}
            \item $\hpatmatch{\hinl{\tau_2}{e_1}}{\hinlp{p_1}}{\theta}$ \BY{assumption} \pflabel{[inl2]match}
            \end{pfsteps*}
            By rule induction over \rulesref{rules:match} on \pfref{[inl2]match}, only one case applies.
            \begin{byCases}
            \item[\text{(\ref{rule:MInl})}]
                \begin{pfsteps*}
                \item $\hpatmatch{e_1}{p_1}{\theta}$ \BY{assumption} \pflabel{[inl2]match1}
                \item $\csatisfy{e_1}{\xi_1}$ \BY{\pfref{[inl]true-equiv} on \pfref{[inl2]match1}} \pflabel{[inl2]satisfy1}
                \item $\csatisfy{\hinl{\tau_2}{e_1}}{\cinl{\xi_1}}$ \BY{\ruleref{rule:CSInl} on \pfref{[inl2]satisfy1}}
                \end{pfsteps*}
            \end{byCases}
        \restorelocalsteps{2}
        \item Prove $\cmaysatisfy{\hinl{\tau_2}{e_1}}{\cinl{\xi_1}}$ implies $\hmaymatch{\hinl{\tau_2}{e_1}}{\hinlp{p_1}}$.
            \begin{pfsteps*}
            \item $\cmaysatisfy{\hinl{\tau_2}{e_1}}{\cinl{\xi_1}}$ \BY{assumption} \pflabel{[inl3]maysat}
            \end{pfsteps*}
            By rule induction over \rulesref{rules:MaySatisfy} on \pfref{[inl3]maysat}, only two cases apply.
            \begin{byCases}
            \savelocalsteps{3}
            \item[\text{(\ref{rule:CMSNotIntro})}]
                \begin{pfsteps*}
                \item $\notIntro{\hinl{\tau_2}{e_1}}$ \BY{assumption}
                \end{pfsteps*}
                Contradicts \autoref{lem:no-inl-notintro}.
            \restorelocalsteps{3}
            \item[\text{(\ref{rule:CMSInl})}]
                \begin{pfsteps*}
                \item $\cmaysatisfy{e_1}{\xi_1}$ \BY{assumption} \pflabel{[inl3]maysat1}
                \item $\hmaymatch{e_1}{p_1}$ \BY{\pfref{[inl]may-equiv} on \pfref{[inl3]maysat1}} \pflabel{[inl3]maymatch1}
                \item $\hmaymatch{\hinl{\tau_2}{e_1}}{\hinlp{p_1}}$ \BY{\ruleref{rule:MMInl} on \pfref{[inl3]maymatch1}}
                \end{pfsteps*}
            \end{byCases}
        \restorelocalsteps{2}
        \item Prove $\hmaymatch{\hinl{\tau_2}{e_1}}{\hinlp{p_1}}$ implies $\cmaysatisfy{\hinl{\tau_2}{e_1}}{\cinl{\xi_1}}$.
            \begin{pfsteps*}
            \item $\hmaymatch{\hinl{\tau_2}{e_1}}{\hinlp{p_1}}$ \BY{assumption} \pflabel{[inl4]maymatch}
            \end{pfsteps*}
            By rule induction over \rulesref{rules:maymatch} on \pfref{[inl4]maymatch}, only two cases apply.
            \begin{byCases}
            \savelocalsteps{3}
            \item[\text{(\ref{rule:MMNotIntro})}]
                \begin{pfsteps*}
                \item $\notIntro{\hinl{\tau_2}{e_1}}$ \BY{assumption}
                \end{pfsteps*}
                Contradicts \autoref{lem:no-inl-notintro}.
            \restorelocalsteps{3}
            \item[\text{(\ref{rule:MMInl})}]
                \begin{pfsteps*}
                \item $\hmaymatch{e_1}{p_1}$ \BY{assumption} \pflabel{[inl4]maymatch1}
                \item $\cmaysatisfy{e_1}{\xi_1}$ \BY{\pfref{[inl]may-equiv} on \pfref{[inl4]maymatch1}} \pflabel{[inl4]maysat1}
                \item $\cmaysatisfy{\hinl{\tau_2}{e_1}}{\cinl{\xi_1}}$ \BY{\ruleref{rule:CMSInl} on \pfref{[inl4]maysat1}}
                \end{pfsteps*}
            \end{byCases}
        \end{enumerate}
    \end{byCases}
\restorelocalsteps{0}
\item[\text{(\ref{rule:PTInr})}]
    \begin{pfsteps*}
    \item $p=\hinrp{p_2}$ \BY{assumption}
    \item $\xi=\cinr{\xi_2}$ \BY{assumption}
    \item $\tau=\tsum{\tau_1}{\tau_2}$ \BY{assumption}
    \item $\chpattyp{p_2}{\tau_2}{\xi_2}{\Gamma}{\Delta}$ \BY{assumption} \pflabel{[inr]p1-typ}
    \end{pfsteps*}
    By rule induction over Rules (\ref{rules:TExp}) on \pfref{eTyp}, the following cases apply.
    \begin{byCases}
    \savelocalsteps{1}
    \item[\text{(\ref{rule:TEHole}),(\ref{rule:THole}),(\ref{rule:TAp}),(\ref{rule:TPrl}),(\ref{rule:TPrr}),(\ref{rule:TMatchZPre}),(\ref{rule:TMatchNZPre})}]
        \begin{pfsteps*}
        \item $e=\hehole{u},\hhole{e_0}{u},\hap{e_1}{e_2},\hprl{e_0},\hprr{e_0},\hmatch{e_0}{\zrules}$ \BY{assumption}
        \item $\notIntro{e}$ \BY{Rule (\ref{rule:NVEHole}),(\ref{rule:NVHole}),(\ref{rule:NVAp}),(\ref{rule:NVMatch}),(\ref{rule:NVPrl}),(\ref{rule:NVPrr})} \pflabel{[inr]notintro}
        \end{pfsteps*}
        \begin{enumerate}
        \savelocalsteps{2}
        \item Prove $\csatisfy{e}{\cinr{\xi_2}}$ implies $\hpatmatch{e}{\hinrp{p_2}}{\theta}$ for some $\theta$.
        By rule induction over \rulesref{rules:Satisfy} on $\csatisfy{e}{\cinr{\xi_2}}$, no case applies due to syntactic contradiction. \\ Therefore, vacuously true.
        \restorelocalsteps{2}
        \item Prove $\hpatmatch{e}{\hinrp{p_2}}{\theta}$ implies $\csatisfy{e}{\cinr{\xi_2}}$.
        By rule induction over \rulesref{rules:match} on $\hpatmatch{e}{\hinrp{p_2}}{\theta}$, no case applies due to syntactic contradiction. \\ Therefore, vacuously true.
        \restorelocalsteps{2}
        \item Prove $\cmaysatisfy{e}{\cinr{\xi_2}}$ implies $\hmaymatch{e}{\hinrp{p_2}}$.
            \begin{pfsteps*}
            \item $\refutable{\hinrp{p_2}}$ \BY{\ruleref{rule:RInr}} \pflabel{[inr]rft}
            \item $\hmaymatch{e}{\hinrp{p_2}}$ \BY{\ruleref{rule:MMNotIntro} on \pfref{[inr]notintro} and \pfref{[inr]rft}}
            \end{pfsteps*}
        \restorelocalsteps{2}
        \item Prove $\hmaymatch{e}{\hinrp{p_2}}$ implies $\cmaysatisfy{e}{\cinr{\xi_2}}$.
            \begin{pfsteps*}
            \item $\refutable{\hinrp{\xi_2}}$ \BY{\ruleref{rule:RXInr}} \pflabel{[inr]xi-rft}
            \item $\cmaysatisfy{e}{\cinr{\xi_2}}$ \BY{\ruleref{rule:CMSNotIntro} on \pfref{[inr]notintro} and \pfref{[inr]xi-rft}}
            \end{pfsteps*}
        \end{enumerate}
    \restorelocalsteps{1}
    \item[\text{(\ref{rule:TInr})}]
        \begin{pfsteps*}
        \item $e=\hinr{\tau_1}{e_2}$ \BY{assumption}
        \item $\hexptyp{\cdot}{\Delta_e}{e_2}{\tau_2}$ \BY{assumption} \pflabel{[inr]e1-typ}
        \item $\isFinal{e_2}$ \BY{\autoref{lem:inl-final} on \pfref{eFinal}} \pflabel{[inr]e1-final}
        \end{pfsteps*}
        By inductive hypothesis on \pfref{[inr]e1-final} and \pfref{[inr]e1-typ} and \pfref{[inr]p1-typ}.
        \begin{pfsteps*}
        \item $\csatisfy{e_2}{\xi_2}$ iff $\hpatmatch{e_2}{p_2}{\theta}$ for some $\theta$ \pflabel{[inr]true-equiv}
        \item $\cmaysatisfy{e_2}{\xi_2}$ iff $\hmaymatch{e_2}{p_2}$ \pflabel{[inr]may-equiv}
        \end{pfsteps*}
        \begin{enumerate}
        \savelocalsteps{2}
        \item Prove $\csatisfy{\hinr{\tau_1}{e_2}}{\cinr{\xi_2}}$ implies $\hpatmatch{\hinr{\tau_1}{e_2}}{\hinrp{p_2}}{\theta}$ for some $\theta$.
            \begin{pfsteps*}
            \item $\csatisfy{\hinr{\tau_1}{e_2}}{\cinr{\xi_2}}$ \BY{assumption} \pflabel{[inr1]satisfy}
            \end{pfsteps*}
            By rule induction over \rulesref{rules:Satisfy} on \pfref{[inr1]satisfy}, only one case applies.
            \begin{byCases}
            \item[\text{(\ref{rule:CSInl})}]
                \begin{pfsteps*}
                \item $\csatisfy{e_2}{\xi_2}$ \BY{assumption} \pflabel{[inr1]satisfy1}
                \item $\hpatmatch{e_2}{p_2}{\theta_1}$ for some $\theta_1$ \BY{\pfref{[inr]true-equiv} on \pfref{[inr1]satisfy1}} \pflabel{[inr1]match1}
                \item $\hpatmatch{\hinr{\tau_1}{e_2}}{\hinrp{p_2}}{\theta_1}$ \BY{\ruleref{rule:MInl} on \pfref{[inr1]match1}}
                \end{pfsteps*}
            \end{byCases}
        \restorelocalsteps{2}
        \item Prove $\hpatmatch{\hinr{\tau_1}{e_2}}{\hinrp{p_2}}{\theta}$ implies $\csatisfy{\hinr{\tau_1}{e_2}}{\cinr{\xi_2}}$.
            \begin{pfsteps*}
            \item $\hpatmatch{\hinr{\tau_1}{e_2}}{\hinrp{p_2}}{\theta}$ \BY{assumption} \pflabel{[inr2]match}
            \end{pfsteps*}
            By rule induction over \rulesref{rules:match} on \pfref{[inr2]match}, only one case applies.
            \begin{byCases}
            \item[\text{(\ref{rule:MInl})}]
                \begin{pfsteps*}
                \item $\hpatmatch{e_2}{p_2}{\theta}$ \BY{assumption} \pflabel{[inr2]match1}
                \item $\csatisfy{e_2}{\xi_2}$ \BY{\pfref{[inr]true-equiv} on \pfref{[inr2]match1}} \pflabel{[inr2]satisfy1}
                \item $\csatisfy{\hinr{\tau_1}{e_2}}{\cinr{\xi_2}}$ \BY{\ruleref{rule:CSInl} on \pfref{[inr2]satisfy1}}
                \end{pfsteps*}
            \end{byCases}
        \restorelocalsteps{2}
        \item Prove $\cmaysatisfy{\hinr{\tau_1}{e_2}}{\cinr{\xi_2}}$ implies $\hmaymatch{\hinr{\tau_1}{e_2}}{\hinrp{p_2}}$.
            \begin{pfsteps*}
            \item $\cmaysatisfy{\hinr{\tau_1}{e_2}}{\cinr{\xi_2}}$ \BY{assumption} \pflabel{[inr3]maysat}
            \end{pfsteps*}
            By rule induction over \rulesref{rules:MaySatisfy} on \pfref{[inr3]maysat}, only two cases apply.
            \begin{byCases}
            \savelocalsteps{3}
            \item[\text{(\ref{rule:CMSNotIntro})}]
                \begin{pfsteps*}
                \item $\notIntro{\hinr{\tau_1}{e_2}}$ \BY{assumption}
                \end{pfsteps*}
                Contradicts \autoref{lem:no-inl-notintro}.
            \restorelocalsteps{3}
            \item[\text{(\ref{rule:CMSInl})}]
                \begin{pfsteps*}
                \item $\cmaysatisfy{e_2}{\xi_2}$ \BY{assumption} \pflabel{[inr3]maysat1}
                \item $\hmaymatch{e_2}{p_2}$ \BY{\pfref{[inr]may-equiv} on \pfref{[inr3]maysat1}} \pflabel{[inr3]maymatch1}
                \item $\hmaymatch{\hinr{\tau_1}{e_2}}{\hinrp{p_2}}$ \BY{\ruleref{rule:MMInl} on \pfref{[inr3]maymatch1}}
                \end{pfsteps*}
            \end{byCases}
        \restorelocalsteps{2}
        \item Prove $\hmaymatch{\hinr{\tau_1}{e_2}}{\hinrp{p_2}}$ implies $\cmaysatisfy{\hinr{\tau_1}{e_2}}{\cinr{\xi_2}}$.
            \begin{pfsteps*}
            \item $\hmaymatch{\hinr{\tau_1}{e_2}}{\hinrp{p_2}}$ \BY{assumption} \pflabel{[inr4]maymatch}
            \end{pfsteps*}
            By rule induction over \rulesref{rules:maymatch} on \pfref{[inr4]maymatch}, only two cases apply.
            \begin{byCases}
            \savelocalsteps{3}
            \item[\text{(\ref{rule:MMNotIntro})}]
                \begin{pfsteps*}
                \item $\notIntro{\hinr{\tau_1}{e_2}}$ \BY{assumption}
                \end{pfsteps*}
                Contradicts \autoref{lem:no-inl-notintro}.
            \restorelocalsteps{3}
            \item[\text{(\ref{rule:MMInl})}]
                \begin{pfsteps*}
                \item $\hmaymatch{e_2}{p_2}$ \BY{assumption} \pflabel{[inr4]maymatch1}
                \item $\cmaysatisfy{e_2}{\xi_2}$ \BY{\pfref{[inr]may-equiv} on \pfref{[inr4]maymatch1}} \pflabel{[inr4]maysat1}
                \item $\cmaysatisfy{\hinr{\tau_1}{e_2}}{\cinr{\xi_2}}$ \BY{\ruleref{rule:CMSInl} on \pfref{[inr4]maysat1}}
                \end{pfsteps*}
            \end{byCases}
        \end{enumerate}
    \end{byCases}
\restorelocalsteps{0}
\item[\text{(\ref{rule:PTPair})}]
    \begin{pfsteps*}
    \item $p=\hpair{p_1}{p_2}$ \BY{assumption}
    \item $\xi=\cpair{\xi_1}{\xi_2}$ \BY{assumption}
    \item $\tau=\tprod{\tau_1}{\tau_2}$ \BY{assumption}
    \item $\Gamma=\Gamma_1\uplus \Gamma_2$ \BY{assumption}
    \item $\Delta=\Delta_1\uplus \Delta_2$ \BY{assumption}
    \item $\chpattyp{p_1}{\tau_1}{\xi_1}{\Gamma_1}{\Delta_1}$ \BY{assumption} \pflabel{[pair]p1-typ}
    \item $\chpattyp{p_2}{\tau_2}{\xi_2}{\Gamma_2}{\Delta_2}$ \BY{assumption} \pflabel{[pair]p2-typ}
    \end{pfsteps*}
    By rule induction over Rules (\ref{rules:TExp}) on \pfref{eTyp}, the following cases apply.
    \begin{byCases}
    \savelocalsteps{1}
    \item[\text{(\ref{rule:TEHole}),(\ref{rule:THole}),(\ref{rule:TAp}),(\ref{rule:TPrl}),(\ref{rule:TPrr}),(\ref{rule:TMatchZPre}),(\ref{rule:TMatchNZPre})}]
        \begin{pfsteps*}
        \item $e=\hehole{u},\hhole{e_0}{u},\hap{e_1}{e_2},\hprl{e_0},\hprr{e_0},\hmatch{e_0}{\zrules}$ \BY{assumption}
        \item $\notIntro{e}$ \BY{Rule (\ref{rule:NVEHole}),(\ref{rule:NVHole}),(\ref{rule:NVAp}),(\ref{rule:NVMatch}),(\ref{rule:NVPrl}),(\ref{rule:NVPrr})} \pflabel{[epair]notintro}
        \item $\isIndet{e}$ \BY{\autoref{lem:final-notintro-indet} on \pfref{eFinal} and \pfref{[epair]notintro}} \pflabel{[epair]indet}
        \item $\isIndet{\hprl{e}}$ \BY{Rule (\ref{rule:IPrl}) on \pfref{[epair]indet}} \pflabel{[epair]prl-indet}
        \item $\isFinal{\hprl{e}}$ \BY{Rule (\ref{rule:FIndet}) on \pfref{[epair]prl-indet}} \pflabel{[epair]prl-final}
        \item $\isIndet{\hprr{e}}$ \BY{Rule (\ref{rule:IPrr}) on \pfref{[epair]indet}} \pflabel{[epair]prr-indet}
        \item $\isFinal{\hprr{e}}$ \BY{Rule (\ref{rule:FIndet}) on \pfref{[epair]prr-indet}} \pflabel{[epair]prr-final}
        \item $\hexptyp{\cdot}{\Delta}{\hprl{e}}{\tau_1}$ \BY{Rule (\ref{rule:TPrl}) on \pfref{eTyp}} \pflabel{[epair]prl-typ}
        \item $\hexptyp{\cdot}{\Delta}{\hprr{e}}{\tau_2}$ \BY{Rule (\ref{rule:TPrr}) on \pfref{eTyp}} \pflabel{[epair]prr-typ}
        \end{pfsteps*}
        By inductive hypothesis on \pfref{[pair]p1-typ} and \pfref{[epair]prl-typ} and \pfref{[epair]prl-final} and by inductive hypothesis on \pfref{[pair]p2-typ} and \pfref{[epair]prr-typ} and \pfref{[epair]prr-final}.
        \begin{pfsteps*}
        \item $\csatisfy{\hprl{e}}{\xi_1}$ iff $\hpatmatch{\hprl{e}}{p_1}{\theta_1}$ for some $\theta_1$ \pflabel{[epair]true-equiv1}
        \item $\cmaysatisfy{\hprl{e}}{\xi_1}$ iff $\hmaymatch{\hprl{e}}{p_1}$ \pflabel{[epair]may-equiv1}
        \item $\csatisfy{\hprr{e}}{\xi_2}$ iff $\hpatmatch{\hprr{e}}{p_2}{\theta_2}$ for some $\theta_2$ \pflabel{[epair]true-equiv2}
        \item $\cmaysatisfy{\hprr{e}}{\xi_2}$ iff $\hmaymatch{\hprr{e}}{p_2}$ \pflabel{[epair]may-equiv2}
        \end{pfsteps*}
        \begin{enumerate}
        \savelocalsteps{2}
        \item Prove $\csatisfy{e}{\cpair{\xi_1}{\xi_2}}$ implies $\hpatmatch{e}{\hpair{p_1}{p_2}}{\theta}$ for some $\theta$.
            \begin{pfsteps*}
            \item $\csatisfy{e}{\cpair{\xi_1}{\xi_2}}$ \BY{assumption} \pflabel{[epair1]satisfy}
            \end{pfsteps*}
            By rule induction over \rulesref{rules:Satisfy} on \pfref{[epair1]satisfy}, only one case applies.
            \begin{byCases}
            \item[\text{(\ref{rule:CSNotIntroPair})}]
                \begin{pfsteps*}
                \item $\csatisfy{\hprl{e}}{\xi_1}$ \BY{assumption} \pflabel{[epair1]satisfy1}
                \item $\csatisfy{\hprr{e}}{\xi_2}$ \BY{assumption} \pflabel{[epair1]satisfy2}
                \item $\hpatmatch{\hprl{e}}{p_1}{\theta_1}$ \BY{\pfref{[epair]true-equiv1} on \pfref{[epair1]satisfy1}} \pflabel{[epair1]match1}
                \item $\hpatmatch{\hprr{e}}{p_2}{\theta_2}$ \BY{\pfref{[epair]true-equiv2} on \pfref{[epair1]satisfy2}} \pflabel{[epair1]match2}
                \item $\hpatmatch{e}{\hpair{p_1}{p_2}}{\theta_1\uplus\theta_2}$ \BY{\ruleref{rule:MNotIntroPair} on \pfref{[epair]notintro} and \pfref{[epair1]match1} and \pfref{[epair1]match2}}
                \end{pfsteps*}
            \end{byCases}
        \restorelocalsteps{2}
        \item Prove $\hpatmatch{e}{\hpair{p_1}{p_2}}{\theta}$ implies $\csatisfy{e}{\cpair{\xi_1}{\xi_2}}$.
            \begin{pfsteps*}
            \item $\hpatmatch{e}{\hpair{p_1}{p_2}}{\theta}$ \BY{assumption} \pflabel{[epair2]match}
            \end{pfsteps*}
            By rule induction over \rulesref{rules:match} on \pfref{[epair2]match}, only one case applies.
            \begin{byCases}
            \item[\text{(\ref{rule:MNotIntroPair})}]
                \begin{pfsteps*}
                \item $\theta=\theta_1\uplus\theta_2$ \BY{assumption}
                \item $\hpatmatch{\hprl{e}}{\xi_1}{\theta_1}$ \BY{assumption} \pflabel{[epair2]match1}
                \item $\hpatmatch{\hprr{e}}{\xi_2}{\theta_2}$ \BY{assumption} \pflabel{[epair2]match2}
                \item $\csatisfy{\hprl{e}}{\xi_1}$ \BY{\pfref{[epair]true-equiv1} on \pfref{[epair2]match1}} \pflabel{[epair2]satisfy1}
                \item $\csatisfy{\hprr{e}}{\xi_2}$ \BY{\pfref{[epair]true-equiv2} on \pfref{[epair2]match2}} \pflabel{[epair2]satisfy2}
                \item $\csatisfy{e}{\cpair{\xi_1}{\xi_2}}$ \BY{\ruleref{rule:CSNotIntroPair} on \pfref{[epair]notintro} and \pfref{[epair2]satisfy1} and \pfref{[epair2]satisfy2}}
                \end{pfsteps*}
            \end{byCases}
        \restorelocalsteps{2}
        \item Prove $\cmaysatisfy{e}{\cpair{\xi_1}{\xi_2}}$ implies $\hmaymatch{e}{\hpair{p_1}{p_2}}$.
            \begin{pfsteps*}
            \item $\cmaysatisfy{e}{\cpair{\xi_1}{\xi_2}}$ \BY{assumption} \pflabel{[epair2]satisfy}
            \end{pfsteps*}
            By rule induction over \rulesref{rules:MaySatisfy} on \pfref{[epair2]satisfy}, only one case applies.
            \begin{byCases}
            \item[\text{(\ref{rule:CMSNotIntro})}]
                \begin{pfsteps*}
                \item $\refutable{\cpair{\xi_1}{\xi_2}}$ \BY{assumption} \pflabel{[epair3]rft}
                \end{pfsteps*}
                By rule induction over \rulesref{rules:xi-refutable} on \pfref{[epair3]rft}, only two cases apply.
                \begin{byCases}
                \savelocalsteps{3}
                \item[\text{(\ref{rule:RXPairL})}]
                    \begin{pfsteps*}
                    \item $\refutable{\xi_1}$ \BY{assumption} \pflabel{[epair3]rft1}
                    \item $\notIntro{\hprl{e}}$ \BY{\ruleref{rule:NVPrl}} \pflabel{[epair3]notintro1}
                    \item $\cmaysatisfy{\hprl{e}}{\xi_1}$ \BY{\ruleref{rule:CMSNotIntro} on \pfref{[epair3]rft1} and \pfref{[epair3]notintro1}} \pflabel{[epair3]satisfy1}
                    \item $\hmaymatch{\hprl{e}}{p_1}$ \BY{\pfref{[epair]may-equiv1} on \pfref{[epair3]satisfy1}} \pflabel{[epair3]match1}
                    \end{pfsteps*}
                    By rule induction over \rulesref{rules:maymatch} on \pfref{[epair3]match1}, only three cases apply.
                    \begin{byCases}
                    \savelocalsteps{4}
                    \item[\text{(\ref{rule:MMEHole}),(\ref{rule:MMHole})}]
                        \begin{pfsteps*}
                        \item $p_1=\heholep{w},\hholep{p_0}{w}{\tau'}$ \BY{assumption}
                        \item $\refutable{p_1}$ \BY{\ruleref{rule:REHole} and \ruleref{rule:RHole}} \pflabel{[epair3]p-rft1}
                        \item $\refutable{\hpair{p_1}{p_2}}$ \BY{\ruleref{rule:RPairL} on \pfref{[epair3]p-rft1}} \pflabel{[epair3]p-rft}
                        \item $\hmaymatch{e}{\hpair{p_1}{p_2}}$ \BY{\ruleref{rule:MMNotIntro} on \pfref{[epair]notintro} and \pfref{[epair3]p-rft}}
                        \end{pfsteps*}
                    \restorelocalsteps{4}
                    \item[\text{(\ref{rule:MMNotIntro})}]
                        \begin{pfsteps*}
                        \item $\refutable{p_1}$ \BY{assumption} \pflabel{[epair3]p-rft1'}
                        \item $\refutable{\hpair{p_1}{p_2}}$ \BY{\ruleref{rule:RPairL} on \pfref{[epair3]p-rft1'}} \pflabel{[epair3]p-rft'}
                        \item $\hmaymatch{e}{\hpair{p_1}{p_2}}$ \BY{\ruleref{rule:MMNotIntro} on \pfref{[epair]notintro} and \pfref{[epair3]p-rft'}}
                        \end{pfsteps*}
                    \end{byCases}
                \restorelocalsteps{3}
                \item[\text{(\ref{rule:RXPairR})}]
                    \begin{pfsteps*}
                    \item $\refutable{\xi_2}$ \BY{assumption} \pflabel{[epair3']rft1}
                    \item $\notIntro{\hprr{e}}$ \BY{\ruleref{rule:NVPrl}} \pflabel{[epair3']notintro1}
                    \item $\cmaysatisfy{\hprr{e}}{\xi_2}$ \BY{\ruleref{rule:CMSNotIntro} on \pfref{[epair3']rft1} and \pfref{[epair3']notintro1}} \pflabel{[epair3']satisfy1}
                    \item $\hmaymatch{\hprr{e}}{p_2}$ \BY{\pfref{[epair]may-equiv2} on \pfref{[epair3']satisfy1}} \pflabel{[epair3']match1}
                    \end{pfsteps*}
                    By rule induction over \rulesref{rules:maymatch} on \pfref{[epair3']match1}, only three cases apply.
                    \begin{byCases}
                    \savelocalsteps{4}
                    \item[\text{(\ref{rule:MMEHole}),(\ref{rule:MMHole})}]
                        \begin{pfsteps*}
                        \item $p_2=\heholep{w},\hholep{p_0}{w}{\tau'}$ \BY{assumption}
                        \item $\refutable{p_2}$ \BY{\ruleref{rule:REHole} and \ruleref{rule:RHole}} \pflabel{[epair3']p-rft1}
                        \item $\refutable{\hpair{p_1}{p_2}}$ \BY{\ruleref{rule:RPairR} on \pfref{[epair3']p-rft1}} \pflabel{[epair3']p-rft}
                        \item $\hmaymatch{e}{\hpair{p_1}{p_2}}$ \BY{\ruleref{rule:MMNotIntro} on \pfref{[epair]notintro} and \pfref{[epair3']p-rft}}
                        \end{pfsteps*}
                    \restorelocalsteps{4}
                    \item[\text{(\ref{rule:MMNotIntro})}]
                        \begin{pfsteps*}
                        \item $\refutable{p_2}$ \BY{assumption} \pflabel{[epair3']p-rft1'}
                        \item $\refutable{\hpair{p_1}{p_2}}$ \BY{\ruleref{rule:RPairR} on \pfref{[epair3']p-rft1'}} \pflabel{[epair3']p-rft'}
                        \item $\hmaymatch{e}{\hpair{p_1}{p_2}}$ \BY{\ruleref{rule:MMNotIntro} on \pfref{[epair]notintro} and \pfref{[epair3']p-rft'}}
                        \end{pfsteps*}
                    \end{byCases}
                \end{byCases}
            \end{byCases}
        \restorelocalsteps{2}
        \item Prove $\hmaymatch{e}{\hpair{p_1}{p_2}}$ implies $\cmaysatisfy{e}{\cpair{\xi_1}{\xi_2}}$.
            \begin{pfsteps*}
            \item $\hmaymatch{e}{\hpair{p_1}{p_2}}$ \BY{assumption} \pflabel{[epair4]match}
            \end{pfsteps*}
            By rule induction over \rulesref{rules:maymatch} on \pfref{[epair4]match}, only one case applies.
            \begin{byCases}
            \item[\text{(\ref{rule:MMNotIntro})}]
                \begin{pfsteps*}
                \item $\refutable{\hpair{p_1}{p_2}}$ \BY{assumption} \pflabel{[epair4]rft}
                \end{pfsteps*}
                By rule induction over \rulesref{rules:p-refutable} on \pfref{[epair4]rft}, only two cases apply.
                \begin{byCases}
                \savelocalsteps{3}
                \item[\text{(\ref{rule:RPairL})}]
                    \begin{pfsteps*}
                    \item $\refutable{p_1}$ \BY{assumption} \pflabel{[epair4]rft1}
                    \item $\notIntro{\hprl{e}}$ \BY{\ruleref{rule:NVPrl}} \pflabel{[epair4]notintro1}
                    \item $\hmaymatch{\hprl{e}}{p_1}$ \BY{\ruleref{rule:MMNotIntro} on \pfref{[epair4]rft1} and \pfref{[epair4]notintro1}} \pflabel{[epair4]match1}
                    \item $\cmaysatisfy{\hprl{e}}{\xi_1}$ \BY{\pfref{[epair]may-equiv1} on \pfref{[epair4]match1}} \pflabel{[epair4]satisfy1}
                    \end{pfsteps*}
                    By rule induction over \rulesref{rules:MaySatisfy} on \pfref{[epair4]satisfy1}, only three cases apply.
                    \begin{byCases}
                    \savelocalsteps{4}
                    \item[\text{(\ref{rule:CMSUnknown})}]
                        \begin{pfsteps*}
                        \item $\xi_1=\cunknown$ \BY{assumption}
                        \item $\refutable{\xi_1}$ \BY{\ruleref{rule:RXUnknown}} \pflabel{[epair4]p-rft1}
                        \item $\refutable{\cpair{\xi_1}{\xi_2}}$ \BY{\ruleref{rule:RXPairL} on \pfref{[epair4]p-rft1}} \pflabel{[epair4]p-rft}
                        \item $\cmaysatisfy{e}{\cpair{\xi_1}{\xi_2}}$ \BY{\ruleref{rule:CMSNotIntro} on \pfref{[epair]notintro} and \pfref{[epair4]p-rft}}
                        \end{pfsteps*}
                    \restorelocalsteps{4}
                    \item[\text{(\ref{rule:CMSNotIntro})}]
                        \begin{pfsteps*}
                        \item $\refutable{\xi_1}$ \BY{assumption} \pflabel{[epair4]p-rft1'}
                        \item $\refutable{\cpair{\xi_1}{\xi_2}}$ \BY{\ruleref{rule:RXPairL} on \pfref{[epair4]p-rft1'}} \pflabel{[epair4]p-rft'}
                        \item $\cmaysatisfy{e}{\cpair{\xi_1}{\xi_2}}$ \BY{\ruleref{rule:CMSNotIntro} on \pfref{[epair]notintro} and \pfref{[epair4]p-rft'}}
                        \end{pfsteps*}
                    \end{byCases}
                \restorelocalsteps{3}
                \item[\text{(\ref{rule:RPairR})}]
                    \begin{pfsteps*}
                    \item $\refutable{p_2}$ \BY{assumption} \pflabel{[epair4']rft1}
                    \item $\notIntro{\hprr{e}}$ \BY{\ruleref{rule:NVPrl}} \pflabel{[epair4']notintro1}
                    \item $\hmaymatch{\hprr{e}}{p_2}$ \BY{\ruleref{rule:MMNotIntro} on \pfref{[epair4']rft1} and \pfref{[epair4']notintro1}} \pflabel{[epair4']satisfy1}
                    \item $\cmaysatisfy{\hprr{e}}{\xi_2}$ \BY{\pfref{[epair]may-equiv2} on \pfref{[epair4']satisfy1}} \pflabel{[epair4']match1}
                    \end{pfsteps*}
                    By rule induction over \rulesref{rules:MaySatisfy} on \pfref{[epair4']match1}, only three cases apply.
                    \begin{byCases}
                    \savelocalsteps{4}
                    \item[\text{(\ref{rule:CMSUnknown})}]
                        \begin{pfsteps*}
                        \item $\xi_2=\cunknown$ \BY{assumption}
                        \item $\refutable{\xi_2}$ \BY{\ruleref{rule:RXUnknown}} \pflabel{[epair4']p-rft1}
                        \item $\refutable{\cpair{\xi_1}{\xi_2}}$ \BY{\ruleref{rule:RXPairR} on \pfref{[epair4']p-rft1}} \pflabel{[epair4']p-rft}
                        \item $\cmaysatisfy{e}{\cpair{\xi_1}{\xi_2}}$ \BY{\ruleref{rule:CMSNotIntro} on \pfref{[epair]notintro} and \pfref{[epair4']p-rft}}
                        \end{pfsteps*}
                    \restorelocalsteps{4}
                    \item[\text{(\ref{rule:CMSNotIntro})}]
                        \begin{pfsteps*}
                        \item $\refutable{\xi_2}$ \BY{assumption} \pflabel{[epair4']p-rft1'}
                        \item $\refutable{\cpair{\xi_1}{\xi_2}}$ \BY{\ruleref{rule:RXPairR} on \pfref{[epair4']p-rft1'}} \pflabel{[epair4']p-rft'}
                        \item $\cmaysatisfy{e}{\cpair{\xi_1}{\xi_2}}$ \BY{\ruleref{rule:CMSNotIntro} on \pfref{[epair]notintro} and \pfref{[epair4']p-rft'}}
                        \end{pfsteps*}
                    \end{byCases}
                \end{byCases}
            \end{byCases}
        \end{enumerate}
    \restorelocalsteps{1}
    \item[\text{(\ref{rule:TPair})}]
        \begin{pfsteps*}
        \item $e=\hpair{e_1}{e_2}$ \BY{assumption}
        \item $\hexptyp{\cdot}{\Delta_e}{e_1}{\tau_1}$ \BY{assumption} \pflabel{[pair]e1-typ}
        \item $\hexptyp{\cdot}{\Delta_e}{e_2}{\tau_2}$ \BY{assumption} \pflabel{[pair]e2-typ}
        \item $\isFinal{e_1}$ \BY{\autoref{lem:pair-final} on \pfref{eFinal}} \pflabel{[pair]e1-final}
        \item $\isFinal{e_2}$ \BY{\autoref{lem:pair-final} on \pfref{eFinal}} \pflabel{[pair]e2-final}
        \end{pfsteps*}
        By inductive hypothesis on \pfref{[pair]e1-final} and \pfref{[pair]e1-typ} and \pfref{[pair]p1-typ} and by inductive hypothesis on \pfref{[pair]e2-final} and \pfref{[pair]e2-typ} and \pfref{[pair]p2-typ}.
        \begin{pfsteps*}
        \item $\csatisfy{e_1}{\xi_1}$ iff $\hpatmatch{e_1}{p_1}{\theta_1}$ for some $\theta_1$ \pflabel{[pair]true-equiv1}
        \item $\cmaysatisfy{e_1}{\xi_1}$ iff $\hmaymatch{e_1}{p_1}$ \pflabel{[pair]may-equiv1}
        \item $\csatisfy{e_2}{\xi_2}$ iff $\hpatmatch{e_2}{p_2}{\theta_2}$ for some $\theta_2$ \pflabel{[pair]true-equiv2}
        \item $\cmaysatisfy{e_2}{\xi_2}$ iff $\hmaymatch{e_2}{p_2}$ \pflabel{[pair]may-equiv2}
        \end{pfsteps*}
        
        \begin{enumerate}
        \savelocalsteps{2}
        \item Prove $\csatisfy{\hpair{e_1}{e_2}}{\cpair{\xi_1}{\xi_2}}$ implies $\hpatmatch{\hpair{e_1}{e_2}}{\hpair{p_1}{p_2}}{\theta}$ for some $\theta$.
            \begin{pfsteps*}
            \item $\csatisfy{\hpair{e_1}{e_2}}{\cpair{\xi_1}{\xi_2}}$ \BY{assumption} \pflabel{[pair1]satisfy}
            \end{pfsteps*}
            By rule induction over \rulesref{rules:Satisfy} on \pfref{[pair1]satisfy}, only two cases apply.
            \begin{byCases}
            \savelocalsteps{3}
            \item[\text{(\ref{rule:CSPair})}]
                \begin{pfsteps*}
                \item $\csatisfy{e_1}{\xi_1}$ \BY{assumption} \pflabel{[pair1]satisfy1}
                \item $\csatisfy{e_2}{\xi_2}$ \BY{assumption} \pflabel{[pair1]satisfy2}
                \item $\hpatmatch{e_1}{p_1}{\theta_1}$ for some $\theta_1$ \BY{\pfref{[pair]true-equiv1} on \pfref{[pair1]satisfy1}} \pflabel{[pair1]match1}
                \item $\hpatmatch{e_2}{p_2}{\theta_2}$ for some $\theta_2$ \BY{\pfref{[pair]true-equiv2} on \pfref{[pair1]satisfy2}} \pflabel{[pair1]match2}
                \item $\hpatmatch{\hpair{e_1}{e_2}}{\hpair{p_1}{p_2}}{\theta_1\uplus\theta_2}$ \BY{\ruleref{rule:MPair} on \pfref{[pair1]match1} and \pfref{[pair1]match2}}
                \end{pfsteps*}
            \restorelocalsteps{3}
            \item[\text{(\ref{rule:CSNotIntroPair})}]
                \begin{pfsteps*}
                \item $\notIntro{\hpair{e_1}{e_2}}$ \BY{assumption}
                \end{pfsteps*}
                Contradicts \autoref{lem:no-pair-notintro}.
            \end{byCases}
        \restorelocalsteps{2}
        \item Prove $\hpatmatch{\hpair{e_1}{e_2}}{\hpair{p_1}{p_2}}{\theta}$ implies $\csatisfy{\hpair{e_1}{e_2}}{\cpair{\xi_1}{\xi_2}}$.
            \begin{pfsteps*}
            \item $\hpatmatch{\hpair{e_1}{e_2}}{\hpair{p_1}{p_2}}{\theta}$ \BY{assumption} \pflabel{[pair2]match}
            \end{pfsteps*}
            By rule induction over \rulesref{rules:match} on \pfref{[pair2]match}, only two cases apply.
            \begin{byCases}
            \savelocalsteps{3}
            \item[\text{(\ref{rule:MPair})}]
                \begin{pfsteps*}
                \item $\hpatmatch{e_1}{p_1}{\theta_1}$ for some $\theta_1$ \BY{assumption}  \pflabel{[pair2]match1}
                \item $\hpatmatch{e_2}{p_2}{\theta_2}$ for some $\theta_2$ \BY{assumption} \pflabel{[pair2]match2}
                \item $\csatisfy{e_1}{\xi_1}$ \BY{\pfref{[pair]true-equiv1} on \pfref{[pair2]match1}} \pflabel{[pair2]satisfy1}
                \item $\csatisfy{e_2}{\xi_2}$ \BY{\pfref{[pair]true-equiv2} on \pfref{[pair2]match2}} \pflabel{[pair2]satisfy2}
                \item $\csatisfy{\hpair{e_1}{e_2}}{\cpair{\xi_1}{\xi_2}}$ \BY{\ruleref{rule:CSPair} on \pfref{[pair2]satisfy1} and \pfref{[pair2]satisfy2}}
                \end{pfsteps*}
            \restorelocalsteps{3}
            \item[\text{(\ref{rule:MNotIntroPair})}]
                \begin{pfsteps*}
                \item $\notIntro{\hpair{e_1}{e_2}}$ \BY{assumption}
                \end{pfsteps*}
                Contradicts \autoref{lem:no-pair-notintro}.
            \end{byCases}
        \restorelocalsteps{2}
        \item Prove $\cmaysatisfy{\hpair{e_1}{e_2}}{\cpair{\xi_1}{\xi_2}}$ implies $\hmaymatch{\hpair{e_1}{e_2}}{\hpair{p_1}{p_2}}$.
            \begin{pfsteps*}
            \item $\cmaysatisfy{\hpair{e_1}{e_2}}{\cpair{\xi_1}{\xi_2}}$ \BY{assumption} \pflabel{[pair3]maysat}
            \end{pfsteps*}
            By rule induction over \rulesref{rules:MaySatisfy} on \pfref{[pair3]maysat}, only four cases apply.
            \begin{byCases}
            \savelocalsteps{3}
            \item[\text{(\ref{rule:CMSNotIntro})}]
                \begin{pfsteps*}
                \item $\notIntro{\hpair{e_1}{e_2}}$ \BY{assumption}
                \end{pfsteps*}
                Contradicts \autoref{lem:no-pair-notintro}.
            \restorelocalsteps{3}
            \item[\text{(\ref{rule:CMSPair1})}]
                \begin{pfsteps*}
                \item $\cmaysatisfy{e_1}{\xi_1}$ \BY{assumption} \pflabel{[pair3]maysat1}
                \item $\csatisfy{e_2}{\xi_2}$ \BY{assumption} \pflabel{[pair3]satisfy2}
                \item $\hmaymatch{e_1}{p_1}$ \BY{\pfref{[pair]may-equiv1} on \pfref{[pair3]maysat1}} \pflabel{[pair3]maymatch1}
                \item $\hpatmatch{e_2}{p_2}{\theta_2}$ \BY{\pfref{[pair]true-equiv2} on \pfref{[pair3]satisfy2}} \pflabel{[pair3]match2}
                \item $\hmaymatch{\hpair{e_1}{e_2}}{\hpair{p_1}{p_2}}$ \BY{\ruleref{rule:MMPairL} on \pfref{[pair3]maymatch1} and \pfref{[pair3]match2}}
                \end{pfsteps*}
            \restorelocalsteps{3}
            \item[\text{(\ref{rule:CMSPair2})}]
                \begin{pfsteps*}
                \item $\csatisfy{e_1}{\xi_1}$ \BY{assumption} \pflabel{[pair3]satisfy1}
                \item $\cmaysatisfy{e_2}{\xi_2}$ \BY{assumption} \pflabel{[pair3]maysat2}
                \item $\hpatmatch{e_1}{p_1}{\theta_1}$ \BY{\pfref{[pair]true-equiv1} on \pfref{[pair3]satisfy1}} \pflabel{[pair3]match1}
                \item $\hmaymatch{e_2}{p_2}$ \BY{\pfref{[pair]may-equiv2} on \pfref{[pair3]maysat2}} \pflabel{[pair3]maymatch2}
                \item $\hmaymatch{\hpair{e_1}{e_2}}{\hpair{p_1}{p_2}}$ \BY{\ruleref{rule:MMPairR} on \pfref{[pair3]match1} and \pfref{[pair3]maymatch2}}
                \end{pfsteps*}
            \restorelocalsteps{3}
            \item[\text{(\ref{rule:CMSPair3})}]
                \begin{pfsteps*}
                \item $\cmaysatisfy{e_1}{\xi_1}$ \BY{assumption} \pflabel{[pair3]maysat1'}
                \item $\cmaysatisfy{e_2}{\xi_2}$ \BY{assumption} \pflabel{[pair3]maysat2'}
                \item $\hmaymatch{e_1}{p_1}$ \BY{\pfref{[pair]may-equiv1} on \pfref{[pair3]maysat1'}} \pflabel{[pair3]maymatch1'}
                \item $\hmaymatch{e_2}{p_2}$ \BY{\pfref{[pair]may-equiv2} on \pfref{[pair3]maysat2'}} \pflabel{[pair3]maymatch2'}
                \item $\hmaymatch{\hpair{e_1}{e_2}}{\hpair{p_1}{p_2}}$ \BY{\ruleref{rule:MMPair} on \pfref{[pair3]maymatch1'} and \pfref{[pair3]maymatch2'}}
                \end{pfsteps*}
            \end{byCases}
        \restorelocalsteps{2}
        \item Prove $\hmaymatch{\hpair{e_1}{e_2}}{\hpair{p_1}{p_2}}$ implies $\cmaysatisfy{\hpair{e_1}{e_2}}{\cpair{\xi_1}{\xi_2}}$.
            \begin{pfsteps*}
            \item $\hmaymatch{\hpair{e_1}{e_2}}{\hpair{p_1}{p_2}}$ \BY{assumption} \pflabel{[pair4]maymatch}
            \end{pfsteps*}
            By rule induction over \rulesref{rules:maymatch} on \pfref{[pair4]maymatch}, only four cases apply.
            \begin{byCases}
            \savelocalsteps{3}
            \item[\text{(\ref{rule:MMNotIntro})}]
                \begin{pfsteps*}
                \item $\notIntro{\hpair{e_1}{e_2}}$ \BY{assumption}
                \end{pfsteps*}
                Contradicts \autoref{lem:no-pair-notintro}.
            \restorelocalsteps{3}
            \item[\text{(\ref{rule:MMPairL})}]
                \begin{pfsteps*}
                \item $\hmaymatch{e_1}{p_1}$ \BY{assumption} \pflabel{[pair4]maymatch1}
                \item $\hpatmatch{e_2}{p_2}{\theta_2}$ \BY{assumption} \pflabel{[pair4]match2}
                \item $\cmaysatisfy{e_1}{\xi_1}$ \BY{\pfref{[pair]may-equiv1} on \pfref{[pair4]maymatch1}} \pflabel{[pair4]maysat1}
                \item $\csatisfy{e_2}{\xi_2}$ \BY{\pfref{[pair]true-equiv2} on \pfref{[pair4]match2}} \pflabel{[pair4]satisfy2}
                \item $\hmaymatch{\hpair{e_1}{e_2}}{\hpair{p_1}{p_2}}$ \BY{\ruleref{rule:CMSPair1} on \pfref{[pair4]maysat1} and \pfref{[pair4]satisfy2}}
                \end{pfsteps*}
            \restorelocalsteps{3}
            \item[\text{(\ref{rule:MMPairR})}]
                \begin{pfsteps*}
                \item $\hpatmatch{e_1}{p_1}{\theta_1}$ \BY{assumption} \pflabel{[pair4]match1}
                \item $\hmaymatch{e_2}{p_2}$ \BY{assumption} \pflabel{[pair4]maymatch2}
                \item $\csatisfy{e_1}{\xi_1}$ \BY{\pfref{[pair]true-equiv1} on \pfref{[pair4]match1}} \pflabel{[pair4]satisfy1}
                \item $\cmaysatisfy{e_2}{\xi_2}$ \BY{\pfref{[pair]may-equiv2} on \pfref{[pair4]maymatch2}} \pflabel{[pair4]maysat2}
                \item $\hmaymatch{\hpair{e_1}{e_2}}{\hpair{p_1}{p_2}}$ \BY{\ruleref{rule:CMSPair2} on \pfref{[pair4]satisfy1} and \pfref{[pair4]maysat2}}
                \end{pfsteps*}
            \restorelocalsteps{3}
            \item[\text{(\ref{rule:MMPair})}]
                \begin{pfsteps*}
                \item $\hmaymatch{e_1}{p_1}$ \BY{assumption} \pflabel{[pair4]maymatch1'}
                \item $\hmaymatch{e_2}{p_2}$ \BY{assumption} \pflabel{[pair4]maymatch2'}
                \item $\cmaysatisfy{e_1}{\xi_1}$ \BY{\pfref{[pair]may-equiv1} on \pfref{[pair4]maymatch1'}} \pflabel{[pair4]maysat1'}
                \item $\cmaysatisfy{e_2}{\xi_2}$ \BY{\pfref{[pair]may-equiv2} on \pfref{[pair4]maymatch2'}} \pflabel{[pair4]maysat2'}
                \item $\hmaymatch{\hpair{e_1}{e_2}}{\hpair{p_1}{p_2}}$ \BY{\ruleref{rule:CMSPair3} on \pfref{[pair4]maysat1'} and \pfref{[pair4]maysat2'}}
                \end{pfsteps*}
            \end{byCases}
        \end{enumerate}
    \end{byCases}
\end{byCases}
\resetpfcounter
\end{proof}
\section{Static Semantics}
$\arraycolsep=4pt\begin{array}{lll}
\tau & ::= &
  \tnum ~\vert~
  \tarr{\tau_1}{\tau_2} ~\vert~
  \tprod{\tau_1}{\tau_2} ~\vert~
  \tsum{\tau_1}{\tau_2} \\
e & ::= &
  x ~\vert~
  \hnum{n} \\
  & ~\vert~ &
  \hlam{x}{\tau}{e} ~\vert~
  \hap{e_1}{e_2} \\
  & ~\vert~ &
  \hpair{e_1}{e_2} \\
  & ~\vert~ &
  \hinl{\tau}{e} ~\vert~
  \hinr{\tau}{e} ~\vert~
  \hmatch{e}{\hat{rs}} \\
  & ~\vert~ &
  \hehole{u} ~\vert~
  \hhole{e}{u} \\
\hat{rs} & ::= &
  \inparens{\zruls{rs}{r}{rs}} \\
rs & ::= &
  \cdot ~\vert~ \hrulesP{r}{rs'} \\
r & ::= &
  \hrul{p}{e} \\
p & ::= &
  x ~\vert~
  \hnum{n} ~\vert~
  \_ ~\vert~
  \hpair{p_1}{p_2} ~\vert~
  \hinlp{p} ~\vert~
  \hinrp{p} ~\vert~
  \hehole{w} ~\vert~
  \hhole{p}{w}
\end{array}$

\judgboxa{\rmpointer{\zrules} = rs}
        {$rs$ can be obtained by erasing pointer from $\zrules$}
\begin{subequations}\label{defn:rmpointer}
\begin{align}
  \rmpointer{\zruls{\cdot}{r}{rs}} &= \hrules{r}{rs} \\
  \rmpointer{\zruls{\hrulesP{r'}{rs'}}{r}{rs}} &= \hrules{r'}{\rmpointer{\zruls{rs'}{r}{rs}}}
\end{align}
\end{subequations}

\judgboxa{
  \hexptyp{\Gamma}{\Delta}{e}{\tau}
}{
  $e$ is of type \(\tau\)
}
\begin{subequations}\label{rules:TExp}
\begin{equation}\label{rule:TVar}
\inferrule[TVar]{ }{
  \hexptyp{\Gamma, x : \tau}{\Delta}{x}{\tau}
}
\end{equation}
\begin{equation}\label{rule:TEHole}
\inferrule[TEHole]{ }{
  \hexptyp{\Gamma}{\Delta, u::\tau}{\hehole{u}}{\tau}
}
\end{equation}
\begin{equation}\label{rule:THole}
\inferrule[THole]{
  \hexptyp{\Gamma}{\Delta, u::\tau}{e}{\tau'}
}{
  \hexptyp{\Gamma}{\Delta, u::\tau}{\hhole{e}{u}}{\tau}
}
\end{equation}
\begin{equation}\label{rule:TNum}
\inferrule[TNum]{ }{
  \hexptyp{\Gamma}{\Delta}{\hnum{n}}{\tnum}
}
\end{equation}
\begin{equation}\label{rule:TLam}
\inferrule[TLam]{
  \hexptyp{\Gamma, x : \tau_1}{\Delta}{e}{\tau_2}
}{
  \hexptyp{\Gamma}{\Delta}{\hlam{x}{\tau_1}{e}}{\tarr{\tau_1}{\tau_2}}
}
\end{equation}
\begin{equation}\label{rule:TAp}
\inferrule[TAp]{
  \hexptyp{\Gamma}{\Delta}{e_1}{\tarr{\tau_2}{\tau}} \\
  \hexptyp{\Gamma}{\Delta}{e_2}{\tau_2}
}{
  \hexptyp{\Gamma}{\Delta}{\hap{e_1}{e_2}}{\tau}
}
\end{equation}
\begin{equation}\label{rule:TPair}
\inferrule[TPair]{
  \hexptyp{\Gamma}{\Delta}{e_1}{\tau_1} \\
  \hexptyp{\Gamma}{\Delta}{e_2}{\tau_2}
}{
  \hexptyp{\Gamma}{\Delta}{\hpair{e_1}{e_2}}{\tprod{\tau_1}{\tau_2}}
}
\end{equation}
\begin{equation}\label{rule:TPrl}
\inferrule[TPrl]{
    \hexptyp{\Gamma}{\Delta}{e}{\tprod{\tau_1}{\tau_2}}
}{
    \hexptyp{\Gamma}{\Delta}{\hprl{e}}{\tau_1}
} 
\end{equation}
\begin{equation}\label{rule:TPrr}
  \inferrule[TPrr]{
    \hexptyp{\Gamma}{\Delta}{e}{\tprod{\tau_1}{\tau_2}}
  }{
    \hexptyp{\Gamma}{\Delta}{\hprr{e}}{\tau_2}
  }
\end{equation}
\begin{equation}\label{rule:TInl}
\inferrule[TInl]{
  \hexptyp{\Gamma}{\Delta}{e}{\tau_1}
}{
  \hexptyp{\Gamma}{\Delta}{\hinl{\tau_2}{e}}{\tsum{\tau_1}{\tau_2}}
}
\end{equation}
\begin{equation}\label{rule:TInr}
\inferrule[TInr]{
  \hexptyp{\Gamma}{\Delta}{e}{\tau_2}
}{
  \hexptyp{\Gamma}{\Delta}{\hinr{\tau_1}{e}}{\tsum{\tau_1}{\tau_2}}
}
\end{equation}
\begin{equation}\label{rule:TMatchZPre}
\inferrule[TMatchZPre]{
  \hexptyp{\Gamma}{\Delta}{e}{\tau} \\
  \chrulstyp{\Gamma}{\Delta}{\cfalsity}{\hrules{r}{rs}}{\tau}{\xi}{\tau'} \\
  \csatisfyormay{\ctruth}{\xi}
}{
\hexptyp{\Gamma}{\Delta}{\hmatch{e}{\zruls{\cdot}{r}{rs}}}{\tau'}
}
\end{equation}
\begin{equation}\label{rule:TMatchNZPre}
\inferrule[TMatchNZPre]{
  \hexptyp{\Gamma}{\Delta}{e}{\tau} \\
  \isFinal{e} \\
  \chrulstyp{\Gamma}{\Delta}{\cfalsity}{rs_{pre}}{\tau}{\xi_{pre}}{\tau'} \\
  \chrulstyp{\Gamma}{\Delta}{\cor{\cfalsity}{\xi_{pre}}}{\hrules{r}{rs_{post}}}{\tau}{\xi_{rest}}{\tau'} \\
  \cnotsatisfyormay{e}{\xi_{pre}} \\
  \csatisfyormay{\ctruth}{\cor{\xi_{pre}}{\xi_{rest}}}
}{
  \hexptyp{\Gamma}{\Delta}{\hmatch{e}{\zruls{rs_{pre}}{r}{rs_{post}}}}{\tau'}
}
\end{equation}
\end{subequations}

\judgboxa{
    \chpattyp{p}{\tau}{\xi}{\Gamma}{\Delta}
  }{
    $p$ is assigned type $\tau$ and emits constraint $\xi$
  }
\begin{subequations}\label{rules:PatTyp}
\begin{equation}\label{rule:PTVar}
\inferrule[PTVar]{ }{
  \chpattyp{x}{\tau}{\ctruth}{\cdot}{x : \tau}
}
\end{equation}
\begin{equation}\label{rule:PTWild}
\inferrule[PTWild]{ }{
  \chpattyp{\_}{\tau}{\ctruth}{\cdot}{\cdot}
}
\end{equation}
\begin{equation}\label{rule:PTEHole}
\inferrule[PTEHole]{ }{
  \chpattyp{\hehole{w}}{\tau}{\cunknown}{\cdot}{w :: \tau}
}
\end{equation}
\begin{equation}\label{rule:PTHole}
\inferrule[PTHole]{
  \chpattyp{p}{\tau}{\xi}{\Gamma}{\Delta}
}{
  \chpattyp{\hhole{p}{w}}{\tau'}{\cunknown}
  {\Gamma}{\Delta , w :: \tau'}
}
\end{equation}
\begin{equation}\label{rule:PTNum}
\inferrule[PTNum]{ }{
  \chpattyp{\hnum{n}}{\tnum}{\cnum{n}}{\cdot}{\cdot}
}
\end{equation}
\begin{equation}\label{rule:PTInl}
\inferrule[PTInl]{
  \chpattyp{p}{\tau_1}{\xi}{\Gamma}{\Delta}
}{
  \chpattyp{\hinlp{p}}{\tsum{\tau_1}{\tau_2}}{\cinl{\xi}}{\Gamma}{\Delta}
}
\end{equation}
\begin{equation}\label{rule:PTInr}
\inferrule[PTInr]{
  \chpattyp{p}{\tau_2}{\xi}{\Gamma}{\Delta}
}{
  \chpattyp{\hinrp{p}}{\tsum{\tau_1}{\tau_2}}{\cinr{\xi}}{\Gamma}{\Delta}
}
\end{equation}
\begin{equation}\label{rule:PTPair}
\inferrule[PTPair]{
  \chpattyp{p_1}{\tau_1}{\xi_1}{\Gamma_1}{\Delta_1} \\
  \chpattyp{p_2}{\tau_2}{\xi_2}{\Gamma_2}{\Delta_2}
}{
  \chpattyp{\hpair{p_1}{p_2}}{\tprod{\tau_1}{\tau_2}}
  {\cpair{\xi_1}{\xi_2}}{\Gamma_1 \uplus \Gamma_2}{\Delta_1 \uplus \Delta_2}
}
\end{equation}
\end{subequations}

\judgboxa{
\chrultyp{\Gamma}{\Delta}{\hrulP{p}{e}}{\tau}{\xi}{\tau'}
}{$r$ transforms a final expression of type $\tau$ \\ to a final expression of type $\tau'$}
\begin{subequations}\label{rules:CTRule}
\begin{equation}\label{rule:CTRule}
\inferrule[CTRule]{
    \chpattyp{p}{\tau}{\xi}{\Gamma_p}{\Delta_p} \\
    \hexptyp{\Gamma \uplus \Gamma_p}{\Delta \uplus \Delta_p}{e}{\tau'}
}{
  \chrultyp{\Gamma}{\Delta}{\hrul{p}{e}}{\tau}{\xi}{\tau'}
}
\end{equation}
\end{subequations}

\judgboxa{\chrulstyp{\Gamma}{\Delta}{\xi_{pre}}{rs}{\tau}{\xi_{rs}}{\tau'}}
{$rs$ transforms a final expression of type $\tau$ \\ to a final expression of type $\tau'$}
\begin{subequations}\label{rules:CTRules}
\begin{equation}\label{rule:CTOneRules}
\inferrule[CTOneRules]{
  \chrultyp{\Gamma}{\Delta}{r}{\tau}{\xi_r}{\tau'} \\
  \cnotsatisfy{\xi_r}{\xi_{pre}}
}{
  \chrulstyp{\Gamma}{\Delta}{\xi_{pre}}{\hrulesP{r}{\cdot}}{\tau}{\xi_r}{\tau'}
}
\end{equation}
\begin{equation}\label{rule:CTRules}
\inferrule[CTRules]{
  \chrultyp{\Gamma}{\Delta}{r}{\tau}{\xi_r}{\tau'} \\
  \chrulstyp{\Gamma}{\Delta}{\cor{\xi_{pre}}{\xi_r}}{rs}
  {\tau}{\xi_{rs}}{\tau'} \\
  \cnotsatisfy{\xi_r}{\xi_{pre}}
}{
  \chrulstyp{\Gamma}{\Delta}{\xi_{pre}}{\hrules{r}{rs}}
  {\tau}{\cor{\xi_r}{\xi_{rs}}}{\tau'}
}
\end{equation}
\end{subequations}

\begin{lemma}
  \label{lem:pat-xi-type}
  If $\chpattyp{p}{\tau}{\xi}{\Gamma}{\Delta}$ then $\ctyp{\xi}{\tau}$.
\end{lemma}
\begin{proof}
By rule induction over \rulesref{rules:PatTyp}.
\end{proof}

\begin{lemma}
  \label{lem:rule-constraint-typ}
  If $\chrultyp{\cdot}{\Delta}{r}{\tau_1}{\xi_r}{\tau}$ then $\ctyp{\xi_r}{\tau_1}$.
\end{lemma}
\begin{proof}
By rule induction over \rulesref{rules:CTRule}.
\end{proof}

\begin{lemma}
  \label{lem:rules-constraint-typ}
  If $\chrulstyp{\cdot}{\Delta}{\xi_{pre}}{rs}{\tau_1}{\xi_{rs}}{\tau}$ then $\ctyp{\xi_{rs}}{\tau_1}$.
\end{lemma}
\begin{proof}
By rule induction over \rulesref{rules:CTRules}.
\end{proof}

\begin{lemma}
  \label{lem:rule-append}
  If $\chrulstyp{\Gamma}{\Delta}{\xi_{pre}}{rs}{\tau}{\xi_{rs}}{\tau'}$ and $\chrultyp{\Gamma}{\Delta}{r}{\tau}{\xi_r}{\tau'}$ and $\cnotsatisfy{\xi_r}{\cor{\xi_{pre}}{\xi_{rs}}}$ then $\chrulstyp{\Gamma}{\Delta}{\xi_{pre}}{\rmpointer{\zruls{rs}{r}{\cdot}}}{\tau}{\cor{\xi_{rs}}{\xi_r}}{\tau'}$
\end{lemma}
\begin{proof}
  \begin{pfsteps*}
  \item $\chrulstyp{\Gamma}{\Delta}{\xi_{pre}}{rs}{\tau}{\xi_{rs}}{\tau'}$ \BY{assumption} \pflabel{rsType}
  \item $\chrultyp{\Gamma}{\Delta}{r}{\tau}{\xi_r}{\tau'}$ \BY{assumption} \pflabel{rType}
  \item $\cnotsatisfy{\xi_r}{\cor{\xi_{pre}}{\xi_{rs}}}$ \BY{assumption} \pflabel{r|/=pre+rs}
  \end{pfsteps*}
  By rule induction over Rules (\ref{rules:CTRules}) on \pfref{rsType}.
  \begin{byCases}
    
  \savelocalsteps{lem:rule-append-1}
  \item[\text{(\ref{rule:CTOneRules})}]
    \begin{pfsteps*}
    \item $rs = \hrules{r'}{\cdot}$ \BY{assumption}
    \item $\xi_{rs} = \xi_r'$ \BY{assumption}
    \item $\chrultyp{\Gamma}{\Delta}{r'}{\tau}{\xi_r'}{\tau'}$ \BY{assumption} \pflabel{[one]r'Type}
    \item $\cnotsatisfy{\xi_r'}{\xi_{pre}}$ \BY{assumption} \pflabel{r'|/=pre}
    \item $\chrulstyp{\Gamma}{\Delta}{\cor{\xi_{pre}}{\xi_r'}}{\hrules{r}{\cdot}}{\tau}{\xi_r}{\tau'}$ \BY{Rule (\ref{rule:CTOneRules}) on \pfref{rType} and \pfref{r|/=pre+rs}} \pflabel{r+dotType}
    \item $\chrulstyp{\Gamma}{\Delta}{\xi_{pre}}{\hrulesP{r'}{\hrules{r}{\cdot}}}{\tau}{\cor{\xi_r'}{\xi_r}}{\tau'}$ \BY{Rule (\ref{rule:CTRules}) on \pfref{r'Type} and \pfref{r+dotType} and \pfref{r'|/=pre}} \pflabel{[one]conc}
    \item $\chrulstyp{\Gamma}{\Delta}{\xi_{pre}}{\rmpointer{\zruls{\hrulesP{r'}{\cdot}}{r}{\cdot}}}{\tau}{\cor{\xi_r'}{\xi_r}}{\tau'}$ \BY{Definition \ref{defn:rmpointer} on \pfref{[one]conc}}
    \end{pfsteps*}

  \restorelocalsteps{lem:rule-append-1}
  \item[\text{(\ref{rule:CTRules})}]
    \begin{pfsteps*}
    \item $rs = \hrules{r'}{rs'}$ \BY{assumption}
    \item $\xi_{rs} = \cor{\xi_r'}{\xi_{rs}'}$ \BY{assumption}
    \item $\chrultyp{\Gamma}{\Delta}{r'}{\tau}{\xi_r'}{\tau'}$ \BY{assumption} \pflabel{r'Type}
    \item $\chrulstyp{\Gamma}{\Delta}{\cor{\xi_{pre}}{\xi_r'}}{rs'}{\tau}{\xi_{rs}'}{\tau'}$ \BY{assumption} \pflabel{rs'Type}
    \item $\cnotsatisfy{\xi_r'}{\xi_{pre}}$ \BY{assumption} \pflabel{prenotredundant}
    \item $\chrulstyp{\Gamma}{\Delta}{\cor{\xi_{pre}}{\xi_r'}}{\rmpointer{\zruls{rs'}{r}{\cdot}}}{\tau}{\cor{\xi_{rs}'}{\xi_r}}{\tau'}$ \BY{IH on \pfref{rs'Type} and \pfref{rType} and \pfref{r|/=pre+rs}} \pflabel{rs'+rType}
    \item $\chrulstyp{\Gamma}{\Delta}{\xi_{pre}}{\hrulesP{r'}{\rmpointer{\zruls{rs'}{r}{\cdot}}}}{\tau}{\cor{\xi_r'}{\cor{\xi_{rs}'}{\xi_r}}}{\tau'}$ \BY{Rule (\ref{rule:CTRules}) on \pfref{r'Type} and \pfref{rs'+rType} and \pfref{prenotredundant}} \pflabel{conc}
    \item $\chrulstyp{\Gamma}{\Delta}{\xi_{pre}}{\rmpointer{\zruls{\hrulesP{r'}{rs'}}{r}{\cdot}}}{\tau}{\cor{\xi_r'}{\cor{\xi_{rs}'}{\xi_r}}}{\tau'}$ \BY{Definition \ref{defn:rmpointer} on \pfref{conc}}
    \end{pfsteps*}
  \resetpfcounter
  \end{byCases}
\end{proof}

\begin{lemma}[Substitution]
  \label{lem:substitution}
  If $\hexptyp{\Gamma, x : \tau}{\Delta}{e_0}{\tau_0}$ and $\hexptyp{\Gamma}{\Delta}{e}{\tau}$
  then $\hexptyp{\Gamma}{\Delta}{[e/x]e_0}{\tau_0}$
\end{lemma}

\begin{lemma}[Simultaneous Substitution]
  \label{lem:simult-substitution}
  If $\hexptyp{\Gamma \uplus \Gamma'}{\Delta}{e}{\tau}$ and $\hsubstyp{\theta}{\Gamma'}$
  then $\hexptyp{\Gamma}{\Delta}{[\theta]e}{\tau}$
\end{lemma}

\begin{lemma}[Substitution Typing]
  \label{lem:subs-typing}
  If $\hpatmatch{e}{p}{\theta}$ and $\hexptyp{\cdot}{\Delta_e}{e}{\tau}$ and $\chpattyp{p}{\tau}{\xi}{\Gamma}{\Delta}$
  then $\hsubstyp{\theta}{\Gamma}$
\end{lemma}
Proof by induction on the derivation of $\hpatmatch{e}{p}{\theta}$.

\begin{theorem}[Determinism]
  \label{thrm:determinism}
  If $\hexptyp{\cdot}{\Delta}{e}{\tau}$ then exactly one of the following holds
  \begin{enumerate}
    \item $\isVal{e}$
    \item $\isIndet{e}$
    \item $\htrans{e}{e'}$ for some unique $e'$
  \end{enumerate}
\end{theorem}

\subsection{Decidability}
\label{sec:decidability}
 We have described the core calculus of Peanut and discussed its mechanization, but it is yet to be shown that the validity of a "fully known" constraint (\autoref{definition:valid-constraint}) is decidable.
\subsubsection{SAT Encoding}
 One approach is to reduce it to a boolean satisfiability problem (SAT). 
If we revisit the analogy between constraint and set of expressions discussed in \autoref{sec:constraint}, we can think of constraints as subsets of values of type $\tau$.
Then $\ccsatisfy{}{\xi}$ basically says that $\xi$ exactly represents the set of all values of type $\tau$. 
However, such set may be infinite (e.g. top constraint $\ctruth$),
and thus defining operations on such infinite sets is nontrivial. 

Nevertheless, we may use logical predicates to encode the subset of values corresponding to a constraint. 
For example, $\xi=\cnum{2}$ represents a set with one value $\hnum{2}$, and thus can be encoded as a predicate $x=2$. 
If there is any connectives ($\cand{}{}$ and $\cor{}{}$) in $\xi$, we can use the connectives of the same form in logical formula. 
It is tricky to encode $inl$, $inr$, and $pair$ constraint as predicate. 
If we think of a constraint as a set again, a value $e=\cpair{e_1}{e_2}$ belongs to $\xi=\cpair{\xi_1}{\xi_2}$ iff $e_1$ belongs to $\xi_1$ and $e_2$ belongs to $\xi_2$. 
Therefore, the logical encoding of $\cpair{\xi_1}{\xi_2}$ would be a conjunction of encoding of both side of the pair, with a variable for each.
For injection, we consider the encoding of $\cinl{\xi_1}$ without loss of generality. Because values of sum type are either injection left or right, we can use a boolean variable as a flag, and conjunct it with the encoding of $\xi_1$.

One last thing to notice here is that we need to make sure when transforming constraints on the same set of values into a predicate, the same variable would be used. To demonstrate how that might work, let's consider a more involved example:
\[ \cpair{\cinl{\cnum{1}}}{\cinl{\cnum{3}}} \vee \cpair{\cinl{\cnum{2}}}{\cinr{\cnum{1}}} \]
Both $\cpair{\cinl{\cnum{1}}}{\cinl{\cnum{3}}}$ and $\cpair{\cinl{\cnum{2}}}{\cinr{\cnum{1}}}$ place constraints on the same variable $x$.
Therefore, their left(right) side also place constraints on the same variable $x_l$ ($x_r$). 
We may use $x_l'$($x_r'$) to encode the number constraints under injections. 
That is we encode $\cinl{\cnum{1}}$ as $b_{x_l} \wedge (x_l'=1)$, 
$\cinl{\cnum{3}}$ as $b_{x_r} \wedge (x_r'=3)$,
$\cinl{\cnum{2}}$ as $b_{x_l} \wedge (x_l'=2)$,
$\cinr{\cnum{1}}$ as $\neg b_{x_r} \wedge (x_r'=1)$. 
Put them together and we get the logical encoding of the entire constraint, 
\[
(b_{x_l} \wedge (x_l'=1) \wedge
b_{x_r} \wedge (x_r'=3))
\vee
(b_{x_l} \wedge (x_l'=2) \wedge
\neg b_{x_r} \wedge (x_r'=1))
\]

As a result, the validity of a constraint $\xi$, $\ccsatisfy{}{\xi}$, is equivalent to the validity of its logical encoding. Exhaustiveness and redundancy checking are reduced to boolean satisfiability problem, which is NP-complete but decidable, and several tools exist for doing so. For handling numeric patterns we only need SAT modulo numeric equality and disequality.

\subsection{Constraint Inconsistency Checking}

Using an SMT solver to decide constraint entailments is, however, overkill. Moreover, it may incur run-time and space overhead in a  development environment. When incorporating Peanut into Hazel, we use a different but more lightweight decision procedure. \figurename~\ref{fig:incon} describes such procedure by defining a new judgment $\cincon{\Xi}$. Assuming constraint $\xi$ is fully known and is of type $\tau$, $\cincon{\xi}$ means constraint $\xi$ is inconsistent in the sense that no values of type $\tau$ satisfy $\xi$, which corresponds to the insatisfiability of $\xi$'s logical encoding. Therefore, a constraint is valid, written as $\ccsatisfy{}{\xi}$, iff its dual is inconsistent, written as $\cincon{\cdual{\xi}}$. Note that this is not fully mechanized in Agda. Such proofs require reasoning about finite sets in a non-structurally recursive way, making them inordinately difficult to verify in Agda, but our implementation uses this algorithm.

% !TEX root= pattern-paper.tex

\begin{figure}[bp]
\judgbox{\cincon{\Xi}}{}

\begin{mathpar}
\Infer{\CINCTruth}{
  \cincon{\Xi}
}{
  \cincon{\Xi, \ctruth}
}

\Infer{\CINCFalsity}{ }{
  \cincon{\Xi, \cfalsity}
}

\Infer{\CINCNum}{
  n_1 \neq n_2
}{
  \cincon{\Xi, \cnum{n_1}, \cnum{n_2}}
}

\Infer{\CINCNotNum}{ }{
  \cincon{\Xi, \cnum{n}, \cnotnum{n}}
}

\Infer{\CINCAnd}{
  \cincon{\Xi, \xi_1, \xi_2}
}{
  \cincon{\Xi, \cand{\xi_1}{\xi_2}}
}

\Infer{\CINCOr}{
  \cincon{\Xi, \xi_1} \\
  \cincon{\Xi, \xi_2}
}{
  \cincon{\Xi, \cor{\xi_1}{\xi_2}}
}

\Infer{\CINCInj}{ }{
  \cincon{\Xi, \cinl{\xi_1}, \cinr{\xi_2}}
}

\Infer{\CINCInl}{
  \cincon{\setof{\xi' | \cinl{\xi'} \in \Xi},\xi}
}{
  \cincon{\Xi, \cinl{\xi}}
}

\Infer{\CINCInr}{
  \cincon{\setof{\xi' | \cinr{\xi'} \in \Xi},\xi}
}{
  \cincon{\Xi, \cinr{\xi}}
}

\Infer{\CINCPairL}{
    \cincon{\setof{\xi_1' | \cpair{\xi_1'}{\xi_2'} \in \Xi},\xi_1}
}{
    \cincon{\Xi, \cpair{\xi_1}{\xi_2}}
}

\Infer{\CINCPairR}{
    \cincon{\setof{\xi_2' | \cpair{\xi_1'}{\xi_2'} \in \Xi},\xi_2}
}{
    \cincon{\Xi, \cpair{\xi_1}{\xi_2}}
}
\end{mathpar}

  \caption{Inconsistency of Constraints}
  \label{fig:incon}
\end{figure}


%\begin{theorem}
%\textbf{}  Assume $\ctruify{\xi}=\xi$. It is decidable whether $\cincon{\xi}$.
%\end{theorem}
%
%\begin{theorem}
%  Assume $\ctruify{\xi}=\xi$. Then $\cincon{\cdual{\xi}}$ iff $\csatisfy{\ctruth}{\xi}$.
%\end{theorem}

\end{document}
%%% Local Variables:
%%% mode: latex
%%% TeX-master: t
%%% End:
