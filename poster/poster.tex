\documentclass[notheorems]{beamer}
\usepackage[orientation=landscape,size=a0,scale=1.4]{beamerposter}
\usepackage[absolute,overlay]{textpos}

\usepackage{amsmath,amsthm}

\usepackage{stmaryrd} % llparenthesis
\usepackage{anyfontsize} % workaround for font size difference warning

\usepackage{cancel} % slash over symbol

\newcommand{\highlight}[1]{\colorbox{yellow}{$\displaystyle #1$}}

\newcommand{\todo}[1]{{\color{red} TODO: #1}}

%% Joshua Dunfield macros
\def\OPTIONConf{1}
\usepackage{jdunfield}
\usepackage{rulelinks} % hyperlink of rule name

\newtheoremstyle{slplain}% name
  {.15\baselineskip\@plus.1\baselineskip\@minus.1\baselineskip}% Space above
  {.15\baselineskip\@plus.1\baselineskip\@minus.1\baselineskip}% Space below
  {\slshape}% Body font
  {\parindent}%Indent amount (empty = no indent, \parindent = para indent)
  {\bfseries}%  Thm head font
  {.}%       Punctuation after thm head
  { }%      Space after thm head: " " = normal interword space;
       %       \newline = linebreak
  {}%       Thm head spec
\theoremstyle{slplain}
\newtheorem{thm}{Theorem}  % Numbered with the equation counter
\numberwithin{thm}{section}
\newtheorem{defn}[thm]{Definition}
\newtheorem{lem}[thm]{Lemma}
\newtheorem{prop}[thm]{Proposition}
\newtheorem{corol}[thm]{Corollary}

% BEAMER
\definecolor{hazelgreen}{RGB}{7,63,36}
\definecolor{hazellightgreen}{RGB}{103,138,97}
\definecolor{hazelyellow}{RGB}{245,222,179}
\definecolor{hazellightyellow}{RGB}{254,254,234}

\definecolor{green}{RGB}{132, 194, 70}
\definecolor{yellow}{RGB}{227, 202, 43}
\definecolor{red}{RGB}{249,130,147}

\setbeamercolor{background canvas}{bg=hazellightyellow}
\setbeamercolor{itemize item}{fg=hazelgreen}
\setbeamercolor{block title}{fg=white,bg=hazellightgreen}
\setbeamercolor{block body}{fg=black,bg=white}

\newlength{\sepwid}
\newlength{\onecolwid}
\newlength{\twocolwid}
\setlength{\sepwid}{0.025\paperwidth}
\setlength{\onecolwid}{0.3\paperwidth}
\setlength{\twocolwid}{0.625\paperwidth}

\usepackage{xcolor}
\usepackage{graphicx}
\usepackage{listings}
\usepackage{tikz}
\usetikzlibrary{calc,fit,tikzmark,plotmarks,arrows.meta,positioning}
\usetikzmarklibrary{listings}

\lstset{
  language=caml,
  breaklines=true,
       showspaces=false,                          % show spaces (with underscores)
     showstringspaces=false,            % underline spaces within strings
     showtabs=false,                            % show tabs using underscores
     tabsize=4,                     % default tabsize
     columns=fullflexible,
     breakautoindent=false,
     framerule=1pt,
     xleftmargin=0pt,
     xrightmargin=0pt,
     breakindent=0pt,
     resetmargins=true
  }

\tikzset{
  next page=below,
%  every highlighter/.style={rounded corners},
  line/.style={
    draw,
    rounded corners=3pt,
    -latex
  },
  balloon/.style={
    draw,
    fill=red!20,
    opacity=0.4,
    inner sep=4pt,
    inner ysep=0pt,
    rounded corners=2pt,
    shift={(0,-0.5)}
  },
  comment/.style={
    draw,
    fill=blue!70,
    text=white,
%    text width=3cm,
%    minimum height=1cm,
    rounded corners,
    drop shadow,
    align=left,
%    font=\scriptsize
  },
}

\newcommand\balloon[4]{%
  \pgfmathtruncatemacro\firstline{%
    #3-1
  }%
  \iftikzmark{line-#2-\firstline-start}{%
    \iftikzmark{line-#2-#3-first}{%
      \xdef\blines{({pic cs:line-#2-\firstline-start} -| {pic           cs:line-#2-#3-first})}%
    }{%
      \iftikzmark{line-#2-#3-start}{%
        \xdef\blines{({pic cs:line-#2-\firstline-start} -| {pic             cs:line-#2-#3-start})}%
      }{%
        \xdef\blines{(pic cs:line-#2-\firstline-start)}%
      }%
    }%
  }{%
    \xdef\blines{}%
  }%
  \foreach \k in {#3,...,#4} {%
    \iftikzmark{line-#2-\k-first}{%
      \xdef\blines{\blines (pic cs:line-#2-\k-first) }
    }{}
    \iftikzmark{line-#2-\k-end}{%
      \xdef\blines{\blines (pic cs:line-#2-\k-end) }
    }{}
  }%
  \ifx\blines\empty
  \else
  \edef\temp{\noexpand\tikz[remember picture,overlay]\noexpand\node[fit={\blines},balloon] (#1) {};}%
\temp
  \fi
}

\title{Pattern Matching with Typed Holes}
\author{Yongwei Yuan}
\institute{University of Michigan}

% !TEX root = ./patterns-paper.tex

% reverse Vdash
\newcommand{\dashV}{\mathbin{\rotatebox[origin=c]{180}{$\Vdash$}}}

% Violet hotdogs; highlight color helps distinguish them
\newcommand{\llparenthesiscolor}{\textcolor{violet}{\llparenthesis}}
\newcommand{\rrparenthesiscolor}{\textcolor{violet}{\rrparenthesis}}

% HTyp and HExp
\newcommand{\hcomplete}[1]{#1~\mathsf{complete}}

% HTyp
\newcommand{\htau}{\dot{\tau}}
\newcommand{\tarr}[2]{\inparens{#1 \rightarrow #2}}
\newcommand{\tarrnp}[2]{#1 \rightarrow #2}
\newcommand{\trul}[2]{\inparens{#1 \Rightarrow #2}}
\newcommand{\tnum}{\mathtt{num}}
\newcommand{\tehole}{\llparenthesiscolor\rrparenthesiscolor}
\newcommand{\tsum}[2]{\inparens{{#1} + {#2}}}
\newcommand{\tprod}[2]{\inparens{{#1} \times {#2}}}
\newcommand{\tunit}{\mathtt{1}}
\newcommand{\tvoid}{\mathtt{0}}

\newcommand{\tcompat}[2]{#1 \sim #2}
\newcommand{\tincompat}[2]{#1 \nsim #2}

% HExp
\newcommand{\hexp}{\dot{e}}
\newcommand{\hlam}[3]{\inparens{\lambda #1:#2.#3}}
\newcommand{\hap}[2]{#1(#2)}
\newcommand{\hapP}[2]{(#1)~(#2)} % Extra paren around function term
\newcommand{\hnum}[1]{\underline{#1}}
\newcommand{\hadd}[2]{\inparens{#1 + #2}}
\newcommand{\hpair}[2]{\inparens{#1 , #2}}
\newcommand{\htriv}{()}
\newcommand{\hehole}{\llparenthesiscolor\rrparenthesiscolor}
\newcommand{\hhole}[1]{\llparenthesiscolor#1\rrparenthesiscolor}
\newcommand{\hindet}[1]{\lceil#1\rceil}
\newcommand{\hinj}[2]{\mathtt{inj}_{#1}({#2})}
\newcommand{\hinl}[2]{\mathtt{inl}_{#1}({#2})}
\newcommand{\hinr}[2]{\mathtt{inr}_{#1}({#2})}
\newcommand{\hinlp}[1]{\mathtt{inl}(#1)}
\newcommand{\hinrp}[1]{\mathtt{inr}(#1)}
\newcommand{\hmatch}[2]{\mathtt{match}(#1) \{#2\}}
\newcommand{\hcase}[5]{\mathtt{case}({#1},{#2}.{#3},{#4}.{#5})}
\newcommand{\hrules}[2]{\inparens{#1 \mid #2}}
\newcommand{\hrul}[2]{#1 \Rightarrow #2}

\newcommand{\hGamma}{\dot{\Gamma}}
\newcommand{\domof}[1]{\text{dom}(#1)}
\newcommand{\hsyn}[3]{#1 \vdash #2 \Rightarrow #3}
\newcommand{\hana}[3]{#1 \vdash #2 \Leftarrow #3}
\newcommand{\hexptyp}[3]{#1 \vdash #2 : #3}
\newcommand{\hpattyp}[3]{#1 : #2 \dashV #3}
\newcommand{\hpatmatch}[3]{#1 \vartriangleright #2 \dashV #3}
\newcommand{\hval}[1]{#1 ~\mathtt{val}}
\newcommand{\herr}[1]{#1 ~\mathtt{err}}

% ZTyp and ZExp
\newcommand{\zlsel}[1]{{\bowtie}{#1}}
\newcommand{\zrsel}[1]{{#1}{\bowtie}}
\newcommand{\zwsel}[1]{
  \setlength{\fboxsep}{0pt}
  \colorbox{green!10!white!100}{
    \ensuremath{{{\textcolor{Green}{{\hspace{-2px}\triangleright}}}}{#1}{\textcolor{Green}{\triangleleft{\vphantom{\tehole}}}}}}
}

\newcommand{\removeSel}[1]{#1^{\diamond}}

% ZTyp
\newcommand{\ztau}{\hat{\tau}}

% ZExp
\newcommand{\zexp}{\hat{e}}

% Direction
\newcommand{\dParent}{\mathtt{parent}}
\newcommand{\dChildn}[1]{\mathtt{child}~\mathtt{{#1}}}
\newcommand{\dChildnm}[1]{\mathtt{child}~{#1}}

% Action
\newcommand{\aMove}[1]{\mathtt{move}~#1}
	\newcommand{\zrightmost}[1]{\mathsf{rightmost}(#1)}
	\newcommand{\zleftmost}[1]{\mathsf{leftmost}(#1)}
\newcommand{\aSelect}[1]{\mathtt{sel}~#1}
\newcommand{\aDel}{\mathtt{del}}
\newcommand{\aReplace}[1]{\mathtt{replace}~#1}
\newcommand{\aConstruct}[1]{\mathtt{construct}~#1}
\newcommand{\aConstructx}[1]{#1}
\newcommand{\aFinish}{\mathtt{finish}}

\newcommand{\performAna}[5]{#1 \vdash #2 \xlongrightarrow{#4} #5 \Leftarrow #3}
\newcommand{\performAnaI}[5]{#1 \vdash #2 \xlongrightarrow{#4}\hspace{-3px}{}^{*}~ #5 \Leftarrow #3}
\newcommand{\performSyn}[6]{#1 \vdash #2 \Rightarrow #3 \xlongrightarrow{#4} #5 \Rightarrow #6}
\newcommand{\performSynI}[6]{#1 \vdash #2 \Rightarrow #3 \xlongrightarrow{#4}\hspace{-3px}{}^{*}~ #5 \Rightarrow #6}
\newcommand{\performTyp}[3]{#1 \xlongrightarrow{#2} #3}
\newcommand{\performTypI}[3]{#1 \xlongrightarrow{#2}\hspace{-3px}{}^{*}~#3}

\newcommand{\performMove}[3]{#1 \xlongrightarrow{#2} #3}
\newcommand{\performDel}[2]{#1 \xlongrightarrow{\aDel} #2}

% Form
\newcommand{\farr}{\mathtt{arrow}}
\newcommand{\fnum}{\mathtt{num}}
\newcommand{\fsum}{\mathtt{sum}}

\newcommand{\fasc}{\mathtt{asc}}
\newcommand{\fvar}[1]{\mathtt{var}~#1}
\newcommand{\flam}[1]{\mathtt{lam}~#1}
\newcommand{\fap}{\mathtt{ap}}
% \newcommand{\farg}{\mathtt{arg}}
\newcommand{\fnumlit}[1]{\mathtt{lit}~#1}
\newcommand{\fplus}{\mathtt{plus}}
\newcommand{\fhole}{\mathtt{hole}}
\newcommand{\fnehole}{\mathtt{nehole}}

\newcommand{\finj}[1]{\mathtt{inj}~#1}
\newcommand{\fcase}[2]{\mathtt{case}~#1~#2}

% Talk about formal rules in example
\newcommand{\refrule}[1]{\textrm{Rule~(#1)}}

\newcommand{\herase}[1]{\left|#1\right|_\textsf{erase}}

\newcommand{\arrmatch}[2]{#1 \blacktriangleright_{\rightarrow} #2}


\newcommand{\TABperformAna}[5]{#1 \vdash & #2                & \xlongrightarrow{#4} & #5 & \Leftarrow #3}
\newcommand{\TABperformSyn}[6]{#1 \vdash & #2 \Rightarrow #3 & \xlongrightarrow{#4} & #5 \Rightarrow #6}
\newcommand{\TABperformTyp}[3]{& #1 & \xlongrightarrow{#2} & #3}

\newcommand{\TABperformMove}[3]{#1 & \xlongrightarrow{#2} & #3}
\newcommand{\TABperformDel}[2]{#1 \xlongrightarrow{\aDel} #2}

\newcommand{\sumhasmatched}[2]{#1 \mathrel{\textcolor{black}{\blacktriangleright_{+}}} #2}

\newcommand{\subminsyn}[1]{\mathsf{submin}_{\Rightarrow}(#1)}
\newcommand{\subminana}[1]{\mathsf{submin}_{\Leftarrow}(#1)}


\newcommand{\inparens}[1]{{\color{gray}(}#1{\color{gray})}}

%% rule names for appendix
\newcommand{\rname}[1]{\textsc{#1}}
\newcommand{\gap}{\vspace{7pt}}


% !TEX root = pattern-paper.tex
%% Statics

% typing of internal expressions
\newrulecommand{TVar}{TVar}
\newrulecommand{TNum}{TNum}
\newrulecommand{TLam}{TLam}
\newrulecommand{TAp}{TAp}
\newrulecommand{TPair}{TPair}
\newrulecommand{TPrl}{TPrl}
\newrulecommand{TPrr}{TPrr}
\newrulecommand{TInl}{TInl}
\newrulecommand{TInr}{TInr}
\newrulecommand{TMatchZPre}{TMatchZPre}
\newrulecommand{TMatchNZPre}{TMatchNZPre}
\newrulecommand{TEHole}{TEHole}
\newrulecommand{THole}{THole}

% typing of rule and rules
\newrulecommand{TRule}{TRule}
\newrulecommand{TOneRules}{TOneRules}
\newrulecommand{TRules}{TRules}

% typing of pattern
\newrulecommand{PTVar}{PTVar}
\newrulecommand{PTNum}{PTNum}
\newrulecommand{PTWild}{PTWild}
\newrulecommand{PTPair}{PTPair}
\newrulecommand{PTInl}{PTInl}
\newrulecommand{PTInr}{PTInr}
\newrulecommand{PTEHole}{PTEHole}
\newrulecommand{PTHole}{PTHole}

%% Dynamics

% typing of substitution
\newrulecommand{STEmpty}{STEmpty}
\newrulecommand{STExtend}{STExtend}

% refutable pattern
\newrulecommand{RNum}{RNum}
\newrulecommand{REHole}{REHole}
\newrulecommand{RHole}{RHole}
\newrulecommand{RInl}{RInl}
\newrulecommand{RInr}{RInr}
\newrulecommand{RPairL}{RPairL}
\newrulecommand{RPairR}{RPairR}

% match
\newrulecommand{MVar}{MVar}
\newrulecommand{MNum}{MNum}
\newrulecommand{MWild}{MWild}
\newrulecommand{MPair}{MPair}
\newrulecommand{MEHolePair}{MEHolePair}
\newrulecommand{MHolePair}{MHolePair}
\newrulecommand{MApPair}{MApPair}
\newrulecommand{MMatchPair}{MMatchPair}
\newrulecommand{MPrlPair}{MPrlPair}
\newrulecommand{MPrrPair}{MPrrPair}
\newrulecommand{MInl}{MInl}
\newrulecommand{MInr}{MInr}

% may match
\newrulecommand{MMEHole}{MMEHole}
\newrulecommand{MMHole}{MMHole}
\newrulecommand{MMExpEHole}{MMExpEHole}
\newrulecommand{MMExpHole}{MMExpHole}
\newrulecommand{MMAp}{MMAp}
\newrulecommand{MMMatch}{MMMatch}
\newrulecommand{MMPrl}{MMPrl}
\newrulecommand{MMPrr}{MMPrr}
\newrulecommand{MMPairL}{MMPairL}
\newrulecommand{MMPairR}{MMPairR}
\newrulecommand{MMPair}{MMPair}
\newrulecommand{MMInl}{MMInl}
\newrulecommand{MMInr}{MMInr}

% not match
\newrulecommand{NMNum}{NMNum}
\newrulecommand{NMPairL}{NMPairL}
\newrulecommand{NMPairR}{NMPairR}
\newrulecommand{NMConfL}{NMConfL}
\newrulecommand{NMConfR}{NMConfR}
\newrulecommand{NMInl}{NMInl}
\newrulecommand{NMInr}{NMInr}

% value
\newrulecommand{VNum}{VNum}
\newrulecommand{VLam}{VLam}
\newrulecommand{VPair}{VPair}
\newrulecommand{VInl}{VInl}
\newrulecommand{VInr}{VInr}

% indeterminate
\newrulecommand{IEHole}{IEHole}
\newrulecommand{IHole}{IHole}
\newrulecommand{IAp}{IAp}
\newrulecommand{IPairL}{IPairL}
\newrulecommand{IPairR}{IPairR}
\newrulecommand{IPair}{IPair}
\newrulecommand{IPrl}{IPrl}
\newrulecommand{IPrr}{IPrr}
\newrulecommand{IInl}{IInl}
\newrulecommand{IInr}{IInr}
\newrulecommand{IMatch}{IMatch}

% final
\newrulecommand{FVal}{FVal}
\newrulecommand{FIndet}{FIndet}

% step
\newrulecommand{ITHole}{ITHole}
\newrulecommand{ITApFun}{ITApFun}
\newrulecommand{ITApArg}{ITApArg}
\newrulecommand{ITAp}{ITAp}
\newrulecommand{ITPairL}{ITPairL}
\newrulecommand{ITPairR}{ITPairR}
\newrulecommand{ITPrl}{ITPrl}
\newrulecommand{ITPrr}{ITPrr}
\newrulecommand{ITInl}{ITInl}
\newrulecommand{ITInr}{ITInr}
\newrulecommand{ITExpMatch}{ITExpMatch}
\newrulecommand{ITSuccMatch}{ITSuccMatch}
\newrulecommand{ITFailMatch}{ITFailMatch}

% typing of constraints
\newrulecommand{CTTruth}{CTTruth}
\newrulecommand{CTFalsity}{CTFalsity}
\newrulecommand{CTUnknown}{CTUnknown}
\newrulecommand{CTNum}{CTNum}
\newrulecommand{CTNotNum}{CTNotNum}
\newrulecommand{CTAnd}{CTAnd}
\newrulecommand{CTOr}{CTOr}
\newrulecommand{CTInl}{CTInl}
\newrulecommand{CTInr}{CTInr}
\newrulecommand{CTPair}{CTPair}

% satisfaction judgment
\newrulecommand{CSTruth}{CSTruth}
\newrulecommand{CSNum}{CSNum}
\newrulecommand{CSNotNum}{CSNotNum}
\newrulecommand{CSAnd}{CSAnd}
\newrulecommandONE{CSOrL}{CSOrL}
\newrulecommandONE{CSOrR}{CSOrR}
\newrulecommand{CSInl}{CSInl}
\newrulecommand{CSInr}{CSInr}
\newrulecommand{CSPair}{CSPair}
\newrulecommand{CSEHolePair}{CSEHolePair}
\newrulecommand{CSHolePair}{CSHolePair}
\newrulecommand{CSApPair}{CSApPair}
\newrulecommand{CSMatchPair}{CSMatchPair}
\newrulecommand{CSPrlPair}{CSPrlPair}
\newrulecommand{CSPrrPair}{CSPrrPair}

% maybe satisfaction judgment
\newrulecommand{CMSUnknown}{CMSUnknown}
\newrulecommand{CMSExpEHole}{CMSExpEHole}
\newrulecommand{CMSExpHole}{CMSExpHole}
\newrulecommand{CMSAp}{CMSAp}
\newrulecommand{CMSMatch}{CMSMatch}
\newrulecommand{CMSPrl}{CMSPrl}
\newrulecommand{CMSPrr}{CMSPrr}
\newrulecommand{CMSAndL}{CMSAndL}
\newrulecommand{CMSAndR}{CMSAndR}
\newrulecommand{CMSAnd}{CMSAnd}
\newrulecommand{CMSOrL}{CMSOrL}
\newrulecommand{CMSOrR}{CMSOrR}
\newrulecommand{CMSInl}{CMSInl}
\newrulecommand{CMSInr}{CMSInr}
\newrulecommand{CMSPairL}{CMSPairL}
\newrulecommand{CMSPairR}{CMSPairR}
\newrulecommand{CMSPair}{CMSPair}

% must or maybe satisfaction judgment
\newrulecommand{CSMSMay}{CSMSMay}
\newrulecommand{CSMSSat}{CSMSSat}

% incon
\newrulecommand{CINCTruth}{CINCTruth}
\newrulecommand{CINCFalsity}{CINCFalsity}
\newrulecommand{CINCNum}{CINCNum}
\newrulecommand{CINCNotNum}{CINCNotNum}
\newrulecommand{CINCAnd}{CINCAnd}
\newrulecommand{CINCOr}{CINCOr}
\newrulecommand{CINCInj}{CINCInj}
\newrulecommand{CINCInl}{CINCInl}
\newrulecommand{CINCInr}{CINCInr}
\newrulecommand{CINCPairL}{CINCPairL}
\newrulecommand{CINCPairR}{CINCPairR}
\begin{document}
\addtobeamertemplate{block end}{}{\vspace*{2ex}} % White space under blocks
\addtobeamertemplate{block example end}{}{\vspace*{2ex}} % White space under example blocks
\addtobeamertemplate{block alerted end}{}{\vspace*{2ex}} % White space under highlighted (alert) blocks

\setlength{\belowcaptionskip}{2ex} % White space under figures
\setlength\belowdisplayshortskip{2ex} % White space under equations
%\begin{darkframes} % Uncomment for dark theme, don't forget to \end{darkframes}
\begin{frame}[containsverbatim] % The whole poster is enclosed in one beamer frame

%==========================Begin Head===============================
  \begin{columns}
   \begin{column}{\linewidth}
    \vskip1cm
    \centering
    \usebeamercolor{title in headline}{\color{fg}\Huge{\textbf{\inserttitle}}\\[0.5ex]}
    \usebeamercolor{author in headline}{\color{fg}\Large{\insertauthor}\\[1ex]}
    \usebeamercolor{institute in headline}{\color{fg}\large{\insertinstitute}\\[1ex]}
    \vskip1cm
   \end{column}
   \vspace{1cm}
  \end{columns}
 \vspace{1cm}

%==========================End Head===============================

\begin{columns}[t] % The whole poster consists of three major columns, the second of which is split into two columns twice - the [t] option aligns each column's content to the top

\begin{column}{\sepwid}\end{column} % Empty spacer column

\begin{column}{\onecolwid} % The first column

%----------------------------------------------------------------------------------------
%	OBJECTIVES
%----------------------------------------------------------------------------------------

\begin{block}{What is Pattern Matching}
  \begin{columns}
    \begin{column}{0.5\textwidth}
      \begin{lstlisting}[name=not exhaustive,caption={Not Exhaustive},captionpos=b,escapechar=!]
        match (3::[ ]) {
        | [ ] -> "empty"
        | 1::xs -> "1,..."
        | 2::xs -> "2,..."
        }
      \end{lstlisting}
      \balloon{comment}{not exhaustive}{2}{4}
    \end{column}

    \begin{column}{0.5\textwidth}
      \begin{lstlisting}[name=redundant,caption={Redundant Branch},captionpos=b,escapechar=!]
        match (2::[ ]) {
        | [ ] -> "empty"
        | 1::xs -> "1,..."
        | 1::2::xs -> "1,2,..."
        }
      \end{lstlisting}
      \balloon{comment}{redundant}{4}{4}
    \end{column}
  \end{columns}
\end{block}

\begin{block}{Why Adding Holes to Pattern Matching}

\begin{itemize}
  \item Chapter \textit{Pattern Matching} in \textsf{PFPL} \cite{Harper2012} introduces a match constraint language to check exhaustiveness of a match expression and redundancy of a single rule. We \textcolor{red}{extend the constraint language with \textit{Unknown} constraint} and adapt the checking algorithm to our setting.
  \item \textsf{Hazel} is a programming environment featuring typed holes \cite{DBLP:conf/popl/OmarVHAH17,DBLP:journals/pacmpl/OmarVCH19}, but it only supports simple case analysis on binary sum types. We want to \textcolor{red}{formalize the full-fledged pattern matching with Typed Holes}.
  \item \textsf{Agda} allows the programmer to automatically generate code through "case splitting", our work is focused on \textcolor{red}{giving live feedbacks and guidance} as the programmer enters a match expression.
\end{itemize}
\end{block}

\begin{block}{How Pattern Matching with Typed Holes Works}
  \begin{columns}
    \begin{column}{.8\textwidth}
    \begin{lstlisting}[name=syntax,caption={A match expression in our internal language},captionpos=b,escapechar=!,mathescape]
      match (!\tikzmark{scrut-left}!inr((1, $\hehole{u}$))!\tikzmark{scrut-right}!) {
      | !\tikzmark{empty-left}!inl(())!\tikzmark{empty-right}! -> "empty"
      | !\tikzmark{hole-xs-left}!inr(($\hehole{w}$, xs))!\tikzmark{hole-xs-right}! -> "1,..."
      | !\tikzmark{1-2-xs-left}!inr((1, inr((2, xs))))!\tikzmark{1-2-xs-right}! -> "1,2,..."
      }
    \end{lstlisting}
    \end{column}
    \begin{column}{.2\textwidth}
      \begin{itemize}
        \item[\tikzmark{scrut}] 1::?
        \item[\tikzmark{empty}] [\,]
        \item[\tikzmark{hole-xs}] ?::xs
        \item[\tikzmark{1-2-xs}] 1::2::xs
      \end{itemize}
    \end{column}
      \begin{tikzpicture}[remember picture, overlay]
        \draw[hazellightgreen,rounded corners]
        ([shift={(-3pt,2ex)}]pic cs:scrut-left) 
          rectangle 
        ([shift={(3pt,-0.65ex)}]pic cs:scrut-right);
        \draw[line width=3,hazellightgreen,->]
        (pic cs:scrut-right) -> ($(pic cs:scrut)+(0,0.5)$);

        \draw[hazellightgreen,rounded corners]
        ([shift={(-3pt,2ex)}]pic cs:empty-left) 
          rectangle 
        ([shift={(3pt,-0.65ex)}]pic cs:empty-right);
        \draw[line width=3,hazellightgreen,->]
        (pic cs:empty-right) -> ($(pic cs:empty)+(0,0.5)$);

        \draw[hazellightgreen,rounded corners]
        ([shift={(-3pt,2ex)}]pic cs:hole-xs-left) 
          rectangle 
        ([shift={(3pt,-0.65ex)}]pic cs:hole-xs-right);
        \draw[line width=3,hazellightgreen,->]
        (pic cs:hole-xs-right) -> ($(pic cs:hole-xs)+(0,0.5)$);

        \draw[hazellightgreen,rounded corners]
        ([shift={(-3pt,2ex)}]pic cs:1-2-xs-left) 
          rectangle 
        ([shift={(3pt,-0.65ex)}]pic cs:1-2-xs-right);
        \draw[line width=3,hazellightgreen,->]
          (pic cs:1-2-xs-right) -> ($(pic cs:1-2-xs)+(0,0.5)$);
      \end{tikzpicture}
  \end{columns}

  \begin{columns}
    \begin{column}{0.5\textwidth}
    \begin{lstlisting}[basicstyle=\small,name=syntax 1,caption={Does Not Match},captionpos=b,escapechar=!,mathescape]
      match (inr((1, $\hehole{u}$))) {
      !\tikzmark{rule-1}!| inl(()) => "empty"
      | inr(($\hehole{w}$, xs)) => "1,..."!\tikzmark{step-left}!
      | inr((1, inr((2, xs)))) => "1,2,..."
      }
    \end{lstlisting}
    \end{column}
    \begin{column}{0.5\textwidth}
    \begin{lstlisting}[basicstyle=\small,name=syntax 2,caption={May Match},captionpos=b,escapechar=!,mathescape]
      match (inr((1, $\hehole{u}$))) {
      | inl(()) => "empty"
      !\tikzmark{rule-2}!| inr(($\hehole{w}$, xs)) => "1,..."
      | inr((1, inr((2, xs)))) => "1,2,..."
      }
    \end{lstlisting}
    \end{column}
    \begin{tikzpicture}[remember picture, overlay]
      \node[rotate=-90,mark size=10pt,color=hazellightgreen,shift={(-0.25,-0.5)}] at (pic cs:rule-1) {\pgfuseplotmark{triangle*}}; 
      \node[rotate=-90,mark size=10pt,color=hazellightgreen,shift={(-0.25,-0.5)}] at (pic cs:rule-2) {\pgfuseplotmark{triangle*}}; 
      \draw[line width=3, |->] ($(pic cs:step-left)+(3,0)$) -- ($(pic cs:rule-2)-(2,0)$);
    \end{tikzpicture}
  \end{columns}
  We know that an expression that cannot be further evaluated is a \textbf{value}.
  When the value of an expression is indeterminate due to unfilled holes, such an expression is \textbf{indeterminate}. For example,
  \begin{itemize}
    \item any expression that contains an expression hole $\hehole{u}$ or $\hhole{e}{u}$
    \item a match expression in which the scrutinee may match the branch under the pointer
  \end{itemize}
  And an expression is $\textbf{final}$ when it is either a value or indeterminate.
\end{block}

%----------------------------------------------------------------------------------------

\end{column} % End of the first column

\begin{column}{\sepwid}\end{column} % Empty spacer column

\begin{column}{\onecolwid} % Begin a column which is two columns wide (column 2)

\begin{block}{Expressions, Patterns and Constraints}
  % !TEX root= extended-abstract.tex

\begin{figure}[h]
$\arraycolsep=4pt\begin{array}{lll}
e & ::= &
  x ~\vert~
  \htriv ~\vert~
  \hnum{n} ~\vert~
  \hpair{e_1}{e_2} ~\vert~
  \hinl{\tau}{e} ~\vert~
  \hinr{\tau}{e} ~\vert~
  \hehole{u} ~\vert~
  \hhole{e}{u} \\
  & ~\vert~ &
  \hlam{x}{\tau}{e} ~\vert~
  \hap{e_1}{e_2} ~\vert~
  \hmatch{e}{\hat{rs}} \\
p & ::= &
  x ~\vert~
  \_ ~\vert~
  \htriv ~\vert~
  \hnum{n} ~\vert~
  \hpair{p_1}{p_2} ~\vert~
  \hinlp{p} ~\vert~
  \hinrp{p} ~\vert~
  \hehole{w} ~\vert~
  \hhole{p}{w} \\
\xi & ::= &
  \ctruth ~\vert~
  \cfalsity ~\vert~
  \cunit ~\vert~
  \cnum{n} ~\vert~
  \cnotnum{n} ~\vert~
  \cpair{\xi_1}{\xi_2} ~\vert~
  \cinl{\xi} ~\vert~
  \cinr{\xi} ~\vert~
  \cunknown ~\vert~
  \cand{\xi_1}{\xi_2} ~\vert~
  \cor{\xi_1}{\xi_2}
\end{array}$
\label{fig:exp-pat-constraint}
\end{figure}

  \begin{tikzpicture}
    \node (E) at (0,0) {expression e};
    \node (P) [below left=12cm and 8cm of E] {pattern p};
    \node (Xi) [below right=12cm and 8cm of E] {constraint $\xi$};

    \draw [line width=5, ->] (E) -- (P) node (Match) [midway, above, sloped] {\begin{tabular}{c} \textcolor{green}{match} \\ \textcolor{yellow}{may match} \\ \textcolor{red}{not match} \end{tabular}};
    \draw [line width=5, ->] (E) -- (Xi) node (Satisfy) [midway, above, sloped] {\begin{tabular}{c} \textcolor{green}{satisfy $\csatisfy{e}{\xi}$} \\ \textcolor{yellow}{may satisfy $\cmaysatisfy{e}{\xi}$} \\ \textcolor{red}{not satisfy $\csatisfy{e}{\cdual{\xi}}$} \end{tabular}};
    \draw [color=hazellightgreen, line width=5, ->] (P) -- (Xi) node [midway, below, sloped] {emit};
    \draw [color=hazellightgreen, line width=4, {Latex[length=1cm]}-{Latex[length=1cm]},double] ($(Match)+(5,0)$) -- ($(Satisfy)-(5,0)$) node [midway, below, sloped] {one-to-one equivalence};
  \end{tikzpicture}
  For example,
  \begin{tikzpicture}
    \node (H) at (0,0) {pattern hole $\hehole{w}$ or $\hhole{p}{w}$};
    \node (U) [right=10cm of H] {unknown constraint $?$};
%    \node (N) at (0,1.5) {pattern $\hnum{1}$};
%    \node (X) [right=12cm of N] {number constraint $\cnum{1}$};
%    \draw [color=hazellightgreen, line width=5, ->] (N) -- (X) node [midway, above, sloped] {emit};
    \draw [color=hazellightgreen, line width=5, ->] (H) -- (U) node [midway, above, sloped] {emit};

    \node (R) at (3.5,-1.75) {branch(rule) $\hrul{\hinrp{\hpair{\hehole{w}}{xs}}}{\text{"empty"}}$};
    \node (C) [right=3cm of R] {constraint $\cinr{\cpair{\cunknown}{\ctruth}}$};
    \draw [color=hazellightgreen, line width=5, ->] (R) -- (C) node [midway, above, sloped] {emit};

    \node (Rs) at (0,-3.5) {branches(rules) $r_1|r_2|r_3|\dots$};
    \node (Cs) [right=9cm of Rs] {constraint $\cor{\xi_1}{\cor{\xi_2}{\dots}}$};
    \draw [color=hazellightgreen, line width=5, ->] (Rs) -- (Cs) node [midway, above, sloped] {emit};
  \end{tikzpicture}
\end{block}

\begin{block}{Typing Judgment of Branches (Rules rs) and Redundancy}
  \begin{align*}
    & Branch & & Constraint \\
    ~\vert~ &
    \hrul{\hinlp{\htriv}}{\text{"empty"}} & \longrightarrow & \hinlp{\cunit} \\
    ~\vert~ &
    \hrul{\hinrp{\hpair{\hehole{w}}{xs}}}{\text{"empty"}} & \longrightarrow & \hinrp{\cpair{\cunknown}{\ctruth}} \\
    ~\vert~ &
    \hrul{\hinrp{\hpair{\hnum{1}}{\hinrp{\hpair{\hnum{2}}{xs}}}}}{\text{"empty"}} & \longrightarrow & \cinr{\cpair{\cnum{1}}{\cinr{\cpair{\cnum{2}}{\ctruth}}}}
  \end{align*}

  A branch is redundant $\textit{iff}$ all expressions that match or may match that branch, must match one of its preceding branches.
  
  We use emitted constraints to check redundancy of every single branch
  \begin{itemize}
    \item the second branch is not redundant, $\cnotsatisfy{\hinrp{\cpair{\cunknown}{\ctruth}}}{\hinlp{\cunit}}$
  \item the third branch is not redundant, $\cnotsatisfy{\cinr{\cpair{\cnum{1}}{\cinr{\cpair{\cnum{2}}{\ctruth}}}}}{\cor{\hinlp{\cunit}}{\hinrp{\cpair{\cunknown}{\ctruth}}}}$
  \end{itemize}
\end{block}

\begin{block}{Typing Judgment of Match Expression and Exhaustiveness}
  Branches must or may be exhaustive \textit{iff} all expressions either matches one of the branches or may match one of the branches.

  We use the constraint emitted from the three branches to enforce that the match expression must or may be exhaustive
\[
  \csatisfyormay{\ctruth}{
    \cor{\hinlp{\cunit}}{\cor{\hinrp{\cpair{\cunknown}{\ctruth}}}{\cinr{\cpair{\cnum{1}}{\cinr{\cpair{\cnum{2}}{\ctruth}}}}}}
  }
\]
\end{block}

%----------------------------------------------------------------------------------------

\end{column} % End of column 2.1

\begin{column}{\sepwid}\end{column} % Empty spacer column

%\begin{alertblock}{Important Result}
%
%Lorem ipsum dolor \textbf{sit amet}, consectetur adipiscing elit. Sed commodo molestie porta. Sed ultrices scelerisque sapien ac commodo. Donec ut volutpat elit.
%
%\end{alertblock} 

\begin{column}{\onecolwid} % The second column within column 2 (column 2.2)

%----------------------------------------------------------------------------------------
%	CONCLUSION
%----------------------------------------------------------------------------------------


\begin{defn}["Must" or "May" Entailment]
  \label{defn:exhaustiveness}
  $\csatisfyormay{\xi_1}{\xi_2}$ iff $\ctyp{\xi_1}{\tau}$ and $\ctyp{\xi_2}{\tau}$ and for all $e$ such that $\hexptyp{\cdot}{\Gamma}{e}{\tau}$ and $\isFinal{e}$ we have \textcolor{green}{$\csatisfy{e}{\xi_1}$} or \textcolor{yellow}{$\cmaysatisfy{e}{\xi_1}$} implies \textcolor{green}{$\csatisfy{e}{\xi_2}$} or \textcolor{yellow}{$\cmaysatisfy{e}{\xi_2}$}.
\end{defn}

\begin{defn}["Must" Entailment]
  \label{defn:redundancy}
  $\csatisfy{\xi_1}{\xi_2}$ iff $\ctyp{\xi_1}{\tau}$ and $\ctyp{\xi_2}{\tau}$ and for all $e$ such that $\hexptyp{\cdot}{\Gamma}{e}{\tau}$ and $\isVal{e}$ we have \textcolor{green}{$\csatisfy{e}{\xi_1}$} or \textcolor{yellow}{$\cmaysatisfy{e}{\xi_1}$} implies \textcolor{green}{$\csatisfy{e}{\xi_2}$}.
\end{defn}

\begin{block}{De-unknown}
  \begin{center}
  \begin{tikzpicture}
    \node (Exh) at (0,0) {\begin{tabular}{c} Exhaustiveness or Maybe Exhaustiveness \\ $\csatisfyormay{\ctruth}{\cor{\hinlp{\cunit}}{\cor{\hinrp{\cpair{\cunknown}{\ctruth}}}{\cinr{\cpair{\cnum{1}}{\cinr{\cpair{\cnum{2}}{\ctruth}}}}}}}$\end{tabular}};
    \node (De-Exh) at (0,-3) {$\csatisfy{\ctruth}{\cor{\hinlp{\cunit}}{\cor{\hinrp{\cpair{\ctruth}{\ctruth}}}{\cinr{\cpair{\cnum{1}}{\cinr{\cpair{\cnum{2}}{\ctruth}}}}}}}$};
    \node (Red2) at (0,-6) {\begin{tabular}{c} Redundancy of Second Branch \\ $\csatisfy{\hinrp{\cpair{\cunknown}{\ctruth}}}{\hinlp{\cunit}}$\end{tabular}};
    \node (De-Red2) at (0,-9) {$\csatisfy{\hinrp{\cpair{\ctruth}{\ctruth}}}{\hinlp{\cunit}}$};
    \node (Red3) at (0,-12) {\begin{tabular}{c} Redundancy of Second Branch \\ $\csatisfy{\cinr{\cpair{\cnum{1}}{\cinr{\cpair{\cnum{2}}{\ctruth}}}}}{\cor{\hinlp{\cunit}}{\hinrp{\cpair{\cunknown}{\ctruth}}}}$\end{tabular}};
    \node (De-Red3) at (0,-15) {$\csatisfy{\cinr{\cpair{\cnum{1}}{\cinr{\cpair{\cnum{2}}{\ctruth}}}}}{\cor{\hinlp{\cunit}}{\hinrp{\cpair{\cfalsity}{\ctruth}}}}$};

    \draw [color=hazellightgreen,line width=2,Latex-Latex,double] ($(Exh.west)+(0.15,-0.5)$) to [out=200,in=160] (De-Exh.west);
    \draw [color=hazellightgreen,line width=2,Latex-Latex,double] ($(Red2.west)+(1.7,-0.5)$) to [out=200,in=160] (De-Red2.west);
    \draw [color=hazellightgreen,line width=2,Latex-Latex,double] ($(Red3.west)+(0,-0.5)$) to [out=200,in=160] (De-Red3.west);
  \end{tikzpicture}
  \end{center}
  Then, we can apply similar checking algorithm as described in Chapter $\textit{Pattern Matching}$ of PFPL \cite{Harper2012}.
\end{block}

\begin{block}{Conclusion}
  We have formalized the type system and are still working on the proof of the correctness of exhaustiveness checking and redundancy checking. The idea has been implemented in a toy programming language
  (\url{https://github.com/fplab/pattern-paper/tree/master/src}).
\end{block}

%----------------------------------------------------------------------------------------
%	REFERENCES
%----------------------------------------------------------------------------------------

\begin{block}{References}

\small{\bibliographystyle{apalike}
\bibliography{references}\vspace{1cm}}
\end{block}


%----------------------------------------------------------------------------------------

\end{column} % End of column 2.2


\begin{column}{\sepwid}\end{column} % Empty spacer column

\end{columns} % End of all the columns in the poster

\end{frame} % End of the enclosing frame
%\end{darkframes} % Uncomment for dark theme

\end{document}
