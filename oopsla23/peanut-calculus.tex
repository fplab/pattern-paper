\section{Peanut: A Typed Pattern Hole Calculus}
\label{sec:formalism}
We now formalize the mechanisms described in use in \autoref{sec:examples} with a core calculus, called Peanut, that supports pattern matching with typed holes in both expression and pattern position. We develop both a static semantics that enforces exhaustiveness and irredundancy as described in \Autoref{sec:hazel-exhaustiveness,sec:hazel-redundancy}, and a dynamic semantics, based on the Hazelnut Live internal language \cite{DBLP:journals/pacmpl/OmarVCH19}, that allows us to engage in live programming with expression and pattern holes as described in Sec.~\ref{sec:hazel-live-eval}.

% Conventionally, matching of an expression to a pattern leads to success or failure (or divergence in the lazy setting). In the setting of typed holes, there is an additional outcome: \emph{indeterminate}.
% An indeterminate match occurs when the pattern or expression contains holes and the match could succeed or fail depending on how the programmer modifies those holes. Such dependence on yet unspecified programmer input similarly colors the outcomes of corresponding coverage and redundancy checkers: a collection of patterns may be \emph{indeterminately inexhaustive}, and a pattern may be \emph{indeterminately redundant} with respect to some preceding collection of patterns. \todo{is "indeterminately" here confusing as in the inexhaustiveness and redundancy are actually determinate}

\Autoref{sec:Syntax} defines the syntax of Peanut and \autoref{sec:dynamics} defines its small-step operational semantics. \Autoref{sec:constraint,sec:statics,sec:algorithm} define a corresponding static semantics as a type assignment system equipped with a match constraint language that we use in reasoning about exhaustiveness and redundancy in the presence of
holes. \autoref{sec:agda} describes the Agda mechanization of Peanut's metatheory. Finally, \Autoref{sec:decidability} describes algorithmic procedures for exhaustiveness and redundancy checking.

% !TEX root= pattern-paper.tex

\begin{figure}[ht]
    \centering
    \begin{minipage}{.5\linewidth}
  $\arraycolsep=4pt\begin{array}{lll}
    e & ::= &
      x ~\vert~
      \hnum{n} \\
      & ~\vert~ &
      \hlam{x}{\tau}{e} ~\vert~
      \hap{e_1}{e_2} \\
      & ~\vert~ &
      \hpair{e_1}{e_2} ~\vert~
      \hfst{e} ~\vert~ \hsnd{e} \\
      & ~\vert~ &
      \hinl{\tau}{e} ~\vert~
      \hinr{\tau}{e} \\
      & ~\vert~ &
      \hmatch{e}{\hat{rs}} \\
      & ~\vert~ &
      \hehole{u} ~\vert~
      \hhole{e}{u} \\
    \end{array}$
    \end{minipage}%
    \begin{minipage}{.5\linewidth}
  $\arraycolsep=4pt\begin{array}{lll}
    \tau & ::= &
      \tnum ~\vert~
      \tarr{\tau_1}{\tau_2} ~\vert~
      \tprod{\tau_1}{\tau_2} ~\vert~
      \tsum{\tau_1}{\tau_2} \\
    \hat{rs} & ::= &
      \zrulsP{rs}{r}{rs} \\
    rs & ::= &
      \cdot ~\vert~ \hrulesP{r}{rs'} \\
    r & ::= &
      \hrul{p}{e} \\
    p & ::= &
      x ~\vert~
      \_ ~\vert~
      \hnum{n} ~\vert~
      \hpair{p_1}{p_2} \\
      & ~\vert~ &
      \hinlp{p} ~\vert~
      \hinrp{p} ~\vert~
      \heholep{w} ~\vert~
      \hholep{p}{w}{\tau}
    \end{array}$
    \end{minipage}
\caption{Syntax}
\label{fig:syntax}
\end{figure}
% !TEX root= pattern-paper.tex

\begin{figure}[ht]

\judgbox{\rmpointer{\zrules} = rs}
        {$rs$ can be obtained by erasing pointer from $\zrules$}
\begin{align*}
  \rmpointer{\zruls{\cdot}{r}{rs}} &= \hrules{r}{rs} \\
  \rmpointer{\zruls{\hrulesP{r'}{rs'}}{r}{rs}} &= \hrules{r'}{\rmpointer{\zruls{rs'}{r}{rs}}}
\end{align*}

\caption{Pointer Eraser flattens the rules with pointer.}
\label{fig:pointer-eraser}
\end{figure}


\subsection{Syntax}
\label{sec:Syntax}
\autoref{fig:syntax} presents the syntax of Peanut.
Peanut is based on the internal language of Hazelnut Live, a typed lambda calculus featuring holes in expression position \cite{DBLP:journals/pacmpl/OmarVCH19}.
We choose numbers as the base type and add binary sums and binary products so that we have interesting
patterns to consider. We also remove the machinery related to gradual typing (casts and failed casts) and expression hole closures to focus our attention on pattern matching in particular, though these features are retained in our Hazel implementation. Most forms are standard (we base our formulation on \cite{Harper2012}). We include functions, function application, pairs, explicit projection operators (for reasons we will consider below), and left and right injections, which are the introductory forms for sum types. Functions and injections include type annotations so that we can define a simple type assignment system. In our Hazel implementation, there is a corresponding bidirectionally typed external language where type annotations are not always necessary, given meaning as in Hazelnut Live by an elaboration process \cite{DBLP:journals/pacmpl/OmarVCH19}, but the details follow straightforwardly from this prior work, so we favor a minimal presentation in this paper. 

Empty expression holes are written $\hehole{u}$ and non-empty expression holes, which serve as membranes around type inconsistencies, are written $\hhole{e}{u}$. Similarly, empty pattern holes are written $\heholep{w}$ and non-empty pattern holes, which are analogous, are written $\hholep{p}{w}{\tau}$. Here, $u$ denotes the name of an expression hole, $w$ denotes the name of a pattern hole, and $\tau$ is the type of the enclosed pattern $p$ (which serves to ensure that the typing judgment in Sec.~\ref{sec:statics} is straightforwardly algorithmic).
We assume that hole names 
correspond to unique holes in the external language, which we do not model here. We do not impose a uniqueness constraint, however, because we are defining an internal language so substitution can cause a hole can appear multiple times during evaluation.

A match expression, $\hmatch{e}{\zrules}$, 
consists of a scrutinee, $e$, and a zipper of rules, $\zrules$, which takes the form of a triple, $\zruls{rs_{pre}}{r}{rs_{post}}$, consisting of a prefix rule sequence, $rs_{pre}$, a current rule, $r$, and a suffix rule sequence, $rs_{post}$. In other words, it is a sequence of one or more rules with a pointer marking the rule currently being considered during evaluation (starting on the first rule). This zippered representation, which is unique to Peanut, is necessary to provide an intermediate result when the evaluation is stuck due to indeterminate pattern matching as shown in Fig.~\ref{fig:pat-hole}. Specifically, $rs_{pre}$ corresponds the first two rules that are crossed out, $r$ corresponds to the rule where pattern matching was indeterminate, and $rs_{post}$ corresponds to the last rule that is yet to be considered. We define a \text{pointer erasure operator} $\rmpointer{\zrules}$ in \autoref{fig:pointer-eraser}. 
Each rule, $r$, consists of a pattern, $p$, and a branch expression, $e$.

\subsection{Live Operational Semantics}\label{sec:dynamics}


The dynamic semantics of Peanut extends Hazelnut Live \cite{DBLP:journals/pacmpl/OmarVCH19} with support for pattern matching with holes. Hazelnut Live defines its dynamic semantics as a contextual operational  semantics \cite{DBLP:conf/ppdp/PientkaD08} capable of evaluating expressions with holes, whereas we choose a structural operational semantics \cite{DBLP:journals/jlp/Plotkin04a} for Peanut because it slightly simplifies the presentation of stepping through individual rules in a match expression. In this section, we present primarily the rules related to pattern matching (see the Supplemental Material for the full definition), as the rules for the other forms in Peanut follow directly from Hazelnut Live. 


% !TEX root= pattern-paper.tex

\begin{figure}[th]

\judgbox{\isVal{e}}{$e$ is a value}

\begin{mathpar}
\Infer{\VNum}{ }{
  \isVal{\hnum{n}}
}

\Infer{\VLam}{ }{
  \isVal{\hlam{x}{\tau}{e}}
}

\Infer{\VPair}{
  \isVal{e_1} \\
  \isVal{e_2}
}{
  \isVal{\hpair{e_1}{e_2}}
}

\Infer{\VInl}{
  \isVal{e}
}{
  \isVal{\hinl{\tau}{e}}
}

\Infer{\VInr}{
  \isVal{e}
}{
  \isVal{\hinr{\tau}{e}}
}
\end{mathpar}

\judgbox{\isIndet{e}}{$e$ is indeterminate}

\begin{mathpar}
\Infer{\IEHole}{ }{
  \isIndet{\hehole{u}}
}

\Infer{\IHole}{
  \isFinal{e}
}{
  \isIndet{\hhole{e}{u}}
}

\Infer{\IAp}{
  \isIndet{e_1} \\ \isFinal{e_2}
}{
  \isIndet{\hap{e_1}{e_2}}
}

\Infer{\IPairL}{
  \isIndet{e_1} \\ \isVal{e_2}
}{
  \isIndet{\hpair{e_1}{e_2}}
}

\Infer{\IPairR}{
  \isVal{e_1} \\ \isIndet{e_2}
}{
  \isIndet{\hpair{e_1}{e_2}}
}

\Infer{\IPair}{
  \isIndet{e_1} \\ \isIndet{e_2}
}{
  \isIndet{\hpair{e_1}{e_2}}
}

\Infer{\IFst}{
  \isIndet{e} \\ e \neq \hpair{e_1}{e_2}
}{
  \isIndet{\hfst{e}}
}

\Infer{\ISnd}{
  \isIndet{e} \\ e \neq \hpair{e_1}{e_2}
}{
  \isIndet{\hsnd{e}}
}

\Infer{\IInl}{
  \isIndet{e}
}{
  \isIndet{\hinl{\tau}{e}}
}

\Infer{\IInr}{
  \isIndet{e}
}{
  \isIndet{\hinr{\tau}{e}}
}

\Infer{\IMatch}{
  \isFinal{e} \\
  \hmaymatch{e}{p_r}
}{
  \isIndet{
    \hmatch{e}{\zruls{rs_{pre}}{\hrulP{p_r}{e_r}}{rs_{post}}}
  }
}
\end{mathpar}

\judgbox{\isFinal{e}}{$e$ is final}

\begin{mathpar}
\Infer{\FVal}{
  \isVal{e}
}{
  \isFinal{e}
}

\Infer{\FIndet}{
  \isIndet{e}
}{
  \isFinal{e}
}
\end{mathpar}

  \caption{Final expressions}
  \label{fig:final}
\end{figure}


We say an expression is \textit{final} if it cannot take any more steps. \autoref{fig:final} presents the finality judgment, $\isFinal{e}$. A final expression can be either (1) a value, specified by $\isVal{e}$, which is standard, or (2) an  \emph{indeterminate} expression, $\isIndet{e}$, where the next step of evaluation would require filling one or more holes. For example, expression holes are indeterminate (Rules \IEHole and \IHole), as are function applications with indeterminate expressions in function position (omitted). 
% The \ITSuccMatch, \ITFailMatch, and \IMatch rules, which feature pattern matching premises of the form $\hpatmatch{e}{p_r}{\theta}$ and $\hnotmatch{e}{p_r}$ and $\hmaymatch{e}{p_r}$, respectively, are central and we will return to these shortly.
% !TEX root= pattern-paper.tex

\begin{figure}[ht]

\judgbox{\htrans{e}{e'}}{$e$ takes a step to $e'$}

\begin{mathpar}
\Infer{\ITHole}{
  \htrans{e}{e'}
}{
  \htrans{\hhole{e}{u}}{\hhole{e'}{u}}
}

\Infer{\ITApFun}{
  \htrans{e_1}{e_1'}
}{
  \htrans{\hap{e_1}{e_2}}{\hap{e_1'}{e_2}}
}

\Infer{\ITApArg}{
  \isFinal{e_1} \\
  \htrans{e_2}{e_2'}
}{
  \htrans{\hap{e_1}{e_2}}{\hap{e_1}{e_2'}}
}

\Infer{\ITAp}{
  \isFinal{e_2}
}{
  \hap{\hlam{x}{\tau}{e_1}}{e_2} \mapsto
    [e_2/x]e_1
}

\Infer{\ITPairL}{
  \htrans{e_1}{e_1'}
}{
  \htrans{\hpair{e_1}{e_2}}{\hpair{e_1'}{e_2}}
}

\Infer{\ITPairR}{
  \isFinal{e_1} \\
  \htrans{e_2}{e_2'}
}{
  \htrans{\hpair{e_1}{e_2}}{\hpair{e_1}{e_2'}}
}

\Infer{\ITPrl}{
  \isFinal{\hpair{e_1}{e_2}}
}{
  \htrans{\hprl{\hpair{e_1}{e_2}}}{e_1}
}

\Infer{\ITPrr}{
  \isFinal{\hpair{e_1}{e_2}}
}{
  \htrans{\hprr{\hpair{e_1}{e_2}}}{e_2}
}

\Infer{\ITInl}{
  \htrans{e}{e'}
}{
  \htrans{\hinl{\tau}{e}}{\hinl{\tau}{e'}}
}

\Infer{\ITInr}{
  \htrans{e}{e'}
}{
  \htrans{\hinr{\tau}{e}}{\hinr{\tau}{e'}}
}

\Infer{\ITExpMatch}{
  \htrans{e}{e'}
}{
  \htrans{\hmatch{e}{\zrules}}{\hmatch{e'}{\zrules}}
}

\Infer{\ITSuccMatch}{
  \isFinal{e} \\
  \hpatmatch{e}{p_r}{\theta}
}{
  \htrans{
    \hmatch{e}{\zruls{rs_{pre}}{\hrulP{p_r}{e_r}}{rs_{post}}}
  }{
    [\theta](e_r)
  }
}

\Infer{\ITFailMatch}{
  \isFinal{e} \\
  \hnotmatch{e}{p_r}
}{
  \htrans{
    \hmatch{e}{\zruls{rs}{\hrulP{p_r}{e_r}}{\hrulesP{r'}{rs'}}}
  }{
    \hmatch{e}{
      \zruls{
        \rmpointer{\zruls{rs}{\hrulP{p_r}{e_r}}{\cdot}}
      }{r'}{rs'}
    }
  }
}
\end{mathpar}

  \caption{Stepping}
  \label{fig:step}
\end{figure}



As with Hazelnut Live, evaluation proceeds around holes, taking those steps that do not depend on how the holes are filled. 
\autoref{fig:step} presents the novel Rules of the stepping judgment, $\htrans{e}{e'}$. \ITExpMatch steps the scrutinee of a match expression and is applied until the scrutinee is final. Once the scrutinee is final, the match expression's rules are considered in turn by stepping through the zipper. At each step, one of three situations can arise, corresponding to Rules \ITSuccMatch{} (a successful match), \ITMisMatch (a mismatch), and \IMatch (an indeterminate match). The second premise of each of these Rules invokes a corresponding judgment, presented in \autoref{fig:patmatch},  relating the scrutinee with the pattern being considered. Exactly one of these three judgments holds, given a well-typed expression and a pattern of the same type (see Sec.~\ref{sec:statics}).
% !TEX root= pattern-paper.tex

\begin{figure}[!b]

\judgbox{
  \hpatmatch{e}{p}{\theta}
}{
  $e$ matches $p$, emitting $\theta$
}

\begin{mathpar}
\Infer{\MVar}{ }{
  \hpatmatch{e}{x}{e / x}
}

\Infer{\MWild}{ }{
  \hpatmatch{e}{\_}{\cdot}
}

\Infer{\MNum}{ }{
  \hpatmatch{\hnum{n}}{\hnum{n}}{\cdot}
}

\Infer{\MPair}{
  \hpatmatch{e_1}{p_1}{\theta_1} \\
  \hpatmatch{e_2}{p_2}{\theta_2}
}{
  \hpatmatch{\hpair{e_1}{e_2}}{\hpair{p_1}{p_2}}{\theta_1 \uplus \theta_2}
}

\Infer{\MInl}{
  \hpatmatch{e}{p}{\theta}
}{
  \hpatmatch{\hinl{\tau}{e}}{\hinlp{p}}{\theta}
}

\Infer{\MInr}{
  \hpatmatch{e}{p}{\theta}
}{
  \hpatmatch{\hinr{\tau}{e}}{\hinrp{p}}{\theta}
}

\Infer{\MNotIntroPair}{
  \notIntro{e} \\
  \hpatmatch{\hprl{e}}{p_1}{\theta_1} \\
  \hpatmatch{\hprr{e}}{p_2}{\theta_2}
}{
  \hpatmatch{e}{\hpair{p_1}{p_2}}{\theta_1 \uplus \theta_2}
}
\end{mathpar}

\judgbox{
  \hnotmatch{e}{p}
}{
  $e$ does not match $p$
}

\begin{mathpar}

\Infer{\NMNum}{
  n_1 \neq n_2
}{
  \hnotmatch{\hnum{n_1}}{\hnum{n_2}}
}

\Infer{\NMPairL}{
  \hnotmatch{e_1}{p_1}
}{
  \hnotmatch{\hpair{e_1}{e_2}}{\hpair{p_1}{p_2}}
}

\Infer{\NMPairR}{
  \hnotmatch{e_2}{p_2}
}{
  \hnotmatch{\hpair{e_1}{e_2}}{\hpair{p_1}{p_2}}
}

\Infer{\NMConfL}{ }{
  \hnotmatch{\hinr{\tau}{e}}{\hinlp{p}}
}

\Infer{\NMConfR}{ }{
  \hnotmatch{\hinl{\tau}{e}}{\hinrp{p}}
}

\Infer{\NMInl}{
  \hnotmatch{e}{p}
}{
  \hnotmatch{\hinr{\tau}{e}}{\hinlp{p}}
}

\Infer{\NMInr}{
  \hnotmatch{e}{p}
}{
  \hnotmatch{\hinl{\tau}{e}}{\hinrp{p}}
}
\end{mathpar}

\judgbox{
  \hmaymatch{e}{p}
}{
  $e$ indeterminately matches $p$
}

\begin{mathpar}
\Infer{\MMEHole}{ }{
  \hmaymatch{e}{\heholep{w}}
}

\Infer{\MMHole}{ }{
  \hmaymatch{e}{\hholep{p}{w}{\tau}}
}
\\
\Infer{\MMNotIntro}{
  \notIntro{e} \\
  \refutable{p}
}{
  \hmaymatch{e}{p}
}

\Infer{\MMPairL}{
  \hmaymatch{e_1}{p_1} \\
  \hpatmatch{e_2}{p_2}{\theta_2}
}{
  \hmaymatch{\hpair{e_1}{e_2}}{\hpair{p_1}{p_2}}
}

\Infer{\MMPairR}{
  \hpatmatch{e_1}{p_1}{\theta_1} \\
  \hmaymatch{e_2}{p_2}
}{
  \hmaymatch{\hpair{e_1}{e_2}}{\hpair{p_1}{p_2}}
}

\Infer{\MMPair}{
  \hmaymatch{e_1}{p_1} \\
  \hmaymatch{e_2}{p_2}
}{
  \hmaymatch{\hpair{e_1}{e_2}}{\hpair{p_1}{p_2}}
}

\Infer{\MMInl}{
  \hmaymatch{e}{p}
}{
  \hmaymatch{\hinl{\tau}{e}}{\hinlp{p}}
}

\Infer{\MMInr}{
  \hmaymatch{e}{p}
}{
  \hmaymatch{\hinr{\tau}{e}}{\hinrp{p}}
}
\end{mathpar}

\caption{Three possible outcomes of pattern matching}
\label{fig:patmatch}
\end{figure}

% !TEX root= pattern-paper.tex

\begin{figure}[ht]
\judgbox{
  \notIntro{e}
}{
  $e$ cannot be a value syntactically
}

\begin{mathpar}
\Infer{NVEHole}{ }{
  \notIntro{\hehole{u}}
}

\Infer{NVHole}{ }{
  \notIntro{\hhole{e}{u}}
}

\Infer{NVAp}{ }{
  \notIntro{\hap{e_1}{e_2}}
}

\Infer{NVMatch}{ }{
  \notIntro{\hmatch{e}{\zrules}}
}

\Infer{NVPrl}{ }{
  \notIntro{\hprl{e}}
}

\Infer{NVPrr}{ }{
  \notIntro{\hprr{e}}
}
\end{mathpar}

  \caption{Expressions that are syntactically not values}
  \label{fig:notintro}
\end{figure}%


\begin{lemma}[Matching Determinism]
  \label{lemma:match-determinism}
  If $\isFinal{e}$ and $\hexptyp{\cdot}{\Delta}{e}{\tau}$ and $\hpattyp{p}{\tau}{\Gamma}{\Delta}$ then exactly one of the following holds:
  \begin{enumerate*}
    \item $\hpatmatch{e}{p}{\theta}$ for some $\theta$;
    \item $\hmaymatch{e}{p}$; or
    \item $\hnotmatch{e}{p}$.
  \end{enumerate*}
\end{lemma}
From this lemma follow the determinism and progress theorems governing the stepping judgment, which are combined below.
\begin{theorem}[Deterministic Progress]
  \label{theorem:determinism}
  If $\hexptyp{\cdot}{\Delta}{e}{\tau}$ then exactly one of the following holds:
  \begin{enumerate*}
    \item $\isVal{e}$;
    \item $\isIndet{e}$; or 
    \item $\htrans{e}{e'}$ for some unique $e'$.
  \end{enumerate*}
\end{theorem}

Let us now consider each of the three possible outcomes in more detail.

%Rules \ITSuccMatch, \ITFailMatch, and \IMatch describe the cases where a match expression takes a step to evaluate the matched branch, move the rule cursor to the next branch due to match failure, and are already indeterminate due to indeterminate match. The second premises of these three rules corresponds to three possible outcomes of matching $e$ against $p$.

The judgment $\hpatmatch{e}{p}{\theta}$ denotes a successful match as witnessed by the substitution $\theta$ over the variables bound in $p$. This substitution is applied, written $[\theta](e_r)$, when the corresponding branch is taken in the conclusion of \ITSuccMatch. Notice that for an irrefutable pattern like $x$ and $\_$, all scrutinees match, including indeterminate scrutinees. 
Pair patterns are an interesting case. If the scrutinee is a pair, then \MPair matches the corresponding expressions and patterns. However, the scrutinee may also be an indeterminate expression of product type that is not syntactically a pair. Because pairs are the only introductory form of product type, we know that no matter how the scrutinee is eventually completed, it will eventually become a pair, so we can successfully match even in this situation. %
The \MNotIntroPair rule handles this case by taking the first and second projections of the scrutinee in any case where the scrutinee is not an introductory form of the language, defined by the judgment $\notIntro{e}$ in \autoref{fig:notintro}.

The judgment $\hnotmatch{e}{p}$ denotes that $e$ does not match $p$. In the case of such a mismatch,
\ITMisMatch moves the focus of the zipper to the next rule. Note that if there are no remaining rules, 
the operational semantics are undefined (i.e. we do not define run-time match failure judgmentally). This does not violate \autoref{theorem:determinism} because
in our static semantics in Sec.~\ref{sec:statics}, exhaustiveness will be a requirement. In settings where exhaustiveness checking
only generates a warning, it would be straightforward (but tedious) to add match failure errors and their corresponding propagation rules to the semantics.


% !TEX root= pattern-paper.tex

\begin{figure}[!ht]
\judgbox{
  \refutable{p}
}{$p$ is refutable}

\begin{mathpar}
\Infer{\RNum}{ }{
  \refutable{\hnum{n}}
}

\Infer{\REHole}{ }{
  \refutable{\heholep{w}}
}

\Infer{\RHole}{ }{
  \refutable{\hholep{p}{w}{\tau}}
}

\Infer{\RInl}{ }{
  \refutable{\hinlp{p}}
}

\Infer{\RInr}{ }{
  \refutable{\hinrp{p}}
}

\Infer{\RPairL}{
  \refutable{p_1}
}{
  \refutable{\hpair{p_1}{p_2}}
}

\Infer{\RPairR}{
  \refutable{p_2}
}{
  \refutable{\hpair{p_1}{p_2}}
}
\end{mathpar}
\caption{Refutable Patterns}
\label{fig:refutable}
\end{figure}
Finally, the judgment $\hmaymatch{e}{p}$ indicates an \emph{indeterminate match} due to the presence of holes in $e$ or $p$. \IMatch designates the entire match expression indeterminate once it has stepped to a rule where there is such an indeterminate match. \MMEHole and \MMHole specify that pattern holes always indeterminate match any scrutinee, because the hole has not been filled. \MMPairL, \MMPairR, and \MMPair specify that pairs indeterminately match as long as at least one of the corresponding projections of the scrutinee indeterminately match and neither mismatches. \MMInl and \MMInr specify that matching injections indeterminately match if the arguments indeterminately match. Finally, \MMNotIntro handles all situations where the scrutinee is not an introductory form (e.g. a hole or indeterminate function application), as specified in \autoref{fig:notintro}, and $p$ is indeterminately refutable, i.e. it is not a variable, wildcard, or pair with irrefutable patterns on both sides, as specified by the judgment $\refutable{p}$ defined in \autoref{fig:refutable}. (Irrefutable patterns lead to matches, as described above, so failing to exclude these would violate \autoref{lemma:match-determinism}.)

\subsection{Match Constraint Language}\label{sec:constraint}
We can now move on in this and the remaining subsections to the static semantics of Peanut. Before continuing to describe the typing judgments in \autoref{sec:statics}, we must introduce \emph{match constraints}, $\xi$,
which generalize patterns to form a simple first-order logic. Constraints are generated from patterns and rules during typechecking and are the structures on which we will perform exhaustiveness and redundancy checking. 

% !TEX root= pattern-paper.tex

\begin{figure}[ht]
$\arraycolsep=4pt\begin{array}{lll}
\xi & ::= &
  \ctruth ~\vert~
  \cfalsity ~\vert~
  \cunknown ~\vert~
  \cnum{n} ~\vert~
  \cnotnum{n} ~\vert~
  \cand{\xi_1}{\xi_2} ~\vert~
  \cor{\xi_1}{\xi_2} ~\vert~
  \cinl{\xi} ~\vert~
  \cinr{\xi} ~\vert~
  \cpair{\xi_1}{\xi_2}
\end{array}$

\judgbox{\ctyp{\xi}{\tau}}{$\xi$ constrains values of type $\tau$}

\begin{mathpar}
\Infer{\CTTruth}{ }{
  \ctyp{\ctruth}{\tau}
}

\Infer{\CTFalsity}{ }{
  \ctyp{\cfalsity}{\tau}
}

\Infer{\CTUnknown}{ }{
  \ctyp{\cunknown}{\tau}
}

\Infer{\CTNum}{ }{
  \ctyp{\cnum{n}}{\tnum}
}

\Infer{\CTNotNum}{ }{
  \ctyp{\cnotnum{n}}{\tnum}
}

\Infer{\CTAnd}{
  \ctyp{\xi_1}{\tau} \\ \ctyp{\xi_2}{\tau}
}{
  \ctyp{\cand{\xi_1}{\xi_2}}{\tau}
}

\Infer{\CTOr}{
  \ctyp{\xi_1}{\tau} \\ \ctyp{\xi_2}{\tau}
}{
  \ctyp{\cor{\xi_1}{\xi_2}}{\tau}
}

\Infer{\CTInl}{
  \ctyp{\xi_1}{\tau_1}
}{
  \ctyp{\cinl{\xi_1}}{\tsum{\tau_1}{\tau_2}}
}

\Infer{\CTInr}{
  \ctyp{\xi_2}{\tau_2}
}{
  \ctyp{\cinr{\xi_2}}{\tsum{\tau_1}{\tau_2}}
}

\Infer{\CTPair}{
  \ctyp{\xi_1}{\tau} \\ \ctyp{\xi_2}{\tau}
}{
  \ctyp{\cpair{\xi_1}{\xi_2}}{\tau}
}
\end{mathpar}

\judgbox{\cdual{\xi_1} = \xi_2}{dual of $\xi_1$ is $\xi_2$}
\begin{subequations}\label{defn:dual}
\begin{align}
  \cdual{\ctruth} &= \cfalsity \\
  \cdual{\cfalsity} &= \ctruth \\
  \cdual{\cunknown} &= \cunknown \\
  \cdual{\cnum{n}} &= \cnotnum{n} \\
  \cdual{\cnotnum{n}} &= \cnum{n} \\
  \cdual{\cand{\xi_1}{\xi_2}} &= \cor{\cdual{\xi_1}}{\cdual{\xi_2}} \\
  \cdual{\cor{\xi_1}{\xi_2}} &= \cand{\cdual{\xi_1}}{\cdual{\xi_2}} \\
  \cdual{\cinl{\xi_1}} &= \cor{ \cinl{\cdual{\xi_1}} }{ \cinr{\ctruth} } \\
  \cdual{\cinr{\xi_2}} &= \cor{ \cinr{\cdual{\xi_2}} }{ \cinl{\ctruth} } \\
  \cdual{\cpair{\xi_1}{\xi_2}} &=
  \cor{ \cor{ 
    \cpair{\xi_1}{\cdual{\xi_2}}
  }{
    \cpair{\cdual{\xi_1}}{\xi_2}
  }}{
    \cpair{\cdual{\xi_1}}{\cdual{\xi_2}}
  }
\end{align}
\end{subequations}
  \caption{Match Constraints}
  \label{fig:constraint}
\end{figure}

% !TEX root= pattern-paper.tex

\begin{figure}[!ht]

\judgbox{\csatisfy{e}{\xi}}{$e$ satisfies $\xi$}

\begin{mathpar}
\Infer{\CSTruth}{ }{
  \csatisfy{e}{\ctruth}
}

\Infer{\CSNum}{ }{
  \csatisfy{\hnum{n}}{\cnum{n}}
}

\Infer{\CSNotNum}{
  n_1 \neq n_2
}{
  \csatisfy{\hnum{n_1}}{\cnotnum{n_2}}
}

\Infer{\CSAnd}{
  \csatisfy{e}{\xi_1} \\
  \csatisfy{e}{\xi_2}
}{
  \csatisfy{e}{\cand{\xi_1}{\xi_2}}
}

\Infer{\CSOrL}{
  \csatisfy{e}{\xi_1}
}{
  \csatisfy{e}{\cor{\xi_1}{\xi_2}}
}

\Infer{\CSOrR}{
  \csatisfy{e}{\xi_2}
}{
  \csatisfy{e}{\cor{\xi_1}{\xi_2}}
}

\Infer{\CSInl}{
  \csatisfy{e_1}{\xi_1}
}{
  \csatisfy{
    \hinl{\tau_2}{e_1}
  }{
    \cinl{\xi_1}
  }
}

\Infer{\CSInr}{
  \csatisfy{e_2}{\xi_2}
}{
  \csatisfy{
    \hinr{\tau_1}{e_2}
  }{
    \cinr{\xi_2}
  }
}

\Infer{\CSPair}{
  \csatisfy{e_1}{\xi_1} \\
  \csatisfy{e_2}{\xi_2}
}{
\csatisfy{\hpair{e_1}{e_2}}{\cpair{\xi_1}{\xi_2}}
}

\Infer{\CSNotIntroPair}{
  \notIntro{e} \\
  \csatisfy{\hprl{e}}{\xi_1} \\
  \csatisfy{\hprr{e}}{\xi_2}
}{
  \csatisfy{e}{\cpair{\xi_1}{\xi_2}}
}
\end{mathpar}

\judgbox{\cmaysatisfy{e}{\xi}}{$e$ may satisfy $\xi$}

\begin{mathpar}
\Infer{\CMSUnknown}{ }{
  \cmaysatisfy{e}{\cunknown}
}

\Infer{\CMSNotIntro}{
  \notIntro{e} \\
  \refutable{\xi}
}{
  \cmaysatisfy{e}{\xi}
}

\Infer{\CMSAndL}{
  \cmaysatisfy{e}{\xi_1} \\
  \csatisfy{e}{\xi_2}
}{
  \cmaysatisfy{e}{\cand{\xi_1}{\xi_2}}
}

\Infer{\CMSAndR}{
  \csatisfy{e}{\xi_1} \\
  \cmaysatisfy{e}{\xi_2}
}{
  \cmaysatisfy{e}{\cand{\xi_1}{\xi_2}}
}

\Infer{\CMSAnd}{
  \cmaysatisfy{e}{\xi_1} \\
  \cmaysatisfy{e}{\xi_2}
}{
  \cmaysatisfy{e}{\cand{\xi_1}{\xi_2}}
}

\Infer{\CMSOrL}{
  \cmaysatisfy{e}{\xi_1} \\
  \cnotsatisfy{e}{\xi_2}
}{
  \cmaysatisfy{e}{\cor{\xi_1}{\xi_2}}
}

\Infer{\CMSOrR}{
  \cnotsatisfy{e}{\xi_1} \\
  \cmaysatisfy{e}{\xi_2}
}{
  \cmaysatisfy{e}{\cor{\xi_1}{\xi_2}}
}

\Infer{\CMSInl}{
  \cmaysatisfy{e_1}{\xi_1}
}{
  \cmaysatisfy{
    \hinl{\tau_2}{e_1}
  }{
    \cinl{\xi_1}
  }
}

\Infer{\CMSInr}{
  \cmaysatisfy{e_2}{\xi_2}
}{
  \cmaysatisfy{
    \hinr{\tau_1}{e_2}
  }{
    \cinr{\xi_2}
  }
}

\Infer{\CMSPairL}{
  \cmaysatisfy{e_1}{\xi_1} \\
  \csatisfy{e_2}{\xi_2}
}{
  \cmaysatisfy{\hpair{e_1}{e_2}}{\cpair{\xi_1}{\xi_2}}
}

\Infer{\CMSPairR}{
  \csatisfy{e_1}{\xi_1} \\
  \cmaysatisfy{e_2}{\xi_2}
}{
  \cmaysatisfy{\hpair{e_1}{e_2}}{\cpair{\xi_1}{\xi_2}}
}

\Infer{\CMSPair}{
  \cmaysatisfy{e_1}{\xi_1} \\
  \cmaysatisfy{e_2}{\xi_2}
}{
  \cmaysatisfy{\hpair{e_1}{e_2}}{\cpair{\xi_1}{\xi_2}}
}
\end{mathpar}

\judgbox{\csatisfyormay{e}{\xi}}{$e$ satisfies or may satisfy $\xi$}

\begin{mathpar}
\Infer{\CSMSMay}{
  \cmaysatisfy{e}{\xi}
}{
  \csatisfyormay{e}{\xi}
}

\Infer{\CSMSSat}{
  \csatisfy{e}{\xi}
}{
  \csatisfyormay{e}{\xi}
}
\end{mathpar}

  \caption{Satisfaction Judgments}
  \label{fig:satisfy}
\end{figure}

% !TEX root= pattern-paper.tex

\begin{figure}[!ht]
\judgbox{
  \possible{\xi}
}{$\xi$ is possible}

\begin{mathpar}
\Infer{\PTruth}{ }{
  \possible{\ctruth}
}

\Infer{\PUnknown}{ }{
  \possible{\cunknown}
}

\Infer{\PNum}{ }{
  \possible{\cnum{n}}
}

\Infer{\PNotNum}{ }{
  \possible{\cnotnum{n}}
}

\Infer{\PInl}{
  \possible{\xi}
}{
  \possible{\cinl{\xi}}
}

\Infer{\PInr}{
  \possible{\xi}
}{
  \possible{\cinr{\xi}}
}

\Infer{\PPair}{
  \possible{\xi_1} \\ \possible{\xi_2}
}{
  \possible{\cpair{\xi_1}{\xi_2}}
}

\Infer{\POrL}{
  \possible{\xi_1}
}{
  \possible{\cor{\xi_1}{\xi_2}}
}

\Infer{\POrR}{
  \possible{\xi_2}
}{
  \possible{\cor{\xi_1}{\xi_2}}
}

\end{mathpar}
\caption{Possible Constraints}
\label{fig:xi-possible}
\end{figure}

\autoref{fig:constraint} defines the syntax of the match constraint language. 
Excluding the unknown constraint $\cunknown$ for a moment, a constraint $\xi$ identifies a satisfying subset of final expressions of some type $\tau$ as specified by the constraint typing judgment,  $\ctyp{\xi}{\tau}$, in \autoref{fig:constraint}.
Assuming $\ctyp{\xi}{\tau}$, then the dual of $\xi$, denoted by $\cdual{\xi}$ and defined also in \autoref{fig:constraint}, represents the complement of the subset identified by $\xi$ under the set of final expressions of type $\tau$.
We must also model indeterminacy in our constraint language. So just as there are three possible outcomes of matching a final expression against a pattern, there are three possible relationships between a final expression $e$ and a constraint $\xi$, assuming they can be checked against the same type---$e$ necessarily satisfies $\xi$, $e$ indeterminately satisfies $\xi$, and $e$ necessarily does not satisfy $\xi$. 
%% !TEX root= pattern-paper.tex

\begin{figure}[!ht]
\judgbox{
  \refutable{\xi}
}{$\xi$ is refutable}

\begin{mathpar}
\Infer{RXFalsity}{ }{
  \refutable{\cfalsity}
}

\Infer{\RXUnknown}{ }{
  \refutable{\cunknown}
}

\Infer{\RXNum}{ }{
  \refutable{\cnum{n}}
}

\Infer{\RXNotNum}{ }{
  \refutable{\cnotnum{n}}
}

\Infer{\RXInl}{ }{
  \refutable{\cinl{\xi}}
}

\Infer{\RXInr}{ }{
  \refutable{\cinr{\xi}}
}

\Infer{\RXPairL}{
  \refutable{\xi_1}
}{
  \refutable{\cpair{\xi_1}{\xi_2}}
}

\Infer{\RXPairR}{
  \refutable{\xi_2}
}{
  \refutable{\cpair{\xi_1}{\xi_2}}
}
\end{mathpar}
\caption{Refutable Constraints}
\label{fig:xi-refutable}
\end{figure}

\autoref{fig:satisfy} defines the satisfaction judgments that encode such relationships. Assuming
final expression $e$ and constraint $\xi$ are of the same type, the
judgment $\csatisfy{e}{\xi}$ specifies that $e$ satisfies $\xi$ while
the judgment $\cmaysatisfy{e}{\xi}$ specifies that $e$ indeterminately satisfies $\xi$. Both judgments closely follow the definition of pattern matching judgments $\hpatmatch{e}{p}{\theta}$ and $\hmaymatch{e}{p}$, adding disjunction and conjunction. The main differences of note are in Rule \CSNotIntroPair and Rule \CMSNotIntro. The refutability premise follows the one in the dynamics (the definition of $\refutable{\xi}$ is in the Supplementary Material). The \CMSNotIntro rule includes a third premise, $\possible{\xi}$. This is due to the presence of a bottom constraint, $\cfalsity$, which does not correspond to a pattern but rather is the dual of the top constraint, $\ctruth$, which corresponds to variable and wildcard patterns. 
$\possible{\xi}$ rules out the cases where $\xi$ contains $\cfalsity$. $\refutable{\xi}$ and $\possible{\xi}$ together says that constraint $\xi$ is neither necessarily irrefutable nor necessarily refuted.
The judgment $\csatisfyormay{e}{\xi}$ is simply the disjunction of necessary and indeterminate satisfaction. 
We say $e$ does not satisfy $\xi$ when $\csatisfyormay{e}{\xi}$ is not derivable, written $\cnotsatisfyormay{e}{\xi}$.

\autoref{lemma:const-matching-coherence} establishes a correspondence
between pattern matching results and satisfaction judgments, which makes
reasoning about pattern matching in the type system possible and helps prove Preservation (\autoref{theorem:preservation}).

\begin{lemma}[Matching Coherence of Constraint]
  \label{lemma:const-matching-coherence}
  If $\hexptyp{\cdot}{\Delta_e}{e}{\tau}$ and $\isFinal{e}$ and $\chpattyp{p}{\tau}{\xi}{\Gamma}{\Delta}$ then
  \begin{enumerate*}
  \item $\csatisfy{e}{\xi}$ iff $\hpatmatch{e}{p}{\theta}$ for some $\theta$; and 
  \item $\cmaysatisfy{e}{\xi}$ iff $\hmaymatch{e}{p}$; and 
  \item $\cnotsatisfyormay{e}{\xi}$ iff $\hnotmatch{e}{p}$.
  \end{enumerate*}
\end{lemma}

And the exclusiveness of satisfaction judgments directly follows from the exclusiveness of pattern matching (\autoref{lemma:match-determinism}).
\begin{theorem}[Exclusiveness of Constraint Satisfaction]
  \label{theorem:exclusive-constraint-satisfaction}
  If $\ctyp{\xi}{\tau}$ and $\hexptyp{\cdot}{\Delta}{e}{\tau}$ and $\isFinal{e}$ then exactly one of the following holds:
  \begin{enumerate*}
  \item $\csatisfy{e}{\xi}$; or
  \item $\cmaysatisfy{e}{\xi}$; or 
  \item $\cnotsatisfyormay{e}{\xi}$.
  \end{enumerate*}
\end{theorem}

Exhaustiveness and redundancy checking can be reduced to the problem of deciding \emph{entailments} between two constraints. We need to define two kinds of entailment: indeterminate entailment and potential entailment.
Indetermine entailment is defined as follows.
\begin{definition}[Indeterminate Entailment of Constraints]
  \label{definition:const-entailment}
  Suppose that $\ctyp{\xi_1}{\tau}$ and $\ctyp{\xi_2}{\tau}$.
  Then $\csatisfy{\xi_1}{\xi_2}$ iff for all $e$ such that $\hexptyp{\cdot}{\Delta}{e}{\tau}$ and $\isVal{e}$ we have $\csatisfyormay{e}{\xi_1}$ implies $\csatisfy{e}{\xi_2}$
\end{definition}
Indeterminate entailment allows us to reason about redundancy, namely we need $\cnotsatisfy{\xi_r}{\xi_{pre}}$ to ensure the constraint for the branch $r$ of interest is not redundant with respect to its preceding branches, conjoined in $rs_{pre}$ (see Rules \TOneRules and \TRules in the next section).
The branch $r$ would never be taken only when for all values that determinately or indeterminately match the pattern in $r$, they determinately match the pattern in one of the previous branches.


Potential entailment is defined as follows.
\begin{definition}[Potential Entailment of Constraints]
  \label{definition:nn-entailment}
  Suppose that $\ctyp{\xi_1}{\tau}$ and $\ctyp{\xi_2}{\tau}$. Then $\csatisfyormay{\xi_1}{\xi_2}$ iff for all $e$ such that $\hexptyp{\cdot}{\Delta}{e}{\tau}$ and $\isFinal{e}$ we have $\csatisfyormay{e}{\xi_1}$ implies $\csatisfyormay{e}{\xi_2}$ 
\end{definition}
Potential entailment will be used for exhaustiveness checking. In particular, if we choose $\xi_1$ to be $\ctruth$, then the entailment $\csatisfyormay{\ctruth}{\xi}$ ensures that $\xi$ is either necessarily or indeterminately exhaustive (we will see this appear in Rules \TMatchZPre and \TMatchNZPre). 
The type checker should give an error/warning about the inexhaustiveness only when a case must have been missed, i.e. when there is a value that does not match any rule's pattern, no matter how the programmer may fill the holes in the program. That is, a match expression is indeterminately exhaustive if all values indeterminately match one of its constituent rules' patterns.
We consider final expressions in the definition of potential entailment, rather than values as with indeterminate entailment. 
In this way, even when the scrutinee $e$ is indeterminate, we can still be confident that either one of the branches will be taken or evaluation will stop on one branch due to the indeterminacy of pattern matching with holes. As we will later show in \autoref{sec:algorithm}, this is equivalent to quantifying over all potential values of a final expression.


\subsection{Typing, Exhaustiveness, and Irredundancy}\label{sec:statics}

% !TEX root= pattern-paper.tex

\begin{figure}[ht]
\judgbox{
  \hexptyp{\Gamma}{\Delta}{e}{\tau}
}{
  $e$ is of type \(\tau\)
}
  \begin{mathpar}
%  \Infer{\TVar}{ }{
%    \hexptyp{\Gamma, x:\tau}{\Delta}{x}{\tau}
%  }
%
%  \Infer{\TEHole}{ }{
%    \hexptyp{\Gamma}{\Delta, u::\tau}{\hehole{u}}{\tau}
%  }
%
%  \Infer{\THole}{
%    \hexptyp{\Gamma}{\Delta, u::\tau}{e}{\tau'}
%  }{
%    \hexptyp{\Gamma}{\Delta, u::\tau}{\hhole{e}{u}}{\tau}
%  }
%
%  \Infer{\TNum}{ }{
%    \hexptyp{\Gamma}{\Delta}{\hnum{n}}{\tnum}
%  }
%
%  \Infer{\TLam}{
%    \hexptyp{\Gamma, x:\tau_1}{\Delta}{e}{\tau_2}
%  }{
%    \hexptyp{\Gamma}{\Delta}{\hlam{x}{\tau_1}{e}}{\tarr{\tau_1}{\tau_2}}
%  }
%
%  \Infer{\TAp}{
%    \hexptyp{\Gamma}{\Delta}{e_1}{\tarr{\tau_2}{\tau}} \\
%    \hexptyp{\Gamma}{\Delta}{e_2}{\tau_2}
%  }{
%    \hexptyp{\Gamma}{\Delta}{\hap{e_1}{e_2}}{\tau}
%  }
%
%  \Infer{\TPair}{
%    \hexptyp{\Gamma}{\Delta}{e_1}{\tau_1} \\
%    \hexptyp{\Gamma}{\Delta}{e_2}{\tau_2}
%  }{
%    \hexptyp{\Gamma}{\Delta}{\hpair{e_1}{e_2}}{\tprod{\tau_1}{\tau_2}}
%  }
%
%  \Infer{\TFst}{
%    \hexptyp{\Gamma}{\Delta}{e}{\tprod{\tau_1}{\tau_2}}
%  }{
%    \hexptyp{\Gamma}{\Delta}{\hfst{e}}{\tau_1}
%  }
%  
%  \Infer{\TSnd}{
%    \hexptyp{\Gamma}{\Delta}{e}{\tprod{\tau_1}{\tau_2}}
%  }{
%    \hexptyp{\Gamma}{\Delta}{\hsnd{e}}{\tau_2}
%  }
%  
%  \Infer{\TInl}{
%    \hexptyp{\Gamma}{\Delta}{e}{\tau_1}
%  }{
%    \hexptyp{\Gamma}{\Delta}{\hinl{\tau_2}{e}}{\tsum{\tau_1}{\tau_2}}
%  }
%
%  \Infer{\TInr}{
%    \hexptyp{\Gamma}{\Delta}{e}{\tau_2}
%  }{
%    \hexptyp{\Gamma}{\Delta}{\hinr{\tau_1}{e}}{\tsum{\tau_1}{\tau_2}}
%  }
  
  \Infer{TMatchZPre}{
    \hexptyp{\Gamma}{\Delta}{e}{\tau} \\
    \chrulstyp{\Gamma}{\Delta}{\cfalsity}{\hrules{r}{rs}}{\tau}{\xi}{\tau'} \\
    \csatisfyormay{\ctruth}{\xi}
  }{
  \hexptyp{\Gamma}{\Delta}{\hmatch{e}{\zruls{\cdot}{r}{rs}}}{\tau'}
  }

  \Infer{TMatchNZPre}{
    \hexptyp{\Gamma}{\Delta}{e}{\tau} \\
    \isFinal{e} \\
    \chrulstyp{\Gamma}{\Delta}{\cfalsity}{rs_{pre}}{\tau}{\xi_{pre}}{\tau'} \\
    \chrulstyp{\Gamma}{\Delta}{\xi_{pre}}{\hrules{r}{rs_{post}}}{\tau}{\xi_{rest}}{\tau'} \\
    \cnotsatisfyormay{e}{\xi_{pre}} \\
    \csatisfyormay{\ctruth}{\cor{\xi_{pre}}{\xi_{rest}}}
  }{
    \hexptyp{\Gamma}{\Delta}{\hmatch{e}{\zruls{rs_{pre}}{r}{rs_{post}}}}{\tau'}
  }
  \end{mathpar}

\caption{Match Expression Typing}
\label{fig:exptyp}
\end{figure}


We can now describe the type system of Peanut, which utilizes entailments between constraints to enforce exhaustiveness and irredundancy.
  Specifically, we build a similar type system to
\cite{DBLP:journals/pacmpl/OmarVCH19} by defining typing judgments for both
expressions and patterns. The expression typing judgment, $\hexptyp{\Gamma}{\Delta}{e}{\tau}$, is defined in \autoref{fig:exptyp}. It takes both a variable context $\Gamma$ and a hole context $\Delta$ as input, and produces a type as output. The pattern typing judgment, $\chpattyp{p}{\tau}{\xi}{\Gamma}{\Delta}$, is defined in \autoref{fig:pat-rulestyp}. It takes a hole context $\Delta$ and a type as input and produces a variable context $\Gamma$ and a constraint, $\xi$. \autoref{fig:pat-rulestyp} also defines typing judgments for rules, which we consider below.

\subsubsection{Typing of Expressions and Exhaustiveness Checking} \label{sec:exptyp}

We start by specifying the typing of expressions in \autoref{fig:exptyp}, giving particular attention to match expressions. We will see that our definitions enforce exhaustiveness of the constituent branches of any match expression. Alternatively, one could include exhaustiveness as a separate judgment, as is done in our Agda mechanization, but we integrate them for simplicity in our presentation.

Rule \TMatchZPre corresponds to the case when we are yet to start pattern
matching, i.e. the rules zipper has no prefix. The first premise specifies that the scrutinee $e$ is of type $\tau$,
and the second premise specifies that the constituent rules $\hrules{r}{rs}$ are not only
well-typed but also transform a final expression of the same type as the
scrutinee, into a final expression of type $\tau'$. 
Typing of a sequence of rules takes an initial constraint as input and produces a final constraint for the rules, 
which is the disjunction of the constituent constraints (see \autoref{fig:pat-rulestyp}). 
As the inductive base case, we supply $\cfalsity$ as the initial constraint (it can be dropped from any disjunction without changing its meaning). 
The 
third premise $\csatisfyormay{\ctruth}{\xi}$ checks for necessary or indeterminate exhaustiveness. 
In other words, for a well-typed match expression,
it is impossible that the final scrutinee would determinately fail to match all the patterns as we consider branches $rs$ in order.
Intuitively, constraint $\xi$ encodes the set of possible expressions that determinately or indeterminately match any of the branches and the entailment from $\ctruth$ ensures that all expressions are considered.

Rule \TMatchNZPre corresponds to the case that we have already started pattern
matching and have already considered preceding rules $rs_{pre}$. First of all,
the scrutinee should not only be well-typed but also be final. Next, in addition to
ensuring the exhaustiveness of the constituent rules, we want to make sure that
at least one of the remaining rules $r$ or $rs_{post}$ would be taken. Note
that only when the final scrutinee $e$ determinately mismatches the pattern $p$, \ie,
$\hnotmatch{e}{p}$, can we move the rule pointer. By
\autoref{lemma:match-determinism}, for any pattern $p$ in the preceding
branches, neither $\hpatmatch{e}{p}{\theta}$ nor $\hmaymatch{e}{p}$ is derivable.
Then by \autoref{lemma:const-matching-coherence}, we can derive the premise
in Rule \TMatchNZPre, $\cnotsatisfyormay{e}{\xi_{pre}}$. And thus, the type of
the match expression is preserved (\autoref{theorem:preservation}) as we consider rules in order.
%\todo{d: This section needs to be much more clear as to how it relates to exhaustiveness checking. There should be some summarizing idea at the top before getting into the details of the rules. It might help to have some discussion of our overall approach using constraints at the top before talking about the specifics of exhaustiveness or redundancy checks. As it stands, constraints kinda come out of nowhere.}

\subsubsection{Typing of Patterns and Rules, and Redundancy Checking}
\label{sec:pattyp}

% !TEX root= pattern-paper.tex

\begin{figure}[!ht]
  \judgbox{
    \chpattyp{p}{\tau}{\xi}{\Gamma}{\Delta}
  }{
    $p$ is assigned type $\tau$ and emits constraint $\xi$
  }

  \begin{mathpar}
  \Infer{\PTVar}{ }{
    \chpattyp{x}{\tau}{\ctruth}{x : \tau}{\cdot}
  }

  \Infer{\PTWild}{ }{
    \chpattyp{\_}{\tau}{\ctruth}{\cdot}{\cdot}
  }

  \Infer{\PTEHole}{ }{
    \chpattyp{\heholep{w}}{\tau}{\cunknown}{\cdot}{\Delta , w :: \tau}
  }

  \Infer{\PTHole}{
    \chpattyp{p}{\tau}{\xi}{\Gamma}{\Delta , w :: \tau'}
  }{
    \chpattyp{\hholep{p}{w}{\tau}}{\tau'}{\cunknown}
    {\Gamma}{\Delta , w :: \tau'}
  }
  
  \Infer{\PTNum}{ }{
    \chpattyp{\hnum{n}}{\tnum}{\cnum{n}}{\cdot}{\Delta}
  }

  \Infer{\PTInl}{
    \chpattyp{p}{\tau_1}{\xi}{\Gamma}{\Delta}
  }{
    \chpattyp{\hinlp{p}}{\tsum{\tau_1}{\tau_2}}{\cinl{\xi}}{\Gamma}{\Delta}
  }

  \Infer{\PTInr}{
    \chpattyp{p}{\tau_2}{\xi}{\Gamma}{\Delta}
  }{
    \chpattyp{\hinrp{p}}{\tsum{\tau_1}{\tau_2}}{\cinr{\xi}}{\Gamma}{\Delta}
  }

  \Infer{\PTPair}{
    \chpattyp{p_1}{\tau_1}{\xi_1}{\Gamma_1}{\Delta} \\
    \chpattyp{p_2}{\tau_2}{\xi_2}{\Gamma_2}{\Delta}
  }{
    \chpattyp{\hpair{p_1}{p_2}}{\tprod{\tau_1}{\tau_2}}
    {\cpair{\xi_1}{\xi_2}}{\Gamma_1 \uplus \Gamma_2}{\Delta}
  }

  \end{mathpar}

  \judgbox{
    \chrultyp{\Gamma}{\Delta}{r}{\tau}{\xi}{\tau'}
  }{$r$ transforms a final expression of type $\tau$ \\ to a final expression of type $\tau'$}

  \begin{mathpar}
    \Infer{\TRule}{
      \chpattyp{p}{\tau}{\xi}{\Gamma_p}{\Delta_p} \\
      \hexptyp{\Gamma \uplus \Gamma_p}{\Delta \uplus \Delta_p}{e}{\tau'}
    }{
      \chrultyp{\Gamma}{\Delta}{\hrulP{p}{e}}{\tau}{\xi}{\tau'}
    }
  \end{mathpar}

  \judgbox{\chrulstyp{\Gamma}{\Delta}{\xi_{pre}}{rs}{\tau}{\xi_{rs}}{\tau'}}
  {$rs$ transforms a final expression of type $\tau$ \\ to a final expression of type $\tau'$}

  \begin{mathpar}
  \Infer{TOneRules}{
    \chrultyp{\Gamma}{\Delta}{r}{\tau}{\xi_r}{\tau'} \\
    \cnotsatisfy{\xi_r}{\xi_{pre}}
  }{
    \chrulstyp{\Gamma}{\Delta}{\xi_{pre}}{\hrulesP{r}{\cdot}}{\tau}{\xi_r}{\tau'}
  }

  \Infer{TRules}{
    \chrultyp{\Gamma}{\Delta}{r}{\tau}{\xi_r}{\tau'} \\
    \chrulstyp{\Gamma}{\Delta}{\cor{\xi_{pre}}{\xi_r}}{rs}
    {\tau}{\xi_{rs}}{\tau'} \\
    \cnotsatisfy{\xi_r}{\xi_{pre}}
  }{
    \chrulstyp{\Gamma}{\Delta}{\xi_{pre}}{\hrules{r}{rs}}
    {\tau}{\cor{\xi_r}{\xi_{rs}}}{\tau'}
  }
\end{mathpar}
\caption{Typing of Patterns, Single Rules, and Series of Rules}
\label{fig:pat-rulestyp}
\end{figure}


\autoref{fig:pat-rulestyp} defines the typing judgment for patterns $p$, a
single rule $r$, and a series of rules $rs$. We will describe the emission of constraints from and their usage in checking redundancy of a rule $r$ with respect
to its preceding ones.
%\todo{d: Similarly this summarizing sentence needs to be more simply
%stated and perhaps more descriptive as to what will follow, not just leaning
%on the phrase ``we will see'' to defer responsibility to later details)}

The typing judgment of series of rules $rs$ is of the form
$\chrulstyp{\Gamma}{\Delta}{\xi_{pre}}{rs}{\tau_1}{\xi_{rs}}{\tau_2}$. As shown
in Rules \TMatchZPre and \TMatchNZPre, the constituent rules inherit the
variable context $\Gamma$ and hole context $\Delta$ from the outer match
expression. When we check the type of a series of rules, we consider each rule
in order, just as how we do pattern matching in \autoref{sec:dynamics}.

Rule \TRules corresponds to the inductive case. The first premise is to check
the type of the initial rule $r$. It specifies that each rule takes a final expression
of type $\tau$ and returns a final expression of type $\tau'$. It also emits
a constraint $\xi_r$, which is actually emitted from the pattern of rule $r$ as
we will see later. In order to determine if the initial rule $r$ of the rules
$\hrules{r}{rs}$ is redundant with respect to its preceding rules, we use
$\xi_{pre}$ to keep track of the pattern matching information of preceding
rules. To accomplish that, as we drop the initial rule $r$, we append the
constraint $\xi_r$ emitted from the pattern of $r$, to the constraint
$\xi_{pre}$, and use $\cor{\xi_{pre}}{\xi_r}$ as the new input to inductively check the type
of the trailing rules $rs$ in the second premise. Now that we have shown how to maintain the
constraint $\xi_{pre}$ associated with the preceding rules, we can compare it
with the constraint of the current rule, $\xi_r$. As we type check rules, we consider each rule in order and use
$\cnotsatisfy{\xi_r}{\xi_{pre}}$ to ensure that the current rule $r$ is not necessarily redundant with respect to its preceding rules. At the same time, the judgment also outputs the accumulated
constraint collected from rules $\hrules{r}{rs}$, which helps exhaustiveness checking
, as we have shown in \autoref{sec:exptyp}

Rule \TOneRules corresponds to the single rule case. The premises are similar to that of Rule \TRules except that
there is no trailing rules to check the type of. The reason why we regard one
rule as the base case instead of empty rules, is that since our match expression
takes zipper rules, we will never need to check the type of empty rule set. The
only case that it makes sense to allow a match expression with no rule, is when we match on a final expression of \textit{Void} type and thus
don't need to worry about exhaustiveness checking. It turns out that we do not
have to sacrifice the generality.

As mentioned above, Rule \TRule specifies that rule
$\hrul{p}{e}$ transforms final expressions of type $\tau_1$ to final expressions
of type $\tau_2$. The first premise is the typing judgment of patterns---by
assigning pattern $p$ with type $\tau_1$, we collect the typing for all the
variables and holes involved in the pattern $p$ and generate variable context
$\Gamma_p$ and hole context $\Delta_p$. Additionally, it emits constraint $\xi$,
which is closely associated with the pattern itself. For example, a pattern hole emits an 
unknown constraint $\cunknown$, to indicate that the set of final expressions
matching $p$ is yet to be determined (Rules \PTEHole and \PTHole). The second premise strictly extends two contexts of
rule $r$ with bindings from pattern $p$, and checks the type of
sub-expression $e$.

\subsubsection{Type Safety}
The type safety of the language is established by
\autoref{theorem:determinism} and \autoref{theorem:preservation}.

\begin{theorem}[Preservation]
  \label{theorem:preservation}
  If $\hexptyp{\cdot}{\Delta}{e}{\tau}$ and $\htrans{e}{e'}$
  then $\hexptyp{\cdot}{\Delta}{e'}{\tau}$.
\end{theorem}

% % !TEX root= pattern-paper.tex

\begin{figure}[ht]

\judgbox{
  \hsubstyp{\theta}{\Gamma}
}{
  $\theta$ is of type $\Gamma$
}

\begin{mathpar}
\Infer{\STEmpty}{ }{
  \hsubstyp{\emptyset}{\cdot}
}

\Infer{\STExtend}{
  \hsubstyp{\theta}{\Gamma_\theta} \\
  \hexptyp{\Gamma}{\Delta}{e}{\tau}
}{
  \hsubstyp{\theta , x / e}{\Gamma_\theta , x : \tau}
}
\end{mathpar}

\caption{Typing of Substitution}
\label{fig:substyp}
\end{figure}


% \figurename~\ref{fig:substyp} defines the typing of substitution $\theta$.

To prove \autoref{theorem:preservation}, we need standard substitution lemmas, for both single substitutions and simultaneous substitutions $\theta$ (see the appendix or mechanization).
% When considering Rule \ITAp, \autoref{lemma:substitution} is needed.
% When considering Rule \ITSuccMatch, \autoref{lemma:subs-typing} is needed
% to show the typing of substitution $\theta$, and we use
% \autoref{lemma:simult-substitution} to show that type is preserved when pattern
% matching succeeds.

% \begin{lemma}[Substitution]
%   \label{lemma:substitution}
%   If $\hexptyp{\Gamma, x : \tau}{\Delta}{e_0}{\tau_0}$ and $\hexptyp{\Gamma}{\Delta}{e}{\tau}$ and $\isFinal{e}$
%   then $\hexptyp{\Gamma}{\Delta}{[e/x]e_0}{\tau_0}$
% \end{lemma}

% \begin{lemma}[Substitution Typing]
%   \label{lemma:subs-typing}
%   If $\hpatmatch{e}{p}{\theta}$ and $\hexptyp{\cdot}{\Delta_e}{e}{\tau}$ and $\chpattyp{p}{\tau}{\xi}{\Gamma}{\Delta}$ then $\hsubstyp{\cdot}{\Delta_e}{\theta}{\Gamma}$
% \end{lemma}

% \begin{lemma}[Simultaneous Substitution]
%   \label{lemma:simult-substitution}
%   If $\hexptyp{\Gamma \uplus \Gamma'}{\Delta}{e}{\tau}$ and $\hsubstyp{\Gamma}{\Delta}{\theta}{\Gamma'}$ and all expressions in $\theta$ are final
%   then $\hexptyp{\Gamma}{\Delta}{[\theta]e}{\tau}$
% \end{lemma}

\subsection{Elimination of Indeterminacy}\label{sec:algorithm}

\autoref{sec:statics} has already described exhaustiveness checking and redundancy checking using constraint entailments -- \Autoref{definition:const-entailment,definition:nn-entailment} --- defined in \autoref{sec:constraint}. In order for the type system to be decidable, we need to show that the constraint entailments, where both expressions and constraints may involve holes, are decidable. This is nontrivial. But the validity of a complete constraint, as defined in \autoref{definition:valid-constraint}, that only concerns values is decidable, as we will show shortly in \autoref{sec:decidability}.
%
In this section, we eliminate the indeterminacy in constraint entailments by showing that they can be reduced to \autoref{definition:valid-constraint}. 

\begin{definition}[Validity of Complete Constraints]
  \label{definition:valid-constraint}
  Suppose that $\ctruify{\xi}=\xi$ and $\ctyp{\xi}{\tau}$.
  Then $\ccsatisfy{}{\xi}$ iff for all $e$ such that $\hexptyp{\cdot}{\Delta}{e}{\tau}$ and $\isVal{e}$ we have $\ccsatisfy{e}{\xi}$
\end{definition}


As discussed in previous sections, only when a match expression is necessarily inexhaustive, no matter how the programmer fills it, shall the type checking fail. As a human being, such a process can be naturally performed by thinking of all the pattern holes as wildcards and checking if the transformed patterns cover all the possible cases. 

Similarly, if we are going to reason about the redundancy of one branch with respect to its previous ones in the presence of pattern holes, the programmer should only get an error/warning if that branch will never be matched no matter how the pattern holes are filled. It is natural to imagine all the previous branches with pattern holes would be refuted during run-time as the programmer may fill them with any refutable patterns. On the other hand, it makes sense to have the branch of our interest to cover as many cases as possible, by thinking of any of its pattern holes as wild cards.% Then, we may see if there is a case that is covered by the branch but not its previous branches.

% The justification of falsifying an unknown constraint is that there are infinite inhabitants in arbitrary type of Peanut. Let's say we are checking the redundancy of pattern $p$, as we have number as our base type, no matter what $p$ is, we can assume any incomplete patterns in its preceding branches can be completed in a way such that $p$ does not cover all the cases covered by the completed pattern.
% What if we add unit type? (We actually need unit type to represent empty list!) Then we no longer have infinite patterns that is of arbitrary given type. We can simply emit Truth constraint $\ctruth$ when assigning unit type to pattern holes.

As discussed in previous sections, we use a constraint language to capture the semantics of patterns, and then reason about the exhaustiveness and redundancy by reasoning about the constraint language. So we implement the aforementioned intuition through transformation on constraints. Specifically, as a pattern hole emits an unknown constraint $\cunknown$, replacing the pattern hole with a wildcard corresponds to "truify" $\cunknown$, \ie, turning $\cunknown$ into $\ctruth$. Similarly, we turn $\cunknown$ into $\cfalsity$ -- dubbed "falsify" -- to encode the idea of replacing the pattern hole with some pattern that a specific set of expressions won't match. Both "truify" operation $\ctruify{\xi}$ and "falsify" operation $\cfalsify{\xi}$ are defined as functions in \autoref{fig:truify-falsify}. 

% !TEX root= pattern-paper.tex

\begin{figure}[ht]
\judgbox{\ctruify{\xi_1} = \xi_2}{}

\begin{align*}
  \ctruify{\ctruth} &= \ctruth \\
  \ctruify{\cfalsity} &= \cfalsity \\
  \ctruify{\cunknown} &= \ctruth \\
  \ctruify{\cnum{n}} &= \cnum{n} \\
  \ctruify{\cnotnum{n}} &= \cnotnum{n} \\
  \ctruify{\cand{\xi_1}{\xi_2}} &= \cand{\ctruify{\xi_1}}{\ctruify{\xi_2}} \\
  \ctruify{\cor{\xi_1}{\xi_2}} &= \cor{\ctruify{\xi_1}}{\ctruify{\xi_2}} \\
  \ctruify{\cinl{\xi}} &= \cinl{\ctruify{\xi}} \\
  \ctruify{\cinr{\xi}} &= \cinr{\ctruify{\xi}} \\
  \ctruify{\cpair{\xi_1}{\xi_2}} &= \cpair{\ctruify{\xi_1}}{\ctruify{\xi_2}}
\end{align*}

\judgbox{\cfalsify{\xi_1} = \xi_2}{}
\begin{align*}
  \cfalsify{\ctruth} &= \ctruth \\
  \cfalsify{\cfalsity} &= \cfalsity \\
  \cfalsify{\cunknown} &= \cfalsity \\
  \cfalsify{\cnum{n}} &= \cnum{n} \\
  \cfalsify{\cnotnum{n}} &= \cnotnum{n} \\
  \cfalsify{\cand{\xi_1}{\xi_2}} &= \cand{\cfalsify{\xi_1}}{\cfalsify{\xi_2}} \\
  \cfalsify{\cor{\xi_1}{\xi_2}} &= \cor{\cfalsify{\xi_1}}{\cfalsify{\xi_2}} \\
  \cfalsify{\cinl{\xi}} &= \cinl{\cfalsify{\xi}} \\
  \cfalsify{\cinr{\xi}} &= \cinr{\cfalsify{\xi}} \\
  \cfalsify{\cpair{\xi_1}{\xi_2}} &= \cpair{\cfalsify{\xi_1}}{\cfalsify{\xi_2}}
\end{align*}
  \caption{Truify and Falsify Constraints}
  \label{fig:truify-falsify}
\end{figure}

% Now we introduce a different entailment that only operates on values and \textit{fully known} constraints. 
% Here, we use $\ctruify{\xi}=\xi$ to denote that unknown constraint $\cunknown$ does not appear in $\xi$, and thus we say $\xi$ are fully known. 
% To differentiate with \autoref{definition:const-entailment}, we drop the dot over the entailment symbol, indicating that we have got rid of the indeterminacy. 

With the above observation in mind, \Autoref{theorem:exhaustive-truify,theorem:redundant-truify-falsify} establish the "fully-known" equivalence of the entail judgments for exhaustiveness checking $\csatisfyormay{\ctruth}{\xi}$ and redundancy checking $\csatisfy{\xi_r}{\xi_{rs}}$.

\begin{theorem}
\label{theorem:exhaustive-truify}
  $\csatisfyormay{\ctruth}{\xi}$ iff $\ccsatisfy{}{\ctruify{\xi}}$.
\end{theorem}

\begin{theorem}
\label{theorem:redundant-truify-falsify}
  $\csatisfy{\xi_r}{\xi_{rs}}$ iff $\ccsatisfy{\;}{\cor{\cdual{\ctruify{\xi_r}}}{\cfalsify{\xi_{rs}}}}$.
\end{theorem}


%Let's check the exhaustiveness of a match expression with a single branch, whose pattern is $\hinlp{\hehole{w}}$. The emitted constraint is $\cinl{\cunknown}$. 
%In order to check if the match expression is indeterminately exhaustive, we try to make it as "exhaustive" as possible by truifying the emitted constraint and get $\cinl{\ctruth}$ as the programmer may fill hole $w$ with a wildcard.
%The entailment $\csatisfy{\ctruth}{\cinl{\ctruth}}$ encodes the indeterminate exhaustiveness. 
%Based on definition of $dual$ in \autoref{fig:constraint}, $\cdual{\cinl{\ctruth}}=\cor{\cinl{\cfalsity}}{\cinr{\ctruth}}$.
%$\cincon{\cor{\cinl{\cfalsity}}{\cinr{\ctruth}}}$ is not derivable because of the $inr$ branch. Therefore, the match expression must be inexhaustive.

% Consider the program in \autoref{fig:may-exhaustive}, it is unclear whether the match expression is truly exhaustive. But by replacing the pattern hole with a wild card pattern, the match expression may cover every case of the input list. And thus we know the match expression may be exhaustive and the program should type check. To Since the Unknown constraint $\cunknown$ is directly emitted from the pattern hole $w$, analogously, we replace it with Truth constraint $\ctruth$.

% Consider the example in \autoref{fig:must-redundant}, we want to tell if the third rule is redundant with respect to the first and second rules. To maximize the subset of values of list type that matches the pattern $\heholep{w}::xs$, we again replace the pattern hole $w$ with a variable pattern or a wild card pattern and realize that it is redundant. When there are hole(s) in the patterns of preceding rules, it is not obvious what pattern to replace the hole with. Consider the example in \figurename~\ref{fig:may-redundant1}, we want to tell if the third rule is redundant. To accomplish that, we want to minimize the subset of values of list type that matches $x::\heholep{w}$ so that it does not cover all the cases of the last branch. Intuitively, we may fill hole $\heholep{w}$ with $2::\nil$ so that the trailing part of the second pattern is more specific than $xs$ in the third pattern. However, it is not always the case that we can find a more specific (sub)pattern. Consider the example in \figurename~\ref{fig:may-redundant2}, the third pattern only matches one values and thus we cannot find a more specific pattern. Particularly, $2::\nil$ in the third pattern corresponds to the hole pattern $\heholep{w}$ in the second pattern. In this case, we can always find a different pattern to substitute the hole pattern. For example, we can replace $\heholep{w}$ with $3::[]$ and find that the third rule is not redundant.

%%% Local Variables:
%%% mode: latex
%%% TeX-master: "pattern-paper"
%%% End:
