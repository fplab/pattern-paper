\section{Labeled Sums}\label{sec:labeledsums}

In this section, we conservatively extend Peanut with finite labeled
sums, which is a practical necessity in general-purpose functional
languages. Besides adding labeled sums to the syntax, we will only cover
pattern matching for labeled sums as this is the most interesting bit
of the extension, and leave the other details to the Supplementary Material.

Labeled sums introduce a new sort $C$ for labels, a.k.a. datatype constructors, and a new type-level connective $C_1(\tau_1) + \cdots + C_n(\tau_n)$ for gathering and labeling types.
Since we are not usually concerned with the length of any particular sum,
we adopt a slightly more general and compact notation $\tlabeledsum{C_i(\tau_i)}_{C_i \in \tagset}$
for the sum consisting of labels $\tagset = \setof{C_i}_{i \leq n}$ with respective argument types $\setof{\tau_i}_{i \leq n}$.
The introduction form is the \emph{labeled injection expression} $\hinj{C}{\tau}{e}$ for injecting expression $e$ into sum $\tau$ at label $C$.
The elimination form is the \emph{labeled injection pattern} $\hinj{C}{\tau}{p}$ 
for expressions matching pattern $p$ that have been injected into sum $\tau$ at label $C$.
% For the remainder of this section, we assume all injections are labeled unless otherwise stated,
% and we elide sorts that are irrelevant or implied by the context.

To ensure maximal liveness, we distinguish \emph{concrete labels} $\tagC$
from \emph{label holes} which are either empty $\tagehole{u}$ or not empty $\taghole{\tagC}{u}$.
Empty holes arise where labels have yet to be constructed, for example during incremental construction of a sum or injection.
Non-empty holes operate as membranes around labels that are syntactically malformed or that violate a semantic constraint.
Non-empty holes around syntactically valid labels indicate duplication for sum type declarations and non-membership for injections.

% !TEX root= pattern-paper.tex

\begin{figure}[ht]
\begin{minipage}[t][][b]{.47\linewidth}
  \judgbox{\hpatmatch{e}{p}{\theta}}{$e$ matches $p$, emitting $\theta$}

  \begin{mathpar}
    \Infer{\MInj}{
      \hpatmatch{e}{p}{\theta}
    }{
      \hpatmatch{\hinj{C}{\tau}{e}}{\hinjp{C}{p}}{\theta}
    }
  \end{mathpar}

  \judgbox{\hmaymatch{e}{p}}{$e$ indeterminately matches $p$}

  \begin{mathpar}
    \Infer{\MMInjTag}{
      \tagmaymatch{C}{C'} \\
      \hnotnotmatch{e}{p}
    }{
      \hmaymatch{\hinj{C}{\tau}{e}}{\hinjp{C'}{p}}
    }

    \Infer{\MMInjArg}{
      \hmaymatch{e}{p}
    }{
      \hmaymatch{\hinj{C}{\tau}{e}}{\hinjp{C}{p}}
    }
  \end{mathpar}
\end{minipage}
\hfill
\begin{minipage}[t][][b]{.47\linewidth}
  \judgbox{\hnotmatch{e}{p}}{$e$ does not match $p$}

  \begin{mathpar}
    \Infer{\NMInj}{
      \tagC \neq \tagC'
    }{
      \hnotmatch{\hinj{\tagC}{\tau}{e}}{\hinjp{\tagC'}{p}}
    }

    \Infer{\NMInjTag}{
      \tagmaymatch{C}{C'} \\
      \hnotmatch{e}{p}
    }{
      \hnotmatch{\hinj{C}{\tau}{e}}{\hinjp{C'}{p}}
    }

    \Infer{\NMInjArg}{
      \hnotmatch{e}{p}
    }{
      \hnotmatch{\hinj{C}{\tau}{e}}{\hinjp{C}{p}}
    }
  \end{mathpar}
\end{minipage}

  \judgbox{\tagmaymatch{C}{C'}}{$C$ indeterminately matches $C'$}

  \begin{mathpar}
    \Infer{\TMMSym}{
      \tagmaymatch{C'}{C}
    }{
      \tagmaymatch{C}{C'}
    }

    \Infer{\TMMHole}{
      \tagehole{u} \neq C
    }{
      \tagmaymatch{\tagehole{u}}{C}
    }

    \Infer{\TMMEHole}{
      \tagC \neq C
    }{
      \tagmaymatch{\taghole{\tagC}{u}}{C}
    }
  \end{mathpar}

  \judgbox{\refutable{p}}{$p$ is refutable}

  \begin{mathpar}
    \Infer{\RInjSing}{
      C \in \tagset \\
      |\tagset| = 1 \\
      \refutable{p}
    }{
      \refutable{\hinjp{C}{p}}
    }

    \Infer{\RInjMult}{
      C \in \tagset \\
      |\tagset| > 1
    }{
      \refutable{\hinjp{C}{p}}
    }
  \end{mathpar}
  \caption{Pattern matching rules for labeled sums}
  \label{fig:labeled-sums-pattern-matching-rules}
\end{figure}


Extending \autoref{fig:patmatch}, the rules in
\autoref{fig:labeled-sums-pattern-matching-rules} define pattern
matching on injections with label holes, subject to the matching
determinism conditions imposed by \autoref{lemma:match-determinism}.
Rule \MInj is a straightforward rule for matching labeled injections.
Rules \MMInjTag and \MMInjArg allow indeterminate matching against a
pattern whose label or argument indeterminately match, respectively.
Rule \NMInj forbids matching of concrete but unequal labels, and Rules
\NMInjTag and \NMInjArg forbid matching when the arguments do not
match.  Rules \TMMSym, \TMMHole, and \TMMEHole define indeterminate
matching of labels by establishing a partial equivalence among
distinct labels when at least one of them is a hole.  Rule \MMNotIntro
works for refutable patterns, and rules \RInjMult and \RInjSing define
refutable patterns of labeled sum type.  Rule \RInjMult establishes
that patterns for nontrivial sums are refutable, i.e., that the set of
dual constraints is not empty.  The premises of Rule \RInjSing express that there are no alternatives for singleton
sums, and no choices at all for void sums.

% The inclusion of finite labeled sums with label holes adds substantial complexity to type consistency checking.
% For example, given two finite labeled sum types of cardinality $n$,
% whether they are consistent depends on whether there is a consistent pairing of the elements.
% Naively generating all $n^2$ pairings and then checking every subset of size $n$ 
% would require $\binom{n^2}{n} \gg n^2$ checks.
% If we instead model the search space as a complete bipartite graph with the elements of the sums on opposing sides,
% the problem reduces to one of finding a maximum matching in a bipartite graph with $2n$ vertices and $n^2$ edges,
% which can be solved by Hopcroft-Karp\todo{cite} in $O(n^{5/2})$ steps.

%%% Local Variables:
%%% mode: latex
%%% TeX-master: "pattern-paper.tex"
%%% End:
