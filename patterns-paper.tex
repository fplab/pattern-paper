\PassOptionsToPackage{svgnames,dvipsnames,svgnames}{xcolor}
\newif\ifarxiv
\arxivtrue
% \arxivfalse
\ifarxiv
\documentclass[acmsmall,screen,nonacm]{acmart}
\else
%%% Note: arxiv does not want line numbers (they are detected somehow, and are not allowed)
\documentclass[acmsmall,screen]{acmart}
% \settopmatter{printfolios=false,printccs=false,printacmref=false}
% \settopmatter{printccs=false,printacmref=false}
\fi

\ifarxiv
\newcommand{\appendixName}{appendix}
\else
\newcommand{\appendixName}{extended appendix}
\fi

%% For single-blind review submission
%\documentclass[acmlarge,review]{acmart}\settopmatter{printfolios=true}
%% For final camera-ready submission
%\documentclass[acmlarge]{acmart}\settopmatter{}

%% Note: Authors migrating a paper from PACMPL format to traditional
%% SIGPLAN proceedings format should change 'acmlarge' to
%% 'sigplan,10pt'.

% \bibliographystyle{ACM-Reference-Format}


%% Some recommended packages.
\usepackage{booktabs}   %% For formal tables:
                        %% http://ctan.org/pkg/booktabs
\usepackage{subcaption} %% For complex figures with subfigures/subcaptions
                        %% http://ctan.org/pkg/subcaption

%% Cyrus packages
\usepackage{microtype}
\usepackage{mdframed}
\usepackage{colortab}
\usepackage{mathpartir}
\usepackage{enumitem}
\usepackage{bbm}
\usepackage{stmaryrd}
\usepackage{mathtools}
\usepackage{leftidx}
\usepackage{todonotes}
\usepackage{xspace}
\usepackage{wrapfig}
\usepackage{extarrows}
% \usepackage[subtle]{savetrees}

\usepackage{listings}%
\lstloadlanguages{ML}
\lstset{tabsize=2, 
basicstyle=\footnotesize\ttfamily, 
% keywordstyle=\sffamily,
commentstyle=\itshape\ttfamily\color{gray}, 
stringstyle=\ttfamily\color{purple},
mathescape=false,escapechar=\#,
numbers=left, numberstyle=\scriptsize\color{gray}\ttfamily, language=ML, showspaces=false,showstringspaces=false,xleftmargin=15pt, 
morekeywords={string, float, int, bool},
classoffset=0,belowskip=\smallskipamount, aboveskip=\smallskipamount,
moredelim=**[is][\color{red}]{SSTR}{ESTR}
}
\newcommand{\li}[1]{\lstinline[basicstyle=\ttfamily\fontsize{9pt}{1em}\selectfont]{#1}}
\newcommand{\lismall}[1]{\lstinline[basicstyle=\ttfamily\fontsize{9pt}{1em}\selectfont]{#1}}

%% Joshua Dunfield macros
\def\OPTIONConf{1}%
\usepackage{joshuadunfield}

%% Can remove this eventually
\usepackage{blindtext}

\usepackage{enumitem}

%%%%%%%%%%%%%%%%%%%%%%%%%%%%%%%%%%%%%%%%%%%%%%%%%%%%%%%%%%%%%%%%%%%%%%%%%%%%%
%% Matt says: Cyrus, this package `adjustbox` seems directly related
%% to the `clipbox` error; To get rid of the error, I moved it last
%% (after other usepackages) and I added the line just above it, which
%% permits it to redefine `clipbox` (apparently also defined in
%% `pstricks`, and due to latex's complete lack of namespace
%% management, these would otherwise names clash).
\let\clipbox\relax
\usepackage[export]{adjustbox}% http://ctan.org/pkg/adjustbox
%%%%%%%%%%%%%%%%%%%%%%%%%%%%%%%%%%%%%%%%%%%%%%%%%%%%%%%%%%%%%%%%%%%%%%%%%%%%%%%%%


%%%%%%%%%%%%%%%%%%%%%%%%%%%%%%%%%%%%%%%%%%%%%%%%%%%%%%%%%%%%%%%%%%%%%%%%%%%%%%%%%
%\usepackage{draftwatermark}
%\SetWatermarkText{DRAFT}
%\SetWatermarkScale{1}
%%%%%%%%%%%%%%%%%%%%%%%%%%%%%%%%%%%%%%%%%%%%%%%%%%%%%%%%%%%%%%%%%%%%%%%%%%%%%%%%%


% A macro for the name of the system being described by ``this paper''
\newcommand{\HazelnutLive}{\textsf{Hazelnut Live}\xspace}
\newcommand{\Hazelnut}{\textsf{Hazelnut}\xspace}
% The mockup, work-in-progress system.
\newcommand{\Hazel}{\textsf{Hazel}\xspace}

% \newtheorem{theorem}{Theorem}[chapter]
% \newtheorem{lemma}[theorem]{Lemma}
% \newtheorem{corollary}[theorem]{Corollary}
% \newtheorem{definition}[theorem]{Definition}
% \newtheorem{assumption}[theorem]{Assumption}
% \newtheorem{condition}[theorem]{Condition}

\newtheoremstyle{slplain}% name
  {.15\baselineskip\@plus.1\baselineskip\@minus.1\baselineskip}% Space above
  {.15\baselineskip\@plus.1\baselineskip\@minus.1\baselineskip}% Space below
  {\slshape}% Body font
  {\parindent}%Indent amount (empty = no indent, \parindent = para indent)
  {\bfseries}%  Thm head font
  {.}%       Punctuation after thm head
  { }%      Space after thm head: " " = normal interword space;
        %       \newline = linebreak
  {}%       Thm head spec
\theoremstyle{slplain}
\newtheorem{thm}{Theorem}  % Numbered with the equation counter
\numberwithin{thm}{section}
\newtheorem{defn}[thm]{Definition}
\newtheorem{lem}[thm]{Lemma}
\newtheorem{prop}[thm]{Proposition}
\newtheorem{corol}[thm]{Corollary}
% \newtheorem{cor}[section]{Corollary}     
% \newtheorem{lem}[section]{Lemma}         
% \newtheorem{prop}[section]{Proposition}  

% \setlength{\abovedisplayskip}{0pt}
% \setlength{\belowdisplayskip}{0pt}
% \setlength{\abovedisplayshortskip}{0pt}
% \setlength{\belowdisplayshortskip}{0pt}


\ifarxiv
\setcopyright{rightsretained} 
\acmJournal{PACMPL}
\acmYear{2019} \acmVolume{3} \acmNumber{POPL} \ifarxiv \acmArticle{1} \else \acmArticle{14} \fi \acmMonth{1} \acmPrice{}\acmDOI{10.1145/3290327}
\copyrightyear{2019}
\else
%%% The following is specific to POPL '19 and the paper
%%% 'Live Functional Programming with Typed Holes'
%%% by Cyrus Omar, Ian Voysey, Ravi Chugh, and Matthew A. Hammer.
%%%
\setcopyright{rightsretained}
\acmPrice{}
\acmDOI{10.1145/3290327}
\acmYear{2019}
\copyrightyear{2019}
\acmJournal{PACMPL}
\acmVolume{3}
\acmNumber{POPL}
\acmArticle{14}
\acmMonth{1}
\fi


% \fancyfoot{} % suppresses the footer (also need \thispagestyle{empty} after \maketitle below)


%% Bibliography style
\bibliographystyle{ACM-Reference-Format}
%% Citation style
%% Note: author/year citations are required for papers published as an
%% issue of PACMPL.
\citestyle{acmauthoryear}   %% For author/year citations

% !TEX root = ./patterns-paper.tex

\newcommand{\mynote}[3]{\textcolor{#3}{\textsf{{#2}}}}
\newcommand{\rkc}[1]{\mynote{rkc}{#1}{blue}}
\newcommand{\cy}[1]{\mynote{cy}{#1}{purple}}
\newcommand{\mah}[1]{\mynote{cy}{#1}{green}}
\newcommand{\matt}[1]{{\color{blue}{\textit{Matt:~#1}}}}

\newcommand{\cvert}{{\,{\vert}\,}}

%% https://tex.stackexchange.com/questions/9796/how-to-add-todo-notes
\newcommand{\rkcTodo}[1]{\todo[linecolor=blue,backgroundcolor=blue!25,bordercolor=blue]{#1}}

\newcommand{\mattTodo}[1]{\todo[linecolor=green,backgroundcolor=green!2,bordercolor=green]{\tiny\textit{#1}}}
\newcommand{\mattOmit}[1]{\colorbox{yellow}{(Matt omitted stuff here)}}

\def\parahead#1{\paragraph{\textbf{#1.}}}
%% \def\paraheadNoDot#1{\paragraph{{\textbf{#1}}}}
\def\subparahead#1{\paragraph{\textit{#1.}}}
%% \def\paraheadindent#1{\paragraph{}\textit{#1.}}
%% \def\paraheadindentnodot#1{\paragraph{}\textit{#1}}

% \newcommand{\ie}{{\emph{i.e.}}}
% \newcommand{\eg}{{\emph{e.g.}}}
% \newcommand{\etc}{{\emph{etc.}}}
% \newcommand{\cf}{{\emph{cf.}}}
% \newcommand{\etal}{{\emph{et al.}}}

%% \newcommand{\hazel}{\ensuremath{\textsc{Hazel}}}
%% \newcommand{\sns}{\ensuremath{\textsc{Sketch-n-Sketch}}}
%% \newcommand{\deuce}{\ensuremath{\textsc{Deuce}}}
\newcommand{\Elm}{\ensuremath{\textsf{Elm}}}
\newcommand{\sns}{\ensuremath{\textrm{Sketch-n-Sketch}}}
\newcommand{\deuce}{\ensuremath{\textrm{Deuce}}}

\newcommand{\sectionDescription}[1]{\section{#1}}
\newcommand{\subsectionDescription}[1]{\subsection{#1}}
\newcommand{\subsubsectionDescription}[1]{\subsubsection{#1}}
%% \newcommand{\subsectionDescription}[1]{\subsection*{#1}}
\newcommand{\suppMaterials}{the Supplementary Materials}

\newcommand{\defeq}{\overset{\textrm{def}}{=}}

\newcommand{\eap}{action suggestion panel\xspace}
\newcommand{\Eap}{Action suggestion panel\xspace}

\newcommand{\myfootnote}[1]{\footnote{ #1}}

\def\sectionautorefname{Section}
\def\subsectionautorefname{Section}
\def\subsubsectionautorefname{Section}

\newcommand{\code}[1]{\lstinline{#1}}

% Make italic?
%\newcommand{\Property}[1]{\emph{#1}}
\newcommand{\Property}[1]{\textrm{#1}}

% Calling out Cyrus's favorite verb, 'to be' ;)
\newcommand{\IS}{\colorbox{red}{is}\xspace}

\newcommand{\codeSize}
  %% {\footnotesize}
  {\small}

%\newcommand{\JoinTypes}[2]{\textsf{join}~~#1~~#2}
\newcommand{\JoinTypes}[2]{\textsf{join}(#1,#2)}

%%%%%%%%%%%%%%%%%%%%%%%%%%%%%%%%%%%%%%%%%%%%%%%%%%%%%%%%%%%%%%%%%%%%%%%%%%%%%%%%
%% Spacing

\newcommand{\sep}{\hspace{0.06in}}
\newcommand{\sepPremise}{\hspace{0.20in}}
\newcommand{\hsepRule}{\hspace{0.20in}}
\newcommand{\vsepRuleHeight}{0.08in}
\newcommand{\vsepRule}{\vspace{\vsepRuleHeight}}
\newcommand{\miniSepOne}{\hspace{0.01in}}
\newcommand{\miniSepTwo}{\hspace{0.02in}}
\newcommand{\miniSepThree}{\hspace{0.03in}}
\newcommand{\miniSepFour}{\hspace{0.04in}}
\newcommand{\miniSepFive}{\hspace{0.05in}}

%%%%%%%%%%%%%%%%%%%%%%%%%%%%%%%%%%%%%%%%%%%%%%%%%%%%%%%%%%%%%%%%%%%%%%%%%%%%%%%%

% \lstset{
% %mathescape=true,basicstyle=\fontsize{8}{9}\ttfamily,
% literate={=>}{$\Rightarrow$}2
%          {<=}{$\leq$}2
%          {->}{${\rightarrow}$}1
%          {\\\\=}{\color{red}{$\lambda$}}2
%          {\\\\}{$\lambda$}2
%          {**}{$\times$}2
%          {*.}{${\color{blue}{\texttt{*.}}}$}2
%          {+.}{${\color{blue}{\texttt{+.}}}$}2
%          {<}{${\color{green}{\lhd}}$}1
%          {>?}{${\color{green}{\rhd}}$?}2
%          {<<}{${\color{green}{\blacktriangleleft}}$}1
%          {>>?}{${\color{green}{\blacktriangleright}}$?}2
%          {\{}{${\color{blue}{\{}}$}1
%          {\}}{${\color{blue}{\}}}$}1
%          {[}{${\color{purple}{[}}$}1
%          {]}{${\color{purple}{]}}$}1
%          {(}{${\color{darkgray}{\texttt{(}}}$}1
%          {)}{${\color{darkgray}{\texttt{)}}}$}1
%          {]]}{${\color{gray}{\big(}}$}1
%          {]]}{${\color{gray}{\big)}}$}1
% }

% !TEX root = ./patterns-paper.tex

% reverse Vdash
\newcommand{\dashV}{\mathbin{\rotatebox[origin=c]{180}{$\Vdash$}}}

% Violet hotdogs; highlight color helps distinguish them
\newcommand{\llparenthesiscolor}{\textcolor{violet}{\llparenthesis}}
\newcommand{\rrparenthesiscolor}{\textcolor{violet}{\rrparenthesis}}

% HTyp and HExp
\newcommand{\hcomplete}[1]{#1~\mathsf{complete}}

% HTyp
\newcommand{\htau}{\dot{\tau}}
\newcommand{\tarr}[2]{\inparens{#1 \rightarrow #2}}
\newcommand{\tarrnp}[2]{#1 \rightarrow #2}
\newcommand{\trul}[2]{\inparens{#1 \Rightarrow #2}}
\newcommand{\tnum}{\mathtt{num}}
\newcommand{\tehole}{\llparenthesiscolor\rrparenthesiscolor}
\newcommand{\tsum}[2]{\inparens{{#1} + {#2}}}
\newcommand{\tprod}[2]{\inparens{{#1} \times {#2}}}
\newcommand{\tunit}{\mathtt{1}}
\newcommand{\tvoid}{\mathtt{0}}

\newcommand{\tcompat}[2]{#1 \sim #2}
\newcommand{\tincompat}[2]{#1 \nsim #2}

% HExp
\newcommand{\hexp}{\dot{e}}
\newcommand{\hlam}[3]{\inparens{\lambda #1:#2.#3}}
\newcommand{\hap}[2]{#1(#2)}
\newcommand{\hapP}[2]{(#1)~(#2)} % Extra paren around function term
\newcommand{\hnum}[1]{\underline{#1}}
\newcommand{\hadd}[2]{\inparens{#1 + #2}}
\newcommand{\hpair}[2]{\inparens{#1 , #2}}
\newcommand{\htriv}{()}
\newcommand{\hehole}{\llparenthesiscolor\rrparenthesiscolor}
\newcommand{\hhole}[1]{\llparenthesiscolor#1\rrparenthesiscolor}
\newcommand{\hindet}[1]{\lceil#1\rceil}
\newcommand{\hinj}[2]{\mathtt{inj}_{#1}({#2})}
\newcommand{\hinl}[2]{\mathtt{inl}_{#1}({#2})}
\newcommand{\hinr}[2]{\mathtt{inr}_{#1}({#2})}
\newcommand{\hinlp}[1]{\mathtt{inl}(#1)}
\newcommand{\hinrp}[1]{\mathtt{inr}(#1)}
\newcommand{\hmatch}[2]{\mathtt{match}(#1) \{#2\}}
\newcommand{\hcase}[5]{\mathtt{case}({#1},{#2}.{#3},{#4}.{#5})}
\newcommand{\hrules}[2]{\inparens{#1 \mid #2}}
\newcommand{\hrul}[2]{#1 \Rightarrow #2}

\newcommand{\hGamma}{\dot{\Gamma}}
\newcommand{\domof}[1]{\text{dom}(#1)}
\newcommand{\hsyn}[3]{#1 \vdash #2 \Rightarrow #3}
\newcommand{\hana}[3]{#1 \vdash #2 \Leftarrow #3}
\newcommand{\hexptyp}[3]{#1 \vdash #2 : #3}
\newcommand{\hpattyp}[3]{#1 : #2 \dashV #3}
\newcommand{\hpatmatch}[3]{#1 \vartriangleright #2 \dashV #3}
\newcommand{\hval}[1]{#1 ~\mathtt{val}}
\newcommand{\herr}[1]{#1 ~\mathtt{err}}

% ZTyp and ZExp
\newcommand{\zlsel}[1]{{\bowtie}{#1}}
\newcommand{\zrsel}[1]{{#1}{\bowtie}}
\newcommand{\zwsel}[1]{
  \setlength{\fboxsep}{0pt}
  \colorbox{green!10!white!100}{
    \ensuremath{{{\textcolor{Green}{{\hspace{-2px}\triangleright}}}}{#1}{\textcolor{Green}{\triangleleft{\vphantom{\tehole}}}}}}
}

\newcommand{\removeSel}[1]{#1^{\diamond}}

% ZTyp
\newcommand{\ztau}{\hat{\tau}}

% ZExp
\newcommand{\zexp}{\hat{e}}

% Direction
\newcommand{\dParent}{\mathtt{parent}}
\newcommand{\dChildn}[1]{\mathtt{child}~\mathtt{{#1}}}
\newcommand{\dChildnm}[1]{\mathtt{child}~{#1}}

% Action
\newcommand{\aMove}[1]{\mathtt{move}~#1}
	\newcommand{\zrightmost}[1]{\mathsf{rightmost}(#1)}
	\newcommand{\zleftmost}[1]{\mathsf{leftmost}(#1)}
\newcommand{\aSelect}[1]{\mathtt{sel}~#1}
\newcommand{\aDel}{\mathtt{del}}
\newcommand{\aReplace}[1]{\mathtt{replace}~#1}
\newcommand{\aConstruct}[1]{\mathtt{construct}~#1}
\newcommand{\aConstructx}[1]{#1}
\newcommand{\aFinish}{\mathtt{finish}}

\newcommand{\performAna}[5]{#1 \vdash #2 \xlongrightarrow{#4} #5 \Leftarrow #3}
\newcommand{\performAnaI}[5]{#1 \vdash #2 \xlongrightarrow{#4}\hspace{-3px}{}^{*}~ #5 \Leftarrow #3}
\newcommand{\performSyn}[6]{#1 \vdash #2 \Rightarrow #3 \xlongrightarrow{#4} #5 \Rightarrow #6}
\newcommand{\performSynI}[6]{#1 \vdash #2 \Rightarrow #3 \xlongrightarrow{#4}\hspace{-3px}{}^{*}~ #5 \Rightarrow #6}
\newcommand{\performTyp}[3]{#1 \xlongrightarrow{#2} #3}
\newcommand{\performTypI}[3]{#1 \xlongrightarrow{#2}\hspace{-3px}{}^{*}~#3}

\newcommand{\performMove}[3]{#1 \xlongrightarrow{#2} #3}
\newcommand{\performDel}[2]{#1 \xlongrightarrow{\aDel} #2}

% Form
\newcommand{\farr}{\mathtt{arrow}}
\newcommand{\fnum}{\mathtt{num}}
\newcommand{\fsum}{\mathtt{sum}}

\newcommand{\fasc}{\mathtt{asc}}
\newcommand{\fvar}[1]{\mathtt{var}~#1}
\newcommand{\flam}[1]{\mathtt{lam}~#1}
\newcommand{\fap}{\mathtt{ap}}
% \newcommand{\farg}{\mathtt{arg}}
\newcommand{\fnumlit}[1]{\mathtt{lit}~#1}
\newcommand{\fplus}{\mathtt{plus}}
\newcommand{\fhole}{\mathtt{hole}}
\newcommand{\fnehole}{\mathtt{nehole}}

\newcommand{\finj}[1]{\mathtt{inj}~#1}
\newcommand{\fcase}[2]{\mathtt{case}~#1~#2}

% Talk about formal rules in example
\newcommand{\refrule}[1]{\textrm{Rule~(#1)}}

\newcommand{\herase}[1]{\left|#1\right|_\textsf{erase}}

\newcommand{\arrmatch}[2]{#1 \blacktriangleright_{\rightarrow} #2}


\newcommand{\TABperformAna}[5]{#1 \vdash & #2                & \xlongrightarrow{#4} & #5 & \Leftarrow #3}
\newcommand{\TABperformSyn}[6]{#1 \vdash & #2 \Rightarrow #3 & \xlongrightarrow{#4} & #5 \Rightarrow #6}
\newcommand{\TABperformTyp}[3]{& #1 & \xlongrightarrow{#2} & #3}

\newcommand{\TABperformMove}[3]{#1 & \xlongrightarrow{#2} & #3}
\newcommand{\TABperformDel}[2]{#1 \xlongrightarrow{\aDel} #2}

\newcommand{\sumhasmatched}[2]{#1 \mathrel{\textcolor{black}{\blacktriangleright_{+}}} #2}

\newcommand{\subminsyn}[1]{\mathsf{submin}_{\Rightarrow}(#1)}
\newcommand{\subminana}[1]{\mathsf{submin}_{\Leftarrow}(#1)}


\newcommand{\inparens}[1]{{\color{gray}(}#1{\color{gray})}}

%% rule names for appendix
\newcommand{\rname}[1]{\textsc{#1}}
\newcommand{\gap}{\vspace{7pt}}


\setlength{\abovecaptionskip}{4pt plus 3pt minus 2pt} % Chosen fairly arbitrarily
\setlength{\belowcaptionskip}{-4pt plus 3pt minus 2pt} % Chosen fairly arbitrarily


\begin{document}

%% Title information
\title{patterns paper}         %% [Short Title] is optional;
\ifarxiv
\subtitle{Extended Version}
\subtitlenote{The original version of this article was published in the POPL 2019 edition of PACMPL \cite{HazelnutLive}. This extended version includes an additional appendix.}
\fi
                                        %% when present, will be used in

                                        %% header instead of Full Title.
% \titlenote{with title note}             %% \titlenote is optional;
                                        %% can be repeated if necessary;
                                        %% contents suppressed with 'anonymous'
% \subtitle{Subtitle}                     %% \subtitle is optional
% \subtitlenote{with subtitle note}       %% \subtitlenote is optional;
                                        %% can be repeated if necessary;
                                        %% contents suppressed with 'anonymous'


%% Author information
%% Contents and number of authors suppressed with 'anonymous'.
%% Each author should be introduced by \author, followed by
%% \authornote (optional), \orcid (optional), \affiliation, and
%% \email.
%% An author may have multiple affiliations and/or emails; repeat the
%% appropriate command.
%% Many elements are not rendered, but should be provided for metadata
%% extraction tools.

%% Author with single affiliation.
% \author{Cyrus Omar}
% \authornote{with author1 note}          %% \authornote is optional;
                                        %% can be repeated if necessary
% \orcid{nnnn-nnnn-nnnn-nnnn}             %% \orcid is optional
% \affiliation{
  % \position{Position1}
  % \department{Department1}              %% \department is recommended
  % \institution{University of Chicago, USA}            %% \institution is required
  % \streetaddress{Street1 Address1}
  % \city{City1}
  % \state{State1}
  % \postcode{Post-Code1}
  % \country{Country1}
% }
% \email{comar@cs.uchicago.edu}          %% \email is recommended


% %% Author with two affiliations and emails.
% \author{First2 Last2}
% \authornote{with author2 note}          %% \authornote is optional;
%                                         %% can be repeated if necessary
% \orcid{nnnn-nnnn-nnnn-nnnn}             %% \orcid is optional
% \affiliation{
%   \position{Position2a}
%   \department{Department2a}             %% \department is recommended
%   \institution{Institution2a}           %% \institution is required
%   \streetaddress{Street2a Address2a}
%   \city{City2a}
%   \state{State2a}
%   \postcode{Post-Code2a}
%   \country{Country2a}
% }
% \email{first2.last2@inst2a.com}         %% \email is recommended
% \affiliation{
%   \position{Position2b}
%   \department{Department2b}             %% \department is recommended
%   \institution{Institution2b}           %% \institution is required
%   \streetaddress{Street3b Address2b}
%   \city{City2b}
%   \state{State2b}
%   \postcode{Post-Code2b}
%   \country{Country2b}
% }
% \email{first2.last2@inst2b.org}         %% \email is recommended


%% Paper note
%% The \thanks command may be used to create a "paper note" ---
%% similar to a title note or an author note, but not explicitly
%% associated with a particular element.  It will appear immediately
%% above the permission/copyright statement.
% \thanks{with paper note}                %% \thanks is optional
                                        %% can be repeated if necesary
                                        %% contents suppressed with 'anonymous'


%% Abstract
%% Note: \begin{abstract}...\end{abstract} environment must come
%% before \maketitle command
% \begin{abstract}
    Several modern programming systems, including GHC Haskell, Agda, Idris, and Hazel,  support \emph{typed holes}. 
    Assigning static and, to varying degree, dynamic meaning to programs with holes allows program editors and other tools to offer meaningful  feedback and assistance throughout editing, i.e. in a \emph{live} manner.
    Prior work, however, has considered only holes appearing in expressions and types. 
    This paper considers, from type theoretic and logical first principles,
    the problem of typed pattern holes.
    We confront two main difficulties, (1) statically reasoning about exhaustiveness and irredundancy 
    when patterns are not fully known, and (2) live evaluation of expressions
    containing both pattern and expression holes. 
    In both cases, this requires reasoning conservatively about all 
    possible hole fillings.
    We develop a typed lambda calculus, Peanut, 
    where reasoning about exhaustiveness and redundancy is mapped to the
    problem of deriving first order entailments. 
    We equip Peanut with an operational semantics in the style of Hazelnut Live that allows us 
    to evaluate around holes in both expressions and patterns.
    We mechanize the metatheory of Peanut in Agda and formalize a procedure capable of 
    deciding the necessary entailments.
    % We scale up the core calculus to support finite labeled sums, adding support for \emph{datatype constructor holes} in the process.\todo{do we still want to highlight this in abstract?}
    Finally, we scale up and implement these mechanisms within Hazel, a  
    programming environment for a dialect of Elm that automatically inserts holes during editing to provide static and dynamic feedback to the programmer in a maximally live manner, i.e. for every possible editor state. Hazel is the first maximally live environment for a general-purpose functional language.
    % Hazel is the first maximally live environment for a general-purpose functional language. 
        % In these languages, empty holes serve as placeholders for missing program terms, while
    % non-empty holes operate as semantic membranes around erroneous terms, 
    % isolating them so that the remainder of the program remains statically meaningful,
    % and, in certain cases, even dynamically meaningful.

  \keywords{pattern matching \and typed holes \and live programming}
\end{abstract}


%% 2012 ACM Computing Classification System (CSS) concepts
%% Generate at 'http://dl.acm.org/ccs/ccs.cfm'.
% \begin{CCSXML}
% <ccs2012>
% <concept>
% <concept_id>10011007.10011006.10011008</concept_id>
% <concept_desc>Software and its engineering~General programming languages</concept_desc>
% <concept_significance>500</concept_significance>
% </concept>
% <concept>
% <concept_id>10003456.10003457.10003521.10003525</concept_id>
% <concept_desc>Social and professional topics~History of programming languages</concept_desc>
% <concept_significance>300</concept_significance>
% </concept>
% </ccs2012>
% \end{CCSXML}
\begin{CCSXML}
<ccs2012>
<concept>
<concept_id>10011007.10011006.10011008.10011009.10011012</concept_id>
<concept_desc>Software and its engineering~Functional languages</concept_desc>
<concept_significance>500</concept_significance>
</concept>
% <concept>
% <concept_id>10003752.10010124.10010131.10010134</concept_id>
% <concept_desc>Theory of computation~Operational semantics</concept_desc>
% <concept_significance>500</concept_significance>
% </concept>
</ccs2012>
\end{CCSXML}

\ccsdesc[500]{Software and its engineering~Functional languages}
% \ccsdesc[500]{Theory of computation~Operational semantics}
%% End of generated code


%% Keywords
%% comma separated list
% \keywords{live programming, gradual typing, contextual modal type theory, typed holes, structured editing}  %% \keywords is optional


%% \maketitle
%% Note: \maketitle command must come after title commands, author
%% commands, abstract environment, Computing Classification System
%% environment and commands, and keywords command.
% \maketitle
% \thispagestyle{empty} % suppresses the footer

% \section{Introduction}
\label{sec:intro}

Programming language definitions typically assign meaning to programs only once they are fully-formed and fully-typed. 
However, programming tools---type checkers, language-aware editors, interpreters, program synthesizers, and so on---%
are frequently asked to reason about and manipulate programs that are incomplete or erroneous.
This can occur when the programmer has made a mistake, or when the programmer is simply in the midst of an editing task.
These meaningless states are sometimes transient but they can also persist, e.g. through long refactoring tasks, causing programming tools to flicker out of service or to turn to  
\emph{ad hoc} heuristics, e.g. arbitrary token insertion, or deletion of problematic lines of code, to offer best-effort feedback and assistance \cite{HazelnutSNAPL, DBLP:conf/oopsla/KatsJNV09,DBLP:journals/pacmpl/BourRS18}.
In brief, definitional gaps lead to gaps in service.

In recognition of this pernicious \emph{gap problem}, several programming systems, 
including GHC Haskell \cite{GHCHoles}, Agda \cite{norell:thesis}, Idris \cite{brady2013idris}, and Hazel \cite{DBLP:conf/popl/OmarVHAH17,DBLP:journals/pacmpl/OmarVCH19}, have introduced \emph{typed holes}. Typed holes come in two basic forms: \emph{empty holes} 
stand for terms that have yet to be constructed, and 
\emph{non-empty holes} 
operate as membranes around erroneous terms, e.g. as-yet-type-inconsistent
expressions or as-yet-unbound variables, 
isolating them from the rest of the program.
By incorporating holes into the syntax and semantics, 
it is possible to assign meaning to a greater number of notionally incomplete programs.
Language services can thereby avoid gaps without needing to rely on \emph{ad hoc} heuristics.
Services can also be developed specifically for working with holes. For example, all of the systems mentioned above report  
the expected type and the variables in scope at each hole, and they are also able to synthesize hole fillings in various ways \cite{DBLP:conf/haskell/Gissurarson18,DBLP:journals/pacmpl/LubinCOC20}.

In most of these systems the programmer manually inserts holes where necessary.
Luckily, holes are syntactically lightweight: in GHC Haskell, for example, an unnamed empty expression
hole is simply \li{_}, a named hole is \li{_name}, and non-empty holes can be inserted implicitly around static errors with an appropriate compiler flag. 
In Agda, programmers can express  
non-empty holes explicitly as \li{\{e\}n} where \li{e} is an expression and \li{n} is an identifying hole number.

The Hazel structure editor is distinct in that it inserts both empty and non-empty holes fully automatically during editing.
For example, Fig.~\ref{fig:exhaustiveness}(a), discussed further below, shows an automatically inserted empty hole to the right of the \li{::} operator, numbered \li{98}. Hazel goes on to eliminate the gap problem entirely, maintaining a \emph{maximal liveness invariant}: Hazel assigns both static and dynamic meaning to \emph{every} possible editor state \cite{DBLP:conf/popl/OmarVHAH17, DBLP:journals/pacmpl/OmarVCH19}. This allows Hazel language services to remain fully functional at all
times. This includes services that require program evaluation, because Hazel is capable of evaluating ``around'' 
expression holes, producing \emph{indeterminate results} that retain holes \cite{DBLP:journals/pacmpl/OmarVCH19}. Hazel also supports holes that appear in type annotations, including those where type inference is unable to find a solution, because Hazel is gradually typed \cite{DBLP:conf/snapl/SiekVCB15}. Dynamic type errors (and other dynamic errors) are reformulated as run-time holes to localize their effect on evaluation \cite{DBLP:journals/pacmpl/OmarVCH19}.\footnote{GHC Haskell can also run programs with expression holes, but the program crashes when a hole is reached. It also supports holes in type annotations, but the type inference system must be able to uniquely solve for these holes.}

In all of the systems just described, 
holes can appear in expressions and types.
None of these systems have previously supported holes in patterns. 
Pattern holes would, however, be useful for much the same reason as expression holes are useful: patterns are compositional in structure and are governed by a type discipline.
In Hazel, our focus in this paper, pattern holes are in fact critical to scale up beyond the language in the prior work, which included only binary products and sums with primitive eliminators. Programmers will necessarily construct patterns incrementally\footnote{In Agda, Hazel, and various other systems, the user can automatically generate an exhaustive set of patterns, but these match only on the outermost constructor and involve automatically generated variable names. Programmers often rearrange these incrementally into more deeply nested patterns.} and Hazel must be able to assign meaning to each step to maintain its maximal liveness invariant.

While expressions and types are central to functional programming, patterns are also ubiquitous and pattern holes are far from trivial. Pattern matching can involve both a large number of patterns and individually large and complex patterns. For example, central to the Hazel editor implementation is a single match expression with 68 rules due to the combinatoric structure of the scrutinee, which is a pair of values (an action and a state). Several of these patterns have a compositional depth of 5 and span multiple lines of code. Entering individually well-typed patterns and collectively irredundant and exhaustive sequences of patterns in non-trivial programs like this is not always straightforward, and it is easy to make mistakes, e.g. during a non-trivial refactoring. It is also useful to be able to test branches as they become complete, even when many remaining branches remain incomplete. In short, live feedback serves to surface problems as soon as they are certain to occur, rather than in an infrequently batched manner, which helps programmers diagnose problems early and maintain confidence in their mental model of program behavior \cite{tanimoto2013perspective}.

This paper describes our integration of full-scale pattern matching (reviewed in Sec.~\ref{sec:background}) with support for pattern holes and live evaluation into Hazel (Sec.~\ref{sec:examples}). We then distill out the essential ideas with a type-theoretic calculus called Peanut, which extends the Hazelnut Live calculus of \citet{DBLP:journals/pacmpl/OmarVCH19} with pattern matching and pattern holes (Sec.~\ref{sec:formalism}). We have mechanized the metatheory of Peanut using Agda (Sec.~\ref{sec:agda} and Supplemental Material). We also describe a simple decision procedure for the necessary static analyses, expressed declaratively as logical entailments (Sec.~\ref{sec:algorithm}). To go from Peanut to Hazel, we generalize it to support finite labeled sums, including sums with \emph{datatype constructor holes} (Sec.~\ref{sec:labeledsums}). The result, which we integrate into Hazel, leaves us with the first general-purpose programming environment that maintains maximal liveness.

\section{Background}
\label{sec:background}
Before discussing pattern holes, let us briefly review the necessary background and terminology, which will be familiar to users of functional languages. 
Briefly,
\emph{structural pattern matching} combines structural case analysis with destructuring binding. 
Patterns are compositional, so pattern matching can dramatically collapse what would otherwise 
need to be a deeply nested sequence of case analyses and destructurings. The central construct is the \li{match} expression, examples of which appear in  Fig.~\ref{fig:basic-examples}. A match expression consists of a \emph{scrutinee} and an ordered sequence of \li{|}-separated \emph{rules}. Each rule consists of a \emph{pattern} and a \emph{branch expression} separated by \li{->}. The value of the scrutinee is matched against each pattern in order, and if there is a match, the corresponding branch is taken. The examples in  Fig.~\ref{fig:basic-examples} 
case analyze on the outer constructor of the value of the {scrutinee}, \li{tree}. In cases where the scrutinee was constructed by the application of the \li{Node} constructor, they simultaneously match on the structure of the list argument. Pattern variables match any value and become bound to that value in the corresponding branch expression. Wildcard patterns, \li{_}, also match any value, but induce no binding.

\begin{figure}
\begin{subfigure}{.45\textwidth}
\begin{lstlisting}[numbers=none]
match tree
| Node([]) -> Empty
| Node([x]) -> Node([f x, Empty])
| Node([x, y]) -> Node([f x, f y])
| Node(x::y::tl) -> Node(
  [f x, f (Node (y::tl))])
| Leaf x -> Leaf x
| Empty -> Empty
end\end{lstlisting}
\vspace{-6px}
\caption{Exhaustive + Irredundant\label{fig:basic-examples-correct}}
\end{subfigure}
\begin{subfigure}{.5\textwidth}
\begin{lstlisting}[numbers=none]
match tree
| Node(x::y::tl) -> Node(
  [f x, f (Node (y::tl))])
| Node([x, y]) -> Node([f x, f y])
| Node([x]) -> Node([f x, Empty])
| Node([]) -> Empty
| Empty -> Empty
end
##\end{lstlisting}
\vspace{-6px}
\caption{Inexhaustive + Redundant (Second Pattern)\label{fig:basic-examples-wrong}}
\end{subfigure}
\vspace{-3px}
\caption{Two examples demonstrating structural pattern matching and common pitfalls.}
\vspace{-4px}
\label{fig:basic-examples}
\end{figure}

Although superficially similar, Fig.~\ref{fig:basic-examples-correct} and Fig.~\ref{fig:basic-examples-wrong}
behave quite differently. In particular, the \li{match} expression in Fig.~\ref{fig:basic-examples-wrong} is \emph{inexhaustive}: there are values of the scrutinee's type, namely values of the form \li{Leaf n}, for which none of the patterns will match, leading to a run-time error or undefined behavior. 
Moreover, the second pattern in Fig.~\ref{fig:basic-examples-wrong} is \emph{redundant}: there are no values that match \li{Node([x, y])} that do not also match a
previous pattern, here only \li{Node(x::y::tl)}, because \li{[x, y]} is syntactic sugar for \li{x::y::[]}.
%In particular, nodes with exactly two children, e.g. \li{Node([Leaf 1, Leaf 2])}, will match \li{Node(x::y::tl)} because, after desugaring the list literal, we match \li{x} to \li{Leaf 1}, \li{y} to \li{Leaf 2}, and \li{tl} to \li{[]}.
By contrast, Fig.~\ref{fig:basic-examples-correct} has no redundant patterns because \li{Node([x, y])} appears first.
Subtleties like these are easy to miss, particularly for novices but even for experienced programmers when working with complex datatypes.

Fortunately,
modern typed functional languages perform static analyses to detect inexhaustive rule sequences and redundant patterns within a rule sequence.
Exhaustiveness checking compels programmers to consider all possible inputs, including rare cases that may lead to undesirable or undefined behavior. Indeed, many major security issues can be understood as a failure to exhaustively case analyze (e.g. null pointer exceptions).
In the setting of a dependently-typed theorem prover, exhaustiveness checking is necessary to ensure totality and thus logical soundness.
Exhaustiveness checking also supports program evolution: when extending datatype definitions with new constructors, exhaustiveness errors 
serve to alert the programmer of every \li{match} expression that needs to be updated to handle the new case, excepting those that use catch-all wildcard patterns, \li{_} (which for this reason are discouraged in functional programming practice).
Redundancy checking similarly improves software quality by helping programmers avoid subtle order-related bugs and duplicated or unnecessary code paths.


% (Maybe: add gradual typing? add fill-and-resume? add speculative evaluation?)

% (How much to emphasize UI components of Sec 2? Is someone going to ask for a user study??)
% \input{examples}
% \input{calculus}
% \section{Implementation}
% \section{Related Work}

\subsubsection{Typed Holes}
Peanut is based on Hazelnut Live, a typed lambda calculus that includes only expression and type holes.
The Hazel programming environment, which our implementation extends, is based directly on Hazelnut Live. 
Peanut is derived from Hazelnut Live, retaining expression holes and introducing  
structural pattern matching and pattern holes. (Some small technical differences are described in \todo{refer part of Sec 3 where we say more about technical distinctions}.) 
Like Hazelnut Live, evaluation is able to proceed around holes, including pattern holes, in Peanut. An evaluation step is taken 
only if it would be justified for all possible hole fillings.

Peanut omits type holes, i.e. the unknown types from gradual type theory\todo{cite Siek and Taha + Siek et al SNAPL 2015}{}. Type holes obscure type information, so 
is possible to reason statically about exhaustiveness and redundancy in only a limited manner when the scrutinee is of unknown type.\footnote{Constraint-based type inference could be deployed to discover a type for the scrutinee in some cases, at which point it would be possible to use our mechanisms as described. Whether inference is deployed for this purpose is an orthogonal consideration.} Allowing scrutinees of unknown type would also require performing run-time checks, e.g. in the form of casts inserted on terms matched by variables of unknown type. 
We have implemented this cast insertion machinery, which follows straightforwardly from prior work on cast insertion for binary sum types\todo{SNAPL15 paper has it}{}, into Hazel. This machinery is orthogonal to the machinery
we consider in this paper, so we did not include type holes and casts in Peanut.

Hazelnut Live augments each expression hole instance with a closure, which serves to record deferred substitutions 
(it is therefore a contextual type theory\todo{cite cmtt}). This information can be presented to the user and it enables soundly resuming from the current evaluation state when the programmer fills a hole (as long as there are no non-commutative side effects in the language).
The addition of pattern holes does not interfere with this mechanism. Pattern holes do not themselves need closures because patterns bind,
rather than consume, variables. (In a language where patterns contain expressions, e.g. when guards are integrated into patterns\todo{cite ML workshop paper by Reppy}, closures on pattern holes would be necessary to support resumption.) It would not be possible to resume evaluation after filling a pattern hole because doing so can, in general, change the binding structure of the program by introducing shadowing. We leave to future work consideration of 
conditional resumption when pattern hole filling happens not to cause shadowing.

In Hazel, holes are inserted automatically during editing. Formally, Hazel is a type-aware structure editor governed by an edit action semantics derived from Hazelnut, a type-aware structure editor calculus\todo{cite}{} 
for the same language as Hazelnut Live. 
Hazel, by combining machinery from Hazelnut and Hazelnut Live, maintains a powerful continuity invariant: every edit state has a well-defined type
and a well-defined result, both possibly containing holes.
Our extension of Hazel maintains the same invariant, now with the addition of match expressions 
as described in this paper. 
Extending the edit action semantics to allow us to enter patterns presents no special challenges relative to prior work, so we omit formal consideration of editing from this paper.

Moreover, our contributions do not require a structure editor: they are also relevant to languages where typed holes are inserted 
manually by programmers, rather than automatically by an editor. 
For example, GHC Haskell, Agda, and Idris, all feature manually inserted typed holes, both empty and non-empty, in expression position. 
None of these systems support pattern holes as of this writing, however. Haskell does support unbound data constructors in patterns, but these bring compilation to a halt and do not interact with the exhaustiveness and redundancy checker, much less the run-time system. With the appropriate flags set, Haskell will attempt to evaluate programs with expression holes, but it stops with an exception whenever a hole is reached\todo{cite}{}. In contrast, our system supports evaluating around holes of all sorts, as described in \autoref{sec:examples} and formalized in \todo{refer to dynamics section}{}. We hope that this paper will prompt other languages to 
consider adding pattern holes. Our work on the Peanut calculus provides a minimal formal characterization 
that captures the essential character of pattern matching with typed holes and our implementation in Hazel 
demonstrates a practical realization, complete with editor integration.

\subsubsection{Pattern Matching}
Pattern matching has a long history, first appearing in the 1970s in early functional languages
such as NPL, SASL, and ML.
All prior work on pattern matching assumes that the patterns and expressions being matched are
complete, i.e., do not contain any holes.
Uniquely, the indeterminate matching and exhaustiveness and redundancy checks in Peanut
take into account how the programmer may fill or correct these holes in a future edit state.

Much of the early work on pattern matching focused on efficient compilation. Some methods construct decision trees encoding the matching procedure, the goal being to minimize the sizes of the constructed trees, meanwhile obtaining exhaustiveness and redundancy checks as easy by-products \cite{Aitken92smlnj,Baudinet85treepattern,Sestoft96mlpattern}.
Other methods compile pattern matching to backtracking automata \cite{DBLP:journals/jfp/Maranget07}(cite Maranget 1994);
while these methods avoid potential exponential behavior exhibited by decision trees, they are less directly amenable to pattern match analyses and require separate methods for reporting inexhaustive and redundant patterns (cite Maranget 2007).
In our setting of incomplete programs, the question of compilation is not (yet) relevant. Our goal in this paper is to provide a complete semantics of pattern matching with holes that extends the incomplete program evaluation of Hazelnut Live; in so doing, we integrate exhaustiveness and redundancy checks into the type system.
We leave questions of optimization to future work.

Contemporary work focuses on extending pattern match analyses to settings with more sophisticated types and pattern matching features (cite Vaozou et al 2017, Cockx and Abel 2018, Sozeau 2010, GMTM, LYG).
In a similar spirit, our work extends pattern matching to the previously unexplored setting of gradually typed patterns and expressions with holes.
Whereas prior work of this sort concerns itself with analyzing increasingly sophisticated predicates delineating a match from a failure (or divergence in the lazy setting), our work introduces a new pattern matching \emph{outcome}---the indeterminate match---and specifies its static and dynamic semantic underpinnings.
To that end, we narrow our focus to the novel aspects of our system and leave integration with existing, richer pattern matching features and checkers to future work.

% The current state of the art is Lower Your Guards by Graf et al., which can handle GADTs and the wide variety of pattern matching features in Haskell, e.g., guards, view patterns, pattern synonyms, etc.
% This work abandons simple structural pattern matching altogether, instead compiling the various pattern forms into a simpler intermediate representation called a guard tree.



% It is straightforward to check for exhaustiveness and redundancy when pattern-matching on simple algebraic datatypes.
% Much of the literature on pattern matching focuses on extending these checks in the presence of more sophisticated types and pattern matching features.
% \begin{itemize}
%     \item Lower Your Guards
%     \begin{itemize}
%         \item various pattern matching features are compiled to simple guard
%         trees, which are then proceeds into annotated trees that decorate
%         branches with refinement types
%         \item pattern match analyses are performed by generating inhabitants
%         of annotated refinement types
%         \item we do not consider sophisticated pattern matching features
%         and need not resort to such machinery, simple structural matching
%         is sufficient
%         \item matching a guard tree may succeed, fail, or diverge; no indeterminate case
%         \item likely their work could be extended to integrate indeterminate matching
%     \end{itemize}
%     \item other checkers for handling potentially undecidable coverage
%     \begin{itemize}
%         \item GADTs Meet Their Match, Karachalias et al 2015
%         \begin{itemize}
%             \item note: defines redundant clause as one where pattern has no well-typed
%             value that matches (in the context of GADTs)
%         \end{itemize}
%         \item SMT solver for handling guards, Kalvoda and Kerckhove 2019
%         \item case trees for dependently typed / refinement type languages
%         \item emphasize that the failure/uncertainty of checking with these
%         other tools is not the same as indeterminate matching in our work
%     \end{itemize}
% \end{itemize}

% \section{Conclusion}
\begin{quote}
    \emph{As we have struggled through the ages to fathom this strange and wondrous cosmos in which we find ourselves, few ideas have been richer than the concept of nothingness. For to understand anything, as Aristotle argued, we must understand what it is not.}
    \\\null\hfill --- Alan Lightman
\end{quote}

\noindent
Programming can perhaps be understood as a progression, full of fits and starts, from an initial nothingness, an empty hole, toward a complete, meaningful world. This paper continues a promising line of research into applying principled logical methods to understand and support this process \cite{DBLP:conf/popl/OmarVHAH17,HazelnutSNAPL}. In the future, we hope that robust support for typed holes, whether inserted manually or automatically, will be as ubiquitous as other editor services, such as code completion and type inspection, have become today. 
Pattern matching is a central feature of modern typed functional programming languages (and increasingly many other modern 
programming languages), so we believe that the semantics for typed pattern holes contributed by this paper represents a significant step toward realizing the goal of \emph{maximally live} programming environments.

% rntz
% Yuning
% rwh

% \begin{itemize}
% \item richer pattern matching features
% \begin{itemize}
%     \item \cite{DBLP:journals/corr/abs-1909-04160} use SMT solver to handle guards with high precision
%     \item redundancy vs inaccessibility in \cite{DBLP:journals/pacmpl/GrafJS20}, seems relevant to void type
% \end{itemize}
% \item witness generation
% \end{itemize}

% \subsection{Limitations}
% In order to reason about indeterminate exhaustiveness and indeterminate redundancy, we conservatively consider all possible hole fillings. 
% However, we are not encoding exhaustiveness requirement and irredundancy requirement for each branch into a monolithic constraint solving problem. 
% Instead, we check the exhaustiveness and redundancy of each branch separately. 
% The drawback here is that, there are cases where two of the requirements can be indeterminately fulfilled respectively but cannot be fulfilled together regardless of the pattern holes' filling. Given such scenarios, a user may have to, first complete the program in a way such that one of the two requirements is fulfilled, and then get the error/warning from the type checker that the other requirement can not be fulfilled. That is saying, we have not realized the full potential of indeterminacy, giving error/warning at the earliest possible point. On the other hand, it is worth more discussion on whether it is a good thing to just tell user that several requirements cannot be fulfilled at the same time as it may be overwhelming to the user. But that opens up space for pattern synthesis.

% !TEX root = ./patterns-paper.tex
\begin{figure}[t]
$\arraycolsep=4pt\begin{array}{lll}
\tau & ::= &
  \tnum ~\vert~
  \tarr{\tau_1}{\tau_2} ~\vert~
  \tprod{\tau_1}{\tau_2} ~\vert~
  \tsum{\tau_1}{\tau_2} ~\vert~
  \tunit ~\vert~
  \tvoid \\
e & ::= &
  x ~\vert~
  \hnum{n} \\
  & ~\vert~ &
  \hlam{x}{\tau}{e} ~\vert~
  \hap{e_1}{e_2} \\
  & ~\vert~ &
  \hpair{e_1}{e_2} ~\vert~
  \htriv \\
  & ~\vert~ &
  \hinl{\tau}{e} ~\vert~
  \hinr{\tau}{e} ~\vert~
  \hmatch{e}{rs} \\
  & ~\vert~ &
  \hehole{u} ~\vert~
  \hhole{e}{u} \\
rs & ::= &
  \cdot ~\vert~ \hrulesP{r}{rs'} \\
r & ::= &
  \hrul{p}{e} \\
p & ::= &
  x ~\vert~
  \hnum{n} ~\vert~
  \_ ~\vert~
  \hpair{p_1}{p_2} ~\vert~
  \htriv ~\vert~
  \hinlp{p} ~\vert~
  \hinrp{p} ~\vert~
  \hehole{w} ~\vert~
  \hhole{p}{w}
\end{array}$
\end{figure}

\begin{figure}[t]
  \fbox{$\hexptyp{\Gamma}{\Delta}{e}{\tau}$}~~\text{$e$ is of type $\tau$}
\begin{subequations}
\begin{equation}
\inferrule[TVar]{ }{
  \hexptyp{\Gamma, x : \tau}{\Delta}{x}{\tau}
}
\end{equation}
\begin{equation}
\inferrule[TNum]{ }{
  \hexptyp{\Gamma}{\Delta}{\hnum{n}}{\tnum}
}
\end{equation}
\begin{equation}
\inferrule[TLam]{
  \hexptyp{\Gamma, x : \tau_1}{\Delta}{e}{\tau_2}
}{
  \hexptyp{\Gamma}{\Delta}{\hlam{x}{\tau_1}{e}}{\tarr{\tau_1}{\tau_2}}
}
\end{equation}
\begin{equation}
\inferrule[TAp]{
  \hexptyp{\Gamma}{\Delta}{e_1}{\tarr{\tau_2}{\tau}} \\
  \hexptyp{\Gamma}{\Delta}{e_2}{\tau_2}
}{
  \hexptyp{\Gamma}{\Delta}{\hap{e_1}{e_2}}{\tau}
}
\end{equation}
\begin{equation}
\inferrule[TPair]{
  \hexptyp{\Gamma}{\Delta}{e_1}{\tau_1} \\
  \hexptyp{\Gamma}{\Delta}{e_2}{\tau_2}
}{
  \hexptyp{\Gamma}{\Delta}{\hpair{e_1}{e_2}}{\tprod{\tau_1}{\tau_2}}
}
\end{equation}
\begin{equation}
\inferrule[TTriv]{ }{
  \hexptyp{\Gamma}{\Delta}{\htriv}{\tunit}
}
\end{equation}
\begin{equation}
\inferrule[TInL]{
  \hexptyp{\Gamma}{\Delta}{e}{\tau_1}
}{
  \hexptyp{\Gamma}{\Delta}{\hinl{\tau_2}{e}}{\tsum{\tau_1}{\tau_2}}
}
\end{equation}
\begin{equation}
\inferrule[TInR]{
  \hexptyp{\Gamma}{\Delta}{e}{\tau_2}
}{
  \hexptyp{\Gamma}{\Delta}{\hinr{\tau_1}{e}}{\tsum{\tau_1}{\tau_2}}
}
\end{equation}
\begin{equation}
\inferrule[TMatch]{
  \hexptyp{\Gamma}{\Delta}{e_1}{\tau_1} \\
  \hexptyp{\Gamma}{\Delta}{rs}{\trul{\tau_1}{\tau}}
}{
  \hexptyp{\Gamma}{\Delta}{\hmatch{e_1}{rs}}{\tau}
}
\end{equation}
\begin{equation}
\inferrule[TEHole]{ }{
  \hexptyp{\Gamma}{\Delta, u::\tau}{\hehole{u}}{\tau}
}
\end{equation}
\begin{equation}
\inferrule[THole]{
  \hexptyp{\Gamma}{\Delta, u::\tau}{e}{\tau'}
}{
  \hexptyp{\Gamma}{\Delta, u::\tau}{\hhole{e}{u}}{\tau}
}
\end{equation}
\end{subequations}
\end{figure}

\begin{figure}[t]
\fbox{$\hexptyp{\Gamma}{\Delta}{r}{\trul{\tau_1}{\tau_2}}$}~~\text{$r$ is of type $\trulnp{\tau_1}{\tau_2}$}
\begin{subequations}
\begin{equation}
\inferrule[TRule]{
  \hpattyp{p}{\tau_1}{\Gamma'}{\Sigma} \\
  \hexptyp{\Gamma \uplus \Gamma'}{\Delta}{e}{\tau_2}
}{
  \hexptyp{\Gamma}{\Delta \uplus \Sigma}{\hrulP{p}{e}}{\trul{\tau_1}{\tau_2}}
}
\end{equation}
\end{subequations}
\end{figure}

\begin{figure}[t]
\fbox{$\hexptyp{\Gamma}{\Delta}{rs}{\trul{\tau_1}{\tau_2}}$}~~\text{$rs$ is of type $\trulnp{\tau_1}{\tau_2}$}
\begin{subequations}
\begin{equation}
\inferrule[TZeroRule]{ }{
  \hexptyp{\Gamma}{\Delta}{\cdot}{\trul{\tau_1}{\tau_2}}
}
\end{equation}
\begin{equation}
\inferrule[TRules]{
  \hexptyp{\Gamma}{\Delta}{r}{\trul{\tau_1}{\tau_2}} \\
  \hexptyp{\Gamma}{\Delta}{rs}{\trul{\tau_1}{\tau_2}}
}{
  \hexptyp{\Gamma}{\Delta}{\hrulesP{r}{rs}}{\trul{\tau_1}{\tau_2}}
}
\end{equation}
\end{subequations}
\end{figure}

\begin{figure}[t]
\fbox{$\hpattyp{p}{\tau}{\Gamma}{\Sigma}$}~~\text{$p$ matches expression of type $\tau$ and generates context $\Gamma$ and hole context $\Sigma$}
\begin{subequations}
\begin{equation}
\inferrule[PTVar]{ }{
  \hpattyp{x}{\tau}{x : \tau}{\cdot}
}
\end{equation}
\begin{equation}
\inferrule[PTNum]{ }{
  \hpattyp{\hnum{n}}{\tnum}{\cdot}{\cdot}
}
\end{equation}
\begin{equation}
\inferrule[PTWild]{ }{
  \hpattyp{\_}{\tau}{\cdot}{\cdot}
}
\end{equation}
\begin{equation}
\inferrule[PTPair]{
  \hpattyp{p_1}{\tau_1}{\Gamma_1}{\Sigma_1} \\
  \hpattyp{p_2}{\tau_2}{\Gamma_2}{\Sigma_2}
}{
  \hpattyp{\hpair{p_1}{p_2}}{\tprod{\tau_1}{\tau_2}}
    {\Gamma_1 \uplus \Gamma_2}{\Sigma_1 \uplus \Sigma_2}
}
\end{equation}
\begin{equation}
\inferrule[PTTriv]{ }{
  \hpattyp{\htriv}{\tunit}{\cdot}{\cdot}
}
\end{equation}
\begin{equation}
\inferrule[PTInL]{
  \hpattyp{p}{\tau_1}{\Gamma}{\Sigma}
}{
  \hpattyp{\hinlp{p}}{\tsum{\tau_1}{\tau_2}}{\Gamma}{\Sigma}
}
\end{equation}
\begin{equation}
\inferrule[PTInR]{
  \hpattyp{p}{\tau_2}{\Gamma}{\Sigma}
}{
  \hpattyp{\hinrp{p}}{\tsum{\tau_1}{\tau_2}}{\Gamma}{\Sigma}
}
\end{equation}
\begin{equation}
\inferrule[PTEHole]{ }{
  \hpattyp{\hehole{w}}{\tau}{\cdot}{w :: \tau}
}
\end{equation}
\begin{equation}
\inferrule[PTHole]{
  \hpattyp{p}{\tau'}{\Gamma}{\Sigma}
}{
  \hpattyp{\hhole{p}{w}}{\tau}
  {\Gamma}{\Sigma , w :: \tau}
}
\end{equation}
\end{subequations}
\end{figure}

\begin{figure}[t]
\fbox{$\hsubstyp{\theta}{\Gamma}$}~~\text{Substitution $\theta$ is of type $\Gamma$}
\begin{subequations}
\begin{equation}
\inferrule[STEmpty]{ }{
  \hsubstyp{\emptyset}{\cdot}
}
\end{equation}
\begin{equation}
\inferrule[STExtend]{
  \hsubstyp{\theta}{\Gamma} \\
  \hexptyp{\cdot}{\Delta}{e}{\tau}
}{
  \hsubstyp{\theta , x / e}{\Gamma , x : \tau}
}
\end{equation}
\end{subequations}
\end{figure}

\begin{figure}[t]
\fbox{$\hpatmatch{e}{p}{\theta}$}~~\text{Pattern matching $e$ on $p$ emits $\theta$}
\begin{subequations}
\begin{equation}
\inferrule[PMVar]{ }{
  \hpatmatch{e}{x}{x / e}
}
\end{equation}
\begin{equation}
\inferrule[PMNum]{ }{
  \hpatmatch{\hnum{n}}{\hnum{n}}{\cdot}
}
\end{equation}
\begin{equation}
\inferrule[PMWild]{ }{
  \hpatmatch{e}{\_}{\cdot}
}
\end{equation}
\begin{equation}
\inferrule[PMTriv]{ }{
  \hpatmatch{\htriv}{\htriv}{\cdot}
}
\end{equation}
\begin{equation}
\inferrule[PMPair]{
  \hpatmatch{e_1}{p_1}{\theta_1} \\
  \hpatmatch{e_2}{p_2}{\theta_2}
}{
  \hpatmatch{\hpair{e_1}{e_2}}{\hpair{p_1}{p_2}}{\theta_1 \uplus \theta_2}
}
\end{equation}
\begin{equation}
\inferrule[PMInL]{
  \hpatmatch{e}{p}{\theta}
}{
  \hpatmatch{\hinl{\tau}{e}}{\hinlp{p}}{\theta}
}
\end{equation}
\begin{equation}
\inferrule[PMInR]{
  \hpatmatch{e}{p}{\theta}
}{
  \hpatmatch{\hinr{\tau}{e}}{\hinrp{p}}{\theta}
}
\end{equation}
\end{subequations}
\end{figure}

\begin{figure}[t]
\fbox{$\hmaymatch{e}{p}$}~~\text{$e$ may match $p$}
\begin{subequations}
\begin{equation}
\inferrule[MMEHole]{ }{
  \hmaymatch{e}{\hehole{w}}
}
\end{equation}
\begin{equation}
\inferrule[MMHole]{ }{
  \hmaymatch{e}{\hhole{p}{w}}
}
\end{equation}
\begin{equation}
\inferrule[MMExpEHole]{
  p \neq x, \_
}{
  \hmaymatch{\hehole{u}}{p}
}
\end{equation}
\begin{equation}
\inferrule[MMExpHole]{
  p \neq x, \_
}{
  \hmaymatch{\hhole{e}{u}}{p}
}
\end{equation}
\begin{equation}
\inferrule[MMPair1]{
  \hmaymatch{e_1}{p_1} \\
  \hpatmatch{e_2}{p_2}{\theta_2}
}{
  \hmaymatch{\hpair{e_1}{e_2}}{\hpair{p_1}{p_2}}
}
\end{equation}
\begin{equation}
\inferrule[MMPair2]{
  \hpatmatch{e_1}{p_1}{\theta_1} \\
  \hmaymatch{e_2}{p_2}
}{
  \hmaymatch{\hpair{e_1}{e_2}}{\hpair{p_1}{p_2}}
}
\end{equation}
\begin{equation}
\inferrule[MMPair3]{
  \hmaymatch{e_1}{p_1} \\
  \hmaymatch{e_2}{p_2}
}{
  \hmaymatch{\hpair{e_1}{e_2}}{\hpair{p_1}{p_2}}
}
\end{equation}
\begin{equation}
\inferrule[MMInL]{
  \hmaymatch{e}{p}
}{
  \hmaymatch{\hinl{\tau}{e}}{\hinlp{p}}
}
\end{equation}
\begin{equation}
\inferrule[MMInR]{
  \hmaymatch{e}{p}
}{
  \hmaymatch{\hinr{\tau}{e}}{\hinrp{p}}
}
\end{equation}
\end{subequations}
\end{figure}

\begin{figure}[t]
\fbox{$\hnotmatch{e}{p}$}~~\text{$e$ doesn't match $p$}
\begin{subequations}
\begin{equation}
\inferrule[NMPair1]{
  \hnotmatch{e_1}{p_1}
}{
  \hnotmatch{\hpair{e_1}{e_2}}{\hpair{p_1}{p_2}}
}
\end{equation}
\begin{equation}
\inferrule[NMPair2]{
  \hnotmatch{e_2}{p_2}
}{
  \hnotmatch{\hpair{e_1}{e_2}}{\hpair{p_1}{p_2}}
}
\end{equation}
\begin{equation}
\inferrule[NMConfL]{ }{
  \hnotmatch{\hinr{\tau}{e}}{\hinlp{p}}
}
\end{equation}
\begin{equation}
\inferrule[NMConfR]{ }{
  \hnotmatch{\hinl{\tau}{e}}{\hinrp{p}}
}
\end{equation}
\begin{equation}
\inferrule[NMInjL]{
  \hnotmatch{e}{p}
}{
  \hnotmatch{\hinr{\tau}{e}}{\hinlp{p}}
}
\end{equation}
\begin{equation}
\inferrule[NMInjR]{
  \hnotmatch{e}{p}
}{
  \hnotmatch{\hinl{\tau}{e}}{\hinrp{p}}
}
\end{equation}
\end{subequations}
\end{figure}

\begin{figure}[t]
\fbox{$\isVal{e}$}~~\text{$e$ is a value}
\begin{subequations}
\begin{equation}
\inferrule[VNum]{ }{
  \isVal{\hnum{n}}
}
\end{equation}
\begin{equation}
\inferrule[VTriv]{ }{
  \isVal{\htriv}
}
\end{equation}
\begin{equation}
\inferrule[VLam]{ }{
  \isVal{\hlam{x}{\tau}{e}}
}
\end{equation}
\begin{equation}
\inferrule[VPair]{
  \isVal{e_1} \\
  \isVal{e_2}
}{\isVal{\hpair{e_1}{e_2}}}
\end{equation}
\begin{equation}
\inferrule[VInL]{
  \isVal{e}
}{
  \isVal{\hinl{\tau}{e}}
}
\end{equation}
\begin{equation}
\inferrule[VinR]{
  \isVal{e}
}{
  \isVal{\hinr{\tau}{e}}
}
\end{equation}
\end{subequations}
\end{figure}

\begin{figure}[t]
\fbox{$\isIndet{e}$}~~\text{$e$ is indeterminate}
\begin{subequations}
\begin{equation}
\inferrule[IEHole]{ }{
  \isIndet{\hehole{u}}
}
\end{equation}
\begin{equation}
\inferrule[IHole]{
  \isFinal{e}
}{
  \isIndet{\hhole{e}{u}}
}
\end{equation}
\begin{equation}
\inferrule[IAp1]{
  \isIndet{e_1}
}{
  \isIndet{\hap{e_1}{e_2}}
}
\end{equation}
\begin{equation}
\inferrule[IAp2]{
  \isVal{e_1} \\ \isIndet{e_2}
}{
  \isIndet{\hap{e_1}{e_2}}
}
\end{equation}
\begin{equation}
\inferrule[IPair1]{
  \isIndet{e_1}
}{
  \isIndet{\hpair{e_1}{e_2}}
}
\end{equation}
\begin{equation}
\inferrule[IPair2]{
  \isVal{e_1} \\
  \isIndet{e_2}
}{
  \isIndet{\hpair{e_1}{e_2}}
}
\end{equation}
\begin{equation}
\inferrule[IInL]{
  \isIndet{e}
}{
  \isIndet{\hinl{\tau}{e}}
}
\end{equation}
\begin{equation}
\inferrule[IInR]{
  \isIndet{e}
}{
  \isIndet{\hinr{\tau}{e}}
}
\end{equation}
\begin{equation}
\inferrule[IMayMatch]{
  \isFinal{e} \\
  \hmaymatch{e}{p_0}
}{
  \isIndet{\hmatch{e}{\hrules{\hrul{p_0}{e_0}}{rs}}}
}
\end{equation}
\end{subequations}
\end{figure}

\begin{figure}[t]
\fbox{$\isErr{e}$}~~\text{$e$ results in a checked error}
\begin{subequations}
\begin{equation}
\inferrule[EAp1]{
  \isErr{e_1}
}{
  \isErr{\hap{e_1}{e_2}}
}
\end{equation}
\begin{equation}
\inferrule[EAp2]{
  \isVal{e_1} \\ \isErr{e_2}
}{
  \isErr{\hap{e_1}{e_2}}
}
\end{equation}
\begin{equation}
\inferrule[EPair1]{
  \isErr{e_1}
}{
  \isErr{\hpair{e_1}{e_2}}
}
\end{equation}
\begin{equation}
\inferrule[EPair2]{
  \isVal{e_1} \\
  \isErr{e_2}
}{
  \isErr{\hpair{e_1}{e_2}}
}
\end{equation}
\begin{equation}
\inferrule[EInL]{
  \isErr{e}
}{
  \isErr{\hinl{\tau}{e}}
}
\end{equation}
\begin{equation}
\inferrule[EInR]{
  \isErr{e}
}{
  \isErr{\hinr{\tau}{e}}
}
\end{equation}
\begin{equation}
\inferrule[EExpMatch]{
  \isErr{e}
}{
  \isErr{\hmatch{e}{rs}}
}
\end{equation}
\begin{equation}
\inferrule[EExhMatch]{
  \isFinal{e}
}{
  \isErr{\hmatch{e}{\cdot}}
}
\end{equation}
\end{subequations}
\end{figure}

\begin{figure}
\fbox{$\isFinal{e}$}~~\text{$e$ is final}
\begin{subequations}
\begin{equation}
\inferrule[FVal]{
  \isVal{e}
}{
  \isFinal{e}
}
\end{equation}
\begin{equation}
\inferrule[FIndet]{
  \isIndet{e}
}{
  \isFinal{e}
}
\end{equation}
\end{subequations}
\end{figure}

\begin{figure}[t]
\fbox{$e \mapsto e'$}~~\text{$e$ takes an instruction transition to $e'$}
\begin{subequations}
\begin{equation}
\inferrule[ITHole]{
  \htrans{e}{e'}
}{
  \htrans{\hhole{e}{u}}{\hhole{e'}{u}}
}
\end{equation}
\begin{equation}
\inferrule[ITAp1]{
  \htrans{e_1}{e_1'}
}{
  \htrans{\hap{e_1}{e_2}}{\hap{e_1'}{e_2}}
}
\end{equation}
\begin{equation}
\inferrule[ITAp2]{
  \isVal{e_1} \\
  \htrans{e_2}{e_2'}
}{
  \htrans{\hap{e_1}{e_2}}{\hap{e_1}{e_2'}}
}
\end{equation}
\begin{equation}
\inferrule[ITAP]{
  \isVal{e_2}
}{
  \hap{\hlam{x}{\tau}{e_1}}{e_2} \mapsto
    [e_2/x]e_1
}
\end{equation}
\begin{equation}
\inferrule[ITPair1]{
  \htrans{e_1}{e_1'}
}{
  \htrans{\hpair{e_1}{e_2}}{\hpair{e_1'}{e_2}}
}
\end{equation}
\begin{equation}
\inferrule[ITPair2]{
  \isVal{e_1} \\
  \htrans{e_2}{e_2'}
}{
  \htrans{\hpair{e_1}{e_2}}{\hpair{e_1}{e_2'}}
}
\end{equation}
\begin{equation}
\inferrule[ITInL]{
  \htrans{e}{e'}
}{
  \htrans{\hinl{\tau}{e}}{\hinl{\tau}{e'}}
}
\end{equation}
\begin{equation}
\inferrule[ITInR]{
  \htrans{e}{e'}
}{
  \htrans{\hinr{\tau}{e}}{\hinr{\tau}{e'}}
}
\end{equation}
\begin{equation}
\inferrule[ITExpMatch]{
  \htrans{e}{e'}
}{
  \htrans{\hmatch{e}{rs}}{\hmatch{e'}{rs}}
}
\end{equation}
\begin{equation}
\inferrule[ITSuccMatch]{
  \isFinal{e} \\
  \hpatmatch{e}{p_0}{\theta}
}{
  \htrans{
    \hmatch{e}{\hrules{\hrul{p_0}{e_0}}{rs}}
  }{
    [\theta](e_0)
  }
}
\end{equation}
\begin{equation}
\inferrule[ITFailMatch]{
  \isFinal{e} \\
  \hnotmatch{e}{p_0} \\
}{
  \htrans{
    \hmatch{e}{\hrules{\hrul{p_0}{e_0}}{rs}}
  }{
    \hmatch{e}{rs}
  }
}
\end{equation}
\end{subequations}
\end{figure}


\begin{thm}[Preservation]
  \label{thrm:preservation}
  If $\hexptyp{\cdot}{\Delta}{e}{\tau}$ and $\htrans{e}{e'}$
  then $\hexptyp{\cdot}{\Delta}{e'}{\tau}$
\end{thm}

\begin{lem}[Substitution]
  \label{lemma:substitution}
  If $\hexptyp{\Gamma, x : \tau}{\Delta}{e_0}{\tau_0}$ and $\hexptyp{\Gamma}{\Delta}{e}{\tau}$
  then $\hexptyp{\Gamma}{\Delta}{[e/x]e_0}{\tau_0}$
\end{lem}

\begin{lem}[Simultaneous Substitution]
  \label{lemma:simult-substitution}
  If $\hexptyp{\Gamma \uplus \Gamma'}{\Delta}{e}{\tau}$ and $\hsubstyp{\theta}{\Gamma'}$
  then $\hexptyp{\Gamma}{\Delta}{[\theta]e}{\tau}$
\end{lem}
Proof by induction on the derivation of $\hexptyp{\Gamma \uplus \Gamma'}{\Delta}{e}{\tau}$.

\begin{lem}[Substitution Typing]
  \label{lemma:subs-typing}
  If $\hpatmatch{e}{p}{\theta}$ and $\hexptyp{\cdot}{\Delta}{e}{\tau}$ and $\hpattyp{p}{\tau}{\Gamma}{\Sigma}$
  then $\hsubstyp{\theta}{\Gamma}$
\end{lem}
Proof by induction on the derivation of $\hpatmatch{e}{p}{\theta}$.

To apply this lemma in ITSuccMatch case, first apply inversion lemma on premise of preservation theorem,

\begin{thm}[Progress]
  \label{thrm:progrs}
  If $\hexptyp{\cdot}{\Delta}{e}{\tau}$
  then $\isVal{e}$ or $\isErr{e}$ or $\isIndet{e}$ or $\htrans{e}{e'}$ for some $e'$
\end{thm}

\begin{lem}[Matching Progress]
  \label{lemma:match-progress}
  If $\hexptyp{\cdot}{\Delta}{e}{\tau}$ and $\hpattyp{p}{\tau}{\Gamma}{\Sigma}$
  then $\hnotmatch{e}{p}$ or $\hmaymatch{e}{p}$ or $\hpatmatch{e}{p}{\theta}$ for some $\theta$
\end{lem}
Proof by induction on two premises.

\begin{thm}[Stepping Determinism]
  \label{thrm:step-determinism}
  If $\htrans{e}{e'}$ and $\htrans{e}{e''}$ then $e' = e''$
\end{thm}

\begin{thm}[Determinism]
  \label{thrm:determinism}
  If $\hexptyp{\cdot}{\Delta}{e}{\tau}$ then exactly one of the following holds
  \begin{enumerate}
    \item $\isVal{e}$
    \item $\isErr{e}$
    \item $\isIndet{e}$
    \item $\htrans{e}{e'}$ for some unique $e'$
  \end{enumerate}
\end{thm}

\begin{lem}[Matching Determinism]
  \label{lemma:match-determinism}
  If $\hexptyp{\cdot}{\Delta}{e}{\tau}$ and $\hpattyp{p}{\tau}{\Gamma}{\Sigma}$ then exactly one of the following holds
  \begin{enumerate}
    \item $\hpatmatch{e}{p}{\theta}$ for some $\theta$
    \item $\hmaymatch{e}{p}$
    \item $\hnotmatch{e}{p}$
  \end{enumerate}
\end{lem}

\begin{figure}[t]
\fbox{$\ctyp{\xi}{\tau}$}~~\text{Constraint $\xi$ constrains values of type $\tau$}
\begin{subequations}
\begin{equation}
\inferrule[CTruth]{ }{
  \ctyp{\ctruth}{\tau}
}
\end{equation}
\begin{equation}
  \inferrule[CTFalsity]{ }{
  \ctyp{\cfalsity}{\tau}
}
\end{equation}
\begin{equation}
\inferrule[CTNum]{ }{
  \ctyp{\cnum{n}}{\tnum}
}
\end{equation}
\begin{equation}
\inferrule[CTUnit]{ }{
  \ctyp{\cunit}{\tunit}
}
\end{equation}
\begin{equation}
\inferrule[CTAnd]{
  \ctyp{\xi_1}{\tau} \\ \ctyp{\xi_2}{\tau}
}{
  \ctyp{\cand{\xi_1}{\xi_2}}{\tau}
}
\end{equation}
\begin{equation}
\inferrule[CTOr]{
  \ctyp{\xi_1}{\tau} \\ \ctyp{\xi_2}{\tau}
}{
  \ctyp{\cor{\xi_1}{\xi_2}}{\tau}
}
\end{equation}
\begin{equation}
\inferrule[CTInl]{
  \ctyp{\xi_1}{\tau_1}
}{
  \ctyp{\cinl{\xi_1}}{\tsum{\tau_1}{\tau_2}}
}
\end{equation}
\begin{equation}
\inferrule[CTInr]{
  \ctyp{\xi_2}{\tau_2}
}{
  \ctyp{\cinr{\xi_2}}{\tsum{\tau_1}{\tau_2}}
}
\end{equation}
\begin{equation}
\inferrule[CTPair]{
  \ctyp{\xi_1}{\tau} \\ \ctyp{\xi_2}{\tau}
}{
  \ctyp{\cpair{\xi_1}{\xi_2}}{\tau}
}
\end{equation}
\end{subequations}
\end{figure}

\begin{figure}
\fbox{$\cdual{\xi_1} = \xi_2$}~~\text{The dual of a match constraint $\xi_1$ is $\xi_2$}
\begin{subequations}
\begin{align}
  \cdual{\ctruth} &= \cfalsity \\
  \cdual{\cnum{n}} &= \bigvee_{i} \cnum{n_i} \quad \text{where $n_i \neq n$}\\
  \cdual{\cunit} &= \cfalsity \\
  \cdual{\cand{\xi_1}{\xi_2}} &= \cor{\cdual{\xi_1}}{\cdual{\xi_2}} \\
  \cdual{\cor{\xi_1}{\xi_2}} &= \cand{\cdual{\xi_1}}{\cdual{\xi_2}} \\
  \cdual{\cinl{\xi_1}} &= \cor{ \cinl{\cdual{\xi_1}} }{ \cinr{\ctruth} } \\
  \cdual{\cinr{\xi_2}} &= \cor{ \cinr{\cdual{\xi_2}} }{ \cinl{\ctruth} } \\
  \cdual{\cpair{\xi_1}{\xi_2}} &=
  \cor{ \cor{ 
    \cpair{\xi_1}{\cdual{\xi_2}}
  }{
    \cpair{\cdual{\xi_1}}{\xi_2}
  }}{
    \cpair{\cdual{\xi_1}}{\cdual{\xi_2}}
  }
\end{align}
\end{subequations}
\end{figure}

\begin{figure}[t]
\fbox{$\csatisfy{e}{\xi}$}~~\text{Value $e$ satisfies constraint $\xi$}
\begin{subequations}
\begin{equation}
\inferrule[CSTruth]{ }{
  \csatisfy{e}{\ctruth}
}
\end{equation}
\begin{equation}
\inferrule[CSNum]{ }{
  \csatisfy{\hnum{n}}{\cnum{n}}
}
\end{equation}
\begin{equation}
\inferrule[CSUnit]{ }{
  \csatisfy{\htriv}{\cunit}
}
\end{equation}
\begin{equation}
\inferrule[CSAnd]{
  \csatisfy{e}{\xi_1} \\
  \csatisfy{e}{\xi_2}
}{
  \csatisfy{e}{\cand{\xi_1}{\xi_2}}
}
\end{equation}
\begin{equation}
\inferrule[CSOr1]{
  \csatisfy{e}{\xi_1}
}{
  \csatisfy{e}{\cor{\xi_1}{\xi_2}}
}
\end{equation}
\begin{equation}
\inferrule[CSOr2]{
  \csatisfy{e}{\xi_2}
}{
  \csatisfy{e}{\cor{\xi_1}{\xi_2}}
}
\end{equation}
\begin{equation}
\inferrule[CSInl]{
  \csatisfy{e_1}{\xi_1}
}{
  \csatisfy{
    \hinl{\tau_2}{e_1}
  }{
    \cinl{\xi_1}
  }
}
\end{equation}
\begin{equation}
\inferrule[CSInr]{
  \csatisfy{e_2}{\xi_2}
}{
  \csatisfy{
    \hinr{\tau_1}{e_2}
  }{
    \cinr{\xi_2}
  }
}
\end{equation}
\begin{equation}
\inferrule[CSPair]{
  \csatisfy{e_1}{\xi_1} \\
  \csatisfy{e_2}{\xi_2}
}{
\csatisfy{\hpair{e_1}{e_2}}{\cpair{\xi_1}{\xi_2}}
}
\end{equation}
\end{subequations}
\end{figure}

\begin{lem}[Constraint Duality]
  \label{lemma:const-duality}
  If $\ctyp{\xi}{\tau}$ and $\hexptyp{\cdot}{\cdot}{e}{\tau}$ and $\isVal{e}$
  then $\csatisfy{e}{\cdual{\xi}}$ iff $\cnotsatisfy{e}{\xi}$
\end{lem}

\begin{defn}[Constraint Entailment]
  \label{lemma:const-entailment}
  $\csatisfy{\xi_1}{\xi_2}$ iff $\csatisfy{e}{\xi_1}$ implies $\csatisfy{e}{\xi_2}$ for all $e$
\end{defn}

\begin{corol}[Constraint Material Entailment]
  \label{lemma:const-material-entailment}
  $\csatisfy{\xi_1}{\xi_2}$ iff $\csatisfy{\ctruth}{\cor{\cdual{\xi_1}}{\xi_2}}$
\end{corol}

\begin{figure}[t]
\fbox{$\cpattyp{p}{\tau}{\xi}{\Gamma}$}~~\text{$p$ matches expression of type $\tau$, emits match constraint $\xi$ and generates context $\Gamma$}
\begin{subequations}
\begin{equation}
\inferrule[CTPVar]{ }{
  \cpattyp{x}{\tau}{\ctruth}{x : \tau}
}
\end{equation}
\begin{equation}
\inferrule[CTPNum]{ }{
  \cpattyp{\hnum{n}}{\tnum}{\cnum{n}}{\cdot}
}
\end{equation}
\begin{equation}
\inferrule[CTPWild]{ }{
  \cpattyp{\_}{\tau}{\ctruth}{\cdot}
}
\end{equation}
\begin{equation}
\inferrule[CTPTriv]{ }{
  \cpattyp{\htriv}{\tunit}{\cunit}{\cdot}
}
\end{equation}
\begin{equation}
\inferrule[CTPInL]{
  \cpattyp{p}{\tau_1}{\xi}{\Gamma}
}{
  \cpattyp{\hinlp{p}}{\tsum{\tau_1}{\tau_2}}{\cinl{\xi}}{\Gamma}
}
\end{equation}
\begin{equation}
\inferrule[CTPInR]{
  \cpattyp{p}{\tau_2}{\xi}{\Gamma}
}{
  \cpattyp{\hinrp{p}}{\tsum{\tau_1}{\tau_2}}{\cinr{\xi}}{\Gamma}
}
\end{equation}
\begin{equation}
\inferrule[CTPPair]{
  \cpattyp{p_1}{\tau_1}{\xi_1}{\Gamma_1} \\
  \cpattyp{p_2}{\tau_2}{\xi_2}{\Gamma_2}
}{
  \cpattyp{\hpair{p_1}{p_2}}{\tprod{\tau_1}{\tau_2}}
  {\cpair{\xi_1}{\xi_2}}{\Gamma_1 \uplus \Gamma_2}
}
\end{equation}
\end{subequations}
\end{figure}

\begin{figure}[t]
\fbox{$\crultyp{\Gamma}{r}{\tau}{\xi}{\tau'}$}~~\text{$r$ transforms values of type $\tau$, constrained by $\xi$, to type $\tau'$}
\begin{equation}
\inferrule[CTRule]{
  \cpattyp{p}{\tau}{\xi}{\Gamma'} \\
  \hexptyp{\Gamma \uplus \Gamma'}{\cdot}{e}{\tau'}
}{
  \crultyp{\Gamma}{\hrul{p}{e}}{\tau}{\xi}{\tau'}
}
\end{equation}
\end{figure}

\begin{figure}[t]
\fbox{$\crultyp{\Gamma}{rs}{\tau}{\xi}{\tau'}$}~~\text{all rules in $rs$ transform values of type $\tau$, constrained by $\xi$, to type $\tau'$}
\begin{subequations}
\begin{equation}
\inferrule[CTZeroRule]{ }{
  \crultyp{\Gamma}{\cdot}{\tau}{\cfalsity}{\tau'}
}
\end{equation}
\begin{equation}
\inferrule[CTRules]{
  \crultyp{\Gamma}{r}{\tau}{\xi}{\tau'} \\
  \crultyp{\Gamma}{rs}{\tau}{\xi'}{\tau'}
}{
  \crultyp{\Gamma}{\hrules{r}{rs}}{\tau}{\cor{\xi}{\xi'}}{\tau'}
}
\end{equation}
\end{subequations}
\end{figure}

\begin{figure}[t]
\fbox{$\hexptyp{\Gamma}{\Delta}{e}{\tau}$}~~\text{$e$ is of type $\tau$}
\begin{subequations}
\begin{equation}
\inferrule[CTMatch]{
  \hexptyp{\Gamma}{\cdot}{e}{\tau} \\
  \crultyp{\Gamma}{rs}{\tau}{\xi}{\tau'} \\
  \csatisfy{\ctruth}{\xi}
}{
  \hexptyp{\Gamma}{\cdot}{\hmatch{e}{rs}}{\tau'}
}
\end{equation}
\end{subequations}
\end{figure}

\begin{lem}[Constraint Matching Coherence]
  \label{lemma:const-matching-coherence}
  Suppose that $\cpattyp{p}{\tau}{\xi}{\Gamma}$. For all $\hexptyp{\cdot}{\cdot}{e}{\tau}$ such that $\isVal{e}$ we have
  \begin{enumerate}
    \item $\csatisfy{e}{\xi}$ iff $\hpatmatch{e}{p}{\theta}$
    \item $\csatisfy{e}{\cdual{\xi}}$ iff $\hnotmatch{e}{p}$
  \end{enumerate}
\end{lem}

\begin{thm}[Exhaustive progress]
 \label{thrm:exhau-progress}
 If $\hexptyp{\cdot}{\cdot}{e}{\tau}$ then either $\isVal{e}$ or $\htrans{e}{e'}$ for some $e'$.
\end{them}

%\clearpage
\bibliography{all.short}

\ifarxiv
\clearpage
\appendix
%\input{appendix-defns}
%\input{extensions}
\else

\fi
% \input{misc}
% \input{implementation-appendix}

\end{document}
