\documentclass[acmsmall,screen,review,nonacm]{acmart}

\usepackage{amsmath,amsthm}

\usepackage{stmaryrd} % llparenthesis
\usepackage{anyfontsize} % workaround for font size difference warning

\usepackage{cancel} % slash over symbol

\newcommand{\highlight}[1]{\colorbox{yellow}{$\displaystyle #1$}}

\newcommand{\todo}[1]{{\color{red} TODO: #1}}

%% Joshua Dunfield macros
\def\OPTIONConf{1}
\usepackage{jdunfield}
\usepackage{rulelinks} % hyperlink of rule name

\citestyle{acmauthoryear}

\newtheoremstyle{slplain}% name
  {.15\baselineskip\@plus.1\baselineskip\@minus.1\baselineskip}% Space above
  {.15\baselineskip\@plus.1\baselineskip\@minus.1\baselineskip}% Space below
  {\slshape}% Body font
  {\parindent}%Indent amount (empty = no indent, \parindent = para indent)
  {\bfseries}%  Thm head font
  {.}%       Punctuation after thm head
  { }%      Space after thm head: " " = normal interword space;
       %       \newline = linebreak
  {}%       Thm head spec
\theoremstyle{slplain}
\newtheorem{thm}{Theorem}  % Numbered with the equation counter
\numberwithin{thm}{section}
\newtheorem{defn}[thm]{Definition}
\newtheorem{lem}[thm]{Lemma}
\newtheorem{prop}[thm]{Proposition}
\newtheorem{corol}[thm]{Corollary}
% !TEX root = pattern-paper.tex
\usepackage{centernot}
% reverse Vdash
\newcommand{\dashV}{\mathbin{\rotatebox[origin=c]{180}{$\Vdash$}}}

% Violet hotdogs; highlight color helps distinguish them
\newcommand{\llparenthesiscolor}{\textcolor{violet}{\llparenthesis}}
\newcommand{\rrparenthesiscolor}{\textcolor{violet}{\rrparenthesis}}

% HTyp and HExp
\newcommand{\hcomplete}[1]{#1~\mathsf{complete}}

% HTyp
\newcommand{\htau}{\dot{\tau}}
\newcommand{\tarr}[2]{\inparens{#1 \rightarrow #2}}
\newcommand{\tarrnp}[2]{#1 \rightarrow #2}
\newcommand{\trul}[2]{\inparens{#1 \Rightarrow #2}}
\newcommand{\trulnp}[2]{#1 \Rightarrow #2}
\newcommand{\tnum}{\mathtt{num}}
\newcommand{\tehole}{\llparenthesiscolor\rrparenthesiscolor}
\newcommand{\tsum}[2]{\inparens{#1 + #2}}
\newcommand{\tprod}[2]{\inparens{#1 \times #2}}
\newcommand{\tunit}{\mathtt{1}}
\newcommand{\tvoid}{\mathtt{0}}

\newcommand{\tlabeledsum}[1]{+\mathopen{}\left\{#1\right\}}

\newcommand{\tcompat}[2]{#1 \sim #2}
\newcommand{\tincompat}[2]{#1 \nsim #2}


% HTag
\newcommand{\tagC}{\underline{\mathbf{C}}}
\newcommand{\tagset}{\mathcal{C}}
\newcommand{\tagehole}[1]{\llparenthesiscolor\rrparenthesiscolor^{#1}}
\newcommand{\taghole}[2]{\llparenthesiscolor#1\rrparenthesiscolor^{#2}}

\newcommand{\tagmaymatch}[2]{#1 \mathbin{?} #2}

% HExp
\newcommand{\hexp}{\dot{e}}
\newcommand{\hlam}[3]{\lambda #1:#2.#3}
\newcommand{\hap}[2]{#1(#2)}
\newcommand{\hapP}[2]{(#1)~(#2)} % Extra paren around function term
\newcommand{\hnum}[1]{\underline{#1}}
\newcommand{\hadd}[2]{\inparens{#1 + #2}}
\newcommand{\hpair}[2]{(#1 , #2)}
\newcommand{\htriv}{()}
\newcommand{\hehole}[1]{\llparenthesiscolor\rrparenthesiscolor^{#1}}
\newcommand{\hhole}[2]{\llparenthesiscolor#1\rrparenthesiscolor^{#2}}
\newcommand{\heholep}[1]{\llparenthesiscolor\rrparenthesiscolor^{#1}}
\newcommand{\hholep}[3]{\llparenthesiscolor#1\rrparenthesiscolor^{#2}_{#3}}
\newcommand{\hindet}[1]{\lceil#1\rceil}
\newcommand{\hinj}[3]{\mathtt{inj}_{#1}^{#2}({#3})}
\newcommand{\hinl}[2]{\mathtt{inl}_{#1}({#2})}
\newcommand{\hinr}[2]{\mathtt{inr}_{#1}({#2})}
\newcommand{\hinjp}[2]{\mathtt{inj}_{#1}(#2)}
\newcommand{\hinlp}[1]{\mathtt{inl}(#1)}
\newcommand{\hinrp}[1]{\mathtt{inr}(#1)}
\newcommand{\hfst}[1]{\mathtt{fst}(#1)}
\newcommand{\hsnd}[1]{\mathtt{snd}(#1)}
\newcommand{\hmatch}[2]{\mathtt{match}(#1)~\{#2\}}
\newcommand{\hcase}[5]{\mathtt{case}({#1},{#2}.{#3},{#4}.{#5})}
\newcommand{\hrules}[2]{#1 \mid #2}
\newcommand{\hrulesP}[2]{\inparens{#1 \mid #2}}
\newcommand{\hrul}[2]{#1 \Rightarrow #2}
\newcommand{\hrulP}[2]{\inparens{#1 \Rightarrow #2}}

\newcommand{\refutable}[1]{#1 ~\mathtt{refutable}_?}
\newcommand{\frefutable}[1]{\textit{refutable}_?\inparens{#1}}
\newcommand{\possible}[1]{#1 ~\mathtt{possible}}
\newcommand{\fpossible}[1]{\textit{possible}\inparens{#1}}
\newcommand{\inValues}[3]{#1 \in \mathtt{values}[#2]\inparens{#3}}

\newcommand{\hGamma}{\dot{\Gamma}}
\newcommand{\domof}[1]{\text{dom}(#1)}
\newcommand{\hsyn}[3]{#1 \vdash #2 \Rightarrow #3}
\newcommand{\hana}[3]{#1 \vdash #2 \Leftarrow #3}
\newcommand{\hexptyp}[4]{#1 \mathbin{;} #2 \vdash #3 : #4}
\newcommand{\hpattyp}[4]{#4 \vdash #1 : #2 \dashV #3}
\newcommand{\hsubstyp}[4]{#1 \mathbin{;} #2 \vdash #3 : #4}
\newcommand{\hpatmatch}[3]{#1 \vartriangleright #2 \dashV #3}
\newcommand{\hnotnotmatch}[2]{#1 \mathbin{\cancel{\bot}} #2}
\newcommand{\hnotmatch}[2]{#1 \mathbin{\bot} #2}
\newcommand{\hmaymatch}[2]{#1 \mathbin{?} #2}
\newcommand{\htrans}[2]{#1 \mapsto #2}
\newcommand{\hlongtrans}[2]{#1 \newline \mapsto #2}

\newcommand{\isVal}[1]{#1 ~\mathtt{val}}
\newcommand{\isErr}[1]{#1 ~\mathtt{err}}
\newcommand{\isIndet}[1]{#1 ~\mathtt{indet}}
\newcommand{\isFinal}[1]{#1 ~\mathtt{final}}
\newcommand{\notIntro}[1]{#1 ~\mathtt{notintro}}
\newcommand{\fnotIntro}[1]{\textit{notintro}\inparens{#1}}

\newcommand{\setof}[1]{\{#1\}}

% ZTyp and ZExp
\newcommand{\zlsel}[1]{{\bowtie}{#1}}
\newcommand{\zrsel}[1]{{#1}{\bowtie}}
\newcommand{\zwsel}[1]{
  \setlength{\fboxsep}{0pt}
  \colorbox{green!10!white!100}{
    \ensuremath{{{\textcolor{Green}{{\hspace{-2px}\triangleright}}}}{#1}{\textcolor{Green}{\triangleleft{\vphantom{\tehole}}}}}}
}

\newcommand{\removeSel}[1]{#1^{\diamond}}

% ZTyp
\newcommand{\ztau}{\hat{\tau}}

% ZExp
\newcommand{\zexp}{\hat{e}}

% rules with pointer

\newcommand{\zrulsP}[3]{\inparens{#1 \mid #2 \mid #3}}
\newcommand{\zruls}[3]{#1 \mid #2 \mid #3}
\newcommand{\zrules}{\hat{rs}}
\newcommand{\rmpointer}[1]{\inparens{#1}^\diamond}

% Constraint
\newcommand{\hxi}{\dot{\xi}}
\newcommand{\ctyp}[2]{#1 : #2}
\newcommand{\cdual}[1]{\overline{#1}}
\newcommand{\ctruify}[1]{\dot{\top}\inparens{#1}}
\newcommand{\cfalsify}[1]{\dot{\bot}\inparens{#1}}
\newcommand{\ctruth}{\top}
\newcommand{\cfalsity}{\bot}
\newcommand{\cnum}[1]{\underline{#1}}
\newcommand{\cnotnum}[1]{\underline{\cancel{#1}}}
\newcommand{\cunit}{()}
\newcommand{\cunknown}{\mathtt{?}}
\newcommand{\cor}[2]{#1 \vee #2}
\newcommand{\cand}[2]{#1 \wedge #2}
\newcommand{\cinj}[3]{\mathtt{inj}_{#1}^{#2}(#3)}
\newcommand{\notC}{\cancel{C}}
\newcommand{\nottagC}{\cancel{\tagC}}
\newcommand{\cinl}[1]{\mathtt{inl}(#1)}
\newcommand{\cinr}[1]{\mathtt{inr}(#1)}
\newcommand{\cpair}[2]{(#1 , #2)}
\newcommand{\csatisfy}[2]{#1 \models #2}
\newcommand{\cnotsatisfy}[2]{#1 \centernot{\models} #2}
\newcommand{\cmaysatisfy}[2]{#1 \models_{?} #2}
\newcommand{\cnotmaysatisfy}[2]{#1 \centernot{\models}_{?} #2}
\newcommand{\csatisfyormay}[2]{#1 \models_{?}^{\dag} #2}
\newcommand{\cnotsatisfyormay}[2]{#1 \centernot{\models}_{?}^{\dag} #2}
%temporary change, need to get it back to make appendix right
%\newcommand{\csatisfy}[2]{#1 \dot{\models} #2}
%\newcommand{\cnotsatisfy}[2]{#1 \centernot{\dot{\models}} #2}
%\newcommand{\cmaysatisfy}[2]{#1 \dot{\models}_{?} #2}
%\newcommand{\cnotmaysatisfy}[2]{#1 \centernot{\dot{\models}}_{?} #2}
%\newcommand{\csatisfyormay}[2]{#1 \dot{\models}_{?}^{\dag} #2}
%\newcommand{\cnotsatisfyormay}[2]{#1 \centernot{\dot{\models}}_{?}^{\dag} #2}
\newcommand{\ccsatisfy}[2]{#1 \models #2}
\newcommand{\ccnotsatisfy}[2]{#1 \centernot{\models} #2}
\newcommand{\cincon}[1]{#1 ~\mathtt{incon}}
\newcommand{\cmayincon}[1]{#1 ~\cancel{\mathtt{not~incon}}}

\newcommand{\fsatisfy}[2]{\textit{satisfy}\inparens{#1, #2}}
\newcommand{\fmaysatisfy}[2]{\textit{maysatisfy}\inparens{#1, #2}}
\newcommand{\fsatisfyormay}[2]{\textit{satisfyormay}\inparens{#1, #2}}

\newcommand{\chpattyp}[5]{#5 \vdash #1 : #2 [#3] \dashV #4}
\newcommand{\crultyp}[5]{#1 \vdash #2 : #3 [#4] \Rightarrow #5}
\newcommand{\chrultyp}[6]{#1 \mathbin{;} #2 \vdash #3 : #4 [#5] \Rightarrow #6}
\newcommand{\crulstyp}[6]{#1 \vdash [#2] #3 : #4 [#5] \Rightarrow #6}
\newcommand{\chrulstyp}[7]{#1 \mathbin{;} #2 \vdash [#3] #4 : #5 [#6] \Rightarrow #7}
\newcommand{\czrulstyp}[7]{#1 \mathbin{;} #2 \vdash [#3] #4 [#5] : #6 \Rightarrow #7}

% Direction
\newcommand{\dParent}{\mathtt{parent}}
\newcommand{\dChildn}[1]{\mathtt{child}~\mathtt{{#1}}}
\newcommand{\dChildnm}[1]{\mathtt{child}~{#1}}

% Action
\newcommand{\aMove}[1]{\mathtt{move}~#1}
	\newcommand{\zrightmost}[1]{\mathsf{rightmost}(#1)}
	\newcommand{\zleftmost}[1]{\mathsf{leftmost}(#1)}
\newcommand{\aSelect}[1]{\mathtt{sel}~#1}
\newcommand{\aDel}{\mathtt{del}}
\newcommand{\aReplace}[1]{\mathtt{replace}~#1}
\newcommand{\aConstruct}[1]{\mathtt{construct}~#1}
\newcommand{\aConstructx}[1]{#1}
\newcommand{\aFinish}{\mathtt{finish}}

\newcommand{\performAna}[5]{#1 \vdash #2 \xlongrightarrow{#4} #5 \Leftarrow #3}
\newcommand{\performAnaI}[5]{#1 \vdash #2 \xlongrightarrow{#4}\hspace{-3px}{}^{*}~ #5 \Leftarrow #3}
\newcommand{\performSyn}[6]{#1 \vdash #2 \Rightarrow #3 \xlongrightarrow{#4} #5 \Rightarrow #6}
\newcommand{\performSynI}[6]{#1 \vdash #2 \Rightarrow #3 \xlongrightarrow{#4}\hspace{-3px}{}^{*}~ #5 \Rightarrow #6}
\newcommand{\performTyp}[3]{#1 \xlongrightarrow{#2} #3}
\newcommand{\performTypI}[3]{#1 \xlongrightarrow{#2}\hspace{-3px}{}^{*}~#3}

\newcommand{\performMove}[3]{#1 \xlongrightarrow{#2} #3}
\newcommand{\performDel}[2]{#1 \xlongrightarrow{\aDel} #2}

% Form
\newcommand{\farr}{\mathtt{arrow}}
\newcommand{\fnum}{\mathtt{num}}
\newcommand{\fsum}{\mathtt{sum}}

\newcommand{\fasc}{\mathtt{asc}}
\newcommand{\fvar}[1]{\mathtt{var}~#1}
\newcommand{\flam}[1]{\mathtt{lam}~#1}
\newcommand{\fap}{\mathtt{ap}}
% \newcommand{\farg}{\mathtt{arg}}
\newcommand{\fnumlit}[1]{\mathtt{lit}~#1}
\newcommand{\fplus}{\mathtt{plus}}
\newcommand{\fhole}{\mathtt{hole}}
\newcommand{\fnehole}{\mathtt{nehole}}

\newcommand{\finj}[1]{\mathtt{inj}~#1}
\newcommand{\fcase}[2]{\mathtt{case}~#1~#2}

% Talk about formal rules in example
\newcommand{\refrule}[1]{\textrm{Rule~(#1)}}

\newcommand{\herase}[1]{\left|#1\right|_\textsf{erase}}

\newcommand{\arrmatch}[2]{#1 \blacktriangleright_{\rightarrow} #2}


\newcommand{\TABperformAna}[5]{#1 \vdash & #2                & \xlongrightarrow{#4} & #5 & \Leftarrow #3}
\newcommand{\TABperformSyn}[6]{#1 \vdash & #2 \Rightarrow #3 & \xlongrightarrow{#4} & #5 \Rightarrow #6}
\newcommand{\TABperformTyp}[3]{& #1 & \xlongrightarrow{#2} & #3}

\newcommand{\TABperformMove}[3]{#1 & \xlongrightarrow{#2} & #3}
\newcommand{\TABperformDel}[2]{#1 \xlongrightarrow{\aDel} #2}

\newcommand{\sumhasmatched}[2]{#1 \mathrel{\textcolor{black}{\blacktriangleright_{+}}} #2}

\newcommand{\subminsyn}[1]{\mathsf{submin}_{\Rightarrow}(#1)}
\newcommand{\subminana}[1]{\mathsf{submin}_{\Leftarrow}(#1)}


\newcommand{\inparens}[1]{{\color{gray}(}#1{\color{gray})}}
\newcommand{\true}{\text{true}}
\newcommand{\false}{\text{false}}

\newcommand{\textnot}{\text{not }}
\newcommand{\textor}{\text{ or }}
\newcommand{\textand}{\text{ and }}
\newcommand{\textif}{\quad\text{if }}

%% rule names for appendix
\newcommand{\rname}[1]{\textsc{#1}}
\newcommand{\gap}{\vspace{7pt}}

%% just for examples
\newcommand{\match}{\mathtt{match}}
\newcommand{\?}{\mathtt{?}}
\newcommand{\nil}{[~]}
\newcommand{\some}[1]{\mathtt{Some\inparens{#1}}}
\newcommand{\none}{\mathtt{None}}

\let\Autoref\undefined
\makeatletter
% an adaption from 
% https://tex.stackexchange.com/questions/15728/multiple-references-with-autoref
% define a macro \Autoref to allow multiple references to be passed to \autoref
\newcommand\Autoref[1]{\@first@ref#1,@}
\def\@throw@dot#1.#2@{#1}% discard everything after the dot
\def\@set@refname#1{%    % set \@refname to autoefname+s using \getrefbykeydefault
    \edef\@tmp{\getrefbykeydefault{#1}{anchor}{}}%
    \xdef\@tmp{\expandafter\@throw@dot\@tmp.@}%
    \ltx@IfUndefined{\@tmp autorefnameplural}%
         {\def\@refname{\@nameuse{\@tmp autorefname}}}%
         {\def\@refname{\@nameuse{\@tmp autorefnameplural}}}%
}
\def\@first@ref#1,#2{%
  \ifx#2@\autoref{#1}\let\@nextref\@gobble% only one ref, revert to normal \autoref
  \else%
    \@set@refname{#1}%  set \@refname to autoref name
    \@refname~\ref{#1}% add autoefname and first reference
    \let\@nextref\@next@ref% push processing to \@next@ref
  \fi%
  \@nextref#2%
}
\def\@next@ref#1,#2{%
   \ifx#2@ and~\ref{#1}\let\@nextref\@gobble% at end: print and+\ref and stop
   \else, \ref{#1}% print  ,+\ref and continue
   \fi%
   \@nextref#2%
}
\makeatother

%%% Local Variables:
%%% mode: latex
%%% TeX-master: "appendix/appendix"
%%% End:


% !TEX root = pattern-paper.tex
%% Statics

% typing of internal expressions
\newrulecommand{TVar}{TVar}
\newrulecommand{TNum}{TNum}
\newrulecommand{TLam}{TLam}
\newrulecommand{TAp}{TAp}
\newrulecommand{TPair}{TPair}
\newrulecommand{TFst}{TFst}
\newrulecommand{TSnd}{TSnd}
\newrulecommand{TInl}{TInl}
\newrulecommand{TInr}{TInr}
\newrulecommand{TMatchZPre}{TMatchZPre}
\newrulecommand{TMatchNZPre}{TMatchNZPre}
\newrulecommand{TEHole}{TEHole}
\newrulecommand{THole}{THole}

% typing of rule and rules
\newrulecommand{TRule}{TRule}
\newrulecommand{TOneRules}{TOneRules}
\newrulecommand{TRules}{TRules}

% typing of pattern
\newrulecommand{PTVar}{PTVar}
\newrulecommand{PTNum}{PTNum}
\newrulecommand{PTWild}{PTWild}
\newrulecommand{PTPair}{PTPair}
\newrulecommand{PTInl}{PTInl}
\newrulecommand{PTInr}{PTInr}
\newrulecommand{PTEHole}{PTEHole}
\newrulecommand{PTHole}{PTHole}

%% Dynamics

% typing of substitution
\newrulecommand{STEmpty}{STEmpty}
\newrulecommand{STExtend}{STExtend}

% syntactically not value
\newrulecommand{NVEHole}{NVEHole}
\newrulecommand{NVHole}{NVHole}
\newrulecommand{NVAp}{NVAp}
\newrulecommand{NVMatch}{NVMatch}
\newrulecommand{NVFst}{NVFst}
\newrulecommand{NVSnd}{NVSnd}

% refutable pattern
\newrulecommand{RNum}{RNum}
\newrulecommand{REHole}{REHole}
\newrulecommand{RHole}{RHole}
\newrulecommand{RInl}{RInl}
\newrulecommand{RInr}{RInr}
\newrulecommand{RPairL}{RPairL}
\newrulecommand{RPairR}{RPairR}

% refutable constraint
\newrulecommand{RXNum}{RXNum}
\newrulecommand{RXNotNum}{RXNotNum}
\newrulecommand{RXUnknown}{RXUnknown}
\newrulecommand{RXFalsity}{RXFalsity}
\newrulecommand{RXInl}{RXInl}
\newrulecommand{RXInr}{RXInr}
\newrulecommand{RXPairL}{RXPairL}
\newrulecommand{RXPairR}{RXPairR}

% possible constraint
\newrulecommand{PTruth}{PTruth}
\newrulecommand{PUnknown}{PUnknown}
\newrulecommand{PNum}{PNum}
\newrulecommand{PInl}{PInl}
\newrulecommand{PInr}{PInr}
\newrulecommand{PPair}{PPair}
\newrulecommand{POrL}{POrL}
\newrulecommand{POrR}{POrR}

% match
\newrulecommand{MVar}{MVar}
\newrulecommand{MNum}{MNum}
\newrulecommand{MWild}{MWild}
\newrulecommand{MPair}{MPair}
\newrulecommand{MNotIntroPair}{MNotIntroPair}
\newrulecommand{MInl}{MInl}
\newrulecommand{MInr}{MInr}

% may match
\newrulecommand{MMEHole}{MMEHole}
\newrulecommand{MMHole}{MMHole}
\newrulecommand{MMNotIntro}{MMNotIntro}
\newrulecommand{MMPairL}{MMPairL}
\newrulecommand{MMPairR}{MMPairR}
\newrulecommand{MMPair}{MMPair}
\newrulecommand{MMInl}{MMInl}
\newrulecommand{MMInr}{MMInr}

% not match
\newrulecommand{NMNum}{NMNum}
\newrulecommand{NMPairL}{NMPairL}
\newrulecommand{NMPairR}{NMPairR}
\newrulecommand{NMConfL}{NMConfL}
\newrulecommand{NMConfR}{NMConfR}
\newrulecommand{NMInl}{NMInl}
\newrulecommand{NMInr}{NMInr}

% value
\newrulecommand{VNum}{VNum}
\newrulecommand{VLam}{VLam}
\newrulecommand{VPair}{VPair}
\newrulecommand{VInl}{VInl}
\newrulecommand{VInr}{VInr}

% indeterminate
\newrulecommand{IEHole}{IEHole}
\newrulecommand{IHole}{IHole}
\newrulecommand{IAp}{IAp}
\newrulecommand{IPairL}{IPairL}
\newrulecommand{IPairR}{IPairR}
\newrulecommand{IPair}{IPair}
\newrulecommand{IFst}{IFst}
\newrulecommand{ISnd}{ISnd}
\newrulecommand{IInl}{IInl}
\newrulecommand{IInr}{IInr}
\newrulecommand{IMatch}{IMatch}

% final
\newrulecommand{FVal}{FVal}
\newrulecommand{FIndet}{FIndet}

% step
\newrulecommand{ITHole}{ITHole}
\newrulecommand{ITApFun}{ITApFun}
\newrulecommand{ITApArg}{ITApArg}
\newrulecommand{ITAp}{ITAp}
\newrulecommand{ITPairL}{ITPairL}
\newrulecommand{ITPairR}{ITPairR}
\newrulecommand{ITFst}{ITFst}
\newrulecommand{ITSnd}{ITSnd}
\newrulecommand{ITFstPair}{ITFstPair}
\newrulecommand{ITSndPair}{ITSndPair}
\newrulecommand{ITInl}{ITInl}
\newrulecommand{ITInr}{ITInr}
\newrulecommand{ITExpMatch}{ITExpMatch}
\newrulecommand{ITSuccMatch}{ITSuccMatch}
\newrulecommand{ITFailMatch}{ITFailMatch}

% typing of constraints
\newrulecommand{CTTruth}{CTTruth}
\newrulecommand{CTFalsity}{CTFalsity}
\newrulecommand{CTUnknown}{CTUnknown}
\newrulecommand{CTNum}{CTNum}
\newrulecommand{CTNotNum}{CTNotNum}
\newrulecommand{CTAnd}{CTAnd}
\newrulecommand{CTOr}{CTOr}
\newrulecommand{CTInl}{CTInl}
\newrulecommand{CTInr}{CTInr}
\newrulecommand{CTPair}{CTPair}

% satisfaction judgment
\newrulecommand{CSTruth}{CSTruth}
\newrulecommand{CSNum}{CSNum}
\newrulecommand{CSNotNum}{CSNotNum}
\newrulecommand{CSAnd}{CSAnd}
\newrulecommandONE{CSOrL}{CSOrL}
\newrulecommandONE{CSOrR}{CSOrR}
\newrulecommand{CSInl}{CSInl}
\newrulecommand{CSInr}{CSInr}
\newrulecommand{CSPair}{CSPair}
\newrulecommand{CSNotIntroPair}{CSNotIntroPair}

% maybe satisfaction judgment
\newrulecommand{CMSUnknown}{CMSUnknown}
\newrulecommand{CMSNotIntro}{CMSNotIntro}
\newrulecommand{CMSAndL}{CMSAndL}
\newrulecommand{CMSAndR}{CMSAndR}
\newrulecommand{CMSAnd}{CMSAnd}
\newrulecommand{CMSOrL}{CMSOrL}
\newrulecommand{CMSOrR}{CMSOrR}
\newrulecommand{CMSInl}{CMSInl}
\newrulecommand{CMSInr}{CMSInr}
\newrulecommand{CMSPairL}{CMSPairL}
\newrulecommand{CMSPairR}{CMSPairR}
\newrulecommand{CMSPair}{CMSPair}

% must or maybe satisfaction judgment
\newrulecommand{CSMSMay}{CSMSMay}
\newrulecommand{CSMSSat}{CSMSSat}

% incon
\newrulecommand{CINCTruth}{CINCTruth}
\newrulecommand{CINCFalsity}{CINCFalsity}
\newrulecommand{CINCNum}{CINCNum}
\newrulecommand{CINCNotNum}{CINCNotNum}
\newrulecommand{CINCAnd}{CINCAnd}
\newrulecommand{CINCOr}{CINCOr}
\newrulecommand{CINCInj}{CINCInj}
\newrulecommand{CINCInl}{CINCInl}
\newrulecommand{CINCInr}{CINCInr}
\newrulecommand{CINCPairL}{CINCPairL}
\newrulecommand{CINCPairR}{CINCPairR}

%% labeled sum dynamics
\newrulecommand{ITInj}{ITInj}
\newrulecommand{VInj}{VInj}
\newrulecommand{IEInj}{IEInj}
\newrulecommand{IEInjHole}{IEInjHole}
\newrulecommand{MInj}{MInj}
\newrulecommand{NMInj}{NMInj}
\newrulecommand{NMInjTag}{NMInjTag}
\newrulecommand{NMInjArg}{NMInjArg}
\newrulecommand{MMInjTag}{MMInjTag}
\newrulecommand{MMInjArg}{MMInjArg}
\newrulecommand{TMMSym}{TMMSym}
\newrulecommand{TMMHole}{TMMHole}
\newrulecommand{TMMEHole}{TMMEHole}
\newrulecommand{TNMTag}{TNMTag}
\newrulecommand{RInjSing}{RInjSing}
\newrulecommand{RInjMult}{RInjMult}

% labeled sum statics
\newrulecommand{TInj}{TInj}
\newrulecommand{PTInj}{PTInj}
\newrulecommand{CTInj}{CTInj}
\newrulecommand{RXInj}{RXInj}
\newrulecommand{PInj}{PInj}
\newrulecommand{CSInj}{CSInj}
\newrulecommand{CMSInjTag}{CMSInjTag}
\newrulecommand{CMSInjArg}{CMSInjArg}

% labeled sum incon
\newrulecommand{CINCInjTag}{CINCInjTag}
\newrulecommand{CINCInjArg}{CINCInjArg}


\begin{document}
  
\title{Pattern Matching with Typed Holes}

\author{Yongwei Yuan}
\authornote{Research advisor: Cyrus Omar; ACM student member number: 9899292; Category: undergraduate}
\email{slark@umich.edu}
\affiliation{%
  \institution{University of Michigan}
}


\maketitle

\section{Introduction}
\label{sec:intro}
Most contemporary programming environments either provide meaningful feedback only for complete programs, or fall back to limited heuristic approaches when the program is incomplete.
Recent work on modeling incomplete programs as typed expressions with \emph{holes} has focused on tackling this problem. \citet{DBLP:conf/popl/OmarVHAH17} describes a static semantics for incomplete functional programs. Based on that, \citet{DBLP:journals/pacmpl/OmarVCH19} develops a dynamic semantics to evaluate such incomplete programs.
These foundational treatments are being implemented and extended in the Hazel programming environment (hazel.org).

\begin{figure}[h]
\begin{minipage}{0.33\textwidth}
\begin{equation}
\begin{split}
  &\mathtt{match} (\hpair{\hinl{\tnum}{\hehole{u}}}{2}) \{\\
  &|~ \hrul{\hpair{\hinrp{x}}{\_}}{x} \\
  &|~ \hrul{\hpair{\_}{x}}{x} \\
  &\}
\end{split}
\label{eq:complete}
\end{equation}
\end{minipage}%
\begin{minipage}{0.33\textwidth}
\begin{equation}
\begin{split}
  &\mathtt{match} (\hpair{\hinl{\tnum}{1}}{2}) \{\\
  &|~ \hrul{\hpair{\hinlp{x}}{\_}}{x} \\
  &|~ \hrul{\hpair{\_}{\hehole{w}}}{\hehole{u}} \\
  &\}
\end{split}
\label{eq:pathole-exhaustive}
\end{equation}
\end{minipage}%
\begin{minipage}{0.33\textwidth}
\begin{equation}
\begin{split}
  &\mathtt{match} (\hpair{\hinl{\tnum}{1}}{2}) \{\\
  &|~ \hrul{\hpair{\hinlp{x}}{\_}}{x} \\
  &|~ \hrul{\hpair{\hinlp{\hehole{w_1}}}{\hehole{w_2}}}{\hehole{u}} \\
  &\}
\end{split}
\label{eq:pathole-redundant}
\end{equation}
\end{minipage}%
\end{figure}

\emph{Pattern matching} is a cornerstone of functional programming languages in the ML family. 
However, \citet{DBLP:journals/pacmpl/OmarVCH19} only supports simple case analysis on binary sum types and does not support nested patterns nor pattern holes.
This paper addresses this problem, focusing on adding full ML-style pattern matching with support for pattern holes to the Hazelnut core calculus and implementing 
this system at full scale into Hazel. Consider the examples below, which contain expression and pattern holes, denoted $\hehole{u}$ (where each hole has a unique name, $u$). 

For the match expression shown in \Eqref{eq:complete}, the scrutinee contains a hole.
By considering the match two rules in order, we observe that
no matter how the hole $u$ is filled, $\hpair{\hinl{\tnum}{\hehole{u}}}{2}$ doesn't match $\hpair{\hinrp{x}}{\_}$,
and always matches $\hpair{\_}{x}$. And thus the value of the match expression is $2$, despite the fact that the program is incomplete.
If, however, we replaced the pattern in the first rule with $(\hinlp{3}, \_)$, it would be impossible to decide whether the pattern matches
or not, and so the match expression would be \emph{indeterminate}, i.e. awaiting hole filling.
%Similarly, if we had replaced the pattern in the first rule with $(\hinlp{x}, \hehole{w})$,
%then it is impossible to tell whether the pattern hole $w$ will match $2$ or not,
%so the match would also be indeterminate.

One of the benefits of pattern matching is that it allows for static reasoning about exhaustiveness and redundancy.
However, reasoning about these matters is subtle and complicated in the presence of holes. 
For example, consider the problem of checking if the rules in \Eqref{eq:pathole-exhaustive} cover all the possibilities, \ie \emph{exhaustiveness}, or if the second rule in \Eqref{eq:pathole-redundant} can never be reached no matter how the holes are filled, \ie \emph{redundancy}. 

This abstract focuses on the formalism of full-fledged pattern matching with typed holes so that the programming environment could give this sort of feedback even when considering incomplete match expressions.

\section{Static Semantics}
\label{sec:statics}

% !TEX root= extended-abstract.tex

\begin{figure}[h]
\judgbox{\hexptyp{\Gamma}{\Delta}{e}{\tau}}
{$e$ is of type $\tau$}

\begin{mathpar}
\Infer{\TVar}{ }{
  \hexptyp{\Gamma, x : \tau}{\Delta}{x}{\tau}
}

\Infer{\TNum}{ }{
  \hexptyp{\Gamma}{\Delta}{\hnum{n}}{\tnum}
}

\Infer{\TLam}{
  \hexptyp{\Gamma, x : \tau_1}{\Delta}{e}{\tau_2}
}{
  \hexptyp{\Gamma}{\Delta}{\hlam{x}{\tau_1}{e}}{\tarr{\tau_1}{\tau_2}}
}

\Infer{\TAp}{
  \hexptyp{\Gamma}{\Delta}{e_1}{\tarr{\tau_2}{\tau}} \\
  \hexptyp{\Gamma}{\Delta}{e_2}{\tau_2}
}{
  \hexptyp{\Gamma}{\Delta}{\hap{e_1}{e_2}}{\tau}
}

\Infer{\TPair}{
  \hexptyp{\Gamma}{\Delta}{e_1}{\tau_1} \\
  \hexptyp{\Gamma}{\Delta}{e_2}{\tau_2}
}{
  \hexptyp{\Gamma}{\Delta}{\hpair{e_1}{e_2}}{\tprod{\tau_1}{\tau_2}}
}

\Infer{\TInL}{
  \hexptyp{\Gamma}{\Delta}{e}{\tau_1}
}{
  \hexptyp{\Gamma}{\Delta}{\hinl{\tau_2}{e}}{\tsum{\tau_1}{\tau_2}}
}

\Infer{\TInR}{
  \hexptyp{\Gamma}{\Delta}{e}{\tau_2}
}{
  \hexptyp{\Gamma}{\Delta}{\hinr{\tau_1}{e}}{\tsum{\tau_1}{\tau_2}}
}

\Infer{\TMatchZPre}{
  \hexptyp{\Gamma}{\Delta}{e}{\tau_1} \\
  \chrulstyp{\Gamma}{\Delta}{\cfalsity}{\hrules{r}{rs}}{\tau_1}{\xi}{\tau_2} \\
  \csatisfyormay{\ctruth}{\xi}
}{
  \hexptyp{\Gamma}{\Delta}{\hmatch{e}{\zruls{\cdot}{r}{rs}}}{\tau_2}
}

\Infer{\TMatchNZPre}{
  \hexptyp{\Gamma}{\Delta}{e}{\tau_1} \\
  \isFinal{e} \\
  \chrulstyp{\Gamma}{\Delta}{\cfalsity}{rs_{pre}}{\tau_1}{\xi_{pre}}{\tau_2} \\
  \chrulstyp{\Gamma}{\Delta}{\xi_{pre}}{\hrules{r}{rs_{post}}}{\tau_1}{\xi_{rest}}{\tau_2} \\
  \cnotsatisfyormay{e}{\xi_{pre}} \\
  \csatisfyormay{\ctruth}{\cor{\xi_{pre}}{\xi_{rest}}}
}{
\hexptyp{\Gamma}{\Delta}{\hmatch{e}{\zruls{rs_{pre}}{r}{rs_{post}}}}{\tau_2}
}

\Infer{\TEHole}{ }{
  \hexptyp{\Gamma}{\Delta, u::\tau}{\hehole{u}}{\tau}
}

\Infer{\THole}{
  \hexptyp{\Gamma}{\Delta, u::\tau}{e}{\tau'}
}{
  \hexptyp{\Gamma}{\Delta, u::\tau}{\hhole{e}{u}}{\tau}
}
\end{mathpar}
\caption{Typing of Internal Expressions}
\label{fig:exptyp}
\end{figure}


\Figref{fig:exptyp} defines part of the internal language.
Different from the examples in \Secref{sec:intro}, the rules in a match expression have a pointer to indicate which rule is being considered.
The pointer would move as we evaluate the match expression and 
is particularly useful when returning an indeterminate match expression to the user, because it shows which rules have been checked.

% !TEX root= pattern-paper.tex

\begin{figure}[ht]
\judgbox{\chrulstyp{\Gamma}{\Delta}{\xi_{pre}}{rs}{\tau_1}{\xi_{rs}}{\tau_2}}
{$rs$ is of type $\trulnp{\tau_1}{\tau_2}$}

\begin{mathpar}
\Infer{\TOneRules}{
  \chrultyp{\Gamma}{\Delta}{r}{\tau_1}{\xi_r}{\tau_2} \\
  \cnotsatisfy{\xi_r}{\xi_{pre}}
}{
  \chrulstyp{\Gamma}{\Delta}{\xi_{pre}}{\hrulesP{r}{\cdot}}{\tau_1}{\xi_r}{\tau_2}
}

\Infer{\TRules}{
  \chrultyp{\Gamma}{\Delta}{r}{\tau_1}{\xi_r}{\tau_2} \\
  \chrulstyp{\Gamma}{\Delta}{\cor{\xi_{pre}}{\xi_r}}{rs}
  {\tau_1}{\xi_{rs}}{\tau_2} \\
  \cnotsatisfy{\xi_r}{\xi_{pre}}
}{
  \chrulstyp{\Gamma}{\Delta}{\xi_{pre}}{\hrules{r}{rs}}
  {\tau_1}{\cor{\xi_r}{\xi_{rs}}}{\tau_2}
}
\end{mathpar}
\caption{Typing of Rules}
\label{fig:rulestyp}
\end{figure}


\Figref{fig:rulestyp} describes the inductive construction of rules. 

Constraints $\xi$ is emitted from patterns. The syntax of them is given in \Figref{fig:pat-constraint}. A constraint predict the set of expressions that match the corresponding pattern.
For example, variable patterns or wild card patterns emit \emph{Truth} constraint $\ctruth$;
a pattern hole emits \emph{Unknown} constraint $?$;
patterns in multiple rules altogether emit disconjunction of constraints $\xi_1 \vee \xi_2 \vee \dots \vee \xi_n$.

The dual of a constraint, $\cdual{\xi}$, predicts the complement of the set of expressions that match the corresponding pattern of $\xi$.
\eg $\cdual{\cnum{n}}=\cnotnum{n}$ means all numbers other than $n$;
$\cdual{\cor{\xi_1}{\xi_2}}=\cand{\cdual{\xi_1}}{\cdual{\xi_2}}$ means the complement of the union of two constraints.

% !TEX root= extended-abstract.tex

\begin{figure}[h]
$\arraycolsep=4pt\begin{array}{lll}
p & ::= &
  x ~\vert~
  \hnum{n} ~\vert~
  \_ ~\vert~
  \hpair{p_1}{p_2} ~\vert~
  \hinlp{p} ~\vert~
  \hinrp{p} ~\vert~
  \hehole{w} ~\vert~
  \hhole{p}{w} \\
\xi & ::= &
  \ctruth ~\vert~
  \cfalsity ~\vert~
  \cnum{n} ~\vert~
  \cnotnum{n} ~\vert~
  \cand{\xi_1}{\xi_2} ~\vert~
  \cor{\xi_1}{\xi_2} ~\vert~
  \cpair{\xi_1}{\xi_2} ~\vert~
  \cinl{\xi} ~\vert~
  \cinr{\xi} ~\vert~
  \cunknown
\end{array}$
\caption{Syntax of patterns and constraints}
\label{fig:pat-constraint}
\end{figure}



\section{Exhaustiveness and Redundancy}
\label{sec:exhaustiveness-redundancy}
We define a satisfaction judgment between an expression and a constraint, $\csatisfy{e}{\xi}$, to represent if the expression match the associated pattern(s),
\eg any expression satisfies Truth constraint, $\csatisfy{e}{\ctruth}$.

Similarly, we define a maybe satisfaction judgment $\cmaysatisfy{e}{\xi}$ with the meaning that $e$ may match the associated pattern(s). For example, an expression hole can be filled by any expression and thus may satisfy any constraint, \ie $\cmaysatisfy{\hehole{u}}{\xi}$ and $\cmaysatisfy{\hhole{e}{u}}{\xi}$.

Note that matching process only makes sense for \emph{final} expressions, \ie either values or indeterminate. The same goes for two sorts of satisfaction judgments.

\Definitionref{defn:exhaustiveness} gives the formal definition of exhaustiveness or maybe exhaustiveness.
Rule \TMatchZPre uses it to enforce the rules associated with $\xi$ cover or may cover all the possibilities of the scrutinee.
In \Eqref{eq:pathole-exhaustive}, we notice that for any final expression $e$, either 
$\csatisfy{e}{\cpair{\cinl{\ctruth}}{\ctruth}}$ or $\cmaysatisfy{e}{\cpair{\ctruth}{\cunknown}}$.
Therefore, \Eqref{eq:pathole-exhaustive} may be exhaustive and we shouldn't generate warning about non-exhaustiveness.

\begin{defn}[Exhaustiveness or Maybe Exhaustiveness]
  \label{defn:exhaustiveness}
  $\csatisfyormay{\ctruth}{\xi}$ iff $\ctyp{\xi}{\tau}$ and for all $e$ such that $\hexptyp{\cdot}{\Gamma}{e}{\tau}$ and $\isFinal{e}$ we have $\csatisfy{e}{\xi}$ or $\cmaysatisfy{e}{\xi}$.
\end{defn}

\Definitionref{defn:redundancy} gives the formal definition of redundancy. When rules are constructed, Rule \TRules uses it to ensures that no single rule is redundant with respect to its previous rules.
In \Eqref{eq:pathole-redundant}, no matter how we fill the hole $w_1$, all final expressions that match $\hinlp{\hehole{w_1}}$ also match $\hinlp{x}$ and the same goes for the hole $w_2$.
That means for all final expressions $e$,
$\cmaysatisfy{e}{\cpair{\cinl{\cunknown}}{\cunknown}}$ implies $\csatisfy{e}{\cpair{\cinl{\ctruth}}{\ctruth}}$.
Even if the second rule is not complete, we know it is redundant and thus we can suggest the programmer to either rewrite or eliminate it.

\begin{defn}[Redundancy]
  \label{defn:redundancy}
  $\csatisfy{\xi_r}{\xi_{pre}}$ iff $\ctyp{\xi_r}{\tau}$ and $\ctyp{\xi_{pre}}{\tau}$ and for all $e$ such that $\hexptyp{\cdot}{\Gamma}{e}{\tau}$ and $\isVal{e}$ we have $\csatisfy{e}{\xi_r}$ or $\cmaysatisfy{e}{\xi_r}$ implies $\csatisfy{e}{\xi_{pre}}$.
\end{defn}

However, to determine if the redundancy judgement is true, we have to apply \emph{material entailment of constraints} and $\csatisfy{\xi_r}{\xi_{pre}}$ is equivalent to $\csatisfy{\ctruth}{\cor{\cdual{\xi_r}}{\xi_{pre}}}$. Then, we only need to determine if $\cor{\cdual{\xi_r}}{\xi_{pre}}$ can be satisfied by all expressions.

While \Definitionref{defn:exhaustiveness} ensures that the scrutinee at least matches or may match one of the constinuent rules, redundancy doesn't have anything to do with the scrutinee, it is simply the property of a single rule with respect to previous rules. Therefore, we only consider the expression that is already a value and recognize the redundant rule with confidence.

\section{Dynamic Semantics}
\label{sec:dynamics}

As for pattern matching, either the scrutinee matches or may match or doesn't match the pattern.
When there are only one remaining rule, the exhaustiveness or maybe exhaustiveness checking ensures that the pattern of the rule either must be matched or may be matched by the scrutinee.

\begin{thm}[Preservation]
  \label{thrm:preservation}
  If $\hexptyp{\cdot}{\Delta}{e}{\tau}$ and $\htrans{e}{e'}$
  then $\hexptyp{\cdot}{\Delta}{e'}{\tau}$
\end{thm}

\begin{thm}[Progress]
 \label{thrm:progress}
 If $\hexptyp{\cdot}{\Delta}{e}{\tau}$ then either $\isFinal{e}$ or $\htrans{e}{e'}$ for some $e'$.
\end{thm}

\section{Discussion}
The abstract explores how constraint helps reasoning pattern matching with typed holes.
One main contribution of the work is extending the match constraint language \cite{Harper2012} with Unknown constraint and introduce a three-way logic, including the concept of ``maybe''. The other contribution is the formalism of pattern matching with typed holes based on the development of the extended match constraint language.

Next, type holes and dynamic type casting would be added and our type system would be turned into a gradual type system.
Recently, pattern matching statements have also been proposed to be added to \textsf{Python}, whose \emph{Type Hints} is taken from the idea of gradual typing \cite{pep484,pep622,Siek2006}.

As a result, the work lays a foundation for integrating full-fledged pattern matching into the \textsf{Hazel} programming environment described in \citet{DBLP:journals/pacmpl/OmarVCH19}.

\clearpage

\bibliographystyle{ACM-Reference-Format}
\bibliography{references}
\end{document}
